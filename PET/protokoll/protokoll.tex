\documentclass[slug=PET, room=Andreas-Schubert-Bau\,\ 424A, supervisor=Carsten\ Bittrich, coursedate=10.\ 01.\ 2020]{../../Lab_Report_LaTeX/lab_report}

\title{Postitronenemissionstomographie}
\author{Oliver Matthes, Valentin Boettcher}
\usepackage{todonotes}
\graphicspath{ {figs/} }
\usepackage{tikz}
\usepackage{pgf}
\usepackage[version=4]{mhchem}
\usepackage[ngerman]{babel}
\usepackage{subcaption}
\usepackage{amssymb}

% bib
\addbibresource{protokoll.bib}

\begin{document}
\maketitle

\section{Einleitung}
\label{sec:einl}

Dieser Versuch beschäftigt sich mit der Positronen-Emissions-Tomographie (PET), die ein
wichtiges bildgebendes Verfahren in der Medizin darstellt, um beispielsweise einen Tumore oder
allgemein Stoffwechselvorgänge sichtbar zu machen.
Dazu muss der Patient einen so genannten Tracer aufnehmen. Dabei handelt es sich um eine
radioaktive Substanz mit einer Halbwertszeit von mehreren Minuten oder Stunden, die sich an
bestimmte Geweberegionen im Körper anlagert. Bei dieser radioaktiven Substanz handelt es sich
um ein Material, das überwiegend über den \(\beta^+\) - Zerfall zerfällt.
Wie der Name des Verfahrens besagt, benötigt es für dieses Positronen. Diese werden durch eben
erwähnten \(\beta^+\) - Zerfall erzeugt:

\begin{equation}\label{eq:betazerf}
        p^+ \rightarrow n + e^+ + \nu_e
\end{equation}

Wie die Zerfallsgleichung~\eqref{eq:betazerf} zeigt, zerfällt beim \(\beta^+\) - Zerfall ein Proton
in ein Neutron, das für die PET wichtige Positron und ein Elektron-Neutrino. Weswegen die
Tracer-Materialien einen Protonenüberschuss im Kern haben.
Neutrinos interagieren nur sehr selten mit Materie, weshalb die beim Zerfall entstehenden einfach
durch den Körper durchgehen und somit hier nicht interessant sind. Das Neutron verbleibt im Kern
und das Positron propagiert durch das Gewebe des Körpers mit einer Reichweite von wenigen
Millimetern und annihiliert dann mit einem Elektron aus der Hülle eines Atoms zu zwei Photonen.
Je besser die Auflösung dieses Verfahrens sein soll, desto kürzer darf die Reichweite der
Positronen sein, das heißt: Je näher sie am Entstehungsort annihilieren, desto besser.

\begin{equation}\label{eq:annihi}
        e^+ + e^- \rightarrow \gamma + \gamma
\end{equation}

Die entstehenden Photonen haben stets die gleiche Energie. Die Ruhemasse von Elektron und Positron
beträgt \(\SI{1022}{\kilo\electronvolt}\) und teilt sich bei der Paarvernichtung gleichmäßig auf
die Photonen auf, sodass diese ergo eine Energie von \(E_\gamma = \SI{511}{\kilo\electronvolt}\).
Da die Annihilation in Ruhe stattfindet und Energie und Impulserhaltung gilt, schließen die beiden
Photonen einen Winkel von \(180^\circ\) ein, bewegen sich also antiparallel.\\

Um den Beobachtungsort sind in einem Ring (in diesem Versuch nur zwei gegenüberliegende)
Detektoren angebracht, die die entstandenen Photonen registrieren.
Allerdings können zum Beispiel durch andere Zerfallsprozesse natürlich auch andere Photonen
entstehen, die die Messungen stören. Um solche zufällige Koinzidenzen möglichst gering zu halten,
müssen die eintreffenden Lichtquanten bestimmte Kriterien erfüllen:\\
Wie eben beschrieben haben die Photonen immer die gleiche Energie, sodass Photonen, die nicht in
ein Energiefenster passen, nicht berücksichtigt werden. Des Weiteren haben die Detektoren einen
bestimmten Abstand zu einander, was bedeutet, dass die Photonen mit einer maximalen zeitlichen
Differenz von Detektorabstand geteilt durch Lichtgeschwindigkeit eintreffen müssen, sofern sie
innerhalb des PET erzeugt wurden.\\

Die Zählrate der wahren Koinzidenzen, also der für uns interessanten ergibt sich folgendermaßen:

\begin{equation}\label{eq:wahrkoinz}
        \dot N_K = \qty(\frac{\Omega_{min}}{2 \pi}) \cdot P_\beta A \cdot \epsilon_1 \epsilon_2
\end{equation}

\begin{conditions}
        \Omega_{min} & Raumwinkelelement des von der Quelle am weitesten entfernten Detektors\\
        P_\beta & Zerfallswahrscheinlichkeit des Nuklids für \(\beta^+\) - Zerfall\\
        A & Aktivität der Quelle\\
        \epsilon_1/\epsilon_2 & intrinsische Nachweiseffektivitäten der Detektoren
\end{conditions}

Die Detektoren, die für die PET verwendet werden, sind Szintillationsdetektoren. Einfach
beschrieben absorbiert ein Szintillator ein eintreffendes Photon und wandelt dieses in Photonen
mit einer anderen Wellenlänge, die meist im sichtbaren oder ultravioletten Bereich liegt, um.
Dabei dient das Szintillatormaterial, das in diesem Versuch aus Kristallen besteht (es gibt
aber auch weitere Szintillatortypen, beispielsweise organische, die mit Plastik als Material
arbeiten) auch als Lichtleiter.
Hinter dem Kristall befindet sich eine Photokathode, die die eintreffenden Photonen über den
Photoelektrischen Effekt in Elektronen umwandeln. Da die durch die Photonen herausgelösten
Elektronen sehr wenige sein können, zu wenige, um ein Signal zu messen, ist nun noch ein
Photomultiplier angeschlossen, der die Elektronen mittels Hochspannung beschleunigt und die
Anzahl der Elektronen vervielfältigt. Diese Vervielfältigung funktioniert über Dynoden. Auf diese
prallen die Elektronen auf und lösen mehrere Sekundärelektronen heraus. Durch
Hintereinanderreihung von mehreren Dynoden, steigt die Elektronenanzahl exponentiell an.\\

Um mit Hilfe der Szintillatoren auf die Energie sowie den Ort der im Detektor erzeugten Photonen
schließen zu können, ist ein Szintillatorkristall von vier Photomultipliern unterschiedlich tief
eingeschnitten. Diese Art von Detektoren nennt sich Blockdetektoren. In diesem Versuch ist der
Kristall vor den vier Photomultipliern in einer 8 x 8 - Matrix unterteilt.
Die so aufgenommenen Amplituden sind aufsummiert proportional zu der Energie, die von den
Photonen im Detektor deponiert wurde. Bildet man den Schwerpunkt der Amplituden kann man
den Ort, an dem die Photonen mit dem Kristall wechselwirkten, bestimmen.\\

Bei diesem Verfahren wird also eine zweidimensionale Abbildung, eine Quellverteilung, die man
untersuchen will, auf eine eindimensionale Funktion, die eine Intensitätsverteilung beschreibt,
projiziert.
Dieser Vorgang wird mathematisch durch die Radon-Transformation beschrieben. Diese Transformation
projiziert eine zweidimensionale Funktion auf eine Gerade \(s\), die durch den Koordinatenursprung
läuft und einen Winkel \(\vartheta\) mit der x-Achse einschließt.
Die Projektion ordnet dabei den auf der Projektionsgeraden \(s\) befindlichen Punkten ein
Linienintegral \(p(s, \vartheta)\) zu:

\begin{equation}\label{eq:linienint}
        p(s, \vartheta) = \int_{-R_I}^{+R_I} f_I(s \cdot \cos
        \vartheta - t \cdot \sin \vartheta, s \cdot \sin\vartheta + t
        \cdot \cos\vartheta) \dd{t}
\end{equation}

Wobei folgende Beziehungen genutzt wurden:

\begin{align}
        s &= x \cdot \cos\vartheta + y \cdot \sin\vartheta \\
        x &= s \cdot \cos \vartheta - t \cdot \sin \vartheta \\
        y &= s \cdot \sin\vartheta + t \cdot \cos\vartheta \\
        R_I &= \sqrt{x^2 + y^2}
\end{align}

Stellt man die Funktion \(p(s, \vartheta)\) zweidimensional dar, erhält man ein so genanntes
\emph{Sinogramm}.

\section{Auswertung}
\label{sec:ausw}

\subsection{Kalibrierung}
\label{sec:kalib}

\subsubsection{Festlegung der Energie und Koinzidenzzeitfenster}
Beschreibung Messung etc....
Messzeit:

\begin{equation}
  \label{eq:caltime}
  T = \SI{647\pm}{\second}
\end{equation}

\label{sec:energkozeit}

Zur bestimmung der Energiefenster wurden die (normierten) Z\"ahlraten
der beiden Detektoren f\"ur die mittige Quellposition \"uber die
Energie aufgetragen.

\begin{figure}[h]\centering
  %% Creator: Matplotlib, PGF backend
%%
%% To include the figure in your LaTeX document, write
%%   \input{<filename>.pgf}
%%
%% Make sure the required packages are loaded in your preamble
%%   \usepackage{pgf}
%%
%% Figures using additional raster images can only be included by \input if
%% they are in the same directory as the main LaTeX file. For loading figures
%% from other directories you can use the `import` package
%%   \usepackage{import}
%% and then include the figures with
%%   \import{<path to file>}{<filename>.pgf}
%%
%% Matplotlib used the following preamble
%%   \usepackage{fontspec}
%%   \setmainfont{DejaVuSerif.ttf}[Path=/usr/lib/python3.7/site-packages/matplotlib/mpl-data/fonts/ttf/]
%%   \setsansfont{DejaVuSans.ttf}[Path=/usr/lib/python3.7/site-packages/matplotlib/mpl-data/fonts/ttf/]
%%   \setmonofont{DejaVuSansMono.ttf}[Path=/usr/lib/python3.7/site-packages/matplotlib/mpl-data/fonts/ttf/]
%%
\begingroup%
\makeatletter%
\begin{pgfpicture}%
\pgfpathrectangle{\pgfpointorigin}{\pgfqpoint{5.000000in}{4.000000in}}%
\pgfusepath{use as bounding box, clip}%
\begin{pgfscope}%
\pgfsetbuttcap%
\pgfsetmiterjoin%
\definecolor{currentfill}{rgb}{1.000000,1.000000,1.000000}%
\pgfsetfillcolor{currentfill}%
\pgfsetlinewidth{0.000000pt}%
\definecolor{currentstroke}{rgb}{1.000000,1.000000,1.000000}%
\pgfsetstrokecolor{currentstroke}%
\pgfsetdash{}{0pt}%
\pgfpathmoveto{\pgfqpoint{0.000000in}{0.000000in}}%
\pgfpathlineto{\pgfqpoint{5.000000in}{0.000000in}}%
\pgfpathlineto{\pgfqpoint{5.000000in}{4.000000in}}%
\pgfpathlineto{\pgfqpoint{0.000000in}{4.000000in}}%
\pgfpathclose%
\pgfusepath{fill}%
\end{pgfscope}%
\begin{pgfscope}%
\pgfsetbuttcap%
\pgfsetmiterjoin%
\definecolor{currentfill}{rgb}{1.000000,1.000000,1.000000}%
\pgfsetfillcolor{currentfill}%
\pgfsetlinewidth{0.000000pt}%
\definecolor{currentstroke}{rgb}{0.000000,0.000000,0.000000}%
\pgfsetstrokecolor{currentstroke}%
\pgfsetstrokeopacity{0.000000}%
\pgfsetdash{}{0pt}%
\pgfpathmoveto{\pgfqpoint{0.752778in}{0.582778in}}%
\pgfpathlineto{\pgfqpoint{4.801389in}{0.582778in}}%
\pgfpathlineto{\pgfqpoint{4.801389in}{3.795000in}}%
\pgfpathlineto{\pgfqpoint{0.752778in}{3.795000in}}%
\pgfpathclose%
\pgfusepath{fill}%
\end{pgfscope}%
\begin{pgfscope}%
\pgfpathrectangle{\pgfqpoint{0.752778in}{0.582778in}}{\pgfqpoint{4.048611in}{3.212222in}}%
\pgfusepath{clip}%
\pgfsetrectcap%
\pgfsetroundjoin%
\pgfsetlinewidth{0.803000pt}%
\definecolor{currentstroke}{rgb}{0.690196,0.690196,0.690196}%
\pgfsetstrokecolor{currentstroke}%
\pgfsetstrokeopacity{0.800000}%
\pgfsetdash{}{0pt}%
\pgfpathmoveto{\pgfqpoint{0.846784in}{0.582778in}}%
\pgfpathlineto{\pgfqpoint{0.846784in}{3.795000in}}%
\pgfusepath{stroke}%
\end{pgfscope}%
\begin{pgfscope}%
\pgfsetbuttcap%
\pgfsetroundjoin%
\definecolor{currentfill}{rgb}{0.000000,0.000000,0.000000}%
\pgfsetfillcolor{currentfill}%
\pgfsetlinewidth{0.803000pt}%
\definecolor{currentstroke}{rgb}{0.000000,0.000000,0.000000}%
\pgfsetstrokecolor{currentstroke}%
\pgfsetdash{}{0pt}%
\pgfsys@defobject{currentmarker}{\pgfqpoint{0.000000in}{-0.048611in}}{\pgfqpoint{0.000000in}{0.000000in}}{%
\pgfpathmoveto{\pgfqpoint{0.000000in}{0.000000in}}%
\pgfpathlineto{\pgfqpoint{0.000000in}{-0.048611in}}%
\pgfusepath{stroke,fill}%
}%
\begin{pgfscope}%
\pgfsys@transformshift{0.846784in}{0.582778in}%
\pgfsys@useobject{currentmarker}{}%
\end{pgfscope}%
\end{pgfscope}%
\begin{pgfscope}%
\definecolor{textcolor}{rgb}{0.000000,0.000000,0.000000}%
\pgfsetstrokecolor{textcolor}%
\pgfsetfillcolor{textcolor}%
\pgftext[x=0.846784in,y=0.485556in,,top]{\color{textcolor}\sffamily\fontsize{10.000000}{12.000000}\selectfont 0}%
\end{pgfscope}%
\begin{pgfscope}%
\pgfpathrectangle{\pgfqpoint{0.752778in}{0.582778in}}{\pgfqpoint{4.048611in}{3.212222in}}%
\pgfusepath{clip}%
\pgfsetrectcap%
\pgfsetroundjoin%
\pgfsetlinewidth{0.803000pt}%
\definecolor{currentstroke}{rgb}{0.690196,0.690196,0.690196}%
\pgfsetstrokecolor{currentstroke}%
\pgfsetstrokeopacity{0.800000}%
\pgfsetdash{}{0pt}%
\pgfpathmoveto{\pgfqpoint{1.290353in}{0.582778in}}%
\pgfpathlineto{\pgfqpoint{1.290353in}{3.795000in}}%
\pgfusepath{stroke}%
\end{pgfscope}%
\begin{pgfscope}%
\pgfsetbuttcap%
\pgfsetroundjoin%
\definecolor{currentfill}{rgb}{0.000000,0.000000,0.000000}%
\pgfsetfillcolor{currentfill}%
\pgfsetlinewidth{0.803000pt}%
\definecolor{currentstroke}{rgb}{0.000000,0.000000,0.000000}%
\pgfsetstrokecolor{currentstroke}%
\pgfsetdash{}{0pt}%
\pgfsys@defobject{currentmarker}{\pgfqpoint{0.000000in}{-0.048611in}}{\pgfqpoint{0.000000in}{0.000000in}}{%
\pgfpathmoveto{\pgfqpoint{0.000000in}{0.000000in}}%
\pgfpathlineto{\pgfqpoint{0.000000in}{-0.048611in}}%
\pgfusepath{stroke,fill}%
}%
\begin{pgfscope}%
\pgfsys@transformshift{1.290353in}{0.582778in}%
\pgfsys@useobject{currentmarker}{}%
\end{pgfscope}%
\end{pgfscope}%
\begin{pgfscope}%
\definecolor{textcolor}{rgb}{0.000000,0.000000,0.000000}%
\pgfsetstrokecolor{textcolor}%
\pgfsetfillcolor{textcolor}%
\pgftext[x=1.290353in,y=0.485556in,,top]{\color{textcolor}\sffamily\fontsize{10.000000}{12.000000}\selectfont 400}%
\end{pgfscope}%
\begin{pgfscope}%
\pgfpathrectangle{\pgfqpoint{0.752778in}{0.582778in}}{\pgfqpoint{4.048611in}{3.212222in}}%
\pgfusepath{clip}%
\pgfsetrectcap%
\pgfsetroundjoin%
\pgfsetlinewidth{0.803000pt}%
\definecolor{currentstroke}{rgb}{0.690196,0.690196,0.690196}%
\pgfsetstrokecolor{currentstroke}%
\pgfsetstrokeopacity{0.800000}%
\pgfsetdash{}{0pt}%
\pgfpathmoveto{\pgfqpoint{1.733922in}{0.582778in}}%
\pgfpathlineto{\pgfqpoint{1.733922in}{3.795000in}}%
\pgfusepath{stroke}%
\end{pgfscope}%
\begin{pgfscope}%
\pgfsetbuttcap%
\pgfsetroundjoin%
\definecolor{currentfill}{rgb}{0.000000,0.000000,0.000000}%
\pgfsetfillcolor{currentfill}%
\pgfsetlinewidth{0.803000pt}%
\definecolor{currentstroke}{rgb}{0.000000,0.000000,0.000000}%
\pgfsetstrokecolor{currentstroke}%
\pgfsetdash{}{0pt}%
\pgfsys@defobject{currentmarker}{\pgfqpoint{0.000000in}{-0.048611in}}{\pgfqpoint{0.000000in}{0.000000in}}{%
\pgfpathmoveto{\pgfqpoint{0.000000in}{0.000000in}}%
\pgfpathlineto{\pgfqpoint{0.000000in}{-0.048611in}}%
\pgfusepath{stroke,fill}%
}%
\begin{pgfscope}%
\pgfsys@transformshift{1.733922in}{0.582778in}%
\pgfsys@useobject{currentmarker}{}%
\end{pgfscope}%
\end{pgfscope}%
\begin{pgfscope}%
\definecolor{textcolor}{rgb}{0.000000,0.000000,0.000000}%
\pgfsetstrokecolor{textcolor}%
\pgfsetfillcolor{textcolor}%
\pgftext[x=1.733922in,y=0.485556in,,top]{\color{textcolor}\sffamily\fontsize{10.000000}{12.000000}\selectfont 800}%
\end{pgfscope}%
\begin{pgfscope}%
\pgfpathrectangle{\pgfqpoint{0.752778in}{0.582778in}}{\pgfqpoint{4.048611in}{3.212222in}}%
\pgfusepath{clip}%
\pgfsetrectcap%
\pgfsetroundjoin%
\pgfsetlinewidth{0.803000pt}%
\definecolor{currentstroke}{rgb}{0.690196,0.690196,0.690196}%
\pgfsetstrokecolor{currentstroke}%
\pgfsetstrokeopacity{0.800000}%
\pgfsetdash{}{0pt}%
\pgfpathmoveto{\pgfqpoint{2.177490in}{0.582778in}}%
\pgfpathlineto{\pgfqpoint{2.177490in}{3.795000in}}%
\pgfusepath{stroke}%
\end{pgfscope}%
\begin{pgfscope}%
\pgfsetbuttcap%
\pgfsetroundjoin%
\definecolor{currentfill}{rgb}{0.000000,0.000000,0.000000}%
\pgfsetfillcolor{currentfill}%
\pgfsetlinewidth{0.803000pt}%
\definecolor{currentstroke}{rgb}{0.000000,0.000000,0.000000}%
\pgfsetstrokecolor{currentstroke}%
\pgfsetdash{}{0pt}%
\pgfsys@defobject{currentmarker}{\pgfqpoint{0.000000in}{-0.048611in}}{\pgfqpoint{0.000000in}{0.000000in}}{%
\pgfpathmoveto{\pgfqpoint{0.000000in}{0.000000in}}%
\pgfpathlineto{\pgfqpoint{0.000000in}{-0.048611in}}%
\pgfusepath{stroke,fill}%
}%
\begin{pgfscope}%
\pgfsys@transformshift{2.177490in}{0.582778in}%
\pgfsys@useobject{currentmarker}{}%
\end{pgfscope}%
\end{pgfscope}%
\begin{pgfscope}%
\definecolor{textcolor}{rgb}{0.000000,0.000000,0.000000}%
\pgfsetstrokecolor{textcolor}%
\pgfsetfillcolor{textcolor}%
\pgftext[x=2.177490in,y=0.485556in,,top]{\color{textcolor}\sffamily\fontsize{10.000000}{12.000000}\selectfont 1200}%
\end{pgfscope}%
\begin{pgfscope}%
\pgfpathrectangle{\pgfqpoint{0.752778in}{0.582778in}}{\pgfqpoint{4.048611in}{3.212222in}}%
\pgfusepath{clip}%
\pgfsetrectcap%
\pgfsetroundjoin%
\pgfsetlinewidth{0.803000pt}%
\definecolor{currentstroke}{rgb}{0.690196,0.690196,0.690196}%
\pgfsetstrokecolor{currentstroke}%
\pgfsetstrokeopacity{0.800000}%
\pgfsetdash{}{0pt}%
\pgfpathmoveto{\pgfqpoint{2.621059in}{0.582778in}}%
\pgfpathlineto{\pgfqpoint{2.621059in}{3.795000in}}%
\pgfusepath{stroke}%
\end{pgfscope}%
\begin{pgfscope}%
\pgfsetbuttcap%
\pgfsetroundjoin%
\definecolor{currentfill}{rgb}{0.000000,0.000000,0.000000}%
\pgfsetfillcolor{currentfill}%
\pgfsetlinewidth{0.803000pt}%
\definecolor{currentstroke}{rgb}{0.000000,0.000000,0.000000}%
\pgfsetstrokecolor{currentstroke}%
\pgfsetdash{}{0pt}%
\pgfsys@defobject{currentmarker}{\pgfqpoint{0.000000in}{-0.048611in}}{\pgfqpoint{0.000000in}{0.000000in}}{%
\pgfpathmoveto{\pgfqpoint{0.000000in}{0.000000in}}%
\pgfpathlineto{\pgfqpoint{0.000000in}{-0.048611in}}%
\pgfusepath{stroke,fill}%
}%
\begin{pgfscope}%
\pgfsys@transformshift{2.621059in}{0.582778in}%
\pgfsys@useobject{currentmarker}{}%
\end{pgfscope}%
\end{pgfscope}%
\begin{pgfscope}%
\definecolor{textcolor}{rgb}{0.000000,0.000000,0.000000}%
\pgfsetstrokecolor{textcolor}%
\pgfsetfillcolor{textcolor}%
\pgftext[x=2.621059in,y=0.485556in,,top]{\color{textcolor}\sffamily\fontsize{10.000000}{12.000000}\selectfont 1600}%
\end{pgfscope}%
\begin{pgfscope}%
\pgfpathrectangle{\pgfqpoint{0.752778in}{0.582778in}}{\pgfqpoint{4.048611in}{3.212222in}}%
\pgfusepath{clip}%
\pgfsetrectcap%
\pgfsetroundjoin%
\pgfsetlinewidth{0.803000pt}%
\definecolor{currentstroke}{rgb}{0.690196,0.690196,0.690196}%
\pgfsetstrokecolor{currentstroke}%
\pgfsetstrokeopacity{0.800000}%
\pgfsetdash{}{0pt}%
\pgfpathmoveto{\pgfqpoint{3.064627in}{0.582778in}}%
\pgfpathlineto{\pgfqpoint{3.064627in}{3.795000in}}%
\pgfusepath{stroke}%
\end{pgfscope}%
\begin{pgfscope}%
\pgfsetbuttcap%
\pgfsetroundjoin%
\definecolor{currentfill}{rgb}{0.000000,0.000000,0.000000}%
\pgfsetfillcolor{currentfill}%
\pgfsetlinewidth{0.803000pt}%
\definecolor{currentstroke}{rgb}{0.000000,0.000000,0.000000}%
\pgfsetstrokecolor{currentstroke}%
\pgfsetdash{}{0pt}%
\pgfsys@defobject{currentmarker}{\pgfqpoint{0.000000in}{-0.048611in}}{\pgfqpoint{0.000000in}{0.000000in}}{%
\pgfpathmoveto{\pgfqpoint{0.000000in}{0.000000in}}%
\pgfpathlineto{\pgfqpoint{0.000000in}{-0.048611in}}%
\pgfusepath{stroke,fill}%
}%
\begin{pgfscope}%
\pgfsys@transformshift{3.064627in}{0.582778in}%
\pgfsys@useobject{currentmarker}{}%
\end{pgfscope}%
\end{pgfscope}%
\begin{pgfscope}%
\definecolor{textcolor}{rgb}{0.000000,0.000000,0.000000}%
\pgfsetstrokecolor{textcolor}%
\pgfsetfillcolor{textcolor}%
\pgftext[x=3.064627in,y=0.485556in,,top]{\color{textcolor}\sffamily\fontsize{10.000000}{12.000000}\selectfont 2000}%
\end{pgfscope}%
\begin{pgfscope}%
\pgfpathrectangle{\pgfqpoint{0.752778in}{0.582778in}}{\pgfqpoint{4.048611in}{3.212222in}}%
\pgfusepath{clip}%
\pgfsetrectcap%
\pgfsetroundjoin%
\pgfsetlinewidth{0.803000pt}%
\definecolor{currentstroke}{rgb}{0.690196,0.690196,0.690196}%
\pgfsetstrokecolor{currentstroke}%
\pgfsetstrokeopacity{0.800000}%
\pgfsetdash{}{0pt}%
\pgfpathmoveto{\pgfqpoint{3.508196in}{0.582778in}}%
\pgfpathlineto{\pgfqpoint{3.508196in}{3.795000in}}%
\pgfusepath{stroke}%
\end{pgfscope}%
\begin{pgfscope}%
\pgfsetbuttcap%
\pgfsetroundjoin%
\definecolor{currentfill}{rgb}{0.000000,0.000000,0.000000}%
\pgfsetfillcolor{currentfill}%
\pgfsetlinewidth{0.803000pt}%
\definecolor{currentstroke}{rgb}{0.000000,0.000000,0.000000}%
\pgfsetstrokecolor{currentstroke}%
\pgfsetdash{}{0pt}%
\pgfsys@defobject{currentmarker}{\pgfqpoint{0.000000in}{-0.048611in}}{\pgfqpoint{0.000000in}{0.000000in}}{%
\pgfpathmoveto{\pgfqpoint{0.000000in}{0.000000in}}%
\pgfpathlineto{\pgfqpoint{0.000000in}{-0.048611in}}%
\pgfusepath{stroke,fill}%
}%
\begin{pgfscope}%
\pgfsys@transformshift{3.508196in}{0.582778in}%
\pgfsys@useobject{currentmarker}{}%
\end{pgfscope}%
\end{pgfscope}%
\begin{pgfscope}%
\definecolor{textcolor}{rgb}{0.000000,0.000000,0.000000}%
\pgfsetstrokecolor{textcolor}%
\pgfsetfillcolor{textcolor}%
\pgftext[x=3.508196in,y=0.485556in,,top]{\color{textcolor}\sffamily\fontsize{10.000000}{12.000000}\selectfont 2400}%
\end{pgfscope}%
\begin{pgfscope}%
\pgfpathrectangle{\pgfqpoint{0.752778in}{0.582778in}}{\pgfqpoint{4.048611in}{3.212222in}}%
\pgfusepath{clip}%
\pgfsetrectcap%
\pgfsetroundjoin%
\pgfsetlinewidth{0.803000pt}%
\definecolor{currentstroke}{rgb}{0.690196,0.690196,0.690196}%
\pgfsetstrokecolor{currentstroke}%
\pgfsetstrokeopacity{0.800000}%
\pgfsetdash{}{0pt}%
\pgfpathmoveto{\pgfqpoint{3.951764in}{0.582778in}}%
\pgfpathlineto{\pgfqpoint{3.951764in}{3.795000in}}%
\pgfusepath{stroke}%
\end{pgfscope}%
\begin{pgfscope}%
\pgfsetbuttcap%
\pgfsetroundjoin%
\definecolor{currentfill}{rgb}{0.000000,0.000000,0.000000}%
\pgfsetfillcolor{currentfill}%
\pgfsetlinewidth{0.803000pt}%
\definecolor{currentstroke}{rgb}{0.000000,0.000000,0.000000}%
\pgfsetstrokecolor{currentstroke}%
\pgfsetdash{}{0pt}%
\pgfsys@defobject{currentmarker}{\pgfqpoint{0.000000in}{-0.048611in}}{\pgfqpoint{0.000000in}{0.000000in}}{%
\pgfpathmoveto{\pgfqpoint{0.000000in}{0.000000in}}%
\pgfpathlineto{\pgfqpoint{0.000000in}{-0.048611in}}%
\pgfusepath{stroke,fill}%
}%
\begin{pgfscope}%
\pgfsys@transformshift{3.951764in}{0.582778in}%
\pgfsys@useobject{currentmarker}{}%
\end{pgfscope}%
\end{pgfscope}%
\begin{pgfscope}%
\definecolor{textcolor}{rgb}{0.000000,0.000000,0.000000}%
\pgfsetstrokecolor{textcolor}%
\pgfsetfillcolor{textcolor}%
\pgftext[x=3.951764in,y=0.485556in,,top]{\color{textcolor}\sffamily\fontsize{10.000000}{12.000000}\selectfont 2800}%
\end{pgfscope}%
\begin{pgfscope}%
\pgfpathrectangle{\pgfqpoint{0.752778in}{0.582778in}}{\pgfqpoint{4.048611in}{3.212222in}}%
\pgfusepath{clip}%
\pgfsetrectcap%
\pgfsetroundjoin%
\pgfsetlinewidth{0.803000pt}%
\definecolor{currentstroke}{rgb}{0.690196,0.690196,0.690196}%
\pgfsetstrokecolor{currentstroke}%
\pgfsetstrokeopacity{0.800000}%
\pgfsetdash{}{0pt}%
\pgfpathmoveto{\pgfqpoint{4.395333in}{0.582778in}}%
\pgfpathlineto{\pgfqpoint{4.395333in}{3.795000in}}%
\pgfusepath{stroke}%
\end{pgfscope}%
\begin{pgfscope}%
\pgfsetbuttcap%
\pgfsetroundjoin%
\definecolor{currentfill}{rgb}{0.000000,0.000000,0.000000}%
\pgfsetfillcolor{currentfill}%
\pgfsetlinewidth{0.803000pt}%
\definecolor{currentstroke}{rgb}{0.000000,0.000000,0.000000}%
\pgfsetstrokecolor{currentstroke}%
\pgfsetdash{}{0pt}%
\pgfsys@defobject{currentmarker}{\pgfqpoint{0.000000in}{-0.048611in}}{\pgfqpoint{0.000000in}{0.000000in}}{%
\pgfpathmoveto{\pgfqpoint{0.000000in}{0.000000in}}%
\pgfpathlineto{\pgfqpoint{0.000000in}{-0.048611in}}%
\pgfusepath{stroke,fill}%
}%
\begin{pgfscope}%
\pgfsys@transformshift{4.395333in}{0.582778in}%
\pgfsys@useobject{currentmarker}{}%
\end{pgfscope}%
\end{pgfscope}%
\begin{pgfscope}%
\definecolor{textcolor}{rgb}{0.000000,0.000000,0.000000}%
\pgfsetstrokecolor{textcolor}%
\pgfsetfillcolor{textcolor}%
\pgftext[x=4.395333in,y=0.485556in,,top]{\color{textcolor}\sffamily\fontsize{10.000000}{12.000000}\selectfont 3200}%
\end{pgfscope}%
\begin{pgfscope}%
\pgfpathrectangle{\pgfqpoint{0.752778in}{0.582778in}}{\pgfqpoint{4.048611in}{3.212222in}}%
\pgfusepath{clip}%
\pgfsetrectcap%
\pgfsetroundjoin%
\pgfsetlinewidth{0.803000pt}%
\definecolor{currentstroke}{rgb}{0.690196,0.690196,0.690196}%
\pgfsetstrokecolor{currentstroke}%
\pgfsetstrokeopacity{0.300000}%
\pgfsetdash{}{0pt}%
\pgfpathmoveto{\pgfqpoint{0.758071in}{0.582778in}}%
\pgfpathlineto{\pgfqpoint{0.758071in}{3.795000in}}%
\pgfusepath{stroke}%
\end{pgfscope}%
\begin{pgfscope}%
\pgfsetbuttcap%
\pgfsetroundjoin%
\definecolor{currentfill}{rgb}{0.000000,0.000000,0.000000}%
\pgfsetfillcolor{currentfill}%
\pgfsetlinewidth{0.602250pt}%
\definecolor{currentstroke}{rgb}{0.000000,0.000000,0.000000}%
\pgfsetstrokecolor{currentstroke}%
\pgfsetdash{}{0pt}%
\pgfsys@defobject{currentmarker}{\pgfqpoint{0.000000in}{-0.027778in}}{\pgfqpoint{0.000000in}{0.000000in}}{%
\pgfpathmoveto{\pgfqpoint{0.000000in}{0.000000in}}%
\pgfpathlineto{\pgfqpoint{0.000000in}{-0.027778in}}%
\pgfusepath{stroke,fill}%
}%
\begin{pgfscope}%
\pgfsys@transformshift{0.758071in}{0.582778in}%
\pgfsys@useobject{currentmarker}{}%
\end{pgfscope}%
\end{pgfscope}%
\begin{pgfscope}%
\pgfpathrectangle{\pgfqpoint{0.752778in}{0.582778in}}{\pgfqpoint{4.048611in}{3.212222in}}%
\pgfusepath{clip}%
\pgfsetrectcap%
\pgfsetroundjoin%
\pgfsetlinewidth{0.803000pt}%
\definecolor{currentstroke}{rgb}{0.690196,0.690196,0.690196}%
\pgfsetstrokecolor{currentstroke}%
\pgfsetstrokeopacity{0.300000}%
\pgfsetdash{}{0pt}%
\pgfpathmoveto{\pgfqpoint{0.802428in}{0.582778in}}%
\pgfpathlineto{\pgfqpoint{0.802428in}{3.795000in}}%
\pgfusepath{stroke}%
\end{pgfscope}%
\begin{pgfscope}%
\pgfsetbuttcap%
\pgfsetroundjoin%
\definecolor{currentfill}{rgb}{0.000000,0.000000,0.000000}%
\pgfsetfillcolor{currentfill}%
\pgfsetlinewidth{0.602250pt}%
\definecolor{currentstroke}{rgb}{0.000000,0.000000,0.000000}%
\pgfsetstrokecolor{currentstroke}%
\pgfsetdash{}{0pt}%
\pgfsys@defobject{currentmarker}{\pgfqpoint{0.000000in}{-0.027778in}}{\pgfqpoint{0.000000in}{0.000000in}}{%
\pgfpathmoveto{\pgfqpoint{0.000000in}{0.000000in}}%
\pgfpathlineto{\pgfqpoint{0.000000in}{-0.027778in}}%
\pgfusepath{stroke,fill}%
}%
\begin{pgfscope}%
\pgfsys@transformshift{0.802428in}{0.582778in}%
\pgfsys@useobject{currentmarker}{}%
\end{pgfscope}%
\end{pgfscope}%
\begin{pgfscope}%
\pgfpathrectangle{\pgfqpoint{0.752778in}{0.582778in}}{\pgfqpoint{4.048611in}{3.212222in}}%
\pgfusepath{clip}%
\pgfsetrectcap%
\pgfsetroundjoin%
\pgfsetlinewidth{0.803000pt}%
\definecolor{currentstroke}{rgb}{0.690196,0.690196,0.690196}%
\pgfsetstrokecolor{currentstroke}%
\pgfsetstrokeopacity{0.300000}%
\pgfsetdash{}{0pt}%
\pgfpathmoveto{\pgfqpoint{0.891141in}{0.582778in}}%
\pgfpathlineto{\pgfqpoint{0.891141in}{3.795000in}}%
\pgfusepath{stroke}%
\end{pgfscope}%
\begin{pgfscope}%
\pgfsetbuttcap%
\pgfsetroundjoin%
\definecolor{currentfill}{rgb}{0.000000,0.000000,0.000000}%
\pgfsetfillcolor{currentfill}%
\pgfsetlinewidth{0.602250pt}%
\definecolor{currentstroke}{rgb}{0.000000,0.000000,0.000000}%
\pgfsetstrokecolor{currentstroke}%
\pgfsetdash{}{0pt}%
\pgfsys@defobject{currentmarker}{\pgfqpoint{0.000000in}{-0.027778in}}{\pgfqpoint{0.000000in}{0.000000in}}{%
\pgfpathmoveto{\pgfqpoint{0.000000in}{0.000000in}}%
\pgfpathlineto{\pgfqpoint{0.000000in}{-0.027778in}}%
\pgfusepath{stroke,fill}%
}%
\begin{pgfscope}%
\pgfsys@transformshift{0.891141in}{0.582778in}%
\pgfsys@useobject{currentmarker}{}%
\end{pgfscope}%
\end{pgfscope}%
\begin{pgfscope}%
\pgfpathrectangle{\pgfqpoint{0.752778in}{0.582778in}}{\pgfqpoint{4.048611in}{3.212222in}}%
\pgfusepath{clip}%
\pgfsetrectcap%
\pgfsetroundjoin%
\pgfsetlinewidth{0.803000pt}%
\definecolor{currentstroke}{rgb}{0.690196,0.690196,0.690196}%
\pgfsetstrokecolor{currentstroke}%
\pgfsetstrokeopacity{0.300000}%
\pgfsetdash{}{0pt}%
\pgfpathmoveto{\pgfqpoint{0.935498in}{0.582778in}}%
\pgfpathlineto{\pgfqpoint{0.935498in}{3.795000in}}%
\pgfusepath{stroke}%
\end{pgfscope}%
\begin{pgfscope}%
\pgfsetbuttcap%
\pgfsetroundjoin%
\definecolor{currentfill}{rgb}{0.000000,0.000000,0.000000}%
\pgfsetfillcolor{currentfill}%
\pgfsetlinewidth{0.602250pt}%
\definecolor{currentstroke}{rgb}{0.000000,0.000000,0.000000}%
\pgfsetstrokecolor{currentstroke}%
\pgfsetdash{}{0pt}%
\pgfsys@defobject{currentmarker}{\pgfqpoint{0.000000in}{-0.027778in}}{\pgfqpoint{0.000000in}{0.000000in}}{%
\pgfpathmoveto{\pgfqpoint{0.000000in}{0.000000in}}%
\pgfpathlineto{\pgfqpoint{0.000000in}{-0.027778in}}%
\pgfusepath{stroke,fill}%
}%
\begin{pgfscope}%
\pgfsys@transformshift{0.935498in}{0.582778in}%
\pgfsys@useobject{currentmarker}{}%
\end{pgfscope}%
\end{pgfscope}%
\begin{pgfscope}%
\pgfpathrectangle{\pgfqpoint{0.752778in}{0.582778in}}{\pgfqpoint{4.048611in}{3.212222in}}%
\pgfusepath{clip}%
\pgfsetrectcap%
\pgfsetroundjoin%
\pgfsetlinewidth{0.803000pt}%
\definecolor{currentstroke}{rgb}{0.690196,0.690196,0.690196}%
\pgfsetstrokecolor{currentstroke}%
\pgfsetstrokeopacity{0.300000}%
\pgfsetdash{}{0pt}%
\pgfpathmoveto{\pgfqpoint{0.979855in}{0.582778in}}%
\pgfpathlineto{\pgfqpoint{0.979855in}{3.795000in}}%
\pgfusepath{stroke}%
\end{pgfscope}%
\begin{pgfscope}%
\pgfsetbuttcap%
\pgfsetroundjoin%
\definecolor{currentfill}{rgb}{0.000000,0.000000,0.000000}%
\pgfsetfillcolor{currentfill}%
\pgfsetlinewidth{0.602250pt}%
\definecolor{currentstroke}{rgb}{0.000000,0.000000,0.000000}%
\pgfsetstrokecolor{currentstroke}%
\pgfsetdash{}{0pt}%
\pgfsys@defobject{currentmarker}{\pgfqpoint{0.000000in}{-0.027778in}}{\pgfqpoint{0.000000in}{0.000000in}}{%
\pgfpathmoveto{\pgfqpoint{0.000000in}{0.000000in}}%
\pgfpathlineto{\pgfqpoint{0.000000in}{-0.027778in}}%
\pgfusepath{stroke,fill}%
}%
\begin{pgfscope}%
\pgfsys@transformshift{0.979855in}{0.582778in}%
\pgfsys@useobject{currentmarker}{}%
\end{pgfscope}%
\end{pgfscope}%
\begin{pgfscope}%
\pgfpathrectangle{\pgfqpoint{0.752778in}{0.582778in}}{\pgfqpoint{4.048611in}{3.212222in}}%
\pgfusepath{clip}%
\pgfsetrectcap%
\pgfsetroundjoin%
\pgfsetlinewidth{0.803000pt}%
\definecolor{currentstroke}{rgb}{0.690196,0.690196,0.690196}%
\pgfsetstrokecolor{currentstroke}%
\pgfsetstrokeopacity{0.300000}%
\pgfsetdash{}{0pt}%
\pgfpathmoveto{\pgfqpoint{1.024212in}{0.582778in}}%
\pgfpathlineto{\pgfqpoint{1.024212in}{3.795000in}}%
\pgfusepath{stroke}%
\end{pgfscope}%
\begin{pgfscope}%
\pgfsetbuttcap%
\pgfsetroundjoin%
\definecolor{currentfill}{rgb}{0.000000,0.000000,0.000000}%
\pgfsetfillcolor{currentfill}%
\pgfsetlinewidth{0.602250pt}%
\definecolor{currentstroke}{rgb}{0.000000,0.000000,0.000000}%
\pgfsetstrokecolor{currentstroke}%
\pgfsetdash{}{0pt}%
\pgfsys@defobject{currentmarker}{\pgfqpoint{0.000000in}{-0.027778in}}{\pgfqpoint{0.000000in}{0.000000in}}{%
\pgfpathmoveto{\pgfqpoint{0.000000in}{0.000000in}}%
\pgfpathlineto{\pgfqpoint{0.000000in}{-0.027778in}}%
\pgfusepath{stroke,fill}%
}%
\begin{pgfscope}%
\pgfsys@transformshift{1.024212in}{0.582778in}%
\pgfsys@useobject{currentmarker}{}%
\end{pgfscope}%
\end{pgfscope}%
\begin{pgfscope}%
\pgfpathrectangle{\pgfqpoint{0.752778in}{0.582778in}}{\pgfqpoint{4.048611in}{3.212222in}}%
\pgfusepath{clip}%
\pgfsetrectcap%
\pgfsetroundjoin%
\pgfsetlinewidth{0.803000pt}%
\definecolor{currentstroke}{rgb}{0.690196,0.690196,0.690196}%
\pgfsetstrokecolor{currentstroke}%
\pgfsetstrokeopacity{0.300000}%
\pgfsetdash{}{0pt}%
\pgfpathmoveto{\pgfqpoint{1.068569in}{0.582778in}}%
\pgfpathlineto{\pgfqpoint{1.068569in}{3.795000in}}%
\pgfusepath{stroke}%
\end{pgfscope}%
\begin{pgfscope}%
\pgfsetbuttcap%
\pgfsetroundjoin%
\definecolor{currentfill}{rgb}{0.000000,0.000000,0.000000}%
\pgfsetfillcolor{currentfill}%
\pgfsetlinewidth{0.602250pt}%
\definecolor{currentstroke}{rgb}{0.000000,0.000000,0.000000}%
\pgfsetstrokecolor{currentstroke}%
\pgfsetdash{}{0pt}%
\pgfsys@defobject{currentmarker}{\pgfqpoint{0.000000in}{-0.027778in}}{\pgfqpoint{0.000000in}{0.000000in}}{%
\pgfpathmoveto{\pgfqpoint{0.000000in}{0.000000in}}%
\pgfpathlineto{\pgfqpoint{0.000000in}{-0.027778in}}%
\pgfusepath{stroke,fill}%
}%
\begin{pgfscope}%
\pgfsys@transformshift{1.068569in}{0.582778in}%
\pgfsys@useobject{currentmarker}{}%
\end{pgfscope}%
\end{pgfscope}%
\begin{pgfscope}%
\pgfpathrectangle{\pgfqpoint{0.752778in}{0.582778in}}{\pgfqpoint{4.048611in}{3.212222in}}%
\pgfusepath{clip}%
\pgfsetrectcap%
\pgfsetroundjoin%
\pgfsetlinewidth{0.803000pt}%
\definecolor{currentstroke}{rgb}{0.690196,0.690196,0.690196}%
\pgfsetstrokecolor{currentstroke}%
\pgfsetstrokeopacity{0.300000}%
\pgfsetdash{}{0pt}%
\pgfpathmoveto{\pgfqpoint{1.112926in}{0.582778in}}%
\pgfpathlineto{\pgfqpoint{1.112926in}{3.795000in}}%
\pgfusepath{stroke}%
\end{pgfscope}%
\begin{pgfscope}%
\pgfsetbuttcap%
\pgfsetroundjoin%
\definecolor{currentfill}{rgb}{0.000000,0.000000,0.000000}%
\pgfsetfillcolor{currentfill}%
\pgfsetlinewidth{0.602250pt}%
\definecolor{currentstroke}{rgb}{0.000000,0.000000,0.000000}%
\pgfsetstrokecolor{currentstroke}%
\pgfsetdash{}{0pt}%
\pgfsys@defobject{currentmarker}{\pgfqpoint{0.000000in}{-0.027778in}}{\pgfqpoint{0.000000in}{0.000000in}}{%
\pgfpathmoveto{\pgfqpoint{0.000000in}{0.000000in}}%
\pgfpathlineto{\pgfqpoint{0.000000in}{-0.027778in}}%
\pgfusepath{stroke,fill}%
}%
\begin{pgfscope}%
\pgfsys@transformshift{1.112926in}{0.582778in}%
\pgfsys@useobject{currentmarker}{}%
\end{pgfscope}%
\end{pgfscope}%
\begin{pgfscope}%
\pgfpathrectangle{\pgfqpoint{0.752778in}{0.582778in}}{\pgfqpoint{4.048611in}{3.212222in}}%
\pgfusepath{clip}%
\pgfsetrectcap%
\pgfsetroundjoin%
\pgfsetlinewidth{0.803000pt}%
\definecolor{currentstroke}{rgb}{0.690196,0.690196,0.690196}%
\pgfsetstrokecolor{currentstroke}%
\pgfsetstrokeopacity{0.300000}%
\pgfsetdash{}{0pt}%
\pgfpathmoveto{\pgfqpoint{1.157282in}{0.582778in}}%
\pgfpathlineto{\pgfqpoint{1.157282in}{3.795000in}}%
\pgfusepath{stroke}%
\end{pgfscope}%
\begin{pgfscope}%
\pgfsetbuttcap%
\pgfsetroundjoin%
\definecolor{currentfill}{rgb}{0.000000,0.000000,0.000000}%
\pgfsetfillcolor{currentfill}%
\pgfsetlinewidth{0.602250pt}%
\definecolor{currentstroke}{rgb}{0.000000,0.000000,0.000000}%
\pgfsetstrokecolor{currentstroke}%
\pgfsetdash{}{0pt}%
\pgfsys@defobject{currentmarker}{\pgfqpoint{0.000000in}{-0.027778in}}{\pgfqpoint{0.000000in}{0.000000in}}{%
\pgfpathmoveto{\pgfqpoint{0.000000in}{0.000000in}}%
\pgfpathlineto{\pgfqpoint{0.000000in}{-0.027778in}}%
\pgfusepath{stroke,fill}%
}%
\begin{pgfscope}%
\pgfsys@transformshift{1.157282in}{0.582778in}%
\pgfsys@useobject{currentmarker}{}%
\end{pgfscope}%
\end{pgfscope}%
\begin{pgfscope}%
\pgfpathrectangle{\pgfqpoint{0.752778in}{0.582778in}}{\pgfqpoint{4.048611in}{3.212222in}}%
\pgfusepath{clip}%
\pgfsetrectcap%
\pgfsetroundjoin%
\pgfsetlinewidth{0.803000pt}%
\definecolor{currentstroke}{rgb}{0.690196,0.690196,0.690196}%
\pgfsetstrokecolor{currentstroke}%
\pgfsetstrokeopacity{0.300000}%
\pgfsetdash{}{0pt}%
\pgfpathmoveto{\pgfqpoint{1.201639in}{0.582778in}}%
\pgfpathlineto{\pgfqpoint{1.201639in}{3.795000in}}%
\pgfusepath{stroke}%
\end{pgfscope}%
\begin{pgfscope}%
\pgfsetbuttcap%
\pgfsetroundjoin%
\definecolor{currentfill}{rgb}{0.000000,0.000000,0.000000}%
\pgfsetfillcolor{currentfill}%
\pgfsetlinewidth{0.602250pt}%
\definecolor{currentstroke}{rgb}{0.000000,0.000000,0.000000}%
\pgfsetstrokecolor{currentstroke}%
\pgfsetdash{}{0pt}%
\pgfsys@defobject{currentmarker}{\pgfqpoint{0.000000in}{-0.027778in}}{\pgfqpoint{0.000000in}{0.000000in}}{%
\pgfpathmoveto{\pgfqpoint{0.000000in}{0.000000in}}%
\pgfpathlineto{\pgfqpoint{0.000000in}{-0.027778in}}%
\pgfusepath{stroke,fill}%
}%
\begin{pgfscope}%
\pgfsys@transformshift{1.201639in}{0.582778in}%
\pgfsys@useobject{currentmarker}{}%
\end{pgfscope}%
\end{pgfscope}%
\begin{pgfscope}%
\pgfpathrectangle{\pgfqpoint{0.752778in}{0.582778in}}{\pgfqpoint{4.048611in}{3.212222in}}%
\pgfusepath{clip}%
\pgfsetrectcap%
\pgfsetroundjoin%
\pgfsetlinewidth{0.803000pt}%
\definecolor{currentstroke}{rgb}{0.690196,0.690196,0.690196}%
\pgfsetstrokecolor{currentstroke}%
\pgfsetstrokeopacity{0.300000}%
\pgfsetdash{}{0pt}%
\pgfpathmoveto{\pgfqpoint{1.245996in}{0.582778in}}%
\pgfpathlineto{\pgfqpoint{1.245996in}{3.795000in}}%
\pgfusepath{stroke}%
\end{pgfscope}%
\begin{pgfscope}%
\pgfsetbuttcap%
\pgfsetroundjoin%
\definecolor{currentfill}{rgb}{0.000000,0.000000,0.000000}%
\pgfsetfillcolor{currentfill}%
\pgfsetlinewidth{0.602250pt}%
\definecolor{currentstroke}{rgb}{0.000000,0.000000,0.000000}%
\pgfsetstrokecolor{currentstroke}%
\pgfsetdash{}{0pt}%
\pgfsys@defobject{currentmarker}{\pgfqpoint{0.000000in}{-0.027778in}}{\pgfqpoint{0.000000in}{0.000000in}}{%
\pgfpathmoveto{\pgfqpoint{0.000000in}{0.000000in}}%
\pgfpathlineto{\pgfqpoint{0.000000in}{-0.027778in}}%
\pgfusepath{stroke,fill}%
}%
\begin{pgfscope}%
\pgfsys@transformshift{1.245996in}{0.582778in}%
\pgfsys@useobject{currentmarker}{}%
\end{pgfscope}%
\end{pgfscope}%
\begin{pgfscope}%
\pgfpathrectangle{\pgfqpoint{0.752778in}{0.582778in}}{\pgfqpoint{4.048611in}{3.212222in}}%
\pgfusepath{clip}%
\pgfsetrectcap%
\pgfsetroundjoin%
\pgfsetlinewidth{0.803000pt}%
\definecolor{currentstroke}{rgb}{0.690196,0.690196,0.690196}%
\pgfsetstrokecolor{currentstroke}%
\pgfsetstrokeopacity{0.300000}%
\pgfsetdash{}{0pt}%
\pgfpathmoveto{\pgfqpoint{1.334710in}{0.582778in}}%
\pgfpathlineto{\pgfqpoint{1.334710in}{3.795000in}}%
\pgfusepath{stroke}%
\end{pgfscope}%
\begin{pgfscope}%
\pgfsetbuttcap%
\pgfsetroundjoin%
\definecolor{currentfill}{rgb}{0.000000,0.000000,0.000000}%
\pgfsetfillcolor{currentfill}%
\pgfsetlinewidth{0.602250pt}%
\definecolor{currentstroke}{rgb}{0.000000,0.000000,0.000000}%
\pgfsetstrokecolor{currentstroke}%
\pgfsetdash{}{0pt}%
\pgfsys@defobject{currentmarker}{\pgfqpoint{0.000000in}{-0.027778in}}{\pgfqpoint{0.000000in}{0.000000in}}{%
\pgfpathmoveto{\pgfqpoint{0.000000in}{0.000000in}}%
\pgfpathlineto{\pgfqpoint{0.000000in}{-0.027778in}}%
\pgfusepath{stroke,fill}%
}%
\begin{pgfscope}%
\pgfsys@transformshift{1.334710in}{0.582778in}%
\pgfsys@useobject{currentmarker}{}%
\end{pgfscope}%
\end{pgfscope}%
\begin{pgfscope}%
\pgfpathrectangle{\pgfqpoint{0.752778in}{0.582778in}}{\pgfqpoint{4.048611in}{3.212222in}}%
\pgfusepath{clip}%
\pgfsetrectcap%
\pgfsetroundjoin%
\pgfsetlinewidth{0.803000pt}%
\definecolor{currentstroke}{rgb}{0.690196,0.690196,0.690196}%
\pgfsetstrokecolor{currentstroke}%
\pgfsetstrokeopacity{0.300000}%
\pgfsetdash{}{0pt}%
\pgfpathmoveto{\pgfqpoint{1.379067in}{0.582778in}}%
\pgfpathlineto{\pgfqpoint{1.379067in}{3.795000in}}%
\pgfusepath{stroke}%
\end{pgfscope}%
\begin{pgfscope}%
\pgfsetbuttcap%
\pgfsetroundjoin%
\definecolor{currentfill}{rgb}{0.000000,0.000000,0.000000}%
\pgfsetfillcolor{currentfill}%
\pgfsetlinewidth{0.602250pt}%
\definecolor{currentstroke}{rgb}{0.000000,0.000000,0.000000}%
\pgfsetstrokecolor{currentstroke}%
\pgfsetdash{}{0pt}%
\pgfsys@defobject{currentmarker}{\pgfqpoint{0.000000in}{-0.027778in}}{\pgfqpoint{0.000000in}{0.000000in}}{%
\pgfpathmoveto{\pgfqpoint{0.000000in}{0.000000in}}%
\pgfpathlineto{\pgfqpoint{0.000000in}{-0.027778in}}%
\pgfusepath{stroke,fill}%
}%
\begin{pgfscope}%
\pgfsys@transformshift{1.379067in}{0.582778in}%
\pgfsys@useobject{currentmarker}{}%
\end{pgfscope}%
\end{pgfscope}%
\begin{pgfscope}%
\pgfpathrectangle{\pgfqpoint{0.752778in}{0.582778in}}{\pgfqpoint{4.048611in}{3.212222in}}%
\pgfusepath{clip}%
\pgfsetrectcap%
\pgfsetroundjoin%
\pgfsetlinewidth{0.803000pt}%
\definecolor{currentstroke}{rgb}{0.690196,0.690196,0.690196}%
\pgfsetstrokecolor{currentstroke}%
\pgfsetstrokeopacity{0.300000}%
\pgfsetdash{}{0pt}%
\pgfpathmoveto{\pgfqpoint{1.423424in}{0.582778in}}%
\pgfpathlineto{\pgfqpoint{1.423424in}{3.795000in}}%
\pgfusepath{stroke}%
\end{pgfscope}%
\begin{pgfscope}%
\pgfsetbuttcap%
\pgfsetroundjoin%
\definecolor{currentfill}{rgb}{0.000000,0.000000,0.000000}%
\pgfsetfillcolor{currentfill}%
\pgfsetlinewidth{0.602250pt}%
\definecolor{currentstroke}{rgb}{0.000000,0.000000,0.000000}%
\pgfsetstrokecolor{currentstroke}%
\pgfsetdash{}{0pt}%
\pgfsys@defobject{currentmarker}{\pgfqpoint{0.000000in}{-0.027778in}}{\pgfqpoint{0.000000in}{0.000000in}}{%
\pgfpathmoveto{\pgfqpoint{0.000000in}{0.000000in}}%
\pgfpathlineto{\pgfqpoint{0.000000in}{-0.027778in}}%
\pgfusepath{stroke,fill}%
}%
\begin{pgfscope}%
\pgfsys@transformshift{1.423424in}{0.582778in}%
\pgfsys@useobject{currentmarker}{}%
\end{pgfscope}%
\end{pgfscope}%
\begin{pgfscope}%
\pgfpathrectangle{\pgfqpoint{0.752778in}{0.582778in}}{\pgfqpoint{4.048611in}{3.212222in}}%
\pgfusepath{clip}%
\pgfsetrectcap%
\pgfsetroundjoin%
\pgfsetlinewidth{0.803000pt}%
\definecolor{currentstroke}{rgb}{0.690196,0.690196,0.690196}%
\pgfsetstrokecolor{currentstroke}%
\pgfsetstrokeopacity{0.300000}%
\pgfsetdash{}{0pt}%
\pgfpathmoveto{\pgfqpoint{1.467780in}{0.582778in}}%
\pgfpathlineto{\pgfqpoint{1.467780in}{3.795000in}}%
\pgfusepath{stroke}%
\end{pgfscope}%
\begin{pgfscope}%
\pgfsetbuttcap%
\pgfsetroundjoin%
\definecolor{currentfill}{rgb}{0.000000,0.000000,0.000000}%
\pgfsetfillcolor{currentfill}%
\pgfsetlinewidth{0.602250pt}%
\definecolor{currentstroke}{rgb}{0.000000,0.000000,0.000000}%
\pgfsetstrokecolor{currentstroke}%
\pgfsetdash{}{0pt}%
\pgfsys@defobject{currentmarker}{\pgfqpoint{0.000000in}{-0.027778in}}{\pgfqpoint{0.000000in}{0.000000in}}{%
\pgfpathmoveto{\pgfqpoint{0.000000in}{0.000000in}}%
\pgfpathlineto{\pgfqpoint{0.000000in}{-0.027778in}}%
\pgfusepath{stroke,fill}%
}%
\begin{pgfscope}%
\pgfsys@transformshift{1.467780in}{0.582778in}%
\pgfsys@useobject{currentmarker}{}%
\end{pgfscope}%
\end{pgfscope}%
\begin{pgfscope}%
\pgfpathrectangle{\pgfqpoint{0.752778in}{0.582778in}}{\pgfqpoint{4.048611in}{3.212222in}}%
\pgfusepath{clip}%
\pgfsetrectcap%
\pgfsetroundjoin%
\pgfsetlinewidth{0.803000pt}%
\definecolor{currentstroke}{rgb}{0.690196,0.690196,0.690196}%
\pgfsetstrokecolor{currentstroke}%
\pgfsetstrokeopacity{0.300000}%
\pgfsetdash{}{0pt}%
\pgfpathmoveto{\pgfqpoint{1.512137in}{0.582778in}}%
\pgfpathlineto{\pgfqpoint{1.512137in}{3.795000in}}%
\pgfusepath{stroke}%
\end{pgfscope}%
\begin{pgfscope}%
\pgfsetbuttcap%
\pgfsetroundjoin%
\definecolor{currentfill}{rgb}{0.000000,0.000000,0.000000}%
\pgfsetfillcolor{currentfill}%
\pgfsetlinewidth{0.602250pt}%
\definecolor{currentstroke}{rgb}{0.000000,0.000000,0.000000}%
\pgfsetstrokecolor{currentstroke}%
\pgfsetdash{}{0pt}%
\pgfsys@defobject{currentmarker}{\pgfqpoint{0.000000in}{-0.027778in}}{\pgfqpoint{0.000000in}{0.000000in}}{%
\pgfpathmoveto{\pgfqpoint{0.000000in}{0.000000in}}%
\pgfpathlineto{\pgfqpoint{0.000000in}{-0.027778in}}%
\pgfusepath{stroke,fill}%
}%
\begin{pgfscope}%
\pgfsys@transformshift{1.512137in}{0.582778in}%
\pgfsys@useobject{currentmarker}{}%
\end{pgfscope}%
\end{pgfscope}%
\begin{pgfscope}%
\pgfpathrectangle{\pgfqpoint{0.752778in}{0.582778in}}{\pgfqpoint{4.048611in}{3.212222in}}%
\pgfusepath{clip}%
\pgfsetrectcap%
\pgfsetroundjoin%
\pgfsetlinewidth{0.803000pt}%
\definecolor{currentstroke}{rgb}{0.690196,0.690196,0.690196}%
\pgfsetstrokecolor{currentstroke}%
\pgfsetstrokeopacity{0.300000}%
\pgfsetdash{}{0pt}%
\pgfpathmoveto{\pgfqpoint{1.556494in}{0.582778in}}%
\pgfpathlineto{\pgfqpoint{1.556494in}{3.795000in}}%
\pgfusepath{stroke}%
\end{pgfscope}%
\begin{pgfscope}%
\pgfsetbuttcap%
\pgfsetroundjoin%
\definecolor{currentfill}{rgb}{0.000000,0.000000,0.000000}%
\pgfsetfillcolor{currentfill}%
\pgfsetlinewidth{0.602250pt}%
\definecolor{currentstroke}{rgb}{0.000000,0.000000,0.000000}%
\pgfsetstrokecolor{currentstroke}%
\pgfsetdash{}{0pt}%
\pgfsys@defobject{currentmarker}{\pgfqpoint{0.000000in}{-0.027778in}}{\pgfqpoint{0.000000in}{0.000000in}}{%
\pgfpathmoveto{\pgfqpoint{0.000000in}{0.000000in}}%
\pgfpathlineto{\pgfqpoint{0.000000in}{-0.027778in}}%
\pgfusepath{stroke,fill}%
}%
\begin{pgfscope}%
\pgfsys@transformshift{1.556494in}{0.582778in}%
\pgfsys@useobject{currentmarker}{}%
\end{pgfscope}%
\end{pgfscope}%
\begin{pgfscope}%
\pgfpathrectangle{\pgfqpoint{0.752778in}{0.582778in}}{\pgfqpoint{4.048611in}{3.212222in}}%
\pgfusepath{clip}%
\pgfsetrectcap%
\pgfsetroundjoin%
\pgfsetlinewidth{0.803000pt}%
\definecolor{currentstroke}{rgb}{0.690196,0.690196,0.690196}%
\pgfsetstrokecolor{currentstroke}%
\pgfsetstrokeopacity{0.300000}%
\pgfsetdash{}{0pt}%
\pgfpathmoveto{\pgfqpoint{1.600851in}{0.582778in}}%
\pgfpathlineto{\pgfqpoint{1.600851in}{3.795000in}}%
\pgfusepath{stroke}%
\end{pgfscope}%
\begin{pgfscope}%
\pgfsetbuttcap%
\pgfsetroundjoin%
\definecolor{currentfill}{rgb}{0.000000,0.000000,0.000000}%
\pgfsetfillcolor{currentfill}%
\pgfsetlinewidth{0.602250pt}%
\definecolor{currentstroke}{rgb}{0.000000,0.000000,0.000000}%
\pgfsetstrokecolor{currentstroke}%
\pgfsetdash{}{0pt}%
\pgfsys@defobject{currentmarker}{\pgfqpoint{0.000000in}{-0.027778in}}{\pgfqpoint{0.000000in}{0.000000in}}{%
\pgfpathmoveto{\pgfqpoint{0.000000in}{0.000000in}}%
\pgfpathlineto{\pgfqpoint{0.000000in}{-0.027778in}}%
\pgfusepath{stroke,fill}%
}%
\begin{pgfscope}%
\pgfsys@transformshift{1.600851in}{0.582778in}%
\pgfsys@useobject{currentmarker}{}%
\end{pgfscope}%
\end{pgfscope}%
\begin{pgfscope}%
\pgfpathrectangle{\pgfqpoint{0.752778in}{0.582778in}}{\pgfqpoint{4.048611in}{3.212222in}}%
\pgfusepath{clip}%
\pgfsetrectcap%
\pgfsetroundjoin%
\pgfsetlinewidth{0.803000pt}%
\definecolor{currentstroke}{rgb}{0.690196,0.690196,0.690196}%
\pgfsetstrokecolor{currentstroke}%
\pgfsetstrokeopacity{0.300000}%
\pgfsetdash{}{0pt}%
\pgfpathmoveto{\pgfqpoint{1.645208in}{0.582778in}}%
\pgfpathlineto{\pgfqpoint{1.645208in}{3.795000in}}%
\pgfusepath{stroke}%
\end{pgfscope}%
\begin{pgfscope}%
\pgfsetbuttcap%
\pgfsetroundjoin%
\definecolor{currentfill}{rgb}{0.000000,0.000000,0.000000}%
\pgfsetfillcolor{currentfill}%
\pgfsetlinewidth{0.602250pt}%
\definecolor{currentstroke}{rgb}{0.000000,0.000000,0.000000}%
\pgfsetstrokecolor{currentstroke}%
\pgfsetdash{}{0pt}%
\pgfsys@defobject{currentmarker}{\pgfqpoint{0.000000in}{-0.027778in}}{\pgfqpoint{0.000000in}{0.000000in}}{%
\pgfpathmoveto{\pgfqpoint{0.000000in}{0.000000in}}%
\pgfpathlineto{\pgfqpoint{0.000000in}{-0.027778in}}%
\pgfusepath{stroke,fill}%
}%
\begin{pgfscope}%
\pgfsys@transformshift{1.645208in}{0.582778in}%
\pgfsys@useobject{currentmarker}{}%
\end{pgfscope}%
\end{pgfscope}%
\begin{pgfscope}%
\pgfpathrectangle{\pgfqpoint{0.752778in}{0.582778in}}{\pgfqpoint{4.048611in}{3.212222in}}%
\pgfusepath{clip}%
\pgfsetrectcap%
\pgfsetroundjoin%
\pgfsetlinewidth{0.803000pt}%
\definecolor{currentstroke}{rgb}{0.690196,0.690196,0.690196}%
\pgfsetstrokecolor{currentstroke}%
\pgfsetstrokeopacity{0.300000}%
\pgfsetdash{}{0pt}%
\pgfpathmoveto{\pgfqpoint{1.689565in}{0.582778in}}%
\pgfpathlineto{\pgfqpoint{1.689565in}{3.795000in}}%
\pgfusepath{stroke}%
\end{pgfscope}%
\begin{pgfscope}%
\pgfsetbuttcap%
\pgfsetroundjoin%
\definecolor{currentfill}{rgb}{0.000000,0.000000,0.000000}%
\pgfsetfillcolor{currentfill}%
\pgfsetlinewidth{0.602250pt}%
\definecolor{currentstroke}{rgb}{0.000000,0.000000,0.000000}%
\pgfsetstrokecolor{currentstroke}%
\pgfsetdash{}{0pt}%
\pgfsys@defobject{currentmarker}{\pgfqpoint{0.000000in}{-0.027778in}}{\pgfqpoint{0.000000in}{0.000000in}}{%
\pgfpathmoveto{\pgfqpoint{0.000000in}{0.000000in}}%
\pgfpathlineto{\pgfqpoint{0.000000in}{-0.027778in}}%
\pgfusepath{stroke,fill}%
}%
\begin{pgfscope}%
\pgfsys@transformshift{1.689565in}{0.582778in}%
\pgfsys@useobject{currentmarker}{}%
\end{pgfscope}%
\end{pgfscope}%
\begin{pgfscope}%
\pgfpathrectangle{\pgfqpoint{0.752778in}{0.582778in}}{\pgfqpoint{4.048611in}{3.212222in}}%
\pgfusepath{clip}%
\pgfsetrectcap%
\pgfsetroundjoin%
\pgfsetlinewidth{0.803000pt}%
\definecolor{currentstroke}{rgb}{0.690196,0.690196,0.690196}%
\pgfsetstrokecolor{currentstroke}%
\pgfsetstrokeopacity{0.300000}%
\pgfsetdash{}{0pt}%
\pgfpathmoveto{\pgfqpoint{1.778278in}{0.582778in}}%
\pgfpathlineto{\pgfqpoint{1.778278in}{3.795000in}}%
\pgfusepath{stroke}%
\end{pgfscope}%
\begin{pgfscope}%
\pgfsetbuttcap%
\pgfsetroundjoin%
\definecolor{currentfill}{rgb}{0.000000,0.000000,0.000000}%
\pgfsetfillcolor{currentfill}%
\pgfsetlinewidth{0.602250pt}%
\definecolor{currentstroke}{rgb}{0.000000,0.000000,0.000000}%
\pgfsetstrokecolor{currentstroke}%
\pgfsetdash{}{0pt}%
\pgfsys@defobject{currentmarker}{\pgfqpoint{0.000000in}{-0.027778in}}{\pgfqpoint{0.000000in}{0.000000in}}{%
\pgfpathmoveto{\pgfqpoint{0.000000in}{0.000000in}}%
\pgfpathlineto{\pgfqpoint{0.000000in}{-0.027778in}}%
\pgfusepath{stroke,fill}%
}%
\begin{pgfscope}%
\pgfsys@transformshift{1.778278in}{0.582778in}%
\pgfsys@useobject{currentmarker}{}%
\end{pgfscope}%
\end{pgfscope}%
\begin{pgfscope}%
\pgfpathrectangle{\pgfqpoint{0.752778in}{0.582778in}}{\pgfqpoint{4.048611in}{3.212222in}}%
\pgfusepath{clip}%
\pgfsetrectcap%
\pgfsetroundjoin%
\pgfsetlinewidth{0.803000pt}%
\definecolor{currentstroke}{rgb}{0.690196,0.690196,0.690196}%
\pgfsetstrokecolor{currentstroke}%
\pgfsetstrokeopacity{0.300000}%
\pgfsetdash{}{0pt}%
\pgfpathmoveto{\pgfqpoint{1.822635in}{0.582778in}}%
\pgfpathlineto{\pgfqpoint{1.822635in}{3.795000in}}%
\pgfusepath{stroke}%
\end{pgfscope}%
\begin{pgfscope}%
\pgfsetbuttcap%
\pgfsetroundjoin%
\definecolor{currentfill}{rgb}{0.000000,0.000000,0.000000}%
\pgfsetfillcolor{currentfill}%
\pgfsetlinewidth{0.602250pt}%
\definecolor{currentstroke}{rgb}{0.000000,0.000000,0.000000}%
\pgfsetstrokecolor{currentstroke}%
\pgfsetdash{}{0pt}%
\pgfsys@defobject{currentmarker}{\pgfqpoint{0.000000in}{-0.027778in}}{\pgfqpoint{0.000000in}{0.000000in}}{%
\pgfpathmoveto{\pgfqpoint{0.000000in}{0.000000in}}%
\pgfpathlineto{\pgfqpoint{0.000000in}{-0.027778in}}%
\pgfusepath{stroke,fill}%
}%
\begin{pgfscope}%
\pgfsys@transformshift{1.822635in}{0.582778in}%
\pgfsys@useobject{currentmarker}{}%
\end{pgfscope}%
\end{pgfscope}%
\begin{pgfscope}%
\pgfpathrectangle{\pgfqpoint{0.752778in}{0.582778in}}{\pgfqpoint{4.048611in}{3.212222in}}%
\pgfusepath{clip}%
\pgfsetrectcap%
\pgfsetroundjoin%
\pgfsetlinewidth{0.803000pt}%
\definecolor{currentstroke}{rgb}{0.690196,0.690196,0.690196}%
\pgfsetstrokecolor{currentstroke}%
\pgfsetstrokeopacity{0.300000}%
\pgfsetdash{}{0pt}%
\pgfpathmoveto{\pgfqpoint{1.866992in}{0.582778in}}%
\pgfpathlineto{\pgfqpoint{1.866992in}{3.795000in}}%
\pgfusepath{stroke}%
\end{pgfscope}%
\begin{pgfscope}%
\pgfsetbuttcap%
\pgfsetroundjoin%
\definecolor{currentfill}{rgb}{0.000000,0.000000,0.000000}%
\pgfsetfillcolor{currentfill}%
\pgfsetlinewidth{0.602250pt}%
\definecolor{currentstroke}{rgb}{0.000000,0.000000,0.000000}%
\pgfsetstrokecolor{currentstroke}%
\pgfsetdash{}{0pt}%
\pgfsys@defobject{currentmarker}{\pgfqpoint{0.000000in}{-0.027778in}}{\pgfqpoint{0.000000in}{0.000000in}}{%
\pgfpathmoveto{\pgfqpoint{0.000000in}{0.000000in}}%
\pgfpathlineto{\pgfqpoint{0.000000in}{-0.027778in}}%
\pgfusepath{stroke,fill}%
}%
\begin{pgfscope}%
\pgfsys@transformshift{1.866992in}{0.582778in}%
\pgfsys@useobject{currentmarker}{}%
\end{pgfscope}%
\end{pgfscope}%
\begin{pgfscope}%
\pgfpathrectangle{\pgfqpoint{0.752778in}{0.582778in}}{\pgfqpoint{4.048611in}{3.212222in}}%
\pgfusepath{clip}%
\pgfsetrectcap%
\pgfsetroundjoin%
\pgfsetlinewidth{0.803000pt}%
\definecolor{currentstroke}{rgb}{0.690196,0.690196,0.690196}%
\pgfsetstrokecolor{currentstroke}%
\pgfsetstrokeopacity{0.300000}%
\pgfsetdash{}{0pt}%
\pgfpathmoveto{\pgfqpoint{1.911349in}{0.582778in}}%
\pgfpathlineto{\pgfqpoint{1.911349in}{3.795000in}}%
\pgfusepath{stroke}%
\end{pgfscope}%
\begin{pgfscope}%
\pgfsetbuttcap%
\pgfsetroundjoin%
\definecolor{currentfill}{rgb}{0.000000,0.000000,0.000000}%
\pgfsetfillcolor{currentfill}%
\pgfsetlinewidth{0.602250pt}%
\definecolor{currentstroke}{rgb}{0.000000,0.000000,0.000000}%
\pgfsetstrokecolor{currentstroke}%
\pgfsetdash{}{0pt}%
\pgfsys@defobject{currentmarker}{\pgfqpoint{0.000000in}{-0.027778in}}{\pgfqpoint{0.000000in}{0.000000in}}{%
\pgfpathmoveto{\pgfqpoint{0.000000in}{0.000000in}}%
\pgfpathlineto{\pgfqpoint{0.000000in}{-0.027778in}}%
\pgfusepath{stroke,fill}%
}%
\begin{pgfscope}%
\pgfsys@transformshift{1.911349in}{0.582778in}%
\pgfsys@useobject{currentmarker}{}%
\end{pgfscope}%
\end{pgfscope}%
\begin{pgfscope}%
\pgfpathrectangle{\pgfqpoint{0.752778in}{0.582778in}}{\pgfqpoint{4.048611in}{3.212222in}}%
\pgfusepath{clip}%
\pgfsetrectcap%
\pgfsetroundjoin%
\pgfsetlinewidth{0.803000pt}%
\definecolor{currentstroke}{rgb}{0.690196,0.690196,0.690196}%
\pgfsetstrokecolor{currentstroke}%
\pgfsetstrokeopacity{0.300000}%
\pgfsetdash{}{0pt}%
\pgfpathmoveto{\pgfqpoint{1.955706in}{0.582778in}}%
\pgfpathlineto{\pgfqpoint{1.955706in}{3.795000in}}%
\pgfusepath{stroke}%
\end{pgfscope}%
\begin{pgfscope}%
\pgfsetbuttcap%
\pgfsetroundjoin%
\definecolor{currentfill}{rgb}{0.000000,0.000000,0.000000}%
\pgfsetfillcolor{currentfill}%
\pgfsetlinewidth{0.602250pt}%
\definecolor{currentstroke}{rgb}{0.000000,0.000000,0.000000}%
\pgfsetstrokecolor{currentstroke}%
\pgfsetdash{}{0pt}%
\pgfsys@defobject{currentmarker}{\pgfqpoint{0.000000in}{-0.027778in}}{\pgfqpoint{0.000000in}{0.000000in}}{%
\pgfpathmoveto{\pgfqpoint{0.000000in}{0.000000in}}%
\pgfpathlineto{\pgfqpoint{0.000000in}{-0.027778in}}%
\pgfusepath{stroke,fill}%
}%
\begin{pgfscope}%
\pgfsys@transformshift{1.955706in}{0.582778in}%
\pgfsys@useobject{currentmarker}{}%
\end{pgfscope}%
\end{pgfscope}%
\begin{pgfscope}%
\pgfpathrectangle{\pgfqpoint{0.752778in}{0.582778in}}{\pgfqpoint{4.048611in}{3.212222in}}%
\pgfusepath{clip}%
\pgfsetrectcap%
\pgfsetroundjoin%
\pgfsetlinewidth{0.803000pt}%
\definecolor{currentstroke}{rgb}{0.690196,0.690196,0.690196}%
\pgfsetstrokecolor{currentstroke}%
\pgfsetstrokeopacity{0.300000}%
\pgfsetdash{}{0pt}%
\pgfpathmoveto{\pgfqpoint{2.000063in}{0.582778in}}%
\pgfpathlineto{\pgfqpoint{2.000063in}{3.795000in}}%
\pgfusepath{stroke}%
\end{pgfscope}%
\begin{pgfscope}%
\pgfsetbuttcap%
\pgfsetroundjoin%
\definecolor{currentfill}{rgb}{0.000000,0.000000,0.000000}%
\pgfsetfillcolor{currentfill}%
\pgfsetlinewidth{0.602250pt}%
\definecolor{currentstroke}{rgb}{0.000000,0.000000,0.000000}%
\pgfsetstrokecolor{currentstroke}%
\pgfsetdash{}{0pt}%
\pgfsys@defobject{currentmarker}{\pgfqpoint{0.000000in}{-0.027778in}}{\pgfqpoint{0.000000in}{0.000000in}}{%
\pgfpathmoveto{\pgfqpoint{0.000000in}{0.000000in}}%
\pgfpathlineto{\pgfqpoint{0.000000in}{-0.027778in}}%
\pgfusepath{stroke,fill}%
}%
\begin{pgfscope}%
\pgfsys@transformshift{2.000063in}{0.582778in}%
\pgfsys@useobject{currentmarker}{}%
\end{pgfscope}%
\end{pgfscope}%
\begin{pgfscope}%
\pgfpathrectangle{\pgfqpoint{0.752778in}{0.582778in}}{\pgfqpoint{4.048611in}{3.212222in}}%
\pgfusepath{clip}%
\pgfsetrectcap%
\pgfsetroundjoin%
\pgfsetlinewidth{0.803000pt}%
\definecolor{currentstroke}{rgb}{0.690196,0.690196,0.690196}%
\pgfsetstrokecolor{currentstroke}%
\pgfsetstrokeopacity{0.300000}%
\pgfsetdash{}{0pt}%
\pgfpathmoveto{\pgfqpoint{2.044420in}{0.582778in}}%
\pgfpathlineto{\pgfqpoint{2.044420in}{3.795000in}}%
\pgfusepath{stroke}%
\end{pgfscope}%
\begin{pgfscope}%
\pgfsetbuttcap%
\pgfsetroundjoin%
\definecolor{currentfill}{rgb}{0.000000,0.000000,0.000000}%
\pgfsetfillcolor{currentfill}%
\pgfsetlinewidth{0.602250pt}%
\definecolor{currentstroke}{rgb}{0.000000,0.000000,0.000000}%
\pgfsetstrokecolor{currentstroke}%
\pgfsetdash{}{0pt}%
\pgfsys@defobject{currentmarker}{\pgfqpoint{0.000000in}{-0.027778in}}{\pgfqpoint{0.000000in}{0.000000in}}{%
\pgfpathmoveto{\pgfqpoint{0.000000in}{0.000000in}}%
\pgfpathlineto{\pgfqpoint{0.000000in}{-0.027778in}}%
\pgfusepath{stroke,fill}%
}%
\begin{pgfscope}%
\pgfsys@transformshift{2.044420in}{0.582778in}%
\pgfsys@useobject{currentmarker}{}%
\end{pgfscope}%
\end{pgfscope}%
\begin{pgfscope}%
\pgfpathrectangle{\pgfqpoint{0.752778in}{0.582778in}}{\pgfqpoint{4.048611in}{3.212222in}}%
\pgfusepath{clip}%
\pgfsetrectcap%
\pgfsetroundjoin%
\pgfsetlinewidth{0.803000pt}%
\definecolor{currentstroke}{rgb}{0.690196,0.690196,0.690196}%
\pgfsetstrokecolor{currentstroke}%
\pgfsetstrokeopacity{0.300000}%
\pgfsetdash{}{0pt}%
\pgfpathmoveto{\pgfqpoint{2.088776in}{0.582778in}}%
\pgfpathlineto{\pgfqpoint{2.088776in}{3.795000in}}%
\pgfusepath{stroke}%
\end{pgfscope}%
\begin{pgfscope}%
\pgfsetbuttcap%
\pgfsetroundjoin%
\definecolor{currentfill}{rgb}{0.000000,0.000000,0.000000}%
\pgfsetfillcolor{currentfill}%
\pgfsetlinewidth{0.602250pt}%
\definecolor{currentstroke}{rgb}{0.000000,0.000000,0.000000}%
\pgfsetstrokecolor{currentstroke}%
\pgfsetdash{}{0pt}%
\pgfsys@defobject{currentmarker}{\pgfqpoint{0.000000in}{-0.027778in}}{\pgfqpoint{0.000000in}{0.000000in}}{%
\pgfpathmoveto{\pgfqpoint{0.000000in}{0.000000in}}%
\pgfpathlineto{\pgfqpoint{0.000000in}{-0.027778in}}%
\pgfusepath{stroke,fill}%
}%
\begin{pgfscope}%
\pgfsys@transformshift{2.088776in}{0.582778in}%
\pgfsys@useobject{currentmarker}{}%
\end{pgfscope}%
\end{pgfscope}%
\begin{pgfscope}%
\pgfpathrectangle{\pgfqpoint{0.752778in}{0.582778in}}{\pgfqpoint{4.048611in}{3.212222in}}%
\pgfusepath{clip}%
\pgfsetrectcap%
\pgfsetroundjoin%
\pgfsetlinewidth{0.803000pt}%
\definecolor{currentstroke}{rgb}{0.690196,0.690196,0.690196}%
\pgfsetstrokecolor{currentstroke}%
\pgfsetstrokeopacity{0.300000}%
\pgfsetdash{}{0pt}%
\pgfpathmoveto{\pgfqpoint{2.133133in}{0.582778in}}%
\pgfpathlineto{\pgfqpoint{2.133133in}{3.795000in}}%
\pgfusepath{stroke}%
\end{pgfscope}%
\begin{pgfscope}%
\pgfsetbuttcap%
\pgfsetroundjoin%
\definecolor{currentfill}{rgb}{0.000000,0.000000,0.000000}%
\pgfsetfillcolor{currentfill}%
\pgfsetlinewidth{0.602250pt}%
\definecolor{currentstroke}{rgb}{0.000000,0.000000,0.000000}%
\pgfsetstrokecolor{currentstroke}%
\pgfsetdash{}{0pt}%
\pgfsys@defobject{currentmarker}{\pgfqpoint{0.000000in}{-0.027778in}}{\pgfqpoint{0.000000in}{0.000000in}}{%
\pgfpathmoveto{\pgfqpoint{0.000000in}{0.000000in}}%
\pgfpathlineto{\pgfqpoint{0.000000in}{-0.027778in}}%
\pgfusepath{stroke,fill}%
}%
\begin{pgfscope}%
\pgfsys@transformshift{2.133133in}{0.582778in}%
\pgfsys@useobject{currentmarker}{}%
\end{pgfscope}%
\end{pgfscope}%
\begin{pgfscope}%
\pgfpathrectangle{\pgfqpoint{0.752778in}{0.582778in}}{\pgfqpoint{4.048611in}{3.212222in}}%
\pgfusepath{clip}%
\pgfsetrectcap%
\pgfsetroundjoin%
\pgfsetlinewidth{0.803000pt}%
\definecolor{currentstroke}{rgb}{0.690196,0.690196,0.690196}%
\pgfsetstrokecolor{currentstroke}%
\pgfsetstrokeopacity{0.300000}%
\pgfsetdash{}{0pt}%
\pgfpathmoveto{\pgfqpoint{2.221847in}{0.582778in}}%
\pgfpathlineto{\pgfqpoint{2.221847in}{3.795000in}}%
\pgfusepath{stroke}%
\end{pgfscope}%
\begin{pgfscope}%
\pgfsetbuttcap%
\pgfsetroundjoin%
\definecolor{currentfill}{rgb}{0.000000,0.000000,0.000000}%
\pgfsetfillcolor{currentfill}%
\pgfsetlinewidth{0.602250pt}%
\definecolor{currentstroke}{rgb}{0.000000,0.000000,0.000000}%
\pgfsetstrokecolor{currentstroke}%
\pgfsetdash{}{0pt}%
\pgfsys@defobject{currentmarker}{\pgfqpoint{0.000000in}{-0.027778in}}{\pgfqpoint{0.000000in}{0.000000in}}{%
\pgfpathmoveto{\pgfqpoint{0.000000in}{0.000000in}}%
\pgfpathlineto{\pgfqpoint{0.000000in}{-0.027778in}}%
\pgfusepath{stroke,fill}%
}%
\begin{pgfscope}%
\pgfsys@transformshift{2.221847in}{0.582778in}%
\pgfsys@useobject{currentmarker}{}%
\end{pgfscope}%
\end{pgfscope}%
\begin{pgfscope}%
\pgfpathrectangle{\pgfqpoint{0.752778in}{0.582778in}}{\pgfqpoint{4.048611in}{3.212222in}}%
\pgfusepath{clip}%
\pgfsetrectcap%
\pgfsetroundjoin%
\pgfsetlinewidth{0.803000pt}%
\definecolor{currentstroke}{rgb}{0.690196,0.690196,0.690196}%
\pgfsetstrokecolor{currentstroke}%
\pgfsetstrokeopacity{0.300000}%
\pgfsetdash{}{0pt}%
\pgfpathmoveto{\pgfqpoint{2.266204in}{0.582778in}}%
\pgfpathlineto{\pgfqpoint{2.266204in}{3.795000in}}%
\pgfusepath{stroke}%
\end{pgfscope}%
\begin{pgfscope}%
\pgfsetbuttcap%
\pgfsetroundjoin%
\definecolor{currentfill}{rgb}{0.000000,0.000000,0.000000}%
\pgfsetfillcolor{currentfill}%
\pgfsetlinewidth{0.602250pt}%
\definecolor{currentstroke}{rgb}{0.000000,0.000000,0.000000}%
\pgfsetstrokecolor{currentstroke}%
\pgfsetdash{}{0pt}%
\pgfsys@defobject{currentmarker}{\pgfqpoint{0.000000in}{-0.027778in}}{\pgfqpoint{0.000000in}{0.000000in}}{%
\pgfpathmoveto{\pgfqpoint{0.000000in}{0.000000in}}%
\pgfpathlineto{\pgfqpoint{0.000000in}{-0.027778in}}%
\pgfusepath{stroke,fill}%
}%
\begin{pgfscope}%
\pgfsys@transformshift{2.266204in}{0.582778in}%
\pgfsys@useobject{currentmarker}{}%
\end{pgfscope}%
\end{pgfscope}%
\begin{pgfscope}%
\pgfpathrectangle{\pgfqpoint{0.752778in}{0.582778in}}{\pgfqpoint{4.048611in}{3.212222in}}%
\pgfusepath{clip}%
\pgfsetrectcap%
\pgfsetroundjoin%
\pgfsetlinewidth{0.803000pt}%
\definecolor{currentstroke}{rgb}{0.690196,0.690196,0.690196}%
\pgfsetstrokecolor{currentstroke}%
\pgfsetstrokeopacity{0.300000}%
\pgfsetdash{}{0pt}%
\pgfpathmoveto{\pgfqpoint{2.310561in}{0.582778in}}%
\pgfpathlineto{\pgfqpoint{2.310561in}{3.795000in}}%
\pgfusepath{stroke}%
\end{pgfscope}%
\begin{pgfscope}%
\pgfsetbuttcap%
\pgfsetroundjoin%
\definecolor{currentfill}{rgb}{0.000000,0.000000,0.000000}%
\pgfsetfillcolor{currentfill}%
\pgfsetlinewidth{0.602250pt}%
\definecolor{currentstroke}{rgb}{0.000000,0.000000,0.000000}%
\pgfsetstrokecolor{currentstroke}%
\pgfsetdash{}{0pt}%
\pgfsys@defobject{currentmarker}{\pgfqpoint{0.000000in}{-0.027778in}}{\pgfqpoint{0.000000in}{0.000000in}}{%
\pgfpathmoveto{\pgfqpoint{0.000000in}{0.000000in}}%
\pgfpathlineto{\pgfqpoint{0.000000in}{-0.027778in}}%
\pgfusepath{stroke,fill}%
}%
\begin{pgfscope}%
\pgfsys@transformshift{2.310561in}{0.582778in}%
\pgfsys@useobject{currentmarker}{}%
\end{pgfscope}%
\end{pgfscope}%
\begin{pgfscope}%
\pgfpathrectangle{\pgfqpoint{0.752778in}{0.582778in}}{\pgfqpoint{4.048611in}{3.212222in}}%
\pgfusepath{clip}%
\pgfsetrectcap%
\pgfsetroundjoin%
\pgfsetlinewidth{0.803000pt}%
\definecolor{currentstroke}{rgb}{0.690196,0.690196,0.690196}%
\pgfsetstrokecolor{currentstroke}%
\pgfsetstrokeopacity{0.300000}%
\pgfsetdash{}{0pt}%
\pgfpathmoveto{\pgfqpoint{2.354918in}{0.582778in}}%
\pgfpathlineto{\pgfqpoint{2.354918in}{3.795000in}}%
\pgfusepath{stroke}%
\end{pgfscope}%
\begin{pgfscope}%
\pgfsetbuttcap%
\pgfsetroundjoin%
\definecolor{currentfill}{rgb}{0.000000,0.000000,0.000000}%
\pgfsetfillcolor{currentfill}%
\pgfsetlinewidth{0.602250pt}%
\definecolor{currentstroke}{rgb}{0.000000,0.000000,0.000000}%
\pgfsetstrokecolor{currentstroke}%
\pgfsetdash{}{0pt}%
\pgfsys@defobject{currentmarker}{\pgfqpoint{0.000000in}{-0.027778in}}{\pgfqpoint{0.000000in}{0.000000in}}{%
\pgfpathmoveto{\pgfqpoint{0.000000in}{0.000000in}}%
\pgfpathlineto{\pgfqpoint{0.000000in}{-0.027778in}}%
\pgfusepath{stroke,fill}%
}%
\begin{pgfscope}%
\pgfsys@transformshift{2.354918in}{0.582778in}%
\pgfsys@useobject{currentmarker}{}%
\end{pgfscope}%
\end{pgfscope}%
\begin{pgfscope}%
\pgfpathrectangle{\pgfqpoint{0.752778in}{0.582778in}}{\pgfqpoint{4.048611in}{3.212222in}}%
\pgfusepath{clip}%
\pgfsetrectcap%
\pgfsetroundjoin%
\pgfsetlinewidth{0.803000pt}%
\definecolor{currentstroke}{rgb}{0.690196,0.690196,0.690196}%
\pgfsetstrokecolor{currentstroke}%
\pgfsetstrokeopacity{0.300000}%
\pgfsetdash{}{0pt}%
\pgfpathmoveto{\pgfqpoint{2.399274in}{0.582778in}}%
\pgfpathlineto{\pgfqpoint{2.399274in}{3.795000in}}%
\pgfusepath{stroke}%
\end{pgfscope}%
\begin{pgfscope}%
\pgfsetbuttcap%
\pgfsetroundjoin%
\definecolor{currentfill}{rgb}{0.000000,0.000000,0.000000}%
\pgfsetfillcolor{currentfill}%
\pgfsetlinewidth{0.602250pt}%
\definecolor{currentstroke}{rgb}{0.000000,0.000000,0.000000}%
\pgfsetstrokecolor{currentstroke}%
\pgfsetdash{}{0pt}%
\pgfsys@defobject{currentmarker}{\pgfqpoint{0.000000in}{-0.027778in}}{\pgfqpoint{0.000000in}{0.000000in}}{%
\pgfpathmoveto{\pgfqpoint{0.000000in}{0.000000in}}%
\pgfpathlineto{\pgfqpoint{0.000000in}{-0.027778in}}%
\pgfusepath{stroke,fill}%
}%
\begin{pgfscope}%
\pgfsys@transformshift{2.399274in}{0.582778in}%
\pgfsys@useobject{currentmarker}{}%
\end{pgfscope}%
\end{pgfscope}%
\begin{pgfscope}%
\pgfpathrectangle{\pgfqpoint{0.752778in}{0.582778in}}{\pgfqpoint{4.048611in}{3.212222in}}%
\pgfusepath{clip}%
\pgfsetrectcap%
\pgfsetroundjoin%
\pgfsetlinewidth{0.803000pt}%
\definecolor{currentstroke}{rgb}{0.690196,0.690196,0.690196}%
\pgfsetstrokecolor{currentstroke}%
\pgfsetstrokeopacity{0.300000}%
\pgfsetdash{}{0pt}%
\pgfpathmoveto{\pgfqpoint{2.443631in}{0.582778in}}%
\pgfpathlineto{\pgfqpoint{2.443631in}{3.795000in}}%
\pgfusepath{stroke}%
\end{pgfscope}%
\begin{pgfscope}%
\pgfsetbuttcap%
\pgfsetroundjoin%
\definecolor{currentfill}{rgb}{0.000000,0.000000,0.000000}%
\pgfsetfillcolor{currentfill}%
\pgfsetlinewidth{0.602250pt}%
\definecolor{currentstroke}{rgb}{0.000000,0.000000,0.000000}%
\pgfsetstrokecolor{currentstroke}%
\pgfsetdash{}{0pt}%
\pgfsys@defobject{currentmarker}{\pgfqpoint{0.000000in}{-0.027778in}}{\pgfqpoint{0.000000in}{0.000000in}}{%
\pgfpathmoveto{\pgfqpoint{0.000000in}{0.000000in}}%
\pgfpathlineto{\pgfqpoint{0.000000in}{-0.027778in}}%
\pgfusepath{stroke,fill}%
}%
\begin{pgfscope}%
\pgfsys@transformshift{2.443631in}{0.582778in}%
\pgfsys@useobject{currentmarker}{}%
\end{pgfscope}%
\end{pgfscope}%
\begin{pgfscope}%
\pgfpathrectangle{\pgfqpoint{0.752778in}{0.582778in}}{\pgfqpoint{4.048611in}{3.212222in}}%
\pgfusepath{clip}%
\pgfsetrectcap%
\pgfsetroundjoin%
\pgfsetlinewidth{0.803000pt}%
\definecolor{currentstroke}{rgb}{0.690196,0.690196,0.690196}%
\pgfsetstrokecolor{currentstroke}%
\pgfsetstrokeopacity{0.300000}%
\pgfsetdash{}{0pt}%
\pgfpathmoveto{\pgfqpoint{2.487988in}{0.582778in}}%
\pgfpathlineto{\pgfqpoint{2.487988in}{3.795000in}}%
\pgfusepath{stroke}%
\end{pgfscope}%
\begin{pgfscope}%
\pgfsetbuttcap%
\pgfsetroundjoin%
\definecolor{currentfill}{rgb}{0.000000,0.000000,0.000000}%
\pgfsetfillcolor{currentfill}%
\pgfsetlinewidth{0.602250pt}%
\definecolor{currentstroke}{rgb}{0.000000,0.000000,0.000000}%
\pgfsetstrokecolor{currentstroke}%
\pgfsetdash{}{0pt}%
\pgfsys@defobject{currentmarker}{\pgfqpoint{0.000000in}{-0.027778in}}{\pgfqpoint{0.000000in}{0.000000in}}{%
\pgfpathmoveto{\pgfqpoint{0.000000in}{0.000000in}}%
\pgfpathlineto{\pgfqpoint{0.000000in}{-0.027778in}}%
\pgfusepath{stroke,fill}%
}%
\begin{pgfscope}%
\pgfsys@transformshift{2.487988in}{0.582778in}%
\pgfsys@useobject{currentmarker}{}%
\end{pgfscope}%
\end{pgfscope}%
\begin{pgfscope}%
\pgfpathrectangle{\pgfqpoint{0.752778in}{0.582778in}}{\pgfqpoint{4.048611in}{3.212222in}}%
\pgfusepath{clip}%
\pgfsetrectcap%
\pgfsetroundjoin%
\pgfsetlinewidth{0.803000pt}%
\definecolor{currentstroke}{rgb}{0.690196,0.690196,0.690196}%
\pgfsetstrokecolor{currentstroke}%
\pgfsetstrokeopacity{0.300000}%
\pgfsetdash{}{0pt}%
\pgfpathmoveto{\pgfqpoint{2.532345in}{0.582778in}}%
\pgfpathlineto{\pgfqpoint{2.532345in}{3.795000in}}%
\pgfusepath{stroke}%
\end{pgfscope}%
\begin{pgfscope}%
\pgfsetbuttcap%
\pgfsetroundjoin%
\definecolor{currentfill}{rgb}{0.000000,0.000000,0.000000}%
\pgfsetfillcolor{currentfill}%
\pgfsetlinewidth{0.602250pt}%
\definecolor{currentstroke}{rgb}{0.000000,0.000000,0.000000}%
\pgfsetstrokecolor{currentstroke}%
\pgfsetdash{}{0pt}%
\pgfsys@defobject{currentmarker}{\pgfqpoint{0.000000in}{-0.027778in}}{\pgfqpoint{0.000000in}{0.000000in}}{%
\pgfpathmoveto{\pgfqpoint{0.000000in}{0.000000in}}%
\pgfpathlineto{\pgfqpoint{0.000000in}{-0.027778in}}%
\pgfusepath{stroke,fill}%
}%
\begin{pgfscope}%
\pgfsys@transformshift{2.532345in}{0.582778in}%
\pgfsys@useobject{currentmarker}{}%
\end{pgfscope}%
\end{pgfscope}%
\begin{pgfscope}%
\pgfpathrectangle{\pgfqpoint{0.752778in}{0.582778in}}{\pgfqpoint{4.048611in}{3.212222in}}%
\pgfusepath{clip}%
\pgfsetrectcap%
\pgfsetroundjoin%
\pgfsetlinewidth{0.803000pt}%
\definecolor{currentstroke}{rgb}{0.690196,0.690196,0.690196}%
\pgfsetstrokecolor{currentstroke}%
\pgfsetstrokeopacity{0.300000}%
\pgfsetdash{}{0pt}%
\pgfpathmoveto{\pgfqpoint{2.576702in}{0.582778in}}%
\pgfpathlineto{\pgfqpoint{2.576702in}{3.795000in}}%
\pgfusepath{stroke}%
\end{pgfscope}%
\begin{pgfscope}%
\pgfsetbuttcap%
\pgfsetroundjoin%
\definecolor{currentfill}{rgb}{0.000000,0.000000,0.000000}%
\pgfsetfillcolor{currentfill}%
\pgfsetlinewidth{0.602250pt}%
\definecolor{currentstroke}{rgb}{0.000000,0.000000,0.000000}%
\pgfsetstrokecolor{currentstroke}%
\pgfsetdash{}{0pt}%
\pgfsys@defobject{currentmarker}{\pgfqpoint{0.000000in}{-0.027778in}}{\pgfqpoint{0.000000in}{0.000000in}}{%
\pgfpathmoveto{\pgfqpoint{0.000000in}{0.000000in}}%
\pgfpathlineto{\pgfqpoint{0.000000in}{-0.027778in}}%
\pgfusepath{stroke,fill}%
}%
\begin{pgfscope}%
\pgfsys@transformshift{2.576702in}{0.582778in}%
\pgfsys@useobject{currentmarker}{}%
\end{pgfscope}%
\end{pgfscope}%
\begin{pgfscope}%
\pgfpathrectangle{\pgfqpoint{0.752778in}{0.582778in}}{\pgfqpoint{4.048611in}{3.212222in}}%
\pgfusepath{clip}%
\pgfsetrectcap%
\pgfsetroundjoin%
\pgfsetlinewidth{0.803000pt}%
\definecolor{currentstroke}{rgb}{0.690196,0.690196,0.690196}%
\pgfsetstrokecolor{currentstroke}%
\pgfsetstrokeopacity{0.300000}%
\pgfsetdash{}{0pt}%
\pgfpathmoveto{\pgfqpoint{2.665416in}{0.582778in}}%
\pgfpathlineto{\pgfqpoint{2.665416in}{3.795000in}}%
\pgfusepath{stroke}%
\end{pgfscope}%
\begin{pgfscope}%
\pgfsetbuttcap%
\pgfsetroundjoin%
\definecolor{currentfill}{rgb}{0.000000,0.000000,0.000000}%
\pgfsetfillcolor{currentfill}%
\pgfsetlinewidth{0.602250pt}%
\definecolor{currentstroke}{rgb}{0.000000,0.000000,0.000000}%
\pgfsetstrokecolor{currentstroke}%
\pgfsetdash{}{0pt}%
\pgfsys@defobject{currentmarker}{\pgfqpoint{0.000000in}{-0.027778in}}{\pgfqpoint{0.000000in}{0.000000in}}{%
\pgfpathmoveto{\pgfqpoint{0.000000in}{0.000000in}}%
\pgfpathlineto{\pgfqpoint{0.000000in}{-0.027778in}}%
\pgfusepath{stroke,fill}%
}%
\begin{pgfscope}%
\pgfsys@transformshift{2.665416in}{0.582778in}%
\pgfsys@useobject{currentmarker}{}%
\end{pgfscope}%
\end{pgfscope}%
\begin{pgfscope}%
\pgfpathrectangle{\pgfqpoint{0.752778in}{0.582778in}}{\pgfqpoint{4.048611in}{3.212222in}}%
\pgfusepath{clip}%
\pgfsetrectcap%
\pgfsetroundjoin%
\pgfsetlinewidth{0.803000pt}%
\definecolor{currentstroke}{rgb}{0.690196,0.690196,0.690196}%
\pgfsetstrokecolor{currentstroke}%
\pgfsetstrokeopacity{0.300000}%
\pgfsetdash{}{0pt}%
\pgfpathmoveto{\pgfqpoint{2.709772in}{0.582778in}}%
\pgfpathlineto{\pgfqpoint{2.709772in}{3.795000in}}%
\pgfusepath{stroke}%
\end{pgfscope}%
\begin{pgfscope}%
\pgfsetbuttcap%
\pgfsetroundjoin%
\definecolor{currentfill}{rgb}{0.000000,0.000000,0.000000}%
\pgfsetfillcolor{currentfill}%
\pgfsetlinewidth{0.602250pt}%
\definecolor{currentstroke}{rgb}{0.000000,0.000000,0.000000}%
\pgfsetstrokecolor{currentstroke}%
\pgfsetdash{}{0pt}%
\pgfsys@defobject{currentmarker}{\pgfqpoint{0.000000in}{-0.027778in}}{\pgfqpoint{0.000000in}{0.000000in}}{%
\pgfpathmoveto{\pgfqpoint{0.000000in}{0.000000in}}%
\pgfpathlineto{\pgfqpoint{0.000000in}{-0.027778in}}%
\pgfusepath{stroke,fill}%
}%
\begin{pgfscope}%
\pgfsys@transformshift{2.709772in}{0.582778in}%
\pgfsys@useobject{currentmarker}{}%
\end{pgfscope}%
\end{pgfscope}%
\begin{pgfscope}%
\pgfpathrectangle{\pgfqpoint{0.752778in}{0.582778in}}{\pgfqpoint{4.048611in}{3.212222in}}%
\pgfusepath{clip}%
\pgfsetrectcap%
\pgfsetroundjoin%
\pgfsetlinewidth{0.803000pt}%
\definecolor{currentstroke}{rgb}{0.690196,0.690196,0.690196}%
\pgfsetstrokecolor{currentstroke}%
\pgfsetstrokeopacity{0.300000}%
\pgfsetdash{}{0pt}%
\pgfpathmoveto{\pgfqpoint{2.754129in}{0.582778in}}%
\pgfpathlineto{\pgfqpoint{2.754129in}{3.795000in}}%
\pgfusepath{stroke}%
\end{pgfscope}%
\begin{pgfscope}%
\pgfsetbuttcap%
\pgfsetroundjoin%
\definecolor{currentfill}{rgb}{0.000000,0.000000,0.000000}%
\pgfsetfillcolor{currentfill}%
\pgfsetlinewidth{0.602250pt}%
\definecolor{currentstroke}{rgb}{0.000000,0.000000,0.000000}%
\pgfsetstrokecolor{currentstroke}%
\pgfsetdash{}{0pt}%
\pgfsys@defobject{currentmarker}{\pgfqpoint{0.000000in}{-0.027778in}}{\pgfqpoint{0.000000in}{0.000000in}}{%
\pgfpathmoveto{\pgfqpoint{0.000000in}{0.000000in}}%
\pgfpathlineto{\pgfqpoint{0.000000in}{-0.027778in}}%
\pgfusepath{stroke,fill}%
}%
\begin{pgfscope}%
\pgfsys@transformshift{2.754129in}{0.582778in}%
\pgfsys@useobject{currentmarker}{}%
\end{pgfscope}%
\end{pgfscope}%
\begin{pgfscope}%
\pgfpathrectangle{\pgfqpoint{0.752778in}{0.582778in}}{\pgfqpoint{4.048611in}{3.212222in}}%
\pgfusepath{clip}%
\pgfsetrectcap%
\pgfsetroundjoin%
\pgfsetlinewidth{0.803000pt}%
\definecolor{currentstroke}{rgb}{0.690196,0.690196,0.690196}%
\pgfsetstrokecolor{currentstroke}%
\pgfsetstrokeopacity{0.300000}%
\pgfsetdash{}{0pt}%
\pgfpathmoveto{\pgfqpoint{2.798486in}{0.582778in}}%
\pgfpathlineto{\pgfqpoint{2.798486in}{3.795000in}}%
\pgfusepath{stroke}%
\end{pgfscope}%
\begin{pgfscope}%
\pgfsetbuttcap%
\pgfsetroundjoin%
\definecolor{currentfill}{rgb}{0.000000,0.000000,0.000000}%
\pgfsetfillcolor{currentfill}%
\pgfsetlinewidth{0.602250pt}%
\definecolor{currentstroke}{rgb}{0.000000,0.000000,0.000000}%
\pgfsetstrokecolor{currentstroke}%
\pgfsetdash{}{0pt}%
\pgfsys@defobject{currentmarker}{\pgfqpoint{0.000000in}{-0.027778in}}{\pgfqpoint{0.000000in}{0.000000in}}{%
\pgfpathmoveto{\pgfqpoint{0.000000in}{0.000000in}}%
\pgfpathlineto{\pgfqpoint{0.000000in}{-0.027778in}}%
\pgfusepath{stroke,fill}%
}%
\begin{pgfscope}%
\pgfsys@transformshift{2.798486in}{0.582778in}%
\pgfsys@useobject{currentmarker}{}%
\end{pgfscope}%
\end{pgfscope}%
\begin{pgfscope}%
\pgfpathrectangle{\pgfqpoint{0.752778in}{0.582778in}}{\pgfqpoint{4.048611in}{3.212222in}}%
\pgfusepath{clip}%
\pgfsetrectcap%
\pgfsetroundjoin%
\pgfsetlinewidth{0.803000pt}%
\definecolor{currentstroke}{rgb}{0.690196,0.690196,0.690196}%
\pgfsetstrokecolor{currentstroke}%
\pgfsetstrokeopacity{0.300000}%
\pgfsetdash{}{0pt}%
\pgfpathmoveto{\pgfqpoint{2.842843in}{0.582778in}}%
\pgfpathlineto{\pgfqpoint{2.842843in}{3.795000in}}%
\pgfusepath{stroke}%
\end{pgfscope}%
\begin{pgfscope}%
\pgfsetbuttcap%
\pgfsetroundjoin%
\definecolor{currentfill}{rgb}{0.000000,0.000000,0.000000}%
\pgfsetfillcolor{currentfill}%
\pgfsetlinewidth{0.602250pt}%
\definecolor{currentstroke}{rgb}{0.000000,0.000000,0.000000}%
\pgfsetstrokecolor{currentstroke}%
\pgfsetdash{}{0pt}%
\pgfsys@defobject{currentmarker}{\pgfqpoint{0.000000in}{-0.027778in}}{\pgfqpoint{0.000000in}{0.000000in}}{%
\pgfpathmoveto{\pgfqpoint{0.000000in}{0.000000in}}%
\pgfpathlineto{\pgfqpoint{0.000000in}{-0.027778in}}%
\pgfusepath{stroke,fill}%
}%
\begin{pgfscope}%
\pgfsys@transformshift{2.842843in}{0.582778in}%
\pgfsys@useobject{currentmarker}{}%
\end{pgfscope}%
\end{pgfscope}%
\begin{pgfscope}%
\pgfpathrectangle{\pgfqpoint{0.752778in}{0.582778in}}{\pgfqpoint{4.048611in}{3.212222in}}%
\pgfusepath{clip}%
\pgfsetrectcap%
\pgfsetroundjoin%
\pgfsetlinewidth{0.803000pt}%
\definecolor{currentstroke}{rgb}{0.690196,0.690196,0.690196}%
\pgfsetstrokecolor{currentstroke}%
\pgfsetstrokeopacity{0.300000}%
\pgfsetdash{}{0pt}%
\pgfpathmoveto{\pgfqpoint{2.887200in}{0.582778in}}%
\pgfpathlineto{\pgfqpoint{2.887200in}{3.795000in}}%
\pgfusepath{stroke}%
\end{pgfscope}%
\begin{pgfscope}%
\pgfsetbuttcap%
\pgfsetroundjoin%
\definecolor{currentfill}{rgb}{0.000000,0.000000,0.000000}%
\pgfsetfillcolor{currentfill}%
\pgfsetlinewidth{0.602250pt}%
\definecolor{currentstroke}{rgb}{0.000000,0.000000,0.000000}%
\pgfsetstrokecolor{currentstroke}%
\pgfsetdash{}{0pt}%
\pgfsys@defobject{currentmarker}{\pgfqpoint{0.000000in}{-0.027778in}}{\pgfqpoint{0.000000in}{0.000000in}}{%
\pgfpathmoveto{\pgfqpoint{0.000000in}{0.000000in}}%
\pgfpathlineto{\pgfqpoint{0.000000in}{-0.027778in}}%
\pgfusepath{stroke,fill}%
}%
\begin{pgfscope}%
\pgfsys@transformshift{2.887200in}{0.582778in}%
\pgfsys@useobject{currentmarker}{}%
\end{pgfscope}%
\end{pgfscope}%
\begin{pgfscope}%
\pgfpathrectangle{\pgfqpoint{0.752778in}{0.582778in}}{\pgfqpoint{4.048611in}{3.212222in}}%
\pgfusepath{clip}%
\pgfsetrectcap%
\pgfsetroundjoin%
\pgfsetlinewidth{0.803000pt}%
\definecolor{currentstroke}{rgb}{0.690196,0.690196,0.690196}%
\pgfsetstrokecolor{currentstroke}%
\pgfsetstrokeopacity{0.300000}%
\pgfsetdash{}{0pt}%
\pgfpathmoveto{\pgfqpoint{2.931557in}{0.582778in}}%
\pgfpathlineto{\pgfqpoint{2.931557in}{3.795000in}}%
\pgfusepath{stroke}%
\end{pgfscope}%
\begin{pgfscope}%
\pgfsetbuttcap%
\pgfsetroundjoin%
\definecolor{currentfill}{rgb}{0.000000,0.000000,0.000000}%
\pgfsetfillcolor{currentfill}%
\pgfsetlinewidth{0.602250pt}%
\definecolor{currentstroke}{rgb}{0.000000,0.000000,0.000000}%
\pgfsetstrokecolor{currentstroke}%
\pgfsetdash{}{0pt}%
\pgfsys@defobject{currentmarker}{\pgfqpoint{0.000000in}{-0.027778in}}{\pgfqpoint{0.000000in}{0.000000in}}{%
\pgfpathmoveto{\pgfqpoint{0.000000in}{0.000000in}}%
\pgfpathlineto{\pgfqpoint{0.000000in}{-0.027778in}}%
\pgfusepath{stroke,fill}%
}%
\begin{pgfscope}%
\pgfsys@transformshift{2.931557in}{0.582778in}%
\pgfsys@useobject{currentmarker}{}%
\end{pgfscope}%
\end{pgfscope}%
\begin{pgfscope}%
\pgfpathrectangle{\pgfqpoint{0.752778in}{0.582778in}}{\pgfqpoint{4.048611in}{3.212222in}}%
\pgfusepath{clip}%
\pgfsetrectcap%
\pgfsetroundjoin%
\pgfsetlinewidth{0.803000pt}%
\definecolor{currentstroke}{rgb}{0.690196,0.690196,0.690196}%
\pgfsetstrokecolor{currentstroke}%
\pgfsetstrokeopacity{0.300000}%
\pgfsetdash{}{0pt}%
\pgfpathmoveto{\pgfqpoint{2.975913in}{0.582778in}}%
\pgfpathlineto{\pgfqpoint{2.975913in}{3.795000in}}%
\pgfusepath{stroke}%
\end{pgfscope}%
\begin{pgfscope}%
\pgfsetbuttcap%
\pgfsetroundjoin%
\definecolor{currentfill}{rgb}{0.000000,0.000000,0.000000}%
\pgfsetfillcolor{currentfill}%
\pgfsetlinewidth{0.602250pt}%
\definecolor{currentstroke}{rgb}{0.000000,0.000000,0.000000}%
\pgfsetstrokecolor{currentstroke}%
\pgfsetdash{}{0pt}%
\pgfsys@defobject{currentmarker}{\pgfqpoint{0.000000in}{-0.027778in}}{\pgfqpoint{0.000000in}{0.000000in}}{%
\pgfpathmoveto{\pgfqpoint{0.000000in}{0.000000in}}%
\pgfpathlineto{\pgfqpoint{0.000000in}{-0.027778in}}%
\pgfusepath{stroke,fill}%
}%
\begin{pgfscope}%
\pgfsys@transformshift{2.975913in}{0.582778in}%
\pgfsys@useobject{currentmarker}{}%
\end{pgfscope}%
\end{pgfscope}%
\begin{pgfscope}%
\pgfpathrectangle{\pgfqpoint{0.752778in}{0.582778in}}{\pgfqpoint{4.048611in}{3.212222in}}%
\pgfusepath{clip}%
\pgfsetrectcap%
\pgfsetroundjoin%
\pgfsetlinewidth{0.803000pt}%
\definecolor{currentstroke}{rgb}{0.690196,0.690196,0.690196}%
\pgfsetstrokecolor{currentstroke}%
\pgfsetstrokeopacity{0.300000}%
\pgfsetdash{}{0pt}%
\pgfpathmoveto{\pgfqpoint{3.020270in}{0.582778in}}%
\pgfpathlineto{\pgfqpoint{3.020270in}{3.795000in}}%
\pgfusepath{stroke}%
\end{pgfscope}%
\begin{pgfscope}%
\pgfsetbuttcap%
\pgfsetroundjoin%
\definecolor{currentfill}{rgb}{0.000000,0.000000,0.000000}%
\pgfsetfillcolor{currentfill}%
\pgfsetlinewidth{0.602250pt}%
\definecolor{currentstroke}{rgb}{0.000000,0.000000,0.000000}%
\pgfsetstrokecolor{currentstroke}%
\pgfsetdash{}{0pt}%
\pgfsys@defobject{currentmarker}{\pgfqpoint{0.000000in}{-0.027778in}}{\pgfqpoint{0.000000in}{0.000000in}}{%
\pgfpathmoveto{\pgfqpoint{0.000000in}{0.000000in}}%
\pgfpathlineto{\pgfqpoint{0.000000in}{-0.027778in}}%
\pgfusepath{stroke,fill}%
}%
\begin{pgfscope}%
\pgfsys@transformshift{3.020270in}{0.582778in}%
\pgfsys@useobject{currentmarker}{}%
\end{pgfscope}%
\end{pgfscope}%
\begin{pgfscope}%
\pgfpathrectangle{\pgfqpoint{0.752778in}{0.582778in}}{\pgfqpoint{4.048611in}{3.212222in}}%
\pgfusepath{clip}%
\pgfsetrectcap%
\pgfsetroundjoin%
\pgfsetlinewidth{0.803000pt}%
\definecolor{currentstroke}{rgb}{0.690196,0.690196,0.690196}%
\pgfsetstrokecolor{currentstroke}%
\pgfsetstrokeopacity{0.300000}%
\pgfsetdash{}{0pt}%
\pgfpathmoveto{\pgfqpoint{3.108984in}{0.582778in}}%
\pgfpathlineto{\pgfqpoint{3.108984in}{3.795000in}}%
\pgfusepath{stroke}%
\end{pgfscope}%
\begin{pgfscope}%
\pgfsetbuttcap%
\pgfsetroundjoin%
\definecolor{currentfill}{rgb}{0.000000,0.000000,0.000000}%
\pgfsetfillcolor{currentfill}%
\pgfsetlinewidth{0.602250pt}%
\definecolor{currentstroke}{rgb}{0.000000,0.000000,0.000000}%
\pgfsetstrokecolor{currentstroke}%
\pgfsetdash{}{0pt}%
\pgfsys@defobject{currentmarker}{\pgfqpoint{0.000000in}{-0.027778in}}{\pgfqpoint{0.000000in}{0.000000in}}{%
\pgfpathmoveto{\pgfqpoint{0.000000in}{0.000000in}}%
\pgfpathlineto{\pgfqpoint{0.000000in}{-0.027778in}}%
\pgfusepath{stroke,fill}%
}%
\begin{pgfscope}%
\pgfsys@transformshift{3.108984in}{0.582778in}%
\pgfsys@useobject{currentmarker}{}%
\end{pgfscope}%
\end{pgfscope}%
\begin{pgfscope}%
\pgfpathrectangle{\pgfqpoint{0.752778in}{0.582778in}}{\pgfqpoint{4.048611in}{3.212222in}}%
\pgfusepath{clip}%
\pgfsetrectcap%
\pgfsetroundjoin%
\pgfsetlinewidth{0.803000pt}%
\definecolor{currentstroke}{rgb}{0.690196,0.690196,0.690196}%
\pgfsetstrokecolor{currentstroke}%
\pgfsetstrokeopacity{0.300000}%
\pgfsetdash{}{0pt}%
\pgfpathmoveto{\pgfqpoint{3.153341in}{0.582778in}}%
\pgfpathlineto{\pgfqpoint{3.153341in}{3.795000in}}%
\pgfusepath{stroke}%
\end{pgfscope}%
\begin{pgfscope}%
\pgfsetbuttcap%
\pgfsetroundjoin%
\definecolor{currentfill}{rgb}{0.000000,0.000000,0.000000}%
\pgfsetfillcolor{currentfill}%
\pgfsetlinewidth{0.602250pt}%
\definecolor{currentstroke}{rgb}{0.000000,0.000000,0.000000}%
\pgfsetstrokecolor{currentstroke}%
\pgfsetdash{}{0pt}%
\pgfsys@defobject{currentmarker}{\pgfqpoint{0.000000in}{-0.027778in}}{\pgfqpoint{0.000000in}{0.000000in}}{%
\pgfpathmoveto{\pgfqpoint{0.000000in}{0.000000in}}%
\pgfpathlineto{\pgfqpoint{0.000000in}{-0.027778in}}%
\pgfusepath{stroke,fill}%
}%
\begin{pgfscope}%
\pgfsys@transformshift{3.153341in}{0.582778in}%
\pgfsys@useobject{currentmarker}{}%
\end{pgfscope}%
\end{pgfscope}%
\begin{pgfscope}%
\pgfpathrectangle{\pgfqpoint{0.752778in}{0.582778in}}{\pgfqpoint{4.048611in}{3.212222in}}%
\pgfusepath{clip}%
\pgfsetrectcap%
\pgfsetroundjoin%
\pgfsetlinewidth{0.803000pt}%
\definecolor{currentstroke}{rgb}{0.690196,0.690196,0.690196}%
\pgfsetstrokecolor{currentstroke}%
\pgfsetstrokeopacity{0.300000}%
\pgfsetdash{}{0pt}%
\pgfpathmoveto{\pgfqpoint{3.197698in}{0.582778in}}%
\pgfpathlineto{\pgfqpoint{3.197698in}{3.795000in}}%
\pgfusepath{stroke}%
\end{pgfscope}%
\begin{pgfscope}%
\pgfsetbuttcap%
\pgfsetroundjoin%
\definecolor{currentfill}{rgb}{0.000000,0.000000,0.000000}%
\pgfsetfillcolor{currentfill}%
\pgfsetlinewidth{0.602250pt}%
\definecolor{currentstroke}{rgb}{0.000000,0.000000,0.000000}%
\pgfsetstrokecolor{currentstroke}%
\pgfsetdash{}{0pt}%
\pgfsys@defobject{currentmarker}{\pgfqpoint{0.000000in}{-0.027778in}}{\pgfqpoint{0.000000in}{0.000000in}}{%
\pgfpathmoveto{\pgfqpoint{0.000000in}{0.000000in}}%
\pgfpathlineto{\pgfqpoint{0.000000in}{-0.027778in}}%
\pgfusepath{stroke,fill}%
}%
\begin{pgfscope}%
\pgfsys@transformshift{3.197698in}{0.582778in}%
\pgfsys@useobject{currentmarker}{}%
\end{pgfscope}%
\end{pgfscope}%
\begin{pgfscope}%
\pgfpathrectangle{\pgfqpoint{0.752778in}{0.582778in}}{\pgfqpoint{4.048611in}{3.212222in}}%
\pgfusepath{clip}%
\pgfsetrectcap%
\pgfsetroundjoin%
\pgfsetlinewidth{0.803000pt}%
\definecolor{currentstroke}{rgb}{0.690196,0.690196,0.690196}%
\pgfsetstrokecolor{currentstroke}%
\pgfsetstrokeopacity{0.300000}%
\pgfsetdash{}{0pt}%
\pgfpathmoveto{\pgfqpoint{3.242055in}{0.582778in}}%
\pgfpathlineto{\pgfqpoint{3.242055in}{3.795000in}}%
\pgfusepath{stroke}%
\end{pgfscope}%
\begin{pgfscope}%
\pgfsetbuttcap%
\pgfsetroundjoin%
\definecolor{currentfill}{rgb}{0.000000,0.000000,0.000000}%
\pgfsetfillcolor{currentfill}%
\pgfsetlinewidth{0.602250pt}%
\definecolor{currentstroke}{rgb}{0.000000,0.000000,0.000000}%
\pgfsetstrokecolor{currentstroke}%
\pgfsetdash{}{0pt}%
\pgfsys@defobject{currentmarker}{\pgfqpoint{0.000000in}{-0.027778in}}{\pgfqpoint{0.000000in}{0.000000in}}{%
\pgfpathmoveto{\pgfqpoint{0.000000in}{0.000000in}}%
\pgfpathlineto{\pgfqpoint{0.000000in}{-0.027778in}}%
\pgfusepath{stroke,fill}%
}%
\begin{pgfscope}%
\pgfsys@transformshift{3.242055in}{0.582778in}%
\pgfsys@useobject{currentmarker}{}%
\end{pgfscope}%
\end{pgfscope}%
\begin{pgfscope}%
\pgfpathrectangle{\pgfqpoint{0.752778in}{0.582778in}}{\pgfqpoint{4.048611in}{3.212222in}}%
\pgfusepath{clip}%
\pgfsetrectcap%
\pgfsetroundjoin%
\pgfsetlinewidth{0.803000pt}%
\definecolor{currentstroke}{rgb}{0.690196,0.690196,0.690196}%
\pgfsetstrokecolor{currentstroke}%
\pgfsetstrokeopacity{0.300000}%
\pgfsetdash{}{0pt}%
\pgfpathmoveto{\pgfqpoint{3.286411in}{0.582778in}}%
\pgfpathlineto{\pgfqpoint{3.286411in}{3.795000in}}%
\pgfusepath{stroke}%
\end{pgfscope}%
\begin{pgfscope}%
\pgfsetbuttcap%
\pgfsetroundjoin%
\definecolor{currentfill}{rgb}{0.000000,0.000000,0.000000}%
\pgfsetfillcolor{currentfill}%
\pgfsetlinewidth{0.602250pt}%
\definecolor{currentstroke}{rgb}{0.000000,0.000000,0.000000}%
\pgfsetstrokecolor{currentstroke}%
\pgfsetdash{}{0pt}%
\pgfsys@defobject{currentmarker}{\pgfqpoint{0.000000in}{-0.027778in}}{\pgfqpoint{0.000000in}{0.000000in}}{%
\pgfpathmoveto{\pgfqpoint{0.000000in}{0.000000in}}%
\pgfpathlineto{\pgfqpoint{0.000000in}{-0.027778in}}%
\pgfusepath{stroke,fill}%
}%
\begin{pgfscope}%
\pgfsys@transformshift{3.286411in}{0.582778in}%
\pgfsys@useobject{currentmarker}{}%
\end{pgfscope}%
\end{pgfscope}%
\begin{pgfscope}%
\pgfpathrectangle{\pgfqpoint{0.752778in}{0.582778in}}{\pgfqpoint{4.048611in}{3.212222in}}%
\pgfusepath{clip}%
\pgfsetrectcap%
\pgfsetroundjoin%
\pgfsetlinewidth{0.803000pt}%
\definecolor{currentstroke}{rgb}{0.690196,0.690196,0.690196}%
\pgfsetstrokecolor{currentstroke}%
\pgfsetstrokeopacity{0.300000}%
\pgfsetdash{}{0pt}%
\pgfpathmoveto{\pgfqpoint{3.330768in}{0.582778in}}%
\pgfpathlineto{\pgfqpoint{3.330768in}{3.795000in}}%
\pgfusepath{stroke}%
\end{pgfscope}%
\begin{pgfscope}%
\pgfsetbuttcap%
\pgfsetroundjoin%
\definecolor{currentfill}{rgb}{0.000000,0.000000,0.000000}%
\pgfsetfillcolor{currentfill}%
\pgfsetlinewidth{0.602250pt}%
\definecolor{currentstroke}{rgb}{0.000000,0.000000,0.000000}%
\pgfsetstrokecolor{currentstroke}%
\pgfsetdash{}{0pt}%
\pgfsys@defobject{currentmarker}{\pgfqpoint{0.000000in}{-0.027778in}}{\pgfqpoint{0.000000in}{0.000000in}}{%
\pgfpathmoveto{\pgfqpoint{0.000000in}{0.000000in}}%
\pgfpathlineto{\pgfqpoint{0.000000in}{-0.027778in}}%
\pgfusepath{stroke,fill}%
}%
\begin{pgfscope}%
\pgfsys@transformshift{3.330768in}{0.582778in}%
\pgfsys@useobject{currentmarker}{}%
\end{pgfscope}%
\end{pgfscope}%
\begin{pgfscope}%
\pgfpathrectangle{\pgfqpoint{0.752778in}{0.582778in}}{\pgfqpoint{4.048611in}{3.212222in}}%
\pgfusepath{clip}%
\pgfsetrectcap%
\pgfsetroundjoin%
\pgfsetlinewidth{0.803000pt}%
\definecolor{currentstroke}{rgb}{0.690196,0.690196,0.690196}%
\pgfsetstrokecolor{currentstroke}%
\pgfsetstrokeopacity{0.300000}%
\pgfsetdash{}{0pt}%
\pgfpathmoveto{\pgfqpoint{3.375125in}{0.582778in}}%
\pgfpathlineto{\pgfqpoint{3.375125in}{3.795000in}}%
\pgfusepath{stroke}%
\end{pgfscope}%
\begin{pgfscope}%
\pgfsetbuttcap%
\pgfsetroundjoin%
\definecolor{currentfill}{rgb}{0.000000,0.000000,0.000000}%
\pgfsetfillcolor{currentfill}%
\pgfsetlinewidth{0.602250pt}%
\definecolor{currentstroke}{rgb}{0.000000,0.000000,0.000000}%
\pgfsetstrokecolor{currentstroke}%
\pgfsetdash{}{0pt}%
\pgfsys@defobject{currentmarker}{\pgfqpoint{0.000000in}{-0.027778in}}{\pgfqpoint{0.000000in}{0.000000in}}{%
\pgfpathmoveto{\pgfqpoint{0.000000in}{0.000000in}}%
\pgfpathlineto{\pgfqpoint{0.000000in}{-0.027778in}}%
\pgfusepath{stroke,fill}%
}%
\begin{pgfscope}%
\pgfsys@transformshift{3.375125in}{0.582778in}%
\pgfsys@useobject{currentmarker}{}%
\end{pgfscope}%
\end{pgfscope}%
\begin{pgfscope}%
\pgfpathrectangle{\pgfqpoint{0.752778in}{0.582778in}}{\pgfqpoint{4.048611in}{3.212222in}}%
\pgfusepath{clip}%
\pgfsetrectcap%
\pgfsetroundjoin%
\pgfsetlinewidth{0.803000pt}%
\definecolor{currentstroke}{rgb}{0.690196,0.690196,0.690196}%
\pgfsetstrokecolor{currentstroke}%
\pgfsetstrokeopacity{0.300000}%
\pgfsetdash{}{0pt}%
\pgfpathmoveto{\pgfqpoint{3.419482in}{0.582778in}}%
\pgfpathlineto{\pgfqpoint{3.419482in}{3.795000in}}%
\pgfusepath{stroke}%
\end{pgfscope}%
\begin{pgfscope}%
\pgfsetbuttcap%
\pgfsetroundjoin%
\definecolor{currentfill}{rgb}{0.000000,0.000000,0.000000}%
\pgfsetfillcolor{currentfill}%
\pgfsetlinewidth{0.602250pt}%
\definecolor{currentstroke}{rgb}{0.000000,0.000000,0.000000}%
\pgfsetstrokecolor{currentstroke}%
\pgfsetdash{}{0pt}%
\pgfsys@defobject{currentmarker}{\pgfqpoint{0.000000in}{-0.027778in}}{\pgfqpoint{0.000000in}{0.000000in}}{%
\pgfpathmoveto{\pgfqpoint{0.000000in}{0.000000in}}%
\pgfpathlineto{\pgfqpoint{0.000000in}{-0.027778in}}%
\pgfusepath{stroke,fill}%
}%
\begin{pgfscope}%
\pgfsys@transformshift{3.419482in}{0.582778in}%
\pgfsys@useobject{currentmarker}{}%
\end{pgfscope}%
\end{pgfscope}%
\begin{pgfscope}%
\pgfpathrectangle{\pgfqpoint{0.752778in}{0.582778in}}{\pgfqpoint{4.048611in}{3.212222in}}%
\pgfusepath{clip}%
\pgfsetrectcap%
\pgfsetroundjoin%
\pgfsetlinewidth{0.803000pt}%
\definecolor{currentstroke}{rgb}{0.690196,0.690196,0.690196}%
\pgfsetstrokecolor{currentstroke}%
\pgfsetstrokeopacity{0.300000}%
\pgfsetdash{}{0pt}%
\pgfpathmoveto{\pgfqpoint{3.463839in}{0.582778in}}%
\pgfpathlineto{\pgfqpoint{3.463839in}{3.795000in}}%
\pgfusepath{stroke}%
\end{pgfscope}%
\begin{pgfscope}%
\pgfsetbuttcap%
\pgfsetroundjoin%
\definecolor{currentfill}{rgb}{0.000000,0.000000,0.000000}%
\pgfsetfillcolor{currentfill}%
\pgfsetlinewidth{0.602250pt}%
\definecolor{currentstroke}{rgb}{0.000000,0.000000,0.000000}%
\pgfsetstrokecolor{currentstroke}%
\pgfsetdash{}{0pt}%
\pgfsys@defobject{currentmarker}{\pgfqpoint{0.000000in}{-0.027778in}}{\pgfqpoint{0.000000in}{0.000000in}}{%
\pgfpathmoveto{\pgfqpoint{0.000000in}{0.000000in}}%
\pgfpathlineto{\pgfqpoint{0.000000in}{-0.027778in}}%
\pgfusepath{stroke,fill}%
}%
\begin{pgfscope}%
\pgfsys@transformshift{3.463839in}{0.582778in}%
\pgfsys@useobject{currentmarker}{}%
\end{pgfscope}%
\end{pgfscope}%
\begin{pgfscope}%
\pgfpathrectangle{\pgfqpoint{0.752778in}{0.582778in}}{\pgfqpoint{4.048611in}{3.212222in}}%
\pgfusepath{clip}%
\pgfsetrectcap%
\pgfsetroundjoin%
\pgfsetlinewidth{0.803000pt}%
\definecolor{currentstroke}{rgb}{0.690196,0.690196,0.690196}%
\pgfsetstrokecolor{currentstroke}%
\pgfsetstrokeopacity{0.300000}%
\pgfsetdash{}{0pt}%
\pgfpathmoveto{\pgfqpoint{3.552553in}{0.582778in}}%
\pgfpathlineto{\pgfqpoint{3.552553in}{3.795000in}}%
\pgfusepath{stroke}%
\end{pgfscope}%
\begin{pgfscope}%
\pgfsetbuttcap%
\pgfsetroundjoin%
\definecolor{currentfill}{rgb}{0.000000,0.000000,0.000000}%
\pgfsetfillcolor{currentfill}%
\pgfsetlinewidth{0.602250pt}%
\definecolor{currentstroke}{rgb}{0.000000,0.000000,0.000000}%
\pgfsetstrokecolor{currentstroke}%
\pgfsetdash{}{0pt}%
\pgfsys@defobject{currentmarker}{\pgfqpoint{0.000000in}{-0.027778in}}{\pgfqpoint{0.000000in}{0.000000in}}{%
\pgfpathmoveto{\pgfqpoint{0.000000in}{0.000000in}}%
\pgfpathlineto{\pgfqpoint{0.000000in}{-0.027778in}}%
\pgfusepath{stroke,fill}%
}%
\begin{pgfscope}%
\pgfsys@transformshift{3.552553in}{0.582778in}%
\pgfsys@useobject{currentmarker}{}%
\end{pgfscope}%
\end{pgfscope}%
\begin{pgfscope}%
\pgfpathrectangle{\pgfqpoint{0.752778in}{0.582778in}}{\pgfqpoint{4.048611in}{3.212222in}}%
\pgfusepath{clip}%
\pgfsetrectcap%
\pgfsetroundjoin%
\pgfsetlinewidth{0.803000pt}%
\definecolor{currentstroke}{rgb}{0.690196,0.690196,0.690196}%
\pgfsetstrokecolor{currentstroke}%
\pgfsetstrokeopacity{0.300000}%
\pgfsetdash{}{0pt}%
\pgfpathmoveto{\pgfqpoint{3.596909in}{0.582778in}}%
\pgfpathlineto{\pgfqpoint{3.596909in}{3.795000in}}%
\pgfusepath{stroke}%
\end{pgfscope}%
\begin{pgfscope}%
\pgfsetbuttcap%
\pgfsetroundjoin%
\definecolor{currentfill}{rgb}{0.000000,0.000000,0.000000}%
\pgfsetfillcolor{currentfill}%
\pgfsetlinewidth{0.602250pt}%
\definecolor{currentstroke}{rgb}{0.000000,0.000000,0.000000}%
\pgfsetstrokecolor{currentstroke}%
\pgfsetdash{}{0pt}%
\pgfsys@defobject{currentmarker}{\pgfqpoint{0.000000in}{-0.027778in}}{\pgfqpoint{0.000000in}{0.000000in}}{%
\pgfpathmoveto{\pgfqpoint{0.000000in}{0.000000in}}%
\pgfpathlineto{\pgfqpoint{0.000000in}{-0.027778in}}%
\pgfusepath{stroke,fill}%
}%
\begin{pgfscope}%
\pgfsys@transformshift{3.596909in}{0.582778in}%
\pgfsys@useobject{currentmarker}{}%
\end{pgfscope}%
\end{pgfscope}%
\begin{pgfscope}%
\pgfpathrectangle{\pgfqpoint{0.752778in}{0.582778in}}{\pgfqpoint{4.048611in}{3.212222in}}%
\pgfusepath{clip}%
\pgfsetrectcap%
\pgfsetroundjoin%
\pgfsetlinewidth{0.803000pt}%
\definecolor{currentstroke}{rgb}{0.690196,0.690196,0.690196}%
\pgfsetstrokecolor{currentstroke}%
\pgfsetstrokeopacity{0.300000}%
\pgfsetdash{}{0pt}%
\pgfpathmoveto{\pgfqpoint{3.641266in}{0.582778in}}%
\pgfpathlineto{\pgfqpoint{3.641266in}{3.795000in}}%
\pgfusepath{stroke}%
\end{pgfscope}%
\begin{pgfscope}%
\pgfsetbuttcap%
\pgfsetroundjoin%
\definecolor{currentfill}{rgb}{0.000000,0.000000,0.000000}%
\pgfsetfillcolor{currentfill}%
\pgfsetlinewidth{0.602250pt}%
\definecolor{currentstroke}{rgb}{0.000000,0.000000,0.000000}%
\pgfsetstrokecolor{currentstroke}%
\pgfsetdash{}{0pt}%
\pgfsys@defobject{currentmarker}{\pgfqpoint{0.000000in}{-0.027778in}}{\pgfqpoint{0.000000in}{0.000000in}}{%
\pgfpathmoveto{\pgfqpoint{0.000000in}{0.000000in}}%
\pgfpathlineto{\pgfqpoint{0.000000in}{-0.027778in}}%
\pgfusepath{stroke,fill}%
}%
\begin{pgfscope}%
\pgfsys@transformshift{3.641266in}{0.582778in}%
\pgfsys@useobject{currentmarker}{}%
\end{pgfscope}%
\end{pgfscope}%
\begin{pgfscope}%
\pgfpathrectangle{\pgfqpoint{0.752778in}{0.582778in}}{\pgfqpoint{4.048611in}{3.212222in}}%
\pgfusepath{clip}%
\pgfsetrectcap%
\pgfsetroundjoin%
\pgfsetlinewidth{0.803000pt}%
\definecolor{currentstroke}{rgb}{0.690196,0.690196,0.690196}%
\pgfsetstrokecolor{currentstroke}%
\pgfsetstrokeopacity{0.300000}%
\pgfsetdash{}{0pt}%
\pgfpathmoveto{\pgfqpoint{3.685623in}{0.582778in}}%
\pgfpathlineto{\pgfqpoint{3.685623in}{3.795000in}}%
\pgfusepath{stroke}%
\end{pgfscope}%
\begin{pgfscope}%
\pgfsetbuttcap%
\pgfsetroundjoin%
\definecolor{currentfill}{rgb}{0.000000,0.000000,0.000000}%
\pgfsetfillcolor{currentfill}%
\pgfsetlinewidth{0.602250pt}%
\definecolor{currentstroke}{rgb}{0.000000,0.000000,0.000000}%
\pgfsetstrokecolor{currentstroke}%
\pgfsetdash{}{0pt}%
\pgfsys@defobject{currentmarker}{\pgfqpoint{0.000000in}{-0.027778in}}{\pgfqpoint{0.000000in}{0.000000in}}{%
\pgfpathmoveto{\pgfqpoint{0.000000in}{0.000000in}}%
\pgfpathlineto{\pgfqpoint{0.000000in}{-0.027778in}}%
\pgfusepath{stroke,fill}%
}%
\begin{pgfscope}%
\pgfsys@transformshift{3.685623in}{0.582778in}%
\pgfsys@useobject{currentmarker}{}%
\end{pgfscope}%
\end{pgfscope}%
\begin{pgfscope}%
\pgfpathrectangle{\pgfqpoint{0.752778in}{0.582778in}}{\pgfqpoint{4.048611in}{3.212222in}}%
\pgfusepath{clip}%
\pgfsetrectcap%
\pgfsetroundjoin%
\pgfsetlinewidth{0.803000pt}%
\definecolor{currentstroke}{rgb}{0.690196,0.690196,0.690196}%
\pgfsetstrokecolor{currentstroke}%
\pgfsetstrokeopacity{0.300000}%
\pgfsetdash{}{0pt}%
\pgfpathmoveto{\pgfqpoint{3.729980in}{0.582778in}}%
\pgfpathlineto{\pgfqpoint{3.729980in}{3.795000in}}%
\pgfusepath{stroke}%
\end{pgfscope}%
\begin{pgfscope}%
\pgfsetbuttcap%
\pgfsetroundjoin%
\definecolor{currentfill}{rgb}{0.000000,0.000000,0.000000}%
\pgfsetfillcolor{currentfill}%
\pgfsetlinewidth{0.602250pt}%
\definecolor{currentstroke}{rgb}{0.000000,0.000000,0.000000}%
\pgfsetstrokecolor{currentstroke}%
\pgfsetdash{}{0pt}%
\pgfsys@defobject{currentmarker}{\pgfqpoint{0.000000in}{-0.027778in}}{\pgfqpoint{0.000000in}{0.000000in}}{%
\pgfpathmoveto{\pgfqpoint{0.000000in}{0.000000in}}%
\pgfpathlineto{\pgfqpoint{0.000000in}{-0.027778in}}%
\pgfusepath{stroke,fill}%
}%
\begin{pgfscope}%
\pgfsys@transformshift{3.729980in}{0.582778in}%
\pgfsys@useobject{currentmarker}{}%
\end{pgfscope}%
\end{pgfscope}%
\begin{pgfscope}%
\pgfpathrectangle{\pgfqpoint{0.752778in}{0.582778in}}{\pgfqpoint{4.048611in}{3.212222in}}%
\pgfusepath{clip}%
\pgfsetrectcap%
\pgfsetroundjoin%
\pgfsetlinewidth{0.803000pt}%
\definecolor{currentstroke}{rgb}{0.690196,0.690196,0.690196}%
\pgfsetstrokecolor{currentstroke}%
\pgfsetstrokeopacity{0.300000}%
\pgfsetdash{}{0pt}%
\pgfpathmoveto{\pgfqpoint{3.774337in}{0.582778in}}%
\pgfpathlineto{\pgfqpoint{3.774337in}{3.795000in}}%
\pgfusepath{stroke}%
\end{pgfscope}%
\begin{pgfscope}%
\pgfsetbuttcap%
\pgfsetroundjoin%
\definecolor{currentfill}{rgb}{0.000000,0.000000,0.000000}%
\pgfsetfillcolor{currentfill}%
\pgfsetlinewidth{0.602250pt}%
\definecolor{currentstroke}{rgb}{0.000000,0.000000,0.000000}%
\pgfsetstrokecolor{currentstroke}%
\pgfsetdash{}{0pt}%
\pgfsys@defobject{currentmarker}{\pgfqpoint{0.000000in}{-0.027778in}}{\pgfqpoint{0.000000in}{0.000000in}}{%
\pgfpathmoveto{\pgfqpoint{0.000000in}{0.000000in}}%
\pgfpathlineto{\pgfqpoint{0.000000in}{-0.027778in}}%
\pgfusepath{stroke,fill}%
}%
\begin{pgfscope}%
\pgfsys@transformshift{3.774337in}{0.582778in}%
\pgfsys@useobject{currentmarker}{}%
\end{pgfscope}%
\end{pgfscope}%
\begin{pgfscope}%
\pgfpathrectangle{\pgfqpoint{0.752778in}{0.582778in}}{\pgfqpoint{4.048611in}{3.212222in}}%
\pgfusepath{clip}%
\pgfsetrectcap%
\pgfsetroundjoin%
\pgfsetlinewidth{0.803000pt}%
\definecolor{currentstroke}{rgb}{0.690196,0.690196,0.690196}%
\pgfsetstrokecolor{currentstroke}%
\pgfsetstrokeopacity{0.300000}%
\pgfsetdash{}{0pt}%
\pgfpathmoveto{\pgfqpoint{3.818694in}{0.582778in}}%
\pgfpathlineto{\pgfqpoint{3.818694in}{3.795000in}}%
\pgfusepath{stroke}%
\end{pgfscope}%
\begin{pgfscope}%
\pgfsetbuttcap%
\pgfsetroundjoin%
\definecolor{currentfill}{rgb}{0.000000,0.000000,0.000000}%
\pgfsetfillcolor{currentfill}%
\pgfsetlinewidth{0.602250pt}%
\definecolor{currentstroke}{rgb}{0.000000,0.000000,0.000000}%
\pgfsetstrokecolor{currentstroke}%
\pgfsetdash{}{0pt}%
\pgfsys@defobject{currentmarker}{\pgfqpoint{0.000000in}{-0.027778in}}{\pgfqpoint{0.000000in}{0.000000in}}{%
\pgfpathmoveto{\pgfqpoint{0.000000in}{0.000000in}}%
\pgfpathlineto{\pgfqpoint{0.000000in}{-0.027778in}}%
\pgfusepath{stroke,fill}%
}%
\begin{pgfscope}%
\pgfsys@transformshift{3.818694in}{0.582778in}%
\pgfsys@useobject{currentmarker}{}%
\end{pgfscope}%
\end{pgfscope}%
\begin{pgfscope}%
\pgfpathrectangle{\pgfqpoint{0.752778in}{0.582778in}}{\pgfqpoint{4.048611in}{3.212222in}}%
\pgfusepath{clip}%
\pgfsetrectcap%
\pgfsetroundjoin%
\pgfsetlinewidth{0.803000pt}%
\definecolor{currentstroke}{rgb}{0.690196,0.690196,0.690196}%
\pgfsetstrokecolor{currentstroke}%
\pgfsetstrokeopacity{0.300000}%
\pgfsetdash{}{0pt}%
\pgfpathmoveto{\pgfqpoint{3.863051in}{0.582778in}}%
\pgfpathlineto{\pgfqpoint{3.863051in}{3.795000in}}%
\pgfusepath{stroke}%
\end{pgfscope}%
\begin{pgfscope}%
\pgfsetbuttcap%
\pgfsetroundjoin%
\definecolor{currentfill}{rgb}{0.000000,0.000000,0.000000}%
\pgfsetfillcolor{currentfill}%
\pgfsetlinewidth{0.602250pt}%
\definecolor{currentstroke}{rgb}{0.000000,0.000000,0.000000}%
\pgfsetstrokecolor{currentstroke}%
\pgfsetdash{}{0pt}%
\pgfsys@defobject{currentmarker}{\pgfqpoint{0.000000in}{-0.027778in}}{\pgfqpoint{0.000000in}{0.000000in}}{%
\pgfpathmoveto{\pgfqpoint{0.000000in}{0.000000in}}%
\pgfpathlineto{\pgfqpoint{0.000000in}{-0.027778in}}%
\pgfusepath{stroke,fill}%
}%
\begin{pgfscope}%
\pgfsys@transformshift{3.863051in}{0.582778in}%
\pgfsys@useobject{currentmarker}{}%
\end{pgfscope}%
\end{pgfscope}%
\begin{pgfscope}%
\pgfpathrectangle{\pgfqpoint{0.752778in}{0.582778in}}{\pgfqpoint{4.048611in}{3.212222in}}%
\pgfusepath{clip}%
\pgfsetrectcap%
\pgfsetroundjoin%
\pgfsetlinewidth{0.803000pt}%
\definecolor{currentstroke}{rgb}{0.690196,0.690196,0.690196}%
\pgfsetstrokecolor{currentstroke}%
\pgfsetstrokeopacity{0.300000}%
\pgfsetdash{}{0pt}%
\pgfpathmoveto{\pgfqpoint{3.907407in}{0.582778in}}%
\pgfpathlineto{\pgfqpoint{3.907407in}{3.795000in}}%
\pgfusepath{stroke}%
\end{pgfscope}%
\begin{pgfscope}%
\pgfsetbuttcap%
\pgfsetroundjoin%
\definecolor{currentfill}{rgb}{0.000000,0.000000,0.000000}%
\pgfsetfillcolor{currentfill}%
\pgfsetlinewidth{0.602250pt}%
\definecolor{currentstroke}{rgb}{0.000000,0.000000,0.000000}%
\pgfsetstrokecolor{currentstroke}%
\pgfsetdash{}{0pt}%
\pgfsys@defobject{currentmarker}{\pgfqpoint{0.000000in}{-0.027778in}}{\pgfqpoint{0.000000in}{0.000000in}}{%
\pgfpathmoveto{\pgfqpoint{0.000000in}{0.000000in}}%
\pgfpathlineto{\pgfqpoint{0.000000in}{-0.027778in}}%
\pgfusepath{stroke,fill}%
}%
\begin{pgfscope}%
\pgfsys@transformshift{3.907407in}{0.582778in}%
\pgfsys@useobject{currentmarker}{}%
\end{pgfscope}%
\end{pgfscope}%
\begin{pgfscope}%
\pgfpathrectangle{\pgfqpoint{0.752778in}{0.582778in}}{\pgfqpoint{4.048611in}{3.212222in}}%
\pgfusepath{clip}%
\pgfsetrectcap%
\pgfsetroundjoin%
\pgfsetlinewidth{0.803000pt}%
\definecolor{currentstroke}{rgb}{0.690196,0.690196,0.690196}%
\pgfsetstrokecolor{currentstroke}%
\pgfsetstrokeopacity{0.300000}%
\pgfsetdash{}{0pt}%
\pgfpathmoveto{\pgfqpoint{3.996121in}{0.582778in}}%
\pgfpathlineto{\pgfqpoint{3.996121in}{3.795000in}}%
\pgfusepath{stroke}%
\end{pgfscope}%
\begin{pgfscope}%
\pgfsetbuttcap%
\pgfsetroundjoin%
\definecolor{currentfill}{rgb}{0.000000,0.000000,0.000000}%
\pgfsetfillcolor{currentfill}%
\pgfsetlinewidth{0.602250pt}%
\definecolor{currentstroke}{rgb}{0.000000,0.000000,0.000000}%
\pgfsetstrokecolor{currentstroke}%
\pgfsetdash{}{0pt}%
\pgfsys@defobject{currentmarker}{\pgfqpoint{0.000000in}{-0.027778in}}{\pgfqpoint{0.000000in}{0.000000in}}{%
\pgfpathmoveto{\pgfqpoint{0.000000in}{0.000000in}}%
\pgfpathlineto{\pgfqpoint{0.000000in}{-0.027778in}}%
\pgfusepath{stroke,fill}%
}%
\begin{pgfscope}%
\pgfsys@transformshift{3.996121in}{0.582778in}%
\pgfsys@useobject{currentmarker}{}%
\end{pgfscope}%
\end{pgfscope}%
\begin{pgfscope}%
\pgfpathrectangle{\pgfqpoint{0.752778in}{0.582778in}}{\pgfqpoint{4.048611in}{3.212222in}}%
\pgfusepath{clip}%
\pgfsetrectcap%
\pgfsetroundjoin%
\pgfsetlinewidth{0.803000pt}%
\definecolor{currentstroke}{rgb}{0.690196,0.690196,0.690196}%
\pgfsetstrokecolor{currentstroke}%
\pgfsetstrokeopacity{0.300000}%
\pgfsetdash{}{0pt}%
\pgfpathmoveto{\pgfqpoint{4.040478in}{0.582778in}}%
\pgfpathlineto{\pgfqpoint{4.040478in}{3.795000in}}%
\pgfusepath{stroke}%
\end{pgfscope}%
\begin{pgfscope}%
\pgfsetbuttcap%
\pgfsetroundjoin%
\definecolor{currentfill}{rgb}{0.000000,0.000000,0.000000}%
\pgfsetfillcolor{currentfill}%
\pgfsetlinewidth{0.602250pt}%
\definecolor{currentstroke}{rgb}{0.000000,0.000000,0.000000}%
\pgfsetstrokecolor{currentstroke}%
\pgfsetdash{}{0pt}%
\pgfsys@defobject{currentmarker}{\pgfqpoint{0.000000in}{-0.027778in}}{\pgfqpoint{0.000000in}{0.000000in}}{%
\pgfpathmoveto{\pgfqpoint{0.000000in}{0.000000in}}%
\pgfpathlineto{\pgfqpoint{0.000000in}{-0.027778in}}%
\pgfusepath{stroke,fill}%
}%
\begin{pgfscope}%
\pgfsys@transformshift{4.040478in}{0.582778in}%
\pgfsys@useobject{currentmarker}{}%
\end{pgfscope}%
\end{pgfscope}%
\begin{pgfscope}%
\pgfpathrectangle{\pgfqpoint{0.752778in}{0.582778in}}{\pgfqpoint{4.048611in}{3.212222in}}%
\pgfusepath{clip}%
\pgfsetrectcap%
\pgfsetroundjoin%
\pgfsetlinewidth{0.803000pt}%
\definecolor{currentstroke}{rgb}{0.690196,0.690196,0.690196}%
\pgfsetstrokecolor{currentstroke}%
\pgfsetstrokeopacity{0.300000}%
\pgfsetdash{}{0pt}%
\pgfpathmoveto{\pgfqpoint{4.084835in}{0.582778in}}%
\pgfpathlineto{\pgfqpoint{4.084835in}{3.795000in}}%
\pgfusepath{stroke}%
\end{pgfscope}%
\begin{pgfscope}%
\pgfsetbuttcap%
\pgfsetroundjoin%
\definecolor{currentfill}{rgb}{0.000000,0.000000,0.000000}%
\pgfsetfillcolor{currentfill}%
\pgfsetlinewidth{0.602250pt}%
\definecolor{currentstroke}{rgb}{0.000000,0.000000,0.000000}%
\pgfsetstrokecolor{currentstroke}%
\pgfsetdash{}{0pt}%
\pgfsys@defobject{currentmarker}{\pgfqpoint{0.000000in}{-0.027778in}}{\pgfqpoint{0.000000in}{0.000000in}}{%
\pgfpathmoveto{\pgfqpoint{0.000000in}{0.000000in}}%
\pgfpathlineto{\pgfqpoint{0.000000in}{-0.027778in}}%
\pgfusepath{stroke,fill}%
}%
\begin{pgfscope}%
\pgfsys@transformshift{4.084835in}{0.582778in}%
\pgfsys@useobject{currentmarker}{}%
\end{pgfscope}%
\end{pgfscope}%
\begin{pgfscope}%
\pgfpathrectangle{\pgfqpoint{0.752778in}{0.582778in}}{\pgfqpoint{4.048611in}{3.212222in}}%
\pgfusepath{clip}%
\pgfsetrectcap%
\pgfsetroundjoin%
\pgfsetlinewidth{0.803000pt}%
\definecolor{currentstroke}{rgb}{0.690196,0.690196,0.690196}%
\pgfsetstrokecolor{currentstroke}%
\pgfsetstrokeopacity{0.300000}%
\pgfsetdash{}{0pt}%
\pgfpathmoveto{\pgfqpoint{4.129192in}{0.582778in}}%
\pgfpathlineto{\pgfqpoint{4.129192in}{3.795000in}}%
\pgfusepath{stroke}%
\end{pgfscope}%
\begin{pgfscope}%
\pgfsetbuttcap%
\pgfsetroundjoin%
\definecolor{currentfill}{rgb}{0.000000,0.000000,0.000000}%
\pgfsetfillcolor{currentfill}%
\pgfsetlinewidth{0.602250pt}%
\definecolor{currentstroke}{rgb}{0.000000,0.000000,0.000000}%
\pgfsetstrokecolor{currentstroke}%
\pgfsetdash{}{0pt}%
\pgfsys@defobject{currentmarker}{\pgfqpoint{0.000000in}{-0.027778in}}{\pgfqpoint{0.000000in}{0.000000in}}{%
\pgfpathmoveto{\pgfqpoint{0.000000in}{0.000000in}}%
\pgfpathlineto{\pgfqpoint{0.000000in}{-0.027778in}}%
\pgfusepath{stroke,fill}%
}%
\begin{pgfscope}%
\pgfsys@transformshift{4.129192in}{0.582778in}%
\pgfsys@useobject{currentmarker}{}%
\end{pgfscope}%
\end{pgfscope}%
\begin{pgfscope}%
\pgfpathrectangle{\pgfqpoint{0.752778in}{0.582778in}}{\pgfqpoint{4.048611in}{3.212222in}}%
\pgfusepath{clip}%
\pgfsetrectcap%
\pgfsetroundjoin%
\pgfsetlinewidth{0.803000pt}%
\definecolor{currentstroke}{rgb}{0.690196,0.690196,0.690196}%
\pgfsetstrokecolor{currentstroke}%
\pgfsetstrokeopacity{0.300000}%
\pgfsetdash{}{0pt}%
\pgfpathmoveto{\pgfqpoint{4.173549in}{0.582778in}}%
\pgfpathlineto{\pgfqpoint{4.173549in}{3.795000in}}%
\pgfusepath{stroke}%
\end{pgfscope}%
\begin{pgfscope}%
\pgfsetbuttcap%
\pgfsetroundjoin%
\definecolor{currentfill}{rgb}{0.000000,0.000000,0.000000}%
\pgfsetfillcolor{currentfill}%
\pgfsetlinewidth{0.602250pt}%
\definecolor{currentstroke}{rgb}{0.000000,0.000000,0.000000}%
\pgfsetstrokecolor{currentstroke}%
\pgfsetdash{}{0pt}%
\pgfsys@defobject{currentmarker}{\pgfqpoint{0.000000in}{-0.027778in}}{\pgfqpoint{0.000000in}{0.000000in}}{%
\pgfpathmoveto{\pgfqpoint{0.000000in}{0.000000in}}%
\pgfpathlineto{\pgfqpoint{0.000000in}{-0.027778in}}%
\pgfusepath{stroke,fill}%
}%
\begin{pgfscope}%
\pgfsys@transformshift{4.173549in}{0.582778in}%
\pgfsys@useobject{currentmarker}{}%
\end{pgfscope}%
\end{pgfscope}%
\begin{pgfscope}%
\pgfpathrectangle{\pgfqpoint{0.752778in}{0.582778in}}{\pgfqpoint{4.048611in}{3.212222in}}%
\pgfusepath{clip}%
\pgfsetrectcap%
\pgfsetroundjoin%
\pgfsetlinewidth{0.803000pt}%
\definecolor{currentstroke}{rgb}{0.690196,0.690196,0.690196}%
\pgfsetstrokecolor{currentstroke}%
\pgfsetstrokeopacity{0.300000}%
\pgfsetdash{}{0pt}%
\pgfpathmoveto{\pgfqpoint{4.217905in}{0.582778in}}%
\pgfpathlineto{\pgfqpoint{4.217905in}{3.795000in}}%
\pgfusepath{stroke}%
\end{pgfscope}%
\begin{pgfscope}%
\pgfsetbuttcap%
\pgfsetroundjoin%
\definecolor{currentfill}{rgb}{0.000000,0.000000,0.000000}%
\pgfsetfillcolor{currentfill}%
\pgfsetlinewidth{0.602250pt}%
\definecolor{currentstroke}{rgb}{0.000000,0.000000,0.000000}%
\pgfsetstrokecolor{currentstroke}%
\pgfsetdash{}{0pt}%
\pgfsys@defobject{currentmarker}{\pgfqpoint{0.000000in}{-0.027778in}}{\pgfqpoint{0.000000in}{0.000000in}}{%
\pgfpathmoveto{\pgfqpoint{0.000000in}{0.000000in}}%
\pgfpathlineto{\pgfqpoint{0.000000in}{-0.027778in}}%
\pgfusepath{stroke,fill}%
}%
\begin{pgfscope}%
\pgfsys@transformshift{4.217905in}{0.582778in}%
\pgfsys@useobject{currentmarker}{}%
\end{pgfscope}%
\end{pgfscope}%
\begin{pgfscope}%
\pgfpathrectangle{\pgfqpoint{0.752778in}{0.582778in}}{\pgfqpoint{4.048611in}{3.212222in}}%
\pgfusepath{clip}%
\pgfsetrectcap%
\pgfsetroundjoin%
\pgfsetlinewidth{0.803000pt}%
\definecolor{currentstroke}{rgb}{0.690196,0.690196,0.690196}%
\pgfsetstrokecolor{currentstroke}%
\pgfsetstrokeopacity{0.300000}%
\pgfsetdash{}{0pt}%
\pgfpathmoveto{\pgfqpoint{4.262262in}{0.582778in}}%
\pgfpathlineto{\pgfqpoint{4.262262in}{3.795000in}}%
\pgfusepath{stroke}%
\end{pgfscope}%
\begin{pgfscope}%
\pgfsetbuttcap%
\pgfsetroundjoin%
\definecolor{currentfill}{rgb}{0.000000,0.000000,0.000000}%
\pgfsetfillcolor{currentfill}%
\pgfsetlinewidth{0.602250pt}%
\definecolor{currentstroke}{rgb}{0.000000,0.000000,0.000000}%
\pgfsetstrokecolor{currentstroke}%
\pgfsetdash{}{0pt}%
\pgfsys@defobject{currentmarker}{\pgfqpoint{0.000000in}{-0.027778in}}{\pgfqpoint{0.000000in}{0.000000in}}{%
\pgfpathmoveto{\pgfqpoint{0.000000in}{0.000000in}}%
\pgfpathlineto{\pgfqpoint{0.000000in}{-0.027778in}}%
\pgfusepath{stroke,fill}%
}%
\begin{pgfscope}%
\pgfsys@transformshift{4.262262in}{0.582778in}%
\pgfsys@useobject{currentmarker}{}%
\end{pgfscope}%
\end{pgfscope}%
\begin{pgfscope}%
\pgfpathrectangle{\pgfqpoint{0.752778in}{0.582778in}}{\pgfqpoint{4.048611in}{3.212222in}}%
\pgfusepath{clip}%
\pgfsetrectcap%
\pgfsetroundjoin%
\pgfsetlinewidth{0.803000pt}%
\definecolor{currentstroke}{rgb}{0.690196,0.690196,0.690196}%
\pgfsetstrokecolor{currentstroke}%
\pgfsetstrokeopacity{0.300000}%
\pgfsetdash{}{0pt}%
\pgfpathmoveto{\pgfqpoint{4.306619in}{0.582778in}}%
\pgfpathlineto{\pgfqpoint{4.306619in}{3.795000in}}%
\pgfusepath{stroke}%
\end{pgfscope}%
\begin{pgfscope}%
\pgfsetbuttcap%
\pgfsetroundjoin%
\definecolor{currentfill}{rgb}{0.000000,0.000000,0.000000}%
\pgfsetfillcolor{currentfill}%
\pgfsetlinewidth{0.602250pt}%
\definecolor{currentstroke}{rgb}{0.000000,0.000000,0.000000}%
\pgfsetstrokecolor{currentstroke}%
\pgfsetdash{}{0pt}%
\pgfsys@defobject{currentmarker}{\pgfqpoint{0.000000in}{-0.027778in}}{\pgfqpoint{0.000000in}{0.000000in}}{%
\pgfpathmoveto{\pgfqpoint{0.000000in}{0.000000in}}%
\pgfpathlineto{\pgfqpoint{0.000000in}{-0.027778in}}%
\pgfusepath{stroke,fill}%
}%
\begin{pgfscope}%
\pgfsys@transformshift{4.306619in}{0.582778in}%
\pgfsys@useobject{currentmarker}{}%
\end{pgfscope}%
\end{pgfscope}%
\begin{pgfscope}%
\pgfpathrectangle{\pgfqpoint{0.752778in}{0.582778in}}{\pgfqpoint{4.048611in}{3.212222in}}%
\pgfusepath{clip}%
\pgfsetrectcap%
\pgfsetroundjoin%
\pgfsetlinewidth{0.803000pt}%
\definecolor{currentstroke}{rgb}{0.690196,0.690196,0.690196}%
\pgfsetstrokecolor{currentstroke}%
\pgfsetstrokeopacity{0.300000}%
\pgfsetdash{}{0pt}%
\pgfpathmoveto{\pgfqpoint{4.350976in}{0.582778in}}%
\pgfpathlineto{\pgfqpoint{4.350976in}{3.795000in}}%
\pgfusepath{stroke}%
\end{pgfscope}%
\begin{pgfscope}%
\pgfsetbuttcap%
\pgfsetroundjoin%
\definecolor{currentfill}{rgb}{0.000000,0.000000,0.000000}%
\pgfsetfillcolor{currentfill}%
\pgfsetlinewidth{0.602250pt}%
\definecolor{currentstroke}{rgb}{0.000000,0.000000,0.000000}%
\pgfsetstrokecolor{currentstroke}%
\pgfsetdash{}{0pt}%
\pgfsys@defobject{currentmarker}{\pgfqpoint{0.000000in}{-0.027778in}}{\pgfqpoint{0.000000in}{0.000000in}}{%
\pgfpathmoveto{\pgfqpoint{0.000000in}{0.000000in}}%
\pgfpathlineto{\pgfqpoint{0.000000in}{-0.027778in}}%
\pgfusepath{stroke,fill}%
}%
\begin{pgfscope}%
\pgfsys@transformshift{4.350976in}{0.582778in}%
\pgfsys@useobject{currentmarker}{}%
\end{pgfscope}%
\end{pgfscope}%
\begin{pgfscope}%
\pgfpathrectangle{\pgfqpoint{0.752778in}{0.582778in}}{\pgfqpoint{4.048611in}{3.212222in}}%
\pgfusepath{clip}%
\pgfsetrectcap%
\pgfsetroundjoin%
\pgfsetlinewidth{0.803000pt}%
\definecolor{currentstroke}{rgb}{0.690196,0.690196,0.690196}%
\pgfsetstrokecolor{currentstroke}%
\pgfsetstrokeopacity{0.300000}%
\pgfsetdash{}{0pt}%
\pgfpathmoveto{\pgfqpoint{4.439690in}{0.582778in}}%
\pgfpathlineto{\pgfqpoint{4.439690in}{3.795000in}}%
\pgfusepath{stroke}%
\end{pgfscope}%
\begin{pgfscope}%
\pgfsetbuttcap%
\pgfsetroundjoin%
\definecolor{currentfill}{rgb}{0.000000,0.000000,0.000000}%
\pgfsetfillcolor{currentfill}%
\pgfsetlinewidth{0.602250pt}%
\definecolor{currentstroke}{rgb}{0.000000,0.000000,0.000000}%
\pgfsetstrokecolor{currentstroke}%
\pgfsetdash{}{0pt}%
\pgfsys@defobject{currentmarker}{\pgfqpoint{0.000000in}{-0.027778in}}{\pgfqpoint{0.000000in}{0.000000in}}{%
\pgfpathmoveto{\pgfqpoint{0.000000in}{0.000000in}}%
\pgfpathlineto{\pgfqpoint{0.000000in}{-0.027778in}}%
\pgfusepath{stroke,fill}%
}%
\begin{pgfscope}%
\pgfsys@transformshift{4.439690in}{0.582778in}%
\pgfsys@useobject{currentmarker}{}%
\end{pgfscope}%
\end{pgfscope}%
\begin{pgfscope}%
\pgfpathrectangle{\pgfqpoint{0.752778in}{0.582778in}}{\pgfqpoint{4.048611in}{3.212222in}}%
\pgfusepath{clip}%
\pgfsetrectcap%
\pgfsetroundjoin%
\pgfsetlinewidth{0.803000pt}%
\definecolor{currentstroke}{rgb}{0.690196,0.690196,0.690196}%
\pgfsetstrokecolor{currentstroke}%
\pgfsetstrokeopacity{0.300000}%
\pgfsetdash{}{0pt}%
\pgfpathmoveto{\pgfqpoint{4.484047in}{0.582778in}}%
\pgfpathlineto{\pgfqpoint{4.484047in}{3.795000in}}%
\pgfusepath{stroke}%
\end{pgfscope}%
\begin{pgfscope}%
\pgfsetbuttcap%
\pgfsetroundjoin%
\definecolor{currentfill}{rgb}{0.000000,0.000000,0.000000}%
\pgfsetfillcolor{currentfill}%
\pgfsetlinewidth{0.602250pt}%
\definecolor{currentstroke}{rgb}{0.000000,0.000000,0.000000}%
\pgfsetstrokecolor{currentstroke}%
\pgfsetdash{}{0pt}%
\pgfsys@defobject{currentmarker}{\pgfqpoint{0.000000in}{-0.027778in}}{\pgfqpoint{0.000000in}{0.000000in}}{%
\pgfpathmoveto{\pgfqpoint{0.000000in}{0.000000in}}%
\pgfpathlineto{\pgfqpoint{0.000000in}{-0.027778in}}%
\pgfusepath{stroke,fill}%
}%
\begin{pgfscope}%
\pgfsys@transformshift{4.484047in}{0.582778in}%
\pgfsys@useobject{currentmarker}{}%
\end{pgfscope}%
\end{pgfscope}%
\begin{pgfscope}%
\pgfpathrectangle{\pgfqpoint{0.752778in}{0.582778in}}{\pgfqpoint{4.048611in}{3.212222in}}%
\pgfusepath{clip}%
\pgfsetrectcap%
\pgfsetroundjoin%
\pgfsetlinewidth{0.803000pt}%
\definecolor{currentstroke}{rgb}{0.690196,0.690196,0.690196}%
\pgfsetstrokecolor{currentstroke}%
\pgfsetstrokeopacity{0.300000}%
\pgfsetdash{}{0pt}%
\pgfpathmoveto{\pgfqpoint{4.528403in}{0.582778in}}%
\pgfpathlineto{\pgfqpoint{4.528403in}{3.795000in}}%
\pgfusepath{stroke}%
\end{pgfscope}%
\begin{pgfscope}%
\pgfsetbuttcap%
\pgfsetroundjoin%
\definecolor{currentfill}{rgb}{0.000000,0.000000,0.000000}%
\pgfsetfillcolor{currentfill}%
\pgfsetlinewidth{0.602250pt}%
\definecolor{currentstroke}{rgb}{0.000000,0.000000,0.000000}%
\pgfsetstrokecolor{currentstroke}%
\pgfsetdash{}{0pt}%
\pgfsys@defobject{currentmarker}{\pgfqpoint{0.000000in}{-0.027778in}}{\pgfqpoint{0.000000in}{0.000000in}}{%
\pgfpathmoveto{\pgfqpoint{0.000000in}{0.000000in}}%
\pgfpathlineto{\pgfqpoint{0.000000in}{-0.027778in}}%
\pgfusepath{stroke,fill}%
}%
\begin{pgfscope}%
\pgfsys@transformshift{4.528403in}{0.582778in}%
\pgfsys@useobject{currentmarker}{}%
\end{pgfscope}%
\end{pgfscope}%
\begin{pgfscope}%
\pgfpathrectangle{\pgfqpoint{0.752778in}{0.582778in}}{\pgfqpoint{4.048611in}{3.212222in}}%
\pgfusepath{clip}%
\pgfsetrectcap%
\pgfsetroundjoin%
\pgfsetlinewidth{0.803000pt}%
\definecolor{currentstroke}{rgb}{0.690196,0.690196,0.690196}%
\pgfsetstrokecolor{currentstroke}%
\pgfsetstrokeopacity{0.300000}%
\pgfsetdash{}{0pt}%
\pgfpathmoveto{\pgfqpoint{4.572760in}{0.582778in}}%
\pgfpathlineto{\pgfqpoint{4.572760in}{3.795000in}}%
\pgfusepath{stroke}%
\end{pgfscope}%
\begin{pgfscope}%
\pgfsetbuttcap%
\pgfsetroundjoin%
\definecolor{currentfill}{rgb}{0.000000,0.000000,0.000000}%
\pgfsetfillcolor{currentfill}%
\pgfsetlinewidth{0.602250pt}%
\definecolor{currentstroke}{rgb}{0.000000,0.000000,0.000000}%
\pgfsetstrokecolor{currentstroke}%
\pgfsetdash{}{0pt}%
\pgfsys@defobject{currentmarker}{\pgfqpoint{0.000000in}{-0.027778in}}{\pgfqpoint{0.000000in}{0.000000in}}{%
\pgfpathmoveto{\pgfqpoint{0.000000in}{0.000000in}}%
\pgfpathlineto{\pgfqpoint{0.000000in}{-0.027778in}}%
\pgfusepath{stroke,fill}%
}%
\begin{pgfscope}%
\pgfsys@transformshift{4.572760in}{0.582778in}%
\pgfsys@useobject{currentmarker}{}%
\end{pgfscope}%
\end{pgfscope}%
\begin{pgfscope}%
\pgfpathrectangle{\pgfqpoint{0.752778in}{0.582778in}}{\pgfqpoint{4.048611in}{3.212222in}}%
\pgfusepath{clip}%
\pgfsetrectcap%
\pgfsetroundjoin%
\pgfsetlinewidth{0.803000pt}%
\definecolor{currentstroke}{rgb}{0.690196,0.690196,0.690196}%
\pgfsetstrokecolor{currentstroke}%
\pgfsetstrokeopacity{0.300000}%
\pgfsetdash{}{0pt}%
\pgfpathmoveto{\pgfqpoint{4.617117in}{0.582778in}}%
\pgfpathlineto{\pgfqpoint{4.617117in}{3.795000in}}%
\pgfusepath{stroke}%
\end{pgfscope}%
\begin{pgfscope}%
\pgfsetbuttcap%
\pgfsetroundjoin%
\definecolor{currentfill}{rgb}{0.000000,0.000000,0.000000}%
\pgfsetfillcolor{currentfill}%
\pgfsetlinewidth{0.602250pt}%
\definecolor{currentstroke}{rgb}{0.000000,0.000000,0.000000}%
\pgfsetstrokecolor{currentstroke}%
\pgfsetdash{}{0pt}%
\pgfsys@defobject{currentmarker}{\pgfqpoint{0.000000in}{-0.027778in}}{\pgfqpoint{0.000000in}{0.000000in}}{%
\pgfpathmoveto{\pgfqpoint{0.000000in}{0.000000in}}%
\pgfpathlineto{\pgfqpoint{0.000000in}{-0.027778in}}%
\pgfusepath{stroke,fill}%
}%
\begin{pgfscope}%
\pgfsys@transformshift{4.617117in}{0.582778in}%
\pgfsys@useobject{currentmarker}{}%
\end{pgfscope}%
\end{pgfscope}%
\begin{pgfscope}%
\pgfpathrectangle{\pgfqpoint{0.752778in}{0.582778in}}{\pgfqpoint{4.048611in}{3.212222in}}%
\pgfusepath{clip}%
\pgfsetrectcap%
\pgfsetroundjoin%
\pgfsetlinewidth{0.803000pt}%
\definecolor{currentstroke}{rgb}{0.690196,0.690196,0.690196}%
\pgfsetstrokecolor{currentstroke}%
\pgfsetstrokeopacity{0.300000}%
\pgfsetdash{}{0pt}%
\pgfpathmoveto{\pgfqpoint{4.661474in}{0.582778in}}%
\pgfpathlineto{\pgfqpoint{4.661474in}{3.795000in}}%
\pgfusepath{stroke}%
\end{pgfscope}%
\begin{pgfscope}%
\pgfsetbuttcap%
\pgfsetroundjoin%
\definecolor{currentfill}{rgb}{0.000000,0.000000,0.000000}%
\pgfsetfillcolor{currentfill}%
\pgfsetlinewidth{0.602250pt}%
\definecolor{currentstroke}{rgb}{0.000000,0.000000,0.000000}%
\pgfsetstrokecolor{currentstroke}%
\pgfsetdash{}{0pt}%
\pgfsys@defobject{currentmarker}{\pgfqpoint{0.000000in}{-0.027778in}}{\pgfqpoint{0.000000in}{0.000000in}}{%
\pgfpathmoveto{\pgfqpoint{0.000000in}{0.000000in}}%
\pgfpathlineto{\pgfqpoint{0.000000in}{-0.027778in}}%
\pgfusepath{stroke,fill}%
}%
\begin{pgfscope}%
\pgfsys@transformshift{4.661474in}{0.582778in}%
\pgfsys@useobject{currentmarker}{}%
\end{pgfscope}%
\end{pgfscope}%
\begin{pgfscope}%
\pgfpathrectangle{\pgfqpoint{0.752778in}{0.582778in}}{\pgfqpoint{4.048611in}{3.212222in}}%
\pgfusepath{clip}%
\pgfsetrectcap%
\pgfsetroundjoin%
\pgfsetlinewidth{0.803000pt}%
\definecolor{currentstroke}{rgb}{0.690196,0.690196,0.690196}%
\pgfsetstrokecolor{currentstroke}%
\pgfsetstrokeopacity{0.300000}%
\pgfsetdash{}{0pt}%
\pgfpathmoveto{\pgfqpoint{4.705831in}{0.582778in}}%
\pgfpathlineto{\pgfqpoint{4.705831in}{3.795000in}}%
\pgfusepath{stroke}%
\end{pgfscope}%
\begin{pgfscope}%
\pgfsetbuttcap%
\pgfsetroundjoin%
\definecolor{currentfill}{rgb}{0.000000,0.000000,0.000000}%
\pgfsetfillcolor{currentfill}%
\pgfsetlinewidth{0.602250pt}%
\definecolor{currentstroke}{rgb}{0.000000,0.000000,0.000000}%
\pgfsetstrokecolor{currentstroke}%
\pgfsetdash{}{0pt}%
\pgfsys@defobject{currentmarker}{\pgfqpoint{0.000000in}{-0.027778in}}{\pgfqpoint{0.000000in}{0.000000in}}{%
\pgfpathmoveto{\pgfqpoint{0.000000in}{0.000000in}}%
\pgfpathlineto{\pgfqpoint{0.000000in}{-0.027778in}}%
\pgfusepath{stroke,fill}%
}%
\begin{pgfscope}%
\pgfsys@transformshift{4.705831in}{0.582778in}%
\pgfsys@useobject{currentmarker}{}%
\end{pgfscope}%
\end{pgfscope}%
\begin{pgfscope}%
\pgfpathrectangle{\pgfqpoint{0.752778in}{0.582778in}}{\pgfqpoint{4.048611in}{3.212222in}}%
\pgfusepath{clip}%
\pgfsetrectcap%
\pgfsetroundjoin%
\pgfsetlinewidth{0.803000pt}%
\definecolor{currentstroke}{rgb}{0.690196,0.690196,0.690196}%
\pgfsetstrokecolor{currentstroke}%
\pgfsetstrokeopacity{0.300000}%
\pgfsetdash{}{0pt}%
\pgfpathmoveto{\pgfqpoint{4.750188in}{0.582778in}}%
\pgfpathlineto{\pgfqpoint{4.750188in}{3.795000in}}%
\pgfusepath{stroke}%
\end{pgfscope}%
\begin{pgfscope}%
\pgfsetbuttcap%
\pgfsetroundjoin%
\definecolor{currentfill}{rgb}{0.000000,0.000000,0.000000}%
\pgfsetfillcolor{currentfill}%
\pgfsetlinewidth{0.602250pt}%
\definecolor{currentstroke}{rgb}{0.000000,0.000000,0.000000}%
\pgfsetstrokecolor{currentstroke}%
\pgfsetdash{}{0pt}%
\pgfsys@defobject{currentmarker}{\pgfqpoint{0.000000in}{-0.027778in}}{\pgfqpoint{0.000000in}{0.000000in}}{%
\pgfpathmoveto{\pgfqpoint{0.000000in}{0.000000in}}%
\pgfpathlineto{\pgfqpoint{0.000000in}{-0.027778in}}%
\pgfusepath{stroke,fill}%
}%
\begin{pgfscope}%
\pgfsys@transformshift{4.750188in}{0.582778in}%
\pgfsys@useobject{currentmarker}{}%
\end{pgfscope}%
\end{pgfscope}%
\begin{pgfscope}%
\pgfpathrectangle{\pgfqpoint{0.752778in}{0.582778in}}{\pgfqpoint{4.048611in}{3.212222in}}%
\pgfusepath{clip}%
\pgfsetrectcap%
\pgfsetroundjoin%
\pgfsetlinewidth{0.803000pt}%
\definecolor{currentstroke}{rgb}{0.690196,0.690196,0.690196}%
\pgfsetstrokecolor{currentstroke}%
\pgfsetstrokeopacity{0.300000}%
\pgfsetdash{}{0pt}%
\pgfpathmoveto{\pgfqpoint{4.794545in}{0.582778in}}%
\pgfpathlineto{\pgfqpoint{4.794545in}{3.795000in}}%
\pgfusepath{stroke}%
\end{pgfscope}%
\begin{pgfscope}%
\pgfsetbuttcap%
\pgfsetroundjoin%
\definecolor{currentfill}{rgb}{0.000000,0.000000,0.000000}%
\pgfsetfillcolor{currentfill}%
\pgfsetlinewidth{0.602250pt}%
\definecolor{currentstroke}{rgb}{0.000000,0.000000,0.000000}%
\pgfsetstrokecolor{currentstroke}%
\pgfsetdash{}{0pt}%
\pgfsys@defobject{currentmarker}{\pgfqpoint{0.000000in}{-0.027778in}}{\pgfqpoint{0.000000in}{0.000000in}}{%
\pgfpathmoveto{\pgfqpoint{0.000000in}{0.000000in}}%
\pgfpathlineto{\pgfqpoint{0.000000in}{-0.027778in}}%
\pgfusepath{stroke,fill}%
}%
\begin{pgfscope}%
\pgfsys@transformshift{4.794545in}{0.582778in}%
\pgfsys@useobject{currentmarker}{}%
\end{pgfscope}%
\end{pgfscope}%
\begin{pgfscope}%
\definecolor{textcolor}{rgb}{0.000000,0.000000,0.000000}%
\pgfsetstrokecolor{textcolor}%
\pgfsetfillcolor{textcolor}%
\pgftext[x=2.777083in,y=0.295587in,,top]{\color{textcolor}\sffamily\fontsize{10.000000}{12.000000}\selectfont Energie [keV]}%
\end{pgfscope}%
\begin{pgfscope}%
\pgfpathrectangle{\pgfqpoint{0.752778in}{0.582778in}}{\pgfqpoint{4.048611in}{3.212222in}}%
\pgfusepath{clip}%
\pgfsetrectcap%
\pgfsetroundjoin%
\pgfsetlinewidth{0.803000pt}%
\definecolor{currentstroke}{rgb}{0.690196,0.690196,0.690196}%
\pgfsetstrokecolor{currentstroke}%
\pgfsetstrokeopacity{0.800000}%
\pgfsetdash{}{0pt}%
\pgfpathmoveto{\pgfqpoint{0.752778in}{0.728788in}}%
\pgfpathlineto{\pgfqpoint{4.801389in}{0.728788in}}%
\pgfusepath{stroke}%
\end{pgfscope}%
\begin{pgfscope}%
\pgfsetbuttcap%
\pgfsetroundjoin%
\definecolor{currentfill}{rgb}{0.000000,0.000000,0.000000}%
\pgfsetfillcolor{currentfill}%
\pgfsetlinewidth{0.803000pt}%
\definecolor{currentstroke}{rgb}{0.000000,0.000000,0.000000}%
\pgfsetstrokecolor{currentstroke}%
\pgfsetdash{}{0pt}%
\pgfsys@defobject{currentmarker}{\pgfqpoint{-0.048611in}{0.000000in}}{\pgfqpoint{0.000000in}{0.000000in}}{%
\pgfpathmoveto{\pgfqpoint{0.000000in}{0.000000in}}%
\pgfpathlineto{\pgfqpoint{-0.048611in}{0.000000in}}%
\pgfusepath{stroke,fill}%
}%
\begin{pgfscope}%
\pgfsys@transformshift{0.752778in}{0.728788in}%
\pgfsys@useobject{currentmarker}{}%
\end{pgfscope}%
\end{pgfscope}%
\begin{pgfscope}%
\pgfsetbuttcap%
\pgfsetroundjoin%
\definecolor{currentfill}{rgb}{0.000000,0.000000,0.000000}%
\pgfsetfillcolor{currentfill}%
\pgfsetlinewidth{0.803000pt}%
\definecolor{currentstroke}{rgb}{0.000000,0.000000,0.000000}%
\pgfsetstrokecolor{currentstroke}%
\pgfsetdash{}{0pt}%
\pgfsys@defobject{currentmarker}{\pgfqpoint{0.000000in}{0.000000in}}{\pgfqpoint{0.048611in}{0.000000in}}{%
\pgfpathmoveto{\pgfqpoint{0.000000in}{0.000000in}}%
\pgfpathlineto{\pgfqpoint{0.048611in}{0.000000in}}%
\pgfusepath{stroke,fill}%
}%
\begin{pgfscope}%
\pgfsys@transformshift{4.801389in}{0.728788in}%
\pgfsys@useobject{currentmarker}{}%
\end{pgfscope}%
\end{pgfscope}%
\begin{pgfscope}%
\definecolor{textcolor}{rgb}{0.000000,0.000000,0.000000}%
\pgfsetstrokecolor{textcolor}%
\pgfsetfillcolor{textcolor}%
\pgftext[x=0.346311in,y=0.676026in,left,base]{\color{textcolor}\sffamily\fontsize{10.000000}{12.000000}\selectfont 0.00}%
\end{pgfscope}%
\begin{pgfscope}%
\pgfpathrectangle{\pgfqpoint{0.752778in}{0.582778in}}{\pgfqpoint{4.048611in}{3.212222in}}%
\pgfusepath{clip}%
\pgfsetrectcap%
\pgfsetroundjoin%
\pgfsetlinewidth{0.803000pt}%
\definecolor{currentstroke}{rgb}{0.690196,0.690196,0.690196}%
\pgfsetstrokecolor{currentstroke}%
\pgfsetstrokeopacity{0.800000}%
\pgfsetdash{}{0pt}%
\pgfpathmoveto{\pgfqpoint{0.752778in}{1.166818in}}%
\pgfpathlineto{\pgfqpoint{4.801389in}{1.166818in}}%
\pgfusepath{stroke}%
\end{pgfscope}%
\begin{pgfscope}%
\pgfsetbuttcap%
\pgfsetroundjoin%
\definecolor{currentfill}{rgb}{0.000000,0.000000,0.000000}%
\pgfsetfillcolor{currentfill}%
\pgfsetlinewidth{0.803000pt}%
\definecolor{currentstroke}{rgb}{0.000000,0.000000,0.000000}%
\pgfsetstrokecolor{currentstroke}%
\pgfsetdash{}{0pt}%
\pgfsys@defobject{currentmarker}{\pgfqpoint{-0.048611in}{0.000000in}}{\pgfqpoint{0.000000in}{0.000000in}}{%
\pgfpathmoveto{\pgfqpoint{0.000000in}{0.000000in}}%
\pgfpathlineto{\pgfqpoint{-0.048611in}{0.000000in}}%
\pgfusepath{stroke,fill}%
}%
\begin{pgfscope}%
\pgfsys@transformshift{0.752778in}{1.166818in}%
\pgfsys@useobject{currentmarker}{}%
\end{pgfscope}%
\end{pgfscope}%
\begin{pgfscope}%
\pgfsetbuttcap%
\pgfsetroundjoin%
\definecolor{currentfill}{rgb}{0.000000,0.000000,0.000000}%
\pgfsetfillcolor{currentfill}%
\pgfsetlinewidth{0.803000pt}%
\definecolor{currentstroke}{rgb}{0.000000,0.000000,0.000000}%
\pgfsetstrokecolor{currentstroke}%
\pgfsetdash{}{0pt}%
\pgfsys@defobject{currentmarker}{\pgfqpoint{0.000000in}{0.000000in}}{\pgfqpoint{0.048611in}{0.000000in}}{%
\pgfpathmoveto{\pgfqpoint{0.000000in}{0.000000in}}%
\pgfpathlineto{\pgfqpoint{0.048611in}{0.000000in}}%
\pgfusepath{stroke,fill}%
}%
\begin{pgfscope}%
\pgfsys@transformshift{4.801389in}{1.166818in}%
\pgfsys@useobject{currentmarker}{}%
\end{pgfscope}%
\end{pgfscope}%
\begin{pgfscope}%
\definecolor{textcolor}{rgb}{0.000000,0.000000,0.000000}%
\pgfsetstrokecolor{textcolor}%
\pgfsetfillcolor{textcolor}%
\pgftext[x=0.346311in,y=1.114057in,left,base]{\color{textcolor}\sffamily\fontsize{10.000000}{12.000000}\selectfont 0.15}%
\end{pgfscope}%
\begin{pgfscope}%
\pgfpathrectangle{\pgfqpoint{0.752778in}{0.582778in}}{\pgfqpoint{4.048611in}{3.212222in}}%
\pgfusepath{clip}%
\pgfsetrectcap%
\pgfsetroundjoin%
\pgfsetlinewidth{0.803000pt}%
\definecolor{currentstroke}{rgb}{0.690196,0.690196,0.690196}%
\pgfsetstrokecolor{currentstroke}%
\pgfsetstrokeopacity{0.800000}%
\pgfsetdash{}{0pt}%
\pgfpathmoveto{\pgfqpoint{0.752778in}{1.604848in}}%
\pgfpathlineto{\pgfqpoint{4.801389in}{1.604848in}}%
\pgfusepath{stroke}%
\end{pgfscope}%
\begin{pgfscope}%
\pgfsetbuttcap%
\pgfsetroundjoin%
\definecolor{currentfill}{rgb}{0.000000,0.000000,0.000000}%
\pgfsetfillcolor{currentfill}%
\pgfsetlinewidth{0.803000pt}%
\definecolor{currentstroke}{rgb}{0.000000,0.000000,0.000000}%
\pgfsetstrokecolor{currentstroke}%
\pgfsetdash{}{0pt}%
\pgfsys@defobject{currentmarker}{\pgfqpoint{-0.048611in}{0.000000in}}{\pgfqpoint{0.000000in}{0.000000in}}{%
\pgfpathmoveto{\pgfqpoint{0.000000in}{0.000000in}}%
\pgfpathlineto{\pgfqpoint{-0.048611in}{0.000000in}}%
\pgfusepath{stroke,fill}%
}%
\begin{pgfscope}%
\pgfsys@transformshift{0.752778in}{1.604848in}%
\pgfsys@useobject{currentmarker}{}%
\end{pgfscope}%
\end{pgfscope}%
\begin{pgfscope}%
\pgfsetbuttcap%
\pgfsetroundjoin%
\definecolor{currentfill}{rgb}{0.000000,0.000000,0.000000}%
\pgfsetfillcolor{currentfill}%
\pgfsetlinewidth{0.803000pt}%
\definecolor{currentstroke}{rgb}{0.000000,0.000000,0.000000}%
\pgfsetstrokecolor{currentstroke}%
\pgfsetdash{}{0pt}%
\pgfsys@defobject{currentmarker}{\pgfqpoint{0.000000in}{0.000000in}}{\pgfqpoint{0.048611in}{0.000000in}}{%
\pgfpathmoveto{\pgfqpoint{0.000000in}{0.000000in}}%
\pgfpathlineto{\pgfqpoint{0.048611in}{0.000000in}}%
\pgfusepath{stroke,fill}%
}%
\begin{pgfscope}%
\pgfsys@transformshift{4.801389in}{1.604848in}%
\pgfsys@useobject{currentmarker}{}%
\end{pgfscope}%
\end{pgfscope}%
\begin{pgfscope}%
\definecolor{textcolor}{rgb}{0.000000,0.000000,0.000000}%
\pgfsetstrokecolor{textcolor}%
\pgfsetfillcolor{textcolor}%
\pgftext[x=0.346311in,y=1.552087in,left,base]{\color{textcolor}\sffamily\fontsize{10.000000}{12.000000}\selectfont 0.30}%
\end{pgfscope}%
\begin{pgfscope}%
\pgfpathrectangle{\pgfqpoint{0.752778in}{0.582778in}}{\pgfqpoint{4.048611in}{3.212222in}}%
\pgfusepath{clip}%
\pgfsetrectcap%
\pgfsetroundjoin%
\pgfsetlinewidth{0.803000pt}%
\definecolor{currentstroke}{rgb}{0.690196,0.690196,0.690196}%
\pgfsetstrokecolor{currentstroke}%
\pgfsetstrokeopacity{0.800000}%
\pgfsetdash{}{0pt}%
\pgfpathmoveto{\pgfqpoint{0.752778in}{2.042879in}}%
\pgfpathlineto{\pgfqpoint{4.801389in}{2.042879in}}%
\pgfusepath{stroke}%
\end{pgfscope}%
\begin{pgfscope}%
\pgfsetbuttcap%
\pgfsetroundjoin%
\definecolor{currentfill}{rgb}{0.000000,0.000000,0.000000}%
\pgfsetfillcolor{currentfill}%
\pgfsetlinewidth{0.803000pt}%
\definecolor{currentstroke}{rgb}{0.000000,0.000000,0.000000}%
\pgfsetstrokecolor{currentstroke}%
\pgfsetdash{}{0pt}%
\pgfsys@defobject{currentmarker}{\pgfqpoint{-0.048611in}{0.000000in}}{\pgfqpoint{0.000000in}{0.000000in}}{%
\pgfpathmoveto{\pgfqpoint{0.000000in}{0.000000in}}%
\pgfpathlineto{\pgfqpoint{-0.048611in}{0.000000in}}%
\pgfusepath{stroke,fill}%
}%
\begin{pgfscope}%
\pgfsys@transformshift{0.752778in}{2.042879in}%
\pgfsys@useobject{currentmarker}{}%
\end{pgfscope}%
\end{pgfscope}%
\begin{pgfscope}%
\pgfsetbuttcap%
\pgfsetroundjoin%
\definecolor{currentfill}{rgb}{0.000000,0.000000,0.000000}%
\pgfsetfillcolor{currentfill}%
\pgfsetlinewidth{0.803000pt}%
\definecolor{currentstroke}{rgb}{0.000000,0.000000,0.000000}%
\pgfsetstrokecolor{currentstroke}%
\pgfsetdash{}{0pt}%
\pgfsys@defobject{currentmarker}{\pgfqpoint{0.000000in}{0.000000in}}{\pgfqpoint{0.048611in}{0.000000in}}{%
\pgfpathmoveto{\pgfqpoint{0.000000in}{0.000000in}}%
\pgfpathlineto{\pgfqpoint{0.048611in}{0.000000in}}%
\pgfusepath{stroke,fill}%
}%
\begin{pgfscope}%
\pgfsys@transformshift{4.801389in}{2.042879in}%
\pgfsys@useobject{currentmarker}{}%
\end{pgfscope}%
\end{pgfscope}%
\begin{pgfscope}%
\definecolor{textcolor}{rgb}{0.000000,0.000000,0.000000}%
\pgfsetstrokecolor{textcolor}%
\pgfsetfillcolor{textcolor}%
\pgftext[x=0.346311in,y=1.990117in,left,base]{\color{textcolor}\sffamily\fontsize{10.000000}{12.000000}\selectfont 0.45}%
\end{pgfscope}%
\begin{pgfscope}%
\pgfpathrectangle{\pgfqpoint{0.752778in}{0.582778in}}{\pgfqpoint{4.048611in}{3.212222in}}%
\pgfusepath{clip}%
\pgfsetrectcap%
\pgfsetroundjoin%
\pgfsetlinewidth{0.803000pt}%
\definecolor{currentstroke}{rgb}{0.690196,0.690196,0.690196}%
\pgfsetstrokecolor{currentstroke}%
\pgfsetstrokeopacity{0.800000}%
\pgfsetdash{}{0pt}%
\pgfpathmoveto{\pgfqpoint{0.752778in}{2.480909in}}%
\pgfpathlineto{\pgfqpoint{4.801389in}{2.480909in}}%
\pgfusepath{stroke}%
\end{pgfscope}%
\begin{pgfscope}%
\pgfsetbuttcap%
\pgfsetroundjoin%
\definecolor{currentfill}{rgb}{0.000000,0.000000,0.000000}%
\pgfsetfillcolor{currentfill}%
\pgfsetlinewidth{0.803000pt}%
\definecolor{currentstroke}{rgb}{0.000000,0.000000,0.000000}%
\pgfsetstrokecolor{currentstroke}%
\pgfsetdash{}{0pt}%
\pgfsys@defobject{currentmarker}{\pgfqpoint{-0.048611in}{0.000000in}}{\pgfqpoint{0.000000in}{0.000000in}}{%
\pgfpathmoveto{\pgfqpoint{0.000000in}{0.000000in}}%
\pgfpathlineto{\pgfqpoint{-0.048611in}{0.000000in}}%
\pgfusepath{stroke,fill}%
}%
\begin{pgfscope}%
\pgfsys@transformshift{0.752778in}{2.480909in}%
\pgfsys@useobject{currentmarker}{}%
\end{pgfscope}%
\end{pgfscope}%
\begin{pgfscope}%
\pgfsetbuttcap%
\pgfsetroundjoin%
\definecolor{currentfill}{rgb}{0.000000,0.000000,0.000000}%
\pgfsetfillcolor{currentfill}%
\pgfsetlinewidth{0.803000pt}%
\definecolor{currentstroke}{rgb}{0.000000,0.000000,0.000000}%
\pgfsetstrokecolor{currentstroke}%
\pgfsetdash{}{0pt}%
\pgfsys@defobject{currentmarker}{\pgfqpoint{0.000000in}{0.000000in}}{\pgfqpoint{0.048611in}{0.000000in}}{%
\pgfpathmoveto{\pgfqpoint{0.000000in}{0.000000in}}%
\pgfpathlineto{\pgfqpoint{0.048611in}{0.000000in}}%
\pgfusepath{stroke,fill}%
}%
\begin{pgfscope}%
\pgfsys@transformshift{4.801389in}{2.480909in}%
\pgfsys@useobject{currentmarker}{}%
\end{pgfscope}%
\end{pgfscope}%
\begin{pgfscope}%
\definecolor{textcolor}{rgb}{0.000000,0.000000,0.000000}%
\pgfsetstrokecolor{textcolor}%
\pgfsetfillcolor{textcolor}%
\pgftext[x=0.346311in,y=2.428148in,left,base]{\color{textcolor}\sffamily\fontsize{10.000000}{12.000000}\selectfont 0.60}%
\end{pgfscope}%
\begin{pgfscope}%
\pgfpathrectangle{\pgfqpoint{0.752778in}{0.582778in}}{\pgfqpoint{4.048611in}{3.212222in}}%
\pgfusepath{clip}%
\pgfsetrectcap%
\pgfsetroundjoin%
\pgfsetlinewidth{0.803000pt}%
\definecolor{currentstroke}{rgb}{0.690196,0.690196,0.690196}%
\pgfsetstrokecolor{currentstroke}%
\pgfsetstrokeopacity{0.800000}%
\pgfsetdash{}{0pt}%
\pgfpathmoveto{\pgfqpoint{0.752778in}{2.918939in}}%
\pgfpathlineto{\pgfqpoint{4.801389in}{2.918939in}}%
\pgfusepath{stroke}%
\end{pgfscope}%
\begin{pgfscope}%
\pgfsetbuttcap%
\pgfsetroundjoin%
\definecolor{currentfill}{rgb}{0.000000,0.000000,0.000000}%
\pgfsetfillcolor{currentfill}%
\pgfsetlinewidth{0.803000pt}%
\definecolor{currentstroke}{rgb}{0.000000,0.000000,0.000000}%
\pgfsetstrokecolor{currentstroke}%
\pgfsetdash{}{0pt}%
\pgfsys@defobject{currentmarker}{\pgfqpoint{-0.048611in}{0.000000in}}{\pgfqpoint{0.000000in}{0.000000in}}{%
\pgfpathmoveto{\pgfqpoint{0.000000in}{0.000000in}}%
\pgfpathlineto{\pgfqpoint{-0.048611in}{0.000000in}}%
\pgfusepath{stroke,fill}%
}%
\begin{pgfscope}%
\pgfsys@transformshift{0.752778in}{2.918939in}%
\pgfsys@useobject{currentmarker}{}%
\end{pgfscope}%
\end{pgfscope}%
\begin{pgfscope}%
\pgfsetbuttcap%
\pgfsetroundjoin%
\definecolor{currentfill}{rgb}{0.000000,0.000000,0.000000}%
\pgfsetfillcolor{currentfill}%
\pgfsetlinewidth{0.803000pt}%
\definecolor{currentstroke}{rgb}{0.000000,0.000000,0.000000}%
\pgfsetstrokecolor{currentstroke}%
\pgfsetdash{}{0pt}%
\pgfsys@defobject{currentmarker}{\pgfqpoint{0.000000in}{0.000000in}}{\pgfqpoint{0.048611in}{0.000000in}}{%
\pgfpathmoveto{\pgfqpoint{0.000000in}{0.000000in}}%
\pgfpathlineto{\pgfqpoint{0.048611in}{0.000000in}}%
\pgfusepath{stroke,fill}%
}%
\begin{pgfscope}%
\pgfsys@transformshift{4.801389in}{2.918939in}%
\pgfsys@useobject{currentmarker}{}%
\end{pgfscope}%
\end{pgfscope}%
\begin{pgfscope}%
\definecolor{textcolor}{rgb}{0.000000,0.000000,0.000000}%
\pgfsetstrokecolor{textcolor}%
\pgfsetfillcolor{textcolor}%
\pgftext[x=0.346311in,y=2.866178in,left,base]{\color{textcolor}\sffamily\fontsize{10.000000}{12.000000}\selectfont 0.75}%
\end{pgfscope}%
\begin{pgfscope}%
\pgfpathrectangle{\pgfqpoint{0.752778in}{0.582778in}}{\pgfqpoint{4.048611in}{3.212222in}}%
\pgfusepath{clip}%
\pgfsetrectcap%
\pgfsetroundjoin%
\pgfsetlinewidth{0.803000pt}%
\definecolor{currentstroke}{rgb}{0.690196,0.690196,0.690196}%
\pgfsetstrokecolor{currentstroke}%
\pgfsetstrokeopacity{0.800000}%
\pgfsetdash{}{0pt}%
\pgfpathmoveto{\pgfqpoint{0.752778in}{3.356970in}}%
\pgfpathlineto{\pgfqpoint{4.801389in}{3.356970in}}%
\pgfusepath{stroke}%
\end{pgfscope}%
\begin{pgfscope}%
\pgfsetbuttcap%
\pgfsetroundjoin%
\definecolor{currentfill}{rgb}{0.000000,0.000000,0.000000}%
\pgfsetfillcolor{currentfill}%
\pgfsetlinewidth{0.803000pt}%
\definecolor{currentstroke}{rgb}{0.000000,0.000000,0.000000}%
\pgfsetstrokecolor{currentstroke}%
\pgfsetdash{}{0pt}%
\pgfsys@defobject{currentmarker}{\pgfqpoint{-0.048611in}{0.000000in}}{\pgfqpoint{0.000000in}{0.000000in}}{%
\pgfpathmoveto{\pgfqpoint{0.000000in}{0.000000in}}%
\pgfpathlineto{\pgfqpoint{-0.048611in}{0.000000in}}%
\pgfusepath{stroke,fill}%
}%
\begin{pgfscope}%
\pgfsys@transformshift{0.752778in}{3.356970in}%
\pgfsys@useobject{currentmarker}{}%
\end{pgfscope}%
\end{pgfscope}%
\begin{pgfscope}%
\pgfsetbuttcap%
\pgfsetroundjoin%
\definecolor{currentfill}{rgb}{0.000000,0.000000,0.000000}%
\pgfsetfillcolor{currentfill}%
\pgfsetlinewidth{0.803000pt}%
\definecolor{currentstroke}{rgb}{0.000000,0.000000,0.000000}%
\pgfsetstrokecolor{currentstroke}%
\pgfsetdash{}{0pt}%
\pgfsys@defobject{currentmarker}{\pgfqpoint{0.000000in}{0.000000in}}{\pgfqpoint{0.048611in}{0.000000in}}{%
\pgfpathmoveto{\pgfqpoint{0.000000in}{0.000000in}}%
\pgfpathlineto{\pgfqpoint{0.048611in}{0.000000in}}%
\pgfusepath{stroke,fill}%
}%
\begin{pgfscope}%
\pgfsys@transformshift{4.801389in}{3.356970in}%
\pgfsys@useobject{currentmarker}{}%
\end{pgfscope}%
\end{pgfscope}%
\begin{pgfscope}%
\definecolor{textcolor}{rgb}{0.000000,0.000000,0.000000}%
\pgfsetstrokecolor{textcolor}%
\pgfsetfillcolor{textcolor}%
\pgftext[x=0.346311in,y=3.304208in,left,base]{\color{textcolor}\sffamily\fontsize{10.000000}{12.000000}\selectfont 0.90}%
\end{pgfscope}%
\begin{pgfscope}%
\pgfpathrectangle{\pgfqpoint{0.752778in}{0.582778in}}{\pgfqpoint{4.048611in}{3.212222in}}%
\pgfusepath{clip}%
\pgfsetrectcap%
\pgfsetroundjoin%
\pgfsetlinewidth{0.803000pt}%
\definecolor{currentstroke}{rgb}{0.690196,0.690196,0.690196}%
\pgfsetstrokecolor{currentstroke}%
\pgfsetstrokeopacity{0.800000}%
\pgfsetdash{}{0pt}%
\pgfpathmoveto{\pgfqpoint{0.752778in}{3.795000in}}%
\pgfpathlineto{\pgfqpoint{4.801389in}{3.795000in}}%
\pgfusepath{stroke}%
\end{pgfscope}%
\begin{pgfscope}%
\pgfsetbuttcap%
\pgfsetroundjoin%
\definecolor{currentfill}{rgb}{0.000000,0.000000,0.000000}%
\pgfsetfillcolor{currentfill}%
\pgfsetlinewidth{0.803000pt}%
\definecolor{currentstroke}{rgb}{0.000000,0.000000,0.000000}%
\pgfsetstrokecolor{currentstroke}%
\pgfsetdash{}{0pt}%
\pgfsys@defobject{currentmarker}{\pgfqpoint{-0.048611in}{0.000000in}}{\pgfqpoint{0.000000in}{0.000000in}}{%
\pgfpathmoveto{\pgfqpoint{0.000000in}{0.000000in}}%
\pgfpathlineto{\pgfqpoint{-0.048611in}{0.000000in}}%
\pgfusepath{stroke,fill}%
}%
\begin{pgfscope}%
\pgfsys@transformshift{0.752778in}{3.795000in}%
\pgfsys@useobject{currentmarker}{}%
\end{pgfscope}%
\end{pgfscope}%
\begin{pgfscope}%
\pgfsetbuttcap%
\pgfsetroundjoin%
\definecolor{currentfill}{rgb}{0.000000,0.000000,0.000000}%
\pgfsetfillcolor{currentfill}%
\pgfsetlinewidth{0.803000pt}%
\definecolor{currentstroke}{rgb}{0.000000,0.000000,0.000000}%
\pgfsetstrokecolor{currentstroke}%
\pgfsetdash{}{0pt}%
\pgfsys@defobject{currentmarker}{\pgfqpoint{0.000000in}{0.000000in}}{\pgfqpoint{0.048611in}{0.000000in}}{%
\pgfpathmoveto{\pgfqpoint{0.000000in}{0.000000in}}%
\pgfpathlineto{\pgfqpoint{0.048611in}{0.000000in}}%
\pgfusepath{stroke,fill}%
}%
\begin{pgfscope}%
\pgfsys@transformshift{4.801389in}{3.795000in}%
\pgfsys@useobject{currentmarker}{}%
\end{pgfscope}%
\end{pgfscope}%
\begin{pgfscope}%
\definecolor{textcolor}{rgb}{0.000000,0.000000,0.000000}%
\pgfsetstrokecolor{textcolor}%
\pgfsetfillcolor{textcolor}%
\pgftext[x=0.346311in,y=3.742238in,left,base]{\color{textcolor}\sffamily\fontsize{10.000000}{12.000000}\selectfont 1.05}%
\end{pgfscope}%
\begin{pgfscope}%
\pgfpathrectangle{\pgfqpoint{0.752778in}{0.582778in}}{\pgfqpoint{4.048611in}{3.212222in}}%
\pgfusepath{clip}%
\pgfsetrectcap%
\pgfsetroundjoin%
\pgfsetlinewidth{0.803000pt}%
\definecolor{currentstroke}{rgb}{0.690196,0.690196,0.690196}%
\pgfsetstrokecolor{currentstroke}%
\pgfsetstrokeopacity{0.300000}%
\pgfsetdash{}{0pt}%
\pgfpathmoveto{\pgfqpoint{0.752778in}{0.597379in}}%
\pgfpathlineto{\pgfqpoint{4.801389in}{0.597379in}}%
\pgfusepath{stroke}%
\end{pgfscope}%
\begin{pgfscope}%
\pgfsetbuttcap%
\pgfsetroundjoin%
\definecolor{currentfill}{rgb}{0.000000,0.000000,0.000000}%
\pgfsetfillcolor{currentfill}%
\pgfsetlinewidth{0.602250pt}%
\definecolor{currentstroke}{rgb}{0.000000,0.000000,0.000000}%
\pgfsetstrokecolor{currentstroke}%
\pgfsetdash{}{0pt}%
\pgfsys@defobject{currentmarker}{\pgfqpoint{-0.027778in}{0.000000in}}{\pgfqpoint{0.000000in}{0.000000in}}{%
\pgfpathmoveto{\pgfqpoint{0.000000in}{0.000000in}}%
\pgfpathlineto{\pgfqpoint{-0.027778in}{0.000000in}}%
\pgfusepath{stroke,fill}%
}%
\begin{pgfscope}%
\pgfsys@transformshift{0.752778in}{0.597379in}%
\pgfsys@useobject{currentmarker}{}%
\end{pgfscope}%
\end{pgfscope}%
\begin{pgfscope}%
\pgfsetbuttcap%
\pgfsetroundjoin%
\definecolor{currentfill}{rgb}{0.000000,0.000000,0.000000}%
\pgfsetfillcolor{currentfill}%
\pgfsetlinewidth{0.602250pt}%
\definecolor{currentstroke}{rgb}{0.000000,0.000000,0.000000}%
\pgfsetstrokecolor{currentstroke}%
\pgfsetdash{}{0pt}%
\pgfsys@defobject{currentmarker}{\pgfqpoint{0.000000in}{0.000000in}}{\pgfqpoint{0.027778in}{0.000000in}}{%
\pgfpathmoveto{\pgfqpoint{0.000000in}{0.000000in}}%
\pgfpathlineto{\pgfqpoint{0.027778in}{0.000000in}}%
\pgfusepath{stroke,fill}%
}%
\begin{pgfscope}%
\pgfsys@transformshift{4.801389in}{0.597379in}%
\pgfsys@useobject{currentmarker}{}%
\end{pgfscope}%
\end{pgfscope}%
\begin{pgfscope}%
\pgfpathrectangle{\pgfqpoint{0.752778in}{0.582778in}}{\pgfqpoint{4.048611in}{3.212222in}}%
\pgfusepath{clip}%
\pgfsetrectcap%
\pgfsetroundjoin%
\pgfsetlinewidth{0.803000pt}%
\definecolor{currentstroke}{rgb}{0.690196,0.690196,0.690196}%
\pgfsetstrokecolor{currentstroke}%
\pgfsetstrokeopacity{0.300000}%
\pgfsetdash{}{0pt}%
\pgfpathmoveto{\pgfqpoint{0.752778in}{0.641182in}}%
\pgfpathlineto{\pgfqpoint{4.801389in}{0.641182in}}%
\pgfusepath{stroke}%
\end{pgfscope}%
\begin{pgfscope}%
\pgfsetbuttcap%
\pgfsetroundjoin%
\definecolor{currentfill}{rgb}{0.000000,0.000000,0.000000}%
\pgfsetfillcolor{currentfill}%
\pgfsetlinewidth{0.602250pt}%
\definecolor{currentstroke}{rgb}{0.000000,0.000000,0.000000}%
\pgfsetstrokecolor{currentstroke}%
\pgfsetdash{}{0pt}%
\pgfsys@defobject{currentmarker}{\pgfqpoint{-0.027778in}{0.000000in}}{\pgfqpoint{0.000000in}{0.000000in}}{%
\pgfpathmoveto{\pgfqpoint{0.000000in}{0.000000in}}%
\pgfpathlineto{\pgfqpoint{-0.027778in}{0.000000in}}%
\pgfusepath{stroke,fill}%
}%
\begin{pgfscope}%
\pgfsys@transformshift{0.752778in}{0.641182in}%
\pgfsys@useobject{currentmarker}{}%
\end{pgfscope}%
\end{pgfscope}%
\begin{pgfscope}%
\pgfsetbuttcap%
\pgfsetroundjoin%
\definecolor{currentfill}{rgb}{0.000000,0.000000,0.000000}%
\pgfsetfillcolor{currentfill}%
\pgfsetlinewidth{0.602250pt}%
\definecolor{currentstroke}{rgb}{0.000000,0.000000,0.000000}%
\pgfsetstrokecolor{currentstroke}%
\pgfsetdash{}{0pt}%
\pgfsys@defobject{currentmarker}{\pgfqpoint{0.000000in}{0.000000in}}{\pgfqpoint{0.027778in}{0.000000in}}{%
\pgfpathmoveto{\pgfqpoint{0.000000in}{0.000000in}}%
\pgfpathlineto{\pgfqpoint{0.027778in}{0.000000in}}%
\pgfusepath{stroke,fill}%
}%
\begin{pgfscope}%
\pgfsys@transformshift{4.801389in}{0.641182in}%
\pgfsys@useobject{currentmarker}{}%
\end{pgfscope}%
\end{pgfscope}%
\begin{pgfscope}%
\pgfpathrectangle{\pgfqpoint{0.752778in}{0.582778in}}{\pgfqpoint{4.048611in}{3.212222in}}%
\pgfusepath{clip}%
\pgfsetrectcap%
\pgfsetroundjoin%
\pgfsetlinewidth{0.803000pt}%
\definecolor{currentstroke}{rgb}{0.690196,0.690196,0.690196}%
\pgfsetstrokecolor{currentstroke}%
\pgfsetstrokeopacity{0.300000}%
\pgfsetdash{}{0pt}%
\pgfpathmoveto{\pgfqpoint{0.752778in}{0.684985in}}%
\pgfpathlineto{\pgfqpoint{4.801389in}{0.684985in}}%
\pgfusepath{stroke}%
\end{pgfscope}%
\begin{pgfscope}%
\pgfsetbuttcap%
\pgfsetroundjoin%
\definecolor{currentfill}{rgb}{0.000000,0.000000,0.000000}%
\pgfsetfillcolor{currentfill}%
\pgfsetlinewidth{0.602250pt}%
\definecolor{currentstroke}{rgb}{0.000000,0.000000,0.000000}%
\pgfsetstrokecolor{currentstroke}%
\pgfsetdash{}{0pt}%
\pgfsys@defobject{currentmarker}{\pgfqpoint{-0.027778in}{0.000000in}}{\pgfqpoint{0.000000in}{0.000000in}}{%
\pgfpathmoveto{\pgfqpoint{0.000000in}{0.000000in}}%
\pgfpathlineto{\pgfqpoint{-0.027778in}{0.000000in}}%
\pgfusepath{stroke,fill}%
}%
\begin{pgfscope}%
\pgfsys@transformshift{0.752778in}{0.684985in}%
\pgfsys@useobject{currentmarker}{}%
\end{pgfscope}%
\end{pgfscope}%
\begin{pgfscope}%
\pgfsetbuttcap%
\pgfsetroundjoin%
\definecolor{currentfill}{rgb}{0.000000,0.000000,0.000000}%
\pgfsetfillcolor{currentfill}%
\pgfsetlinewidth{0.602250pt}%
\definecolor{currentstroke}{rgb}{0.000000,0.000000,0.000000}%
\pgfsetstrokecolor{currentstroke}%
\pgfsetdash{}{0pt}%
\pgfsys@defobject{currentmarker}{\pgfqpoint{0.000000in}{0.000000in}}{\pgfqpoint{0.027778in}{0.000000in}}{%
\pgfpathmoveto{\pgfqpoint{0.000000in}{0.000000in}}%
\pgfpathlineto{\pgfqpoint{0.027778in}{0.000000in}}%
\pgfusepath{stroke,fill}%
}%
\begin{pgfscope}%
\pgfsys@transformshift{4.801389in}{0.684985in}%
\pgfsys@useobject{currentmarker}{}%
\end{pgfscope}%
\end{pgfscope}%
\begin{pgfscope}%
\pgfpathrectangle{\pgfqpoint{0.752778in}{0.582778in}}{\pgfqpoint{4.048611in}{3.212222in}}%
\pgfusepath{clip}%
\pgfsetrectcap%
\pgfsetroundjoin%
\pgfsetlinewidth{0.803000pt}%
\definecolor{currentstroke}{rgb}{0.690196,0.690196,0.690196}%
\pgfsetstrokecolor{currentstroke}%
\pgfsetstrokeopacity{0.300000}%
\pgfsetdash{}{0pt}%
\pgfpathmoveto{\pgfqpoint{0.752778in}{0.772591in}}%
\pgfpathlineto{\pgfqpoint{4.801389in}{0.772591in}}%
\pgfusepath{stroke}%
\end{pgfscope}%
\begin{pgfscope}%
\pgfsetbuttcap%
\pgfsetroundjoin%
\definecolor{currentfill}{rgb}{0.000000,0.000000,0.000000}%
\pgfsetfillcolor{currentfill}%
\pgfsetlinewidth{0.602250pt}%
\definecolor{currentstroke}{rgb}{0.000000,0.000000,0.000000}%
\pgfsetstrokecolor{currentstroke}%
\pgfsetdash{}{0pt}%
\pgfsys@defobject{currentmarker}{\pgfqpoint{-0.027778in}{0.000000in}}{\pgfqpoint{0.000000in}{0.000000in}}{%
\pgfpathmoveto{\pgfqpoint{0.000000in}{0.000000in}}%
\pgfpathlineto{\pgfqpoint{-0.027778in}{0.000000in}}%
\pgfusepath{stroke,fill}%
}%
\begin{pgfscope}%
\pgfsys@transformshift{0.752778in}{0.772591in}%
\pgfsys@useobject{currentmarker}{}%
\end{pgfscope}%
\end{pgfscope}%
\begin{pgfscope}%
\pgfsetbuttcap%
\pgfsetroundjoin%
\definecolor{currentfill}{rgb}{0.000000,0.000000,0.000000}%
\pgfsetfillcolor{currentfill}%
\pgfsetlinewidth{0.602250pt}%
\definecolor{currentstroke}{rgb}{0.000000,0.000000,0.000000}%
\pgfsetstrokecolor{currentstroke}%
\pgfsetdash{}{0pt}%
\pgfsys@defobject{currentmarker}{\pgfqpoint{0.000000in}{0.000000in}}{\pgfqpoint{0.027778in}{0.000000in}}{%
\pgfpathmoveto{\pgfqpoint{0.000000in}{0.000000in}}%
\pgfpathlineto{\pgfqpoint{0.027778in}{0.000000in}}%
\pgfusepath{stroke,fill}%
}%
\begin{pgfscope}%
\pgfsys@transformshift{4.801389in}{0.772591in}%
\pgfsys@useobject{currentmarker}{}%
\end{pgfscope}%
\end{pgfscope}%
\begin{pgfscope}%
\pgfpathrectangle{\pgfqpoint{0.752778in}{0.582778in}}{\pgfqpoint{4.048611in}{3.212222in}}%
\pgfusepath{clip}%
\pgfsetrectcap%
\pgfsetroundjoin%
\pgfsetlinewidth{0.803000pt}%
\definecolor{currentstroke}{rgb}{0.690196,0.690196,0.690196}%
\pgfsetstrokecolor{currentstroke}%
\pgfsetstrokeopacity{0.300000}%
\pgfsetdash{}{0pt}%
\pgfpathmoveto{\pgfqpoint{0.752778in}{0.816394in}}%
\pgfpathlineto{\pgfqpoint{4.801389in}{0.816394in}}%
\pgfusepath{stroke}%
\end{pgfscope}%
\begin{pgfscope}%
\pgfsetbuttcap%
\pgfsetroundjoin%
\definecolor{currentfill}{rgb}{0.000000,0.000000,0.000000}%
\pgfsetfillcolor{currentfill}%
\pgfsetlinewidth{0.602250pt}%
\definecolor{currentstroke}{rgb}{0.000000,0.000000,0.000000}%
\pgfsetstrokecolor{currentstroke}%
\pgfsetdash{}{0pt}%
\pgfsys@defobject{currentmarker}{\pgfqpoint{-0.027778in}{0.000000in}}{\pgfqpoint{0.000000in}{0.000000in}}{%
\pgfpathmoveto{\pgfqpoint{0.000000in}{0.000000in}}%
\pgfpathlineto{\pgfqpoint{-0.027778in}{0.000000in}}%
\pgfusepath{stroke,fill}%
}%
\begin{pgfscope}%
\pgfsys@transformshift{0.752778in}{0.816394in}%
\pgfsys@useobject{currentmarker}{}%
\end{pgfscope}%
\end{pgfscope}%
\begin{pgfscope}%
\pgfsetbuttcap%
\pgfsetroundjoin%
\definecolor{currentfill}{rgb}{0.000000,0.000000,0.000000}%
\pgfsetfillcolor{currentfill}%
\pgfsetlinewidth{0.602250pt}%
\definecolor{currentstroke}{rgb}{0.000000,0.000000,0.000000}%
\pgfsetstrokecolor{currentstroke}%
\pgfsetdash{}{0pt}%
\pgfsys@defobject{currentmarker}{\pgfqpoint{0.000000in}{0.000000in}}{\pgfqpoint{0.027778in}{0.000000in}}{%
\pgfpathmoveto{\pgfqpoint{0.000000in}{0.000000in}}%
\pgfpathlineto{\pgfqpoint{0.027778in}{0.000000in}}%
\pgfusepath{stroke,fill}%
}%
\begin{pgfscope}%
\pgfsys@transformshift{4.801389in}{0.816394in}%
\pgfsys@useobject{currentmarker}{}%
\end{pgfscope}%
\end{pgfscope}%
\begin{pgfscope}%
\pgfpathrectangle{\pgfqpoint{0.752778in}{0.582778in}}{\pgfqpoint{4.048611in}{3.212222in}}%
\pgfusepath{clip}%
\pgfsetrectcap%
\pgfsetroundjoin%
\pgfsetlinewidth{0.803000pt}%
\definecolor{currentstroke}{rgb}{0.690196,0.690196,0.690196}%
\pgfsetstrokecolor{currentstroke}%
\pgfsetstrokeopacity{0.300000}%
\pgfsetdash{}{0pt}%
\pgfpathmoveto{\pgfqpoint{0.752778in}{0.860197in}}%
\pgfpathlineto{\pgfqpoint{4.801389in}{0.860197in}}%
\pgfusepath{stroke}%
\end{pgfscope}%
\begin{pgfscope}%
\pgfsetbuttcap%
\pgfsetroundjoin%
\definecolor{currentfill}{rgb}{0.000000,0.000000,0.000000}%
\pgfsetfillcolor{currentfill}%
\pgfsetlinewidth{0.602250pt}%
\definecolor{currentstroke}{rgb}{0.000000,0.000000,0.000000}%
\pgfsetstrokecolor{currentstroke}%
\pgfsetdash{}{0pt}%
\pgfsys@defobject{currentmarker}{\pgfqpoint{-0.027778in}{0.000000in}}{\pgfqpoint{0.000000in}{0.000000in}}{%
\pgfpathmoveto{\pgfqpoint{0.000000in}{0.000000in}}%
\pgfpathlineto{\pgfqpoint{-0.027778in}{0.000000in}}%
\pgfusepath{stroke,fill}%
}%
\begin{pgfscope}%
\pgfsys@transformshift{0.752778in}{0.860197in}%
\pgfsys@useobject{currentmarker}{}%
\end{pgfscope}%
\end{pgfscope}%
\begin{pgfscope}%
\pgfsetbuttcap%
\pgfsetroundjoin%
\definecolor{currentfill}{rgb}{0.000000,0.000000,0.000000}%
\pgfsetfillcolor{currentfill}%
\pgfsetlinewidth{0.602250pt}%
\definecolor{currentstroke}{rgb}{0.000000,0.000000,0.000000}%
\pgfsetstrokecolor{currentstroke}%
\pgfsetdash{}{0pt}%
\pgfsys@defobject{currentmarker}{\pgfqpoint{0.000000in}{0.000000in}}{\pgfqpoint{0.027778in}{0.000000in}}{%
\pgfpathmoveto{\pgfqpoint{0.000000in}{0.000000in}}%
\pgfpathlineto{\pgfqpoint{0.027778in}{0.000000in}}%
\pgfusepath{stroke,fill}%
}%
\begin{pgfscope}%
\pgfsys@transformshift{4.801389in}{0.860197in}%
\pgfsys@useobject{currentmarker}{}%
\end{pgfscope}%
\end{pgfscope}%
\begin{pgfscope}%
\pgfpathrectangle{\pgfqpoint{0.752778in}{0.582778in}}{\pgfqpoint{4.048611in}{3.212222in}}%
\pgfusepath{clip}%
\pgfsetrectcap%
\pgfsetroundjoin%
\pgfsetlinewidth{0.803000pt}%
\definecolor{currentstroke}{rgb}{0.690196,0.690196,0.690196}%
\pgfsetstrokecolor{currentstroke}%
\pgfsetstrokeopacity{0.300000}%
\pgfsetdash{}{0pt}%
\pgfpathmoveto{\pgfqpoint{0.752778in}{0.904000in}}%
\pgfpathlineto{\pgfqpoint{4.801389in}{0.904000in}}%
\pgfusepath{stroke}%
\end{pgfscope}%
\begin{pgfscope}%
\pgfsetbuttcap%
\pgfsetroundjoin%
\definecolor{currentfill}{rgb}{0.000000,0.000000,0.000000}%
\pgfsetfillcolor{currentfill}%
\pgfsetlinewidth{0.602250pt}%
\definecolor{currentstroke}{rgb}{0.000000,0.000000,0.000000}%
\pgfsetstrokecolor{currentstroke}%
\pgfsetdash{}{0pt}%
\pgfsys@defobject{currentmarker}{\pgfqpoint{-0.027778in}{0.000000in}}{\pgfqpoint{0.000000in}{0.000000in}}{%
\pgfpathmoveto{\pgfqpoint{0.000000in}{0.000000in}}%
\pgfpathlineto{\pgfqpoint{-0.027778in}{0.000000in}}%
\pgfusepath{stroke,fill}%
}%
\begin{pgfscope}%
\pgfsys@transformshift{0.752778in}{0.904000in}%
\pgfsys@useobject{currentmarker}{}%
\end{pgfscope}%
\end{pgfscope}%
\begin{pgfscope}%
\pgfsetbuttcap%
\pgfsetroundjoin%
\definecolor{currentfill}{rgb}{0.000000,0.000000,0.000000}%
\pgfsetfillcolor{currentfill}%
\pgfsetlinewidth{0.602250pt}%
\definecolor{currentstroke}{rgb}{0.000000,0.000000,0.000000}%
\pgfsetstrokecolor{currentstroke}%
\pgfsetdash{}{0pt}%
\pgfsys@defobject{currentmarker}{\pgfqpoint{0.000000in}{0.000000in}}{\pgfqpoint{0.027778in}{0.000000in}}{%
\pgfpathmoveto{\pgfqpoint{0.000000in}{0.000000in}}%
\pgfpathlineto{\pgfqpoint{0.027778in}{0.000000in}}%
\pgfusepath{stroke,fill}%
}%
\begin{pgfscope}%
\pgfsys@transformshift{4.801389in}{0.904000in}%
\pgfsys@useobject{currentmarker}{}%
\end{pgfscope}%
\end{pgfscope}%
\begin{pgfscope}%
\pgfpathrectangle{\pgfqpoint{0.752778in}{0.582778in}}{\pgfqpoint{4.048611in}{3.212222in}}%
\pgfusepath{clip}%
\pgfsetrectcap%
\pgfsetroundjoin%
\pgfsetlinewidth{0.803000pt}%
\definecolor{currentstroke}{rgb}{0.690196,0.690196,0.690196}%
\pgfsetstrokecolor{currentstroke}%
\pgfsetstrokeopacity{0.300000}%
\pgfsetdash{}{0pt}%
\pgfpathmoveto{\pgfqpoint{0.752778in}{0.947803in}}%
\pgfpathlineto{\pgfqpoint{4.801389in}{0.947803in}}%
\pgfusepath{stroke}%
\end{pgfscope}%
\begin{pgfscope}%
\pgfsetbuttcap%
\pgfsetroundjoin%
\definecolor{currentfill}{rgb}{0.000000,0.000000,0.000000}%
\pgfsetfillcolor{currentfill}%
\pgfsetlinewidth{0.602250pt}%
\definecolor{currentstroke}{rgb}{0.000000,0.000000,0.000000}%
\pgfsetstrokecolor{currentstroke}%
\pgfsetdash{}{0pt}%
\pgfsys@defobject{currentmarker}{\pgfqpoint{-0.027778in}{0.000000in}}{\pgfqpoint{0.000000in}{0.000000in}}{%
\pgfpathmoveto{\pgfqpoint{0.000000in}{0.000000in}}%
\pgfpathlineto{\pgfqpoint{-0.027778in}{0.000000in}}%
\pgfusepath{stroke,fill}%
}%
\begin{pgfscope}%
\pgfsys@transformshift{0.752778in}{0.947803in}%
\pgfsys@useobject{currentmarker}{}%
\end{pgfscope}%
\end{pgfscope}%
\begin{pgfscope}%
\pgfsetbuttcap%
\pgfsetroundjoin%
\definecolor{currentfill}{rgb}{0.000000,0.000000,0.000000}%
\pgfsetfillcolor{currentfill}%
\pgfsetlinewidth{0.602250pt}%
\definecolor{currentstroke}{rgb}{0.000000,0.000000,0.000000}%
\pgfsetstrokecolor{currentstroke}%
\pgfsetdash{}{0pt}%
\pgfsys@defobject{currentmarker}{\pgfqpoint{0.000000in}{0.000000in}}{\pgfqpoint{0.027778in}{0.000000in}}{%
\pgfpathmoveto{\pgfqpoint{0.000000in}{0.000000in}}%
\pgfpathlineto{\pgfqpoint{0.027778in}{0.000000in}}%
\pgfusepath{stroke,fill}%
}%
\begin{pgfscope}%
\pgfsys@transformshift{4.801389in}{0.947803in}%
\pgfsys@useobject{currentmarker}{}%
\end{pgfscope}%
\end{pgfscope}%
\begin{pgfscope}%
\pgfpathrectangle{\pgfqpoint{0.752778in}{0.582778in}}{\pgfqpoint{4.048611in}{3.212222in}}%
\pgfusepath{clip}%
\pgfsetrectcap%
\pgfsetroundjoin%
\pgfsetlinewidth{0.803000pt}%
\definecolor{currentstroke}{rgb}{0.690196,0.690196,0.690196}%
\pgfsetstrokecolor{currentstroke}%
\pgfsetstrokeopacity{0.300000}%
\pgfsetdash{}{0pt}%
\pgfpathmoveto{\pgfqpoint{0.752778in}{0.991606in}}%
\pgfpathlineto{\pgfqpoint{4.801389in}{0.991606in}}%
\pgfusepath{stroke}%
\end{pgfscope}%
\begin{pgfscope}%
\pgfsetbuttcap%
\pgfsetroundjoin%
\definecolor{currentfill}{rgb}{0.000000,0.000000,0.000000}%
\pgfsetfillcolor{currentfill}%
\pgfsetlinewidth{0.602250pt}%
\definecolor{currentstroke}{rgb}{0.000000,0.000000,0.000000}%
\pgfsetstrokecolor{currentstroke}%
\pgfsetdash{}{0pt}%
\pgfsys@defobject{currentmarker}{\pgfqpoint{-0.027778in}{0.000000in}}{\pgfqpoint{0.000000in}{0.000000in}}{%
\pgfpathmoveto{\pgfqpoint{0.000000in}{0.000000in}}%
\pgfpathlineto{\pgfqpoint{-0.027778in}{0.000000in}}%
\pgfusepath{stroke,fill}%
}%
\begin{pgfscope}%
\pgfsys@transformshift{0.752778in}{0.991606in}%
\pgfsys@useobject{currentmarker}{}%
\end{pgfscope}%
\end{pgfscope}%
\begin{pgfscope}%
\pgfsetbuttcap%
\pgfsetroundjoin%
\definecolor{currentfill}{rgb}{0.000000,0.000000,0.000000}%
\pgfsetfillcolor{currentfill}%
\pgfsetlinewidth{0.602250pt}%
\definecolor{currentstroke}{rgb}{0.000000,0.000000,0.000000}%
\pgfsetstrokecolor{currentstroke}%
\pgfsetdash{}{0pt}%
\pgfsys@defobject{currentmarker}{\pgfqpoint{0.000000in}{0.000000in}}{\pgfqpoint{0.027778in}{0.000000in}}{%
\pgfpathmoveto{\pgfqpoint{0.000000in}{0.000000in}}%
\pgfpathlineto{\pgfqpoint{0.027778in}{0.000000in}}%
\pgfusepath{stroke,fill}%
}%
\begin{pgfscope}%
\pgfsys@transformshift{4.801389in}{0.991606in}%
\pgfsys@useobject{currentmarker}{}%
\end{pgfscope}%
\end{pgfscope}%
\begin{pgfscope}%
\pgfpathrectangle{\pgfqpoint{0.752778in}{0.582778in}}{\pgfqpoint{4.048611in}{3.212222in}}%
\pgfusepath{clip}%
\pgfsetrectcap%
\pgfsetroundjoin%
\pgfsetlinewidth{0.803000pt}%
\definecolor{currentstroke}{rgb}{0.690196,0.690196,0.690196}%
\pgfsetstrokecolor{currentstroke}%
\pgfsetstrokeopacity{0.300000}%
\pgfsetdash{}{0pt}%
\pgfpathmoveto{\pgfqpoint{0.752778in}{1.035409in}}%
\pgfpathlineto{\pgfqpoint{4.801389in}{1.035409in}}%
\pgfusepath{stroke}%
\end{pgfscope}%
\begin{pgfscope}%
\pgfsetbuttcap%
\pgfsetroundjoin%
\definecolor{currentfill}{rgb}{0.000000,0.000000,0.000000}%
\pgfsetfillcolor{currentfill}%
\pgfsetlinewidth{0.602250pt}%
\definecolor{currentstroke}{rgb}{0.000000,0.000000,0.000000}%
\pgfsetstrokecolor{currentstroke}%
\pgfsetdash{}{0pt}%
\pgfsys@defobject{currentmarker}{\pgfqpoint{-0.027778in}{0.000000in}}{\pgfqpoint{0.000000in}{0.000000in}}{%
\pgfpathmoveto{\pgfqpoint{0.000000in}{0.000000in}}%
\pgfpathlineto{\pgfqpoint{-0.027778in}{0.000000in}}%
\pgfusepath{stroke,fill}%
}%
\begin{pgfscope}%
\pgfsys@transformshift{0.752778in}{1.035409in}%
\pgfsys@useobject{currentmarker}{}%
\end{pgfscope}%
\end{pgfscope}%
\begin{pgfscope}%
\pgfsetbuttcap%
\pgfsetroundjoin%
\definecolor{currentfill}{rgb}{0.000000,0.000000,0.000000}%
\pgfsetfillcolor{currentfill}%
\pgfsetlinewidth{0.602250pt}%
\definecolor{currentstroke}{rgb}{0.000000,0.000000,0.000000}%
\pgfsetstrokecolor{currentstroke}%
\pgfsetdash{}{0pt}%
\pgfsys@defobject{currentmarker}{\pgfqpoint{0.000000in}{0.000000in}}{\pgfqpoint{0.027778in}{0.000000in}}{%
\pgfpathmoveto{\pgfqpoint{0.000000in}{0.000000in}}%
\pgfpathlineto{\pgfqpoint{0.027778in}{0.000000in}}%
\pgfusepath{stroke,fill}%
}%
\begin{pgfscope}%
\pgfsys@transformshift{4.801389in}{1.035409in}%
\pgfsys@useobject{currentmarker}{}%
\end{pgfscope}%
\end{pgfscope}%
\begin{pgfscope}%
\pgfpathrectangle{\pgfqpoint{0.752778in}{0.582778in}}{\pgfqpoint{4.048611in}{3.212222in}}%
\pgfusepath{clip}%
\pgfsetrectcap%
\pgfsetroundjoin%
\pgfsetlinewidth{0.803000pt}%
\definecolor{currentstroke}{rgb}{0.690196,0.690196,0.690196}%
\pgfsetstrokecolor{currentstroke}%
\pgfsetstrokeopacity{0.300000}%
\pgfsetdash{}{0pt}%
\pgfpathmoveto{\pgfqpoint{0.752778in}{1.079212in}}%
\pgfpathlineto{\pgfqpoint{4.801389in}{1.079212in}}%
\pgfusepath{stroke}%
\end{pgfscope}%
\begin{pgfscope}%
\pgfsetbuttcap%
\pgfsetroundjoin%
\definecolor{currentfill}{rgb}{0.000000,0.000000,0.000000}%
\pgfsetfillcolor{currentfill}%
\pgfsetlinewidth{0.602250pt}%
\definecolor{currentstroke}{rgb}{0.000000,0.000000,0.000000}%
\pgfsetstrokecolor{currentstroke}%
\pgfsetdash{}{0pt}%
\pgfsys@defobject{currentmarker}{\pgfqpoint{-0.027778in}{0.000000in}}{\pgfqpoint{0.000000in}{0.000000in}}{%
\pgfpathmoveto{\pgfqpoint{0.000000in}{0.000000in}}%
\pgfpathlineto{\pgfqpoint{-0.027778in}{0.000000in}}%
\pgfusepath{stroke,fill}%
}%
\begin{pgfscope}%
\pgfsys@transformshift{0.752778in}{1.079212in}%
\pgfsys@useobject{currentmarker}{}%
\end{pgfscope}%
\end{pgfscope}%
\begin{pgfscope}%
\pgfsetbuttcap%
\pgfsetroundjoin%
\definecolor{currentfill}{rgb}{0.000000,0.000000,0.000000}%
\pgfsetfillcolor{currentfill}%
\pgfsetlinewidth{0.602250pt}%
\definecolor{currentstroke}{rgb}{0.000000,0.000000,0.000000}%
\pgfsetstrokecolor{currentstroke}%
\pgfsetdash{}{0pt}%
\pgfsys@defobject{currentmarker}{\pgfqpoint{0.000000in}{0.000000in}}{\pgfqpoint{0.027778in}{0.000000in}}{%
\pgfpathmoveto{\pgfqpoint{0.000000in}{0.000000in}}%
\pgfpathlineto{\pgfqpoint{0.027778in}{0.000000in}}%
\pgfusepath{stroke,fill}%
}%
\begin{pgfscope}%
\pgfsys@transformshift{4.801389in}{1.079212in}%
\pgfsys@useobject{currentmarker}{}%
\end{pgfscope}%
\end{pgfscope}%
\begin{pgfscope}%
\pgfpathrectangle{\pgfqpoint{0.752778in}{0.582778in}}{\pgfqpoint{4.048611in}{3.212222in}}%
\pgfusepath{clip}%
\pgfsetrectcap%
\pgfsetroundjoin%
\pgfsetlinewidth{0.803000pt}%
\definecolor{currentstroke}{rgb}{0.690196,0.690196,0.690196}%
\pgfsetstrokecolor{currentstroke}%
\pgfsetstrokeopacity{0.300000}%
\pgfsetdash{}{0pt}%
\pgfpathmoveto{\pgfqpoint{0.752778in}{1.123015in}}%
\pgfpathlineto{\pgfqpoint{4.801389in}{1.123015in}}%
\pgfusepath{stroke}%
\end{pgfscope}%
\begin{pgfscope}%
\pgfsetbuttcap%
\pgfsetroundjoin%
\definecolor{currentfill}{rgb}{0.000000,0.000000,0.000000}%
\pgfsetfillcolor{currentfill}%
\pgfsetlinewidth{0.602250pt}%
\definecolor{currentstroke}{rgb}{0.000000,0.000000,0.000000}%
\pgfsetstrokecolor{currentstroke}%
\pgfsetdash{}{0pt}%
\pgfsys@defobject{currentmarker}{\pgfqpoint{-0.027778in}{0.000000in}}{\pgfqpoint{0.000000in}{0.000000in}}{%
\pgfpathmoveto{\pgfqpoint{0.000000in}{0.000000in}}%
\pgfpathlineto{\pgfqpoint{-0.027778in}{0.000000in}}%
\pgfusepath{stroke,fill}%
}%
\begin{pgfscope}%
\pgfsys@transformshift{0.752778in}{1.123015in}%
\pgfsys@useobject{currentmarker}{}%
\end{pgfscope}%
\end{pgfscope}%
\begin{pgfscope}%
\pgfsetbuttcap%
\pgfsetroundjoin%
\definecolor{currentfill}{rgb}{0.000000,0.000000,0.000000}%
\pgfsetfillcolor{currentfill}%
\pgfsetlinewidth{0.602250pt}%
\definecolor{currentstroke}{rgb}{0.000000,0.000000,0.000000}%
\pgfsetstrokecolor{currentstroke}%
\pgfsetdash{}{0pt}%
\pgfsys@defobject{currentmarker}{\pgfqpoint{0.000000in}{0.000000in}}{\pgfqpoint{0.027778in}{0.000000in}}{%
\pgfpathmoveto{\pgfqpoint{0.000000in}{0.000000in}}%
\pgfpathlineto{\pgfqpoint{0.027778in}{0.000000in}}%
\pgfusepath{stroke,fill}%
}%
\begin{pgfscope}%
\pgfsys@transformshift{4.801389in}{1.123015in}%
\pgfsys@useobject{currentmarker}{}%
\end{pgfscope}%
\end{pgfscope}%
\begin{pgfscope}%
\pgfpathrectangle{\pgfqpoint{0.752778in}{0.582778in}}{\pgfqpoint{4.048611in}{3.212222in}}%
\pgfusepath{clip}%
\pgfsetrectcap%
\pgfsetroundjoin%
\pgfsetlinewidth{0.803000pt}%
\definecolor{currentstroke}{rgb}{0.690196,0.690196,0.690196}%
\pgfsetstrokecolor{currentstroke}%
\pgfsetstrokeopacity{0.300000}%
\pgfsetdash{}{0pt}%
\pgfpathmoveto{\pgfqpoint{0.752778in}{1.210621in}}%
\pgfpathlineto{\pgfqpoint{4.801389in}{1.210621in}}%
\pgfusepath{stroke}%
\end{pgfscope}%
\begin{pgfscope}%
\pgfsetbuttcap%
\pgfsetroundjoin%
\definecolor{currentfill}{rgb}{0.000000,0.000000,0.000000}%
\pgfsetfillcolor{currentfill}%
\pgfsetlinewidth{0.602250pt}%
\definecolor{currentstroke}{rgb}{0.000000,0.000000,0.000000}%
\pgfsetstrokecolor{currentstroke}%
\pgfsetdash{}{0pt}%
\pgfsys@defobject{currentmarker}{\pgfqpoint{-0.027778in}{0.000000in}}{\pgfqpoint{0.000000in}{0.000000in}}{%
\pgfpathmoveto{\pgfqpoint{0.000000in}{0.000000in}}%
\pgfpathlineto{\pgfqpoint{-0.027778in}{0.000000in}}%
\pgfusepath{stroke,fill}%
}%
\begin{pgfscope}%
\pgfsys@transformshift{0.752778in}{1.210621in}%
\pgfsys@useobject{currentmarker}{}%
\end{pgfscope}%
\end{pgfscope}%
\begin{pgfscope}%
\pgfsetbuttcap%
\pgfsetroundjoin%
\definecolor{currentfill}{rgb}{0.000000,0.000000,0.000000}%
\pgfsetfillcolor{currentfill}%
\pgfsetlinewidth{0.602250pt}%
\definecolor{currentstroke}{rgb}{0.000000,0.000000,0.000000}%
\pgfsetstrokecolor{currentstroke}%
\pgfsetdash{}{0pt}%
\pgfsys@defobject{currentmarker}{\pgfqpoint{0.000000in}{0.000000in}}{\pgfqpoint{0.027778in}{0.000000in}}{%
\pgfpathmoveto{\pgfqpoint{0.000000in}{0.000000in}}%
\pgfpathlineto{\pgfqpoint{0.027778in}{0.000000in}}%
\pgfusepath{stroke,fill}%
}%
\begin{pgfscope}%
\pgfsys@transformshift{4.801389in}{1.210621in}%
\pgfsys@useobject{currentmarker}{}%
\end{pgfscope}%
\end{pgfscope}%
\begin{pgfscope}%
\pgfpathrectangle{\pgfqpoint{0.752778in}{0.582778in}}{\pgfqpoint{4.048611in}{3.212222in}}%
\pgfusepath{clip}%
\pgfsetrectcap%
\pgfsetroundjoin%
\pgfsetlinewidth{0.803000pt}%
\definecolor{currentstroke}{rgb}{0.690196,0.690196,0.690196}%
\pgfsetstrokecolor{currentstroke}%
\pgfsetstrokeopacity{0.300000}%
\pgfsetdash{}{0pt}%
\pgfpathmoveto{\pgfqpoint{0.752778in}{1.254424in}}%
\pgfpathlineto{\pgfqpoint{4.801389in}{1.254424in}}%
\pgfusepath{stroke}%
\end{pgfscope}%
\begin{pgfscope}%
\pgfsetbuttcap%
\pgfsetroundjoin%
\definecolor{currentfill}{rgb}{0.000000,0.000000,0.000000}%
\pgfsetfillcolor{currentfill}%
\pgfsetlinewidth{0.602250pt}%
\definecolor{currentstroke}{rgb}{0.000000,0.000000,0.000000}%
\pgfsetstrokecolor{currentstroke}%
\pgfsetdash{}{0pt}%
\pgfsys@defobject{currentmarker}{\pgfqpoint{-0.027778in}{0.000000in}}{\pgfqpoint{0.000000in}{0.000000in}}{%
\pgfpathmoveto{\pgfqpoint{0.000000in}{0.000000in}}%
\pgfpathlineto{\pgfqpoint{-0.027778in}{0.000000in}}%
\pgfusepath{stroke,fill}%
}%
\begin{pgfscope}%
\pgfsys@transformshift{0.752778in}{1.254424in}%
\pgfsys@useobject{currentmarker}{}%
\end{pgfscope}%
\end{pgfscope}%
\begin{pgfscope}%
\pgfsetbuttcap%
\pgfsetroundjoin%
\definecolor{currentfill}{rgb}{0.000000,0.000000,0.000000}%
\pgfsetfillcolor{currentfill}%
\pgfsetlinewidth{0.602250pt}%
\definecolor{currentstroke}{rgb}{0.000000,0.000000,0.000000}%
\pgfsetstrokecolor{currentstroke}%
\pgfsetdash{}{0pt}%
\pgfsys@defobject{currentmarker}{\pgfqpoint{0.000000in}{0.000000in}}{\pgfqpoint{0.027778in}{0.000000in}}{%
\pgfpathmoveto{\pgfqpoint{0.000000in}{0.000000in}}%
\pgfpathlineto{\pgfqpoint{0.027778in}{0.000000in}}%
\pgfusepath{stroke,fill}%
}%
\begin{pgfscope}%
\pgfsys@transformshift{4.801389in}{1.254424in}%
\pgfsys@useobject{currentmarker}{}%
\end{pgfscope}%
\end{pgfscope}%
\begin{pgfscope}%
\pgfpathrectangle{\pgfqpoint{0.752778in}{0.582778in}}{\pgfqpoint{4.048611in}{3.212222in}}%
\pgfusepath{clip}%
\pgfsetrectcap%
\pgfsetroundjoin%
\pgfsetlinewidth{0.803000pt}%
\definecolor{currentstroke}{rgb}{0.690196,0.690196,0.690196}%
\pgfsetstrokecolor{currentstroke}%
\pgfsetstrokeopacity{0.300000}%
\pgfsetdash{}{0pt}%
\pgfpathmoveto{\pgfqpoint{0.752778in}{1.298227in}}%
\pgfpathlineto{\pgfqpoint{4.801389in}{1.298227in}}%
\pgfusepath{stroke}%
\end{pgfscope}%
\begin{pgfscope}%
\pgfsetbuttcap%
\pgfsetroundjoin%
\definecolor{currentfill}{rgb}{0.000000,0.000000,0.000000}%
\pgfsetfillcolor{currentfill}%
\pgfsetlinewidth{0.602250pt}%
\definecolor{currentstroke}{rgb}{0.000000,0.000000,0.000000}%
\pgfsetstrokecolor{currentstroke}%
\pgfsetdash{}{0pt}%
\pgfsys@defobject{currentmarker}{\pgfqpoint{-0.027778in}{0.000000in}}{\pgfqpoint{0.000000in}{0.000000in}}{%
\pgfpathmoveto{\pgfqpoint{0.000000in}{0.000000in}}%
\pgfpathlineto{\pgfqpoint{-0.027778in}{0.000000in}}%
\pgfusepath{stroke,fill}%
}%
\begin{pgfscope}%
\pgfsys@transformshift{0.752778in}{1.298227in}%
\pgfsys@useobject{currentmarker}{}%
\end{pgfscope}%
\end{pgfscope}%
\begin{pgfscope}%
\pgfsetbuttcap%
\pgfsetroundjoin%
\definecolor{currentfill}{rgb}{0.000000,0.000000,0.000000}%
\pgfsetfillcolor{currentfill}%
\pgfsetlinewidth{0.602250pt}%
\definecolor{currentstroke}{rgb}{0.000000,0.000000,0.000000}%
\pgfsetstrokecolor{currentstroke}%
\pgfsetdash{}{0pt}%
\pgfsys@defobject{currentmarker}{\pgfqpoint{0.000000in}{0.000000in}}{\pgfqpoint{0.027778in}{0.000000in}}{%
\pgfpathmoveto{\pgfqpoint{0.000000in}{0.000000in}}%
\pgfpathlineto{\pgfqpoint{0.027778in}{0.000000in}}%
\pgfusepath{stroke,fill}%
}%
\begin{pgfscope}%
\pgfsys@transformshift{4.801389in}{1.298227in}%
\pgfsys@useobject{currentmarker}{}%
\end{pgfscope}%
\end{pgfscope}%
\begin{pgfscope}%
\pgfpathrectangle{\pgfqpoint{0.752778in}{0.582778in}}{\pgfqpoint{4.048611in}{3.212222in}}%
\pgfusepath{clip}%
\pgfsetrectcap%
\pgfsetroundjoin%
\pgfsetlinewidth{0.803000pt}%
\definecolor{currentstroke}{rgb}{0.690196,0.690196,0.690196}%
\pgfsetstrokecolor{currentstroke}%
\pgfsetstrokeopacity{0.300000}%
\pgfsetdash{}{0pt}%
\pgfpathmoveto{\pgfqpoint{0.752778in}{1.342030in}}%
\pgfpathlineto{\pgfqpoint{4.801389in}{1.342030in}}%
\pgfusepath{stroke}%
\end{pgfscope}%
\begin{pgfscope}%
\pgfsetbuttcap%
\pgfsetroundjoin%
\definecolor{currentfill}{rgb}{0.000000,0.000000,0.000000}%
\pgfsetfillcolor{currentfill}%
\pgfsetlinewidth{0.602250pt}%
\definecolor{currentstroke}{rgb}{0.000000,0.000000,0.000000}%
\pgfsetstrokecolor{currentstroke}%
\pgfsetdash{}{0pt}%
\pgfsys@defobject{currentmarker}{\pgfqpoint{-0.027778in}{0.000000in}}{\pgfqpoint{0.000000in}{0.000000in}}{%
\pgfpathmoveto{\pgfqpoint{0.000000in}{0.000000in}}%
\pgfpathlineto{\pgfqpoint{-0.027778in}{0.000000in}}%
\pgfusepath{stroke,fill}%
}%
\begin{pgfscope}%
\pgfsys@transformshift{0.752778in}{1.342030in}%
\pgfsys@useobject{currentmarker}{}%
\end{pgfscope}%
\end{pgfscope}%
\begin{pgfscope}%
\pgfsetbuttcap%
\pgfsetroundjoin%
\definecolor{currentfill}{rgb}{0.000000,0.000000,0.000000}%
\pgfsetfillcolor{currentfill}%
\pgfsetlinewidth{0.602250pt}%
\definecolor{currentstroke}{rgb}{0.000000,0.000000,0.000000}%
\pgfsetstrokecolor{currentstroke}%
\pgfsetdash{}{0pt}%
\pgfsys@defobject{currentmarker}{\pgfqpoint{0.000000in}{0.000000in}}{\pgfqpoint{0.027778in}{0.000000in}}{%
\pgfpathmoveto{\pgfqpoint{0.000000in}{0.000000in}}%
\pgfpathlineto{\pgfqpoint{0.027778in}{0.000000in}}%
\pgfusepath{stroke,fill}%
}%
\begin{pgfscope}%
\pgfsys@transformshift{4.801389in}{1.342030in}%
\pgfsys@useobject{currentmarker}{}%
\end{pgfscope}%
\end{pgfscope}%
\begin{pgfscope}%
\pgfpathrectangle{\pgfqpoint{0.752778in}{0.582778in}}{\pgfqpoint{4.048611in}{3.212222in}}%
\pgfusepath{clip}%
\pgfsetrectcap%
\pgfsetroundjoin%
\pgfsetlinewidth{0.803000pt}%
\definecolor{currentstroke}{rgb}{0.690196,0.690196,0.690196}%
\pgfsetstrokecolor{currentstroke}%
\pgfsetstrokeopacity{0.300000}%
\pgfsetdash{}{0pt}%
\pgfpathmoveto{\pgfqpoint{0.752778in}{1.385833in}}%
\pgfpathlineto{\pgfqpoint{4.801389in}{1.385833in}}%
\pgfusepath{stroke}%
\end{pgfscope}%
\begin{pgfscope}%
\pgfsetbuttcap%
\pgfsetroundjoin%
\definecolor{currentfill}{rgb}{0.000000,0.000000,0.000000}%
\pgfsetfillcolor{currentfill}%
\pgfsetlinewidth{0.602250pt}%
\definecolor{currentstroke}{rgb}{0.000000,0.000000,0.000000}%
\pgfsetstrokecolor{currentstroke}%
\pgfsetdash{}{0pt}%
\pgfsys@defobject{currentmarker}{\pgfqpoint{-0.027778in}{0.000000in}}{\pgfqpoint{0.000000in}{0.000000in}}{%
\pgfpathmoveto{\pgfqpoint{0.000000in}{0.000000in}}%
\pgfpathlineto{\pgfqpoint{-0.027778in}{0.000000in}}%
\pgfusepath{stroke,fill}%
}%
\begin{pgfscope}%
\pgfsys@transformshift{0.752778in}{1.385833in}%
\pgfsys@useobject{currentmarker}{}%
\end{pgfscope}%
\end{pgfscope}%
\begin{pgfscope}%
\pgfsetbuttcap%
\pgfsetroundjoin%
\definecolor{currentfill}{rgb}{0.000000,0.000000,0.000000}%
\pgfsetfillcolor{currentfill}%
\pgfsetlinewidth{0.602250pt}%
\definecolor{currentstroke}{rgb}{0.000000,0.000000,0.000000}%
\pgfsetstrokecolor{currentstroke}%
\pgfsetdash{}{0pt}%
\pgfsys@defobject{currentmarker}{\pgfqpoint{0.000000in}{0.000000in}}{\pgfqpoint{0.027778in}{0.000000in}}{%
\pgfpathmoveto{\pgfqpoint{0.000000in}{0.000000in}}%
\pgfpathlineto{\pgfqpoint{0.027778in}{0.000000in}}%
\pgfusepath{stroke,fill}%
}%
\begin{pgfscope}%
\pgfsys@transformshift{4.801389in}{1.385833in}%
\pgfsys@useobject{currentmarker}{}%
\end{pgfscope}%
\end{pgfscope}%
\begin{pgfscope}%
\pgfpathrectangle{\pgfqpoint{0.752778in}{0.582778in}}{\pgfqpoint{4.048611in}{3.212222in}}%
\pgfusepath{clip}%
\pgfsetrectcap%
\pgfsetroundjoin%
\pgfsetlinewidth{0.803000pt}%
\definecolor{currentstroke}{rgb}{0.690196,0.690196,0.690196}%
\pgfsetstrokecolor{currentstroke}%
\pgfsetstrokeopacity{0.300000}%
\pgfsetdash{}{0pt}%
\pgfpathmoveto{\pgfqpoint{0.752778in}{1.429636in}}%
\pgfpathlineto{\pgfqpoint{4.801389in}{1.429636in}}%
\pgfusepath{stroke}%
\end{pgfscope}%
\begin{pgfscope}%
\pgfsetbuttcap%
\pgfsetroundjoin%
\definecolor{currentfill}{rgb}{0.000000,0.000000,0.000000}%
\pgfsetfillcolor{currentfill}%
\pgfsetlinewidth{0.602250pt}%
\definecolor{currentstroke}{rgb}{0.000000,0.000000,0.000000}%
\pgfsetstrokecolor{currentstroke}%
\pgfsetdash{}{0pt}%
\pgfsys@defobject{currentmarker}{\pgfqpoint{-0.027778in}{0.000000in}}{\pgfqpoint{0.000000in}{0.000000in}}{%
\pgfpathmoveto{\pgfqpoint{0.000000in}{0.000000in}}%
\pgfpathlineto{\pgfqpoint{-0.027778in}{0.000000in}}%
\pgfusepath{stroke,fill}%
}%
\begin{pgfscope}%
\pgfsys@transformshift{0.752778in}{1.429636in}%
\pgfsys@useobject{currentmarker}{}%
\end{pgfscope}%
\end{pgfscope}%
\begin{pgfscope}%
\pgfsetbuttcap%
\pgfsetroundjoin%
\definecolor{currentfill}{rgb}{0.000000,0.000000,0.000000}%
\pgfsetfillcolor{currentfill}%
\pgfsetlinewidth{0.602250pt}%
\definecolor{currentstroke}{rgb}{0.000000,0.000000,0.000000}%
\pgfsetstrokecolor{currentstroke}%
\pgfsetdash{}{0pt}%
\pgfsys@defobject{currentmarker}{\pgfqpoint{0.000000in}{0.000000in}}{\pgfqpoint{0.027778in}{0.000000in}}{%
\pgfpathmoveto{\pgfqpoint{0.000000in}{0.000000in}}%
\pgfpathlineto{\pgfqpoint{0.027778in}{0.000000in}}%
\pgfusepath{stroke,fill}%
}%
\begin{pgfscope}%
\pgfsys@transformshift{4.801389in}{1.429636in}%
\pgfsys@useobject{currentmarker}{}%
\end{pgfscope}%
\end{pgfscope}%
\begin{pgfscope}%
\pgfpathrectangle{\pgfqpoint{0.752778in}{0.582778in}}{\pgfqpoint{4.048611in}{3.212222in}}%
\pgfusepath{clip}%
\pgfsetrectcap%
\pgfsetroundjoin%
\pgfsetlinewidth{0.803000pt}%
\definecolor{currentstroke}{rgb}{0.690196,0.690196,0.690196}%
\pgfsetstrokecolor{currentstroke}%
\pgfsetstrokeopacity{0.300000}%
\pgfsetdash{}{0pt}%
\pgfpathmoveto{\pgfqpoint{0.752778in}{1.473439in}}%
\pgfpathlineto{\pgfqpoint{4.801389in}{1.473439in}}%
\pgfusepath{stroke}%
\end{pgfscope}%
\begin{pgfscope}%
\pgfsetbuttcap%
\pgfsetroundjoin%
\definecolor{currentfill}{rgb}{0.000000,0.000000,0.000000}%
\pgfsetfillcolor{currentfill}%
\pgfsetlinewidth{0.602250pt}%
\definecolor{currentstroke}{rgb}{0.000000,0.000000,0.000000}%
\pgfsetstrokecolor{currentstroke}%
\pgfsetdash{}{0pt}%
\pgfsys@defobject{currentmarker}{\pgfqpoint{-0.027778in}{0.000000in}}{\pgfqpoint{0.000000in}{0.000000in}}{%
\pgfpathmoveto{\pgfqpoint{0.000000in}{0.000000in}}%
\pgfpathlineto{\pgfqpoint{-0.027778in}{0.000000in}}%
\pgfusepath{stroke,fill}%
}%
\begin{pgfscope}%
\pgfsys@transformshift{0.752778in}{1.473439in}%
\pgfsys@useobject{currentmarker}{}%
\end{pgfscope}%
\end{pgfscope}%
\begin{pgfscope}%
\pgfsetbuttcap%
\pgfsetroundjoin%
\definecolor{currentfill}{rgb}{0.000000,0.000000,0.000000}%
\pgfsetfillcolor{currentfill}%
\pgfsetlinewidth{0.602250pt}%
\definecolor{currentstroke}{rgb}{0.000000,0.000000,0.000000}%
\pgfsetstrokecolor{currentstroke}%
\pgfsetdash{}{0pt}%
\pgfsys@defobject{currentmarker}{\pgfqpoint{0.000000in}{0.000000in}}{\pgfqpoint{0.027778in}{0.000000in}}{%
\pgfpathmoveto{\pgfqpoint{0.000000in}{0.000000in}}%
\pgfpathlineto{\pgfqpoint{0.027778in}{0.000000in}}%
\pgfusepath{stroke,fill}%
}%
\begin{pgfscope}%
\pgfsys@transformshift{4.801389in}{1.473439in}%
\pgfsys@useobject{currentmarker}{}%
\end{pgfscope}%
\end{pgfscope}%
\begin{pgfscope}%
\pgfpathrectangle{\pgfqpoint{0.752778in}{0.582778in}}{\pgfqpoint{4.048611in}{3.212222in}}%
\pgfusepath{clip}%
\pgfsetrectcap%
\pgfsetroundjoin%
\pgfsetlinewidth{0.803000pt}%
\definecolor{currentstroke}{rgb}{0.690196,0.690196,0.690196}%
\pgfsetstrokecolor{currentstroke}%
\pgfsetstrokeopacity{0.300000}%
\pgfsetdash{}{0pt}%
\pgfpathmoveto{\pgfqpoint{0.752778in}{1.517242in}}%
\pgfpathlineto{\pgfqpoint{4.801389in}{1.517242in}}%
\pgfusepath{stroke}%
\end{pgfscope}%
\begin{pgfscope}%
\pgfsetbuttcap%
\pgfsetroundjoin%
\definecolor{currentfill}{rgb}{0.000000,0.000000,0.000000}%
\pgfsetfillcolor{currentfill}%
\pgfsetlinewidth{0.602250pt}%
\definecolor{currentstroke}{rgb}{0.000000,0.000000,0.000000}%
\pgfsetstrokecolor{currentstroke}%
\pgfsetdash{}{0pt}%
\pgfsys@defobject{currentmarker}{\pgfqpoint{-0.027778in}{0.000000in}}{\pgfqpoint{0.000000in}{0.000000in}}{%
\pgfpathmoveto{\pgfqpoint{0.000000in}{0.000000in}}%
\pgfpathlineto{\pgfqpoint{-0.027778in}{0.000000in}}%
\pgfusepath{stroke,fill}%
}%
\begin{pgfscope}%
\pgfsys@transformshift{0.752778in}{1.517242in}%
\pgfsys@useobject{currentmarker}{}%
\end{pgfscope}%
\end{pgfscope}%
\begin{pgfscope}%
\pgfsetbuttcap%
\pgfsetroundjoin%
\definecolor{currentfill}{rgb}{0.000000,0.000000,0.000000}%
\pgfsetfillcolor{currentfill}%
\pgfsetlinewidth{0.602250pt}%
\definecolor{currentstroke}{rgb}{0.000000,0.000000,0.000000}%
\pgfsetstrokecolor{currentstroke}%
\pgfsetdash{}{0pt}%
\pgfsys@defobject{currentmarker}{\pgfqpoint{0.000000in}{0.000000in}}{\pgfqpoint{0.027778in}{0.000000in}}{%
\pgfpathmoveto{\pgfqpoint{0.000000in}{0.000000in}}%
\pgfpathlineto{\pgfqpoint{0.027778in}{0.000000in}}%
\pgfusepath{stroke,fill}%
}%
\begin{pgfscope}%
\pgfsys@transformshift{4.801389in}{1.517242in}%
\pgfsys@useobject{currentmarker}{}%
\end{pgfscope}%
\end{pgfscope}%
\begin{pgfscope}%
\pgfpathrectangle{\pgfqpoint{0.752778in}{0.582778in}}{\pgfqpoint{4.048611in}{3.212222in}}%
\pgfusepath{clip}%
\pgfsetrectcap%
\pgfsetroundjoin%
\pgfsetlinewidth{0.803000pt}%
\definecolor{currentstroke}{rgb}{0.690196,0.690196,0.690196}%
\pgfsetstrokecolor{currentstroke}%
\pgfsetstrokeopacity{0.300000}%
\pgfsetdash{}{0pt}%
\pgfpathmoveto{\pgfqpoint{0.752778in}{1.561045in}}%
\pgfpathlineto{\pgfqpoint{4.801389in}{1.561045in}}%
\pgfusepath{stroke}%
\end{pgfscope}%
\begin{pgfscope}%
\pgfsetbuttcap%
\pgfsetroundjoin%
\definecolor{currentfill}{rgb}{0.000000,0.000000,0.000000}%
\pgfsetfillcolor{currentfill}%
\pgfsetlinewidth{0.602250pt}%
\definecolor{currentstroke}{rgb}{0.000000,0.000000,0.000000}%
\pgfsetstrokecolor{currentstroke}%
\pgfsetdash{}{0pt}%
\pgfsys@defobject{currentmarker}{\pgfqpoint{-0.027778in}{0.000000in}}{\pgfqpoint{0.000000in}{0.000000in}}{%
\pgfpathmoveto{\pgfqpoint{0.000000in}{0.000000in}}%
\pgfpathlineto{\pgfqpoint{-0.027778in}{0.000000in}}%
\pgfusepath{stroke,fill}%
}%
\begin{pgfscope}%
\pgfsys@transformshift{0.752778in}{1.561045in}%
\pgfsys@useobject{currentmarker}{}%
\end{pgfscope}%
\end{pgfscope}%
\begin{pgfscope}%
\pgfsetbuttcap%
\pgfsetroundjoin%
\definecolor{currentfill}{rgb}{0.000000,0.000000,0.000000}%
\pgfsetfillcolor{currentfill}%
\pgfsetlinewidth{0.602250pt}%
\definecolor{currentstroke}{rgb}{0.000000,0.000000,0.000000}%
\pgfsetstrokecolor{currentstroke}%
\pgfsetdash{}{0pt}%
\pgfsys@defobject{currentmarker}{\pgfqpoint{0.000000in}{0.000000in}}{\pgfqpoint{0.027778in}{0.000000in}}{%
\pgfpathmoveto{\pgfqpoint{0.000000in}{0.000000in}}%
\pgfpathlineto{\pgfqpoint{0.027778in}{0.000000in}}%
\pgfusepath{stroke,fill}%
}%
\begin{pgfscope}%
\pgfsys@transformshift{4.801389in}{1.561045in}%
\pgfsys@useobject{currentmarker}{}%
\end{pgfscope}%
\end{pgfscope}%
\begin{pgfscope}%
\pgfpathrectangle{\pgfqpoint{0.752778in}{0.582778in}}{\pgfqpoint{4.048611in}{3.212222in}}%
\pgfusepath{clip}%
\pgfsetrectcap%
\pgfsetroundjoin%
\pgfsetlinewidth{0.803000pt}%
\definecolor{currentstroke}{rgb}{0.690196,0.690196,0.690196}%
\pgfsetstrokecolor{currentstroke}%
\pgfsetstrokeopacity{0.300000}%
\pgfsetdash{}{0pt}%
\pgfpathmoveto{\pgfqpoint{0.752778in}{1.648652in}}%
\pgfpathlineto{\pgfqpoint{4.801389in}{1.648652in}}%
\pgfusepath{stroke}%
\end{pgfscope}%
\begin{pgfscope}%
\pgfsetbuttcap%
\pgfsetroundjoin%
\definecolor{currentfill}{rgb}{0.000000,0.000000,0.000000}%
\pgfsetfillcolor{currentfill}%
\pgfsetlinewidth{0.602250pt}%
\definecolor{currentstroke}{rgb}{0.000000,0.000000,0.000000}%
\pgfsetstrokecolor{currentstroke}%
\pgfsetdash{}{0pt}%
\pgfsys@defobject{currentmarker}{\pgfqpoint{-0.027778in}{0.000000in}}{\pgfqpoint{0.000000in}{0.000000in}}{%
\pgfpathmoveto{\pgfqpoint{0.000000in}{0.000000in}}%
\pgfpathlineto{\pgfqpoint{-0.027778in}{0.000000in}}%
\pgfusepath{stroke,fill}%
}%
\begin{pgfscope}%
\pgfsys@transformshift{0.752778in}{1.648652in}%
\pgfsys@useobject{currentmarker}{}%
\end{pgfscope}%
\end{pgfscope}%
\begin{pgfscope}%
\pgfsetbuttcap%
\pgfsetroundjoin%
\definecolor{currentfill}{rgb}{0.000000,0.000000,0.000000}%
\pgfsetfillcolor{currentfill}%
\pgfsetlinewidth{0.602250pt}%
\definecolor{currentstroke}{rgb}{0.000000,0.000000,0.000000}%
\pgfsetstrokecolor{currentstroke}%
\pgfsetdash{}{0pt}%
\pgfsys@defobject{currentmarker}{\pgfqpoint{0.000000in}{0.000000in}}{\pgfqpoint{0.027778in}{0.000000in}}{%
\pgfpathmoveto{\pgfqpoint{0.000000in}{0.000000in}}%
\pgfpathlineto{\pgfqpoint{0.027778in}{0.000000in}}%
\pgfusepath{stroke,fill}%
}%
\begin{pgfscope}%
\pgfsys@transformshift{4.801389in}{1.648652in}%
\pgfsys@useobject{currentmarker}{}%
\end{pgfscope}%
\end{pgfscope}%
\begin{pgfscope}%
\pgfpathrectangle{\pgfqpoint{0.752778in}{0.582778in}}{\pgfqpoint{4.048611in}{3.212222in}}%
\pgfusepath{clip}%
\pgfsetrectcap%
\pgfsetroundjoin%
\pgfsetlinewidth{0.803000pt}%
\definecolor{currentstroke}{rgb}{0.690196,0.690196,0.690196}%
\pgfsetstrokecolor{currentstroke}%
\pgfsetstrokeopacity{0.300000}%
\pgfsetdash{}{0pt}%
\pgfpathmoveto{\pgfqpoint{0.752778in}{1.692455in}}%
\pgfpathlineto{\pgfqpoint{4.801389in}{1.692455in}}%
\pgfusepath{stroke}%
\end{pgfscope}%
\begin{pgfscope}%
\pgfsetbuttcap%
\pgfsetroundjoin%
\definecolor{currentfill}{rgb}{0.000000,0.000000,0.000000}%
\pgfsetfillcolor{currentfill}%
\pgfsetlinewidth{0.602250pt}%
\definecolor{currentstroke}{rgb}{0.000000,0.000000,0.000000}%
\pgfsetstrokecolor{currentstroke}%
\pgfsetdash{}{0pt}%
\pgfsys@defobject{currentmarker}{\pgfqpoint{-0.027778in}{0.000000in}}{\pgfqpoint{0.000000in}{0.000000in}}{%
\pgfpathmoveto{\pgfqpoint{0.000000in}{0.000000in}}%
\pgfpathlineto{\pgfqpoint{-0.027778in}{0.000000in}}%
\pgfusepath{stroke,fill}%
}%
\begin{pgfscope}%
\pgfsys@transformshift{0.752778in}{1.692455in}%
\pgfsys@useobject{currentmarker}{}%
\end{pgfscope}%
\end{pgfscope}%
\begin{pgfscope}%
\pgfsetbuttcap%
\pgfsetroundjoin%
\definecolor{currentfill}{rgb}{0.000000,0.000000,0.000000}%
\pgfsetfillcolor{currentfill}%
\pgfsetlinewidth{0.602250pt}%
\definecolor{currentstroke}{rgb}{0.000000,0.000000,0.000000}%
\pgfsetstrokecolor{currentstroke}%
\pgfsetdash{}{0pt}%
\pgfsys@defobject{currentmarker}{\pgfqpoint{0.000000in}{0.000000in}}{\pgfqpoint{0.027778in}{0.000000in}}{%
\pgfpathmoveto{\pgfqpoint{0.000000in}{0.000000in}}%
\pgfpathlineto{\pgfqpoint{0.027778in}{0.000000in}}%
\pgfusepath{stroke,fill}%
}%
\begin{pgfscope}%
\pgfsys@transformshift{4.801389in}{1.692455in}%
\pgfsys@useobject{currentmarker}{}%
\end{pgfscope}%
\end{pgfscope}%
\begin{pgfscope}%
\pgfpathrectangle{\pgfqpoint{0.752778in}{0.582778in}}{\pgfqpoint{4.048611in}{3.212222in}}%
\pgfusepath{clip}%
\pgfsetrectcap%
\pgfsetroundjoin%
\pgfsetlinewidth{0.803000pt}%
\definecolor{currentstroke}{rgb}{0.690196,0.690196,0.690196}%
\pgfsetstrokecolor{currentstroke}%
\pgfsetstrokeopacity{0.300000}%
\pgfsetdash{}{0pt}%
\pgfpathmoveto{\pgfqpoint{0.752778in}{1.736258in}}%
\pgfpathlineto{\pgfqpoint{4.801389in}{1.736258in}}%
\pgfusepath{stroke}%
\end{pgfscope}%
\begin{pgfscope}%
\pgfsetbuttcap%
\pgfsetroundjoin%
\definecolor{currentfill}{rgb}{0.000000,0.000000,0.000000}%
\pgfsetfillcolor{currentfill}%
\pgfsetlinewidth{0.602250pt}%
\definecolor{currentstroke}{rgb}{0.000000,0.000000,0.000000}%
\pgfsetstrokecolor{currentstroke}%
\pgfsetdash{}{0pt}%
\pgfsys@defobject{currentmarker}{\pgfqpoint{-0.027778in}{0.000000in}}{\pgfqpoint{0.000000in}{0.000000in}}{%
\pgfpathmoveto{\pgfqpoint{0.000000in}{0.000000in}}%
\pgfpathlineto{\pgfqpoint{-0.027778in}{0.000000in}}%
\pgfusepath{stroke,fill}%
}%
\begin{pgfscope}%
\pgfsys@transformshift{0.752778in}{1.736258in}%
\pgfsys@useobject{currentmarker}{}%
\end{pgfscope}%
\end{pgfscope}%
\begin{pgfscope}%
\pgfsetbuttcap%
\pgfsetroundjoin%
\definecolor{currentfill}{rgb}{0.000000,0.000000,0.000000}%
\pgfsetfillcolor{currentfill}%
\pgfsetlinewidth{0.602250pt}%
\definecolor{currentstroke}{rgb}{0.000000,0.000000,0.000000}%
\pgfsetstrokecolor{currentstroke}%
\pgfsetdash{}{0pt}%
\pgfsys@defobject{currentmarker}{\pgfqpoint{0.000000in}{0.000000in}}{\pgfqpoint{0.027778in}{0.000000in}}{%
\pgfpathmoveto{\pgfqpoint{0.000000in}{0.000000in}}%
\pgfpathlineto{\pgfqpoint{0.027778in}{0.000000in}}%
\pgfusepath{stroke,fill}%
}%
\begin{pgfscope}%
\pgfsys@transformshift{4.801389in}{1.736258in}%
\pgfsys@useobject{currentmarker}{}%
\end{pgfscope}%
\end{pgfscope}%
\begin{pgfscope}%
\pgfpathrectangle{\pgfqpoint{0.752778in}{0.582778in}}{\pgfqpoint{4.048611in}{3.212222in}}%
\pgfusepath{clip}%
\pgfsetrectcap%
\pgfsetroundjoin%
\pgfsetlinewidth{0.803000pt}%
\definecolor{currentstroke}{rgb}{0.690196,0.690196,0.690196}%
\pgfsetstrokecolor{currentstroke}%
\pgfsetstrokeopacity{0.300000}%
\pgfsetdash{}{0pt}%
\pgfpathmoveto{\pgfqpoint{0.752778in}{1.780061in}}%
\pgfpathlineto{\pgfqpoint{4.801389in}{1.780061in}}%
\pgfusepath{stroke}%
\end{pgfscope}%
\begin{pgfscope}%
\pgfsetbuttcap%
\pgfsetroundjoin%
\definecolor{currentfill}{rgb}{0.000000,0.000000,0.000000}%
\pgfsetfillcolor{currentfill}%
\pgfsetlinewidth{0.602250pt}%
\definecolor{currentstroke}{rgb}{0.000000,0.000000,0.000000}%
\pgfsetstrokecolor{currentstroke}%
\pgfsetdash{}{0pt}%
\pgfsys@defobject{currentmarker}{\pgfqpoint{-0.027778in}{0.000000in}}{\pgfqpoint{0.000000in}{0.000000in}}{%
\pgfpathmoveto{\pgfqpoint{0.000000in}{0.000000in}}%
\pgfpathlineto{\pgfqpoint{-0.027778in}{0.000000in}}%
\pgfusepath{stroke,fill}%
}%
\begin{pgfscope}%
\pgfsys@transformshift{0.752778in}{1.780061in}%
\pgfsys@useobject{currentmarker}{}%
\end{pgfscope}%
\end{pgfscope}%
\begin{pgfscope}%
\pgfsetbuttcap%
\pgfsetroundjoin%
\definecolor{currentfill}{rgb}{0.000000,0.000000,0.000000}%
\pgfsetfillcolor{currentfill}%
\pgfsetlinewidth{0.602250pt}%
\definecolor{currentstroke}{rgb}{0.000000,0.000000,0.000000}%
\pgfsetstrokecolor{currentstroke}%
\pgfsetdash{}{0pt}%
\pgfsys@defobject{currentmarker}{\pgfqpoint{0.000000in}{0.000000in}}{\pgfqpoint{0.027778in}{0.000000in}}{%
\pgfpathmoveto{\pgfqpoint{0.000000in}{0.000000in}}%
\pgfpathlineto{\pgfqpoint{0.027778in}{0.000000in}}%
\pgfusepath{stroke,fill}%
}%
\begin{pgfscope}%
\pgfsys@transformshift{4.801389in}{1.780061in}%
\pgfsys@useobject{currentmarker}{}%
\end{pgfscope}%
\end{pgfscope}%
\begin{pgfscope}%
\pgfpathrectangle{\pgfqpoint{0.752778in}{0.582778in}}{\pgfqpoint{4.048611in}{3.212222in}}%
\pgfusepath{clip}%
\pgfsetrectcap%
\pgfsetroundjoin%
\pgfsetlinewidth{0.803000pt}%
\definecolor{currentstroke}{rgb}{0.690196,0.690196,0.690196}%
\pgfsetstrokecolor{currentstroke}%
\pgfsetstrokeopacity{0.300000}%
\pgfsetdash{}{0pt}%
\pgfpathmoveto{\pgfqpoint{0.752778in}{1.823864in}}%
\pgfpathlineto{\pgfqpoint{4.801389in}{1.823864in}}%
\pgfusepath{stroke}%
\end{pgfscope}%
\begin{pgfscope}%
\pgfsetbuttcap%
\pgfsetroundjoin%
\definecolor{currentfill}{rgb}{0.000000,0.000000,0.000000}%
\pgfsetfillcolor{currentfill}%
\pgfsetlinewidth{0.602250pt}%
\definecolor{currentstroke}{rgb}{0.000000,0.000000,0.000000}%
\pgfsetstrokecolor{currentstroke}%
\pgfsetdash{}{0pt}%
\pgfsys@defobject{currentmarker}{\pgfqpoint{-0.027778in}{0.000000in}}{\pgfqpoint{0.000000in}{0.000000in}}{%
\pgfpathmoveto{\pgfqpoint{0.000000in}{0.000000in}}%
\pgfpathlineto{\pgfqpoint{-0.027778in}{0.000000in}}%
\pgfusepath{stroke,fill}%
}%
\begin{pgfscope}%
\pgfsys@transformshift{0.752778in}{1.823864in}%
\pgfsys@useobject{currentmarker}{}%
\end{pgfscope}%
\end{pgfscope}%
\begin{pgfscope}%
\pgfsetbuttcap%
\pgfsetroundjoin%
\definecolor{currentfill}{rgb}{0.000000,0.000000,0.000000}%
\pgfsetfillcolor{currentfill}%
\pgfsetlinewidth{0.602250pt}%
\definecolor{currentstroke}{rgb}{0.000000,0.000000,0.000000}%
\pgfsetstrokecolor{currentstroke}%
\pgfsetdash{}{0pt}%
\pgfsys@defobject{currentmarker}{\pgfqpoint{0.000000in}{0.000000in}}{\pgfqpoint{0.027778in}{0.000000in}}{%
\pgfpathmoveto{\pgfqpoint{0.000000in}{0.000000in}}%
\pgfpathlineto{\pgfqpoint{0.027778in}{0.000000in}}%
\pgfusepath{stroke,fill}%
}%
\begin{pgfscope}%
\pgfsys@transformshift{4.801389in}{1.823864in}%
\pgfsys@useobject{currentmarker}{}%
\end{pgfscope}%
\end{pgfscope}%
\begin{pgfscope}%
\pgfpathrectangle{\pgfqpoint{0.752778in}{0.582778in}}{\pgfqpoint{4.048611in}{3.212222in}}%
\pgfusepath{clip}%
\pgfsetrectcap%
\pgfsetroundjoin%
\pgfsetlinewidth{0.803000pt}%
\definecolor{currentstroke}{rgb}{0.690196,0.690196,0.690196}%
\pgfsetstrokecolor{currentstroke}%
\pgfsetstrokeopacity{0.300000}%
\pgfsetdash{}{0pt}%
\pgfpathmoveto{\pgfqpoint{0.752778in}{1.867667in}}%
\pgfpathlineto{\pgfqpoint{4.801389in}{1.867667in}}%
\pgfusepath{stroke}%
\end{pgfscope}%
\begin{pgfscope}%
\pgfsetbuttcap%
\pgfsetroundjoin%
\definecolor{currentfill}{rgb}{0.000000,0.000000,0.000000}%
\pgfsetfillcolor{currentfill}%
\pgfsetlinewidth{0.602250pt}%
\definecolor{currentstroke}{rgb}{0.000000,0.000000,0.000000}%
\pgfsetstrokecolor{currentstroke}%
\pgfsetdash{}{0pt}%
\pgfsys@defobject{currentmarker}{\pgfqpoint{-0.027778in}{0.000000in}}{\pgfqpoint{0.000000in}{0.000000in}}{%
\pgfpathmoveto{\pgfqpoint{0.000000in}{0.000000in}}%
\pgfpathlineto{\pgfqpoint{-0.027778in}{0.000000in}}%
\pgfusepath{stroke,fill}%
}%
\begin{pgfscope}%
\pgfsys@transformshift{0.752778in}{1.867667in}%
\pgfsys@useobject{currentmarker}{}%
\end{pgfscope}%
\end{pgfscope}%
\begin{pgfscope}%
\pgfsetbuttcap%
\pgfsetroundjoin%
\definecolor{currentfill}{rgb}{0.000000,0.000000,0.000000}%
\pgfsetfillcolor{currentfill}%
\pgfsetlinewidth{0.602250pt}%
\definecolor{currentstroke}{rgb}{0.000000,0.000000,0.000000}%
\pgfsetstrokecolor{currentstroke}%
\pgfsetdash{}{0pt}%
\pgfsys@defobject{currentmarker}{\pgfqpoint{0.000000in}{0.000000in}}{\pgfqpoint{0.027778in}{0.000000in}}{%
\pgfpathmoveto{\pgfqpoint{0.000000in}{0.000000in}}%
\pgfpathlineto{\pgfqpoint{0.027778in}{0.000000in}}%
\pgfusepath{stroke,fill}%
}%
\begin{pgfscope}%
\pgfsys@transformshift{4.801389in}{1.867667in}%
\pgfsys@useobject{currentmarker}{}%
\end{pgfscope}%
\end{pgfscope}%
\begin{pgfscope}%
\pgfpathrectangle{\pgfqpoint{0.752778in}{0.582778in}}{\pgfqpoint{4.048611in}{3.212222in}}%
\pgfusepath{clip}%
\pgfsetrectcap%
\pgfsetroundjoin%
\pgfsetlinewidth{0.803000pt}%
\definecolor{currentstroke}{rgb}{0.690196,0.690196,0.690196}%
\pgfsetstrokecolor{currentstroke}%
\pgfsetstrokeopacity{0.300000}%
\pgfsetdash{}{0pt}%
\pgfpathmoveto{\pgfqpoint{0.752778in}{1.911470in}}%
\pgfpathlineto{\pgfqpoint{4.801389in}{1.911470in}}%
\pgfusepath{stroke}%
\end{pgfscope}%
\begin{pgfscope}%
\pgfsetbuttcap%
\pgfsetroundjoin%
\definecolor{currentfill}{rgb}{0.000000,0.000000,0.000000}%
\pgfsetfillcolor{currentfill}%
\pgfsetlinewidth{0.602250pt}%
\definecolor{currentstroke}{rgb}{0.000000,0.000000,0.000000}%
\pgfsetstrokecolor{currentstroke}%
\pgfsetdash{}{0pt}%
\pgfsys@defobject{currentmarker}{\pgfqpoint{-0.027778in}{0.000000in}}{\pgfqpoint{0.000000in}{0.000000in}}{%
\pgfpathmoveto{\pgfqpoint{0.000000in}{0.000000in}}%
\pgfpathlineto{\pgfqpoint{-0.027778in}{0.000000in}}%
\pgfusepath{stroke,fill}%
}%
\begin{pgfscope}%
\pgfsys@transformshift{0.752778in}{1.911470in}%
\pgfsys@useobject{currentmarker}{}%
\end{pgfscope}%
\end{pgfscope}%
\begin{pgfscope}%
\pgfsetbuttcap%
\pgfsetroundjoin%
\definecolor{currentfill}{rgb}{0.000000,0.000000,0.000000}%
\pgfsetfillcolor{currentfill}%
\pgfsetlinewidth{0.602250pt}%
\definecolor{currentstroke}{rgb}{0.000000,0.000000,0.000000}%
\pgfsetstrokecolor{currentstroke}%
\pgfsetdash{}{0pt}%
\pgfsys@defobject{currentmarker}{\pgfqpoint{0.000000in}{0.000000in}}{\pgfqpoint{0.027778in}{0.000000in}}{%
\pgfpathmoveto{\pgfqpoint{0.000000in}{0.000000in}}%
\pgfpathlineto{\pgfqpoint{0.027778in}{0.000000in}}%
\pgfusepath{stroke,fill}%
}%
\begin{pgfscope}%
\pgfsys@transformshift{4.801389in}{1.911470in}%
\pgfsys@useobject{currentmarker}{}%
\end{pgfscope}%
\end{pgfscope}%
\begin{pgfscope}%
\pgfpathrectangle{\pgfqpoint{0.752778in}{0.582778in}}{\pgfqpoint{4.048611in}{3.212222in}}%
\pgfusepath{clip}%
\pgfsetrectcap%
\pgfsetroundjoin%
\pgfsetlinewidth{0.803000pt}%
\definecolor{currentstroke}{rgb}{0.690196,0.690196,0.690196}%
\pgfsetstrokecolor{currentstroke}%
\pgfsetstrokeopacity{0.300000}%
\pgfsetdash{}{0pt}%
\pgfpathmoveto{\pgfqpoint{0.752778in}{1.955273in}}%
\pgfpathlineto{\pgfqpoint{4.801389in}{1.955273in}}%
\pgfusepath{stroke}%
\end{pgfscope}%
\begin{pgfscope}%
\pgfsetbuttcap%
\pgfsetroundjoin%
\definecolor{currentfill}{rgb}{0.000000,0.000000,0.000000}%
\pgfsetfillcolor{currentfill}%
\pgfsetlinewidth{0.602250pt}%
\definecolor{currentstroke}{rgb}{0.000000,0.000000,0.000000}%
\pgfsetstrokecolor{currentstroke}%
\pgfsetdash{}{0pt}%
\pgfsys@defobject{currentmarker}{\pgfqpoint{-0.027778in}{0.000000in}}{\pgfqpoint{0.000000in}{0.000000in}}{%
\pgfpathmoveto{\pgfqpoint{0.000000in}{0.000000in}}%
\pgfpathlineto{\pgfqpoint{-0.027778in}{0.000000in}}%
\pgfusepath{stroke,fill}%
}%
\begin{pgfscope}%
\pgfsys@transformshift{0.752778in}{1.955273in}%
\pgfsys@useobject{currentmarker}{}%
\end{pgfscope}%
\end{pgfscope}%
\begin{pgfscope}%
\pgfsetbuttcap%
\pgfsetroundjoin%
\definecolor{currentfill}{rgb}{0.000000,0.000000,0.000000}%
\pgfsetfillcolor{currentfill}%
\pgfsetlinewidth{0.602250pt}%
\definecolor{currentstroke}{rgb}{0.000000,0.000000,0.000000}%
\pgfsetstrokecolor{currentstroke}%
\pgfsetdash{}{0pt}%
\pgfsys@defobject{currentmarker}{\pgfqpoint{0.000000in}{0.000000in}}{\pgfqpoint{0.027778in}{0.000000in}}{%
\pgfpathmoveto{\pgfqpoint{0.000000in}{0.000000in}}%
\pgfpathlineto{\pgfqpoint{0.027778in}{0.000000in}}%
\pgfusepath{stroke,fill}%
}%
\begin{pgfscope}%
\pgfsys@transformshift{4.801389in}{1.955273in}%
\pgfsys@useobject{currentmarker}{}%
\end{pgfscope}%
\end{pgfscope}%
\begin{pgfscope}%
\pgfpathrectangle{\pgfqpoint{0.752778in}{0.582778in}}{\pgfqpoint{4.048611in}{3.212222in}}%
\pgfusepath{clip}%
\pgfsetrectcap%
\pgfsetroundjoin%
\pgfsetlinewidth{0.803000pt}%
\definecolor{currentstroke}{rgb}{0.690196,0.690196,0.690196}%
\pgfsetstrokecolor{currentstroke}%
\pgfsetstrokeopacity{0.300000}%
\pgfsetdash{}{0pt}%
\pgfpathmoveto{\pgfqpoint{0.752778in}{1.999076in}}%
\pgfpathlineto{\pgfqpoint{4.801389in}{1.999076in}}%
\pgfusepath{stroke}%
\end{pgfscope}%
\begin{pgfscope}%
\pgfsetbuttcap%
\pgfsetroundjoin%
\definecolor{currentfill}{rgb}{0.000000,0.000000,0.000000}%
\pgfsetfillcolor{currentfill}%
\pgfsetlinewidth{0.602250pt}%
\definecolor{currentstroke}{rgb}{0.000000,0.000000,0.000000}%
\pgfsetstrokecolor{currentstroke}%
\pgfsetdash{}{0pt}%
\pgfsys@defobject{currentmarker}{\pgfqpoint{-0.027778in}{0.000000in}}{\pgfqpoint{0.000000in}{0.000000in}}{%
\pgfpathmoveto{\pgfqpoint{0.000000in}{0.000000in}}%
\pgfpathlineto{\pgfqpoint{-0.027778in}{0.000000in}}%
\pgfusepath{stroke,fill}%
}%
\begin{pgfscope}%
\pgfsys@transformshift{0.752778in}{1.999076in}%
\pgfsys@useobject{currentmarker}{}%
\end{pgfscope}%
\end{pgfscope}%
\begin{pgfscope}%
\pgfsetbuttcap%
\pgfsetroundjoin%
\definecolor{currentfill}{rgb}{0.000000,0.000000,0.000000}%
\pgfsetfillcolor{currentfill}%
\pgfsetlinewidth{0.602250pt}%
\definecolor{currentstroke}{rgb}{0.000000,0.000000,0.000000}%
\pgfsetstrokecolor{currentstroke}%
\pgfsetdash{}{0pt}%
\pgfsys@defobject{currentmarker}{\pgfqpoint{0.000000in}{0.000000in}}{\pgfqpoint{0.027778in}{0.000000in}}{%
\pgfpathmoveto{\pgfqpoint{0.000000in}{0.000000in}}%
\pgfpathlineto{\pgfqpoint{0.027778in}{0.000000in}}%
\pgfusepath{stroke,fill}%
}%
\begin{pgfscope}%
\pgfsys@transformshift{4.801389in}{1.999076in}%
\pgfsys@useobject{currentmarker}{}%
\end{pgfscope}%
\end{pgfscope}%
\begin{pgfscope}%
\pgfpathrectangle{\pgfqpoint{0.752778in}{0.582778in}}{\pgfqpoint{4.048611in}{3.212222in}}%
\pgfusepath{clip}%
\pgfsetrectcap%
\pgfsetroundjoin%
\pgfsetlinewidth{0.803000pt}%
\definecolor{currentstroke}{rgb}{0.690196,0.690196,0.690196}%
\pgfsetstrokecolor{currentstroke}%
\pgfsetstrokeopacity{0.300000}%
\pgfsetdash{}{0pt}%
\pgfpathmoveto{\pgfqpoint{0.752778in}{2.086682in}}%
\pgfpathlineto{\pgfqpoint{4.801389in}{2.086682in}}%
\pgfusepath{stroke}%
\end{pgfscope}%
\begin{pgfscope}%
\pgfsetbuttcap%
\pgfsetroundjoin%
\definecolor{currentfill}{rgb}{0.000000,0.000000,0.000000}%
\pgfsetfillcolor{currentfill}%
\pgfsetlinewidth{0.602250pt}%
\definecolor{currentstroke}{rgb}{0.000000,0.000000,0.000000}%
\pgfsetstrokecolor{currentstroke}%
\pgfsetdash{}{0pt}%
\pgfsys@defobject{currentmarker}{\pgfqpoint{-0.027778in}{0.000000in}}{\pgfqpoint{0.000000in}{0.000000in}}{%
\pgfpathmoveto{\pgfqpoint{0.000000in}{0.000000in}}%
\pgfpathlineto{\pgfqpoint{-0.027778in}{0.000000in}}%
\pgfusepath{stroke,fill}%
}%
\begin{pgfscope}%
\pgfsys@transformshift{0.752778in}{2.086682in}%
\pgfsys@useobject{currentmarker}{}%
\end{pgfscope}%
\end{pgfscope}%
\begin{pgfscope}%
\pgfsetbuttcap%
\pgfsetroundjoin%
\definecolor{currentfill}{rgb}{0.000000,0.000000,0.000000}%
\pgfsetfillcolor{currentfill}%
\pgfsetlinewidth{0.602250pt}%
\definecolor{currentstroke}{rgb}{0.000000,0.000000,0.000000}%
\pgfsetstrokecolor{currentstroke}%
\pgfsetdash{}{0pt}%
\pgfsys@defobject{currentmarker}{\pgfqpoint{0.000000in}{0.000000in}}{\pgfqpoint{0.027778in}{0.000000in}}{%
\pgfpathmoveto{\pgfqpoint{0.000000in}{0.000000in}}%
\pgfpathlineto{\pgfqpoint{0.027778in}{0.000000in}}%
\pgfusepath{stroke,fill}%
}%
\begin{pgfscope}%
\pgfsys@transformshift{4.801389in}{2.086682in}%
\pgfsys@useobject{currentmarker}{}%
\end{pgfscope}%
\end{pgfscope}%
\begin{pgfscope}%
\pgfpathrectangle{\pgfqpoint{0.752778in}{0.582778in}}{\pgfqpoint{4.048611in}{3.212222in}}%
\pgfusepath{clip}%
\pgfsetrectcap%
\pgfsetroundjoin%
\pgfsetlinewidth{0.803000pt}%
\definecolor{currentstroke}{rgb}{0.690196,0.690196,0.690196}%
\pgfsetstrokecolor{currentstroke}%
\pgfsetstrokeopacity{0.300000}%
\pgfsetdash{}{0pt}%
\pgfpathmoveto{\pgfqpoint{0.752778in}{2.130485in}}%
\pgfpathlineto{\pgfqpoint{4.801389in}{2.130485in}}%
\pgfusepath{stroke}%
\end{pgfscope}%
\begin{pgfscope}%
\pgfsetbuttcap%
\pgfsetroundjoin%
\definecolor{currentfill}{rgb}{0.000000,0.000000,0.000000}%
\pgfsetfillcolor{currentfill}%
\pgfsetlinewidth{0.602250pt}%
\definecolor{currentstroke}{rgb}{0.000000,0.000000,0.000000}%
\pgfsetstrokecolor{currentstroke}%
\pgfsetdash{}{0pt}%
\pgfsys@defobject{currentmarker}{\pgfqpoint{-0.027778in}{0.000000in}}{\pgfqpoint{0.000000in}{0.000000in}}{%
\pgfpathmoveto{\pgfqpoint{0.000000in}{0.000000in}}%
\pgfpathlineto{\pgfqpoint{-0.027778in}{0.000000in}}%
\pgfusepath{stroke,fill}%
}%
\begin{pgfscope}%
\pgfsys@transformshift{0.752778in}{2.130485in}%
\pgfsys@useobject{currentmarker}{}%
\end{pgfscope}%
\end{pgfscope}%
\begin{pgfscope}%
\pgfsetbuttcap%
\pgfsetroundjoin%
\definecolor{currentfill}{rgb}{0.000000,0.000000,0.000000}%
\pgfsetfillcolor{currentfill}%
\pgfsetlinewidth{0.602250pt}%
\definecolor{currentstroke}{rgb}{0.000000,0.000000,0.000000}%
\pgfsetstrokecolor{currentstroke}%
\pgfsetdash{}{0pt}%
\pgfsys@defobject{currentmarker}{\pgfqpoint{0.000000in}{0.000000in}}{\pgfqpoint{0.027778in}{0.000000in}}{%
\pgfpathmoveto{\pgfqpoint{0.000000in}{0.000000in}}%
\pgfpathlineto{\pgfqpoint{0.027778in}{0.000000in}}%
\pgfusepath{stroke,fill}%
}%
\begin{pgfscope}%
\pgfsys@transformshift{4.801389in}{2.130485in}%
\pgfsys@useobject{currentmarker}{}%
\end{pgfscope}%
\end{pgfscope}%
\begin{pgfscope}%
\pgfpathrectangle{\pgfqpoint{0.752778in}{0.582778in}}{\pgfqpoint{4.048611in}{3.212222in}}%
\pgfusepath{clip}%
\pgfsetrectcap%
\pgfsetroundjoin%
\pgfsetlinewidth{0.803000pt}%
\definecolor{currentstroke}{rgb}{0.690196,0.690196,0.690196}%
\pgfsetstrokecolor{currentstroke}%
\pgfsetstrokeopacity{0.300000}%
\pgfsetdash{}{0pt}%
\pgfpathmoveto{\pgfqpoint{0.752778in}{2.174288in}}%
\pgfpathlineto{\pgfqpoint{4.801389in}{2.174288in}}%
\pgfusepath{stroke}%
\end{pgfscope}%
\begin{pgfscope}%
\pgfsetbuttcap%
\pgfsetroundjoin%
\definecolor{currentfill}{rgb}{0.000000,0.000000,0.000000}%
\pgfsetfillcolor{currentfill}%
\pgfsetlinewidth{0.602250pt}%
\definecolor{currentstroke}{rgb}{0.000000,0.000000,0.000000}%
\pgfsetstrokecolor{currentstroke}%
\pgfsetdash{}{0pt}%
\pgfsys@defobject{currentmarker}{\pgfqpoint{-0.027778in}{0.000000in}}{\pgfqpoint{0.000000in}{0.000000in}}{%
\pgfpathmoveto{\pgfqpoint{0.000000in}{0.000000in}}%
\pgfpathlineto{\pgfqpoint{-0.027778in}{0.000000in}}%
\pgfusepath{stroke,fill}%
}%
\begin{pgfscope}%
\pgfsys@transformshift{0.752778in}{2.174288in}%
\pgfsys@useobject{currentmarker}{}%
\end{pgfscope}%
\end{pgfscope}%
\begin{pgfscope}%
\pgfsetbuttcap%
\pgfsetroundjoin%
\definecolor{currentfill}{rgb}{0.000000,0.000000,0.000000}%
\pgfsetfillcolor{currentfill}%
\pgfsetlinewidth{0.602250pt}%
\definecolor{currentstroke}{rgb}{0.000000,0.000000,0.000000}%
\pgfsetstrokecolor{currentstroke}%
\pgfsetdash{}{0pt}%
\pgfsys@defobject{currentmarker}{\pgfqpoint{0.000000in}{0.000000in}}{\pgfqpoint{0.027778in}{0.000000in}}{%
\pgfpathmoveto{\pgfqpoint{0.000000in}{0.000000in}}%
\pgfpathlineto{\pgfqpoint{0.027778in}{0.000000in}}%
\pgfusepath{stroke,fill}%
}%
\begin{pgfscope}%
\pgfsys@transformshift{4.801389in}{2.174288in}%
\pgfsys@useobject{currentmarker}{}%
\end{pgfscope}%
\end{pgfscope}%
\begin{pgfscope}%
\pgfpathrectangle{\pgfqpoint{0.752778in}{0.582778in}}{\pgfqpoint{4.048611in}{3.212222in}}%
\pgfusepath{clip}%
\pgfsetrectcap%
\pgfsetroundjoin%
\pgfsetlinewidth{0.803000pt}%
\definecolor{currentstroke}{rgb}{0.690196,0.690196,0.690196}%
\pgfsetstrokecolor{currentstroke}%
\pgfsetstrokeopacity{0.300000}%
\pgfsetdash{}{0pt}%
\pgfpathmoveto{\pgfqpoint{0.752778in}{2.218091in}}%
\pgfpathlineto{\pgfqpoint{4.801389in}{2.218091in}}%
\pgfusepath{stroke}%
\end{pgfscope}%
\begin{pgfscope}%
\pgfsetbuttcap%
\pgfsetroundjoin%
\definecolor{currentfill}{rgb}{0.000000,0.000000,0.000000}%
\pgfsetfillcolor{currentfill}%
\pgfsetlinewidth{0.602250pt}%
\definecolor{currentstroke}{rgb}{0.000000,0.000000,0.000000}%
\pgfsetstrokecolor{currentstroke}%
\pgfsetdash{}{0pt}%
\pgfsys@defobject{currentmarker}{\pgfqpoint{-0.027778in}{0.000000in}}{\pgfqpoint{0.000000in}{0.000000in}}{%
\pgfpathmoveto{\pgfqpoint{0.000000in}{0.000000in}}%
\pgfpathlineto{\pgfqpoint{-0.027778in}{0.000000in}}%
\pgfusepath{stroke,fill}%
}%
\begin{pgfscope}%
\pgfsys@transformshift{0.752778in}{2.218091in}%
\pgfsys@useobject{currentmarker}{}%
\end{pgfscope}%
\end{pgfscope}%
\begin{pgfscope}%
\pgfsetbuttcap%
\pgfsetroundjoin%
\definecolor{currentfill}{rgb}{0.000000,0.000000,0.000000}%
\pgfsetfillcolor{currentfill}%
\pgfsetlinewidth{0.602250pt}%
\definecolor{currentstroke}{rgb}{0.000000,0.000000,0.000000}%
\pgfsetstrokecolor{currentstroke}%
\pgfsetdash{}{0pt}%
\pgfsys@defobject{currentmarker}{\pgfqpoint{0.000000in}{0.000000in}}{\pgfqpoint{0.027778in}{0.000000in}}{%
\pgfpathmoveto{\pgfqpoint{0.000000in}{0.000000in}}%
\pgfpathlineto{\pgfqpoint{0.027778in}{0.000000in}}%
\pgfusepath{stroke,fill}%
}%
\begin{pgfscope}%
\pgfsys@transformshift{4.801389in}{2.218091in}%
\pgfsys@useobject{currentmarker}{}%
\end{pgfscope}%
\end{pgfscope}%
\begin{pgfscope}%
\pgfpathrectangle{\pgfqpoint{0.752778in}{0.582778in}}{\pgfqpoint{4.048611in}{3.212222in}}%
\pgfusepath{clip}%
\pgfsetrectcap%
\pgfsetroundjoin%
\pgfsetlinewidth{0.803000pt}%
\definecolor{currentstroke}{rgb}{0.690196,0.690196,0.690196}%
\pgfsetstrokecolor{currentstroke}%
\pgfsetstrokeopacity{0.300000}%
\pgfsetdash{}{0pt}%
\pgfpathmoveto{\pgfqpoint{0.752778in}{2.261894in}}%
\pgfpathlineto{\pgfqpoint{4.801389in}{2.261894in}}%
\pgfusepath{stroke}%
\end{pgfscope}%
\begin{pgfscope}%
\pgfsetbuttcap%
\pgfsetroundjoin%
\definecolor{currentfill}{rgb}{0.000000,0.000000,0.000000}%
\pgfsetfillcolor{currentfill}%
\pgfsetlinewidth{0.602250pt}%
\definecolor{currentstroke}{rgb}{0.000000,0.000000,0.000000}%
\pgfsetstrokecolor{currentstroke}%
\pgfsetdash{}{0pt}%
\pgfsys@defobject{currentmarker}{\pgfqpoint{-0.027778in}{0.000000in}}{\pgfqpoint{0.000000in}{0.000000in}}{%
\pgfpathmoveto{\pgfqpoint{0.000000in}{0.000000in}}%
\pgfpathlineto{\pgfqpoint{-0.027778in}{0.000000in}}%
\pgfusepath{stroke,fill}%
}%
\begin{pgfscope}%
\pgfsys@transformshift{0.752778in}{2.261894in}%
\pgfsys@useobject{currentmarker}{}%
\end{pgfscope}%
\end{pgfscope}%
\begin{pgfscope}%
\pgfsetbuttcap%
\pgfsetroundjoin%
\definecolor{currentfill}{rgb}{0.000000,0.000000,0.000000}%
\pgfsetfillcolor{currentfill}%
\pgfsetlinewidth{0.602250pt}%
\definecolor{currentstroke}{rgb}{0.000000,0.000000,0.000000}%
\pgfsetstrokecolor{currentstroke}%
\pgfsetdash{}{0pt}%
\pgfsys@defobject{currentmarker}{\pgfqpoint{0.000000in}{0.000000in}}{\pgfqpoint{0.027778in}{0.000000in}}{%
\pgfpathmoveto{\pgfqpoint{0.000000in}{0.000000in}}%
\pgfpathlineto{\pgfqpoint{0.027778in}{0.000000in}}%
\pgfusepath{stroke,fill}%
}%
\begin{pgfscope}%
\pgfsys@transformshift{4.801389in}{2.261894in}%
\pgfsys@useobject{currentmarker}{}%
\end{pgfscope}%
\end{pgfscope}%
\begin{pgfscope}%
\pgfpathrectangle{\pgfqpoint{0.752778in}{0.582778in}}{\pgfqpoint{4.048611in}{3.212222in}}%
\pgfusepath{clip}%
\pgfsetrectcap%
\pgfsetroundjoin%
\pgfsetlinewidth{0.803000pt}%
\definecolor{currentstroke}{rgb}{0.690196,0.690196,0.690196}%
\pgfsetstrokecolor{currentstroke}%
\pgfsetstrokeopacity{0.300000}%
\pgfsetdash{}{0pt}%
\pgfpathmoveto{\pgfqpoint{0.752778in}{2.305697in}}%
\pgfpathlineto{\pgfqpoint{4.801389in}{2.305697in}}%
\pgfusepath{stroke}%
\end{pgfscope}%
\begin{pgfscope}%
\pgfsetbuttcap%
\pgfsetroundjoin%
\definecolor{currentfill}{rgb}{0.000000,0.000000,0.000000}%
\pgfsetfillcolor{currentfill}%
\pgfsetlinewidth{0.602250pt}%
\definecolor{currentstroke}{rgb}{0.000000,0.000000,0.000000}%
\pgfsetstrokecolor{currentstroke}%
\pgfsetdash{}{0pt}%
\pgfsys@defobject{currentmarker}{\pgfqpoint{-0.027778in}{0.000000in}}{\pgfqpoint{0.000000in}{0.000000in}}{%
\pgfpathmoveto{\pgfqpoint{0.000000in}{0.000000in}}%
\pgfpathlineto{\pgfqpoint{-0.027778in}{0.000000in}}%
\pgfusepath{stroke,fill}%
}%
\begin{pgfscope}%
\pgfsys@transformshift{0.752778in}{2.305697in}%
\pgfsys@useobject{currentmarker}{}%
\end{pgfscope}%
\end{pgfscope}%
\begin{pgfscope}%
\pgfsetbuttcap%
\pgfsetroundjoin%
\definecolor{currentfill}{rgb}{0.000000,0.000000,0.000000}%
\pgfsetfillcolor{currentfill}%
\pgfsetlinewidth{0.602250pt}%
\definecolor{currentstroke}{rgb}{0.000000,0.000000,0.000000}%
\pgfsetstrokecolor{currentstroke}%
\pgfsetdash{}{0pt}%
\pgfsys@defobject{currentmarker}{\pgfqpoint{0.000000in}{0.000000in}}{\pgfqpoint{0.027778in}{0.000000in}}{%
\pgfpathmoveto{\pgfqpoint{0.000000in}{0.000000in}}%
\pgfpathlineto{\pgfqpoint{0.027778in}{0.000000in}}%
\pgfusepath{stroke,fill}%
}%
\begin{pgfscope}%
\pgfsys@transformshift{4.801389in}{2.305697in}%
\pgfsys@useobject{currentmarker}{}%
\end{pgfscope}%
\end{pgfscope}%
\begin{pgfscope}%
\pgfpathrectangle{\pgfqpoint{0.752778in}{0.582778in}}{\pgfqpoint{4.048611in}{3.212222in}}%
\pgfusepath{clip}%
\pgfsetrectcap%
\pgfsetroundjoin%
\pgfsetlinewidth{0.803000pt}%
\definecolor{currentstroke}{rgb}{0.690196,0.690196,0.690196}%
\pgfsetstrokecolor{currentstroke}%
\pgfsetstrokeopacity{0.300000}%
\pgfsetdash{}{0pt}%
\pgfpathmoveto{\pgfqpoint{0.752778in}{2.349500in}}%
\pgfpathlineto{\pgfqpoint{4.801389in}{2.349500in}}%
\pgfusepath{stroke}%
\end{pgfscope}%
\begin{pgfscope}%
\pgfsetbuttcap%
\pgfsetroundjoin%
\definecolor{currentfill}{rgb}{0.000000,0.000000,0.000000}%
\pgfsetfillcolor{currentfill}%
\pgfsetlinewidth{0.602250pt}%
\definecolor{currentstroke}{rgb}{0.000000,0.000000,0.000000}%
\pgfsetstrokecolor{currentstroke}%
\pgfsetdash{}{0pt}%
\pgfsys@defobject{currentmarker}{\pgfqpoint{-0.027778in}{0.000000in}}{\pgfqpoint{0.000000in}{0.000000in}}{%
\pgfpathmoveto{\pgfqpoint{0.000000in}{0.000000in}}%
\pgfpathlineto{\pgfqpoint{-0.027778in}{0.000000in}}%
\pgfusepath{stroke,fill}%
}%
\begin{pgfscope}%
\pgfsys@transformshift{0.752778in}{2.349500in}%
\pgfsys@useobject{currentmarker}{}%
\end{pgfscope}%
\end{pgfscope}%
\begin{pgfscope}%
\pgfsetbuttcap%
\pgfsetroundjoin%
\definecolor{currentfill}{rgb}{0.000000,0.000000,0.000000}%
\pgfsetfillcolor{currentfill}%
\pgfsetlinewidth{0.602250pt}%
\definecolor{currentstroke}{rgb}{0.000000,0.000000,0.000000}%
\pgfsetstrokecolor{currentstroke}%
\pgfsetdash{}{0pt}%
\pgfsys@defobject{currentmarker}{\pgfqpoint{0.000000in}{0.000000in}}{\pgfqpoint{0.027778in}{0.000000in}}{%
\pgfpathmoveto{\pgfqpoint{0.000000in}{0.000000in}}%
\pgfpathlineto{\pgfqpoint{0.027778in}{0.000000in}}%
\pgfusepath{stroke,fill}%
}%
\begin{pgfscope}%
\pgfsys@transformshift{4.801389in}{2.349500in}%
\pgfsys@useobject{currentmarker}{}%
\end{pgfscope}%
\end{pgfscope}%
\begin{pgfscope}%
\pgfpathrectangle{\pgfqpoint{0.752778in}{0.582778in}}{\pgfqpoint{4.048611in}{3.212222in}}%
\pgfusepath{clip}%
\pgfsetrectcap%
\pgfsetroundjoin%
\pgfsetlinewidth{0.803000pt}%
\definecolor{currentstroke}{rgb}{0.690196,0.690196,0.690196}%
\pgfsetstrokecolor{currentstroke}%
\pgfsetstrokeopacity{0.300000}%
\pgfsetdash{}{0pt}%
\pgfpathmoveto{\pgfqpoint{0.752778in}{2.393303in}}%
\pgfpathlineto{\pgfqpoint{4.801389in}{2.393303in}}%
\pgfusepath{stroke}%
\end{pgfscope}%
\begin{pgfscope}%
\pgfsetbuttcap%
\pgfsetroundjoin%
\definecolor{currentfill}{rgb}{0.000000,0.000000,0.000000}%
\pgfsetfillcolor{currentfill}%
\pgfsetlinewidth{0.602250pt}%
\definecolor{currentstroke}{rgb}{0.000000,0.000000,0.000000}%
\pgfsetstrokecolor{currentstroke}%
\pgfsetdash{}{0pt}%
\pgfsys@defobject{currentmarker}{\pgfqpoint{-0.027778in}{0.000000in}}{\pgfqpoint{0.000000in}{0.000000in}}{%
\pgfpathmoveto{\pgfqpoint{0.000000in}{0.000000in}}%
\pgfpathlineto{\pgfqpoint{-0.027778in}{0.000000in}}%
\pgfusepath{stroke,fill}%
}%
\begin{pgfscope}%
\pgfsys@transformshift{0.752778in}{2.393303in}%
\pgfsys@useobject{currentmarker}{}%
\end{pgfscope}%
\end{pgfscope}%
\begin{pgfscope}%
\pgfsetbuttcap%
\pgfsetroundjoin%
\definecolor{currentfill}{rgb}{0.000000,0.000000,0.000000}%
\pgfsetfillcolor{currentfill}%
\pgfsetlinewidth{0.602250pt}%
\definecolor{currentstroke}{rgb}{0.000000,0.000000,0.000000}%
\pgfsetstrokecolor{currentstroke}%
\pgfsetdash{}{0pt}%
\pgfsys@defobject{currentmarker}{\pgfqpoint{0.000000in}{0.000000in}}{\pgfqpoint{0.027778in}{0.000000in}}{%
\pgfpathmoveto{\pgfqpoint{0.000000in}{0.000000in}}%
\pgfpathlineto{\pgfqpoint{0.027778in}{0.000000in}}%
\pgfusepath{stroke,fill}%
}%
\begin{pgfscope}%
\pgfsys@transformshift{4.801389in}{2.393303in}%
\pgfsys@useobject{currentmarker}{}%
\end{pgfscope}%
\end{pgfscope}%
\begin{pgfscope}%
\pgfpathrectangle{\pgfqpoint{0.752778in}{0.582778in}}{\pgfqpoint{4.048611in}{3.212222in}}%
\pgfusepath{clip}%
\pgfsetrectcap%
\pgfsetroundjoin%
\pgfsetlinewidth{0.803000pt}%
\definecolor{currentstroke}{rgb}{0.690196,0.690196,0.690196}%
\pgfsetstrokecolor{currentstroke}%
\pgfsetstrokeopacity{0.300000}%
\pgfsetdash{}{0pt}%
\pgfpathmoveto{\pgfqpoint{0.752778in}{2.437106in}}%
\pgfpathlineto{\pgfqpoint{4.801389in}{2.437106in}}%
\pgfusepath{stroke}%
\end{pgfscope}%
\begin{pgfscope}%
\pgfsetbuttcap%
\pgfsetroundjoin%
\definecolor{currentfill}{rgb}{0.000000,0.000000,0.000000}%
\pgfsetfillcolor{currentfill}%
\pgfsetlinewidth{0.602250pt}%
\definecolor{currentstroke}{rgb}{0.000000,0.000000,0.000000}%
\pgfsetstrokecolor{currentstroke}%
\pgfsetdash{}{0pt}%
\pgfsys@defobject{currentmarker}{\pgfqpoint{-0.027778in}{0.000000in}}{\pgfqpoint{0.000000in}{0.000000in}}{%
\pgfpathmoveto{\pgfqpoint{0.000000in}{0.000000in}}%
\pgfpathlineto{\pgfqpoint{-0.027778in}{0.000000in}}%
\pgfusepath{stroke,fill}%
}%
\begin{pgfscope}%
\pgfsys@transformshift{0.752778in}{2.437106in}%
\pgfsys@useobject{currentmarker}{}%
\end{pgfscope}%
\end{pgfscope}%
\begin{pgfscope}%
\pgfsetbuttcap%
\pgfsetroundjoin%
\definecolor{currentfill}{rgb}{0.000000,0.000000,0.000000}%
\pgfsetfillcolor{currentfill}%
\pgfsetlinewidth{0.602250pt}%
\definecolor{currentstroke}{rgb}{0.000000,0.000000,0.000000}%
\pgfsetstrokecolor{currentstroke}%
\pgfsetdash{}{0pt}%
\pgfsys@defobject{currentmarker}{\pgfqpoint{0.000000in}{0.000000in}}{\pgfqpoint{0.027778in}{0.000000in}}{%
\pgfpathmoveto{\pgfqpoint{0.000000in}{0.000000in}}%
\pgfpathlineto{\pgfqpoint{0.027778in}{0.000000in}}%
\pgfusepath{stroke,fill}%
}%
\begin{pgfscope}%
\pgfsys@transformshift{4.801389in}{2.437106in}%
\pgfsys@useobject{currentmarker}{}%
\end{pgfscope}%
\end{pgfscope}%
\begin{pgfscope}%
\pgfpathrectangle{\pgfqpoint{0.752778in}{0.582778in}}{\pgfqpoint{4.048611in}{3.212222in}}%
\pgfusepath{clip}%
\pgfsetrectcap%
\pgfsetroundjoin%
\pgfsetlinewidth{0.803000pt}%
\definecolor{currentstroke}{rgb}{0.690196,0.690196,0.690196}%
\pgfsetstrokecolor{currentstroke}%
\pgfsetstrokeopacity{0.300000}%
\pgfsetdash{}{0pt}%
\pgfpathmoveto{\pgfqpoint{0.752778in}{2.524712in}}%
\pgfpathlineto{\pgfqpoint{4.801389in}{2.524712in}}%
\pgfusepath{stroke}%
\end{pgfscope}%
\begin{pgfscope}%
\pgfsetbuttcap%
\pgfsetroundjoin%
\definecolor{currentfill}{rgb}{0.000000,0.000000,0.000000}%
\pgfsetfillcolor{currentfill}%
\pgfsetlinewidth{0.602250pt}%
\definecolor{currentstroke}{rgb}{0.000000,0.000000,0.000000}%
\pgfsetstrokecolor{currentstroke}%
\pgfsetdash{}{0pt}%
\pgfsys@defobject{currentmarker}{\pgfqpoint{-0.027778in}{0.000000in}}{\pgfqpoint{0.000000in}{0.000000in}}{%
\pgfpathmoveto{\pgfqpoint{0.000000in}{0.000000in}}%
\pgfpathlineto{\pgfqpoint{-0.027778in}{0.000000in}}%
\pgfusepath{stroke,fill}%
}%
\begin{pgfscope}%
\pgfsys@transformshift{0.752778in}{2.524712in}%
\pgfsys@useobject{currentmarker}{}%
\end{pgfscope}%
\end{pgfscope}%
\begin{pgfscope}%
\pgfsetbuttcap%
\pgfsetroundjoin%
\definecolor{currentfill}{rgb}{0.000000,0.000000,0.000000}%
\pgfsetfillcolor{currentfill}%
\pgfsetlinewidth{0.602250pt}%
\definecolor{currentstroke}{rgb}{0.000000,0.000000,0.000000}%
\pgfsetstrokecolor{currentstroke}%
\pgfsetdash{}{0pt}%
\pgfsys@defobject{currentmarker}{\pgfqpoint{0.000000in}{0.000000in}}{\pgfqpoint{0.027778in}{0.000000in}}{%
\pgfpathmoveto{\pgfqpoint{0.000000in}{0.000000in}}%
\pgfpathlineto{\pgfqpoint{0.027778in}{0.000000in}}%
\pgfusepath{stroke,fill}%
}%
\begin{pgfscope}%
\pgfsys@transformshift{4.801389in}{2.524712in}%
\pgfsys@useobject{currentmarker}{}%
\end{pgfscope}%
\end{pgfscope}%
\begin{pgfscope}%
\pgfpathrectangle{\pgfqpoint{0.752778in}{0.582778in}}{\pgfqpoint{4.048611in}{3.212222in}}%
\pgfusepath{clip}%
\pgfsetrectcap%
\pgfsetroundjoin%
\pgfsetlinewidth{0.803000pt}%
\definecolor{currentstroke}{rgb}{0.690196,0.690196,0.690196}%
\pgfsetstrokecolor{currentstroke}%
\pgfsetstrokeopacity{0.300000}%
\pgfsetdash{}{0pt}%
\pgfpathmoveto{\pgfqpoint{0.752778in}{2.568515in}}%
\pgfpathlineto{\pgfqpoint{4.801389in}{2.568515in}}%
\pgfusepath{stroke}%
\end{pgfscope}%
\begin{pgfscope}%
\pgfsetbuttcap%
\pgfsetroundjoin%
\definecolor{currentfill}{rgb}{0.000000,0.000000,0.000000}%
\pgfsetfillcolor{currentfill}%
\pgfsetlinewidth{0.602250pt}%
\definecolor{currentstroke}{rgb}{0.000000,0.000000,0.000000}%
\pgfsetstrokecolor{currentstroke}%
\pgfsetdash{}{0pt}%
\pgfsys@defobject{currentmarker}{\pgfqpoint{-0.027778in}{0.000000in}}{\pgfqpoint{0.000000in}{0.000000in}}{%
\pgfpathmoveto{\pgfqpoint{0.000000in}{0.000000in}}%
\pgfpathlineto{\pgfqpoint{-0.027778in}{0.000000in}}%
\pgfusepath{stroke,fill}%
}%
\begin{pgfscope}%
\pgfsys@transformshift{0.752778in}{2.568515in}%
\pgfsys@useobject{currentmarker}{}%
\end{pgfscope}%
\end{pgfscope}%
\begin{pgfscope}%
\pgfsetbuttcap%
\pgfsetroundjoin%
\definecolor{currentfill}{rgb}{0.000000,0.000000,0.000000}%
\pgfsetfillcolor{currentfill}%
\pgfsetlinewidth{0.602250pt}%
\definecolor{currentstroke}{rgb}{0.000000,0.000000,0.000000}%
\pgfsetstrokecolor{currentstroke}%
\pgfsetdash{}{0pt}%
\pgfsys@defobject{currentmarker}{\pgfqpoint{0.000000in}{0.000000in}}{\pgfqpoint{0.027778in}{0.000000in}}{%
\pgfpathmoveto{\pgfqpoint{0.000000in}{0.000000in}}%
\pgfpathlineto{\pgfqpoint{0.027778in}{0.000000in}}%
\pgfusepath{stroke,fill}%
}%
\begin{pgfscope}%
\pgfsys@transformshift{4.801389in}{2.568515in}%
\pgfsys@useobject{currentmarker}{}%
\end{pgfscope}%
\end{pgfscope}%
\begin{pgfscope}%
\pgfpathrectangle{\pgfqpoint{0.752778in}{0.582778in}}{\pgfqpoint{4.048611in}{3.212222in}}%
\pgfusepath{clip}%
\pgfsetrectcap%
\pgfsetroundjoin%
\pgfsetlinewidth{0.803000pt}%
\definecolor{currentstroke}{rgb}{0.690196,0.690196,0.690196}%
\pgfsetstrokecolor{currentstroke}%
\pgfsetstrokeopacity{0.300000}%
\pgfsetdash{}{0pt}%
\pgfpathmoveto{\pgfqpoint{0.752778in}{2.612318in}}%
\pgfpathlineto{\pgfqpoint{4.801389in}{2.612318in}}%
\pgfusepath{stroke}%
\end{pgfscope}%
\begin{pgfscope}%
\pgfsetbuttcap%
\pgfsetroundjoin%
\definecolor{currentfill}{rgb}{0.000000,0.000000,0.000000}%
\pgfsetfillcolor{currentfill}%
\pgfsetlinewidth{0.602250pt}%
\definecolor{currentstroke}{rgb}{0.000000,0.000000,0.000000}%
\pgfsetstrokecolor{currentstroke}%
\pgfsetdash{}{0pt}%
\pgfsys@defobject{currentmarker}{\pgfqpoint{-0.027778in}{0.000000in}}{\pgfqpoint{0.000000in}{0.000000in}}{%
\pgfpathmoveto{\pgfqpoint{0.000000in}{0.000000in}}%
\pgfpathlineto{\pgfqpoint{-0.027778in}{0.000000in}}%
\pgfusepath{stroke,fill}%
}%
\begin{pgfscope}%
\pgfsys@transformshift{0.752778in}{2.612318in}%
\pgfsys@useobject{currentmarker}{}%
\end{pgfscope}%
\end{pgfscope}%
\begin{pgfscope}%
\pgfsetbuttcap%
\pgfsetroundjoin%
\definecolor{currentfill}{rgb}{0.000000,0.000000,0.000000}%
\pgfsetfillcolor{currentfill}%
\pgfsetlinewidth{0.602250pt}%
\definecolor{currentstroke}{rgb}{0.000000,0.000000,0.000000}%
\pgfsetstrokecolor{currentstroke}%
\pgfsetdash{}{0pt}%
\pgfsys@defobject{currentmarker}{\pgfqpoint{0.000000in}{0.000000in}}{\pgfqpoint{0.027778in}{0.000000in}}{%
\pgfpathmoveto{\pgfqpoint{0.000000in}{0.000000in}}%
\pgfpathlineto{\pgfqpoint{0.027778in}{0.000000in}}%
\pgfusepath{stroke,fill}%
}%
\begin{pgfscope}%
\pgfsys@transformshift{4.801389in}{2.612318in}%
\pgfsys@useobject{currentmarker}{}%
\end{pgfscope}%
\end{pgfscope}%
\begin{pgfscope}%
\pgfpathrectangle{\pgfqpoint{0.752778in}{0.582778in}}{\pgfqpoint{4.048611in}{3.212222in}}%
\pgfusepath{clip}%
\pgfsetrectcap%
\pgfsetroundjoin%
\pgfsetlinewidth{0.803000pt}%
\definecolor{currentstroke}{rgb}{0.690196,0.690196,0.690196}%
\pgfsetstrokecolor{currentstroke}%
\pgfsetstrokeopacity{0.300000}%
\pgfsetdash{}{0pt}%
\pgfpathmoveto{\pgfqpoint{0.752778in}{2.656121in}}%
\pgfpathlineto{\pgfqpoint{4.801389in}{2.656121in}}%
\pgfusepath{stroke}%
\end{pgfscope}%
\begin{pgfscope}%
\pgfsetbuttcap%
\pgfsetroundjoin%
\definecolor{currentfill}{rgb}{0.000000,0.000000,0.000000}%
\pgfsetfillcolor{currentfill}%
\pgfsetlinewidth{0.602250pt}%
\definecolor{currentstroke}{rgb}{0.000000,0.000000,0.000000}%
\pgfsetstrokecolor{currentstroke}%
\pgfsetdash{}{0pt}%
\pgfsys@defobject{currentmarker}{\pgfqpoint{-0.027778in}{0.000000in}}{\pgfqpoint{0.000000in}{0.000000in}}{%
\pgfpathmoveto{\pgfqpoint{0.000000in}{0.000000in}}%
\pgfpathlineto{\pgfqpoint{-0.027778in}{0.000000in}}%
\pgfusepath{stroke,fill}%
}%
\begin{pgfscope}%
\pgfsys@transformshift{0.752778in}{2.656121in}%
\pgfsys@useobject{currentmarker}{}%
\end{pgfscope}%
\end{pgfscope}%
\begin{pgfscope}%
\pgfsetbuttcap%
\pgfsetroundjoin%
\definecolor{currentfill}{rgb}{0.000000,0.000000,0.000000}%
\pgfsetfillcolor{currentfill}%
\pgfsetlinewidth{0.602250pt}%
\definecolor{currentstroke}{rgb}{0.000000,0.000000,0.000000}%
\pgfsetstrokecolor{currentstroke}%
\pgfsetdash{}{0pt}%
\pgfsys@defobject{currentmarker}{\pgfqpoint{0.000000in}{0.000000in}}{\pgfqpoint{0.027778in}{0.000000in}}{%
\pgfpathmoveto{\pgfqpoint{0.000000in}{0.000000in}}%
\pgfpathlineto{\pgfqpoint{0.027778in}{0.000000in}}%
\pgfusepath{stroke,fill}%
}%
\begin{pgfscope}%
\pgfsys@transformshift{4.801389in}{2.656121in}%
\pgfsys@useobject{currentmarker}{}%
\end{pgfscope}%
\end{pgfscope}%
\begin{pgfscope}%
\pgfpathrectangle{\pgfqpoint{0.752778in}{0.582778in}}{\pgfqpoint{4.048611in}{3.212222in}}%
\pgfusepath{clip}%
\pgfsetrectcap%
\pgfsetroundjoin%
\pgfsetlinewidth{0.803000pt}%
\definecolor{currentstroke}{rgb}{0.690196,0.690196,0.690196}%
\pgfsetstrokecolor{currentstroke}%
\pgfsetstrokeopacity{0.300000}%
\pgfsetdash{}{0pt}%
\pgfpathmoveto{\pgfqpoint{0.752778in}{2.699924in}}%
\pgfpathlineto{\pgfqpoint{4.801389in}{2.699924in}}%
\pgfusepath{stroke}%
\end{pgfscope}%
\begin{pgfscope}%
\pgfsetbuttcap%
\pgfsetroundjoin%
\definecolor{currentfill}{rgb}{0.000000,0.000000,0.000000}%
\pgfsetfillcolor{currentfill}%
\pgfsetlinewidth{0.602250pt}%
\definecolor{currentstroke}{rgb}{0.000000,0.000000,0.000000}%
\pgfsetstrokecolor{currentstroke}%
\pgfsetdash{}{0pt}%
\pgfsys@defobject{currentmarker}{\pgfqpoint{-0.027778in}{0.000000in}}{\pgfqpoint{0.000000in}{0.000000in}}{%
\pgfpathmoveto{\pgfqpoint{0.000000in}{0.000000in}}%
\pgfpathlineto{\pgfqpoint{-0.027778in}{0.000000in}}%
\pgfusepath{stroke,fill}%
}%
\begin{pgfscope}%
\pgfsys@transformshift{0.752778in}{2.699924in}%
\pgfsys@useobject{currentmarker}{}%
\end{pgfscope}%
\end{pgfscope}%
\begin{pgfscope}%
\pgfsetbuttcap%
\pgfsetroundjoin%
\definecolor{currentfill}{rgb}{0.000000,0.000000,0.000000}%
\pgfsetfillcolor{currentfill}%
\pgfsetlinewidth{0.602250pt}%
\definecolor{currentstroke}{rgb}{0.000000,0.000000,0.000000}%
\pgfsetstrokecolor{currentstroke}%
\pgfsetdash{}{0pt}%
\pgfsys@defobject{currentmarker}{\pgfqpoint{0.000000in}{0.000000in}}{\pgfqpoint{0.027778in}{0.000000in}}{%
\pgfpathmoveto{\pgfqpoint{0.000000in}{0.000000in}}%
\pgfpathlineto{\pgfqpoint{0.027778in}{0.000000in}}%
\pgfusepath{stroke,fill}%
}%
\begin{pgfscope}%
\pgfsys@transformshift{4.801389in}{2.699924in}%
\pgfsys@useobject{currentmarker}{}%
\end{pgfscope}%
\end{pgfscope}%
\begin{pgfscope}%
\pgfpathrectangle{\pgfqpoint{0.752778in}{0.582778in}}{\pgfqpoint{4.048611in}{3.212222in}}%
\pgfusepath{clip}%
\pgfsetrectcap%
\pgfsetroundjoin%
\pgfsetlinewidth{0.803000pt}%
\definecolor{currentstroke}{rgb}{0.690196,0.690196,0.690196}%
\pgfsetstrokecolor{currentstroke}%
\pgfsetstrokeopacity{0.300000}%
\pgfsetdash{}{0pt}%
\pgfpathmoveto{\pgfqpoint{0.752778in}{2.743727in}}%
\pgfpathlineto{\pgfqpoint{4.801389in}{2.743727in}}%
\pgfusepath{stroke}%
\end{pgfscope}%
\begin{pgfscope}%
\pgfsetbuttcap%
\pgfsetroundjoin%
\definecolor{currentfill}{rgb}{0.000000,0.000000,0.000000}%
\pgfsetfillcolor{currentfill}%
\pgfsetlinewidth{0.602250pt}%
\definecolor{currentstroke}{rgb}{0.000000,0.000000,0.000000}%
\pgfsetstrokecolor{currentstroke}%
\pgfsetdash{}{0pt}%
\pgfsys@defobject{currentmarker}{\pgfqpoint{-0.027778in}{0.000000in}}{\pgfqpoint{0.000000in}{0.000000in}}{%
\pgfpathmoveto{\pgfqpoint{0.000000in}{0.000000in}}%
\pgfpathlineto{\pgfqpoint{-0.027778in}{0.000000in}}%
\pgfusepath{stroke,fill}%
}%
\begin{pgfscope}%
\pgfsys@transformshift{0.752778in}{2.743727in}%
\pgfsys@useobject{currentmarker}{}%
\end{pgfscope}%
\end{pgfscope}%
\begin{pgfscope}%
\pgfsetbuttcap%
\pgfsetroundjoin%
\definecolor{currentfill}{rgb}{0.000000,0.000000,0.000000}%
\pgfsetfillcolor{currentfill}%
\pgfsetlinewidth{0.602250pt}%
\definecolor{currentstroke}{rgb}{0.000000,0.000000,0.000000}%
\pgfsetstrokecolor{currentstroke}%
\pgfsetdash{}{0pt}%
\pgfsys@defobject{currentmarker}{\pgfqpoint{0.000000in}{0.000000in}}{\pgfqpoint{0.027778in}{0.000000in}}{%
\pgfpathmoveto{\pgfqpoint{0.000000in}{0.000000in}}%
\pgfpathlineto{\pgfqpoint{0.027778in}{0.000000in}}%
\pgfusepath{stroke,fill}%
}%
\begin{pgfscope}%
\pgfsys@transformshift{4.801389in}{2.743727in}%
\pgfsys@useobject{currentmarker}{}%
\end{pgfscope}%
\end{pgfscope}%
\begin{pgfscope}%
\pgfpathrectangle{\pgfqpoint{0.752778in}{0.582778in}}{\pgfqpoint{4.048611in}{3.212222in}}%
\pgfusepath{clip}%
\pgfsetrectcap%
\pgfsetroundjoin%
\pgfsetlinewidth{0.803000pt}%
\definecolor{currentstroke}{rgb}{0.690196,0.690196,0.690196}%
\pgfsetstrokecolor{currentstroke}%
\pgfsetstrokeopacity{0.300000}%
\pgfsetdash{}{0pt}%
\pgfpathmoveto{\pgfqpoint{0.752778in}{2.787530in}}%
\pgfpathlineto{\pgfqpoint{4.801389in}{2.787530in}}%
\pgfusepath{stroke}%
\end{pgfscope}%
\begin{pgfscope}%
\pgfsetbuttcap%
\pgfsetroundjoin%
\definecolor{currentfill}{rgb}{0.000000,0.000000,0.000000}%
\pgfsetfillcolor{currentfill}%
\pgfsetlinewidth{0.602250pt}%
\definecolor{currentstroke}{rgb}{0.000000,0.000000,0.000000}%
\pgfsetstrokecolor{currentstroke}%
\pgfsetdash{}{0pt}%
\pgfsys@defobject{currentmarker}{\pgfqpoint{-0.027778in}{0.000000in}}{\pgfqpoint{0.000000in}{0.000000in}}{%
\pgfpathmoveto{\pgfqpoint{0.000000in}{0.000000in}}%
\pgfpathlineto{\pgfqpoint{-0.027778in}{0.000000in}}%
\pgfusepath{stroke,fill}%
}%
\begin{pgfscope}%
\pgfsys@transformshift{0.752778in}{2.787530in}%
\pgfsys@useobject{currentmarker}{}%
\end{pgfscope}%
\end{pgfscope}%
\begin{pgfscope}%
\pgfsetbuttcap%
\pgfsetroundjoin%
\definecolor{currentfill}{rgb}{0.000000,0.000000,0.000000}%
\pgfsetfillcolor{currentfill}%
\pgfsetlinewidth{0.602250pt}%
\definecolor{currentstroke}{rgb}{0.000000,0.000000,0.000000}%
\pgfsetstrokecolor{currentstroke}%
\pgfsetdash{}{0pt}%
\pgfsys@defobject{currentmarker}{\pgfqpoint{0.000000in}{0.000000in}}{\pgfqpoint{0.027778in}{0.000000in}}{%
\pgfpathmoveto{\pgfqpoint{0.000000in}{0.000000in}}%
\pgfpathlineto{\pgfqpoint{0.027778in}{0.000000in}}%
\pgfusepath{stroke,fill}%
}%
\begin{pgfscope}%
\pgfsys@transformshift{4.801389in}{2.787530in}%
\pgfsys@useobject{currentmarker}{}%
\end{pgfscope}%
\end{pgfscope}%
\begin{pgfscope}%
\pgfpathrectangle{\pgfqpoint{0.752778in}{0.582778in}}{\pgfqpoint{4.048611in}{3.212222in}}%
\pgfusepath{clip}%
\pgfsetrectcap%
\pgfsetroundjoin%
\pgfsetlinewidth{0.803000pt}%
\definecolor{currentstroke}{rgb}{0.690196,0.690196,0.690196}%
\pgfsetstrokecolor{currentstroke}%
\pgfsetstrokeopacity{0.300000}%
\pgfsetdash{}{0pt}%
\pgfpathmoveto{\pgfqpoint{0.752778in}{2.831333in}}%
\pgfpathlineto{\pgfqpoint{4.801389in}{2.831333in}}%
\pgfusepath{stroke}%
\end{pgfscope}%
\begin{pgfscope}%
\pgfsetbuttcap%
\pgfsetroundjoin%
\definecolor{currentfill}{rgb}{0.000000,0.000000,0.000000}%
\pgfsetfillcolor{currentfill}%
\pgfsetlinewidth{0.602250pt}%
\definecolor{currentstroke}{rgb}{0.000000,0.000000,0.000000}%
\pgfsetstrokecolor{currentstroke}%
\pgfsetdash{}{0pt}%
\pgfsys@defobject{currentmarker}{\pgfqpoint{-0.027778in}{0.000000in}}{\pgfqpoint{0.000000in}{0.000000in}}{%
\pgfpathmoveto{\pgfqpoint{0.000000in}{0.000000in}}%
\pgfpathlineto{\pgfqpoint{-0.027778in}{0.000000in}}%
\pgfusepath{stroke,fill}%
}%
\begin{pgfscope}%
\pgfsys@transformshift{0.752778in}{2.831333in}%
\pgfsys@useobject{currentmarker}{}%
\end{pgfscope}%
\end{pgfscope}%
\begin{pgfscope}%
\pgfsetbuttcap%
\pgfsetroundjoin%
\definecolor{currentfill}{rgb}{0.000000,0.000000,0.000000}%
\pgfsetfillcolor{currentfill}%
\pgfsetlinewidth{0.602250pt}%
\definecolor{currentstroke}{rgb}{0.000000,0.000000,0.000000}%
\pgfsetstrokecolor{currentstroke}%
\pgfsetdash{}{0pt}%
\pgfsys@defobject{currentmarker}{\pgfqpoint{0.000000in}{0.000000in}}{\pgfqpoint{0.027778in}{0.000000in}}{%
\pgfpathmoveto{\pgfqpoint{0.000000in}{0.000000in}}%
\pgfpathlineto{\pgfqpoint{0.027778in}{0.000000in}}%
\pgfusepath{stroke,fill}%
}%
\begin{pgfscope}%
\pgfsys@transformshift{4.801389in}{2.831333in}%
\pgfsys@useobject{currentmarker}{}%
\end{pgfscope}%
\end{pgfscope}%
\begin{pgfscope}%
\pgfpathrectangle{\pgfqpoint{0.752778in}{0.582778in}}{\pgfqpoint{4.048611in}{3.212222in}}%
\pgfusepath{clip}%
\pgfsetrectcap%
\pgfsetroundjoin%
\pgfsetlinewidth{0.803000pt}%
\definecolor{currentstroke}{rgb}{0.690196,0.690196,0.690196}%
\pgfsetstrokecolor{currentstroke}%
\pgfsetstrokeopacity{0.300000}%
\pgfsetdash{}{0pt}%
\pgfpathmoveto{\pgfqpoint{0.752778in}{2.875136in}}%
\pgfpathlineto{\pgfqpoint{4.801389in}{2.875136in}}%
\pgfusepath{stroke}%
\end{pgfscope}%
\begin{pgfscope}%
\pgfsetbuttcap%
\pgfsetroundjoin%
\definecolor{currentfill}{rgb}{0.000000,0.000000,0.000000}%
\pgfsetfillcolor{currentfill}%
\pgfsetlinewidth{0.602250pt}%
\definecolor{currentstroke}{rgb}{0.000000,0.000000,0.000000}%
\pgfsetstrokecolor{currentstroke}%
\pgfsetdash{}{0pt}%
\pgfsys@defobject{currentmarker}{\pgfqpoint{-0.027778in}{0.000000in}}{\pgfqpoint{0.000000in}{0.000000in}}{%
\pgfpathmoveto{\pgfqpoint{0.000000in}{0.000000in}}%
\pgfpathlineto{\pgfqpoint{-0.027778in}{0.000000in}}%
\pgfusepath{stroke,fill}%
}%
\begin{pgfscope}%
\pgfsys@transformshift{0.752778in}{2.875136in}%
\pgfsys@useobject{currentmarker}{}%
\end{pgfscope}%
\end{pgfscope}%
\begin{pgfscope}%
\pgfsetbuttcap%
\pgfsetroundjoin%
\definecolor{currentfill}{rgb}{0.000000,0.000000,0.000000}%
\pgfsetfillcolor{currentfill}%
\pgfsetlinewidth{0.602250pt}%
\definecolor{currentstroke}{rgb}{0.000000,0.000000,0.000000}%
\pgfsetstrokecolor{currentstroke}%
\pgfsetdash{}{0pt}%
\pgfsys@defobject{currentmarker}{\pgfqpoint{0.000000in}{0.000000in}}{\pgfqpoint{0.027778in}{0.000000in}}{%
\pgfpathmoveto{\pgfqpoint{0.000000in}{0.000000in}}%
\pgfpathlineto{\pgfqpoint{0.027778in}{0.000000in}}%
\pgfusepath{stroke,fill}%
}%
\begin{pgfscope}%
\pgfsys@transformshift{4.801389in}{2.875136in}%
\pgfsys@useobject{currentmarker}{}%
\end{pgfscope}%
\end{pgfscope}%
\begin{pgfscope}%
\pgfpathrectangle{\pgfqpoint{0.752778in}{0.582778in}}{\pgfqpoint{4.048611in}{3.212222in}}%
\pgfusepath{clip}%
\pgfsetrectcap%
\pgfsetroundjoin%
\pgfsetlinewidth{0.803000pt}%
\definecolor{currentstroke}{rgb}{0.690196,0.690196,0.690196}%
\pgfsetstrokecolor{currentstroke}%
\pgfsetstrokeopacity{0.300000}%
\pgfsetdash{}{0pt}%
\pgfpathmoveto{\pgfqpoint{0.752778in}{2.962742in}}%
\pgfpathlineto{\pgfqpoint{4.801389in}{2.962742in}}%
\pgfusepath{stroke}%
\end{pgfscope}%
\begin{pgfscope}%
\pgfsetbuttcap%
\pgfsetroundjoin%
\definecolor{currentfill}{rgb}{0.000000,0.000000,0.000000}%
\pgfsetfillcolor{currentfill}%
\pgfsetlinewidth{0.602250pt}%
\definecolor{currentstroke}{rgb}{0.000000,0.000000,0.000000}%
\pgfsetstrokecolor{currentstroke}%
\pgfsetdash{}{0pt}%
\pgfsys@defobject{currentmarker}{\pgfqpoint{-0.027778in}{0.000000in}}{\pgfqpoint{0.000000in}{0.000000in}}{%
\pgfpathmoveto{\pgfqpoint{0.000000in}{0.000000in}}%
\pgfpathlineto{\pgfqpoint{-0.027778in}{0.000000in}}%
\pgfusepath{stroke,fill}%
}%
\begin{pgfscope}%
\pgfsys@transformshift{0.752778in}{2.962742in}%
\pgfsys@useobject{currentmarker}{}%
\end{pgfscope}%
\end{pgfscope}%
\begin{pgfscope}%
\pgfsetbuttcap%
\pgfsetroundjoin%
\definecolor{currentfill}{rgb}{0.000000,0.000000,0.000000}%
\pgfsetfillcolor{currentfill}%
\pgfsetlinewidth{0.602250pt}%
\definecolor{currentstroke}{rgb}{0.000000,0.000000,0.000000}%
\pgfsetstrokecolor{currentstroke}%
\pgfsetdash{}{0pt}%
\pgfsys@defobject{currentmarker}{\pgfqpoint{0.000000in}{0.000000in}}{\pgfqpoint{0.027778in}{0.000000in}}{%
\pgfpathmoveto{\pgfqpoint{0.000000in}{0.000000in}}%
\pgfpathlineto{\pgfqpoint{0.027778in}{0.000000in}}%
\pgfusepath{stroke,fill}%
}%
\begin{pgfscope}%
\pgfsys@transformshift{4.801389in}{2.962742in}%
\pgfsys@useobject{currentmarker}{}%
\end{pgfscope}%
\end{pgfscope}%
\begin{pgfscope}%
\pgfpathrectangle{\pgfqpoint{0.752778in}{0.582778in}}{\pgfqpoint{4.048611in}{3.212222in}}%
\pgfusepath{clip}%
\pgfsetrectcap%
\pgfsetroundjoin%
\pgfsetlinewidth{0.803000pt}%
\definecolor{currentstroke}{rgb}{0.690196,0.690196,0.690196}%
\pgfsetstrokecolor{currentstroke}%
\pgfsetstrokeopacity{0.300000}%
\pgfsetdash{}{0pt}%
\pgfpathmoveto{\pgfqpoint{0.752778in}{3.006545in}}%
\pgfpathlineto{\pgfqpoint{4.801389in}{3.006545in}}%
\pgfusepath{stroke}%
\end{pgfscope}%
\begin{pgfscope}%
\pgfsetbuttcap%
\pgfsetroundjoin%
\definecolor{currentfill}{rgb}{0.000000,0.000000,0.000000}%
\pgfsetfillcolor{currentfill}%
\pgfsetlinewidth{0.602250pt}%
\definecolor{currentstroke}{rgb}{0.000000,0.000000,0.000000}%
\pgfsetstrokecolor{currentstroke}%
\pgfsetdash{}{0pt}%
\pgfsys@defobject{currentmarker}{\pgfqpoint{-0.027778in}{0.000000in}}{\pgfqpoint{0.000000in}{0.000000in}}{%
\pgfpathmoveto{\pgfqpoint{0.000000in}{0.000000in}}%
\pgfpathlineto{\pgfqpoint{-0.027778in}{0.000000in}}%
\pgfusepath{stroke,fill}%
}%
\begin{pgfscope}%
\pgfsys@transformshift{0.752778in}{3.006545in}%
\pgfsys@useobject{currentmarker}{}%
\end{pgfscope}%
\end{pgfscope}%
\begin{pgfscope}%
\pgfsetbuttcap%
\pgfsetroundjoin%
\definecolor{currentfill}{rgb}{0.000000,0.000000,0.000000}%
\pgfsetfillcolor{currentfill}%
\pgfsetlinewidth{0.602250pt}%
\definecolor{currentstroke}{rgb}{0.000000,0.000000,0.000000}%
\pgfsetstrokecolor{currentstroke}%
\pgfsetdash{}{0pt}%
\pgfsys@defobject{currentmarker}{\pgfqpoint{0.000000in}{0.000000in}}{\pgfqpoint{0.027778in}{0.000000in}}{%
\pgfpathmoveto{\pgfqpoint{0.000000in}{0.000000in}}%
\pgfpathlineto{\pgfqpoint{0.027778in}{0.000000in}}%
\pgfusepath{stroke,fill}%
}%
\begin{pgfscope}%
\pgfsys@transformshift{4.801389in}{3.006545in}%
\pgfsys@useobject{currentmarker}{}%
\end{pgfscope}%
\end{pgfscope}%
\begin{pgfscope}%
\pgfpathrectangle{\pgfqpoint{0.752778in}{0.582778in}}{\pgfqpoint{4.048611in}{3.212222in}}%
\pgfusepath{clip}%
\pgfsetrectcap%
\pgfsetroundjoin%
\pgfsetlinewidth{0.803000pt}%
\definecolor{currentstroke}{rgb}{0.690196,0.690196,0.690196}%
\pgfsetstrokecolor{currentstroke}%
\pgfsetstrokeopacity{0.300000}%
\pgfsetdash{}{0pt}%
\pgfpathmoveto{\pgfqpoint{0.752778in}{3.050348in}}%
\pgfpathlineto{\pgfqpoint{4.801389in}{3.050348in}}%
\pgfusepath{stroke}%
\end{pgfscope}%
\begin{pgfscope}%
\pgfsetbuttcap%
\pgfsetroundjoin%
\definecolor{currentfill}{rgb}{0.000000,0.000000,0.000000}%
\pgfsetfillcolor{currentfill}%
\pgfsetlinewidth{0.602250pt}%
\definecolor{currentstroke}{rgb}{0.000000,0.000000,0.000000}%
\pgfsetstrokecolor{currentstroke}%
\pgfsetdash{}{0pt}%
\pgfsys@defobject{currentmarker}{\pgfqpoint{-0.027778in}{0.000000in}}{\pgfqpoint{0.000000in}{0.000000in}}{%
\pgfpathmoveto{\pgfqpoint{0.000000in}{0.000000in}}%
\pgfpathlineto{\pgfqpoint{-0.027778in}{0.000000in}}%
\pgfusepath{stroke,fill}%
}%
\begin{pgfscope}%
\pgfsys@transformshift{0.752778in}{3.050348in}%
\pgfsys@useobject{currentmarker}{}%
\end{pgfscope}%
\end{pgfscope}%
\begin{pgfscope}%
\pgfsetbuttcap%
\pgfsetroundjoin%
\definecolor{currentfill}{rgb}{0.000000,0.000000,0.000000}%
\pgfsetfillcolor{currentfill}%
\pgfsetlinewidth{0.602250pt}%
\definecolor{currentstroke}{rgb}{0.000000,0.000000,0.000000}%
\pgfsetstrokecolor{currentstroke}%
\pgfsetdash{}{0pt}%
\pgfsys@defobject{currentmarker}{\pgfqpoint{0.000000in}{0.000000in}}{\pgfqpoint{0.027778in}{0.000000in}}{%
\pgfpathmoveto{\pgfqpoint{0.000000in}{0.000000in}}%
\pgfpathlineto{\pgfqpoint{0.027778in}{0.000000in}}%
\pgfusepath{stroke,fill}%
}%
\begin{pgfscope}%
\pgfsys@transformshift{4.801389in}{3.050348in}%
\pgfsys@useobject{currentmarker}{}%
\end{pgfscope}%
\end{pgfscope}%
\begin{pgfscope}%
\pgfpathrectangle{\pgfqpoint{0.752778in}{0.582778in}}{\pgfqpoint{4.048611in}{3.212222in}}%
\pgfusepath{clip}%
\pgfsetrectcap%
\pgfsetroundjoin%
\pgfsetlinewidth{0.803000pt}%
\definecolor{currentstroke}{rgb}{0.690196,0.690196,0.690196}%
\pgfsetstrokecolor{currentstroke}%
\pgfsetstrokeopacity{0.300000}%
\pgfsetdash{}{0pt}%
\pgfpathmoveto{\pgfqpoint{0.752778in}{3.094152in}}%
\pgfpathlineto{\pgfqpoint{4.801389in}{3.094152in}}%
\pgfusepath{stroke}%
\end{pgfscope}%
\begin{pgfscope}%
\pgfsetbuttcap%
\pgfsetroundjoin%
\definecolor{currentfill}{rgb}{0.000000,0.000000,0.000000}%
\pgfsetfillcolor{currentfill}%
\pgfsetlinewidth{0.602250pt}%
\definecolor{currentstroke}{rgb}{0.000000,0.000000,0.000000}%
\pgfsetstrokecolor{currentstroke}%
\pgfsetdash{}{0pt}%
\pgfsys@defobject{currentmarker}{\pgfqpoint{-0.027778in}{0.000000in}}{\pgfqpoint{0.000000in}{0.000000in}}{%
\pgfpathmoveto{\pgfqpoint{0.000000in}{0.000000in}}%
\pgfpathlineto{\pgfqpoint{-0.027778in}{0.000000in}}%
\pgfusepath{stroke,fill}%
}%
\begin{pgfscope}%
\pgfsys@transformshift{0.752778in}{3.094152in}%
\pgfsys@useobject{currentmarker}{}%
\end{pgfscope}%
\end{pgfscope}%
\begin{pgfscope}%
\pgfsetbuttcap%
\pgfsetroundjoin%
\definecolor{currentfill}{rgb}{0.000000,0.000000,0.000000}%
\pgfsetfillcolor{currentfill}%
\pgfsetlinewidth{0.602250pt}%
\definecolor{currentstroke}{rgb}{0.000000,0.000000,0.000000}%
\pgfsetstrokecolor{currentstroke}%
\pgfsetdash{}{0pt}%
\pgfsys@defobject{currentmarker}{\pgfqpoint{0.000000in}{0.000000in}}{\pgfqpoint{0.027778in}{0.000000in}}{%
\pgfpathmoveto{\pgfqpoint{0.000000in}{0.000000in}}%
\pgfpathlineto{\pgfqpoint{0.027778in}{0.000000in}}%
\pgfusepath{stroke,fill}%
}%
\begin{pgfscope}%
\pgfsys@transformshift{4.801389in}{3.094152in}%
\pgfsys@useobject{currentmarker}{}%
\end{pgfscope}%
\end{pgfscope}%
\begin{pgfscope}%
\pgfpathrectangle{\pgfqpoint{0.752778in}{0.582778in}}{\pgfqpoint{4.048611in}{3.212222in}}%
\pgfusepath{clip}%
\pgfsetrectcap%
\pgfsetroundjoin%
\pgfsetlinewidth{0.803000pt}%
\definecolor{currentstroke}{rgb}{0.690196,0.690196,0.690196}%
\pgfsetstrokecolor{currentstroke}%
\pgfsetstrokeopacity{0.300000}%
\pgfsetdash{}{0pt}%
\pgfpathmoveto{\pgfqpoint{0.752778in}{3.137955in}}%
\pgfpathlineto{\pgfqpoint{4.801389in}{3.137955in}}%
\pgfusepath{stroke}%
\end{pgfscope}%
\begin{pgfscope}%
\pgfsetbuttcap%
\pgfsetroundjoin%
\definecolor{currentfill}{rgb}{0.000000,0.000000,0.000000}%
\pgfsetfillcolor{currentfill}%
\pgfsetlinewidth{0.602250pt}%
\definecolor{currentstroke}{rgb}{0.000000,0.000000,0.000000}%
\pgfsetstrokecolor{currentstroke}%
\pgfsetdash{}{0pt}%
\pgfsys@defobject{currentmarker}{\pgfqpoint{-0.027778in}{0.000000in}}{\pgfqpoint{0.000000in}{0.000000in}}{%
\pgfpathmoveto{\pgfqpoint{0.000000in}{0.000000in}}%
\pgfpathlineto{\pgfqpoint{-0.027778in}{0.000000in}}%
\pgfusepath{stroke,fill}%
}%
\begin{pgfscope}%
\pgfsys@transformshift{0.752778in}{3.137955in}%
\pgfsys@useobject{currentmarker}{}%
\end{pgfscope}%
\end{pgfscope}%
\begin{pgfscope}%
\pgfsetbuttcap%
\pgfsetroundjoin%
\definecolor{currentfill}{rgb}{0.000000,0.000000,0.000000}%
\pgfsetfillcolor{currentfill}%
\pgfsetlinewidth{0.602250pt}%
\definecolor{currentstroke}{rgb}{0.000000,0.000000,0.000000}%
\pgfsetstrokecolor{currentstroke}%
\pgfsetdash{}{0pt}%
\pgfsys@defobject{currentmarker}{\pgfqpoint{0.000000in}{0.000000in}}{\pgfqpoint{0.027778in}{0.000000in}}{%
\pgfpathmoveto{\pgfqpoint{0.000000in}{0.000000in}}%
\pgfpathlineto{\pgfqpoint{0.027778in}{0.000000in}}%
\pgfusepath{stroke,fill}%
}%
\begin{pgfscope}%
\pgfsys@transformshift{4.801389in}{3.137955in}%
\pgfsys@useobject{currentmarker}{}%
\end{pgfscope}%
\end{pgfscope}%
\begin{pgfscope}%
\pgfpathrectangle{\pgfqpoint{0.752778in}{0.582778in}}{\pgfqpoint{4.048611in}{3.212222in}}%
\pgfusepath{clip}%
\pgfsetrectcap%
\pgfsetroundjoin%
\pgfsetlinewidth{0.803000pt}%
\definecolor{currentstroke}{rgb}{0.690196,0.690196,0.690196}%
\pgfsetstrokecolor{currentstroke}%
\pgfsetstrokeopacity{0.300000}%
\pgfsetdash{}{0pt}%
\pgfpathmoveto{\pgfqpoint{0.752778in}{3.181758in}}%
\pgfpathlineto{\pgfqpoint{4.801389in}{3.181758in}}%
\pgfusepath{stroke}%
\end{pgfscope}%
\begin{pgfscope}%
\pgfsetbuttcap%
\pgfsetroundjoin%
\definecolor{currentfill}{rgb}{0.000000,0.000000,0.000000}%
\pgfsetfillcolor{currentfill}%
\pgfsetlinewidth{0.602250pt}%
\definecolor{currentstroke}{rgb}{0.000000,0.000000,0.000000}%
\pgfsetstrokecolor{currentstroke}%
\pgfsetdash{}{0pt}%
\pgfsys@defobject{currentmarker}{\pgfqpoint{-0.027778in}{0.000000in}}{\pgfqpoint{0.000000in}{0.000000in}}{%
\pgfpathmoveto{\pgfqpoint{0.000000in}{0.000000in}}%
\pgfpathlineto{\pgfqpoint{-0.027778in}{0.000000in}}%
\pgfusepath{stroke,fill}%
}%
\begin{pgfscope}%
\pgfsys@transformshift{0.752778in}{3.181758in}%
\pgfsys@useobject{currentmarker}{}%
\end{pgfscope}%
\end{pgfscope}%
\begin{pgfscope}%
\pgfsetbuttcap%
\pgfsetroundjoin%
\definecolor{currentfill}{rgb}{0.000000,0.000000,0.000000}%
\pgfsetfillcolor{currentfill}%
\pgfsetlinewidth{0.602250pt}%
\definecolor{currentstroke}{rgb}{0.000000,0.000000,0.000000}%
\pgfsetstrokecolor{currentstroke}%
\pgfsetdash{}{0pt}%
\pgfsys@defobject{currentmarker}{\pgfqpoint{0.000000in}{0.000000in}}{\pgfqpoint{0.027778in}{0.000000in}}{%
\pgfpathmoveto{\pgfqpoint{0.000000in}{0.000000in}}%
\pgfpathlineto{\pgfqpoint{0.027778in}{0.000000in}}%
\pgfusepath{stroke,fill}%
}%
\begin{pgfscope}%
\pgfsys@transformshift{4.801389in}{3.181758in}%
\pgfsys@useobject{currentmarker}{}%
\end{pgfscope}%
\end{pgfscope}%
\begin{pgfscope}%
\pgfpathrectangle{\pgfqpoint{0.752778in}{0.582778in}}{\pgfqpoint{4.048611in}{3.212222in}}%
\pgfusepath{clip}%
\pgfsetrectcap%
\pgfsetroundjoin%
\pgfsetlinewidth{0.803000pt}%
\definecolor{currentstroke}{rgb}{0.690196,0.690196,0.690196}%
\pgfsetstrokecolor{currentstroke}%
\pgfsetstrokeopacity{0.300000}%
\pgfsetdash{}{0pt}%
\pgfpathmoveto{\pgfqpoint{0.752778in}{3.225561in}}%
\pgfpathlineto{\pgfqpoint{4.801389in}{3.225561in}}%
\pgfusepath{stroke}%
\end{pgfscope}%
\begin{pgfscope}%
\pgfsetbuttcap%
\pgfsetroundjoin%
\definecolor{currentfill}{rgb}{0.000000,0.000000,0.000000}%
\pgfsetfillcolor{currentfill}%
\pgfsetlinewidth{0.602250pt}%
\definecolor{currentstroke}{rgb}{0.000000,0.000000,0.000000}%
\pgfsetstrokecolor{currentstroke}%
\pgfsetdash{}{0pt}%
\pgfsys@defobject{currentmarker}{\pgfqpoint{-0.027778in}{0.000000in}}{\pgfqpoint{0.000000in}{0.000000in}}{%
\pgfpathmoveto{\pgfqpoint{0.000000in}{0.000000in}}%
\pgfpathlineto{\pgfqpoint{-0.027778in}{0.000000in}}%
\pgfusepath{stroke,fill}%
}%
\begin{pgfscope}%
\pgfsys@transformshift{0.752778in}{3.225561in}%
\pgfsys@useobject{currentmarker}{}%
\end{pgfscope}%
\end{pgfscope}%
\begin{pgfscope}%
\pgfsetbuttcap%
\pgfsetroundjoin%
\definecolor{currentfill}{rgb}{0.000000,0.000000,0.000000}%
\pgfsetfillcolor{currentfill}%
\pgfsetlinewidth{0.602250pt}%
\definecolor{currentstroke}{rgb}{0.000000,0.000000,0.000000}%
\pgfsetstrokecolor{currentstroke}%
\pgfsetdash{}{0pt}%
\pgfsys@defobject{currentmarker}{\pgfqpoint{0.000000in}{0.000000in}}{\pgfqpoint{0.027778in}{0.000000in}}{%
\pgfpathmoveto{\pgfqpoint{0.000000in}{0.000000in}}%
\pgfpathlineto{\pgfqpoint{0.027778in}{0.000000in}}%
\pgfusepath{stroke,fill}%
}%
\begin{pgfscope}%
\pgfsys@transformshift{4.801389in}{3.225561in}%
\pgfsys@useobject{currentmarker}{}%
\end{pgfscope}%
\end{pgfscope}%
\begin{pgfscope}%
\pgfpathrectangle{\pgfqpoint{0.752778in}{0.582778in}}{\pgfqpoint{4.048611in}{3.212222in}}%
\pgfusepath{clip}%
\pgfsetrectcap%
\pgfsetroundjoin%
\pgfsetlinewidth{0.803000pt}%
\definecolor{currentstroke}{rgb}{0.690196,0.690196,0.690196}%
\pgfsetstrokecolor{currentstroke}%
\pgfsetstrokeopacity{0.300000}%
\pgfsetdash{}{0pt}%
\pgfpathmoveto{\pgfqpoint{0.752778in}{3.269364in}}%
\pgfpathlineto{\pgfqpoint{4.801389in}{3.269364in}}%
\pgfusepath{stroke}%
\end{pgfscope}%
\begin{pgfscope}%
\pgfsetbuttcap%
\pgfsetroundjoin%
\definecolor{currentfill}{rgb}{0.000000,0.000000,0.000000}%
\pgfsetfillcolor{currentfill}%
\pgfsetlinewidth{0.602250pt}%
\definecolor{currentstroke}{rgb}{0.000000,0.000000,0.000000}%
\pgfsetstrokecolor{currentstroke}%
\pgfsetdash{}{0pt}%
\pgfsys@defobject{currentmarker}{\pgfqpoint{-0.027778in}{0.000000in}}{\pgfqpoint{0.000000in}{0.000000in}}{%
\pgfpathmoveto{\pgfqpoint{0.000000in}{0.000000in}}%
\pgfpathlineto{\pgfqpoint{-0.027778in}{0.000000in}}%
\pgfusepath{stroke,fill}%
}%
\begin{pgfscope}%
\pgfsys@transformshift{0.752778in}{3.269364in}%
\pgfsys@useobject{currentmarker}{}%
\end{pgfscope}%
\end{pgfscope}%
\begin{pgfscope}%
\pgfsetbuttcap%
\pgfsetroundjoin%
\definecolor{currentfill}{rgb}{0.000000,0.000000,0.000000}%
\pgfsetfillcolor{currentfill}%
\pgfsetlinewidth{0.602250pt}%
\definecolor{currentstroke}{rgb}{0.000000,0.000000,0.000000}%
\pgfsetstrokecolor{currentstroke}%
\pgfsetdash{}{0pt}%
\pgfsys@defobject{currentmarker}{\pgfqpoint{0.000000in}{0.000000in}}{\pgfqpoint{0.027778in}{0.000000in}}{%
\pgfpathmoveto{\pgfqpoint{0.000000in}{0.000000in}}%
\pgfpathlineto{\pgfqpoint{0.027778in}{0.000000in}}%
\pgfusepath{stroke,fill}%
}%
\begin{pgfscope}%
\pgfsys@transformshift{4.801389in}{3.269364in}%
\pgfsys@useobject{currentmarker}{}%
\end{pgfscope}%
\end{pgfscope}%
\begin{pgfscope}%
\pgfpathrectangle{\pgfqpoint{0.752778in}{0.582778in}}{\pgfqpoint{4.048611in}{3.212222in}}%
\pgfusepath{clip}%
\pgfsetrectcap%
\pgfsetroundjoin%
\pgfsetlinewidth{0.803000pt}%
\definecolor{currentstroke}{rgb}{0.690196,0.690196,0.690196}%
\pgfsetstrokecolor{currentstroke}%
\pgfsetstrokeopacity{0.300000}%
\pgfsetdash{}{0pt}%
\pgfpathmoveto{\pgfqpoint{0.752778in}{3.313167in}}%
\pgfpathlineto{\pgfqpoint{4.801389in}{3.313167in}}%
\pgfusepath{stroke}%
\end{pgfscope}%
\begin{pgfscope}%
\pgfsetbuttcap%
\pgfsetroundjoin%
\definecolor{currentfill}{rgb}{0.000000,0.000000,0.000000}%
\pgfsetfillcolor{currentfill}%
\pgfsetlinewidth{0.602250pt}%
\definecolor{currentstroke}{rgb}{0.000000,0.000000,0.000000}%
\pgfsetstrokecolor{currentstroke}%
\pgfsetdash{}{0pt}%
\pgfsys@defobject{currentmarker}{\pgfqpoint{-0.027778in}{0.000000in}}{\pgfqpoint{0.000000in}{0.000000in}}{%
\pgfpathmoveto{\pgfqpoint{0.000000in}{0.000000in}}%
\pgfpathlineto{\pgfqpoint{-0.027778in}{0.000000in}}%
\pgfusepath{stroke,fill}%
}%
\begin{pgfscope}%
\pgfsys@transformshift{0.752778in}{3.313167in}%
\pgfsys@useobject{currentmarker}{}%
\end{pgfscope}%
\end{pgfscope}%
\begin{pgfscope}%
\pgfsetbuttcap%
\pgfsetroundjoin%
\definecolor{currentfill}{rgb}{0.000000,0.000000,0.000000}%
\pgfsetfillcolor{currentfill}%
\pgfsetlinewidth{0.602250pt}%
\definecolor{currentstroke}{rgb}{0.000000,0.000000,0.000000}%
\pgfsetstrokecolor{currentstroke}%
\pgfsetdash{}{0pt}%
\pgfsys@defobject{currentmarker}{\pgfqpoint{0.000000in}{0.000000in}}{\pgfqpoint{0.027778in}{0.000000in}}{%
\pgfpathmoveto{\pgfqpoint{0.000000in}{0.000000in}}%
\pgfpathlineto{\pgfqpoint{0.027778in}{0.000000in}}%
\pgfusepath{stroke,fill}%
}%
\begin{pgfscope}%
\pgfsys@transformshift{4.801389in}{3.313167in}%
\pgfsys@useobject{currentmarker}{}%
\end{pgfscope}%
\end{pgfscope}%
\begin{pgfscope}%
\pgfpathrectangle{\pgfqpoint{0.752778in}{0.582778in}}{\pgfqpoint{4.048611in}{3.212222in}}%
\pgfusepath{clip}%
\pgfsetrectcap%
\pgfsetroundjoin%
\pgfsetlinewidth{0.803000pt}%
\definecolor{currentstroke}{rgb}{0.690196,0.690196,0.690196}%
\pgfsetstrokecolor{currentstroke}%
\pgfsetstrokeopacity{0.300000}%
\pgfsetdash{}{0pt}%
\pgfpathmoveto{\pgfqpoint{0.752778in}{3.400773in}}%
\pgfpathlineto{\pgfqpoint{4.801389in}{3.400773in}}%
\pgfusepath{stroke}%
\end{pgfscope}%
\begin{pgfscope}%
\pgfsetbuttcap%
\pgfsetroundjoin%
\definecolor{currentfill}{rgb}{0.000000,0.000000,0.000000}%
\pgfsetfillcolor{currentfill}%
\pgfsetlinewidth{0.602250pt}%
\definecolor{currentstroke}{rgb}{0.000000,0.000000,0.000000}%
\pgfsetstrokecolor{currentstroke}%
\pgfsetdash{}{0pt}%
\pgfsys@defobject{currentmarker}{\pgfqpoint{-0.027778in}{0.000000in}}{\pgfqpoint{0.000000in}{0.000000in}}{%
\pgfpathmoveto{\pgfqpoint{0.000000in}{0.000000in}}%
\pgfpathlineto{\pgfqpoint{-0.027778in}{0.000000in}}%
\pgfusepath{stroke,fill}%
}%
\begin{pgfscope}%
\pgfsys@transformshift{0.752778in}{3.400773in}%
\pgfsys@useobject{currentmarker}{}%
\end{pgfscope}%
\end{pgfscope}%
\begin{pgfscope}%
\pgfsetbuttcap%
\pgfsetroundjoin%
\definecolor{currentfill}{rgb}{0.000000,0.000000,0.000000}%
\pgfsetfillcolor{currentfill}%
\pgfsetlinewidth{0.602250pt}%
\definecolor{currentstroke}{rgb}{0.000000,0.000000,0.000000}%
\pgfsetstrokecolor{currentstroke}%
\pgfsetdash{}{0pt}%
\pgfsys@defobject{currentmarker}{\pgfqpoint{0.000000in}{0.000000in}}{\pgfqpoint{0.027778in}{0.000000in}}{%
\pgfpathmoveto{\pgfqpoint{0.000000in}{0.000000in}}%
\pgfpathlineto{\pgfqpoint{0.027778in}{0.000000in}}%
\pgfusepath{stroke,fill}%
}%
\begin{pgfscope}%
\pgfsys@transformshift{4.801389in}{3.400773in}%
\pgfsys@useobject{currentmarker}{}%
\end{pgfscope}%
\end{pgfscope}%
\begin{pgfscope}%
\pgfpathrectangle{\pgfqpoint{0.752778in}{0.582778in}}{\pgfqpoint{4.048611in}{3.212222in}}%
\pgfusepath{clip}%
\pgfsetrectcap%
\pgfsetroundjoin%
\pgfsetlinewidth{0.803000pt}%
\definecolor{currentstroke}{rgb}{0.690196,0.690196,0.690196}%
\pgfsetstrokecolor{currentstroke}%
\pgfsetstrokeopacity{0.300000}%
\pgfsetdash{}{0pt}%
\pgfpathmoveto{\pgfqpoint{0.752778in}{3.444576in}}%
\pgfpathlineto{\pgfqpoint{4.801389in}{3.444576in}}%
\pgfusepath{stroke}%
\end{pgfscope}%
\begin{pgfscope}%
\pgfsetbuttcap%
\pgfsetroundjoin%
\definecolor{currentfill}{rgb}{0.000000,0.000000,0.000000}%
\pgfsetfillcolor{currentfill}%
\pgfsetlinewidth{0.602250pt}%
\definecolor{currentstroke}{rgb}{0.000000,0.000000,0.000000}%
\pgfsetstrokecolor{currentstroke}%
\pgfsetdash{}{0pt}%
\pgfsys@defobject{currentmarker}{\pgfqpoint{-0.027778in}{0.000000in}}{\pgfqpoint{0.000000in}{0.000000in}}{%
\pgfpathmoveto{\pgfqpoint{0.000000in}{0.000000in}}%
\pgfpathlineto{\pgfqpoint{-0.027778in}{0.000000in}}%
\pgfusepath{stroke,fill}%
}%
\begin{pgfscope}%
\pgfsys@transformshift{0.752778in}{3.444576in}%
\pgfsys@useobject{currentmarker}{}%
\end{pgfscope}%
\end{pgfscope}%
\begin{pgfscope}%
\pgfsetbuttcap%
\pgfsetroundjoin%
\definecolor{currentfill}{rgb}{0.000000,0.000000,0.000000}%
\pgfsetfillcolor{currentfill}%
\pgfsetlinewidth{0.602250pt}%
\definecolor{currentstroke}{rgb}{0.000000,0.000000,0.000000}%
\pgfsetstrokecolor{currentstroke}%
\pgfsetdash{}{0pt}%
\pgfsys@defobject{currentmarker}{\pgfqpoint{0.000000in}{0.000000in}}{\pgfqpoint{0.027778in}{0.000000in}}{%
\pgfpathmoveto{\pgfqpoint{0.000000in}{0.000000in}}%
\pgfpathlineto{\pgfqpoint{0.027778in}{0.000000in}}%
\pgfusepath{stroke,fill}%
}%
\begin{pgfscope}%
\pgfsys@transformshift{4.801389in}{3.444576in}%
\pgfsys@useobject{currentmarker}{}%
\end{pgfscope}%
\end{pgfscope}%
\begin{pgfscope}%
\pgfpathrectangle{\pgfqpoint{0.752778in}{0.582778in}}{\pgfqpoint{4.048611in}{3.212222in}}%
\pgfusepath{clip}%
\pgfsetrectcap%
\pgfsetroundjoin%
\pgfsetlinewidth{0.803000pt}%
\definecolor{currentstroke}{rgb}{0.690196,0.690196,0.690196}%
\pgfsetstrokecolor{currentstroke}%
\pgfsetstrokeopacity{0.300000}%
\pgfsetdash{}{0pt}%
\pgfpathmoveto{\pgfqpoint{0.752778in}{3.488379in}}%
\pgfpathlineto{\pgfqpoint{4.801389in}{3.488379in}}%
\pgfusepath{stroke}%
\end{pgfscope}%
\begin{pgfscope}%
\pgfsetbuttcap%
\pgfsetroundjoin%
\definecolor{currentfill}{rgb}{0.000000,0.000000,0.000000}%
\pgfsetfillcolor{currentfill}%
\pgfsetlinewidth{0.602250pt}%
\definecolor{currentstroke}{rgb}{0.000000,0.000000,0.000000}%
\pgfsetstrokecolor{currentstroke}%
\pgfsetdash{}{0pt}%
\pgfsys@defobject{currentmarker}{\pgfqpoint{-0.027778in}{0.000000in}}{\pgfqpoint{0.000000in}{0.000000in}}{%
\pgfpathmoveto{\pgfqpoint{0.000000in}{0.000000in}}%
\pgfpathlineto{\pgfqpoint{-0.027778in}{0.000000in}}%
\pgfusepath{stroke,fill}%
}%
\begin{pgfscope}%
\pgfsys@transformshift{0.752778in}{3.488379in}%
\pgfsys@useobject{currentmarker}{}%
\end{pgfscope}%
\end{pgfscope}%
\begin{pgfscope}%
\pgfsetbuttcap%
\pgfsetroundjoin%
\definecolor{currentfill}{rgb}{0.000000,0.000000,0.000000}%
\pgfsetfillcolor{currentfill}%
\pgfsetlinewidth{0.602250pt}%
\definecolor{currentstroke}{rgb}{0.000000,0.000000,0.000000}%
\pgfsetstrokecolor{currentstroke}%
\pgfsetdash{}{0pt}%
\pgfsys@defobject{currentmarker}{\pgfqpoint{0.000000in}{0.000000in}}{\pgfqpoint{0.027778in}{0.000000in}}{%
\pgfpathmoveto{\pgfqpoint{0.000000in}{0.000000in}}%
\pgfpathlineto{\pgfqpoint{0.027778in}{0.000000in}}%
\pgfusepath{stroke,fill}%
}%
\begin{pgfscope}%
\pgfsys@transformshift{4.801389in}{3.488379in}%
\pgfsys@useobject{currentmarker}{}%
\end{pgfscope}%
\end{pgfscope}%
\begin{pgfscope}%
\pgfpathrectangle{\pgfqpoint{0.752778in}{0.582778in}}{\pgfqpoint{4.048611in}{3.212222in}}%
\pgfusepath{clip}%
\pgfsetrectcap%
\pgfsetroundjoin%
\pgfsetlinewidth{0.803000pt}%
\definecolor{currentstroke}{rgb}{0.690196,0.690196,0.690196}%
\pgfsetstrokecolor{currentstroke}%
\pgfsetstrokeopacity{0.300000}%
\pgfsetdash{}{0pt}%
\pgfpathmoveto{\pgfqpoint{0.752778in}{3.532182in}}%
\pgfpathlineto{\pgfqpoint{4.801389in}{3.532182in}}%
\pgfusepath{stroke}%
\end{pgfscope}%
\begin{pgfscope}%
\pgfsetbuttcap%
\pgfsetroundjoin%
\definecolor{currentfill}{rgb}{0.000000,0.000000,0.000000}%
\pgfsetfillcolor{currentfill}%
\pgfsetlinewidth{0.602250pt}%
\definecolor{currentstroke}{rgb}{0.000000,0.000000,0.000000}%
\pgfsetstrokecolor{currentstroke}%
\pgfsetdash{}{0pt}%
\pgfsys@defobject{currentmarker}{\pgfqpoint{-0.027778in}{0.000000in}}{\pgfqpoint{0.000000in}{0.000000in}}{%
\pgfpathmoveto{\pgfqpoint{0.000000in}{0.000000in}}%
\pgfpathlineto{\pgfqpoint{-0.027778in}{0.000000in}}%
\pgfusepath{stroke,fill}%
}%
\begin{pgfscope}%
\pgfsys@transformshift{0.752778in}{3.532182in}%
\pgfsys@useobject{currentmarker}{}%
\end{pgfscope}%
\end{pgfscope}%
\begin{pgfscope}%
\pgfsetbuttcap%
\pgfsetroundjoin%
\definecolor{currentfill}{rgb}{0.000000,0.000000,0.000000}%
\pgfsetfillcolor{currentfill}%
\pgfsetlinewidth{0.602250pt}%
\definecolor{currentstroke}{rgb}{0.000000,0.000000,0.000000}%
\pgfsetstrokecolor{currentstroke}%
\pgfsetdash{}{0pt}%
\pgfsys@defobject{currentmarker}{\pgfqpoint{0.000000in}{0.000000in}}{\pgfqpoint{0.027778in}{0.000000in}}{%
\pgfpathmoveto{\pgfqpoint{0.000000in}{0.000000in}}%
\pgfpathlineto{\pgfqpoint{0.027778in}{0.000000in}}%
\pgfusepath{stroke,fill}%
}%
\begin{pgfscope}%
\pgfsys@transformshift{4.801389in}{3.532182in}%
\pgfsys@useobject{currentmarker}{}%
\end{pgfscope}%
\end{pgfscope}%
\begin{pgfscope}%
\pgfpathrectangle{\pgfqpoint{0.752778in}{0.582778in}}{\pgfqpoint{4.048611in}{3.212222in}}%
\pgfusepath{clip}%
\pgfsetrectcap%
\pgfsetroundjoin%
\pgfsetlinewidth{0.803000pt}%
\definecolor{currentstroke}{rgb}{0.690196,0.690196,0.690196}%
\pgfsetstrokecolor{currentstroke}%
\pgfsetstrokeopacity{0.300000}%
\pgfsetdash{}{0pt}%
\pgfpathmoveto{\pgfqpoint{0.752778in}{3.575985in}}%
\pgfpathlineto{\pgfqpoint{4.801389in}{3.575985in}}%
\pgfusepath{stroke}%
\end{pgfscope}%
\begin{pgfscope}%
\pgfsetbuttcap%
\pgfsetroundjoin%
\definecolor{currentfill}{rgb}{0.000000,0.000000,0.000000}%
\pgfsetfillcolor{currentfill}%
\pgfsetlinewidth{0.602250pt}%
\definecolor{currentstroke}{rgb}{0.000000,0.000000,0.000000}%
\pgfsetstrokecolor{currentstroke}%
\pgfsetdash{}{0pt}%
\pgfsys@defobject{currentmarker}{\pgfqpoint{-0.027778in}{0.000000in}}{\pgfqpoint{0.000000in}{0.000000in}}{%
\pgfpathmoveto{\pgfqpoint{0.000000in}{0.000000in}}%
\pgfpathlineto{\pgfqpoint{-0.027778in}{0.000000in}}%
\pgfusepath{stroke,fill}%
}%
\begin{pgfscope}%
\pgfsys@transformshift{0.752778in}{3.575985in}%
\pgfsys@useobject{currentmarker}{}%
\end{pgfscope}%
\end{pgfscope}%
\begin{pgfscope}%
\pgfsetbuttcap%
\pgfsetroundjoin%
\definecolor{currentfill}{rgb}{0.000000,0.000000,0.000000}%
\pgfsetfillcolor{currentfill}%
\pgfsetlinewidth{0.602250pt}%
\definecolor{currentstroke}{rgb}{0.000000,0.000000,0.000000}%
\pgfsetstrokecolor{currentstroke}%
\pgfsetdash{}{0pt}%
\pgfsys@defobject{currentmarker}{\pgfqpoint{0.000000in}{0.000000in}}{\pgfqpoint{0.027778in}{0.000000in}}{%
\pgfpathmoveto{\pgfqpoint{0.000000in}{0.000000in}}%
\pgfpathlineto{\pgfqpoint{0.027778in}{0.000000in}}%
\pgfusepath{stroke,fill}%
}%
\begin{pgfscope}%
\pgfsys@transformshift{4.801389in}{3.575985in}%
\pgfsys@useobject{currentmarker}{}%
\end{pgfscope}%
\end{pgfscope}%
\begin{pgfscope}%
\pgfpathrectangle{\pgfqpoint{0.752778in}{0.582778in}}{\pgfqpoint{4.048611in}{3.212222in}}%
\pgfusepath{clip}%
\pgfsetrectcap%
\pgfsetroundjoin%
\pgfsetlinewidth{0.803000pt}%
\definecolor{currentstroke}{rgb}{0.690196,0.690196,0.690196}%
\pgfsetstrokecolor{currentstroke}%
\pgfsetstrokeopacity{0.300000}%
\pgfsetdash{}{0pt}%
\pgfpathmoveto{\pgfqpoint{0.752778in}{3.619788in}}%
\pgfpathlineto{\pgfqpoint{4.801389in}{3.619788in}}%
\pgfusepath{stroke}%
\end{pgfscope}%
\begin{pgfscope}%
\pgfsetbuttcap%
\pgfsetroundjoin%
\definecolor{currentfill}{rgb}{0.000000,0.000000,0.000000}%
\pgfsetfillcolor{currentfill}%
\pgfsetlinewidth{0.602250pt}%
\definecolor{currentstroke}{rgb}{0.000000,0.000000,0.000000}%
\pgfsetstrokecolor{currentstroke}%
\pgfsetdash{}{0pt}%
\pgfsys@defobject{currentmarker}{\pgfqpoint{-0.027778in}{0.000000in}}{\pgfqpoint{0.000000in}{0.000000in}}{%
\pgfpathmoveto{\pgfqpoint{0.000000in}{0.000000in}}%
\pgfpathlineto{\pgfqpoint{-0.027778in}{0.000000in}}%
\pgfusepath{stroke,fill}%
}%
\begin{pgfscope}%
\pgfsys@transformshift{0.752778in}{3.619788in}%
\pgfsys@useobject{currentmarker}{}%
\end{pgfscope}%
\end{pgfscope}%
\begin{pgfscope}%
\pgfsetbuttcap%
\pgfsetroundjoin%
\definecolor{currentfill}{rgb}{0.000000,0.000000,0.000000}%
\pgfsetfillcolor{currentfill}%
\pgfsetlinewidth{0.602250pt}%
\definecolor{currentstroke}{rgb}{0.000000,0.000000,0.000000}%
\pgfsetstrokecolor{currentstroke}%
\pgfsetdash{}{0pt}%
\pgfsys@defobject{currentmarker}{\pgfqpoint{0.000000in}{0.000000in}}{\pgfqpoint{0.027778in}{0.000000in}}{%
\pgfpathmoveto{\pgfqpoint{0.000000in}{0.000000in}}%
\pgfpathlineto{\pgfqpoint{0.027778in}{0.000000in}}%
\pgfusepath{stroke,fill}%
}%
\begin{pgfscope}%
\pgfsys@transformshift{4.801389in}{3.619788in}%
\pgfsys@useobject{currentmarker}{}%
\end{pgfscope}%
\end{pgfscope}%
\begin{pgfscope}%
\pgfpathrectangle{\pgfqpoint{0.752778in}{0.582778in}}{\pgfqpoint{4.048611in}{3.212222in}}%
\pgfusepath{clip}%
\pgfsetrectcap%
\pgfsetroundjoin%
\pgfsetlinewidth{0.803000pt}%
\definecolor{currentstroke}{rgb}{0.690196,0.690196,0.690196}%
\pgfsetstrokecolor{currentstroke}%
\pgfsetstrokeopacity{0.300000}%
\pgfsetdash{}{0pt}%
\pgfpathmoveto{\pgfqpoint{0.752778in}{3.663591in}}%
\pgfpathlineto{\pgfqpoint{4.801389in}{3.663591in}}%
\pgfusepath{stroke}%
\end{pgfscope}%
\begin{pgfscope}%
\pgfsetbuttcap%
\pgfsetroundjoin%
\definecolor{currentfill}{rgb}{0.000000,0.000000,0.000000}%
\pgfsetfillcolor{currentfill}%
\pgfsetlinewidth{0.602250pt}%
\definecolor{currentstroke}{rgb}{0.000000,0.000000,0.000000}%
\pgfsetstrokecolor{currentstroke}%
\pgfsetdash{}{0pt}%
\pgfsys@defobject{currentmarker}{\pgfqpoint{-0.027778in}{0.000000in}}{\pgfqpoint{0.000000in}{0.000000in}}{%
\pgfpathmoveto{\pgfqpoint{0.000000in}{0.000000in}}%
\pgfpathlineto{\pgfqpoint{-0.027778in}{0.000000in}}%
\pgfusepath{stroke,fill}%
}%
\begin{pgfscope}%
\pgfsys@transformshift{0.752778in}{3.663591in}%
\pgfsys@useobject{currentmarker}{}%
\end{pgfscope}%
\end{pgfscope}%
\begin{pgfscope}%
\pgfsetbuttcap%
\pgfsetroundjoin%
\definecolor{currentfill}{rgb}{0.000000,0.000000,0.000000}%
\pgfsetfillcolor{currentfill}%
\pgfsetlinewidth{0.602250pt}%
\definecolor{currentstroke}{rgb}{0.000000,0.000000,0.000000}%
\pgfsetstrokecolor{currentstroke}%
\pgfsetdash{}{0pt}%
\pgfsys@defobject{currentmarker}{\pgfqpoint{0.000000in}{0.000000in}}{\pgfqpoint{0.027778in}{0.000000in}}{%
\pgfpathmoveto{\pgfqpoint{0.000000in}{0.000000in}}%
\pgfpathlineto{\pgfqpoint{0.027778in}{0.000000in}}%
\pgfusepath{stroke,fill}%
}%
\begin{pgfscope}%
\pgfsys@transformshift{4.801389in}{3.663591in}%
\pgfsys@useobject{currentmarker}{}%
\end{pgfscope}%
\end{pgfscope}%
\begin{pgfscope}%
\pgfpathrectangle{\pgfqpoint{0.752778in}{0.582778in}}{\pgfqpoint{4.048611in}{3.212222in}}%
\pgfusepath{clip}%
\pgfsetrectcap%
\pgfsetroundjoin%
\pgfsetlinewidth{0.803000pt}%
\definecolor{currentstroke}{rgb}{0.690196,0.690196,0.690196}%
\pgfsetstrokecolor{currentstroke}%
\pgfsetstrokeopacity{0.300000}%
\pgfsetdash{}{0pt}%
\pgfpathmoveto{\pgfqpoint{0.752778in}{3.707394in}}%
\pgfpathlineto{\pgfqpoint{4.801389in}{3.707394in}}%
\pgfusepath{stroke}%
\end{pgfscope}%
\begin{pgfscope}%
\pgfsetbuttcap%
\pgfsetroundjoin%
\definecolor{currentfill}{rgb}{0.000000,0.000000,0.000000}%
\pgfsetfillcolor{currentfill}%
\pgfsetlinewidth{0.602250pt}%
\definecolor{currentstroke}{rgb}{0.000000,0.000000,0.000000}%
\pgfsetstrokecolor{currentstroke}%
\pgfsetdash{}{0pt}%
\pgfsys@defobject{currentmarker}{\pgfqpoint{-0.027778in}{0.000000in}}{\pgfqpoint{0.000000in}{0.000000in}}{%
\pgfpathmoveto{\pgfqpoint{0.000000in}{0.000000in}}%
\pgfpathlineto{\pgfqpoint{-0.027778in}{0.000000in}}%
\pgfusepath{stroke,fill}%
}%
\begin{pgfscope}%
\pgfsys@transformshift{0.752778in}{3.707394in}%
\pgfsys@useobject{currentmarker}{}%
\end{pgfscope}%
\end{pgfscope}%
\begin{pgfscope}%
\pgfsetbuttcap%
\pgfsetroundjoin%
\definecolor{currentfill}{rgb}{0.000000,0.000000,0.000000}%
\pgfsetfillcolor{currentfill}%
\pgfsetlinewidth{0.602250pt}%
\definecolor{currentstroke}{rgb}{0.000000,0.000000,0.000000}%
\pgfsetstrokecolor{currentstroke}%
\pgfsetdash{}{0pt}%
\pgfsys@defobject{currentmarker}{\pgfqpoint{0.000000in}{0.000000in}}{\pgfqpoint{0.027778in}{0.000000in}}{%
\pgfpathmoveto{\pgfqpoint{0.000000in}{0.000000in}}%
\pgfpathlineto{\pgfqpoint{0.027778in}{0.000000in}}%
\pgfusepath{stroke,fill}%
}%
\begin{pgfscope}%
\pgfsys@transformshift{4.801389in}{3.707394in}%
\pgfsys@useobject{currentmarker}{}%
\end{pgfscope}%
\end{pgfscope}%
\begin{pgfscope}%
\pgfpathrectangle{\pgfqpoint{0.752778in}{0.582778in}}{\pgfqpoint{4.048611in}{3.212222in}}%
\pgfusepath{clip}%
\pgfsetrectcap%
\pgfsetroundjoin%
\pgfsetlinewidth{0.803000pt}%
\definecolor{currentstroke}{rgb}{0.690196,0.690196,0.690196}%
\pgfsetstrokecolor{currentstroke}%
\pgfsetstrokeopacity{0.300000}%
\pgfsetdash{}{0pt}%
\pgfpathmoveto{\pgfqpoint{0.752778in}{3.751197in}}%
\pgfpathlineto{\pgfqpoint{4.801389in}{3.751197in}}%
\pgfusepath{stroke}%
\end{pgfscope}%
\begin{pgfscope}%
\pgfsetbuttcap%
\pgfsetroundjoin%
\definecolor{currentfill}{rgb}{0.000000,0.000000,0.000000}%
\pgfsetfillcolor{currentfill}%
\pgfsetlinewidth{0.602250pt}%
\definecolor{currentstroke}{rgb}{0.000000,0.000000,0.000000}%
\pgfsetstrokecolor{currentstroke}%
\pgfsetdash{}{0pt}%
\pgfsys@defobject{currentmarker}{\pgfqpoint{-0.027778in}{0.000000in}}{\pgfqpoint{0.000000in}{0.000000in}}{%
\pgfpathmoveto{\pgfqpoint{0.000000in}{0.000000in}}%
\pgfpathlineto{\pgfqpoint{-0.027778in}{0.000000in}}%
\pgfusepath{stroke,fill}%
}%
\begin{pgfscope}%
\pgfsys@transformshift{0.752778in}{3.751197in}%
\pgfsys@useobject{currentmarker}{}%
\end{pgfscope}%
\end{pgfscope}%
\begin{pgfscope}%
\pgfsetbuttcap%
\pgfsetroundjoin%
\definecolor{currentfill}{rgb}{0.000000,0.000000,0.000000}%
\pgfsetfillcolor{currentfill}%
\pgfsetlinewidth{0.602250pt}%
\definecolor{currentstroke}{rgb}{0.000000,0.000000,0.000000}%
\pgfsetstrokecolor{currentstroke}%
\pgfsetdash{}{0pt}%
\pgfsys@defobject{currentmarker}{\pgfqpoint{0.000000in}{0.000000in}}{\pgfqpoint{0.027778in}{0.000000in}}{%
\pgfpathmoveto{\pgfqpoint{0.000000in}{0.000000in}}%
\pgfpathlineto{\pgfqpoint{0.027778in}{0.000000in}}%
\pgfusepath{stroke,fill}%
}%
\begin{pgfscope}%
\pgfsys@transformshift{4.801389in}{3.751197in}%
\pgfsys@useobject{currentmarker}{}%
\end{pgfscope}%
\end{pgfscope}%
\begin{pgfscope}%
\definecolor{textcolor}{rgb}{0.000000,0.000000,0.000000}%
\pgfsetstrokecolor{textcolor}%
\pgfsetfillcolor{textcolor}%
\pgftext[x=0.290755in,y=2.188889in,,bottom,rotate=90.000000]{\color{textcolor}\sffamily\fontsize{10.000000}{12.000000}\selectfont Normierte Zaehlrate}%
\end{pgfscope}%
\begin{pgfscope}%
\pgfpathrectangle{\pgfqpoint{0.752778in}{0.582778in}}{\pgfqpoint{4.048611in}{3.212222in}}%
\pgfusepath{clip}%
\pgfsetrectcap%
\pgfsetroundjoin%
\pgfsetlinewidth{1.505625pt}%
\definecolor{currentstroke}{rgb}{0.121569,0.466667,0.705882}%
\pgfsetstrokecolor{currentstroke}%
\pgfsetdash{}{0pt}%
\pgfpathmoveto{\pgfqpoint{0.936806in}{0.728788in}}%
\pgfpathlineto{\pgfqpoint{0.937004in}{0.728788in}}%
\pgfpathlineto{\pgfqpoint{0.937004in}{0.759313in}}%
\pgfpathlineto{\pgfqpoint{0.937997in}{0.749138in}}%
\pgfpathlineto{\pgfqpoint{0.938394in}{0.749138in}}%
\pgfpathlineto{\pgfqpoint{0.939188in}{0.728788in}}%
\pgfpathlineto{\pgfqpoint{0.938592in}{0.759313in}}%
\pgfpathlineto{\pgfqpoint{0.939386in}{0.749138in}}%
\pgfpathlineto{\pgfqpoint{0.939585in}{0.749138in}}%
\pgfpathlineto{\pgfqpoint{0.939982in}{0.779662in}}%
\pgfpathlineto{\pgfqpoint{0.939783in}{0.728788in}}%
\pgfpathlineto{\pgfqpoint{0.940577in}{0.738963in}}%
\pgfpathlineto{\pgfqpoint{0.940974in}{0.738963in}}%
\pgfpathlineto{\pgfqpoint{0.940974in}{0.728788in}}%
\pgfpathlineto{\pgfqpoint{0.941371in}{0.749138in}}%
\pgfpathlineto{\pgfqpoint{0.941966in}{0.749138in}}%
\pgfpathlineto{\pgfqpoint{0.942165in}{0.749138in}}%
\pgfpathlineto{\pgfqpoint{0.942562in}{0.728788in}}%
\pgfpathlineto{\pgfqpoint{0.942959in}{0.759313in}}%
\pgfpathlineto{\pgfqpoint{0.943157in}{0.749138in}}%
\pgfpathlineto{\pgfqpoint{0.943554in}{0.749138in}}%
\pgfpathlineto{\pgfqpoint{0.944348in}{0.728788in}}%
\pgfpathlineto{\pgfqpoint{0.944547in}{0.738963in}}%
\pgfpathlineto{\pgfqpoint{0.944944in}{0.738963in}}%
\pgfpathlineto{\pgfqpoint{0.944944in}{0.749138in}}%
\pgfpathlineto{\pgfqpoint{0.945341in}{0.728788in}}%
\pgfpathlineto{\pgfqpoint{0.945936in}{0.738963in}}%
\pgfpathlineto{\pgfqpoint{0.946333in}{0.738963in}}%
\pgfpathlineto{\pgfqpoint{0.946730in}{0.759313in}}%
\pgfpathlineto{\pgfqpoint{0.946532in}{0.728788in}}%
\pgfpathlineto{\pgfqpoint{0.947326in}{0.728788in}}%
\pgfpathlineto{\pgfqpoint{0.947524in}{0.728788in}}%
\pgfpathlineto{\pgfqpoint{0.947524in}{0.749138in}}%
\pgfpathlineto{\pgfqpoint{0.948517in}{0.738963in}}%
\pgfpathlineto{\pgfqpoint{0.948914in}{0.738963in}}%
\pgfpathlineto{\pgfqpoint{0.948914in}{0.749138in}}%
\pgfpathlineto{\pgfqpoint{0.949112in}{0.728788in}}%
\pgfpathlineto{\pgfqpoint{0.949906in}{0.728788in}}%
\pgfpathlineto{\pgfqpoint{0.950700in}{0.728788in}}%
\pgfpathlineto{\pgfqpoint{0.950700in}{0.738963in}}%
\pgfpathlineto{\pgfqpoint{0.951693in}{0.728788in}}%
\pgfpathlineto{\pgfqpoint{0.952090in}{0.728788in}}%
\pgfpathlineto{\pgfqpoint{0.952884in}{0.749138in}}%
\pgfpathlineto{\pgfqpoint{0.953082in}{0.749138in}}%
\pgfpathlineto{\pgfqpoint{0.953678in}{0.749138in}}%
\pgfpathlineto{\pgfqpoint{0.953678in}{0.728788in}}%
\pgfpathlineto{\pgfqpoint{0.954670in}{0.728788in}}%
\pgfpathlineto{\pgfqpoint{0.954869in}{0.728788in}}%
\pgfpathlineto{\pgfqpoint{0.955067in}{0.749138in}}%
\pgfpathlineto{\pgfqpoint{0.955861in}{0.738963in}}%
\pgfpathlineto{\pgfqpoint{0.956060in}{0.738963in}}%
\pgfpathlineto{\pgfqpoint{0.956060in}{0.728788in}}%
\pgfpathlineto{\pgfqpoint{0.956457in}{0.749138in}}%
\pgfpathlineto{\pgfqpoint{0.957052in}{0.728788in}}%
\pgfpathlineto{\pgfqpoint{0.957846in}{0.728788in}}%
\pgfpathlineto{\pgfqpoint{0.957846in}{0.738963in}}%
\pgfpathlineto{\pgfqpoint{0.958839in}{0.728788in}}%
\pgfpathlineto{\pgfqpoint{0.959037in}{0.728788in}}%
\pgfpathlineto{\pgfqpoint{0.959037in}{0.749138in}}%
\pgfpathlineto{\pgfqpoint{0.960030in}{0.738963in}}%
\pgfpathlineto{\pgfqpoint{0.960228in}{0.738963in}}%
\pgfpathlineto{\pgfqpoint{0.960824in}{0.769488in}}%
\pgfpathlineto{\pgfqpoint{0.961221in}{0.769488in}}%
\pgfpathlineto{\pgfqpoint{0.961419in}{0.769488in}}%
\pgfpathlineto{\pgfqpoint{0.961618in}{0.728788in}}%
\pgfpathlineto{\pgfqpoint{0.962412in}{0.749138in}}%
\pgfpathlineto{\pgfqpoint{0.962809in}{0.749138in}}%
\pgfpathlineto{\pgfqpoint{0.962809in}{0.738963in}}%
\pgfpathlineto{\pgfqpoint{0.963206in}{0.769488in}}%
\pgfpathlineto{\pgfqpoint{0.963801in}{0.759313in}}%
\pgfpathlineto{\pgfqpoint{0.964198in}{0.759313in}}%
\pgfpathlineto{\pgfqpoint{0.964198in}{0.779662in}}%
\pgfpathlineto{\pgfqpoint{0.964794in}{0.738963in}}%
\pgfpathlineto{\pgfqpoint{0.965191in}{0.738963in}}%
\pgfpathlineto{\pgfqpoint{0.965389in}{0.738963in}}%
\pgfpathlineto{\pgfqpoint{0.965786in}{0.728788in}}%
\pgfpathlineto{\pgfqpoint{0.966382in}{0.769488in}}%
\pgfpathlineto{\pgfqpoint{0.966580in}{0.769488in}}%
\pgfpathlineto{\pgfqpoint{0.966580in}{0.738963in}}%
\pgfpathlineto{\pgfqpoint{0.967573in}{0.749138in}}%
\pgfpathlineto{\pgfqpoint{0.967771in}{0.749138in}}%
\pgfpathlineto{\pgfqpoint{0.967771in}{0.738963in}}%
\pgfpathlineto{\pgfqpoint{0.967970in}{0.779662in}}%
\pgfpathlineto{\pgfqpoint{0.968764in}{0.749138in}}%
\pgfpathlineto{\pgfqpoint{0.968962in}{0.749138in}}%
\pgfpathlineto{\pgfqpoint{0.969359in}{0.728788in}}%
\pgfpathlineto{\pgfqpoint{0.969161in}{0.759313in}}%
\pgfpathlineto{\pgfqpoint{0.969955in}{0.749138in}}%
\pgfpathlineto{\pgfqpoint{0.970153in}{0.749138in}}%
\pgfpathlineto{\pgfqpoint{0.970749in}{0.728788in}}%
\pgfpathlineto{\pgfqpoint{0.970352in}{0.769488in}}%
\pgfpathlineto{\pgfqpoint{0.971146in}{0.769488in}}%
\pgfpathlineto{\pgfqpoint{0.971344in}{0.769488in}}%
\pgfpathlineto{\pgfqpoint{0.971741in}{0.738963in}}%
\pgfpathlineto{\pgfqpoint{0.972337in}{0.749138in}}%
\pgfpathlineto{\pgfqpoint{0.972733in}{0.749138in}}%
\pgfpathlineto{\pgfqpoint{0.973130in}{0.800012in}}%
\pgfpathlineto{\pgfqpoint{0.972932in}{0.728788in}}%
\pgfpathlineto{\pgfqpoint{0.973726in}{0.769488in}}%
\pgfpathlineto{\pgfqpoint{0.973924in}{0.769488in}}%
\pgfpathlineto{\pgfqpoint{0.973924in}{0.749138in}}%
\pgfpathlineto{\pgfqpoint{0.974321in}{0.789837in}}%
\pgfpathlineto{\pgfqpoint{0.974917in}{0.789837in}}%
\pgfpathlineto{\pgfqpoint{0.975115in}{0.789837in}}%
\pgfpathlineto{\pgfqpoint{0.975512in}{0.728788in}}%
\pgfpathlineto{\pgfqpoint{0.976108in}{0.728788in}}%
\pgfpathlineto{\pgfqpoint{0.976306in}{0.728788in}}%
\pgfpathlineto{\pgfqpoint{0.977100in}{0.779662in}}%
\pgfpathlineto{\pgfqpoint{0.977299in}{0.749138in}}%
\pgfpathlineto{\pgfqpoint{0.977497in}{0.749138in}}%
\pgfpathlineto{\pgfqpoint{0.978093in}{0.728788in}}%
\pgfpathlineto{\pgfqpoint{0.977696in}{0.789837in}}%
\pgfpathlineto{\pgfqpoint{0.978490in}{0.738963in}}%
\pgfpathlineto{\pgfqpoint{0.978688in}{0.738963in}}%
\pgfpathlineto{\pgfqpoint{0.979085in}{0.789837in}}%
\pgfpathlineto{\pgfqpoint{0.979681in}{0.789837in}}%
\pgfpathlineto{\pgfqpoint{0.979879in}{0.789837in}}%
\pgfpathlineto{\pgfqpoint{0.979879in}{0.749138in}}%
\pgfpathlineto{\pgfqpoint{0.980872in}{0.779662in}}%
\pgfpathlineto{\pgfqpoint{0.981070in}{0.779662in}}%
\pgfpathlineto{\pgfqpoint{0.981269in}{0.749138in}}%
\pgfpathlineto{\pgfqpoint{0.982063in}{0.769488in}}%
\pgfpathlineto{\pgfqpoint{0.982261in}{0.769488in}}%
\pgfpathlineto{\pgfqpoint{0.982658in}{0.789837in}}%
\pgfpathlineto{\pgfqpoint{0.982857in}{0.738963in}}%
\pgfpathlineto{\pgfqpoint{0.983254in}{0.769488in}}%
\pgfpathlineto{\pgfqpoint{0.983651in}{0.769488in}}%
\pgfpathlineto{\pgfqpoint{0.983651in}{0.800012in}}%
\pgfpathlineto{\pgfqpoint{0.984048in}{0.738963in}}%
\pgfpathlineto{\pgfqpoint{0.984643in}{0.749138in}}%
\pgfpathlineto{\pgfqpoint{0.984842in}{0.749138in}}%
\pgfpathlineto{\pgfqpoint{0.985636in}{0.810187in}}%
\pgfpathlineto{\pgfqpoint{0.985834in}{0.738963in}}%
\pgfpathlineto{\pgfqpoint{0.986033in}{0.738963in}}%
\pgfpathlineto{\pgfqpoint{0.986231in}{0.800012in}}%
\pgfpathlineto{\pgfqpoint{0.987025in}{0.749138in}}%
\pgfpathlineto{\pgfqpoint{0.987224in}{0.749138in}}%
\pgfpathlineto{\pgfqpoint{0.987621in}{0.820362in}}%
\pgfpathlineto{\pgfqpoint{0.988216in}{0.769488in}}%
\pgfpathlineto{\pgfqpoint{0.988415in}{0.769488in}}%
\pgfpathlineto{\pgfqpoint{0.988415in}{0.759313in}}%
\pgfpathlineto{\pgfqpoint{0.989209in}{0.810187in}}%
\pgfpathlineto{\pgfqpoint{0.989407in}{0.789837in}}%
\pgfpathlineto{\pgfqpoint{0.989606in}{0.789837in}}%
\pgfpathlineto{\pgfqpoint{0.989606in}{0.759313in}}%
\pgfpathlineto{\pgfqpoint{0.990201in}{0.810187in}}%
\pgfpathlineto{\pgfqpoint{0.990598in}{0.779662in}}%
\pgfpathlineto{\pgfqpoint{0.990797in}{0.779662in}}%
\pgfpathlineto{\pgfqpoint{0.991392in}{0.820362in}}%
\pgfpathlineto{\pgfqpoint{0.990995in}{0.759313in}}%
\pgfpathlineto{\pgfqpoint{0.991789in}{0.769488in}}%
\pgfpathlineto{\pgfqpoint{0.991988in}{0.769488in}}%
\pgfpathlineto{\pgfqpoint{0.991988in}{0.759313in}}%
\pgfpathlineto{\pgfqpoint{0.992385in}{0.779662in}}%
\pgfpathlineto{\pgfqpoint{0.992980in}{0.779662in}}%
\pgfpathlineto{\pgfqpoint{0.993179in}{0.779662in}}%
\pgfpathlineto{\pgfqpoint{0.993179in}{0.850887in}}%
\pgfpathlineto{\pgfqpoint{0.993576in}{0.738963in}}%
\pgfpathlineto{\pgfqpoint{0.994171in}{0.738963in}}%
\pgfpathlineto{\pgfqpoint{0.994370in}{0.738963in}}%
\pgfpathlineto{\pgfqpoint{0.995362in}{0.861062in}}%
\pgfpathlineto{\pgfqpoint{0.995561in}{0.861062in}}%
\pgfpathlineto{\pgfqpoint{0.996156in}{0.759313in}}%
\pgfpathlineto{\pgfqpoint{0.996553in}{0.789837in}}%
\pgfpathlineto{\pgfqpoint{0.996752in}{0.789837in}}%
\pgfpathlineto{\pgfqpoint{0.997149in}{0.810187in}}%
\pgfpathlineto{\pgfqpoint{0.997744in}{0.759313in}}%
\pgfpathlineto{\pgfqpoint{0.997943in}{0.759313in}}%
\pgfpathlineto{\pgfqpoint{0.998737in}{0.840712in}}%
\pgfpathlineto{\pgfqpoint{0.998935in}{0.789837in}}%
\pgfpathlineto{\pgfqpoint{0.999134in}{0.789837in}}%
\pgfpathlineto{\pgfqpoint{0.999531in}{0.820362in}}%
\pgfpathlineto{\pgfqpoint{1.000126in}{0.820362in}}%
\pgfpathlineto{\pgfqpoint{1.000325in}{0.820362in}}%
\pgfpathlineto{\pgfqpoint{1.000920in}{0.779662in}}%
\pgfpathlineto{\pgfqpoint{1.001119in}{0.830537in}}%
\pgfpathlineto{\pgfqpoint{1.001317in}{0.820362in}}%
\pgfpathlineto{\pgfqpoint{1.001516in}{0.820362in}}%
\pgfpathlineto{\pgfqpoint{1.002310in}{0.789837in}}%
\pgfpathlineto{\pgfqpoint{1.001714in}{0.861062in}}%
\pgfpathlineto{\pgfqpoint{1.002508in}{0.800012in}}%
\pgfpathlineto{\pgfqpoint{1.002707in}{0.800012in}}%
\pgfpathlineto{\pgfqpoint{1.003302in}{0.871237in}}%
\pgfpathlineto{\pgfqpoint{1.003699in}{0.779662in}}%
\pgfpathlineto{\pgfqpoint{1.003898in}{0.779662in}}%
\pgfpathlineto{\pgfqpoint{1.004493in}{0.861062in}}%
\pgfpathlineto{\pgfqpoint{1.004295in}{0.759313in}}%
\pgfpathlineto{\pgfqpoint{1.004890in}{0.840712in}}%
\pgfpathlineto{\pgfqpoint{1.005088in}{0.840712in}}%
\pgfpathlineto{\pgfqpoint{1.005287in}{0.800012in}}%
\pgfpathlineto{\pgfqpoint{1.005882in}{0.861062in}}%
\pgfpathlineto{\pgfqpoint{1.006081in}{0.830537in}}%
\pgfpathlineto{\pgfqpoint{1.006279in}{0.830537in}}%
\pgfpathlineto{\pgfqpoint{1.006875in}{0.789837in}}%
\pgfpathlineto{\pgfqpoint{1.007272in}{0.901762in}}%
\pgfpathlineto{\pgfqpoint{1.007470in}{0.901762in}}%
\pgfpathlineto{\pgfqpoint{1.007470in}{0.779662in}}%
\pgfpathlineto{\pgfqpoint{1.008463in}{0.800012in}}%
\pgfpathlineto{\pgfqpoint{1.008661in}{0.800012in}}%
\pgfpathlineto{\pgfqpoint{1.008661in}{0.891587in}}%
\pgfpathlineto{\pgfqpoint{1.009654in}{0.820362in}}%
\pgfpathlineto{\pgfqpoint{1.009852in}{0.820362in}}%
\pgfpathlineto{\pgfqpoint{1.009852in}{0.810187in}}%
\pgfpathlineto{\pgfqpoint{1.010249in}{0.850887in}}%
\pgfpathlineto{\pgfqpoint{1.010845in}{0.810187in}}%
\pgfpathlineto{\pgfqpoint{1.011043in}{0.810187in}}%
\pgfpathlineto{\pgfqpoint{1.011440in}{0.850887in}}%
\pgfpathlineto{\pgfqpoint{1.011639in}{0.769488in}}%
\pgfpathlineto{\pgfqpoint{1.012036in}{0.830537in}}%
\pgfpathlineto{\pgfqpoint{1.012234in}{0.830537in}}%
\pgfpathlineto{\pgfqpoint{1.013227in}{0.942461in}}%
\pgfpathlineto{\pgfqpoint{1.013425in}{0.942461in}}%
\pgfpathlineto{\pgfqpoint{1.014219in}{0.820362in}}%
\pgfpathlineto{\pgfqpoint{1.014418in}{0.911936in}}%
\pgfpathlineto{\pgfqpoint{1.014616in}{0.911936in}}%
\pgfpathlineto{\pgfqpoint{1.015609in}{0.800012in}}%
\pgfpathlineto{\pgfqpoint{1.015807in}{0.800012in}}%
\pgfpathlineto{\pgfqpoint{1.016800in}{0.932286in}}%
\pgfpathlineto{\pgfqpoint{1.016998in}{0.932286in}}%
\pgfpathlineto{\pgfqpoint{1.017991in}{0.769488in}}%
\pgfpathlineto{\pgfqpoint{1.018189in}{0.769488in}}%
\pgfpathlineto{\pgfqpoint{1.018983in}{0.901762in}}%
\pgfpathlineto{\pgfqpoint{1.019182in}{0.871237in}}%
\pgfpathlineto{\pgfqpoint{1.019380in}{0.871237in}}%
\pgfpathlineto{\pgfqpoint{1.019777in}{0.830537in}}%
\pgfpathlineto{\pgfqpoint{1.019579in}{0.932286in}}%
\pgfpathlineto{\pgfqpoint{1.020373in}{0.901762in}}%
\pgfpathlineto{\pgfqpoint{1.020571in}{0.901762in}}%
\pgfpathlineto{\pgfqpoint{1.020571in}{0.830537in}}%
\pgfpathlineto{\pgfqpoint{1.021167in}{0.952636in}}%
\pgfpathlineto{\pgfqpoint{1.021564in}{0.861062in}}%
\pgfpathlineto{\pgfqpoint{1.021762in}{0.861062in}}%
\pgfpathlineto{\pgfqpoint{1.021762in}{0.881412in}}%
\pgfpathlineto{\pgfqpoint{1.021961in}{0.830537in}}%
\pgfpathlineto{\pgfqpoint{1.022755in}{0.840712in}}%
\pgfpathlineto{\pgfqpoint{1.022953in}{0.840712in}}%
\pgfpathlineto{\pgfqpoint{1.023152in}{0.891587in}}%
\pgfpathlineto{\pgfqpoint{1.023350in}{0.820362in}}%
\pgfpathlineto{\pgfqpoint{1.023946in}{0.861062in}}%
\pgfpathlineto{\pgfqpoint{1.024144in}{0.861062in}}%
\pgfpathlineto{\pgfqpoint{1.024144in}{0.789837in}}%
\pgfpathlineto{\pgfqpoint{1.024343in}{0.922111in}}%
\pgfpathlineto{\pgfqpoint{1.025137in}{0.871237in}}%
\pgfpathlineto{\pgfqpoint{1.025335in}{0.871237in}}%
\pgfpathlineto{\pgfqpoint{1.025931in}{0.820362in}}%
\pgfpathlineto{\pgfqpoint{1.026129in}{0.942461in}}%
\pgfpathlineto{\pgfqpoint{1.026328in}{0.820362in}}%
\pgfpathlineto{\pgfqpoint{1.026526in}{0.820362in}}%
\pgfpathlineto{\pgfqpoint{1.026725in}{0.922111in}}%
\pgfpathlineto{\pgfqpoint{1.026923in}{0.810187in}}%
\pgfpathlineto{\pgfqpoint{1.027519in}{0.840712in}}%
\pgfpathlineto{\pgfqpoint{1.027717in}{0.840712in}}%
\pgfpathlineto{\pgfqpoint{1.028313in}{0.911936in}}%
\pgfpathlineto{\pgfqpoint{1.028710in}{0.850887in}}%
\pgfpathlineto{\pgfqpoint{1.028908in}{0.850887in}}%
\pgfpathlineto{\pgfqpoint{1.028908in}{0.840712in}}%
\pgfpathlineto{\pgfqpoint{1.029305in}{0.983161in}}%
\pgfpathlineto{\pgfqpoint{1.029901in}{0.962811in}}%
\pgfpathlineto{\pgfqpoint{1.030099in}{0.962811in}}%
\pgfpathlineto{\pgfqpoint{1.030298in}{0.800012in}}%
\pgfpathlineto{\pgfqpoint{1.031092in}{0.871237in}}%
\pgfpathlineto{\pgfqpoint{1.031290in}{0.871237in}}%
\pgfpathlineto{\pgfqpoint{1.031290in}{0.901762in}}%
\pgfpathlineto{\pgfqpoint{1.032283in}{0.820362in}}%
\pgfpathlineto{\pgfqpoint{1.032481in}{0.820362in}}%
\pgfpathlineto{\pgfqpoint{1.033275in}{0.983161in}}%
\pgfpathlineto{\pgfqpoint{1.033474in}{0.922111in}}%
\pgfpathlineto{\pgfqpoint{1.033672in}{0.922111in}}%
\pgfpathlineto{\pgfqpoint{1.033672in}{0.850887in}}%
\pgfpathlineto{\pgfqpoint{1.034665in}{1.013686in}}%
\pgfpathlineto{\pgfqpoint{1.034863in}{1.013686in}}%
\pgfpathlineto{\pgfqpoint{1.034863in}{0.891587in}}%
\pgfpathlineto{\pgfqpoint{1.035856in}{0.942461in}}%
\pgfpathlineto{\pgfqpoint{1.036054in}{0.942461in}}%
\pgfpathlineto{\pgfqpoint{1.036451in}{0.891587in}}%
\pgfpathlineto{\pgfqpoint{1.036253in}{1.013686in}}%
\pgfpathlineto{\pgfqpoint{1.037047in}{0.922111in}}%
\pgfpathlineto{\pgfqpoint{1.037245in}{0.922111in}}%
\pgfpathlineto{\pgfqpoint{1.037840in}{0.993336in}}%
\pgfpathlineto{\pgfqpoint{1.038039in}{0.871237in}}%
\pgfpathlineto{\pgfqpoint{1.038237in}{0.911936in}}%
\pgfpathlineto{\pgfqpoint{1.038436in}{0.911936in}}%
\pgfpathlineto{\pgfqpoint{1.038634in}{0.881412in}}%
\pgfpathlineto{\pgfqpoint{1.039428in}{0.983161in}}%
\pgfpathlineto{\pgfqpoint{1.039627in}{0.983161in}}%
\pgfpathlineto{\pgfqpoint{1.039627in}{0.901762in}}%
\pgfpathlineto{\pgfqpoint{1.040619in}{0.911936in}}%
\pgfpathlineto{\pgfqpoint{1.041016in}{0.911936in}}%
\pgfpathlineto{\pgfqpoint{1.041016in}{0.881412in}}%
\pgfpathlineto{\pgfqpoint{1.042009in}{0.932286in}}%
\pgfpathlineto{\pgfqpoint{1.042207in}{0.932286in}}%
\pgfpathlineto{\pgfqpoint{1.042406in}{0.881412in}}%
\pgfpathlineto{\pgfqpoint{1.043001in}{0.983161in}}%
\pgfpathlineto{\pgfqpoint{1.043200in}{0.922111in}}%
\pgfpathlineto{\pgfqpoint{1.043398in}{0.922111in}}%
\pgfpathlineto{\pgfqpoint{1.043994in}{1.023861in}}%
\pgfpathlineto{\pgfqpoint{1.043597in}{0.871237in}}%
\pgfpathlineto{\pgfqpoint{1.044391in}{0.972986in}}%
\pgfpathlineto{\pgfqpoint{1.044589in}{0.972986in}}%
\pgfpathlineto{\pgfqpoint{1.045185in}{0.850887in}}%
\pgfpathlineto{\pgfqpoint{1.045582in}{0.891587in}}%
\pgfpathlineto{\pgfqpoint{1.045780in}{0.891587in}}%
\pgfpathlineto{\pgfqpoint{1.046376in}{0.993336in}}%
\pgfpathlineto{\pgfqpoint{1.046574in}{0.861062in}}%
\pgfpathlineto{\pgfqpoint{1.046773in}{0.983161in}}%
\pgfpathlineto{\pgfqpoint{1.046971in}{0.983161in}}%
\pgfpathlineto{\pgfqpoint{1.047567in}{1.013686in}}%
\pgfpathlineto{\pgfqpoint{1.047964in}{0.861062in}}%
\pgfpathlineto{\pgfqpoint{1.048162in}{0.861062in}}%
\pgfpathlineto{\pgfqpoint{1.048559in}{1.034035in}}%
\pgfpathlineto{\pgfqpoint{1.049155in}{0.942461in}}%
\pgfpathlineto{\pgfqpoint{1.049353in}{0.942461in}}%
\pgfpathlineto{\pgfqpoint{1.050147in}{0.830537in}}%
\pgfpathlineto{\pgfqpoint{1.050346in}{1.034035in}}%
\pgfpathlineto{\pgfqpoint{1.050544in}{1.034035in}}%
\pgfpathlineto{\pgfqpoint{1.051537in}{0.901762in}}%
\pgfpathlineto{\pgfqpoint{1.051735in}{0.901762in}}%
\pgfpathlineto{\pgfqpoint{1.051934in}{1.044210in}}%
\pgfpathlineto{\pgfqpoint{1.052728in}{1.023861in}}%
\pgfpathlineto{\pgfqpoint{1.052926in}{1.023861in}}%
\pgfpathlineto{\pgfqpoint{1.052926in}{0.861062in}}%
\pgfpathlineto{\pgfqpoint{1.053919in}{0.942461in}}%
\pgfpathlineto{\pgfqpoint{1.054117in}{0.942461in}}%
\pgfpathlineto{\pgfqpoint{1.054316in}{1.064560in}}%
\pgfpathlineto{\pgfqpoint{1.054713in}{0.901762in}}%
\pgfpathlineto{\pgfqpoint{1.055110in}{1.013686in}}%
\pgfpathlineto{\pgfqpoint{1.055308in}{1.013686in}}%
\pgfpathlineto{\pgfqpoint{1.056102in}{0.952636in}}%
\pgfpathlineto{\pgfqpoint{1.055507in}{1.034035in}}%
\pgfpathlineto{\pgfqpoint{1.056301in}{1.003511in}}%
\pgfpathlineto{\pgfqpoint{1.056499in}{1.003511in}}%
\pgfpathlineto{\pgfqpoint{1.057095in}{0.922111in}}%
\pgfpathlineto{\pgfqpoint{1.057492in}{1.064560in}}%
\pgfpathlineto{\pgfqpoint{1.057690in}{1.064560in}}%
\pgfpathlineto{\pgfqpoint{1.057889in}{0.901762in}}%
\pgfpathlineto{\pgfqpoint{1.058683in}{0.932286in}}%
\pgfpathlineto{\pgfqpoint{1.058881in}{0.932286in}}%
\pgfpathlineto{\pgfqpoint{1.059278in}{1.084910in}}%
\pgfpathlineto{\pgfqpoint{1.059874in}{0.891587in}}%
\pgfpathlineto{\pgfqpoint{1.060072in}{0.891587in}}%
\pgfpathlineto{\pgfqpoint{1.060469in}{1.074735in}}%
\pgfpathlineto{\pgfqpoint{1.061065in}{0.850887in}}%
\pgfpathlineto{\pgfqpoint{1.061263in}{0.850887in}}%
\pgfpathlineto{\pgfqpoint{1.061263in}{1.023861in}}%
\pgfpathlineto{\pgfqpoint{1.062256in}{0.942461in}}%
\pgfpathlineto{\pgfqpoint{1.062454in}{0.942461in}}%
\pgfpathlineto{\pgfqpoint{1.062454in}{0.901762in}}%
\pgfpathlineto{\pgfqpoint{1.063447in}{1.013686in}}%
\pgfpathlineto{\pgfqpoint{1.063645in}{1.013686in}}%
\pgfpathlineto{\pgfqpoint{1.063645in}{0.922111in}}%
\pgfpathlineto{\pgfqpoint{1.064638in}{0.962811in}}%
\pgfpathlineto{\pgfqpoint{1.064836in}{0.962811in}}%
\pgfpathlineto{\pgfqpoint{1.065035in}{1.034035in}}%
\pgfpathlineto{\pgfqpoint{1.065233in}{0.911936in}}%
\pgfpathlineto{\pgfqpoint{1.065829in}{0.993336in}}%
\pgfpathlineto{\pgfqpoint{1.066027in}{0.993336in}}%
\pgfpathlineto{\pgfqpoint{1.066623in}{0.922111in}}%
\pgfpathlineto{\pgfqpoint{1.067020in}{1.013686in}}%
\pgfpathlineto{\pgfqpoint{1.067218in}{1.013686in}}%
\pgfpathlineto{\pgfqpoint{1.067218in}{1.054385in}}%
\pgfpathlineto{\pgfqpoint{1.067615in}{0.952636in}}%
\pgfpathlineto{\pgfqpoint{1.068211in}{0.962811in}}%
\pgfpathlineto{\pgfqpoint{1.068409in}{0.962811in}}%
\pgfpathlineto{\pgfqpoint{1.069203in}{0.932286in}}%
\pgfpathlineto{\pgfqpoint{1.068806in}{1.054385in}}%
\pgfpathlineto{\pgfqpoint{1.069402in}{0.962811in}}%
\pgfpathlineto{\pgfqpoint{1.069600in}{0.962811in}}%
\pgfpathlineto{\pgfqpoint{1.069600in}{1.044210in}}%
\pgfpathlineto{\pgfqpoint{1.069997in}{0.952636in}}%
\pgfpathlineto{\pgfqpoint{1.070592in}{0.962811in}}%
\pgfpathlineto{\pgfqpoint{1.070791in}{0.962811in}}%
\pgfpathlineto{\pgfqpoint{1.070791in}{1.013686in}}%
\pgfpathlineto{\pgfqpoint{1.071188in}{0.901762in}}%
\pgfpathlineto{\pgfqpoint{1.071783in}{1.013686in}}%
\pgfpathlineto{\pgfqpoint{1.071982in}{1.013686in}}%
\pgfpathlineto{\pgfqpoint{1.071982in}{1.044210in}}%
\pgfpathlineto{\pgfqpoint{1.072577in}{0.962811in}}%
\pgfpathlineto{\pgfqpoint{1.072974in}{1.034035in}}%
\pgfpathlineto{\pgfqpoint{1.073173in}{1.034035in}}%
\pgfpathlineto{\pgfqpoint{1.073570in}{1.084910in}}%
\pgfpathlineto{\pgfqpoint{1.073967in}{0.932286in}}%
\pgfpathlineto{\pgfqpoint{1.074165in}{0.962811in}}%
\pgfpathlineto{\pgfqpoint{1.074364in}{0.962811in}}%
\pgfpathlineto{\pgfqpoint{1.074364in}{1.064560in}}%
\pgfpathlineto{\pgfqpoint{1.075356in}{1.034035in}}%
\pgfpathlineto{\pgfqpoint{1.075555in}{1.034035in}}%
\pgfpathlineto{\pgfqpoint{1.075555in}{0.972986in}}%
\pgfpathlineto{\pgfqpoint{1.076349in}{1.064560in}}%
\pgfpathlineto{\pgfqpoint{1.076547in}{1.064560in}}%
\pgfpathlineto{\pgfqpoint{1.076746in}{1.064560in}}%
\pgfpathlineto{\pgfqpoint{1.077143in}{0.952636in}}%
\pgfpathlineto{\pgfqpoint{1.077341in}{1.125610in}}%
\pgfpathlineto{\pgfqpoint{1.077738in}{0.983161in}}%
\pgfpathlineto{\pgfqpoint{1.077937in}{0.983161in}}%
\pgfpathlineto{\pgfqpoint{1.077937in}{1.084910in}}%
\pgfpathlineto{\pgfqpoint{1.078532in}{0.942461in}}%
\pgfpathlineto{\pgfqpoint{1.078929in}{1.054385in}}%
\pgfpathlineto{\pgfqpoint{1.079128in}{1.054385in}}%
\pgfpathlineto{\pgfqpoint{1.079723in}{0.942461in}}%
\pgfpathlineto{\pgfqpoint{1.079525in}{1.125610in}}%
\pgfpathlineto{\pgfqpoint{1.080120in}{1.054385in}}%
\pgfpathlineto{\pgfqpoint{1.080319in}{1.054385in}}%
\pgfpathlineto{\pgfqpoint{1.080517in}{0.972986in}}%
\pgfpathlineto{\pgfqpoint{1.081311in}{1.054385in}}%
\pgfpathlineto{\pgfqpoint{1.081510in}{1.054385in}}%
\pgfpathlineto{\pgfqpoint{1.081510in}{1.064560in}}%
\pgfpathlineto{\pgfqpoint{1.082304in}{0.911936in}}%
\pgfpathlineto{\pgfqpoint{1.082502in}{1.003511in}}%
\pgfpathlineto{\pgfqpoint{1.082701in}{1.003511in}}%
\pgfpathlineto{\pgfqpoint{1.082899in}{1.105260in}}%
\pgfpathlineto{\pgfqpoint{1.083693in}{0.922111in}}%
\pgfpathlineto{\pgfqpoint{1.083892in}{0.922111in}}%
\pgfpathlineto{\pgfqpoint{1.083892in}{1.095085in}}%
\pgfpathlineto{\pgfqpoint{1.084884in}{1.095085in}}%
\pgfpathlineto{\pgfqpoint{1.085083in}{1.095085in}}%
\pgfpathlineto{\pgfqpoint{1.085083in}{1.125610in}}%
\pgfpathlineto{\pgfqpoint{1.085281in}{0.972986in}}%
\pgfpathlineto{\pgfqpoint{1.086075in}{1.003511in}}%
\pgfpathlineto{\pgfqpoint{1.086274in}{1.003511in}}%
\pgfpathlineto{\pgfqpoint{1.086869in}{1.084910in}}%
\pgfpathlineto{\pgfqpoint{1.087266in}{1.074735in}}%
\pgfpathlineto{\pgfqpoint{1.087465in}{1.074735in}}%
\pgfpathlineto{\pgfqpoint{1.087663in}{0.942461in}}%
\pgfpathlineto{\pgfqpoint{1.087862in}{1.095085in}}%
\pgfpathlineto{\pgfqpoint{1.088457in}{1.084910in}}%
\pgfpathlineto{\pgfqpoint{1.088656in}{1.084910in}}%
\pgfpathlineto{\pgfqpoint{1.088656in}{0.962811in}}%
\pgfpathlineto{\pgfqpoint{1.089648in}{1.135785in}}%
\pgfpathlineto{\pgfqpoint{1.089847in}{1.135785in}}%
\pgfpathlineto{\pgfqpoint{1.089847in}{1.186659in}}%
\pgfpathlineto{\pgfqpoint{1.090641in}{0.983161in}}%
\pgfpathlineto{\pgfqpoint{1.090839in}{1.044210in}}%
\pgfpathlineto{\pgfqpoint{1.091038in}{1.044210in}}%
\pgfpathlineto{\pgfqpoint{1.091633in}{0.983161in}}%
\pgfpathlineto{\pgfqpoint{1.091435in}{1.084910in}}%
\pgfpathlineto{\pgfqpoint{1.092030in}{1.013686in}}%
\pgfpathlineto{\pgfqpoint{1.092229in}{1.013686in}}%
\pgfpathlineto{\pgfqpoint{1.092229in}{1.084910in}}%
\pgfpathlineto{\pgfqpoint{1.093221in}{1.034035in}}%
\pgfpathlineto{\pgfqpoint{1.093420in}{1.034035in}}%
\pgfpathlineto{\pgfqpoint{1.093420in}{0.972986in}}%
\pgfpathlineto{\pgfqpoint{1.094214in}{1.084910in}}%
\pgfpathlineto{\pgfqpoint{1.094412in}{1.034035in}}%
\pgfpathlineto{\pgfqpoint{1.094611in}{1.034035in}}%
\pgfpathlineto{\pgfqpoint{1.094809in}{1.166309in}}%
\pgfpathlineto{\pgfqpoint{1.095405in}{0.993336in}}%
\pgfpathlineto{\pgfqpoint{1.095603in}{1.023861in}}%
\pgfpathlineto{\pgfqpoint{1.095802in}{1.023861in}}%
\pgfpathlineto{\pgfqpoint{1.095802in}{1.105260in}}%
\pgfpathlineto{\pgfqpoint{1.096000in}{0.942461in}}%
\pgfpathlineto{\pgfqpoint{1.096794in}{1.074735in}}%
\pgfpathlineto{\pgfqpoint{1.096993in}{1.074735in}}%
\pgfpathlineto{\pgfqpoint{1.097588in}{1.186659in}}%
\pgfpathlineto{\pgfqpoint{1.097985in}{0.942461in}}%
\pgfpathlineto{\pgfqpoint{1.098184in}{0.942461in}}%
\pgfpathlineto{\pgfqpoint{1.098184in}{1.095085in}}%
\pgfpathlineto{\pgfqpoint{1.099176in}{0.993336in}}%
\pgfpathlineto{\pgfqpoint{1.099375in}{0.993336in}}%
\pgfpathlineto{\pgfqpoint{1.099375in}{1.166309in}}%
\pgfpathlineto{\pgfqpoint{1.100367in}{1.054385in}}%
\pgfpathlineto{\pgfqpoint{1.100566in}{1.054385in}}%
\pgfpathlineto{\pgfqpoint{1.101360in}{1.105260in}}%
\pgfpathlineto{\pgfqpoint{1.100764in}{1.044210in}}%
\pgfpathlineto{\pgfqpoint{1.101558in}{1.044210in}}%
\pgfpathlineto{\pgfqpoint{1.101955in}{1.044210in}}%
\pgfpathlineto{\pgfqpoint{1.102550in}{0.942461in}}%
\pgfpathlineto{\pgfqpoint{1.102153in}{1.125610in}}%
\pgfpathlineto{\pgfqpoint{1.102947in}{0.942461in}}%
\pgfpathlineto{\pgfqpoint{1.103146in}{0.942461in}}%
\pgfpathlineto{\pgfqpoint{1.103146in}{0.891587in}}%
\pgfpathlineto{\pgfqpoint{1.103741in}{1.125610in}}%
\pgfpathlineto{\pgfqpoint{1.104138in}{0.993336in}}%
\pgfpathlineto{\pgfqpoint{1.104337in}{0.993336in}}%
\pgfpathlineto{\pgfqpoint{1.104337in}{1.095085in}}%
\pgfpathlineto{\pgfqpoint{1.105131in}{0.972986in}}%
\pgfpathlineto{\pgfqpoint{1.105329in}{0.993336in}}%
\pgfpathlineto{\pgfqpoint{1.105726in}{0.993336in}}%
\pgfpathlineto{\pgfqpoint{1.106520in}{1.156135in}}%
\pgfpathlineto{\pgfqpoint{1.106322in}{0.932286in}}%
\pgfpathlineto{\pgfqpoint{1.106719in}{1.013686in}}%
\pgfpathlineto{\pgfqpoint{1.106917in}{1.013686in}}%
\pgfpathlineto{\pgfqpoint{1.107314in}{1.095085in}}%
\pgfpathlineto{\pgfqpoint{1.107513in}{0.983161in}}%
\pgfpathlineto{\pgfqpoint{1.107910in}{1.064560in}}%
\pgfpathlineto{\pgfqpoint{1.108108in}{1.064560in}}%
\pgfpathlineto{\pgfqpoint{1.108108in}{0.962811in}}%
\pgfpathlineto{\pgfqpoint{1.108307in}{1.156135in}}%
\pgfpathlineto{\pgfqpoint{1.109101in}{1.034035in}}%
\pgfpathlineto{\pgfqpoint{1.109299in}{1.034035in}}%
\pgfpathlineto{\pgfqpoint{1.109696in}{0.952636in}}%
\pgfpathlineto{\pgfqpoint{1.110292in}{1.186659in}}%
\pgfpathlineto{\pgfqpoint{1.110490in}{1.186659in}}%
\pgfpathlineto{\pgfqpoint{1.110689in}{1.003511in}}%
\pgfpathlineto{\pgfqpoint{1.111483in}{1.034035in}}%
\pgfpathlineto{\pgfqpoint{1.111681in}{1.034035in}}%
\pgfpathlineto{\pgfqpoint{1.112475in}{0.952636in}}%
\pgfpathlineto{\pgfqpoint{1.112674in}{1.145960in}}%
\pgfpathlineto{\pgfqpoint{1.112872in}{1.145960in}}%
\pgfpathlineto{\pgfqpoint{1.113666in}{0.983161in}}%
\pgfpathlineto{\pgfqpoint{1.113865in}{1.084910in}}%
\pgfpathlineto{\pgfqpoint{1.114063in}{1.084910in}}%
\pgfpathlineto{\pgfqpoint{1.114857in}{0.993336in}}%
\pgfpathlineto{\pgfqpoint{1.115056in}{1.135785in}}%
\pgfpathlineto{\pgfqpoint{1.115254in}{1.135785in}}%
\pgfpathlineto{\pgfqpoint{1.116048in}{1.044210in}}%
\pgfpathlineto{\pgfqpoint{1.115651in}{1.156135in}}%
\pgfpathlineto{\pgfqpoint{1.116247in}{1.044210in}}%
\pgfpathlineto{\pgfqpoint{1.116445in}{1.044210in}}%
\pgfpathlineto{\pgfqpoint{1.117041in}{1.105260in}}%
\pgfpathlineto{\pgfqpoint{1.116842in}{0.972986in}}%
\pgfpathlineto{\pgfqpoint{1.117438in}{1.023861in}}%
\pgfpathlineto{\pgfqpoint{1.117636in}{1.023861in}}%
\pgfpathlineto{\pgfqpoint{1.117636in}{0.993336in}}%
\pgfpathlineto{\pgfqpoint{1.118232in}{1.115435in}}%
\pgfpathlineto{\pgfqpoint{1.118629in}{1.105260in}}%
\pgfpathlineto{\pgfqpoint{1.118827in}{1.105260in}}%
\pgfpathlineto{\pgfqpoint{1.119621in}{1.166309in}}%
\pgfpathlineto{\pgfqpoint{1.119026in}{1.013686in}}%
\pgfpathlineto{\pgfqpoint{1.119820in}{1.135785in}}%
\pgfpathlineto{\pgfqpoint{1.120018in}{1.135785in}}%
\pgfpathlineto{\pgfqpoint{1.120018in}{1.156135in}}%
\pgfpathlineto{\pgfqpoint{1.121011in}{0.972986in}}%
\pgfpathlineto{\pgfqpoint{1.121209in}{0.972986in}}%
\pgfpathlineto{\pgfqpoint{1.121606in}{1.135785in}}%
\pgfpathlineto{\pgfqpoint{1.122202in}{1.044210in}}%
\pgfpathlineto{\pgfqpoint{1.122400in}{1.044210in}}%
\pgfpathlineto{\pgfqpoint{1.122599in}{0.983161in}}%
\pgfpathlineto{\pgfqpoint{1.122797in}{1.084910in}}%
\pgfpathlineto{\pgfqpoint{1.123393in}{1.044210in}}%
\pgfpathlineto{\pgfqpoint{1.123591in}{1.044210in}}%
\pgfpathlineto{\pgfqpoint{1.124385in}{1.125610in}}%
\pgfpathlineto{\pgfqpoint{1.124584in}{0.993336in}}%
\pgfpathlineto{\pgfqpoint{1.124782in}{0.993336in}}%
\pgfpathlineto{\pgfqpoint{1.124782in}{1.105260in}}%
\pgfpathlineto{\pgfqpoint{1.125775in}{1.034035in}}%
\pgfpathlineto{\pgfqpoint{1.125973in}{1.034035in}}%
\pgfpathlineto{\pgfqpoint{1.126966in}{1.237534in}}%
\pgfpathlineto{\pgfqpoint{1.127164in}{1.237534in}}%
\pgfpathlineto{\pgfqpoint{1.127363in}{0.993336in}}%
\pgfpathlineto{\pgfqpoint{1.128157in}{1.003511in}}%
\pgfpathlineto{\pgfqpoint{1.128355in}{1.003511in}}%
\pgfpathlineto{\pgfqpoint{1.128355in}{1.166309in}}%
\pgfpathlineto{\pgfqpoint{1.128951in}{0.983161in}}%
\pgfpathlineto{\pgfqpoint{1.129348in}{1.074735in}}%
\pgfpathlineto{\pgfqpoint{1.129546in}{1.074735in}}%
\pgfpathlineto{\pgfqpoint{1.129546in}{1.156135in}}%
\pgfpathlineto{\pgfqpoint{1.129943in}{1.003511in}}%
\pgfpathlineto{\pgfqpoint{1.130539in}{1.145960in}}%
\pgfpathlineto{\pgfqpoint{1.130737in}{1.145960in}}%
\pgfpathlineto{\pgfqpoint{1.130737in}{1.034035in}}%
\pgfpathlineto{\pgfqpoint{1.130936in}{1.186659in}}%
\pgfpathlineto{\pgfqpoint{1.131730in}{1.074735in}}%
\pgfpathlineto{\pgfqpoint{1.131928in}{1.074735in}}%
\pgfpathlineto{\pgfqpoint{1.132127in}{0.952636in}}%
\pgfpathlineto{\pgfqpoint{1.132325in}{1.105260in}}%
\pgfpathlineto{\pgfqpoint{1.132921in}{1.105260in}}%
\pgfpathlineto{\pgfqpoint{1.133119in}{1.105260in}}%
\pgfpathlineto{\pgfqpoint{1.133516in}{0.993336in}}%
\pgfpathlineto{\pgfqpoint{1.134112in}{1.013686in}}%
\pgfpathlineto{\pgfqpoint{1.134310in}{1.013686in}}%
\pgfpathlineto{\pgfqpoint{1.134707in}{0.993336in}}%
\pgfpathlineto{\pgfqpoint{1.135302in}{1.135785in}}%
\pgfpathlineto{\pgfqpoint{1.135501in}{1.135785in}}%
\pgfpathlineto{\pgfqpoint{1.136295in}{0.942461in}}%
\pgfpathlineto{\pgfqpoint{1.135898in}{1.196834in}}%
\pgfpathlineto{\pgfqpoint{1.136493in}{1.125610in}}%
\pgfpathlineto{\pgfqpoint{1.136692in}{1.125610in}}%
\pgfpathlineto{\pgfqpoint{1.136692in}{1.145960in}}%
\pgfpathlineto{\pgfqpoint{1.137287in}{1.023861in}}%
\pgfpathlineto{\pgfqpoint{1.137684in}{1.145960in}}%
\pgfpathlineto{\pgfqpoint{1.137883in}{1.145960in}}%
\pgfpathlineto{\pgfqpoint{1.137883in}{1.034035in}}%
\pgfpathlineto{\pgfqpoint{1.138875in}{1.074735in}}%
\pgfpathlineto{\pgfqpoint{1.139074in}{1.074735in}}%
\pgfpathlineto{\pgfqpoint{1.139868in}{0.993336in}}%
\pgfpathlineto{\pgfqpoint{1.140066in}{1.115435in}}%
\pgfpathlineto{\pgfqpoint{1.140265in}{1.115435in}}%
\pgfpathlineto{\pgfqpoint{1.140463in}{1.034035in}}%
\pgfpathlineto{\pgfqpoint{1.140860in}{1.145960in}}%
\pgfpathlineto{\pgfqpoint{1.141257in}{1.135785in}}%
\pgfpathlineto{\pgfqpoint{1.141456in}{1.135785in}}%
\pgfpathlineto{\pgfqpoint{1.141853in}{0.983161in}}%
\pgfpathlineto{\pgfqpoint{1.142051in}{1.145960in}}%
\pgfpathlineto{\pgfqpoint{1.142448in}{1.064560in}}%
\pgfpathlineto{\pgfqpoint{1.142845in}{1.064560in}}%
\pgfpathlineto{\pgfqpoint{1.142845in}{0.972986in}}%
\pgfpathlineto{\pgfqpoint{1.143242in}{1.125610in}}%
\pgfpathlineto{\pgfqpoint{1.143838in}{0.983161in}}%
\pgfpathlineto{\pgfqpoint{1.144036in}{0.983161in}}%
\pgfpathlineto{\pgfqpoint{1.144235in}{1.237534in}}%
\pgfpathlineto{\pgfqpoint{1.145029in}{1.115435in}}%
\pgfpathlineto{\pgfqpoint{1.145227in}{1.115435in}}%
\pgfpathlineto{\pgfqpoint{1.145823in}{0.983161in}}%
\pgfpathlineto{\pgfqpoint{1.146220in}{1.145960in}}%
\pgfpathlineto{\pgfqpoint{1.146418in}{1.145960in}}%
\pgfpathlineto{\pgfqpoint{1.147212in}{1.217184in}}%
\pgfpathlineto{\pgfqpoint{1.147411in}{0.962811in}}%
\pgfpathlineto{\pgfqpoint{1.147609in}{0.962811in}}%
\pgfpathlineto{\pgfqpoint{1.147609in}{0.942461in}}%
\pgfpathlineto{\pgfqpoint{1.148006in}{1.125610in}}%
\pgfpathlineto{\pgfqpoint{1.148602in}{1.054385in}}%
\pgfpathlineto{\pgfqpoint{1.148800in}{1.054385in}}%
\pgfpathlineto{\pgfqpoint{1.148800in}{1.176484in}}%
\pgfpathlineto{\pgfqpoint{1.149793in}{1.125610in}}%
\pgfpathlineto{\pgfqpoint{1.149991in}{1.125610in}}%
\pgfpathlineto{\pgfqpoint{1.150388in}{0.972986in}}%
\pgfpathlineto{\pgfqpoint{1.150984in}{0.993336in}}%
\pgfpathlineto{\pgfqpoint{1.151182in}{0.993336in}}%
\pgfpathlineto{\pgfqpoint{1.151778in}{1.084910in}}%
\pgfpathlineto{\pgfqpoint{1.151381in}{0.972986in}}%
\pgfpathlineto{\pgfqpoint{1.152175in}{1.013686in}}%
\pgfpathlineto{\pgfqpoint{1.152373in}{1.013686in}}%
\pgfpathlineto{\pgfqpoint{1.152572in}{0.962811in}}%
\pgfpathlineto{\pgfqpoint{1.153366in}{1.166309in}}%
\pgfpathlineto{\pgfqpoint{1.153564in}{1.166309in}}%
\pgfpathlineto{\pgfqpoint{1.153961in}{1.044210in}}%
\pgfpathlineto{\pgfqpoint{1.153763in}{1.176484in}}%
\pgfpathlineto{\pgfqpoint{1.154557in}{1.054385in}}%
\pgfpathlineto{\pgfqpoint{1.154755in}{1.054385in}}%
\pgfpathlineto{\pgfqpoint{1.154755in}{1.298583in}}%
\pgfpathlineto{\pgfqpoint{1.155748in}{1.023861in}}%
\pgfpathlineto{\pgfqpoint{1.155946in}{1.023861in}}%
\pgfpathlineto{\pgfqpoint{1.155946in}{1.115435in}}%
\pgfpathlineto{\pgfqpoint{1.156939in}{1.064560in}}%
\pgfpathlineto{\pgfqpoint{1.157137in}{1.064560in}}%
\pgfpathlineto{\pgfqpoint{1.157137in}{1.044210in}}%
\pgfpathlineto{\pgfqpoint{1.157534in}{1.176484in}}%
\pgfpathlineto{\pgfqpoint{1.158130in}{1.074735in}}%
\pgfpathlineto{\pgfqpoint{1.158328in}{1.074735in}}%
\pgfpathlineto{\pgfqpoint{1.159122in}{0.983161in}}%
\pgfpathlineto{\pgfqpoint{1.159321in}{1.186659in}}%
\pgfpathlineto{\pgfqpoint{1.159519in}{1.186659in}}%
\pgfpathlineto{\pgfqpoint{1.160115in}{1.044210in}}%
\pgfpathlineto{\pgfqpoint{1.160512in}{1.095085in}}%
\pgfpathlineto{\pgfqpoint{1.160710in}{1.095085in}}%
\pgfpathlineto{\pgfqpoint{1.161504in}{1.176484in}}%
\pgfpathlineto{\pgfqpoint{1.161107in}{1.023861in}}%
\pgfpathlineto{\pgfqpoint{1.161703in}{1.145960in}}%
\pgfpathlineto{\pgfqpoint{1.161901in}{1.145960in}}%
\pgfpathlineto{\pgfqpoint{1.162695in}{0.993336in}}%
\pgfpathlineto{\pgfqpoint{1.162894in}{1.084910in}}%
\pgfpathlineto{\pgfqpoint{1.163092in}{1.084910in}}%
\pgfpathlineto{\pgfqpoint{1.163688in}{1.054385in}}%
\pgfpathlineto{\pgfqpoint{1.164085in}{1.196834in}}%
\pgfpathlineto{\pgfqpoint{1.164283in}{1.196834in}}%
\pgfpathlineto{\pgfqpoint{1.164680in}{0.972986in}}%
\pgfpathlineto{\pgfqpoint{1.164879in}{1.257884in}}%
\pgfpathlineto{\pgfqpoint{1.165276in}{1.084910in}}%
\pgfpathlineto{\pgfqpoint{1.165474in}{1.084910in}}%
\pgfpathlineto{\pgfqpoint{1.166070in}{1.186659in}}%
\pgfpathlineto{\pgfqpoint{1.165673in}{1.023861in}}%
\pgfpathlineto{\pgfqpoint{1.166467in}{1.186659in}}%
\pgfpathlineto{\pgfqpoint{1.166665in}{1.186659in}}%
\pgfpathlineto{\pgfqpoint{1.167459in}{1.054385in}}%
\pgfpathlineto{\pgfqpoint{1.167062in}{1.257884in}}%
\pgfpathlineto{\pgfqpoint{1.167657in}{1.186659in}}%
\pgfpathlineto{\pgfqpoint{1.167856in}{1.186659in}}%
\pgfpathlineto{\pgfqpoint{1.168650in}{1.023861in}}%
\pgfpathlineto{\pgfqpoint{1.168848in}{1.156135in}}%
\pgfpathlineto{\pgfqpoint{1.169047in}{1.156135in}}%
\pgfpathlineto{\pgfqpoint{1.169642in}{1.044210in}}%
\pgfpathlineto{\pgfqpoint{1.170039in}{1.166309in}}%
\pgfpathlineto{\pgfqpoint{1.170238in}{1.166309in}}%
\pgfpathlineto{\pgfqpoint{1.170635in}{1.064560in}}%
\pgfpathlineto{\pgfqpoint{1.171230in}{1.115435in}}%
\pgfpathlineto{\pgfqpoint{1.171627in}{1.115435in}}%
\pgfpathlineto{\pgfqpoint{1.172223in}{1.176484in}}%
\pgfpathlineto{\pgfqpoint{1.172024in}{0.983161in}}%
\pgfpathlineto{\pgfqpoint{1.172620in}{1.064560in}}%
\pgfpathlineto{\pgfqpoint{1.172818in}{1.064560in}}%
\pgfpathlineto{\pgfqpoint{1.173414in}{1.186659in}}%
\pgfpathlineto{\pgfqpoint{1.173215in}{1.023861in}}%
\pgfpathlineto{\pgfqpoint{1.173811in}{1.135785in}}%
\pgfpathlineto{\pgfqpoint{1.174009in}{1.135785in}}%
\pgfpathlineto{\pgfqpoint{1.174009in}{1.166309in}}%
\pgfpathlineto{\pgfqpoint{1.175002in}{1.023861in}}%
\pgfpathlineto{\pgfqpoint{1.175399in}{1.023861in}}%
\pgfpathlineto{\pgfqpoint{1.175399in}{1.013686in}}%
\pgfpathlineto{\pgfqpoint{1.175994in}{1.207009in}}%
\pgfpathlineto{\pgfqpoint{1.176391in}{1.207009in}}%
\pgfpathlineto{\pgfqpoint{1.176590in}{1.207009in}}%
\pgfpathlineto{\pgfqpoint{1.177185in}{1.034035in}}%
\pgfpathlineto{\pgfqpoint{1.177582in}{1.166309in}}%
\pgfpathlineto{\pgfqpoint{1.177781in}{1.166309in}}%
\pgfpathlineto{\pgfqpoint{1.177979in}{1.034035in}}%
\pgfpathlineto{\pgfqpoint{1.178575in}{1.176484in}}%
\pgfpathlineto{\pgfqpoint{1.178773in}{1.084910in}}%
\pgfpathlineto{\pgfqpoint{1.178972in}{1.084910in}}%
\pgfpathlineto{\pgfqpoint{1.179369in}{1.034035in}}%
\pgfpathlineto{\pgfqpoint{1.179766in}{1.207009in}}%
\pgfpathlineto{\pgfqpoint{1.179964in}{1.074735in}}%
\pgfpathlineto{\pgfqpoint{1.180163in}{1.074735in}}%
\pgfpathlineto{\pgfqpoint{1.180163in}{0.993336in}}%
\pgfpathlineto{\pgfqpoint{1.181155in}{1.186659in}}%
\pgfpathlineto{\pgfqpoint{1.181354in}{1.186659in}}%
\pgfpathlineto{\pgfqpoint{1.181354in}{1.013686in}}%
\pgfpathlineto{\pgfqpoint{1.182346in}{1.156135in}}%
\pgfpathlineto{\pgfqpoint{1.182545in}{1.156135in}}%
\pgfpathlineto{\pgfqpoint{1.183339in}{1.013686in}}%
\pgfpathlineto{\pgfqpoint{1.183537in}{1.156135in}}%
\pgfpathlineto{\pgfqpoint{1.183736in}{1.156135in}}%
\pgfpathlineto{\pgfqpoint{1.184530in}{1.003511in}}%
\pgfpathlineto{\pgfqpoint{1.184728in}{1.084910in}}%
\pgfpathlineto{\pgfqpoint{1.184927in}{1.084910in}}%
\pgfpathlineto{\pgfqpoint{1.184927in}{1.074735in}}%
\pgfpathlineto{\pgfqpoint{1.185522in}{1.166309in}}%
\pgfpathlineto{\pgfqpoint{1.185919in}{1.074735in}}%
\pgfpathlineto{\pgfqpoint{1.186118in}{1.074735in}}%
\pgfpathlineto{\pgfqpoint{1.186118in}{1.003511in}}%
\pgfpathlineto{\pgfqpoint{1.186912in}{1.247709in}}%
\pgfpathlineto{\pgfqpoint{1.187110in}{1.176484in}}%
\pgfpathlineto{\pgfqpoint{1.187309in}{1.176484in}}%
\pgfpathlineto{\pgfqpoint{1.187309in}{1.054385in}}%
\pgfpathlineto{\pgfqpoint{1.188301in}{1.166309in}}%
\pgfpathlineto{\pgfqpoint{1.188500in}{1.166309in}}%
\pgfpathlineto{\pgfqpoint{1.188698in}{1.064560in}}%
\pgfpathlineto{\pgfqpoint{1.189492in}{1.105260in}}%
\pgfpathlineto{\pgfqpoint{1.189691in}{1.105260in}}%
\pgfpathlineto{\pgfqpoint{1.190485in}{1.217184in}}%
\pgfpathlineto{\pgfqpoint{1.189889in}{0.983161in}}%
\pgfpathlineto{\pgfqpoint{1.190683in}{1.034035in}}%
\pgfpathlineto{\pgfqpoint{1.190882in}{1.034035in}}%
\pgfpathlineto{\pgfqpoint{1.191477in}{1.247709in}}%
\pgfpathlineto{\pgfqpoint{1.191279in}{0.983161in}}%
\pgfpathlineto{\pgfqpoint{1.191874in}{1.064560in}}%
\pgfpathlineto{\pgfqpoint{1.192073in}{1.064560in}}%
\pgfpathlineto{\pgfqpoint{1.192867in}{1.237534in}}%
\pgfpathlineto{\pgfqpoint{1.192668in}{1.023861in}}%
\pgfpathlineto{\pgfqpoint{1.193065in}{1.125610in}}%
\pgfpathlineto{\pgfqpoint{1.193264in}{1.125610in}}%
\pgfpathlineto{\pgfqpoint{1.194058in}{1.044210in}}%
\pgfpathlineto{\pgfqpoint{1.194256in}{1.095085in}}%
\pgfpathlineto{\pgfqpoint{1.194455in}{1.095085in}}%
\pgfpathlineto{\pgfqpoint{1.194852in}{0.993336in}}%
\pgfpathlineto{\pgfqpoint{1.194653in}{1.115435in}}%
\pgfpathlineto{\pgfqpoint{1.195447in}{1.105260in}}%
\pgfpathlineto{\pgfqpoint{1.195646in}{1.105260in}}%
\pgfpathlineto{\pgfqpoint{1.195646in}{0.983161in}}%
\pgfpathlineto{\pgfqpoint{1.195844in}{1.125610in}}%
\pgfpathlineto{\pgfqpoint{1.196638in}{1.105260in}}%
\pgfpathlineto{\pgfqpoint{1.197035in}{1.105260in}}%
\pgfpathlineto{\pgfqpoint{1.197035in}{1.278234in}}%
\pgfpathlineto{\pgfqpoint{1.198028in}{1.023861in}}%
\pgfpathlineto{\pgfqpoint{1.198226in}{1.023861in}}%
\pgfpathlineto{\pgfqpoint{1.198822in}{1.227359in}}%
\pgfpathlineto{\pgfqpoint{1.199218in}{1.156135in}}%
\pgfpathlineto{\pgfqpoint{1.199417in}{1.156135in}}%
\pgfpathlineto{\pgfqpoint{1.199615in}{0.972986in}}%
\pgfpathlineto{\pgfqpoint{1.200409in}{1.105260in}}%
\pgfpathlineto{\pgfqpoint{1.200608in}{1.105260in}}%
\pgfpathlineto{\pgfqpoint{1.200608in}{0.993336in}}%
\pgfpathlineto{\pgfqpoint{1.201402in}{1.145960in}}%
\pgfpathlineto{\pgfqpoint{1.201600in}{1.034035in}}%
\pgfpathlineto{\pgfqpoint{1.201799in}{1.034035in}}%
\pgfpathlineto{\pgfqpoint{1.201997in}{1.135785in}}%
\pgfpathlineto{\pgfqpoint{1.202593in}{0.993336in}}%
\pgfpathlineto{\pgfqpoint{1.202791in}{1.105260in}}%
\pgfpathlineto{\pgfqpoint{1.202990in}{1.105260in}}%
\pgfpathlineto{\pgfqpoint{1.203387in}{1.145960in}}%
\pgfpathlineto{\pgfqpoint{1.203982in}{1.044210in}}%
\pgfpathlineto{\pgfqpoint{1.204181in}{1.044210in}}%
\pgfpathlineto{\pgfqpoint{1.204379in}{1.145960in}}%
\pgfpathlineto{\pgfqpoint{1.205173in}{1.064560in}}%
\pgfpathlineto{\pgfqpoint{1.205372in}{1.064560in}}%
\pgfpathlineto{\pgfqpoint{1.205372in}{1.186659in}}%
\pgfpathlineto{\pgfqpoint{1.206364in}{1.125610in}}%
\pgfpathlineto{\pgfqpoint{1.206761in}{1.125610in}}%
\pgfpathlineto{\pgfqpoint{1.207555in}{1.156135in}}%
\pgfpathlineto{\pgfqpoint{1.207754in}{0.983161in}}%
\pgfpathlineto{\pgfqpoint{1.207952in}{0.983161in}}%
\pgfpathlineto{\pgfqpoint{1.207952in}{1.207009in}}%
\pgfpathlineto{\pgfqpoint{1.208945in}{1.023861in}}%
\pgfpathlineto{\pgfqpoint{1.209143in}{1.023861in}}%
\pgfpathlineto{\pgfqpoint{1.209739in}{1.125610in}}%
\pgfpathlineto{\pgfqpoint{1.210136in}{0.962811in}}%
\pgfpathlineto{\pgfqpoint{1.210334in}{0.962811in}}%
\pgfpathlineto{\pgfqpoint{1.210334in}{1.156135in}}%
\pgfpathlineto{\pgfqpoint{1.211327in}{0.983161in}}%
\pgfpathlineto{\pgfqpoint{1.211525in}{0.983161in}}%
\pgfpathlineto{\pgfqpoint{1.212319in}{1.196834in}}%
\pgfpathlineto{\pgfqpoint{1.212518in}{1.156135in}}%
\pgfpathlineto{\pgfqpoint{1.212716in}{1.156135in}}%
\pgfpathlineto{\pgfqpoint{1.213312in}{1.237534in}}%
\pgfpathlineto{\pgfqpoint{1.213709in}{0.962811in}}%
\pgfpathlineto{\pgfqpoint{1.213907in}{0.962811in}}%
\pgfpathlineto{\pgfqpoint{1.213907in}{1.105260in}}%
\pgfpathlineto{\pgfqpoint{1.214900in}{1.023861in}}%
\pgfpathlineto{\pgfqpoint{1.215098in}{1.023861in}}%
\pgfpathlineto{\pgfqpoint{1.215694in}{1.095085in}}%
\pgfpathlineto{\pgfqpoint{1.215892in}{0.972986in}}%
\pgfpathlineto{\pgfqpoint{1.216091in}{1.095085in}}%
\pgfpathlineto{\pgfqpoint{1.216289in}{1.095085in}}%
\pgfpathlineto{\pgfqpoint{1.216488in}{1.186659in}}%
\pgfpathlineto{\pgfqpoint{1.217282in}{0.952636in}}%
\pgfpathlineto{\pgfqpoint{1.217480in}{0.952636in}}%
\pgfpathlineto{\pgfqpoint{1.218473in}{1.217184in}}%
\pgfpathlineto{\pgfqpoint{1.218671in}{1.217184in}}%
\pgfpathlineto{\pgfqpoint{1.218870in}{0.972986in}}%
\pgfpathlineto{\pgfqpoint{1.219664in}{1.095085in}}%
\pgfpathlineto{\pgfqpoint{1.219862in}{1.095085in}}%
\pgfpathlineto{\pgfqpoint{1.219862in}{1.074735in}}%
\pgfpathlineto{\pgfqpoint{1.220855in}{1.176484in}}%
\pgfpathlineto{\pgfqpoint{1.221053in}{1.176484in}}%
\pgfpathlineto{\pgfqpoint{1.222046in}{1.003511in}}%
\pgfpathlineto{\pgfqpoint{1.222244in}{1.003511in}}%
\pgfpathlineto{\pgfqpoint{1.222840in}{1.095085in}}%
\pgfpathlineto{\pgfqpoint{1.223237in}{1.095085in}}%
\pgfpathlineto{\pgfqpoint{1.223435in}{1.095085in}}%
\pgfpathlineto{\pgfqpoint{1.223435in}{1.186659in}}%
\pgfpathlineto{\pgfqpoint{1.224428in}{0.932286in}}%
\pgfpathlineto{\pgfqpoint{1.224626in}{0.932286in}}%
\pgfpathlineto{\pgfqpoint{1.225222in}{1.207009in}}%
\pgfpathlineto{\pgfqpoint{1.225619in}{1.074735in}}%
\pgfpathlineto{\pgfqpoint{1.225817in}{1.074735in}}%
\pgfpathlineto{\pgfqpoint{1.225817in}{0.962811in}}%
\pgfpathlineto{\pgfqpoint{1.226016in}{1.247709in}}%
\pgfpathlineto{\pgfqpoint{1.226810in}{1.196834in}}%
\pgfpathlineto{\pgfqpoint{1.227008in}{1.196834in}}%
\pgfpathlineto{\pgfqpoint{1.227008in}{1.268059in}}%
\pgfpathlineto{\pgfqpoint{1.227802in}{1.054385in}}%
\pgfpathlineto{\pgfqpoint{1.228001in}{1.074735in}}%
\pgfpathlineto{\pgfqpoint{1.228199in}{1.074735in}}%
\pgfpathlineto{\pgfqpoint{1.228596in}{1.196834in}}%
\pgfpathlineto{\pgfqpoint{1.229192in}{0.993336in}}%
\pgfpathlineto{\pgfqpoint{1.229390in}{0.993336in}}%
\pgfpathlineto{\pgfqpoint{1.230383in}{1.207009in}}%
\pgfpathlineto{\pgfqpoint{1.230581in}{1.207009in}}%
\pgfpathlineto{\pgfqpoint{1.230581in}{0.962811in}}%
\pgfpathlineto{\pgfqpoint{1.231573in}{1.074735in}}%
\pgfpathlineto{\pgfqpoint{1.231772in}{1.074735in}}%
\pgfpathlineto{\pgfqpoint{1.231772in}{1.135785in}}%
\pgfpathlineto{\pgfqpoint{1.232764in}{1.084910in}}%
\pgfpathlineto{\pgfqpoint{1.232963in}{1.084910in}}%
\pgfpathlineto{\pgfqpoint{1.233161in}{1.207009in}}%
\pgfpathlineto{\pgfqpoint{1.233360in}{0.972986in}}%
\pgfpathlineto{\pgfqpoint{1.233955in}{1.084910in}}%
\pgfpathlineto{\pgfqpoint{1.234154in}{1.084910in}}%
\pgfpathlineto{\pgfqpoint{1.234749in}{1.034035in}}%
\pgfpathlineto{\pgfqpoint{1.234551in}{1.196834in}}%
\pgfpathlineto{\pgfqpoint{1.235146in}{1.186659in}}%
\pgfpathlineto{\pgfqpoint{1.235345in}{1.186659in}}%
\pgfpathlineto{\pgfqpoint{1.235345in}{1.196834in}}%
\pgfpathlineto{\pgfqpoint{1.235543in}{0.972986in}}%
\pgfpathlineto{\pgfqpoint{1.236337in}{1.034035in}}%
\pgfpathlineto{\pgfqpoint{1.236536in}{1.034035in}}%
\pgfpathlineto{\pgfqpoint{1.237528in}{1.166309in}}%
\pgfpathlineto{\pgfqpoint{1.237727in}{1.166309in}}%
\pgfpathlineto{\pgfqpoint{1.237925in}{0.993336in}}%
\pgfpathlineto{\pgfqpoint{1.238521in}{1.196834in}}%
\pgfpathlineto{\pgfqpoint{1.238719in}{1.176484in}}%
\pgfpathlineto{\pgfqpoint{1.238918in}{1.176484in}}%
\pgfpathlineto{\pgfqpoint{1.239315in}{1.095085in}}%
\pgfpathlineto{\pgfqpoint{1.239910in}{1.095085in}}%
\pgfpathlineto{\pgfqpoint{1.240109in}{1.095085in}}%
\pgfpathlineto{\pgfqpoint{1.240109in}{0.993336in}}%
\pgfpathlineto{\pgfqpoint{1.240903in}{1.186659in}}%
\pgfpathlineto{\pgfqpoint{1.241101in}{1.023861in}}%
\pgfpathlineto{\pgfqpoint{1.241300in}{1.023861in}}%
\pgfpathlineto{\pgfqpoint{1.242292in}{1.156135in}}%
\pgfpathlineto{\pgfqpoint{1.242491in}{1.156135in}}%
\pgfpathlineto{\pgfqpoint{1.242689in}{0.972986in}}%
\pgfpathlineto{\pgfqpoint{1.243086in}{1.237534in}}%
\pgfpathlineto{\pgfqpoint{1.243483in}{1.054385in}}%
\pgfpathlineto{\pgfqpoint{1.243682in}{1.054385in}}%
\pgfpathlineto{\pgfqpoint{1.244277in}{1.145960in}}%
\pgfpathlineto{\pgfqpoint{1.244674in}{1.023861in}}%
\pgfpathlineto{\pgfqpoint{1.244873in}{1.023861in}}%
\pgfpathlineto{\pgfqpoint{1.245270in}{1.227359in}}%
\pgfpathlineto{\pgfqpoint{1.245865in}{1.166309in}}%
\pgfpathlineto{\pgfqpoint{1.246064in}{1.166309in}}%
\pgfpathlineto{\pgfqpoint{1.246461in}{1.227359in}}%
\pgfpathlineto{\pgfqpoint{1.247056in}{1.034035in}}%
\pgfpathlineto{\pgfqpoint{1.247255in}{1.034035in}}%
\pgfpathlineto{\pgfqpoint{1.248049in}{1.145960in}}%
\pgfpathlineto{\pgfqpoint{1.248247in}{1.135785in}}%
\pgfpathlineto{\pgfqpoint{1.248446in}{1.135785in}}%
\pgfpathlineto{\pgfqpoint{1.249041in}{1.166309in}}%
\pgfpathlineto{\pgfqpoint{1.249438in}{0.983161in}}%
\pgfpathlineto{\pgfqpoint{1.249637in}{0.983161in}}%
\pgfpathlineto{\pgfqpoint{1.250629in}{1.329108in}}%
\pgfpathlineto{\pgfqpoint{1.250828in}{1.329108in}}%
\pgfpathlineto{\pgfqpoint{1.250828in}{1.044210in}}%
\pgfpathlineto{\pgfqpoint{1.251820in}{1.095085in}}%
\pgfpathlineto{\pgfqpoint{1.252019in}{1.095085in}}%
\pgfpathlineto{\pgfqpoint{1.252813in}{1.247709in}}%
\pgfpathlineto{\pgfqpoint{1.253011in}{1.166309in}}%
\pgfpathlineto{\pgfqpoint{1.253210in}{1.166309in}}%
\pgfpathlineto{\pgfqpoint{1.253805in}{1.034035in}}%
\pgfpathlineto{\pgfqpoint{1.253408in}{1.237534in}}%
\pgfpathlineto{\pgfqpoint{1.254202in}{1.186659in}}%
\pgfpathlineto{\pgfqpoint{1.254401in}{1.186659in}}%
\pgfpathlineto{\pgfqpoint{1.255195in}{1.237534in}}%
\pgfpathlineto{\pgfqpoint{1.255393in}{1.095085in}}%
\pgfpathlineto{\pgfqpoint{1.255592in}{1.095085in}}%
\pgfpathlineto{\pgfqpoint{1.255790in}{1.268059in}}%
\pgfpathlineto{\pgfqpoint{1.255989in}{1.034035in}}%
\pgfpathlineto{\pgfqpoint{1.256584in}{1.217184in}}%
\pgfpathlineto{\pgfqpoint{1.256783in}{1.217184in}}%
\pgfpathlineto{\pgfqpoint{1.256981in}{1.084910in}}%
\pgfpathlineto{\pgfqpoint{1.257577in}{1.237534in}}%
\pgfpathlineto{\pgfqpoint{1.257775in}{1.237534in}}%
\pgfpathlineto{\pgfqpoint{1.257974in}{1.237534in}}%
\pgfpathlineto{\pgfqpoint{1.257974in}{1.288408in}}%
\pgfpathlineto{\pgfqpoint{1.258966in}{1.054385in}}%
\pgfpathlineto{\pgfqpoint{1.259165in}{1.054385in}}%
\pgfpathlineto{\pgfqpoint{1.259959in}{1.237534in}}%
\pgfpathlineto{\pgfqpoint{1.259562in}{0.983161in}}%
\pgfpathlineto{\pgfqpoint{1.260157in}{1.115435in}}%
\pgfpathlineto{\pgfqpoint{1.260356in}{1.115435in}}%
\pgfpathlineto{\pgfqpoint{1.260356in}{1.166309in}}%
\pgfpathlineto{\pgfqpoint{1.260951in}{1.064560in}}%
\pgfpathlineto{\pgfqpoint{1.261348in}{1.115435in}}%
\pgfpathlineto{\pgfqpoint{1.261547in}{1.115435in}}%
\pgfpathlineto{\pgfqpoint{1.261547in}{1.227359in}}%
\pgfpathlineto{\pgfqpoint{1.262539in}{1.145960in}}%
\pgfpathlineto{\pgfqpoint{1.262936in}{1.145960in}}%
\pgfpathlineto{\pgfqpoint{1.263135in}{1.084910in}}%
\pgfpathlineto{\pgfqpoint{1.263730in}{1.186659in}}%
\pgfpathlineto{\pgfqpoint{1.263928in}{1.145960in}}%
\pgfpathlineto{\pgfqpoint{1.264325in}{1.145960in}}%
\pgfpathlineto{\pgfqpoint{1.264524in}{1.237534in}}%
\pgfpathlineto{\pgfqpoint{1.264722in}{1.135785in}}%
\pgfpathlineto{\pgfqpoint{1.265318in}{1.156135in}}%
\pgfpathlineto{\pgfqpoint{1.265516in}{1.156135in}}%
\pgfpathlineto{\pgfqpoint{1.265516in}{1.074735in}}%
\pgfpathlineto{\pgfqpoint{1.265913in}{1.298583in}}%
\pgfpathlineto{\pgfqpoint{1.266509in}{1.207009in}}%
\pgfpathlineto{\pgfqpoint{1.266707in}{1.207009in}}%
\pgfpathlineto{\pgfqpoint{1.267104in}{1.105260in}}%
\pgfpathlineto{\pgfqpoint{1.267700in}{1.196834in}}%
\pgfpathlineto{\pgfqpoint{1.268097in}{1.196834in}}%
\pgfpathlineto{\pgfqpoint{1.268295in}{1.156135in}}%
\pgfpathlineto{\pgfqpoint{1.269089in}{1.298583in}}%
\pgfpathlineto{\pgfqpoint{1.269288in}{1.298583in}}%
\pgfpathlineto{\pgfqpoint{1.269883in}{1.115435in}}%
\pgfpathlineto{\pgfqpoint{1.270280in}{1.135785in}}%
\pgfpathlineto{\pgfqpoint{1.270479in}{1.135785in}}%
\pgfpathlineto{\pgfqpoint{1.270677in}{1.308758in}}%
\pgfpathlineto{\pgfqpoint{1.271471in}{1.278234in}}%
\pgfpathlineto{\pgfqpoint{1.271670in}{1.278234in}}%
\pgfpathlineto{\pgfqpoint{1.272067in}{1.084910in}}%
\pgfpathlineto{\pgfqpoint{1.272662in}{1.156135in}}%
\pgfpathlineto{\pgfqpoint{1.272861in}{1.156135in}}%
\pgfpathlineto{\pgfqpoint{1.272861in}{1.105260in}}%
\pgfpathlineto{\pgfqpoint{1.273655in}{1.390158in}}%
\pgfpathlineto{\pgfqpoint{1.273853in}{1.227359in}}%
\pgfpathlineto{\pgfqpoint{1.274250in}{1.227359in}}%
\pgfpathlineto{\pgfqpoint{1.274250in}{1.329108in}}%
\pgfpathlineto{\pgfqpoint{1.274846in}{1.125610in}}%
\pgfpathlineto{\pgfqpoint{1.275243in}{1.217184in}}%
\pgfpathlineto{\pgfqpoint{1.275441in}{1.217184in}}%
\pgfpathlineto{\pgfqpoint{1.275441in}{1.298583in}}%
\pgfpathlineto{\pgfqpoint{1.276434in}{1.084910in}}%
\pgfpathlineto{\pgfqpoint{1.276632in}{1.084910in}}%
\pgfpathlineto{\pgfqpoint{1.277426in}{1.268059in}}%
\pgfpathlineto{\pgfqpoint{1.277625in}{1.207009in}}%
\pgfpathlineto{\pgfqpoint{1.277823in}{1.207009in}}%
\pgfpathlineto{\pgfqpoint{1.278022in}{1.135785in}}%
\pgfpathlineto{\pgfqpoint{1.278617in}{1.278234in}}%
\pgfpathlineto{\pgfqpoint{1.278816in}{1.135785in}}%
\pgfpathlineto{\pgfqpoint{1.279014in}{1.135785in}}%
\pgfpathlineto{\pgfqpoint{1.279014in}{1.359633in}}%
\pgfpathlineto{\pgfqpoint{1.280007in}{1.237534in}}%
\pgfpathlineto{\pgfqpoint{1.280205in}{1.237534in}}%
\pgfpathlineto{\pgfqpoint{1.280205in}{1.298583in}}%
\pgfpathlineto{\pgfqpoint{1.281198in}{1.196834in}}%
\pgfpathlineto{\pgfqpoint{1.281396in}{1.196834in}}%
\pgfpathlineto{\pgfqpoint{1.282190in}{1.308758in}}%
\pgfpathlineto{\pgfqpoint{1.281595in}{1.125610in}}%
\pgfpathlineto{\pgfqpoint{1.282389in}{1.237534in}}%
\pgfpathlineto{\pgfqpoint{1.282587in}{1.237534in}}%
\pgfpathlineto{\pgfqpoint{1.282587in}{1.349458in}}%
\pgfpathlineto{\pgfqpoint{1.282984in}{1.145960in}}%
\pgfpathlineto{\pgfqpoint{1.283580in}{1.288408in}}%
\pgfpathlineto{\pgfqpoint{1.283778in}{1.288408in}}%
\pgfpathlineto{\pgfqpoint{1.284175in}{1.115435in}}%
\pgfpathlineto{\pgfqpoint{1.284374in}{1.349458in}}%
\pgfpathlineto{\pgfqpoint{1.284771in}{1.247709in}}%
\pgfpathlineto{\pgfqpoint{1.284969in}{1.247709in}}%
\pgfpathlineto{\pgfqpoint{1.285168in}{1.237534in}}%
\pgfpathlineto{\pgfqpoint{1.285962in}{1.430857in}}%
\pgfpathlineto{\pgfqpoint{1.286160in}{1.430857in}}%
\pgfpathlineto{\pgfqpoint{1.286160in}{1.237534in}}%
\pgfpathlineto{\pgfqpoint{1.287153in}{1.339283in}}%
\pgfpathlineto{\pgfqpoint{1.287351in}{1.339283in}}%
\pgfpathlineto{\pgfqpoint{1.287351in}{1.257884in}}%
\pgfpathlineto{\pgfqpoint{1.288344in}{1.257884in}}%
\pgfpathlineto{\pgfqpoint{1.288542in}{1.257884in}}%
\pgfpathlineto{\pgfqpoint{1.288542in}{1.217184in}}%
\pgfpathlineto{\pgfqpoint{1.289535in}{1.451207in}}%
\pgfpathlineto{\pgfqpoint{1.289733in}{1.451207in}}%
\pgfpathlineto{\pgfqpoint{1.290130in}{1.217184in}}%
\pgfpathlineto{\pgfqpoint{1.290726in}{1.298583in}}%
\pgfpathlineto{\pgfqpoint{1.290924in}{1.298583in}}%
\pgfpathlineto{\pgfqpoint{1.290924in}{1.390158in}}%
\pgfpathlineto{\pgfqpoint{1.291321in}{1.217184in}}%
\pgfpathlineto{\pgfqpoint{1.291917in}{1.308758in}}%
\pgfpathlineto{\pgfqpoint{1.292115in}{1.308758in}}%
\pgfpathlineto{\pgfqpoint{1.292115in}{1.502082in}}%
\pgfpathlineto{\pgfqpoint{1.292711in}{1.207009in}}%
\pgfpathlineto{\pgfqpoint{1.293108in}{1.349458in}}%
\pgfpathlineto{\pgfqpoint{1.293505in}{1.349458in}}%
\pgfpathlineto{\pgfqpoint{1.294299in}{1.481732in}}%
\pgfpathlineto{\pgfqpoint{1.293703in}{1.217184in}}%
\pgfpathlineto{\pgfqpoint{1.294497in}{1.471557in}}%
\pgfpathlineto{\pgfqpoint{1.294696in}{1.471557in}}%
\pgfpathlineto{\pgfqpoint{1.295291in}{1.491907in}}%
\pgfpathlineto{\pgfqpoint{1.295688in}{1.247709in}}%
\pgfpathlineto{\pgfqpoint{1.295886in}{1.247709in}}%
\pgfpathlineto{\pgfqpoint{1.295886in}{1.502082in}}%
\pgfpathlineto{\pgfqpoint{1.296879in}{1.308758in}}%
\pgfpathlineto{\pgfqpoint{1.297077in}{1.308758in}}%
\pgfpathlineto{\pgfqpoint{1.297673in}{1.471557in}}%
\pgfpathlineto{\pgfqpoint{1.298070in}{1.379983in}}%
\pgfpathlineto{\pgfqpoint{1.298268in}{1.379983in}}%
\pgfpathlineto{\pgfqpoint{1.298268in}{1.441032in}}%
\pgfpathlineto{\pgfqpoint{1.299062in}{1.298583in}}%
\pgfpathlineto{\pgfqpoint{1.299261in}{1.298583in}}%
\pgfpathlineto{\pgfqpoint{1.299459in}{1.298583in}}%
\pgfpathlineto{\pgfqpoint{1.299856in}{1.441032in}}%
\pgfpathlineto{\pgfqpoint{1.300452in}{1.339283in}}%
\pgfpathlineto{\pgfqpoint{1.300849in}{1.339283in}}%
\pgfpathlineto{\pgfqpoint{1.301444in}{1.318933in}}%
\pgfpathlineto{\pgfqpoint{1.301841in}{1.563131in}}%
\pgfpathlineto{\pgfqpoint{1.302040in}{1.563131in}}%
\pgfpathlineto{\pgfqpoint{1.302635in}{1.227359in}}%
\pgfpathlineto{\pgfqpoint{1.303032in}{1.308758in}}%
\pgfpathlineto{\pgfqpoint{1.303231in}{1.308758in}}%
\pgfpathlineto{\pgfqpoint{1.304025in}{1.664881in}}%
\pgfpathlineto{\pgfqpoint{1.304223in}{1.451207in}}%
\pgfpathlineto{\pgfqpoint{1.304422in}{1.451207in}}%
\pgfpathlineto{\pgfqpoint{1.305216in}{1.308758in}}%
\pgfpathlineto{\pgfqpoint{1.305017in}{1.532607in}}%
\pgfpathlineto{\pgfqpoint{1.305414in}{1.420682in}}%
\pgfpathlineto{\pgfqpoint{1.305811in}{1.420682in}}%
\pgfpathlineto{\pgfqpoint{1.306010in}{1.491907in}}%
\pgfpathlineto{\pgfqpoint{1.306804in}{1.318933in}}%
\pgfpathlineto{\pgfqpoint{1.307002in}{1.318933in}}%
\pgfpathlineto{\pgfqpoint{1.307598in}{1.288408in}}%
\pgfpathlineto{\pgfqpoint{1.307995in}{1.624181in}}%
\pgfpathlineto{\pgfqpoint{1.308193in}{1.624181in}}%
\pgfpathlineto{\pgfqpoint{1.308789in}{1.339283in}}%
\pgfpathlineto{\pgfqpoint{1.309186in}{1.532607in}}%
\pgfpathlineto{\pgfqpoint{1.309384in}{1.532607in}}%
\pgfpathlineto{\pgfqpoint{1.309384in}{1.329108in}}%
\pgfpathlineto{\pgfqpoint{1.310377in}{1.644531in}}%
\pgfpathlineto{\pgfqpoint{1.310575in}{1.644531in}}%
\pgfpathlineto{\pgfqpoint{1.311568in}{1.247709in}}%
\pgfpathlineto{\pgfqpoint{1.311766in}{1.247709in}}%
\pgfpathlineto{\pgfqpoint{1.312362in}{1.675055in}}%
\pgfpathlineto{\pgfqpoint{1.312759in}{1.522432in}}%
\pgfpathlineto{\pgfqpoint{1.312957in}{1.522432in}}%
\pgfpathlineto{\pgfqpoint{1.313751in}{1.573306in}}%
\pgfpathlineto{\pgfqpoint{1.313354in}{1.288408in}}%
\pgfpathlineto{\pgfqpoint{1.313950in}{1.542781in}}%
\pgfpathlineto{\pgfqpoint{1.314148in}{1.542781in}}%
\pgfpathlineto{\pgfqpoint{1.314545in}{1.491907in}}%
\pgfpathlineto{\pgfqpoint{1.315141in}{1.695405in}}%
\pgfpathlineto{\pgfqpoint{1.315339in}{1.695405in}}%
\pgfpathlineto{\pgfqpoint{1.316332in}{1.491907in}}%
\pgfpathlineto{\pgfqpoint{1.316530in}{1.491907in}}%
\pgfpathlineto{\pgfqpoint{1.316729in}{1.420682in}}%
\pgfpathlineto{\pgfqpoint{1.317523in}{1.725930in}}%
\pgfpathlineto{\pgfqpoint{1.317721in}{1.725930in}}%
\pgfpathlineto{\pgfqpoint{1.317920in}{1.461382in}}%
\pgfpathlineto{\pgfqpoint{1.318714in}{1.614006in}}%
\pgfpathlineto{\pgfqpoint{1.318912in}{1.614006in}}%
\pgfpathlineto{\pgfqpoint{1.318912in}{1.675055in}}%
\pgfpathlineto{\pgfqpoint{1.319111in}{1.441032in}}%
\pgfpathlineto{\pgfqpoint{1.319905in}{1.664881in}}%
\pgfpathlineto{\pgfqpoint{1.320103in}{1.664881in}}%
\pgfpathlineto{\pgfqpoint{1.320103in}{1.786980in}}%
\pgfpathlineto{\pgfqpoint{1.320699in}{1.552956in}}%
\pgfpathlineto{\pgfqpoint{1.321096in}{1.644531in}}%
\pgfpathlineto{\pgfqpoint{1.321294in}{1.644531in}}%
\pgfpathlineto{\pgfqpoint{1.321493in}{1.502082in}}%
\pgfpathlineto{\pgfqpoint{1.322088in}{1.786980in}}%
\pgfpathlineto{\pgfqpoint{1.322287in}{1.664881in}}%
\pgfpathlineto{\pgfqpoint{1.322485in}{1.664881in}}%
\pgfpathlineto{\pgfqpoint{1.322684in}{1.491907in}}%
\pgfpathlineto{\pgfqpoint{1.323279in}{1.797154in}}%
\pgfpathlineto{\pgfqpoint{1.323478in}{1.634356in}}%
\pgfpathlineto{\pgfqpoint{1.323676in}{1.634356in}}%
\pgfpathlineto{\pgfqpoint{1.323875in}{1.766630in}}%
\pgfpathlineto{\pgfqpoint{1.324669in}{1.522432in}}%
\pgfpathlineto{\pgfqpoint{1.324867in}{1.522432in}}%
\pgfpathlineto{\pgfqpoint{1.324867in}{1.756455in}}%
\pgfpathlineto{\pgfqpoint{1.325860in}{1.654706in}}%
\pgfpathlineto{\pgfqpoint{1.326058in}{1.654706in}}%
\pgfpathlineto{\pgfqpoint{1.326058in}{1.624181in}}%
\pgfpathlineto{\pgfqpoint{1.326654in}{1.858204in}}%
\pgfpathlineto{\pgfqpoint{1.327051in}{1.675055in}}%
\pgfpathlineto{\pgfqpoint{1.327249in}{1.675055in}}%
\pgfpathlineto{\pgfqpoint{1.327249in}{1.848029in}}%
\pgfpathlineto{\pgfqpoint{1.328043in}{1.593656in}}%
\pgfpathlineto{\pgfqpoint{1.328241in}{1.634356in}}%
\pgfpathlineto{\pgfqpoint{1.328440in}{1.634356in}}%
\pgfpathlineto{\pgfqpoint{1.328638in}{1.624181in}}%
\pgfpathlineto{\pgfqpoint{1.329432in}{1.837854in}}%
\pgfpathlineto{\pgfqpoint{1.329631in}{1.837854in}}%
\pgfpathlineto{\pgfqpoint{1.330226in}{1.502082in}}%
\pgfpathlineto{\pgfqpoint{1.330028in}{1.878554in}}%
\pgfpathlineto{\pgfqpoint{1.330623in}{1.644531in}}%
\pgfpathlineto{\pgfqpoint{1.330822in}{1.644531in}}%
\pgfpathlineto{\pgfqpoint{1.330822in}{1.878554in}}%
\pgfpathlineto{\pgfqpoint{1.331020in}{1.573306in}}%
\pgfpathlineto{\pgfqpoint{1.331814in}{1.725930in}}%
\pgfpathlineto{\pgfqpoint{1.332013in}{1.725930in}}%
\pgfpathlineto{\pgfqpoint{1.332608in}{1.461382in}}%
\pgfpathlineto{\pgfqpoint{1.332807in}{1.756455in}}%
\pgfpathlineto{\pgfqpoint{1.333005in}{1.654706in}}%
\pgfpathlineto{\pgfqpoint{1.333204in}{1.654706in}}%
\pgfpathlineto{\pgfqpoint{1.333402in}{1.868379in}}%
\pgfpathlineto{\pgfqpoint{1.334196in}{1.715755in}}%
\pgfpathlineto{\pgfqpoint{1.334395in}{1.715755in}}%
\pgfpathlineto{\pgfqpoint{1.334395in}{1.746280in}}%
\pgfpathlineto{\pgfqpoint{1.334990in}{1.583481in}}%
\pgfpathlineto{\pgfqpoint{1.335387in}{1.725930in}}%
\pgfpathlineto{\pgfqpoint{1.335586in}{1.725930in}}%
\pgfpathlineto{\pgfqpoint{1.335586in}{1.868379in}}%
\pgfpathlineto{\pgfqpoint{1.335983in}{1.624181in}}%
\pgfpathlineto{\pgfqpoint{1.336578in}{1.817504in}}%
\pgfpathlineto{\pgfqpoint{1.336777in}{1.817504in}}%
\pgfpathlineto{\pgfqpoint{1.336777in}{1.837854in}}%
\pgfpathlineto{\pgfqpoint{1.337571in}{1.685230in}}%
\pgfpathlineto{\pgfqpoint{1.337769in}{1.807329in}}%
\pgfpathlineto{\pgfqpoint{1.337968in}{1.807329in}}%
\pgfpathlineto{\pgfqpoint{1.337968in}{1.715755in}}%
\pgfpathlineto{\pgfqpoint{1.338762in}{1.858204in}}%
\pgfpathlineto{\pgfqpoint{1.338960in}{1.858204in}}%
\pgfpathlineto{\pgfqpoint{1.339357in}{1.858204in}}%
\pgfpathlineto{\pgfqpoint{1.339357in}{1.980303in}}%
\pgfpathlineto{\pgfqpoint{1.339556in}{1.644531in}}%
\pgfpathlineto{\pgfqpoint{1.340350in}{1.878554in}}%
\pgfpathlineto{\pgfqpoint{1.340548in}{1.878554in}}%
\pgfpathlineto{\pgfqpoint{1.340945in}{1.746280in}}%
\pgfpathlineto{\pgfqpoint{1.340747in}{2.061702in}}%
\pgfpathlineto{\pgfqpoint{1.341541in}{1.980303in}}%
\pgfpathlineto{\pgfqpoint{1.341739in}{1.980303in}}%
\pgfpathlineto{\pgfqpoint{1.342136in}{1.725930in}}%
\pgfpathlineto{\pgfqpoint{1.342533in}{2.051527in}}%
\pgfpathlineto{\pgfqpoint{1.342732in}{1.929428in}}%
\pgfpathlineto{\pgfqpoint{1.342930in}{1.929428in}}%
\pgfpathlineto{\pgfqpoint{1.343327in}{2.021003in}}%
\pgfpathlineto{\pgfqpoint{1.343923in}{1.807329in}}%
\pgfpathlineto{\pgfqpoint{1.344121in}{1.807329in}}%
\pgfpathlineto{\pgfqpoint{1.344518in}{2.010828in}}%
\pgfpathlineto{\pgfqpoint{1.344320in}{1.695405in}}%
\pgfpathlineto{\pgfqpoint{1.345114in}{1.909079in}}%
\pgfpathlineto{\pgfqpoint{1.345312in}{1.909079in}}%
\pgfpathlineto{\pgfqpoint{1.345312in}{1.695405in}}%
\pgfpathlineto{\pgfqpoint{1.345709in}{1.959953in}}%
\pgfpathlineto{\pgfqpoint{1.346305in}{1.858204in}}%
\pgfpathlineto{\pgfqpoint{1.346503in}{1.858204in}}%
\pgfpathlineto{\pgfqpoint{1.347099in}{1.756455in}}%
\pgfpathlineto{\pgfqpoint{1.346702in}{1.898904in}}%
\pgfpathlineto{\pgfqpoint{1.347496in}{1.797154in}}%
\pgfpathlineto{\pgfqpoint{1.347694in}{1.797154in}}%
\pgfpathlineto{\pgfqpoint{1.348091in}{1.542781in}}%
\pgfpathlineto{\pgfqpoint{1.348687in}{2.092227in}}%
\pgfpathlineto{\pgfqpoint{1.348885in}{2.092227in}}%
\pgfpathlineto{\pgfqpoint{1.349679in}{1.725930in}}%
\pgfpathlineto{\pgfqpoint{1.349878in}{1.939603in}}%
\pgfpathlineto{\pgfqpoint{1.350076in}{1.939603in}}%
\pgfpathlineto{\pgfqpoint{1.350076in}{1.756455in}}%
\pgfpathlineto{\pgfqpoint{1.350473in}{1.980303in}}%
\pgfpathlineto{\pgfqpoint{1.351069in}{1.797154in}}%
\pgfpathlineto{\pgfqpoint{1.351267in}{1.797154in}}%
\pgfpathlineto{\pgfqpoint{1.351267in}{1.949778in}}%
\pgfpathlineto{\pgfqpoint{1.351664in}{1.715755in}}%
\pgfpathlineto{\pgfqpoint{1.352260in}{1.868379in}}%
\pgfpathlineto{\pgfqpoint{1.352458in}{1.868379in}}%
\pgfpathlineto{\pgfqpoint{1.352657in}{1.786980in}}%
\pgfpathlineto{\pgfqpoint{1.353451in}{1.990478in}}%
\pgfpathlineto{\pgfqpoint{1.353649in}{1.990478in}}%
\pgfpathlineto{\pgfqpoint{1.353649in}{2.031178in}}%
\pgfpathlineto{\pgfqpoint{1.354245in}{1.685230in}}%
\pgfpathlineto{\pgfqpoint{1.354642in}{1.919254in}}%
\pgfpathlineto{\pgfqpoint{1.354840in}{1.919254in}}%
\pgfpathlineto{\pgfqpoint{1.354840in}{2.092227in}}%
\pgfpathlineto{\pgfqpoint{1.355237in}{1.807329in}}%
\pgfpathlineto{\pgfqpoint{1.355833in}{1.990478in}}%
\pgfpathlineto{\pgfqpoint{1.356031in}{1.990478in}}%
\pgfpathlineto{\pgfqpoint{1.356031in}{1.797154in}}%
\pgfpathlineto{\pgfqpoint{1.356428in}{2.132927in}}%
\pgfpathlineto{\pgfqpoint{1.357024in}{1.878554in}}%
\pgfpathlineto{\pgfqpoint{1.357222in}{1.878554in}}%
\pgfpathlineto{\pgfqpoint{1.358016in}{2.173627in}}%
\pgfpathlineto{\pgfqpoint{1.357421in}{1.786980in}}%
\pgfpathlineto{\pgfqpoint{1.358215in}{2.041353in}}%
\pgfpathlineto{\pgfqpoint{1.358413in}{2.041353in}}%
\pgfpathlineto{\pgfqpoint{1.359207in}{1.898904in}}%
\pgfpathlineto{\pgfqpoint{1.359406in}{1.970128in}}%
\pgfpathlineto{\pgfqpoint{1.359604in}{1.970128in}}%
\pgfpathlineto{\pgfqpoint{1.360398in}{2.163452in}}%
\pgfpathlineto{\pgfqpoint{1.360200in}{1.919254in}}%
\pgfpathlineto{\pgfqpoint{1.360596in}{1.929428in}}%
\pgfpathlineto{\pgfqpoint{1.360795in}{1.929428in}}%
\pgfpathlineto{\pgfqpoint{1.360993in}{2.102402in}}%
\pgfpathlineto{\pgfqpoint{1.361192in}{1.786980in}}%
\pgfpathlineto{\pgfqpoint{1.361787in}{1.878554in}}%
\pgfpathlineto{\pgfqpoint{1.361986in}{1.878554in}}%
\pgfpathlineto{\pgfqpoint{1.362184in}{2.163452in}}%
\pgfpathlineto{\pgfqpoint{1.362978in}{2.021003in}}%
\pgfpathlineto{\pgfqpoint{1.363177in}{2.021003in}}%
\pgfpathlineto{\pgfqpoint{1.363574in}{1.970128in}}%
\pgfpathlineto{\pgfqpoint{1.364169in}{2.061702in}}%
\pgfpathlineto{\pgfqpoint{1.364368in}{2.061702in}}%
\pgfpathlineto{\pgfqpoint{1.364368in}{2.316075in}}%
\pgfpathlineto{\pgfqpoint{1.364566in}{1.837854in}}%
\pgfpathlineto{\pgfqpoint{1.365360in}{2.061702in}}%
\pgfpathlineto{\pgfqpoint{1.365559in}{2.061702in}}%
\pgfpathlineto{\pgfqpoint{1.365559in}{1.858204in}}%
\pgfpathlineto{\pgfqpoint{1.366551in}{1.878554in}}%
\pgfpathlineto{\pgfqpoint{1.366750in}{1.878554in}}%
\pgfpathlineto{\pgfqpoint{1.367544in}{1.776805in}}%
\pgfpathlineto{\pgfqpoint{1.367742in}{2.214326in}}%
\pgfpathlineto{\pgfqpoint{1.367941in}{2.214326in}}%
\pgfpathlineto{\pgfqpoint{1.368536in}{1.970128in}}%
\pgfpathlineto{\pgfqpoint{1.368933in}{2.132927in}}%
\pgfpathlineto{\pgfqpoint{1.369132in}{2.132927in}}%
\pgfpathlineto{\pgfqpoint{1.369727in}{1.878554in}}%
\pgfpathlineto{\pgfqpoint{1.370124in}{1.878554in}}%
\pgfpathlineto{\pgfqpoint{1.370323in}{1.878554in}}%
\pgfpathlineto{\pgfqpoint{1.370720in}{2.153277in}}%
\pgfpathlineto{\pgfqpoint{1.370918in}{1.817504in}}%
\pgfpathlineto{\pgfqpoint{1.371315in}{2.112577in}}%
\pgfpathlineto{\pgfqpoint{1.371514in}{2.112577in}}%
\pgfpathlineto{\pgfqpoint{1.372109in}{1.878554in}}%
\pgfpathlineto{\pgfqpoint{1.372506in}{2.122752in}}%
\pgfpathlineto{\pgfqpoint{1.372705in}{2.122752in}}%
\pgfpathlineto{\pgfqpoint{1.373697in}{1.888729in}}%
\pgfpathlineto{\pgfqpoint{1.373896in}{1.888729in}}%
\pgfpathlineto{\pgfqpoint{1.374094in}{2.183801in}}%
\pgfpathlineto{\pgfqpoint{1.374888in}{2.061702in}}%
\pgfpathlineto{\pgfqpoint{1.375087in}{2.061702in}}%
\pgfpathlineto{\pgfqpoint{1.375087in}{1.837854in}}%
\pgfpathlineto{\pgfqpoint{1.376079in}{2.092227in}}%
\pgfpathlineto{\pgfqpoint{1.376278in}{2.092227in}}%
\pgfpathlineto{\pgfqpoint{1.376278in}{1.919254in}}%
\pgfpathlineto{\pgfqpoint{1.376675in}{2.163452in}}%
\pgfpathlineto{\pgfqpoint{1.377270in}{2.092227in}}%
\pgfpathlineto{\pgfqpoint{1.377469in}{2.092227in}}%
\pgfpathlineto{\pgfqpoint{1.377469in}{1.919254in}}%
\pgfpathlineto{\pgfqpoint{1.378461in}{1.970128in}}%
\pgfpathlineto{\pgfqpoint{1.378660in}{1.970128in}}%
\pgfpathlineto{\pgfqpoint{1.378660in}{1.919254in}}%
\pgfpathlineto{\pgfqpoint{1.379652in}{2.224501in}}%
\pgfpathlineto{\pgfqpoint{1.379851in}{2.224501in}}%
\pgfpathlineto{\pgfqpoint{1.380843in}{1.878554in}}%
\pgfpathlineto{\pgfqpoint{1.381042in}{1.878554in}}%
\pgfpathlineto{\pgfqpoint{1.381836in}{2.102402in}}%
\pgfpathlineto{\pgfqpoint{1.382034in}{1.980303in}}%
\pgfpathlineto{\pgfqpoint{1.382233in}{1.980303in}}%
\pgfpathlineto{\pgfqpoint{1.382630in}{2.193976in}}%
\pgfpathlineto{\pgfqpoint{1.383225in}{1.939603in}}%
\pgfpathlineto{\pgfqpoint{1.383424in}{1.939603in}}%
\pgfpathlineto{\pgfqpoint{1.384019in}{2.255026in}}%
\pgfpathlineto{\pgfqpoint{1.384416in}{1.848029in}}%
\pgfpathlineto{\pgfqpoint{1.384615in}{1.848029in}}%
\pgfpathlineto{\pgfqpoint{1.384813in}{2.183801in}}%
\pgfpathlineto{\pgfqpoint{1.385607in}{2.021003in}}%
\pgfpathlineto{\pgfqpoint{1.386004in}{2.021003in}}%
\pgfpathlineto{\pgfqpoint{1.386203in}{1.786980in}}%
\pgfpathlineto{\pgfqpoint{1.386401in}{2.153277in}}%
\pgfpathlineto{\pgfqpoint{1.386997in}{1.929428in}}%
\pgfpathlineto{\pgfqpoint{1.387195in}{1.929428in}}%
\pgfpathlineto{\pgfqpoint{1.387592in}{2.275376in}}%
\pgfpathlineto{\pgfqpoint{1.387394in}{1.817504in}}%
\pgfpathlineto{\pgfqpoint{1.388188in}{2.122752in}}%
\pgfpathlineto{\pgfqpoint{1.388386in}{2.122752in}}%
\pgfpathlineto{\pgfqpoint{1.388585in}{1.878554in}}%
\pgfpathlineto{\pgfqpoint{1.389379in}{2.163452in}}%
\pgfpathlineto{\pgfqpoint{1.389577in}{2.163452in}}%
\pgfpathlineto{\pgfqpoint{1.390173in}{1.817504in}}%
\pgfpathlineto{\pgfqpoint{1.390570in}{2.244851in}}%
\pgfpathlineto{\pgfqpoint{1.390768in}{2.244851in}}%
\pgfpathlineto{\pgfqpoint{1.390768in}{1.878554in}}%
\pgfpathlineto{\pgfqpoint{1.391761in}{2.051527in}}%
\pgfpathlineto{\pgfqpoint{1.391959in}{2.051527in}}%
\pgfpathlineto{\pgfqpoint{1.392356in}{2.122752in}}%
\pgfpathlineto{\pgfqpoint{1.392158in}{1.888729in}}%
\pgfpathlineto{\pgfqpoint{1.392951in}{2.112577in}}%
\pgfpathlineto{\pgfqpoint{1.393150in}{2.112577in}}%
\pgfpathlineto{\pgfqpoint{1.393547in}{1.848029in}}%
\pgfpathlineto{\pgfqpoint{1.394142in}{2.041353in}}%
\pgfpathlineto{\pgfqpoint{1.394341in}{2.041353in}}%
\pgfpathlineto{\pgfqpoint{1.394738in}{2.285551in}}%
\pgfpathlineto{\pgfqpoint{1.394936in}{1.929428in}}%
\pgfpathlineto{\pgfqpoint{1.395333in}{1.949778in}}%
\pgfpathlineto{\pgfqpoint{1.395532in}{1.949778in}}%
\pgfpathlineto{\pgfqpoint{1.395730in}{1.878554in}}%
\pgfpathlineto{\pgfqpoint{1.396524in}{2.316075in}}%
\pgfpathlineto{\pgfqpoint{1.396723in}{2.316075in}}%
\pgfpathlineto{\pgfqpoint{1.397517in}{1.858204in}}%
\pgfpathlineto{\pgfqpoint{1.397715in}{2.021003in}}%
\pgfpathlineto{\pgfqpoint{1.397914in}{2.021003in}}%
\pgfpathlineto{\pgfqpoint{1.398311in}{2.285551in}}%
\pgfpathlineto{\pgfqpoint{1.398906in}{1.919254in}}%
\pgfpathlineto{\pgfqpoint{1.399105in}{1.919254in}}%
\pgfpathlineto{\pgfqpoint{1.399899in}{2.336425in}}%
\pgfpathlineto{\pgfqpoint{1.399502in}{1.848029in}}%
\pgfpathlineto{\pgfqpoint{1.400097in}{2.071877in}}%
\pgfpathlineto{\pgfqpoint{1.400296in}{2.071877in}}%
\pgfpathlineto{\pgfqpoint{1.400296in}{1.868379in}}%
\pgfpathlineto{\pgfqpoint{1.401090in}{2.438174in}}%
\pgfpathlineto{\pgfqpoint{1.401288in}{2.305900in}}%
\pgfpathlineto{\pgfqpoint{1.401487in}{2.305900in}}%
\pgfpathlineto{\pgfqpoint{1.402281in}{1.837854in}}%
\pgfpathlineto{\pgfqpoint{1.402479in}{2.051527in}}%
\pgfpathlineto{\pgfqpoint{1.402678in}{2.051527in}}%
\pgfpathlineto{\pgfqpoint{1.403472in}{1.766630in}}%
\pgfpathlineto{\pgfqpoint{1.403075in}{2.183801in}}%
\pgfpathlineto{\pgfqpoint{1.403670in}{1.929428in}}%
\pgfpathlineto{\pgfqpoint{1.403869in}{1.929428in}}%
\pgfpathlineto{\pgfqpoint{1.403869in}{2.224501in}}%
\pgfpathlineto{\pgfqpoint{1.404861in}{1.970128in}}%
\pgfpathlineto{\pgfqpoint{1.405060in}{1.970128in}}%
\pgfpathlineto{\pgfqpoint{1.405060in}{1.817504in}}%
\pgfpathlineto{\pgfqpoint{1.406052in}{2.132927in}}%
\pgfpathlineto{\pgfqpoint{1.406251in}{2.132927in}}%
\pgfpathlineto{\pgfqpoint{1.406846in}{2.244851in}}%
\pgfpathlineto{\pgfqpoint{1.407243in}{1.959953in}}%
\pgfpathlineto{\pgfqpoint{1.407442in}{1.959953in}}%
\pgfpathlineto{\pgfqpoint{1.407442in}{2.244851in}}%
\pgfpathlineto{\pgfqpoint{1.407640in}{1.878554in}}%
\pgfpathlineto{\pgfqpoint{1.408434in}{1.949778in}}%
\pgfpathlineto{\pgfqpoint{1.408633in}{1.949778in}}%
\pgfpathlineto{\pgfqpoint{1.408633in}{2.163452in}}%
\pgfpathlineto{\pgfqpoint{1.409625in}{1.858204in}}%
\pgfpathlineto{\pgfqpoint{1.409824in}{1.858204in}}%
\pgfpathlineto{\pgfqpoint{1.410816in}{2.366950in}}%
\pgfpathlineto{\pgfqpoint{1.411015in}{2.366950in}}%
\pgfpathlineto{\pgfqpoint{1.411610in}{1.909079in}}%
\pgfpathlineto{\pgfqpoint{1.412007in}{2.183801in}}%
\pgfpathlineto{\pgfqpoint{1.412206in}{2.183801in}}%
\pgfpathlineto{\pgfqpoint{1.412206in}{2.255026in}}%
\pgfpathlineto{\pgfqpoint{1.413000in}{1.878554in}}%
\pgfpathlineto{\pgfqpoint{1.413198in}{2.143102in}}%
\pgfpathlineto{\pgfqpoint{1.413397in}{2.143102in}}%
\pgfpathlineto{\pgfqpoint{1.413397in}{1.888729in}}%
\pgfpathlineto{\pgfqpoint{1.413595in}{2.255026in}}%
\pgfpathlineto{\pgfqpoint{1.414389in}{2.051527in}}%
\pgfpathlineto{\pgfqpoint{1.414588in}{2.051527in}}%
\pgfpathlineto{\pgfqpoint{1.414588in}{2.265201in}}%
\pgfpathlineto{\pgfqpoint{1.414786in}{1.970128in}}%
\pgfpathlineto{\pgfqpoint{1.415580in}{2.102402in}}%
\pgfpathlineto{\pgfqpoint{1.415779in}{2.102402in}}%
\pgfpathlineto{\pgfqpoint{1.415977in}{1.970128in}}%
\pgfpathlineto{\pgfqpoint{1.416771in}{2.285551in}}%
\pgfpathlineto{\pgfqpoint{1.416970in}{2.285551in}}%
\pgfpathlineto{\pgfqpoint{1.417168in}{1.898904in}}%
\pgfpathlineto{\pgfqpoint{1.417962in}{2.112577in}}%
\pgfpathlineto{\pgfqpoint{1.418161in}{2.112577in}}%
\pgfpathlineto{\pgfqpoint{1.418359in}{1.898904in}}%
\pgfpathlineto{\pgfqpoint{1.418558in}{2.275376in}}%
\pgfpathlineto{\pgfqpoint{1.419153in}{2.173627in}}%
\pgfpathlineto{\pgfqpoint{1.419352in}{2.173627in}}%
\pgfpathlineto{\pgfqpoint{1.419352in}{2.255026in}}%
\pgfpathlineto{\pgfqpoint{1.419947in}{2.102402in}}%
\pgfpathlineto{\pgfqpoint{1.420344in}{2.122752in}}%
\pgfpathlineto{\pgfqpoint{1.420543in}{2.122752in}}%
\pgfpathlineto{\pgfqpoint{1.420741in}{2.193976in}}%
\pgfpathlineto{\pgfqpoint{1.421535in}{1.888729in}}%
\pgfpathlineto{\pgfqpoint{1.421734in}{1.888729in}}%
\pgfpathlineto{\pgfqpoint{1.422131in}{2.193976in}}%
\pgfpathlineto{\pgfqpoint{1.421932in}{1.858204in}}%
\pgfpathlineto{\pgfqpoint{1.422726in}{2.163452in}}%
\pgfpathlineto{\pgfqpoint{1.422925in}{2.163452in}}%
\pgfpathlineto{\pgfqpoint{1.423123in}{1.909079in}}%
\pgfpathlineto{\pgfqpoint{1.423520in}{2.204151in}}%
\pgfpathlineto{\pgfqpoint{1.423917in}{2.031178in}}%
\pgfpathlineto{\pgfqpoint{1.424116in}{2.031178in}}%
\pgfpathlineto{\pgfqpoint{1.424513in}{2.204151in}}%
\pgfpathlineto{\pgfqpoint{1.424910in}{1.868379in}}%
\pgfpathlineto{\pgfqpoint{1.425108in}{1.959953in}}%
\pgfpathlineto{\pgfqpoint{1.425505in}{1.959953in}}%
\pgfpathlineto{\pgfqpoint{1.425505in}{1.878554in}}%
\pgfpathlineto{\pgfqpoint{1.426299in}{2.346600in}}%
\pgfpathlineto{\pgfqpoint{1.426497in}{2.316075in}}%
\pgfpathlineto{\pgfqpoint{1.426696in}{2.316075in}}%
\pgfpathlineto{\pgfqpoint{1.426696in}{1.939603in}}%
\pgfpathlineto{\pgfqpoint{1.427688in}{2.193976in}}%
\pgfpathlineto{\pgfqpoint{1.427887in}{2.193976in}}%
\pgfpathlineto{\pgfqpoint{1.428284in}{1.939603in}}%
\pgfpathlineto{\pgfqpoint{1.428879in}{2.041353in}}%
\pgfpathlineto{\pgfqpoint{1.429078in}{2.041353in}}%
\pgfpathlineto{\pgfqpoint{1.429475in}{1.929428in}}%
\pgfpathlineto{\pgfqpoint{1.429276in}{2.214326in}}%
\pgfpathlineto{\pgfqpoint{1.430070in}{2.021003in}}%
\pgfpathlineto{\pgfqpoint{1.430269in}{2.021003in}}%
\pgfpathlineto{\pgfqpoint{1.430467in}{1.888729in}}%
\pgfpathlineto{\pgfqpoint{1.430666in}{2.224501in}}%
\pgfpathlineto{\pgfqpoint{1.431261in}{2.204151in}}%
\pgfpathlineto{\pgfqpoint{1.431460in}{2.204151in}}%
\pgfpathlineto{\pgfqpoint{1.432055in}{1.919254in}}%
\pgfpathlineto{\pgfqpoint{1.432254in}{2.326250in}}%
\pgfpathlineto{\pgfqpoint{1.432452in}{2.000653in}}%
\pgfpathlineto{\pgfqpoint{1.432651in}{2.000653in}}%
\pgfpathlineto{\pgfqpoint{1.433445in}{2.265201in}}%
\pgfpathlineto{\pgfqpoint{1.433048in}{1.868379in}}%
\pgfpathlineto{\pgfqpoint{1.433643in}{2.193976in}}%
\pgfpathlineto{\pgfqpoint{1.433842in}{2.193976in}}%
\pgfpathlineto{\pgfqpoint{1.434437in}{1.786980in}}%
\pgfpathlineto{\pgfqpoint{1.434636in}{2.265201in}}%
\pgfpathlineto{\pgfqpoint{1.434834in}{2.031178in}}%
\pgfpathlineto{\pgfqpoint{1.435033in}{2.031178in}}%
\pgfpathlineto{\pgfqpoint{1.435430in}{1.939603in}}%
\pgfpathlineto{\pgfqpoint{1.435827in}{2.214326in}}%
\pgfpathlineto{\pgfqpoint{1.436025in}{2.132927in}}%
\pgfpathlineto{\pgfqpoint{1.436224in}{2.132927in}}%
\pgfpathlineto{\pgfqpoint{1.436224in}{2.010828in}}%
\pgfpathlineto{\pgfqpoint{1.436621in}{2.183801in}}%
\pgfpathlineto{\pgfqpoint{1.437216in}{2.041353in}}%
\pgfpathlineto{\pgfqpoint{1.437415in}{2.041353in}}%
\pgfpathlineto{\pgfqpoint{1.437812in}{1.878554in}}%
\pgfpathlineto{\pgfqpoint{1.438407in}{2.193976in}}%
\pgfpathlineto{\pgfqpoint{1.438606in}{2.193976in}}%
\pgfpathlineto{\pgfqpoint{1.439201in}{2.224501in}}%
\pgfpathlineto{\pgfqpoint{1.439598in}{1.786980in}}%
\pgfpathlineto{\pgfqpoint{1.439797in}{1.786980in}}%
\pgfpathlineto{\pgfqpoint{1.440591in}{2.275376in}}%
\pgfpathlineto{\pgfqpoint{1.440789in}{1.848029in}}%
\pgfpathlineto{\pgfqpoint{1.440988in}{1.848029in}}%
\pgfpathlineto{\pgfqpoint{1.441583in}{2.193976in}}%
\pgfpathlineto{\pgfqpoint{1.441980in}{2.092227in}}%
\pgfpathlineto{\pgfqpoint{1.442179in}{2.092227in}}%
\pgfpathlineto{\pgfqpoint{1.442774in}{1.848029in}}%
\pgfpathlineto{\pgfqpoint{1.442576in}{2.214326in}}%
\pgfpathlineto{\pgfqpoint{1.443171in}{1.858204in}}%
\pgfpathlineto{\pgfqpoint{1.443370in}{1.858204in}}%
\pgfpathlineto{\pgfqpoint{1.444362in}{2.163452in}}%
\pgfpathlineto{\pgfqpoint{1.444561in}{2.163452in}}%
\pgfpathlineto{\pgfqpoint{1.445156in}{2.183801in}}%
\pgfpathlineto{\pgfqpoint{1.445553in}{1.776805in}}%
\pgfpathlineto{\pgfqpoint{1.445752in}{1.776805in}}%
\pgfpathlineto{\pgfqpoint{1.445752in}{2.204151in}}%
\pgfpathlineto{\pgfqpoint{1.446744in}{2.010828in}}%
\pgfpathlineto{\pgfqpoint{1.446943in}{2.010828in}}%
\pgfpathlineto{\pgfqpoint{1.447737in}{2.102402in}}%
\pgfpathlineto{\pgfqpoint{1.447340in}{2.000653in}}%
\pgfpathlineto{\pgfqpoint{1.447935in}{2.071877in}}%
\pgfpathlineto{\pgfqpoint{1.448134in}{2.071877in}}%
\pgfpathlineto{\pgfqpoint{1.448531in}{2.173627in}}%
\pgfpathlineto{\pgfqpoint{1.449126in}{1.959953in}}%
\pgfpathlineto{\pgfqpoint{1.449325in}{1.959953in}}%
\pgfpathlineto{\pgfqpoint{1.450119in}{1.797154in}}%
\pgfpathlineto{\pgfqpoint{1.450317in}{2.143102in}}%
\pgfpathlineto{\pgfqpoint{1.450516in}{2.143102in}}%
\pgfpathlineto{\pgfqpoint{1.450714in}{1.919254in}}%
\pgfpathlineto{\pgfqpoint{1.451508in}{2.143102in}}%
\pgfpathlineto{\pgfqpoint{1.451707in}{2.143102in}}%
\pgfpathlineto{\pgfqpoint{1.452104in}{1.858204in}}%
\pgfpathlineto{\pgfqpoint{1.452699in}{1.909079in}}%
\pgfpathlineto{\pgfqpoint{1.452898in}{1.909079in}}%
\pgfpathlineto{\pgfqpoint{1.453295in}{2.102402in}}%
\pgfpathlineto{\pgfqpoint{1.453096in}{1.888729in}}%
\pgfpathlineto{\pgfqpoint{1.453890in}{1.980303in}}%
\pgfpathlineto{\pgfqpoint{1.454089in}{1.980303in}}%
\pgfpathlineto{\pgfqpoint{1.454486in}{2.071877in}}%
\pgfpathlineto{\pgfqpoint{1.454287in}{1.970128in}}%
\pgfpathlineto{\pgfqpoint{1.455081in}{2.051527in}}%
\pgfpathlineto{\pgfqpoint{1.455280in}{2.051527in}}%
\pgfpathlineto{\pgfqpoint{1.455280in}{1.959953in}}%
\pgfpathlineto{\pgfqpoint{1.456272in}{2.183801in}}%
\pgfpathlineto{\pgfqpoint{1.456471in}{2.183801in}}%
\pgfpathlineto{\pgfqpoint{1.456669in}{1.898904in}}%
\pgfpathlineto{\pgfqpoint{1.456868in}{2.204151in}}%
\pgfpathlineto{\pgfqpoint{1.457463in}{2.000653in}}%
\pgfpathlineto{\pgfqpoint{1.457661in}{2.000653in}}%
\pgfpathlineto{\pgfqpoint{1.458455in}{1.837854in}}%
\pgfpathlineto{\pgfqpoint{1.458257in}{2.122752in}}%
\pgfpathlineto{\pgfqpoint{1.458654in}{1.919254in}}%
\pgfpathlineto{\pgfqpoint{1.458852in}{1.919254in}}%
\pgfpathlineto{\pgfqpoint{1.458852in}{2.102402in}}%
\pgfpathlineto{\pgfqpoint{1.459249in}{1.868379in}}%
\pgfpathlineto{\pgfqpoint{1.459845in}{1.909079in}}%
\pgfpathlineto{\pgfqpoint{1.460043in}{1.909079in}}%
\pgfpathlineto{\pgfqpoint{1.460043in}{1.868379in}}%
\pgfpathlineto{\pgfqpoint{1.461036in}{2.295726in}}%
\pgfpathlineto{\pgfqpoint{1.461234in}{2.295726in}}%
\pgfpathlineto{\pgfqpoint{1.462227in}{1.909079in}}%
\pgfpathlineto{\pgfqpoint{1.462425in}{1.909079in}}%
\pgfpathlineto{\pgfqpoint{1.462425in}{2.092227in}}%
\pgfpathlineto{\pgfqpoint{1.463021in}{1.858204in}}%
\pgfpathlineto{\pgfqpoint{1.463418in}{1.949778in}}%
\pgfpathlineto{\pgfqpoint{1.463616in}{1.949778in}}%
\pgfpathlineto{\pgfqpoint{1.463815in}{2.295726in}}%
\pgfpathlineto{\pgfqpoint{1.464212in}{1.929428in}}%
\pgfpathlineto{\pgfqpoint{1.464609in}{2.071877in}}%
\pgfpathlineto{\pgfqpoint{1.464807in}{2.071877in}}%
\pgfpathlineto{\pgfqpoint{1.464807in}{1.858204in}}%
\pgfpathlineto{\pgfqpoint{1.465204in}{2.204151in}}%
\pgfpathlineto{\pgfqpoint{1.465800in}{1.919254in}}%
\pgfpathlineto{\pgfqpoint{1.465998in}{1.919254in}}%
\pgfpathlineto{\pgfqpoint{1.466395in}{2.082052in}}%
\pgfpathlineto{\pgfqpoint{1.466991in}{1.959953in}}%
\pgfpathlineto{\pgfqpoint{1.467189in}{1.959953in}}%
\pgfpathlineto{\pgfqpoint{1.468182in}{2.112577in}}%
\pgfpathlineto{\pgfqpoint{1.468380in}{2.112577in}}%
\pgfpathlineto{\pgfqpoint{1.468777in}{1.949778in}}%
\pgfpathlineto{\pgfqpoint{1.469373in}{1.949778in}}%
\pgfpathlineto{\pgfqpoint{1.469571in}{1.949778in}}%
\pgfpathlineto{\pgfqpoint{1.469968in}{1.797154in}}%
\pgfpathlineto{\pgfqpoint{1.470564in}{2.193976in}}%
\pgfpathlineto{\pgfqpoint{1.470762in}{2.193976in}}%
\pgfpathlineto{\pgfqpoint{1.471358in}{1.888729in}}%
\pgfpathlineto{\pgfqpoint{1.471755in}{2.122752in}}%
\pgfpathlineto{\pgfqpoint{1.471953in}{2.122752in}}%
\pgfpathlineto{\pgfqpoint{1.471953in}{1.827679in}}%
\pgfpathlineto{\pgfqpoint{1.472747in}{2.234676in}}%
\pgfpathlineto{\pgfqpoint{1.472946in}{2.041353in}}%
\pgfpathlineto{\pgfqpoint{1.473144in}{2.041353in}}%
\pgfpathlineto{\pgfqpoint{1.473144in}{1.939603in}}%
\pgfpathlineto{\pgfqpoint{1.473938in}{2.132927in}}%
\pgfpathlineto{\pgfqpoint{1.474137in}{1.959953in}}%
\pgfpathlineto{\pgfqpoint{1.474335in}{1.959953in}}%
\pgfpathlineto{\pgfqpoint{1.474732in}{1.858204in}}%
\pgfpathlineto{\pgfqpoint{1.475328in}{2.204151in}}%
\pgfpathlineto{\pgfqpoint{1.475526in}{2.204151in}}%
\pgfpathlineto{\pgfqpoint{1.475923in}{1.949778in}}%
\pgfpathlineto{\pgfqpoint{1.476519in}{2.071877in}}%
\pgfpathlineto{\pgfqpoint{1.476717in}{2.071877in}}%
\pgfpathlineto{\pgfqpoint{1.477511in}{1.868379in}}%
\pgfpathlineto{\pgfqpoint{1.477710in}{2.010828in}}%
\pgfpathlineto{\pgfqpoint{1.477908in}{2.010828in}}%
\pgfpathlineto{\pgfqpoint{1.477908in}{2.255026in}}%
\pgfpathlineto{\pgfqpoint{1.478305in}{1.919254in}}%
\pgfpathlineto{\pgfqpoint{1.478901in}{2.183801in}}%
\pgfpathlineto{\pgfqpoint{1.479298in}{2.183801in}}%
\pgfpathlineto{\pgfqpoint{1.479893in}{1.898904in}}%
\pgfpathlineto{\pgfqpoint{1.480290in}{2.143102in}}%
\pgfpathlineto{\pgfqpoint{1.480489in}{2.143102in}}%
\pgfpathlineto{\pgfqpoint{1.480489in}{1.909079in}}%
\pgfpathlineto{\pgfqpoint{1.481084in}{2.244851in}}%
\pgfpathlineto{\pgfqpoint{1.481481in}{2.021003in}}%
\pgfpathlineto{\pgfqpoint{1.481680in}{2.021003in}}%
\pgfpathlineto{\pgfqpoint{1.482275in}{2.244851in}}%
\pgfpathlineto{\pgfqpoint{1.482077in}{1.898904in}}%
\pgfpathlineto{\pgfqpoint{1.482672in}{2.041353in}}%
\pgfpathlineto{\pgfqpoint{1.482871in}{2.041353in}}%
\pgfpathlineto{\pgfqpoint{1.483069in}{1.797154in}}%
\pgfpathlineto{\pgfqpoint{1.483665in}{2.275376in}}%
\pgfpathlineto{\pgfqpoint{1.483863in}{2.102402in}}%
\pgfpathlineto{\pgfqpoint{1.484062in}{2.102402in}}%
\pgfpathlineto{\pgfqpoint{1.484657in}{2.285551in}}%
\pgfpathlineto{\pgfqpoint{1.484856in}{2.051527in}}%
\pgfpathlineto{\pgfqpoint{1.485054in}{2.122752in}}%
\pgfpathlineto{\pgfqpoint{1.485253in}{2.122752in}}%
\pgfpathlineto{\pgfqpoint{1.485650in}{2.224501in}}%
\pgfpathlineto{\pgfqpoint{1.485848in}{1.929428in}}%
\pgfpathlineto{\pgfqpoint{1.486245in}{2.143102in}}%
\pgfpathlineto{\pgfqpoint{1.486444in}{2.143102in}}%
\pgfpathlineto{\pgfqpoint{1.486642in}{1.858204in}}%
\pgfpathlineto{\pgfqpoint{1.486841in}{2.326250in}}%
\pgfpathlineto{\pgfqpoint{1.487436in}{2.214326in}}%
\pgfpathlineto{\pgfqpoint{1.487635in}{2.214326in}}%
\pgfpathlineto{\pgfqpoint{1.487635in}{2.326250in}}%
\pgfpathlineto{\pgfqpoint{1.488429in}{1.949778in}}%
\pgfpathlineto{\pgfqpoint{1.488627in}{2.244851in}}%
\pgfpathlineto{\pgfqpoint{1.488826in}{2.244851in}}%
\pgfpathlineto{\pgfqpoint{1.488826in}{2.000653in}}%
\pgfpathlineto{\pgfqpoint{1.489223in}{2.305900in}}%
\pgfpathlineto{\pgfqpoint{1.489818in}{2.193976in}}%
\pgfpathlineto{\pgfqpoint{1.490016in}{2.193976in}}%
\pgfpathlineto{\pgfqpoint{1.490413in}{2.000653in}}%
\pgfpathlineto{\pgfqpoint{1.490810in}{2.529749in}}%
\pgfpathlineto{\pgfqpoint{1.491009in}{2.214326in}}%
\pgfpathlineto{\pgfqpoint{1.491207in}{2.214326in}}%
\pgfpathlineto{\pgfqpoint{1.491207in}{2.102402in}}%
\pgfpathlineto{\pgfqpoint{1.492200in}{2.102402in}}%
\pgfpathlineto{\pgfqpoint{1.492398in}{2.102402in}}%
\pgfpathlineto{\pgfqpoint{1.492398in}{2.316075in}}%
\pgfpathlineto{\pgfqpoint{1.492795in}{2.092227in}}%
\pgfpathlineto{\pgfqpoint{1.493391in}{2.112577in}}%
\pgfpathlineto{\pgfqpoint{1.493589in}{2.112577in}}%
\pgfpathlineto{\pgfqpoint{1.493589in}{2.356775in}}%
\pgfpathlineto{\pgfqpoint{1.494383in}{2.031178in}}%
\pgfpathlineto{\pgfqpoint{1.494582in}{2.082052in}}%
\pgfpathlineto{\pgfqpoint{1.494780in}{2.082052in}}%
\pgfpathlineto{\pgfqpoint{1.495376in}{2.265201in}}%
\pgfpathlineto{\pgfqpoint{1.494979in}{2.051527in}}%
\pgfpathlineto{\pgfqpoint{1.495773in}{2.244851in}}%
\pgfpathlineto{\pgfqpoint{1.495971in}{2.244851in}}%
\pgfpathlineto{\pgfqpoint{1.496765in}{2.092227in}}%
\pgfpathlineto{\pgfqpoint{1.496964in}{2.397475in}}%
\pgfpathlineto{\pgfqpoint{1.497162in}{2.397475in}}%
\pgfpathlineto{\pgfqpoint{1.497956in}{2.163452in}}%
\pgfpathlineto{\pgfqpoint{1.497361in}{2.407650in}}%
\pgfpathlineto{\pgfqpoint{1.498155in}{2.193976in}}%
\pgfpathlineto{\pgfqpoint{1.498353in}{2.193976in}}%
\pgfpathlineto{\pgfqpoint{1.498353in}{1.990478in}}%
\pgfpathlineto{\pgfqpoint{1.498750in}{2.336425in}}%
\pgfpathlineto{\pgfqpoint{1.499346in}{2.112577in}}%
\pgfpathlineto{\pgfqpoint{1.499544in}{2.112577in}}%
\pgfpathlineto{\pgfqpoint{1.499544in}{2.010828in}}%
\pgfpathlineto{\pgfqpoint{1.499743in}{2.468699in}}%
\pgfpathlineto{\pgfqpoint{1.500537in}{2.265201in}}%
\pgfpathlineto{\pgfqpoint{1.500735in}{2.265201in}}%
\pgfpathlineto{\pgfqpoint{1.501331in}{2.458524in}}%
\pgfpathlineto{\pgfqpoint{1.501132in}{2.092227in}}%
\pgfpathlineto{\pgfqpoint{1.501728in}{2.244851in}}%
\pgfpathlineto{\pgfqpoint{1.501926in}{2.244851in}}%
\pgfpathlineto{\pgfqpoint{1.502323in}{2.000653in}}%
\pgfpathlineto{\pgfqpoint{1.502919in}{2.570448in}}%
\pgfpathlineto{\pgfqpoint{1.503117in}{2.570448in}}%
\pgfpathlineto{\pgfqpoint{1.503514in}{2.061702in}}%
\pgfpathlineto{\pgfqpoint{1.504110in}{2.122752in}}%
\pgfpathlineto{\pgfqpoint{1.504308in}{2.122752in}}%
\pgfpathlineto{\pgfqpoint{1.505102in}{2.509399in}}%
\pgfpathlineto{\pgfqpoint{1.504705in}{2.092227in}}%
\pgfpathlineto{\pgfqpoint{1.505301in}{2.489049in}}%
\pgfpathlineto{\pgfqpoint{1.505499in}{2.489049in}}%
\pgfpathlineto{\pgfqpoint{1.506293in}{2.204151in}}%
\pgfpathlineto{\pgfqpoint{1.506492in}{2.397475in}}%
\pgfpathlineto{\pgfqpoint{1.506690in}{2.397475in}}%
\pgfpathlineto{\pgfqpoint{1.506690in}{2.071877in}}%
\pgfpathlineto{\pgfqpoint{1.507484in}{2.509399in}}%
\pgfpathlineto{\pgfqpoint{1.507683in}{2.377125in}}%
\pgfpathlineto{\pgfqpoint{1.507881in}{2.377125in}}%
\pgfpathlineto{\pgfqpoint{1.508477in}{2.275376in}}%
\pgfpathlineto{\pgfqpoint{1.508675in}{2.499224in}}%
\pgfpathlineto{\pgfqpoint{1.508874in}{2.417825in}}%
\pgfpathlineto{\pgfqpoint{1.509072in}{2.417825in}}%
\pgfpathlineto{\pgfqpoint{1.509469in}{2.204151in}}%
\pgfpathlineto{\pgfqpoint{1.510065in}{2.509399in}}%
\pgfpathlineto{\pgfqpoint{1.510263in}{2.509399in}}%
\pgfpathlineto{\pgfqpoint{1.510660in}{2.173627in}}%
\pgfpathlineto{\pgfqpoint{1.511256in}{2.346600in}}%
\pgfpathlineto{\pgfqpoint{1.511454in}{2.346600in}}%
\pgfpathlineto{\pgfqpoint{1.512248in}{2.255026in}}%
\pgfpathlineto{\pgfqpoint{1.512050in}{2.550099in}}%
\pgfpathlineto{\pgfqpoint{1.512447in}{2.458524in}}%
\pgfpathlineto{\pgfqpoint{1.512645in}{2.458524in}}%
\pgfpathlineto{\pgfqpoint{1.513439in}{2.183801in}}%
\pgfpathlineto{\pgfqpoint{1.512844in}{2.509399in}}%
\pgfpathlineto{\pgfqpoint{1.513638in}{2.295726in}}%
\pgfpathlineto{\pgfqpoint{1.513836in}{2.295726in}}%
\pgfpathlineto{\pgfqpoint{1.514829in}{2.600973in}}%
\pgfpathlineto{\pgfqpoint{1.515027in}{2.600973in}}%
\pgfpathlineto{\pgfqpoint{1.515027in}{2.204151in}}%
\pgfpathlineto{\pgfqpoint{1.515424in}{2.692547in}}%
\pgfpathlineto{\pgfqpoint{1.516020in}{2.478874in}}%
\pgfpathlineto{\pgfqpoint{1.516218in}{2.478874in}}%
\pgfpathlineto{\pgfqpoint{1.516814in}{2.438174in}}%
\pgfpathlineto{\pgfqpoint{1.517211in}{2.692547in}}%
\pgfpathlineto{\pgfqpoint{1.517409in}{2.692547in}}%
\pgfpathlineto{\pgfqpoint{1.518005in}{2.397475in}}%
\pgfpathlineto{\pgfqpoint{1.518402in}{2.397475in}}%
\pgfpathlineto{\pgfqpoint{1.518600in}{2.397475in}}%
\pgfpathlineto{\pgfqpoint{1.518600in}{2.865521in}}%
\pgfpathlineto{\pgfqpoint{1.518997in}{2.346600in}}%
\pgfpathlineto{\pgfqpoint{1.519593in}{2.580623in}}%
\pgfpathlineto{\pgfqpoint{1.519791in}{2.580623in}}%
\pgfpathlineto{\pgfqpoint{1.520188in}{2.712897in}}%
\pgfpathlineto{\pgfqpoint{1.519990in}{2.295726in}}%
\pgfpathlineto{\pgfqpoint{1.520784in}{2.539924in}}%
\pgfpathlineto{\pgfqpoint{1.520982in}{2.539924in}}%
\pgfpathlineto{\pgfqpoint{1.520982in}{2.733247in}}%
\pgfpathlineto{\pgfqpoint{1.521379in}{2.366950in}}%
\pgfpathlineto{\pgfqpoint{1.521975in}{2.600973in}}%
\pgfpathlineto{\pgfqpoint{1.522173in}{2.600973in}}%
\pgfpathlineto{\pgfqpoint{1.522173in}{2.428000in}}%
\pgfpathlineto{\pgfqpoint{1.522967in}{2.702722in}}%
\pgfpathlineto{\pgfqpoint{1.523165in}{2.550099in}}%
\pgfpathlineto{\pgfqpoint{1.523364in}{2.550099in}}%
\pgfpathlineto{\pgfqpoint{1.524158in}{2.682373in}}%
\pgfpathlineto{\pgfqpoint{1.523761in}{2.397475in}}%
\pgfpathlineto{\pgfqpoint{1.524356in}{2.662023in}}%
\pgfpathlineto{\pgfqpoint{1.524555in}{2.662023in}}%
\pgfpathlineto{\pgfqpoint{1.525150in}{2.366950in}}%
\pgfpathlineto{\pgfqpoint{1.525349in}{2.723072in}}%
\pgfpathlineto{\pgfqpoint{1.525547in}{2.723072in}}%
\pgfpathlineto{\pgfqpoint{1.525746in}{2.723072in}}%
\pgfpathlineto{\pgfqpoint{1.526341in}{2.519574in}}%
\pgfpathlineto{\pgfqpoint{1.526738in}{2.946920in}}%
\pgfpathlineto{\pgfqpoint{1.526937in}{2.946920in}}%
\pgfpathlineto{\pgfqpoint{1.527532in}{2.509399in}}%
\pgfpathlineto{\pgfqpoint{1.527929in}{2.855346in}}%
\pgfpathlineto{\pgfqpoint{1.528128in}{2.855346in}}%
\pgfpathlineto{\pgfqpoint{1.528922in}{2.366950in}}%
\pgfpathlineto{\pgfqpoint{1.529120in}{2.906221in}}%
\pgfpathlineto{\pgfqpoint{1.529319in}{2.906221in}}%
\pgfpathlineto{\pgfqpoint{1.529319in}{2.417825in}}%
\pgfpathlineto{\pgfqpoint{1.530311in}{2.590798in}}%
\pgfpathlineto{\pgfqpoint{1.530510in}{2.590798in}}%
\pgfpathlineto{\pgfqpoint{1.530510in}{2.865521in}}%
\pgfpathlineto{\pgfqpoint{1.531502in}{2.824821in}}%
\pgfpathlineto{\pgfqpoint{1.531701in}{2.824821in}}%
\pgfpathlineto{\pgfqpoint{1.531899in}{2.682373in}}%
\pgfpathlineto{\pgfqpoint{1.532495in}{2.946920in}}%
\pgfpathlineto{\pgfqpoint{1.532693in}{2.845171in}}%
\pgfpathlineto{\pgfqpoint{1.532892in}{2.845171in}}%
\pgfpathlineto{\pgfqpoint{1.533686in}{2.631498in}}%
\pgfpathlineto{\pgfqpoint{1.533289in}{3.038495in}}%
\pgfpathlineto{\pgfqpoint{1.533884in}{2.845171in}}%
\pgfpathlineto{\pgfqpoint{1.534083in}{2.845171in}}%
\pgfpathlineto{\pgfqpoint{1.534678in}{2.550099in}}%
\pgfpathlineto{\pgfqpoint{1.534480in}{3.069019in}}%
\pgfpathlineto{\pgfqpoint{1.535075in}{2.936746in}}%
\pgfpathlineto{\pgfqpoint{1.535274in}{2.936746in}}%
\pgfpathlineto{\pgfqpoint{1.535472in}{2.377125in}}%
\pgfpathlineto{\pgfqpoint{1.536266in}{2.977445in}}%
\pgfpathlineto{\pgfqpoint{1.536465in}{2.977445in}}%
\pgfpathlineto{\pgfqpoint{1.536465in}{2.733247in}}%
\pgfpathlineto{\pgfqpoint{1.537259in}{2.997795in}}%
\pgfpathlineto{\pgfqpoint{1.537457in}{2.987620in}}%
\pgfpathlineto{\pgfqpoint{1.537656in}{2.987620in}}%
\pgfpathlineto{\pgfqpoint{1.538251in}{3.007970in}}%
\pgfpathlineto{\pgfqpoint{1.538648in}{2.712897in}}%
\pgfpathlineto{\pgfqpoint{1.538847in}{2.712897in}}%
\pgfpathlineto{\pgfqpoint{1.539045in}{3.221643in}}%
\pgfpathlineto{\pgfqpoint{1.539839in}{2.662023in}}%
\pgfpathlineto{\pgfqpoint{1.540038in}{2.662023in}}%
\pgfpathlineto{\pgfqpoint{1.540633in}{3.089369in}}%
\pgfpathlineto{\pgfqpoint{1.541030in}{2.855346in}}%
\pgfpathlineto{\pgfqpoint{1.541229in}{2.855346in}}%
\pgfpathlineto{\pgfqpoint{1.541824in}{2.824821in}}%
\pgfpathlineto{\pgfqpoint{1.542221in}{3.201293in}}%
\pgfpathlineto{\pgfqpoint{1.542420in}{3.201293in}}%
\pgfpathlineto{\pgfqpoint{1.543214in}{2.763772in}}%
\pgfpathlineto{\pgfqpoint{1.543412in}{2.855346in}}%
\pgfpathlineto{\pgfqpoint{1.543611in}{2.855346in}}%
\pgfpathlineto{\pgfqpoint{1.543611in}{2.692547in}}%
\pgfpathlineto{\pgfqpoint{1.544603in}{3.140244in}}%
\pgfpathlineto{\pgfqpoint{1.544802in}{3.140244in}}%
\pgfpathlineto{\pgfqpoint{1.544802in}{3.231818in}}%
\pgfpathlineto{\pgfqpoint{1.545397in}{2.570448in}}%
\pgfpathlineto{\pgfqpoint{1.545794in}{2.926571in}}%
\pgfpathlineto{\pgfqpoint{1.545993in}{2.926571in}}%
\pgfpathlineto{\pgfqpoint{1.546588in}{3.160594in}}%
\pgfpathlineto{\pgfqpoint{1.546787in}{2.865521in}}%
\pgfpathlineto{\pgfqpoint{1.546985in}{3.058845in}}%
\pgfpathlineto{\pgfqpoint{1.547184in}{3.058845in}}%
\pgfpathlineto{\pgfqpoint{1.547779in}{2.906221in}}%
\pgfpathlineto{\pgfqpoint{1.547581in}{3.292868in}}%
\pgfpathlineto{\pgfqpoint{1.548176in}{3.150419in}}%
\pgfpathlineto{\pgfqpoint{1.548375in}{3.150419in}}%
\pgfpathlineto{\pgfqpoint{1.548573in}{2.946920in}}%
\pgfpathlineto{\pgfqpoint{1.549367in}{3.180944in}}%
\pgfpathlineto{\pgfqpoint{1.549566in}{3.180944in}}%
\pgfpathlineto{\pgfqpoint{1.549764in}{2.855346in}}%
\pgfpathlineto{\pgfqpoint{1.550558in}{3.119894in}}%
\pgfpathlineto{\pgfqpoint{1.550757in}{3.119894in}}%
\pgfpathlineto{\pgfqpoint{1.550757in}{2.845171in}}%
\pgfpathlineto{\pgfqpoint{1.551352in}{3.282693in}}%
\pgfpathlineto{\pgfqpoint{1.551749in}{2.957095in}}%
\pgfpathlineto{\pgfqpoint{1.551948in}{2.957095in}}%
\pgfpathlineto{\pgfqpoint{1.552146in}{2.824821in}}%
\pgfpathlineto{\pgfqpoint{1.552543in}{3.303043in}}%
\pgfpathlineto{\pgfqpoint{1.552940in}{2.967270in}}%
\pgfpathlineto{\pgfqpoint{1.553139in}{2.967270in}}%
\pgfpathlineto{\pgfqpoint{1.553139in}{3.048670in}}%
\pgfpathlineto{\pgfqpoint{1.553734in}{2.804472in}}%
\pgfpathlineto{\pgfqpoint{1.554131in}{2.946920in}}%
\pgfpathlineto{\pgfqpoint{1.554330in}{2.946920in}}%
\pgfpathlineto{\pgfqpoint{1.554330in}{3.221643in}}%
\pgfpathlineto{\pgfqpoint{1.555322in}{2.967270in}}%
\pgfpathlineto{\pgfqpoint{1.555520in}{2.967270in}}%
\pgfpathlineto{\pgfqpoint{1.555719in}{2.702722in}}%
\pgfpathlineto{\pgfqpoint{1.555917in}{3.343742in}}%
\pgfpathlineto{\pgfqpoint{1.556513in}{3.018145in}}%
\pgfpathlineto{\pgfqpoint{1.556711in}{3.018145in}}%
\pgfpathlineto{\pgfqpoint{1.556711in}{3.262343in}}%
\pgfpathlineto{\pgfqpoint{1.557704in}{3.130069in}}%
\pgfpathlineto{\pgfqpoint{1.557902in}{3.130069in}}%
\pgfpathlineto{\pgfqpoint{1.557902in}{3.303043in}}%
\pgfpathlineto{\pgfqpoint{1.558696in}{2.641673in}}%
\pgfpathlineto{\pgfqpoint{1.558895in}{3.018145in}}%
\pgfpathlineto{\pgfqpoint{1.559093in}{3.018145in}}%
\pgfpathlineto{\pgfqpoint{1.559093in}{2.896046in}}%
\pgfpathlineto{\pgfqpoint{1.559490in}{3.170769in}}%
\pgfpathlineto{\pgfqpoint{1.560086in}{3.109719in}}%
\pgfpathlineto{\pgfqpoint{1.560284in}{3.109719in}}%
\pgfpathlineto{\pgfqpoint{1.560681in}{2.885871in}}%
\pgfpathlineto{\pgfqpoint{1.561277in}{3.119894in}}%
\pgfpathlineto{\pgfqpoint{1.561475in}{3.119894in}}%
\pgfpathlineto{\pgfqpoint{1.561872in}{3.292868in}}%
\pgfpathlineto{\pgfqpoint{1.562071in}{3.048670in}}%
\pgfpathlineto{\pgfqpoint{1.562468in}{3.109719in}}%
\pgfpathlineto{\pgfqpoint{1.562666in}{3.109719in}}%
\pgfpathlineto{\pgfqpoint{1.563262in}{2.987620in}}%
\pgfpathlineto{\pgfqpoint{1.563659in}{3.282693in}}%
\pgfpathlineto{\pgfqpoint{1.563857in}{3.282693in}}%
\pgfpathlineto{\pgfqpoint{1.564056in}{3.007970in}}%
\pgfpathlineto{\pgfqpoint{1.564254in}{3.476016in}}%
\pgfpathlineto{\pgfqpoint{1.564850in}{3.404792in}}%
\pgfpathlineto{\pgfqpoint{1.565048in}{3.404792in}}%
\pgfpathlineto{\pgfqpoint{1.565048in}{2.987620in}}%
\pgfpathlineto{\pgfqpoint{1.566041in}{3.170769in}}%
\pgfpathlineto{\pgfqpoint{1.566239in}{3.170769in}}%
\pgfpathlineto{\pgfqpoint{1.566835in}{3.007970in}}%
\pgfpathlineto{\pgfqpoint{1.566636in}{3.506541in}}%
\pgfpathlineto{\pgfqpoint{1.567232in}{3.292868in}}%
\pgfpathlineto{\pgfqpoint{1.567430in}{3.292868in}}%
\pgfpathlineto{\pgfqpoint{1.567827in}{2.896046in}}%
\pgfpathlineto{\pgfqpoint{1.567629in}{3.465841in}}%
\pgfpathlineto{\pgfqpoint{1.568423in}{3.007970in}}%
\pgfpathlineto{\pgfqpoint{1.568621in}{3.007970in}}%
\pgfpathlineto{\pgfqpoint{1.568820in}{3.201293in}}%
\pgfpathlineto{\pgfqpoint{1.569614in}{3.048670in}}%
\pgfpathlineto{\pgfqpoint{1.569812in}{3.048670in}}%
\pgfpathlineto{\pgfqpoint{1.570805in}{3.465841in}}%
\pgfpathlineto{\pgfqpoint{1.571003in}{3.465841in}}%
\pgfpathlineto{\pgfqpoint{1.571400in}{3.089369in}}%
\pgfpathlineto{\pgfqpoint{1.571996in}{3.191119in}}%
\pgfpathlineto{\pgfqpoint{1.572194in}{3.191119in}}%
\pgfpathlineto{\pgfqpoint{1.572194in}{3.394617in}}%
\pgfpathlineto{\pgfqpoint{1.572591in}{3.048670in}}%
\pgfpathlineto{\pgfqpoint{1.573187in}{3.333567in}}%
\pgfpathlineto{\pgfqpoint{1.573385in}{3.333567in}}%
\pgfpathlineto{\pgfqpoint{1.573782in}{3.160594in}}%
\pgfpathlineto{\pgfqpoint{1.574378in}{3.160594in}}%
\pgfpathlineto{\pgfqpoint{1.574576in}{3.160594in}}%
\pgfpathlineto{\pgfqpoint{1.574775in}{3.618465in}}%
\pgfpathlineto{\pgfqpoint{1.575569in}{2.916396in}}%
\pgfpathlineto{\pgfqpoint{1.575767in}{2.916396in}}%
\pgfpathlineto{\pgfqpoint{1.576561in}{3.374267in}}%
\pgfpathlineto{\pgfqpoint{1.576760in}{3.364092in}}%
\pgfpathlineto{\pgfqpoint{1.576958in}{3.364092in}}%
\pgfpathlineto{\pgfqpoint{1.576958in}{3.069019in}}%
\pgfpathlineto{\pgfqpoint{1.577951in}{3.282693in}}%
\pgfpathlineto{\pgfqpoint{1.578149in}{3.282693in}}%
\pgfpathlineto{\pgfqpoint{1.578745in}{2.946920in}}%
\pgfpathlineto{\pgfqpoint{1.579142in}{3.547241in}}%
\pgfpathlineto{\pgfqpoint{1.579340in}{3.547241in}}%
\pgfpathlineto{\pgfqpoint{1.579737in}{3.048670in}}%
\pgfpathlineto{\pgfqpoint{1.580333in}{3.313218in}}%
\pgfpathlineto{\pgfqpoint{1.580531in}{3.313218in}}%
\pgfpathlineto{\pgfqpoint{1.580531in}{2.957095in}}%
\pgfpathlineto{\pgfqpoint{1.581325in}{3.435317in}}%
\pgfpathlineto{\pgfqpoint{1.581524in}{3.048670in}}%
\pgfpathlineto{\pgfqpoint{1.581722in}{3.048670in}}%
\pgfpathlineto{\pgfqpoint{1.582119in}{2.794297in}}%
\pgfpathlineto{\pgfqpoint{1.581921in}{3.486191in}}%
\pgfpathlineto{\pgfqpoint{1.582715in}{2.834996in}}%
\pgfpathlineto{\pgfqpoint{1.582913in}{2.834996in}}%
\pgfpathlineto{\pgfqpoint{1.583509in}{3.587940in}}%
\pgfpathlineto{\pgfqpoint{1.583906in}{3.241993in}}%
\pgfpathlineto{\pgfqpoint{1.584104in}{3.241993in}}%
\pgfpathlineto{\pgfqpoint{1.584898in}{2.794297in}}%
\pgfpathlineto{\pgfqpoint{1.584501in}{3.333567in}}%
\pgfpathlineto{\pgfqpoint{1.585097in}{3.272518in}}%
\pgfpathlineto{\pgfqpoint{1.585295in}{3.272518in}}%
\pgfpathlineto{\pgfqpoint{1.586089in}{2.906221in}}%
\pgfpathlineto{\pgfqpoint{1.586288in}{3.333567in}}%
\pgfpathlineto{\pgfqpoint{1.586486in}{3.333567in}}%
\pgfpathlineto{\pgfqpoint{1.587280in}{3.028320in}}%
\pgfpathlineto{\pgfqpoint{1.587478in}{3.364092in}}%
\pgfpathlineto{\pgfqpoint{1.587677in}{3.364092in}}%
\pgfpathlineto{\pgfqpoint{1.587875in}{3.648990in}}%
\pgfpathlineto{\pgfqpoint{1.588272in}{3.007970in}}%
\pgfpathlineto{\pgfqpoint{1.588669in}{3.028320in}}%
\pgfpathlineto{\pgfqpoint{1.588868in}{3.028320in}}%
\pgfpathlineto{\pgfqpoint{1.589265in}{3.414967in}}%
\pgfpathlineto{\pgfqpoint{1.589463in}{2.946920in}}%
\pgfpathlineto{\pgfqpoint{1.589860in}{3.109719in}}%
\pgfpathlineto{\pgfqpoint{1.590059in}{3.109719in}}%
\pgfpathlineto{\pgfqpoint{1.590059in}{2.814646in}}%
\pgfpathlineto{\pgfqpoint{1.590654in}{3.364092in}}%
\pgfpathlineto{\pgfqpoint{1.591051in}{2.814646in}}%
\pgfpathlineto{\pgfqpoint{1.591250in}{2.814646in}}%
\pgfpathlineto{\pgfqpoint{1.591647in}{3.343742in}}%
\pgfpathlineto{\pgfqpoint{1.592044in}{2.743422in}}%
\pgfpathlineto{\pgfqpoint{1.592242in}{3.313218in}}%
\pgfpathlineto{\pgfqpoint{1.592441in}{3.313218in}}%
\pgfpathlineto{\pgfqpoint{1.592639in}{3.394617in}}%
\pgfpathlineto{\pgfqpoint{1.593433in}{2.855346in}}%
\pgfpathlineto{\pgfqpoint{1.593632in}{2.855346in}}%
\pgfpathlineto{\pgfqpoint{1.594426in}{3.384442in}}%
\pgfpathlineto{\pgfqpoint{1.594624in}{2.763772in}}%
\pgfpathlineto{\pgfqpoint{1.594823in}{2.763772in}}%
\pgfpathlineto{\pgfqpoint{1.595220in}{3.537066in}}%
\pgfpathlineto{\pgfqpoint{1.595815in}{3.048670in}}%
\pgfpathlineto{\pgfqpoint{1.596014in}{3.048670in}}%
\pgfpathlineto{\pgfqpoint{1.597006in}{3.313218in}}%
\pgfpathlineto{\pgfqpoint{1.597205in}{3.313218in}}%
\pgfpathlineto{\pgfqpoint{1.597800in}{2.987620in}}%
\pgfpathlineto{\pgfqpoint{1.598197in}{3.150419in}}%
\pgfpathlineto{\pgfqpoint{1.598396in}{3.150419in}}%
\pgfpathlineto{\pgfqpoint{1.598396in}{3.435317in}}%
\pgfpathlineto{\pgfqpoint{1.599388in}{2.967270in}}%
\pgfpathlineto{\pgfqpoint{1.599587in}{2.967270in}}%
\pgfpathlineto{\pgfqpoint{1.599587in}{3.028320in}}%
\pgfpathlineto{\pgfqpoint{1.600579in}{2.865521in}}%
\pgfpathlineto{\pgfqpoint{1.600778in}{2.865521in}}%
\pgfpathlineto{\pgfqpoint{1.601572in}{3.292868in}}%
\pgfpathlineto{\pgfqpoint{1.601770in}{3.018145in}}%
\pgfpathlineto{\pgfqpoint{1.601969in}{3.018145in}}%
\pgfpathlineto{\pgfqpoint{1.602763in}{3.170769in}}%
\pgfpathlineto{\pgfqpoint{1.602366in}{2.824821in}}%
\pgfpathlineto{\pgfqpoint{1.602961in}{3.160594in}}%
\pgfpathlineto{\pgfqpoint{1.603160in}{3.160594in}}%
\pgfpathlineto{\pgfqpoint{1.603557in}{2.784122in}}%
\pgfpathlineto{\pgfqpoint{1.603755in}{3.292868in}}%
\pgfpathlineto{\pgfqpoint{1.604152in}{2.794297in}}%
\pgfpathlineto{\pgfqpoint{1.604351in}{2.794297in}}%
\pgfpathlineto{\pgfqpoint{1.604549in}{3.374267in}}%
\pgfpathlineto{\pgfqpoint{1.605343in}{2.977445in}}%
\pgfpathlineto{\pgfqpoint{1.605542in}{2.977445in}}%
\pgfpathlineto{\pgfqpoint{1.606137in}{3.333567in}}%
\pgfpathlineto{\pgfqpoint{1.605939in}{2.651848in}}%
\pgfpathlineto{\pgfqpoint{1.606534in}{2.773947in}}%
\pgfpathlineto{\pgfqpoint{1.606733in}{2.773947in}}%
\pgfpathlineto{\pgfqpoint{1.607130in}{3.079194in}}%
\pgfpathlineto{\pgfqpoint{1.607725in}{3.058845in}}%
\pgfpathlineto{\pgfqpoint{1.607924in}{3.058845in}}%
\pgfpathlineto{\pgfqpoint{1.608916in}{2.885871in}}%
\pgfpathlineto{\pgfqpoint{1.609115in}{2.885871in}}%
\pgfpathlineto{\pgfqpoint{1.609115in}{2.834996in}}%
\pgfpathlineto{\pgfqpoint{1.609512in}{3.038495in}}%
\pgfpathlineto{\pgfqpoint{1.610107in}{2.987620in}}%
\pgfpathlineto{\pgfqpoint{1.610306in}{2.987620in}}%
\pgfpathlineto{\pgfqpoint{1.610306in}{3.353917in}}%
\pgfpathlineto{\pgfqpoint{1.610504in}{2.885871in}}%
\pgfpathlineto{\pgfqpoint{1.611298in}{2.936746in}}%
\pgfpathlineto{\pgfqpoint{1.611497in}{2.936746in}}%
\pgfpathlineto{\pgfqpoint{1.612291in}{2.834996in}}%
\pgfpathlineto{\pgfqpoint{1.611695in}{3.018145in}}%
\pgfpathlineto{\pgfqpoint{1.612489in}{2.916396in}}%
\pgfpathlineto{\pgfqpoint{1.612688in}{2.916396in}}%
\pgfpathlineto{\pgfqpoint{1.612688in}{3.109719in}}%
\pgfpathlineto{\pgfqpoint{1.613283in}{2.672198in}}%
\pgfpathlineto{\pgfqpoint{1.613680in}{2.875696in}}%
\pgfpathlineto{\pgfqpoint{1.613879in}{2.875696in}}%
\pgfpathlineto{\pgfqpoint{1.613879in}{2.845171in}}%
\pgfpathlineto{\pgfqpoint{1.614871in}{3.048670in}}%
\pgfpathlineto{\pgfqpoint{1.615070in}{3.048670in}}%
\pgfpathlineto{\pgfqpoint{1.615070in}{2.499224in}}%
\pgfpathlineto{\pgfqpoint{1.615665in}{3.201293in}}%
\pgfpathlineto{\pgfqpoint{1.616062in}{2.794297in}}%
\pgfpathlineto{\pgfqpoint{1.616261in}{2.794297in}}%
\pgfpathlineto{\pgfqpoint{1.616658in}{2.590798in}}%
\pgfpathlineto{\pgfqpoint{1.617253in}{3.160594in}}%
\pgfpathlineto{\pgfqpoint{1.617452in}{3.160594in}}%
\pgfpathlineto{\pgfqpoint{1.617452in}{2.824821in}}%
\pgfpathlineto{\pgfqpoint{1.618444in}{2.896046in}}%
\pgfpathlineto{\pgfqpoint{1.618643in}{2.896046in}}%
\pgfpathlineto{\pgfqpoint{1.618643in}{2.600973in}}%
\pgfpathlineto{\pgfqpoint{1.619039in}{3.058845in}}%
\pgfpathlineto{\pgfqpoint{1.619635in}{2.794297in}}%
\pgfpathlineto{\pgfqpoint{1.619833in}{2.794297in}}%
\pgfpathlineto{\pgfqpoint{1.620429in}{3.069019in}}%
\pgfpathlineto{\pgfqpoint{1.620627in}{2.692547in}}%
\pgfpathlineto{\pgfqpoint{1.620826in}{3.028320in}}%
\pgfpathlineto{\pgfqpoint{1.621024in}{3.028320in}}%
\pgfpathlineto{\pgfqpoint{1.621818in}{3.140244in}}%
\pgfpathlineto{\pgfqpoint{1.622017in}{2.682373in}}%
\pgfpathlineto{\pgfqpoint{1.622215in}{2.682373in}}%
\pgfpathlineto{\pgfqpoint{1.622414in}{3.119894in}}%
\pgfpathlineto{\pgfqpoint{1.623208in}{2.845171in}}%
\pgfpathlineto{\pgfqpoint{1.623406in}{2.845171in}}%
\pgfpathlineto{\pgfqpoint{1.623803in}{2.580623in}}%
\pgfpathlineto{\pgfqpoint{1.623605in}{2.977445in}}%
\pgfpathlineto{\pgfqpoint{1.624399in}{2.702722in}}%
\pgfpathlineto{\pgfqpoint{1.624597in}{2.702722in}}%
\pgfpathlineto{\pgfqpoint{1.624597in}{2.590798in}}%
\pgfpathlineto{\pgfqpoint{1.624796in}{3.241993in}}%
\pgfpathlineto{\pgfqpoint{1.625590in}{2.662023in}}%
\pgfpathlineto{\pgfqpoint{1.625788in}{2.662023in}}%
\pgfpathlineto{\pgfqpoint{1.625788in}{2.977445in}}%
\pgfpathlineto{\pgfqpoint{1.626185in}{2.600973in}}%
\pgfpathlineto{\pgfqpoint{1.626781in}{2.733247in}}%
\pgfpathlineto{\pgfqpoint{1.626979in}{2.733247in}}%
\pgfpathlineto{\pgfqpoint{1.627376in}{2.519574in}}%
\pgfpathlineto{\pgfqpoint{1.627773in}{2.896046in}}%
\pgfpathlineto{\pgfqpoint{1.627972in}{2.834996in}}%
\pgfpathlineto{\pgfqpoint{1.628170in}{2.834996in}}%
\pgfpathlineto{\pgfqpoint{1.628170in}{2.519574in}}%
\pgfpathlineto{\pgfqpoint{1.628369in}{3.099544in}}%
\pgfpathlineto{\pgfqpoint{1.629163in}{2.794297in}}%
\pgfpathlineto{\pgfqpoint{1.629361in}{2.794297in}}%
\pgfpathlineto{\pgfqpoint{1.629361in}{2.834996in}}%
\pgfpathlineto{\pgfqpoint{1.629758in}{2.682373in}}%
\pgfpathlineto{\pgfqpoint{1.630354in}{2.763772in}}%
\pgfpathlineto{\pgfqpoint{1.630552in}{2.763772in}}%
\pgfpathlineto{\pgfqpoint{1.630552in}{2.865521in}}%
\pgfpathlineto{\pgfqpoint{1.631545in}{2.651848in}}%
\pgfpathlineto{\pgfqpoint{1.631743in}{2.651848in}}%
\pgfpathlineto{\pgfqpoint{1.632140in}{3.048670in}}%
\pgfpathlineto{\pgfqpoint{1.632736in}{2.438174in}}%
\pgfpathlineto{\pgfqpoint{1.632934in}{2.438174in}}%
\pgfpathlineto{\pgfqpoint{1.633530in}{2.672198in}}%
\pgfpathlineto{\pgfqpoint{1.633927in}{2.631498in}}%
\pgfpathlineto{\pgfqpoint{1.634125in}{2.631498in}}%
\pgfpathlineto{\pgfqpoint{1.634324in}{2.824821in}}%
\pgfpathlineto{\pgfqpoint{1.634919in}{2.417825in}}%
\pgfpathlineto{\pgfqpoint{1.635118in}{2.784122in}}%
\pgfpathlineto{\pgfqpoint{1.635316in}{2.784122in}}%
\pgfpathlineto{\pgfqpoint{1.635316in}{2.336425in}}%
\pgfpathlineto{\pgfqpoint{1.636309in}{2.580623in}}%
\pgfpathlineto{\pgfqpoint{1.636507in}{2.580623in}}%
\pgfpathlineto{\pgfqpoint{1.636507in}{2.733247in}}%
\pgfpathlineto{\pgfqpoint{1.637500in}{2.428000in}}%
\pgfpathlineto{\pgfqpoint{1.637698in}{2.428000in}}%
\pgfpathlineto{\pgfqpoint{1.638294in}{2.896046in}}%
\pgfpathlineto{\pgfqpoint{1.638691in}{2.682373in}}%
\pgfpathlineto{\pgfqpoint{1.638889in}{2.682373in}}%
\pgfpathlineto{\pgfqpoint{1.639088in}{2.438174in}}%
\pgfpathlineto{\pgfqpoint{1.639882in}{2.519574in}}%
\pgfpathlineto{\pgfqpoint{1.640080in}{2.519574in}}%
\pgfpathlineto{\pgfqpoint{1.640676in}{2.743422in}}%
\pgfpathlineto{\pgfqpoint{1.641073in}{2.600973in}}%
\pgfpathlineto{\pgfqpoint{1.641271in}{2.600973in}}%
\pgfpathlineto{\pgfqpoint{1.641271in}{2.682373in}}%
\pgfpathlineto{\pgfqpoint{1.642264in}{2.377125in}}%
\pgfpathlineto{\pgfqpoint{1.642462in}{2.377125in}}%
\pgfpathlineto{\pgfqpoint{1.642462in}{2.611148in}}%
\pgfpathlineto{\pgfqpoint{1.643256in}{2.356775in}}%
\pgfpathlineto{\pgfqpoint{1.643455in}{2.509399in}}%
\pgfpathlineto{\pgfqpoint{1.643653in}{2.509399in}}%
\pgfpathlineto{\pgfqpoint{1.643653in}{2.550099in}}%
\pgfpathlineto{\pgfqpoint{1.643852in}{2.356775in}}%
\pgfpathlineto{\pgfqpoint{1.644646in}{2.550099in}}%
\pgfpathlineto{\pgfqpoint{1.644844in}{2.550099in}}%
\pgfpathlineto{\pgfqpoint{1.645043in}{2.285551in}}%
\pgfpathlineto{\pgfqpoint{1.645440in}{2.590798in}}%
\pgfpathlineto{\pgfqpoint{1.645837in}{2.377125in}}%
\pgfpathlineto{\pgfqpoint{1.646035in}{2.377125in}}%
\pgfpathlineto{\pgfqpoint{1.646035in}{2.651848in}}%
\pgfpathlineto{\pgfqpoint{1.646432in}{2.204151in}}%
\pgfpathlineto{\pgfqpoint{1.647028in}{2.356775in}}%
\pgfpathlineto{\pgfqpoint{1.647226in}{2.356775in}}%
\pgfpathlineto{\pgfqpoint{1.647623in}{2.570448in}}%
\pgfpathlineto{\pgfqpoint{1.647425in}{2.204151in}}%
\pgfpathlineto{\pgfqpoint{1.648219in}{2.255026in}}%
\pgfpathlineto{\pgfqpoint{1.648417in}{2.255026in}}%
\pgfpathlineto{\pgfqpoint{1.649013in}{2.071877in}}%
\pgfpathlineto{\pgfqpoint{1.649410in}{2.692547in}}%
\pgfpathlineto{\pgfqpoint{1.649608in}{2.692547in}}%
\pgfpathlineto{\pgfqpoint{1.649807in}{2.336425in}}%
\pgfpathlineto{\pgfqpoint{1.650601in}{2.580623in}}%
\pgfpathlineto{\pgfqpoint{1.650799in}{2.580623in}}%
\pgfpathlineto{\pgfqpoint{1.650998in}{2.346600in}}%
\pgfpathlineto{\pgfqpoint{1.651791in}{2.407650in}}%
\pgfpathlineto{\pgfqpoint{1.651990in}{2.407650in}}%
\pgfpathlineto{\pgfqpoint{1.651990in}{2.356775in}}%
\pgfpathlineto{\pgfqpoint{1.652188in}{2.641673in}}%
\pgfpathlineto{\pgfqpoint{1.652982in}{2.417825in}}%
\pgfpathlineto{\pgfqpoint{1.653181in}{2.417825in}}%
\pgfpathlineto{\pgfqpoint{1.653379in}{2.478874in}}%
\pgfpathlineto{\pgfqpoint{1.654173in}{2.183801in}}%
\pgfpathlineto{\pgfqpoint{1.654372in}{2.183801in}}%
\pgfpathlineto{\pgfqpoint{1.654570in}{2.499224in}}%
\pgfpathlineto{\pgfqpoint{1.654769in}{2.132927in}}%
\pgfpathlineto{\pgfqpoint{1.655364in}{2.336425in}}%
\pgfpathlineto{\pgfqpoint{1.655563in}{2.336425in}}%
\pgfpathlineto{\pgfqpoint{1.655761in}{2.051527in}}%
\pgfpathlineto{\pgfqpoint{1.655960in}{2.570448in}}%
\pgfpathlineto{\pgfqpoint{1.656555in}{2.356775in}}%
\pgfpathlineto{\pgfqpoint{1.656754in}{2.356775in}}%
\pgfpathlineto{\pgfqpoint{1.657746in}{2.163452in}}%
\pgfpathlineto{\pgfqpoint{1.657945in}{2.163452in}}%
\pgfpathlineto{\pgfqpoint{1.658739in}{2.122752in}}%
\pgfpathlineto{\pgfqpoint{1.658143in}{2.509399in}}%
\pgfpathlineto{\pgfqpoint{1.658937in}{2.234676in}}%
\pgfpathlineto{\pgfqpoint{1.659136in}{2.234676in}}%
\pgfpathlineto{\pgfqpoint{1.659930in}{2.092227in}}%
\pgfpathlineto{\pgfqpoint{1.660128in}{2.336425in}}%
\pgfpathlineto{\pgfqpoint{1.660327in}{2.336425in}}%
\pgfpathlineto{\pgfqpoint{1.660327in}{2.143102in}}%
\pgfpathlineto{\pgfqpoint{1.661319in}{2.234676in}}%
\pgfpathlineto{\pgfqpoint{1.661518in}{2.234676in}}%
\pgfpathlineto{\pgfqpoint{1.662312in}{2.275376in}}%
\pgfpathlineto{\pgfqpoint{1.662510in}{2.051527in}}%
\pgfpathlineto{\pgfqpoint{1.662709in}{2.051527in}}%
\pgfpathlineto{\pgfqpoint{1.663106in}{2.377125in}}%
\pgfpathlineto{\pgfqpoint{1.663701in}{2.193976in}}%
\pgfpathlineto{\pgfqpoint{1.663900in}{2.193976in}}%
\pgfpathlineto{\pgfqpoint{1.664495in}{2.316075in}}%
\pgfpathlineto{\pgfqpoint{1.664694in}{2.143102in}}%
\pgfpathlineto{\pgfqpoint{1.664892in}{2.143102in}}%
\pgfpathlineto{\pgfqpoint{1.665289in}{2.143102in}}%
\pgfpathlineto{\pgfqpoint{1.665686in}{2.428000in}}%
\pgfpathlineto{\pgfqpoint{1.666282in}{2.071877in}}%
\pgfpathlineto{\pgfqpoint{1.666480in}{2.071877in}}%
\pgfpathlineto{\pgfqpoint{1.667076in}{2.051527in}}%
\pgfpathlineto{\pgfqpoint{1.667473in}{2.285551in}}%
\pgfpathlineto{\pgfqpoint{1.667671in}{2.285551in}}%
\pgfpathlineto{\pgfqpoint{1.668267in}{2.000653in}}%
\pgfpathlineto{\pgfqpoint{1.668465in}{2.326250in}}%
\pgfpathlineto{\pgfqpoint{1.668664in}{2.326250in}}%
\pgfpathlineto{\pgfqpoint{1.668862in}{2.326250in}}%
\pgfpathlineto{\pgfqpoint{1.668862in}{2.051527in}}%
\pgfpathlineto{\pgfqpoint{1.669259in}{2.346600in}}%
\pgfpathlineto{\pgfqpoint{1.669855in}{2.132927in}}%
\pgfpathlineto{\pgfqpoint{1.670053in}{2.132927in}}%
\pgfpathlineto{\pgfqpoint{1.670252in}{1.919254in}}%
\pgfpathlineto{\pgfqpoint{1.670450in}{2.234676in}}%
\pgfpathlineto{\pgfqpoint{1.671046in}{2.071877in}}%
\pgfpathlineto{\pgfqpoint{1.671244in}{2.071877in}}%
\pgfpathlineto{\pgfqpoint{1.671443in}{1.939603in}}%
\pgfpathlineto{\pgfqpoint{1.671641in}{2.244851in}}%
\pgfpathlineto{\pgfqpoint{1.672237in}{2.112577in}}%
\pgfpathlineto{\pgfqpoint{1.672435in}{2.112577in}}%
\pgfpathlineto{\pgfqpoint{1.673229in}{2.214326in}}%
\pgfpathlineto{\pgfqpoint{1.672832in}{1.939603in}}%
\pgfpathlineto{\pgfqpoint{1.673428in}{2.092227in}}%
\pgfpathlineto{\pgfqpoint{1.673626in}{2.092227in}}%
\pgfpathlineto{\pgfqpoint{1.674023in}{2.214326in}}%
\pgfpathlineto{\pgfqpoint{1.674619in}{1.888729in}}%
\pgfpathlineto{\pgfqpoint{1.674817in}{1.888729in}}%
\pgfpathlineto{\pgfqpoint{1.675810in}{2.061702in}}%
\pgfpathlineto{\pgfqpoint{1.676008in}{2.061702in}}%
\pgfpathlineto{\pgfqpoint{1.676405in}{1.878554in}}%
\pgfpathlineto{\pgfqpoint{1.676604in}{2.153277in}}%
\pgfpathlineto{\pgfqpoint{1.677001in}{2.082052in}}%
\pgfpathlineto{\pgfqpoint{1.677199in}{2.082052in}}%
\pgfpathlineto{\pgfqpoint{1.677795in}{2.010828in}}%
\pgfpathlineto{\pgfqpoint{1.677398in}{2.173627in}}%
\pgfpathlineto{\pgfqpoint{1.678192in}{2.153277in}}%
\pgfpathlineto{\pgfqpoint{1.678390in}{2.153277in}}%
\pgfpathlineto{\pgfqpoint{1.678787in}{1.797154in}}%
\pgfpathlineto{\pgfqpoint{1.678589in}{2.244851in}}%
\pgfpathlineto{\pgfqpoint{1.679383in}{2.041353in}}%
\pgfpathlineto{\pgfqpoint{1.679581in}{2.041353in}}%
\pgfpathlineto{\pgfqpoint{1.679978in}{2.092227in}}%
\pgfpathlineto{\pgfqpoint{1.680177in}{1.949778in}}%
\pgfpathlineto{\pgfqpoint{1.680574in}{2.010828in}}%
\pgfpathlineto{\pgfqpoint{1.680772in}{2.010828in}}%
\pgfpathlineto{\pgfqpoint{1.681368in}{2.071877in}}%
\pgfpathlineto{\pgfqpoint{1.680971in}{1.909079in}}%
\pgfpathlineto{\pgfqpoint{1.681765in}{1.970128in}}%
\pgfpathlineto{\pgfqpoint{1.681963in}{1.970128in}}%
\pgfpathlineto{\pgfqpoint{1.682162in}{1.817504in}}%
\pgfpathlineto{\pgfqpoint{1.682559in}{2.112577in}}%
\pgfpathlineto{\pgfqpoint{1.682956in}{1.990478in}}%
\pgfpathlineto{\pgfqpoint{1.683154in}{1.990478in}}%
\pgfpathlineto{\pgfqpoint{1.683154in}{2.163452in}}%
\pgfpathlineto{\pgfqpoint{1.684146in}{1.807329in}}%
\pgfpathlineto{\pgfqpoint{1.684345in}{1.807329in}}%
\pgfpathlineto{\pgfqpoint{1.684742in}{2.102402in}}%
\pgfpathlineto{\pgfqpoint{1.685337in}{1.909079in}}%
\pgfpathlineto{\pgfqpoint{1.685536in}{1.909079in}}%
\pgfpathlineto{\pgfqpoint{1.686131in}{1.644531in}}%
\pgfpathlineto{\pgfqpoint{1.685933in}{1.959953in}}%
\pgfpathlineto{\pgfqpoint{1.686528in}{1.929428in}}%
\pgfpathlineto{\pgfqpoint{1.686727in}{1.929428in}}%
\pgfpathlineto{\pgfqpoint{1.687124in}{2.071877in}}%
\pgfpathlineto{\pgfqpoint{1.687719in}{1.705580in}}%
\pgfpathlineto{\pgfqpoint{1.687918in}{1.705580in}}%
\pgfpathlineto{\pgfqpoint{1.687918in}{1.959953in}}%
\pgfpathlineto{\pgfqpoint{1.688910in}{1.786980in}}%
\pgfpathlineto{\pgfqpoint{1.689109in}{1.786980in}}%
\pgfpathlineto{\pgfqpoint{1.689307in}{2.010828in}}%
\pgfpathlineto{\pgfqpoint{1.689704in}{1.776805in}}%
\pgfpathlineto{\pgfqpoint{1.690101in}{1.919254in}}%
\pgfpathlineto{\pgfqpoint{1.690300in}{1.919254in}}%
\pgfpathlineto{\pgfqpoint{1.690498in}{1.980303in}}%
\pgfpathlineto{\pgfqpoint{1.691292in}{1.746280in}}%
\pgfpathlineto{\pgfqpoint{1.691491in}{1.746280in}}%
\pgfpathlineto{\pgfqpoint{1.691888in}{2.051527in}}%
\pgfpathlineto{\pgfqpoint{1.692483in}{1.715755in}}%
\pgfpathlineto{\pgfqpoint{1.692682in}{1.715755in}}%
\pgfpathlineto{\pgfqpoint{1.693079in}{2.061702in}}%
\pgfpathlineto{\pgfqpoint{1.693674in}{1.868379in}}%
\pgfpathlineto{\pgfqpoint{1.693873in}{1.868379in}}%
\pgfpathlineto{\pgfqpoint{1.693873in}{1.929428in}}%
\pgfpathlineto{\pgfqpoint{1.694667in}{1.583481in}}%
\pgfpathlineto{\pgfqpoint{1.694865in}{1.817504in}}%
\pgfpathlineto{\pgfqpoint{1.695064in}{1.817504in}}%
\pgfpathlineto{\pgfqpoint{1.695461in}{1.980303in}}%
\pgfpathlineto{\pgfqpoint{1.695659in}{1.715755in}}%
\pgfpathlineto{\pgfqpoint{1.696056in}{1.868379in}}%
\pgfpathlineto{\pgfqpoint{1.696255in}{1.868379in}}%
\pgfpathlineto{\pgfqpoint{1.696850in}{1.705580in}}%
\pgfpathlineto{\pgfqpoint{1.696652in}{1.898904in}}%
\pgfpathlineto{\pgfqpoint{1.697247in}{1.837854in}}%
\pgfpathlineto{\pgfqpoint{1.697446in}{1.837854in}}%
\pgfpathlineto{\pgfqpoint{1.697644in}{2.051527in}}%
\pgfpathlineto{\pgfqpoint{1.698041in}{1.675055in}}%
\pgfpathlineto{\pgfqpoint{1.698438in}{1.858204in}}%
\pgfpathlineto{\pgfqpoint{1.698637in}{1.858204in}}%
\pgfpathlineto{\pgfqpoint{1.699431in}{1.593656in}}%
\pgfpathlineto{\pgfqpoint{1.699629in}{1.634356in}}%
\pgfpathlineto{\pgfqpoint{1.699828in}{1.634356in}}%
\pgfpathlineto{\pgfqpoint{1.700225in}{1.939603in}}%
\pgfpathlineto{\pgfqpoint{1.700820in}{1.817504in}}%
\pgfpathlineto{\pgfqpoint{1.701019in}{1.817504in}}%
\pgfpathlineto{\pgfqpoint{1.701019in}{1.837854in}}%
\pgfpathlineto{\pgfqpoint{1.701614in}{1.522432in}}%
\pgfpathlineto{\pgfqpoint{1.702011in}{1.776805in}}%
\pgfpathlineto{\pgfqpoint{1.702210in}{1.776805in}}%
\pgfpathlineto{\pgfqpoint{1.703004in}{1.624181in}}%
\pgfpathlineto{\pgfqpoint{1.703202in}{1.766630in}}%
\pgfpathlineto{\pgfqpoint{1.703401in}{1.766630in}}%
\pgfpathlineto{\pgfqpoint{1.704195in}{1.491907in}}%
\pgfpathlineto{\pgfqpoint{1.703798in}{1.786980in}}%
\pgfpathlineto{\pgfqpoint{1.704393in}{1.603831in}}%
\pgfpathlineto{\pgfqpoint{1.704592in}{1.603831in}}%
\pgfpathlineto{\pgfqpoint{1.705187in}{1.797154in}}%
\pgfpathlineto{\pgfqpoint{1.704790in}{1.593656in}}%
\pgfpathlineto{\pgfqpoint{1.705584in}{1.736105in}}%
\pgfpathlineto{\pgfqpoint{1.705981in}{1.736105in}}%
\pgfpathlineto{\pgfqpoint{1.706775in}{1.614006in}}%
\pgfpathlineto{\pgfqpoint{1.706378in}{1.837854in}}%
\pgfpathlineto{\pgfqpoint{1.706974in}{1.776805in}}%
\pgfpathlineto{\pgfqpoint{1.707172in}{1.776805in}}%
\pgfpathlineto{\pgfqpoint{1.707172in}{1.491907in}}%
\pgfpathlineto{\pgfqpoint{1.708165in}{1.644531in}}%
\pgfpathlineto{\pgfqpoint{1.708363in}{1.644531in}}%
\pgfpathlineto{\pgfqpoint{1.708760in}{1.563131in}}%
\pgfpathlineto{\pgfqpoint{1.709356in}{1.837854in}}%
\pgfpathlineto{\pgfqpoint{1.709554in}{1.837854in}}%
\pgfpathlineto{\pgfqpoint{1.710547in}{1.481732in}}%
\pgfpathlineto{\pgfqpoint{1.710745in}{1.481732in}}%
\pgfpathlineto{\pgfqpoint{1.710745in}{1.736105in}}%
\pgfpathlineto{\pgfqpoint{1.711738in}{1.583481in}}%
\pgfpathlineto{\pgfqpoint{1.711936in}{1.583481in}}%
\pgfpathlineto{\pgfqpoint{1.711936in}{1.776805in}}%
\pgfpathlineto{\pgfqpoint{1.712135in}{1.430857in}}%
\pgfpathlineto{\pgfqpoint{1.712929in}{1.614006in}}%
\pgfpathlineto{\pgfqpoint{1.713127in}{1.614006in}}%
\pgfpathlineto{\pgfqpoint{1.713127in}{1.573306in}}%
\pgfpathlineto{\pgfqpoint{1.714120in}{1.837854in}}%
\pgfpathlineto{\pgfqpoint{1.714318in}{1.837854in}}%
\pgfpathlineto{\pgfqpoint{1.714517in}{1.410508in}}%
\pgfpathlineto{\pgfqpoint{1.715311in}{1.552956in}}%
\pgfpathlineto{\pgfqpoint{1.715509in}{1.552956in}}%
\pgfpathlineto{\pgfqpoint{1.715708in}{1.522432in}}%
\pgfpathlineto{\pgfqpoint{1.715906in}{1.695405in}}%
\pgfpathlineto{\pgfqpoint{1.716501in}{1.603831in}}%
\pgfpathlineto{\pgfqpoint{1.716700in}{1.603831in}}%
\pgfpathlineto{\pgfqpoint{1.716898in}{1.461382in}}%
\pgfpathlineto{\pgfqpoint{1.717692in}{1.573306in}}%
\pgfpathlineto{\pgfqpoint{1.717891in}{1.573306in}}%
\pgfpathlineto{\pgfqpoint{1.718685in}{1.410508in}}%
\pgfpathlineto{\pgfqpoint{1.718486in}{1.766630in}}%
\pgfpathlineto{\pgfqpoint{1.718883in}{1.552956in}}%
\pgfpathlineto{\pgfqpoint{1.719082in}{1.552956in}}%
\pgfpathlineto{\pgfqpoint{1.719082in}{1.695405in}}%
\pgfpathlineto{\pgfqpoint{1.720074in}{1.654706in}}%
\pgfpathlineto{\pgfqpoint{1.720273in}{1.654706in}}%
\pgfpathlineto{\pgfqpoint{1.720273in}{1.766630in}}%
\pgfpathlineto{\pgfqpoint{1.721067in}{1.441032in}}%
\pgfpathlineto{\pgfqpoint{1.721265in}{1.471557in}}%
\pgfpathlineto{\pgfqpoint{1.721464in}{1.471557in}}%
\pgfpathlineto{\pgfqpoint{1.721662in}{1.664881in}}%
\pgfpathlineto{\pgfqpoint{1.722059in}{1.461382in}}%
\pgfpathlineto{\pgfqpoint{1.722456in}{1.471557in}}%
\pgfpathlineto{\pgfqpoint{1.722655in}{1.471557in}}%
\pgfpathlineto{\pgfqpoint{1.723052in}{1.685230in}}%
\pgfpathlineto{\pgfqpoint{1.723647in}{1.573306in}}%
\pgfpathlineto{\pgfqpoint{1.723846in}{1.573306in}}%
\pgfpathlineto{\pgfqpoint{1.723846in}{1.359633in}}%
\pgfpathlineto{\pgfqpoint{1.724838in}{1.563131in}}%
\pgfpathlineto{\pgfqpoint{1.725037in}{1.563131in}}%
\pgfpathlineto{\pgfqpoint{1.725632in}{1.379983in}}%
\pgfpathlineto{\pgfqpoint{1.726029in}{1.563131in}}%
\pgfpathlineto{\pgfqpoint{1.726228in}{1.563131in}}%
\pgfpathlineto{\pgfqpoint{1.726228in}{1.329108in}}%
\pgfpathlineto{\pgfqpoint{1.726625in}{1.593656in}}%
\pgfpathlineto{\pgfqpoint{1.727220in}{1.512257in}}%
\pgfpathlineto{\pgfqpoint{1.727419in}{1.512257in}}%
\pgfpathlineto{\pgfqpoint{1.728014in}{1.359633in}}%
\pgfpathlineto{\pgfqpoint{1.727816in}{1.532607in}}%
\pgfpathlineto{\pgfqpoint{1.728411in}{1.461382in}}%
\pgfpathlineto{\pgfqpoint{1.728610in}{1.461382in}}%
\pgfpathlineto{\pgfqpoint{1.729007in}{1.268059in}}%
\pgfpathlineto{\pgfqpoint{1.729602in}{1.532607in}}%
\pgfpathlineto{\pgfqpoint{1.729801in}{1.532607in}}%
\pgfpathlineto{\pgfqpoint{1.729999in}{1.542781in}}%
\pgfpathlineto{\pgfqpoint{1.730793in}{1.329108in}}%
\pgfpathlineto{\pgfqpoint{1.730992in}{1.329108in}}%
\pgfpathlineto{\pgfqpoint{1.731786in}{1.522432in}}%
\pgfpathlineto{\pgfqpoint{1.731984in}{1.481732in}}%
\pgfpathlineto{\pgfqpoint{1.732183in}{1.481732in}}%
\pgfpathlineto{\pgfqpoint{1.732183in}{1.542781in}}%
\pgfpathlineto{\pgfqpoint{1.732580in}{1.339283in}}%
\pgfpathlineto{\pgfqpoint{1.733175in}{1.420682in}}%
\pgfpathlineto{\pgfqpoint{1.733374in}{1.420682in}}%
\pgfpathlineto{\pgfqpoint{1.733572in}{1.451207in}}%
\pgfpathlineto{\pgfqpoint{1.733969in}{1.298583in}}%
\pgfpathlineto{\pgfqpoint{1.734366in}{1.390158in}}%
\pgfpathlineto{\pgfqpoint{1.734565in}{1.390158in}}%
\pgfpathlineto{\pgfqpoint{1.734962in}{1.329108in}}%
\pgfpathlineto{\pgfqpoint{1.735359in}{1.461382in}}%
\pgfpathlineto{\pgfqpoint{1.735557in}{1.441032in}}%
\pgfpathlineto{\pgfqpoint{1.735756in}{1.441032in}}%
\pgfpathlineto{\pgfqpoint{1.736153in}{1.471557in}}%
\pgfpathlineto{\pgfqpoint{1.735954in}{1.379983in}}%
\pgfpathlineto{\pgfqpoint{1.736748in}{1.379983in}}%
\pgfpathlineto{\pgfqpoint{1.736947in}{1.379983in}}%
\pgfpathlineto{\pgfqpoint{1.737344in}{1.318933in}}%
\pgfpathlineto{\pgfqpoint{1.737145in}{1.563131in}}%
\pgfpathlineto{\pgfqpoint{1.737939in}{1.349458in}}%
\pgfpathlineto{\pgfqpoint{1.738138in}{1.349458in}}%
\pgfpathlineto{\pgfqpoint{1.738336in}{1.552956in}}%
\pgfpathlineto{\pgfqpoint{1.738733in}{1.237534in}}%
\pgfpathlineto{\pgfqpoint{1.739130in}{1.379983in}}%
\pgfpathlineto{\pgfqpoint{1.739329in}{1.379983in}}%
\pgfpathlineto{\pgfqpoint{1.739527in}{1.420682in}}%
\pgfpathlineto{\pgfqpoint{1.739726in}{1.359633in}}%
\pgfpathlineto{\pgfqpoint{1.740321in}{1.420682in}}%
\pgfpathlineto{\pgfqpoint{1.740520in}{1.420682in}}%
\pgfpathlineto{\pgfqpoint{1.741314in}{1.441032in}}%
\pgfpathlineto{\pgfqpoint{1.740718in}{1.227359in}}%
\pgfpathlineto{\pgfqpoint{1.741512in}{1.247709in}}%
\pgfpathlineto{\pgfqpoint{1.741711in}{1.247709in}}%
\pgfpathlineto{\pgfqpoint{1.742306in}{1.420682in}}%
\pgfpathlineto{\pgfqpoint{1.742703in}{1.135785in}}%
\pgfpathlineto{\pgfqpoint{1.742902in}{1.135785in}}%
\pgfpathlineto{\pgfqpoint{1.742902in}{1.481732in}}%
\pgfpathlineto{\pgfqpoint{1.743894in}{1.318933in}}%
\pgfpathlineto{\pgfqpoint{1.744093in}{1.318933in}}%
\pgfpathlineto{\pgfqpoint{1.744490in}{1.186659in}}%
\pgfpathlineto{\pgfqpoint{1.744887in}{1.379983in}}%
\pgfpathlineto{\pgfqpoint{1.745085in}{1.329108in}}%
\pgfpathlineto{\pgfqpoint{1.745284in}{1.329108in}}%
\pgfpathlineto{\pgfqpoint{1.745284in}{1.441032in}}%
\pgfpathlineto{\pgfqpoint{1.746078in}{1.298583in}}%
\pgfpathlineto{\pgfqpoint{1.746276in}{1.329108in}}%
\pgfpathlineto{\pgfqpoint{1.746475in}{1.329108in}}%
\pgfpathlineto{\pgfqpoint{1.746475in}{1.491907in}}%
\pgfpathlineto{\pgfqpoint{1.746673in}{1.288408in}}%
\pgfpathlineto{\pgfqpoint{1.747467in}{1.298583in}}%
\pgfpathlineto{\pgfqpoint{1.747666in}{1.298583in}}%
\pgfpathlineto{\pgfqpoint{1.748459in}{1.186659in}}%
\pgfpathlineto{\pgfqpoint{1.747864in}{1.349458in}}%
\pgfpathlineto{\pgfqpoint{1.748658in}{1.318933in}}%
\pgfpathlineto{\pgfqpoint{1.749055in}{1.318933in}}%
\pgfpathlineto{\pgfqpoint{1.749849in}{1.145960in}}%
\pgfpathlineto{\pgfqpoint{1.750047in}{1.247709in}}%
\pgfpathlineto{\pgfqpoint{1.750246in}{1.247709in}}%
\pgfpathlineto{\pgfqpoint{1.750246in}{1.227359in}}%
\pgfpathlineto{\pgfqpoint{1.751238in}{1.349458in}}%
\pgfpathlineto{\pgfqpoint{1.751437in}{1.349458in}}%
\pgfpathlineto{\pgfqpoint{1.752231in}{1.186659in}}%
\pgfpathlineto{\pgfqpoint{1.751834in}{1.441032in}}%
\pgfpathlineto{\pgfqpoint{1.752429in}{1.288408in}}%
\pgfpathlineto{\pgfqpoint{1.752628in}{1.288408in}}%
\pgfpathlineto{\pgfqpoint{1.752628in}{1.349458in}}%
\pgfpathlineto{\pgfqpoint{1.753620in}{1.064560in}}%
\pgfpathlineto{\pgfqpoint{1.753819in}{1.064560in}}%
\pgfpathlineto{\pgfqpoint{1.754613in}{1.390158in}}%
\pgfpathlineto{\pgfqpoint{1.754811in}{1.176484in}}%
\pgfpathlineto{\pgfqpoint{1.755010in}{1.176484in}}%
\pgfpathlineto{\pgfqpoint{1.755804in}{1.349458in}}%
\pgfpathlineto{\pgfqpoint{1.756002in}{1.257884in}}%
\pgfpathlineto{\pgfqpoint{1.756201in}{1.257884in}}%
\pgfpathlineto{\pgfqpoint{1.756201in}{1.176484in}}%
\pgfpathlineto{\pgfqpoint{1.756995in}{1.278234in}}%
\pgfpathlineto{\pgfqpoint{1.757193in}{1.247709in}}%
\pgfpathlineto{\pgfqpoint{1.757392in}{1.247709in}}%
\pgfpathlineto{\pgfqpoint{1.758186in}{1.105260in}}%
\pgfpathlineto{\pgfqpoint{1.757987in}{1.278234in}}%
\pgfpathlineto{\pgfqpoint{1.758384in}{1.176484in}}%
\pgfpathlineto{\pgfqpoint{1.758583in}{1.176484in}}%
\pgfpathlineto{\pgfqpoint{1.759178in}{1.227359in}}%
\pgfpathlineto{\pgfqpoint{1.758980in}{1.135785in}}%
\pgfpathlineto{\pgfqpoint{1.759575in}{1.227359in}}%
\pgfpathlineto{\pgfqpoint{1.759774in}{1.227359in}}%
\pgfpathlineto{\pgfqpoint{1.759972in}{1.318933in}}%
\pgfpathlineto{\pgfqpoint{1.760766in}{1.135785in}}%
\pgfpathlineto{\pgfqpoint{1.760965in}{1.135785in}}%
\pgfpathlineto{\pgfqpoint{1.761560in}{1.074735in}}%
\pgfpathlineto{\pgfqpoint{1.761759in}{1.227359in}}%
\pgfpathlineto{\pgfqpoint{1.761957in}{1.145960in}}%
\pgfpathlineto{\pgfqpoint{1.762156in}{1.145960in}}%
\pgfpathlineto{\pgfqpoint{1.762950in}{1.186659in}}%
\pgfpathlineto{\pgfqpoint{1.763148in}{1.064560in}}%
\pgfpathlineto{\pgfqpoint{1.763347in}{1.064560in}}%
\pgfpathlineto{\pgfqpoint{1.763744in}{1.247709in}}%
\pgfpathlineto{\pgfqpoint{1.764339in}{1.125610in}}%
\pgfpathlineto{\pgfqpoint{1.764538in}{1.125610in}}%
\pgfpathlineto{\pgfqpoint{1.764736in}{1.074735in}}%
\pgfpathlineto{\pgfqpoint{1.765133in}{1.196834in}}%
\pgfpathlineto{\pgfqpoint{1.765530in}{1.176484in}}%
\pgfpathlineto{\pgfqpoint{1.765729in}{1.176484in}}%
\pgfpathlineto{\pgfqpoint{1.766126in}{1.318933in}}%
\pgfpathlineto{\pgfqpoint{1.766721in}{1.013686in}}%
\pgfpathlineto{\pgfqpoint{1.766920in}{1.013686in}}%
\pgfpathlineto{\pgfqpoint{1.767515in}{1.237534in}}%
\pgfpathlineto{\pgfqpoint{1.767912in}{1.095085in}}%
\pgfpathlineto{\pgfqpoint{1.768111in}{1.095085in}}%
\pgfpathlineto{\pgfqpoint{1.768309in}{1.217184in}}%
\pgfpathlineto{\pgfqpoint{1.769103in}{1.145960in}}%
\pgfpathlineto{\pgfqpoint{1.769302in}{1.145960in}}%
\pgfpathlineto{\pgfqpoint{1.769699in}{1.064560in}}%
\pgfpathlineto{\pgfqpoint{1.770294in}{1.145960in}}%
\pgfpathlineto{\pgfqpoint{1.770493in}{1.145960in}}%
\pgfpathlineto{\pgfqpoint{1.770493in}{1.217184in}}%
\pgfpathlineto{\pgfqpoint{1.771088in}{0.972986in}}%
\pgfpathlineto{\pgfqpoint{1.771485in}{1.095085in}}%
\pgfpathlineto{\pgfqpoint{1.771882in}{1.095085in}}%
\pgfpathlineto{\pgfqpoint{1.772676in}{1.054385in}}%
\pgfpathlineto{\pgfqpoint{1.772875in}{1.227359in}}%
\pgfpathlineto{\pgfqpoint{1.773073in}{1.227359in}}%
\pgfpathlineto{\pgfqpoint{1.773470in}{1.074735in}}%
\pgfpathlineto{\pgfqpoint{1.774066in}{1.166309in}}%
\pgfpathlineto{\pgfqpoint{1.774264in}{1.166309in}}%
\pgfpathlineto{\pgfqpoint{1.775058in}{1.034035in}}%
\pgfpathlineto{\pgfqpoint{1.775257in}{1.196834in}}%
\pgfpathlineto{\pgfqpoint{1.775455in}{1.196834in}}%
\pgfpathlineto{\pgfqpoint{1.775654in}{1.034035in}}%
\pgfpathlineto{\pgfqpoint{1.776249in}{1.237534in}}%
\pgfpathlineto{\pgfqpoint{1.776448in}{1.156135in}}%
\pgfpathlineto{\pgfqpoint{1.776646in}{1.156135in}}%
\pgfpathlineto{\pgfqpoint{1.776845in}{1.064560in}}%
\pgfpathlineto{\pgfqpoint{1.777639in}{1.156135in}}%
\pgfpathlineto{\pgfqpoint{1.777837in}{1.156135in}}%
\pgfpathlineto{\pgfqpoint{1.777837in}{1.207009in}}%
\pgfpathlineto{\pgfqpoint{1.778234in}{0.962811in}}%
\pgfpathlineto{\pgfqpoint{1.778830in}{1.084910in}}%
\pgfpathlineto{\pgfqpoint{1.779028in}{1.084910in}}%
\pgfpathlineto{\pgfqpoint{1.779624in}{1.125610in}}%
\pgfpathlineto{\pgfqpoint{1.779425in}{1.023861in}}%
\pgfpathlineto{\pgfqpoint{1.780021in}{1.084910in}}%
\pgfpathlineto{\pgfqpoint{1.780219in}{1.084910in}}%
\pgfpathlineto{\pgfqpoint{1.781211in}{0.972986in}}%
\pgfpathlineto{\pgfqpoint{1.781410in}{0.972986in}}%
\pgfpathlineto{\pgfqpoint{1.781410in}{1.176484in}}%
\pgfpathlineto{\pgfqpoint{1.781608in}{0.922111in}}%
\pgfpathlineto{\pgfqpoint{1.782402in}{1.044210in}}%
\pgfpathlineto{\pgfqpoint{1.782799in}{1.044210in}}%
\pgfpathlineto{\pgfqpoint{1.782799in}{1.156135in}}%
\pgfpathlineto{\pgfqpoint{1.783792in}{1.145960in}}%
\pgfpathlineto{\pgfqpoint{1.783990in}{1.145960in}}%
\pgfpathlineto{\pgfqpoint{1.784189in}{1.013686in}}%
\pgfpathlineto{\pgfqpoint{1.784983in}{1.105260in}}%
\pgfpathlineto{\pgfqpoint{1.785181in}{1.105260in}}%
\pgfpathlineto{\pgfqpoint{1.785975in}{0.942461in}}%
\pgfpathlineto{\pgfqpoint{1.785578in}{1.156135in}}%
\pgfpathlineto{\pgfqpoint{1.786174in}{1.013686in}}%
\pgfpathlineto{\pgfqpoint{1.786372in}{1.013686in}}%
\pgfpathlineto{\pgfqpoint{1.786968in}{1.074735in}}%
\pgfpathlineto{\pgfqpoint{1.787365in}{0.952636in}}%
\pgfpathlineto{\pgfqpoint{1.787563in}{0.952636in}}%
\pgfpathlineto{\pgfqpoint{1.788556in}{1.105260in}}%
\pgfpathlineto{\pgfqpoint{1.788754in}{1.105260in}}%
\pgfpathlineto{\pgfqpoint{1.788754in}{1.003511in}}%
\pgfpathlineto{\pgfqpoint{1.789747in}{1.023861in}}%
\pgfpathlineto{\pgfqpoint{1.789945in}{1.023861in}}%
\pgfpathlineto{\pgfqpoint{1.790144in}{1.095085in}}%
\pgfpathlineto{\pgfqpoint{1.790938in}{0.942461in}}%
\pgfpathlineto{\pgfqpoint{1.791136in}{0.942461in}}%
\pgfpathlineto{\pgfqpoint{1.791930in}{1.054385in}}%
\pgfpathlineto{\pgfqpoint{1.791335in}{0.911936in}}%
\pgfpathlineto{\pgfqpoint{1.792129in}{0.952636in}}%
\pgfpathlineto{\pgfqpoint{1.792327in}{0.952636in}}%
\pgfpathlineto{\pgfqpoint{1.792923in}{1.105260in}}%
\pgfpathlineto{\pgfqpoint{1.792526in}{0.901762in}}%
\pgfpathlineto{\pgfqpoint{1.793320in}{0.952636in}}%
\pgfpathlineto{\pgfqpoint{1.793518in}{0.952636in}}%
\pgfpathlineto{\pgfqpoint{1.793915in}{1.034035in}}%
\pgfpathlineto{\pgfqpoint{1.794511in}{0.983161in}}%
\pgfpathlineto{\pgfqpoint{1.794709in}{0.983161in}}%
\pgfpathlineto{\pgfqpoint{1.794709in}{0.901762in}}%
\pgfpathlineto{\pgfqpoint{1.795106in}{1.064560in}}%
\pgfpathlineto{\pgfqpoint{1.795702in}{1.034035in}}%
\pgfpathlineto{\pgfqpoint{1.795900in}{1.034035in}}%
\pgfpathlineto{\pgfqpoint{1.796496in}{0.942461in}}%
\pgfpathlineto{\pgfqpoint{1.796893in}{1.034035in}}%
\pgfpathlineto{\pgfqpoint{1.797290in}{1.034035in}}%
\pgfpathlineto{\pgfqpoint{1.797488in}{1.044210in}}%
\pgfpathlineto{\pgfqpoint{1.798282in}{0.881412in}}%
\pgfpathlineto{\pgfqpoint{1.798481in}{0.881412in}}%
\pgfpathlineto{\pgfqpoint{1.799076in}{1.074735in}}%
\pgfpathlineto{\pgfqpoint{1.799473in}{0.881412in}}%
\pgfpathlineto{\pgfqpoint{1.799672in}{0.881412in}}%
\pgfpathlineto{\pgfqpoint{1.800069in}{1.095085in}}%
\pgfpathlineto{\pgfqpoint{1.800664in}{0.891587in}}%
\pgfpathlineto{\pgfqpoint{1.800863in}{0.891587in}}%
\pgfpathlineto{\pgfqpoint{1.801458in}{0.993336in}}%
\pgfpathlineto{\pgfqpoint{1.801855in}{0.993336in}}%
\pgfpathlineto{\pgfqpoint{1.802054in}{0.993336in}}%
\pgfpathlineto{\pgfqpoint{1.802054in}{1.013686in}}%
\pgfpathlineto{\pgfqpoint{1.802649in}{0.911936in}}%
\pgfpathlineto{\pgfqpoint{1.803046in}{0.952636in}}%
\pgfpathlineto{\pgfqpoint{1.803245in}{0.952636in}}%
\pgfpathlineto{\pgfqpoint{1.803443in}{0.901762in}}%
\pgfpathlineto{\pgfqpoint{1.803840in}{1.013686in}}%
\pgfpathlineto{\pgfqpoint{1.804237in}{0.972986in}}%
\pgfpathlineto{\pgfqpoint{1.804436in}{0.972986in}}%
\pgfpathlineto{\pgfqpoint{1.804436in}{1.084910in}}%
\pgfpathlineto{\pgfqpoint{1.805230in}{0.820362in}}%
\pgfpathlineto{\pgfqpoint{1.805428in}{0.861062in}}%
\pgfpathlineto{\pgfqpoint{1.805627in}{0.861062in}}%
\pgfpathlineto{\pgfqpoint{1.806222in}{0.972986in}}%
\pgfpathlineto{\pgfqpoint{1.806619in}{0.891587in}}%
\pgfpathlineto{\pgfqpoint{1.806818in}{0.891587in}}%
\pgfpathlineto{\pgfqpoint{1.806818in}{1.044210in}}%
\pgfpathlineto{\pgfqpoint{1.807810in}{0.861062in}}%
\pgfpathlineto{\pgfqpoint{1.808009in}{0.861062in}}%
\pgfpathlineto{\pgfqpoint{1.808406in}{1.013686in}}%
\pgfpathlineto{\pgfqpoint{1.809001in}{0.911936in}}%
\pgfpathlineto{\pgfqpoint{1.809200in}{0.911936in}}%
\pgfpathlineto{\pgfqpoint{1.809200in}{0.952636in}}%
\pgfpathlineto{\pgfqpoint{1.810192in}{0.830537in}}%
\pgfpathlineto{\pgfqpoint{1.810391in}{0.830537in}}%
\pgfpathlineto{\pgfqpoint{1.810589in}{0.983161in}}%
\pgfpathlineto{\pgfqpoint{1.810788in}{0.820362in}}%
\pgfpathlineto{\pgfqpoint{1.811383in}{0.891587in}}%
\pgfpathlineto{\pgfqpoint{1.811582in}{0.891587in}}%
\pgfpathlineto{\pgfqpoint{1.811582in}{0.810187in}}%
\pgfpathlineto{\pgfqpoint{1.812574in}{0.962811in}}%
\pgfpathlineto{\pgfqpoint{1.812773in}{0.962811in}}%
\pgfpathlineto{\pgfqpoint{1.813368in}{0.800012in}}%
\pgfpathlineto{\pgfqpoint{1.813765in}{0.881412in}}%
\pgfpathlineto{\pgfqpoint{1.814162in}{0.881412in}}%
\pgfpathlineto{\pgfqpoint{1.814559in}{1.044210in}}%
\pgfpathlineto{\pgfqpoint{1.814956in}{0.871237in}}%
\pgfpathlineto{\pgfqpoint{1.815154in}{0.881412in}}%
\pgfpathlineto{\pgfqpoint{1.815353in}{0.881412in}}%
\pgfpathlineto{\pgfqpoint{1.815353in}{0.952636in}}%
\pgfpathlineto{\pgfqpoint{1.816147in}{0.871237in}}%
\pgfpathlineto{\pgfqpoint{1.816345in}{0.901762in}}%
\pgfpathlineto{\pgfqpoint{1.816544in}{0.901762in}}%
\pgfpathlineto{\pgfqpoint{1.816544in}{0.942461in}}%
\pgfpathlineto{\pgfqpoint{1.816742in}{0.850887in}}%
\pgfpathlineto{\pgfqpoint{1.817536in}{0.881412in}}%
\pgfpathlineto{\pgfqpoint{1.817735in}{0.881412in}}%
\pgfpathlineto{\pgfqpoint{1.817933in}{0.972986in}}%
\pgfpathlineto{\pgfqpoint{1.818132in}{0.830537in}}%
\pgfpathlineto{\pgfqpoint{1.818727in}{0.901762in}}%
\pgfpathlineto{\pgfqpoint{1.818926in}{0.901762in}}%
\pgfpathlineto{\pgfqpoint{1.819720in}{0.983161in}}%
\pgfpathlineto{\pgfqpoint{1.819124in}{0.810187in}}%
\pgfpathlineto{\pgfqpoint{1.819918in}{0.891587in}}%
\pgfpathlineto{\pgfqpoint{1.820514in}{0.891587in}}%
\pgfpathlineto{\pgfqpoint{1.820514in}{0.800012in}}%
\pgfpathlineto{\pgfqpoint{1.820712in}{0.942461in}}%
\pgfpathlineto{\pgfqpoint{1.821506in}{0.840712in}}%
\pgfpathlineto{\pgfqpoint{1.821705in}{0.840712in}}%
\pgfpathlineto{\pgfqpoint{1.822300in}{0.932286in}}%
\pgfpathlineto{\pgfqpoint{1.821903in}{0.820362in}}%
\pgfpathlineto{\pgfqpoint{1.822697in}{0.871237in}}%
\pgfpathlineto{\pgfqpoint{1.822896in}{0.871237in}}%
\pgfpathlineto{\pgfqpoint{1.823690in}{0.800012in}}%
\pgfpathlineto{\pgfqpoint{1.823094in}{0.942461in}}%
\pgfpathlineto{\pgfqpoint{1.823888in}{0.901762in}}%
\pgfpathlineto{\pgfqpoint{1.824087in}{0.901762in}}%
\pgfpathlineto{\pgfqpoint{1.824087in}{0.850887in}}%
\pgfpathlineto{\pgfqpoint{1.825079in}{1.013686in}}%
\pgfpathlineto{\pgfqpoint{1.825278in}{1.013686in}}%
\pgfpathlineto{\pgfqpoint{1.825278in}{0.840712in}}%
\pgfpathlineto{\pgfqpoint{1.826270in}{0.891587in}}%
\pgfpathlineto{\pgfqpoint{1.826469in}{0.891587in}}%
\pgfpathlineto{\pgfqpoint{1.826469in}{0.911936in}}%
\pgfpathlineto{\pgfqpoint{1.827064in}{0.830537in}}%
\pgfpathlineto{\pgfqpoint{1.827461in}{0.891587in}}%
\pgfpathlineto{\pgfqpoint{1.827660in}{0.891587in}}%
\pgfpathlineto{\pgfqpoint{1.827858in}{0.800012in}}%
\pgfpathlineto{\pgfqpoint{1.828454in}{0.901762in}}%
\pgfpathlineto{\pgfqpoint{1.828652in}{0.850887in}}%
\pgfpathlineto{\pgfqpoint{1.828851in}{0.850887in}}%
\pgfpathlineto{\pgfqpoint{1.828851in}{0.881412in}}%
\pgfpathlineto{\pgfqpoint{1.829843in}{0.820362in}}%
\pgfpathlineto{\pgfqpoint{1.830240in}{0.820362in}}%
\pgfpathlineto{\pgfqpoint{1.830240in}{0.911936in}}%
\pgfpathlineto{\pgfqpoint{1.831034in}{0.800012in}}%
\pgfpathlineto{\pgfqpoint{1.831233in}{0.850887in}}%
\pgfpathlineto{\pgfqpoint{1.831431in}{0.850887in}}%
\pgfpathlineto{\pgfqpoint{1.831431in}{0.891587in}}%
\pgfpathlineto{\pgfqpoint{1.832424in}{0.820362in}}%
\pgfpathlineto{\pgfqpoint{1.833019in}{0.820362in}}%
\pgfpathlineto{\pgfqpoint{1.833218in}{0.891587in}}%
\pgfpathlineto{\pgfqpoint{1.833416in}{0.789837in}}%
\pgfpathlineto{\pgfqpoint{1.834012in}{0.850887in}}%
\pgfpathlineto{\pgfqpoint{1.834210in}{0.850887in}}%
\pgfpathlineto{\pgfqpoint{1.834409in}{0.810187in}}%
\pgfpathlineto{\pgfqpoint{1.834607in}{0.871237in}}%
\pgfpathlineto{\pgfqpoint{1.835203in}{0.840712in}}%
\pgfpathlineto{\pgfqpoint{1.835401in}{0.840712in}}%
\pgfpathlineto{\pgfqpoint{1.835600in}{0.779662in}}%
\pgfpathlineto{\pgfqpoint{1.835798in}{0.881412in}}%
\pgfpathlineto{\pgfqpoint{1.836394in}{0.800012in}}%
\pgfpathlineto{\pgfqpoint{1.836592in}{0.800012in}}%
\pgfpathlineto{\pgfqpoint{1.836989in}{0.789837in}}%
\pgfpathlineto{\pgfqpoint{1.837585in}{0.871237in}}%
\pgfpathlineto{\pgfqpoint{1.837783in}{0.871237in}}%
\pgfpathlineto{\pgfqpoint{1.838577in}{0.800012in}}%
\pgfpathlineto{\pgfqpoint{1.838776in}{0.871237in}}%
\pgfpathlineto{\pgfqpoint{1.838974in}{0.871237in}}%
\pgfpathlineto{\pgfqpoint{1.839173in}{0.820362in}}%
\pgfpathlineto{\pgfqpoint{1.839967in}{0.840712in}}%
\pgfpathlineto{\pgfqpoint{1.840165in}{0.840712in}}%
\pgfpathlineto{\pgfqpoint{1.840165in}{0.820362in}}%
\pgfpathlineto{\pgfqpoint{1.840562in}{0.861062in}}%
\pgfpathlineto{\pgfqpoint{1.841158in}{0.830537in}}%
\pgfpathlineto{\pgfqpoint{1.841356in}{0.830537in}}%
\pgfpathlineto{\pgfqpoint{1.841753in}{0.871237in}}%
\pgfpathlineto{\pgfqpoint{1.841555in}{0.810187in}}%
\pgfpathlineto{\pgfqpoint{1.842349in}{0.820362in}}%
\pgfpathlineto{\pgfqpoint{1.842547in}{0.820362in}}%
\pgfpathlineto{\pgfqpoint{1.842547in}{0.901762in}}%
\pgfpathlineto{\pgfqpoint{1.842944in}{0.810187in}}%
\pgfpathlineto{\pgfqpoint{1.843540in}{0.850887in}}%
\pgfpathlineto{\pgfqpoint{1.843738in}{0.850887in}}%
\pgfpathlineto{\pgfqpoint{1.844532in}{0.901762in}}%
\pgfpathlineto{\pgfqpoint{1.844731in}{0.810187in}}%
\pgfpathlineto{\pgfqpoint{1.844929in}{0.810187in}}%
\pgfpathlineto{\pgfqpoint{1.845921in}{0.881412in}}%
\pgfpathlineto{\pgfqpoint{1.846120in}{0.881412in}}%
\pgfpathlineto{\pgfqpoint{1.846120in}{0.759313in}}%
\pgfpathlineto{\pgfqpoint{1.847112in}{0.810187in}}%
\pgfpathlineto{\pgfqpoint{1.847311in}{0.810187in}}%
\pgfpathlineto{\pgfqpoint{1.847509in}{0.881412in}}%
\pgfpathlineto{\pgfqpoint{1.847708in}{0.769488in}}%
\pgfpathlineto{\pgfqpoint{1.848303in}{0.820362in}}%
\pgfpathlineto{\pgfqpoint{1.848502in}{0.820362in}}%
\pgfpathlineto{\pgfqpoint{1.849296in}{0.850887in}}%
\pgfpathlineto{\pgfqpoint{1.849494in}{0.800012in}}%
\pgfpathlineto{\pgfqpoint{1.849693in}{0.800012in}}%
\pgfpathlineto{\pgfqpoint{1.850288in}{0.881412in}}%
\pgfpathlineto{\pgfqpoint{1.850487in}{0.789837in}}%
\pgfpathlineto{\pgfqpoint{1.850685in}{0.800012in}}%
\pgfpathlineto{\pgfqpoint{1.850884in}{0.800012in}}%
\pgfpathlineto{\pgfqpoint{1.851678in}{0.789837in}}%
\pgfpathlineto{\pgfqpoint{1.851876in}{0.871237in}}%
\pgfpathlineto{\pgfqpoint{1.852075in}{0.871237in}}%
\pgfpathlineto{\pgfqpoint{1.852472in}{0.789837in}}%
\pgfpathlineto{\pgfqpoint{1.853067in}{0.830537in}}%
\pgfpathlineto{\pgfqpoint{1.853464in}{0.830537in}}%
\pgfpathlineto{\pgfqpoint{1.854060in}{0.871237in}}%
\pgfpathlineto{\pgfqpoint{1.853663in}{0.779662in}}%
\pgfpathlineto{\pgfqpoint{1.854457in}{0.779662in}}%
\pgfpathlineto{\pgfqpoint{1.854655in}{0.779662in}}%
\pgfpathlineto{\pgfqpoint{1.855449in}{0.850887in}}%
\pgfpathlineto{\pgfqpoint{1.855251in}{0.759313in}}%
\pgfpathlineto{\pgfqpoint{1.855648in}{0.850887in}}%
\pgfpathlineto{\pgfqpoint{1.855846in}{0.850887in}}%
\pgfpathlineto{\pgfqpoint{1.855846in}{0.769488in}}%
\pgfpathlineto{\pgfqpoint{1.856839in}{0.881412in}}%
\pgfpathlineto{\pgfqpoint{1.857037in}{0.881412in}}%
\pgfpathlineto{\pgfqpoint{1.857434in}{0.800012in}}%
\pgfpathlineto{\pgfqpoint{1.858030in}{0.820362in}}%
\pgfpathlineto{\pgfqpoint{1.858427in}{0.820362in}}%
\pgfpathlineto{\pgfqpoint{1.858824in}{0.749138in}}%
\pgfpathlineto{\pgfqpoint{1.859419in}{0.789837in}}%
\pgfpathlineto{\pgfqpoint{1.859618in}{0.789837in}}%
\pgfpathlineto{\pgfqpoint{1.859816in}{0.820362in}}%
\pgfpathlineto{\pgfqpoint{1.860412in}{0.779662in}}%
\pgfpathlineto{\pgfqpoint{1.860610in}{0.789837in}}%
\pgfpathlineto{\pgfqpoint{1.860809in}{0.789837in}}%
\pgfpathlineto{\pgfqpoint{1.861404in}{0.820362in}}%
\pgfpathlineto{\pgfqpoint{1.861801in}{0.779662in}}%
\pgfpathlineto{\pgfqpoint{1.862000in}{0.779662in}}%
\pgfpathlineto{\pgfqpoint{1.862794in}{0.820362in}}%
\pgfpathlineto{\pgfqpoint{1.862198in}{0.759313in}}%
\pgfpathlineto{\pgfqpoint{1.862992in}{0.800012in}}%
\pgfpathlineto{\pgfqpoint{1.863191in}{0.800012in}}%
\pgfpathlineto{\pgfqpoint{1.863191in}{0.830537in}}%
\pgfpathlineto{\pgfqpoint{1.863786in}{0.769488in}}%
\pgfpathlineto{\pgfqpoint{1.864183in}{0.800012in}}%
\pgfpathlineto{\pgfqpoint{1.864382in}{0.800012in}}%
\pgfpathlineto{\pgfqpoint{1.864580in}{0.759313in}}%
\pgfpathlineto{\pgfqpoint{1.864779in}{0.850887in}}%
\pgfpathlineto{\pgfqpoint{1.865374in}{0.789837in}}%
\pgfpathlineto{\pgfqpoint{1.865573in}{0.789837in}}%
\pgfpathlineto{\pgfqpoint{1.866168in}{0.850887in}}%
\pgfpathlineto{\pgfqpoint{1.866565in}{0.769488in}}%
\pgfpathlineto{\pgfqpoint{1.866764in}{0.769488in}}%
\pgfpathlineto{\pgfqpoint{1.867359in}{0.759313in}}%
\pgfpathlineto{\pgfqpoint{1.867756in}{0.850887in}}%
\pgfpathlineto{\pgfqpoint{1.867955in}{0.850887in}}%
\pgfpathlineto{\pgfqpoint{1.867955in}{0.779662in}}%
\pgfpathlineto{\pgfqpoint{1.868947in}{0.830537in}}%
\pgfpathlineto{\pgfqpoint{1.869146in}{0.830537in}}%
\pgfpathlineto{\pgfqpoint{1.869543in}{0.759313in}}%
\pgfpathlineto{\pgfqpoint{1.870138in}{0.789837in}}%
\pgfpathlineto{\pgfqpoint{1.870535in}{0.789837in}}%
\pgfpathlineto{\pgfqpoint{1.871131in}{0.759313in}}%
\pgfpathlineto{\pgfqpoint{1.871528in}{0.840712in}}%
\pgfpathlineto{\pgfqpoint{1.871726in}{0.840712in}}%
\pgfpathlineto{\pgfqpoint{1.872719in}{0.749138in}}%
\pgfpathlineto{\pgfqpoint{1.872917in}{0.749138in}}%
\pgfpathlineto{\pgfqpoint{1.873910in}{0.810187in}}%
\pgfpathlineto{\pgfqpoint{1.874108in}{0.810187in}}%
\pgfpathlineto{\pgfqpoint{1.874108in}{0.749138in}}%
\pgfpathlineto{\pgfqpoint{1.874902in}{0.830537in}}%
\pgfpathlineto{\pgfqpoint{1.875101in}{0.759313in}}%
\pgfpathlineto{\pgfqpoint{1.875299in}{0.759313in}}%
\pgfpathlineto{\pgfqpoint{1.875895in}{0.861062in}}%
\pgfpathlineto{\pgfqpoint{1.876093in}{0.749138in}}%
\pgfpathlineto{\pgfqpoint{1.876292in}{0.810187in}}%
\pgfpathlineto{\pgfqpoint{1.876490in}{0.810187in}}%
\pgfpathlineto{\pgfqpoint{1.876887in}{0.769488in}}%
\pgfpathlineto{\pgfqpoint{1.877483in}{0.789837in}}%
\pgfpathlineto{\pgfqpoint{1.877681in}{0.789837in}}%
\pgfpathlineto{\pgfqpoint{1.878078in}{0.769488in}}%
\pgfpathlineto{\pgfqpoint{1.878673in}{0.789837in}}%
\pgfpathlineto{\pgfqpoint{1.879070in}{0.789837in}}%
\pgfpathlineto{\pgfqpoint{1.879666in}{0.820362in}}%
\pgfpathlineto{\pgfqpoint{1.880063in}{0.759313in}}%
\pgfpathlineto{\pgfqpoint{1.880261in}{0.759313in}}%
\pgfpathlineto{\pgfqpoint{1.880460in}{0.850887in}}%
\pgfpathlineto{\pgfqpoint{1.880857in}{0.749138in}}%
\pgfpathlineto{\pgfqpoint{1.881254in}{0.800012in}}%
\pgfpathlineto{\pgfqpoint{1.881651in}{0.800012in}}%
\pgfpathlineto{\pgfqpoint{1.882445in}{0.830537in}}%
\pgfpathlineto{\pgfqpoint{1.882643in}{0.769488in}}%
\pgfpathlineto{\pgfqpoint{1.883040in}{0.769488in}}%
\pgfpathlineto{\pgfqpoint{1.883040in}{0.810187in}}%
\pgfpathlineto{\pgfqpoint{1.884033in}{0.800012in}}%
\pgfpathlineto{\pgfqpoint{1.884231in}{0.800012in}}%
\pgfpathlineto{\pgfqpoint{1.884231in}{0.749138in}}%
\pgfpathlineto{\pgfqpoint{1.885224in}{0.779662in}}%
\pgfpathlineto{\pgfqpoint{1.885422in}{0.779662in}}%
\pgfpathlineto{\pgfqpoint{1.885422in}{0.769488in}}%
\pgfpathlineto{\pgfqpoint{1.885819in}{0.800012in}}%
\pgfpathlineto{\pgfqpoint{1.886415in}{0.800012in}}%
\pgfpathlineto{\pgfqpoint{1.886613in}{0.800012in}}%
\pgfpathlineto{\pgfqpoint{1.887606in}{0.738963in}}%
\pgfpathlineto{\pgfqpoint{1.887804in}{0.738963in}}%
\pgfpathlineto{\pgfqpoint{1.888201in}{0.820362in}}%
\pgfpathlineto{\pgfqpoint{1.888797in}{0.759313in}}%
\pgfpathlineto{\pgfqpoint{1.888995in}{0.759313in}}%
\pgfpathlineto{\pgfqpoint{1.888995in}{0.800012in}}%
\pgfpathlineto{\pgfqpoint{1.889988in}{0.779662in}}%
\pgfpathlineto{\pgfqpoint{1.890186in}{0.779662in}}%
\pgfpathlineto{\pgfqpoint{1.890782in}{0.810187in}}%
\pgfpathlineto{\pgfqpoint{1.890980in}{0.749138in}}%
\pgfpathlineto{\pgfqpoint{1.891179in}{0.779662in}}%
\pgfpathlineto{\pgfqpoint{1.891377in}{0.779662in}}%
\pgfpathlineto{\pgfqpoint{1.891576in}{0.769488in}}%
\pgfpathlineto{\pgfqpoint{1.892370in}{0.810187in}}%
\pgfpathlineto{\pgfqpoint{1.892568in}{0.810187in}}%
\pgfpathlineto{\pgfqpoint{1.892568in}{0.738963in}}%
\pgfpathlineto{\pgfqpoint{1.893561in}{0.759313in}}%
\pgfpathlineto{\pgfqpoint{1.893759in}{0.759313in}}%
\pgfpathlineto{\pgfqpoint{1.893958in}{0.820362in}}%
\pgfpathlineto{\pgfqpoint{1.894156in}{0.738963in}}%
\pgfpathlineto{\pgfqpoint{1.894752in}{0.779662in}}%
\pgfpathlineto{\pgfqpoint{1.894950in}{0.779662in}}%
\pgfpathlineto{\pgfqpoint{1.894950in}{0.830537in}}%
\pgfpathlineto{\pgfqpoint{1.895943in}{0.749138in}}%
\pgfpathlineto{\pgfqpoint{1.896141in}{0.749138in}}%
\pgfpathlineto{\pgfqpoint{1.896141in}{0.800012in}}%
\pgfpathlineto{\pgfqpoint{1.897134in}{0.789837in}}%
\pgfpathlineto{\pgfqpoint{1.897332in}{0.789837in}}%
\pgfpathlineto{\pgfqpoint{1.897531in}{0.749138in}}%
\pgfpathlineto{\pgfqpoint{1.898325in}{0.779662in}}%
\pgfpathlineto{\pgfqpoint{1.898523in}{0.779662in}}%
\pgfpathlineto{\pgfqpoint{1.898523in}{0.769488in}}%
\pgfpathlineto{\pgfqpoint{1.899119in}{0.800012in}}%
\pgfpathlineto{\pgfqpoint{1.899516in}{0.769488in}}%
\pgfpathlineto{\pgfqpoint{1.899714in}{0.769488in}}%
\pgfpathlineto{\pgfqpoint{1.899913in}{0.800012in}}%
\pgfpathlineto{\pgfqpoint{1.900111in}{0.749138in}}%
\pgfpathlineto{\pgfqpoint{1.900707in}{0.759313in}}%
\pgfpathlineto{\pgfqpoint{1.900905in}{0.759313in}}%
\pgfpathlineto{\pgfqpoint{1.901501in}{0.830537in}}%
\pgfpathlineto{\pgfqpoint{1.901104in}{0.749138in}}%
\pgfpathlineto{\pgfqpoint{1.901898in}{0.749138in}}%
\pgfpathlineto{\pgfqpoint{1.902096in}{0.749138in}}%
\pgfpathlineto{\pgfqpoint{1.902493in}{0.820362in}}%
\pgfpathlineto{\pgfqpoint{1.903089in}{0.759313in}}%
\pgfpathlineto{\pgfqpoint{1.903486in}{0.759313in}}%
\pgfpathlineto{\pgfqpoint{1.903486in}{0.800012in}}%
\pgfpathlineto{\pgfqpoint{1.904081in}{0.749138in}}%
\pgfpathlineto{\pgfqpoint{1.904478in}{0.759313in}}%
\pgfpathlineto{\pgfqpoint{1.904677in}{0.759313in}}%
\pgfpathlineto{\pgfqpoint{1.904677in}{0.728788in}}%
\pgfpathlineto{\pgfqpoint{1.904875in}{0.789837in}}%
\pgfpathlineto{\pgfqpoint{1.905669in}{0.759313in}}%
\pgfpathlineto{\pgfqpoint{1.905868in}{0.759313in}}%
\pgfpathlineto{\pgfqpoint{1.905868in}{0.779662in}}%
\pgfpathlineto{\pgfqpoint{1.906860in}{0.738963in}}%
\pgfpathlineto{\pgfqpoint{1.907059in}{0.738963in}}%
\pgfpathlineto{\pgfqpoint{1.907257in}{0.789837in}}%
\pgfpathlineto{\pgfqpoint{1.908051in}{0.759313in}}%
\pgfpathlineto{\pgfqpoint{1.908250in}{0.759313in}}%
\pgfpathlineto{\pgfqpoint{1.908250in}{0.810187in}}%
\pgfpathlineto{\pgfqpoint{1.909044in}{0.738963in}}%
\pgfpathlineto{\pgfqpoint{1.909242in}{0.769488in}}%
\pgfpathlineto{\pgfqpoint{1.909441in}{0.769488in}}%
\pgfpathlineto{\pgfqpoint{1.909441in}{0.738963in}}%
\pgfpathlineto{\pgfqpoint{1.909639in}{0.800012in}}%
\pgfpathlineto{\pgfqpoint{1.910433in}{0.779662in}}%
\pgfpathlineto{\pgfqpoint{1.910631in}{0.779662in}}%
\pgfpathlineto{\pgfqpoint{1.910830in}{0.738963in}}%
\pgfpathlineto{\pgfqpoint{1.911227in}{0.800012in}}%
\pgfpathlineto{\pgfqpoint{1.911624in}{0.779662in}}%
\pgfpathlineto{\pgfqpoint{1.911822in}{0.779662in}}%
\pgfpathlineto{\pgfqpoint{1.912616in}{0.749138in}}%
\pgfpathlineto{\pgfqpoint{1.912815in}{0.800012in}}%
\pgfpathlineto{\pgfqpoint{1.913013in}{0.800012in}}%
\pgfpathlineto{\pgfqpoint{1.913013in}{0.749138in}}%
\pgfpathlineto{\pgfqpoint{1.914006in}{0.759313in}}%
\pgfpathlineto{\pgfqpoint{1.914204in}{0.759313in}}%
\pgfpathlineto{\pgfqpoint{1.914204in}{0.789837in}}%
\pgfpathlineto{\pgfqpoint{1.915197in}{0.759313in}}%
\pgfpathlineto{\pgfqpoint{1.915395in}{0.759313in}}%
\pgfpathlineto{\pgfqpoint{1.915792in}{0.810187in}}%
\pgfpathlineto{\pgfqpoint{1.916388in}{0.738963in}}%
\pgfpathlineto{\pgfqpoint{1.916586in}{0.738963in}}%
\pgfpathlineto{\pgfqpoint{1.917380in}{0.789837in}}%
\pgfpathlineto{\pgfqpoint{1.917579in}{0.738963in}}%
\pgfpathlineto{\pgfqpoint{1.917777in}{0.738963in}}%
\pgfpathlineto{\pgfqpoint{1.918174in}{0.779662in}}%
\pgfpathlineto{\pgfqpoint{1.918770in}{0.759313in}}%
\pgfpathlineto{\pgfqpoint{1.918968in}{0.759313in}}%
\pgfpathlineto{\pgfqpoint{1.918968in}{0.749138in}}%
\pgfpathlineto{\pgfqpoint{1.919167in}{0.800012in}}%
\pgfpathlineto{\pgfqpoint{1.919961in}{0.759313in}}%
\pgfpathlineto{\pgfqpoint{1.920556in}{0.759313in}}%
\pgfpathlineto{\pgfqpoint{1.920953in}{0.800012in}}%
\pgfpathlineto{\pgfqpoint{1.921350in}{0.738963in}}%
\pgfpathlineto{\pgfqpoint{1.921549in}{0.779662in}}%
\pgfpathlineto{\pgfqpoint{1.921747in}{0.779662in}}%
\pgfpathlineto{\pgfqpoint{1.921946in}{0.738963in}}%
\pgfpathlineto{\pgfqpoint{1.922740in}{0.749138in}}%
\pgfpathlineto{\pgfqpoint{1.923137in}{0.749138in}}%
\pgfpathlineto{\pgfqpoint{1.923137in}{0.738963in}}%
\pgfpathlineto{\pgfqpoint{1.923534in}{0.779662in}}%
\pgfpathlineto{\pgfqpoint{1.924129in}{0.769488in}}%
\pgfpathlineto{\pgfqpoint{1.924328in}{0.769488in}}%
\pgfpathlineto{\pgfqpoint{1.924328in}{0.728788in}}%
\pgfpathlineto{\pgfqpoint{1.925320in}{0.800012in}}%
\pgfpathlineto{\pgfqpoint{1.925519in}{0.800012in}}%
\pgfpathlineto{\pgfqpoint{1.926313in}{0.728788in}}%
\pgfpathlineto{\pgfqpoint{1.926511in}{0.779662in}}%
\pgfpathlineto{\pgfqpoint{1.926710in}{0.779662in}}%
\pgfpathlineto{\pgfqpoint{1.926710in}{0.789837in}}%
\pgfpathlineto{\pgfqpoint{1.927702in}{0.738963in}}%
\pgfpathlineto{\pgfqpoint{1.927901in}{0.738963in}}%
\pgfpathlineto{\pgfqpoint{1.927901in}{0.779662in}}%
\pgfpathlineto{\pgfqpoint{1.928695in}{0.728788in}}%
\pgfpathlineto{\pgfqpoint{1.928893in}{0.749138in}}%
\pgfpathlineto{\pgfqpoint{1.929092in}{0.749138in}}%
\pgfpathlineto{\pgfqpoint{1.929092in}{0.738963in}}%
\pgfpathlineto{\pgfqpoint{1.929290in}{0.779662in}}%
\pgfpathlineto{\pgfqpoint{1.930084in}{0.759313in}}%
\pgfpathlineto{\pgfqpoint{1.930283in}{0.759313in}}%
\pgfpathlineto{\pgfqpoint{1.930283in}{0.800012in}}%
\pgfpathlineto{\pgfqpoint{1.930481in}{0.749138in}}%
\pgfpathlineto{\pgfqpoint{1.931275in}{0.749138in}}%
\pgfpathlineto{\pgfqpoint{1.931474in}{0.749138in}}%
\pgfpathlineto{\pgfqpoint{1.932069in}{0.800012in}}%
\pgfpathlineto{\pgfqpoint{1.931672in}{0.728788in}}%
\pgfpathlineto{\pgfqpoint{1.932466in}{0.738963in}}%
\pgfpathlineto{\pgfqpoint{1.932665in}{0.738963in}}%
\pgfpathlineto{\pgfqpoint{1.932665in}{0.800012in}}%
\pgfpathlineto{\pgfqpoint{1.933657in}{0.728788in}}%
\pgfpathlineto{\pgfqpoint{1.933856in}{0.728788in}}%
\pgfpathlineto{\pgfqpoint{1.934253in}{0.779662in}}%
\pgfpathlineto{\pgfqpoint{1.934848in}{0.749138in}}%
\pgfpathlineto{\pgfqpoint{1.935245in}{0.749138in}}%
\pgfpathlineto{\pgfqpoint{1.935642in}{0.779662in}}%
\pgfpathlineto{\pgfqpoint{1.935444in}{0.738963in}}%
\pgfpathlineto{\pgfqpoint{1.936238in}{0.759313in}}%
\pgfpathlineto{\pgfqpoint{1.936635in}{0.759313in}}%
\pgfpathlineto{\pgfqpoint{1.936635in}{0.779662in}}%
\pgfpathlineto{\pgfqpoint{1.937627in}{0.728788in}}%
\pgfpathlineto{\pgfqpoint{1.937826in}{0.728788in}}%
\pgfpathlineto{\pgfqpoint{1.938620in}{0.779662in}}%
\pgfpathlineto{\pgfqpoint{1.938818in}{0.738963in}}%
\pgfpathlineto{\pgfqpoint{1.939017in}{0.738963in}}%
\pgfpathlineto{\pgfqpoint{1.939215in}{0.800012in}}%
\pgfpathlineto{\pgfqpoint{1.940009in}{0.749138in}}%
\pgfpathlineto{\pgfqpoint{1.940208in}{0.749138in}}%
\pgfpathlineto{\pgfqpoint{1.940208in}{0.769488in}}%
\pgfpathlineto{\pgfqpoint{1.940605in}{0.728788in}}%
\pgfpathlineto{\pgfqpoint{1.941200in}{0.759313in}}%
\pgfpathlineto{\pgfqpoint{1.941399in}{0.759313in}}%
\pgfpathlineto{\pgfqpoint{1.941399in}{0.738963in}}%
\pgfpathlineto{\pgfqpoint{1.942391in}{0.738963in}}%
\pgfpathlineto{\pgfqpoint{1.942788in}{0.738963in}}%
\pgfpathlineto{\pgfqpoint{1.942788in}{0.779662in}}%
\pgfpathlineto{\pgfqpoint{1.943780in}{0.749138in}}%
\pgfpathlineto{\pgfqpoint{1.944376in}{0.749138in}}%
\pgfpathlineto{\pgfqpoint{1.944376in}{0.779662in}}%
\pgfpathlineto{\pgfqpoint{1.944574in}{0.738963in}}%
\pgfpathlineto{\pgfqpoint{1.945368in}{0.759313in}}%
\pgfpathlineto{\pgfqpoint{1.945567in}{0.759313in}}%
\pgfpathlineto{\pgfqpoint{1.946162in}{0.738963in}}%
\pgfpathlineto{\pgfqpoint{1.946361in}{0.769488in}}%
\pgfpathlineto{\pgfqpoint{1.946559in}{0.759313in}}%
\pgfpathlineto{\pgfqpoint{1.946758in}{0.759313in}}%
\pgfpathlineto{\pgfqpoint{1.946758in}{0.779662in}}%
\pgfpathlineto{\pgfqpoint{1.947552in}{0.728788in}}%
\pgfpathlineto{\pgfqpoint{1.947750in}{0.779662in}}%
\pgfpathlineto{\pgfqpoint{1.947949in}{0.779662in}}%
\pgfpathlineto{\pgfqpoint{1.948544in}{0.738963in}}%
\pgfpathlineto{\pgfqpoint{1.948941in}{0.789837in}}%
\pgfpathlineto{\pgfqpoint{1.949140in}{0.789837in}}%
\pgfpathlineto{\pgfqpoint{1.949140in}{0.738963in}}%
\pgfpathlineto{\pgfqpoint{1.950132in}{0.779662in}}%
\pgfpathlineto{\pgfqpoint{1.950331in}{0.779662in}}%
\pgfpathlineto{\pgfqpoint{1.951323in}{0.728788in}}%
\pgfpathlineto{\pgfqpoint{1.951522in}{0.728788in}}%
\pgfpathlineto{\pgfqpoint{1.951522in}{0.779662in}}%
\pgfpathlineto{\pgfqpoint{1.952514in}{0.749138in}}%
\pgfpathlineto{\pgfqpoint{1.952911in}{0.749138in}}%
\pgfpathlineto{\pgfqpoint{1.953705in}{0.789837in}}%
\pgfpathlineto{\pgfqpoint{1.953904in}{0.738963in}}%
\pgfpathlineto{\pgfqpoint{1.954301in}{0.738963in}}%
\pgfpathlineto{\pgfqpoint{1.954301in}{0.769488in}}%
\pgfpathlineto{\pgfqpoint{1.954698in}{0.728788in}}%
\pgfpathlineto{\pgfqpoint{1.955293in}{0.738963in}}%
\pgfpathlineto{\pgfqpoint{1.955492in}{0.738963in}}%
\pgfpathlineto{\pgfqpoint{1.956087in}{0.789837in}}%
\pgfpathlineto{\pgfqpoint{1.956484in}{0.738963in}}%
\pgfpathlineto{\pgfqpoint{1.956683in}{0.738963in}}%
\pgfpathlineto{\pgfqpoint{1.956683in}{0.728788in}}%
\pgfpathlineto{\pgfqpoint{1.956881in}{0.769488in}}%
\pgfpathlineto{\pgfqpoint{1.957675in}{0.738963in}}%
\pgfpathlineto{\pgfqpoint{1.957874in}{0.738963in}}%
\pgfpathlineto{\pgfqpoint{1.958271in}{0.779662in}}%
\pgfpathlineto{\pgfqpoint{1.958866in}{0.759313in}}%
\pgfpathlineto{\pgfqpoint{1.959065in}{0.759313in}}%
\pgfpathlineto{\pgfqpoint{1.959660in}{0.728788in}}%
\pgfpathlineto{\pgfqpoint{1.959859in}{0.769488in}}%
\pgfpathlineto{\pgfqpoint{1.960057in}{0.759313in}}%
\pgfpathlineto{\pgfqpoint{1.960454in}{0.759313in}}%
\pgfpathlineto{\pgfqpoint{1.960454in}{0.738963in}}%
\pgfpathlineto{\pgfqpoint{1.960653in}{0.769488in}}%
\pgfpathlineto{\pgfqpoint{1.961447in}{0.749138in}}%
\pgfpathlineto{\pgfqpoint{1.961645in}{0.749138in}}%
\pgfpathlineto{\pgfqpoint{1.961645in}{0.728788in}}%
\pgfpathlineto{\pgfqpoint{1.961844in}{0.779662in}}%
\pgfpathlineto{\pgfqpoint{1.962638in}{0.779662in}}%
\pgfpathlineto{\pgfqpoint{1.962836in}{0.779662in}}%
\pgfpathlineto{\pgfqpoint{1.962836in}{0.738963in}}%
\pgfpathlineto{\pgfqpoint{1.963829in}{0.738963in}}%
\pgfpathlineto{\pgfqpoint{1.964027in}{0.738963in}}%
\pgfpathlineto{\pgfqpoint{1.964821in}{0.779662in}}%
\pgfpathlineto{\pgfqpoint{1.965020in}{0.759313in}}%
\pgfpathlineto{\pgfqpoint{1.965218in}{0.759313in}}%
\pgfpathlineto{\pgfqpoint{1.965218in}{0.728788in}}%
\pgfpathlineto{\pgfqpoint{1.966211in}{0.728788in}}%
\pgfpathlineto{\pgfqpoint{1.966409in}{0.728788in}}%
\pgfpathlineto{\pgfqpoint{1.967005in}{0.769488in}}%
\pgfpathlineto{\pgfqpoint{1.967402in}{0.738963in}}%
\pgfpathlineto{\pgfqpoint{1.967600in}{0.738963in}}%
\pgfpathlineto{\pgfqpoint{1.968196in}{0.769488in}}%
\pgfpathlineto{\pgfqpoint{1.968593in}{0.738963in}}%
\pgfpathlineto{\pgfqpoint{1.968791in}{0.738963in}}%
\pgfpathlineto{\pgfqpoint{1.969188in}{0.779662in}}%
\pgfpathlineto{\pgfqpoint{1.969784in}{0.779662in}}%
\pgfpathlineto{\pgfqpoint{1.969982in}{0.779662in}}%
\pgfpathlineto{\pgfqpoint{1.970181in}{0.728788in}}%
\pgfpathlineto{\pgfqpoint{1.970975in}{0.738963in}}%
\pgfpathlineto{\pgfqpoint{1.971173in}{0.738963in}}%
\pgfpathlineto{\pgfqpoint{1.971173in}{0.769488in}}%
\pgfpathlineto{\pgfqpoint{1.972166in}{0.749138in}}%
\pgfpathlineto{\pgfqpoint{1.972364in}{0.749138in}}%
\pgfpathlineto{\pgfqpoint{1.972761in}{0.728788in}}%
\pgfpathlineto{\pgfqpoint{1.972563in}{0.779662in}}%
\pgfpathlineto{\pgfqpoint{1.973357in}{0.779662in}}%
\pgfpathlineto{\pgfqpoint{1.973555in}{0.779662in}}%
\pgfpathlineto{\pgfqpoint{1.974547in}{0.728788in}}%
\pgfpathlineto{\pgfqpoint{1.974746in}{0.728788in}}%
\pgfpathlineto{\pgfqpoint{1.974746in}{0.789837in}}%
\pgfpathlineto{\pgfqpoint{1.975738in}{0.769488in}}%
\pgfpathlineto{\pgfqpoint{1.975937in}{0.769488in}}%
\pgfpathlineto{\pgfqpoint{1.975937in}{0.779662in}}%
\pgfpathlineto{\pgfqpoint{1.976135in}{0.728788in}}%
\pgfpathlineto{\pgfqpoint{1.976929in}{0.759313in}}%
\pgfpathlineto{\pgfqpoint{1.977128in}{0.759313in}}%
\pgfpathlineto{\pgfqpoint{1.977128in}{0.749138in}}%
\pgfpathlineto{\pgfqpoint{1.977525in}{0.769488in}}%
\pgfpathlineto{\pgfqpoint{1.978120in}{0.769488in}}%
\pgfpathlineto{\pgfqpoint{1.978319in}{0.769488in}}%
\pgfpathlineto{\pgfqpoint{1.978517in}{0.728788in}}%
\pgfpathlineto{\pgfqpoint{1.979311in}{0.759313in}}%
\pgfpathlineto{\pgfqpoint{1.979708in}{0.759313in}}%
\pgfpathlineto{\pgfqpoint{1.980105in}{0.779662in}}%
\pgfpathlineto{\pgfqpoint{1.979907in}{0.728788in}}%
\pgfpathlineto{\pgfqpoint{1.980701in}{0.738963in}}%
\pgfpathlineto{\pgfqpoint{1.980899in}{0.738963in}}%
\pgfpathlineto{\pgfqpoint{1.980899in}{0.749138in}}%
\pgfpathlineto{\pgfqpoint{1.981892in}{0.738963in}}%
\pgfpathlineto{\pgfqpoint{1.982090in}{0.738963in}}%
\pgfpathlineto{\pgfqpoint{1.982090in}{0.728788in}}%
\pgfpathlineto{\pgfqpoint{1.983083in}{0.789837in}}%
\pgfpathlineto{\pgfqpoint{1.983281in}{0.789837in}}%
\pgfpathlineto{\pgfqpoint{1.983281in}{0.738963in}}%
\pgfpathlineto{\pgfqpoint{1.984274in}{0.738963in}}%
\pgfpathlineto{\pgfqpoint{1.984472in}{0.738963in}}%
\pgfpathlineto{\pgfqpoint{1.984472in}{0.759313in}}%
\pgfpathlineto{\pgfqpoint{1.984869in}{0.728788in}}%
\pgfpathlineto{\pgfqpoint{1.985465in}{0.738963in}}%
\pgfpathlineto{\pgfqpoint{1.985663in}{0.738963in}}%
\pgfpathlineto{\pgfqpoint{1.986457in}{0.759313in}}%
\pgfpathlineto{\pgfqpoint{1.986656in}{0.728788in}}%
\pgfpathlineto{\pgfqpoint{1.986854in}{0.728788in}}%
\pgfpathlineto{\pgfqpoint{1.986854in}{0.779662in}}%
\pgfpathlineto{\pgfqpoint{1.987847in}{0.738963in}}%
\pgfpathlineto{\pgfqpoint{1.988244in}{0.738963in}}%
\pgfpathlineto{\pgfqpoint{1.988839in}{0.728788in}}%
\pgfpathlineto{\pgfqpoint{1.989236in}{0.800012in}}%
\pgfpathlineto{\pgfqpoint{1.989435in}{0.800012in}}%
\pgfpathlineto{\pgfqpoint{1.990427in}{0.728788in}}%
\pgfpathlineto{\pgfqpoint{1.990626in}{0.728788in}}%
\pgfpathlineto{\pgfqpoint{1.991618in}{0.779662in}}%
\pgfpathlineto{\pgfqpoint{1.991817in}{0.779662in}}%
\pgfpathlineto{\pgfqpoint{1.991817in}{0.789837in}}%
\pgfpathlineto{\pgfqpoint{1.992611in}{0.738963in}}%
\pgfpathlineto{\pgfqpoint{1.992809in}{0.749138in}}%
\pgfpathlineto{\pgfqpoint{1.993008in}{0.749138in}}%
\pgfpathlineto{\pgfqpoint{1.993405in}{0.728788in}}%
\pgfpathlineto{\pgfqpoint{1.993802in}{0.769488in}}%
\pgfpathlineto{\pgfqpoint{1.994000in}{0.738963in}}%
\pgfpathlineto{\pgfqpoint{1.994199in}{0.738963in}}%
\pgfpathlineto{\pgfqpoint{1.994199in}{0.749138in}}%
\pgfpathlineto{\pgfqpoint{1.994993in}{0.728788in}}%
\pgfpathlineto{\pgfqpoint{1.995191in}{0.749138in}}%
\pgfpathlineto{\pgfqpoint{1.995390in}{0.749138in}}%
\pgfpathlineto{\pgfqpoint{1.995390in}{0.738963in}}%
\pgfpathlineto{\pgfqpoint{1.995588in}{0.769488in}}%
\pgfpathlineto{\pgfqpoint{1.996382in}{0.738963in}}%
\pgfpathlineto{\pgfqpoint{1.996581in}{0.738963in}}%
\pgfpathlineto{\pgfqpoint{1.996779in}{0.759313in}}%
\pgfpathlineto{\pgfqpoint{1.997573in}{0.738963in}}%
\pgfpathlineto{\pgfqpoint{1.997772in}{0.738963in}}%
\pgfpathlineto{\pgfqpoint{1.997772in}{0.759313in}}%
\pgfpathlineto{\pgfqpoint{1.998764in}{0.738963in}}%
\pgfpathlineto{\pgfqpoint{1.999558in}{0.738963in}}%
\pgfpathlineto{\pgfqpoint{1.999558in}{0.749138in}}%
\pgfpathlineto{\pgfqpoint{2.000551in}{0.738963in}}%
\pgfpathlineto{\pgfqpoint{2.000749in}{0.738963in}}%
\pgfpathlineto{\pgfqpoint{2.001146in}{0.769488in}}%
\pgfpathlineto{\pgfqpoint{2.000948in}{0.728788in}}%
\pgfpathlineto{\pgfqpoint{2.001742in}{0.749138in}}%
\pgfpathlineto{\pgfqpoint{2.002536in}{0.749138in}}%
\pgfpathlineto{\pgfqpoint{2.002536in}{0.738963in}}%
\pgfpathlineto{\pgfqpoint{2.003330in}{0.769488in}}%
\pgfpathlineto{\pgfqpoint{2.003528in}{0.759313in}}%
\pgfpathlineto{\pgfqpoint{2.003727in}{0.759313in}}%
\pgfpathlineto{\pgfqpoint{2.003727in}{0.738963in}}%
\pgfpathlineto{\pgfqpoint{2.004322in}{0.769488in}}%
\pgfpathlineto{\pgfqpoint{2.004719in}{0.738963in}}%
\pgfpathlineto{\pgfqpoint{2.004918in}{0.738963in}}%
\pgfpathlineto{\pgfqpoint{2.004918in}{0.769488in}}%
\pgfpathlineto{\pgfqpoint{2.005513in}{0.728788in}}%
\pgfpathlineto{\pgfqpoint{2.005910in}{0.769488in}}%
\pgfpathlineto{\pgfqpoint{2.006109in}{0.769488in}}%
\pgfpathlineto{\pgfqpoint{2.006109in}{0.738963in}}%
\pgfpathlineto{\pgfqpoint{2.007101in}{0.738963in}}%
\pgfpathlineto{\pgfqpoint{2.007498in}{0.738963in}}%
\pgfpathlineto{\pgfqpoint{2.007498in}{0.769488in}}%
\pgfpathlineto{\pgfqpoint{2.008093in}{0.728788in}}%
\pgfpathlineto{\pgfqpoint{2.008490in}{0.759313in}}%
\pgfpathlineto{\pgfqpoint{2.008689in}{0.759313in}}%
\pgfpathlineto{\pgfqpoint{2.008689in}{0.769488in}}%
\pgfpathlineto{\pgfqpoint{2.008887in}{0.728788in}}%
\pgfpathlineto{\pgfqpoint{2.009681in}{0.728788in}}%
\pgfpathlineto{\pgfqpoint{2.009880in}{0.728788in}}%
\pgfpathlineto{\pgfqpoint{2.010475in}{0.769488in}}%
\pgfpathlineto{\pgfqpoint{2.010872in}{0.738963in}}%
\pgfpathlineto{\pgfqpoint{2.011071in}{0.738963in}}%
\pgfpathlineto{\pgfqpoint{2.011071in}{0.728788in}}%
\pgfpathlineto{\pgfqpoint{2.012063in}{0.769488in}}%
\pgfpathlineto{\pgfqpoint{2.012262in}{0.769488in}}%
\pgfpathlineto{\pgfqpoint{2.012460in}{0.728788in}}%
\pgfpathlineto{\pgfqpoint{2.013254in}{0.728788in}}%
\pgfpathlineto{\pgfqpoint{2.013453in}{0.728788in}}%
\pgfpathlineto{\pgfqpoint{2.013651in}{0.779662in}}%
\pgfpathlineto{\pgfqpoint{2.014445in}{0.728788in}}%
\pgfpathlineto{\pgfqpoint{2.014644in}{0.728788in}}%
\pgfpathlineto{\pgfqpoint{2.015239in}{0.769488in}}%
\pgfpathlineto{\pgfqpoint{2.015636in}{0.738963in}}%
\pgfpathlineto{\pgfqpoint{2.016033in}{0.738963in}}%
\pgfpathlineto{\pgfqpoint{2.016033in}{0.769488in}}%
\pgfpathlineto{\pgfqpoint{2.017026in}{0.749138in}}%
\pgfpathlineto{\pgfqpoint{2.017224in}{0.749138in}}%
\pgfpathlineto{\pgfqpoint{2.018018in}{0.769488in}}%
\pgfpathlineto{\pgfqpoint{2.017621in}{0.728788in}}%
\pgfpathlineto{\pgfqpoint{2.018217in}{0.759313in}}%
\pgfpathlineto{\pgfqpoint{2.018614in}{0.759313in}}%
\pgfpathlineto{\pgfqpoint{2.018614in}{0.738963in}}%
\pgfpathlineto{\pgfqpoint{2.019209in}{0.769488in}}%
\pgfpathlineto{\pgfqpoint{2.019606in}{0.749138in}}%
\pgfpathlineto{\pgfqpoint{2.019805in}{0.749138in}}%
\pgfpathlineto{\pgfqpoint{2.020003in}{0.759313in}}%
\pgfpathlineto{\pgfqpoint{2.020797in}{0.728788in}}%
\pgfpathlineto{\pgfqpoint{2.021194in}{0.728788in}}%
\pgfpathlineto{\pgfqpoint{2.021988in}{0.759313in}}%
\pgfpathlineto{\pgfqpoint{2.022187in}{0.759313in}}%
\pgfpathlineto{\pgfqpoint{2.022584in}{0.759313in}}%
\pgfpathlineto{\pgfqpoint{2.022584in}{0.728788in}}%
\pgfpathlineto{\pgfqpoint{2.022782in}{0.769488in}}%
\pgfpathlineto{\pgfqpoint{2.023576in}{0.738963in}}%
\pgfpathlineto{\pgfqpoint{2.023973in}{0.738963in}}%
\pgfpathlineto{\pgfqpoint{2.023973in}{0.749138in}}%
\pgfpathlineto{\pgfqpoint{2.024569in}{0.728788in}}%
\pgfpathlineto{\pgfqpoint{2.024966in}{0.749138in}}%
\pgfpathlineto{\pgfqpoint{2.025164in}{0.749138in}}%
\pgfpathlineto{\pgfqpoint{2.025760in}{0.728788in}}%
\pgfpathlineto{\pgfqpoint{2.025561in}{0.759313in}}%
\pgfpathlineto{\pgfqpoint{2.026157in}{0.749138in}}%
\pgfpathlineto{\pgfqpoint{2.026355in}{0.749138in}}%
\pgfpathlineto{\pgfqpoint{2.026752in}{0.728788in}}%
\pgfpathlineto{\pgfqpoint{2.027348in}{0.728788in}}%
\pgfpathlineto{\pgfqpoint{2.027745in}{0.728788in}}%
\pgfpathlineto{\pgfqpoint{2.027745in}{0.769488in}}%
\pgfpathlineto{\pgfqpoint{2.028737in}{0.759313in}}%
\pgfpathlineto{\pgfqpoint{2.028936in}{0.759313in}}%
\pgfpathlineto{\pgfqpoint{2.028936in}{0.779662in}}%
\pgfpathlineto{\pgfqpoint{2.029134in}{0.728788in}}%
\pgfpathlineto{\pgfqpoint{2.029928in}{0.738963in}}%
\pgfpathlineto{\pgfqpoint{2.030127in}{0.738963in}}%
\pgfpathlineto{\pgfqpoint{2.030127in}{0.728788in}}%
\pgfpathlineto{\pgfqpoint{2.031119in}{0.769488in}}%
\pgfpathlineto{\pgfqpoint{2.031318in}{0.769488in}}%
\pgfpathlineto{\pgfqpoint{2.031318in}{0.728788in}}%
\pgfpathlineto{\pgfqpoint{2.032310in}{0.779662in}}%
\pgfpathlineto{\pgfqpoint{2.032509in}{0.779662in}}%
\pgfpathlineto{\pgfqpoint{2.033104in}{0.728788in}}%
\pgfpathlineto{\pgfqpoint{2.033501in}{0.728788in}}%
\pgfpathlineto{\pgfqpoint{2.033700in}{0.728788in}}%
\pgfpathlineto{\pgfqpoint{2.034295in}{0.759313in}}%
\pgfpathlineto{\pgfqpoint{2.034692in}{0.749138in}}%
\pgfpathlineto{\pgfqpoint{2.035089in}{0.749138in}}%
\pgfpathlineto{\pgfqpoint{2.035685in}{0.728788in}}%
\pgfpathlineto{\pgfqpoint{2.035288in}{0.759313in}}%
\pgfpathlineto{\pgfqpoint{2.036082in}{0.728788in}}%
\pgfpathlineto{\pgfqpoint{2.036280in}{0.728788in}}%
\pgfpathlineto{\pgfqpoint{2.036479in}{0.769488in}}%
\pgfpathlineto{\pgfqpoint{2.037273in}{0.759313in}}%
\pgfpathlineto{\pgfqpoint{2.037471in}{0.759313in}}%
\pgfpathlineto{\pgfqpoint{2.037471in}{0.728788in}}%
\pgfpathlineto{\pgfqpoint{2.038464in}{0.738963in}}%
\pgfpathlineto{\pgfqpoint{2.038662in}{0.738963in}}%
\pgfpathlineto{\pgfqpoint{2.039456in}{0.779662in}}%
\pgfpathlineto{\pgfqpoint{2.039059in}{0.728788in}}%
\pgfpathlineto{\pgfqpoint{2.039654in}{0.728788in}}%
\pgfpathlineto{\pgfqpoint{2.039853in}{0.728788in}}%
\pgfpathlineto{\pgfqpoint{2.039853in}{0.759313in}}%
\pgfpathlineto{\pgfqpoint{2.040845in}{0.728788in}}%
\pgfpathlineto{\pgfqpoint{2.041044in}{0.728788in}}%
\pgfpathlineto{\pgfqpoint{2.041044in}{0.759313in}}%
\pgfpathlineto{\pgfqpoint{2.042036in}{0.749138in}}%
\pgfpathlineto{\pgfqpoint{2.042830in}{0.749138in}}%
\pgfpathlineto{\pgfqpoint{2.042830in}{0.728788in}}%
\pgfpathlineto{\pgfqpoint{2.043426in}{0.759313in}}%
\pgfpathlineto{\pgfqpoint{2.043823in}{0.738963in}}%
\pgfpathlineto{\pgfqpoint{2.044418in}{0.738963in}}%
\pgfpathlineto{\pgfqpoint{2.044617in}{0.759313in}}%
\pgfpathlineto{\pgfqpoint{2.045411in}{0.738963in}}%
\pgfpathlineto{\pgfqpoint{2.045609in}{0.738963in}}%
\pgfpathlineto{\pgfqpoint{2.045609in}{0.728788in}}%
\pgfpathlineto{\pgfqpoint{2.046205in}{0.759313in}}%
\pgfpathlineto{\pgfqpoint{2.046602in}{0.749138in}}%
\pgfpathlineto{\pgfqpoint{2.046800in}{0.749138in}}%
\pgfpathlineto{\pgfqpoint{2.047197in}{0.728788in}}%
\pgfpathlineto{\pgfqpoint{2.047793in}{0.769488in}}%
\pgfpathlineto{\pgfqpoint{2.047991in}{0.769488in}}%
\pgfpathlineto{\pgfqpoint{2.047991in}{0.749138in}}%
\pgfpathlineto{\pgfqpoint{2.048984in}{0.789837in}}%
\pgfpathlineto{\pgfqpoint{2.049182in}{0.789837in}}%
\pgfpathlineto{\pgfqpoint{2.049182in}{0.728788in}}%
\pgfpathlineto{\pgfqpoint{2.050175in}{0.738963in}}%
\pgfpathlineto{\pgfqpoint{2.050770in}{0.738963in}}%
\pgfpathlineto{\pgfqpoint{2.050770in}{0.759313in}}%
\pgfpathlineto{\pgfqpoint{2.051763in}{0.728788in}}%
\pgfpathlineto{\pgfqpoint{2.051961in}{0.728788in}}%
\pgfpathlineto{\pgfqpoint{2.051961in}{0.759313in}}%
\pgfpathlineto{\pgfqpoint{2.052954in}{0.749138in}}%
\pgfpathlineto{\pgfqpoint{2.053152in}{0.749138in}}%
\pgfpathlineto{\pgfqpoint{2.053351in}{0.728788in}}%
\pgfpathlineto{\pgfqpoint{2.054145in}{0.728788in}}%
\pgfpathlineto{\pgfqpoint{2.054343in}{0.728788in}}%
\pgfpathlineto{\pgfqpoint{2.054542in}{0.769488in}}%
\pgfpathlineto{\pgfqpoint{2.055336in}{0.749138in}}%
\pgfpathlineto{\pgfqpoint{2.055534in}{0.749138in}}%
\pgfpathlineto{\pgfqpoint{2.055534in}{0.738963in}}%
\pgfpathlineto{\pgfqpoint{2.055733in}{0.779662in}}%
\pgfpathlineto{\pgfqpoint{2.056527in}{0.738963in}}%
\pgfpathlineto{\pgfqpoint{2.056924in}{0.738963in}}%
\pgfpathlineto{\pgfqpoint{2.057916in}{0.779662in}}%
\pgfpathlineto{\pgfqpoint{2.058115in}{0.779662in}}%
\pgfpathlineto{\pgfqpoint{2.058909in}{0.728788in}}%
\pgfpathlineto{\pgfqpoint{2.059107in}{0.749138in}}%
\pgfpathlineto{\pgfqpoint{2.059504in}{0.749138in}}%
\pgfpathlineto{\pgfqpoint{2.060298in}{0.769488in}}%
\pgfpathlineto{\pgfqpoint{2.060100in}{0.728788in}}%
\pgfpathlineto{\pgfqpoint{2.060497in}{0.749138in}}%
\pgfpathlineto{\pgfqpoint{2.060695in}{0.749138in}}%
\pgfpathlineto{\pgfqpoint{2.060695in}{0.759313in}}%
\pgfpathlineto{\pgfqpoint{2.060894in}{0.728788in}}%
\pgfpathlineto{\pgfqpoint{2.061688in}{0.759313in}}%
\pgfpathlineto{\pgfqpoint{2.061886in}{0.759313in}}%
\pgfpathlineto{\pgfqpoint{2.061886in}{0.728788in}}%
\pgfpathlineto{\pgfqpoint{2.062879in}{0.738963in}}%
\pgfpathlineto{\pgfqpoint{2.063077in}{0.738963in}}%
\pgfpathlineto{\pgfqpoint{2.063077in}{0.728788in}}%
\pgfpathlineto{\pgfqpoint{2.063871in}{0.759313in}}%
\pgfpathlineto{\pgfqpoint{2.064070in}{0.738963in}}%
\pgfpathlineto{\pgfqpoint{2.064268in}{0.738963in}}%
\pgfpathlineto{\pgfqpoint{2.064268in}{0.728788in}}%
\pgfpathlineto{\pgfqpoint{2.064467in}{0.749138in}}%
\pgfpathlineto{\pgfqpoint{2.065261in}{0.728788in}}%
\pgfpathlineto{\pgfqpoint{2.065459in}{0.728788in}}%
\pgfpathlineto{\pgfqpoint{2.066055in}{0.749138in}}%
\pgfpathlineto{\pgfqpoint{2.066452in}{0.749138in}}%
\pgfpathlineto{\pgfqpoint{2.066650in}{0.749138in}}%
\pgfpathlineto{\pgfqpoint{2.067246in}{0.728788in}}%
\pgfpathlineto{\pgfqpoint{2.066849in}{0.779662in}}%
\pgfpathlineto{\pgfqpoint{2.067643in}{0.728788in}}%
\pgfpathlineto{\pgfqpoint{2.067841in}{0.728788in}}%
\pgfpathlineto{\pgfqpoint{2.067841in}{0.759313in}}%
\pgfpathlineto{\pgfqpoint{2.068834in}{0.738963in}}%
\pgfpathlineto{\pgfqpoint{2.069032in}{0.738963in}}%
\pgfpathlineto{\pgfqpoint{2.069032in}{0.759313in}}%
\pgfpathlineto{\pgfqpoint{2.070025in}{0.738963in}}%
\pgfpathlineto{\pgfqpoint{2.070422in}{0.738963in}}%
\pgfpathlineto{\pgfqpoint{2.070422in}{0.769488in}}%
\pgfpathlineto{\pgfqpoint{2.071017in}{0.728788in}}%
\pgfpathlineto{\pgfqpoint{2.071414in}{0.738963in}}%
\pgfpathlineto{\pgfqpoint{2.071612in}{0.738963in}}%
\pgfpathlineto{\pgfqpoint{2.071612in}{0.769488in}}%
\pgfpathlineto{\pgfqpoint{2.071811in}{0.728788in}}%
\pgfpathlineto{\pgfqpoint{2.072605in}{0.738963in}}%
\pgfpathlineto{\pgfqpoint{2.073002in}{0.738963in}}%
\pgfpathlineto{\pgfqpoint{2.073002in}{0.749138in}}%
\pgfpathlineto{\pgfqpoint{2.073994in}{0.749138in}}%
\pgfpathlineto{\pgfqpoint{2.074193in}{0.749138in}}%
\pgfpathlineto{\pgfqpoint{2.074987in}{0.728788in}}%
\pgfpathlineto{\pgfqpoint{2.075185in}{0.728788in}}%
\pgfpathlineto{\pgfqpoint{2.075384in}{0.728788in}}%
\pgfpathlineto{\pgfqpoint{2.075582in}{0.779662in}}%
\pgfpathlineto{\pgfqpoint{2.076376in}{0.749138in}}%
\pgfpathlineto{\pgfqpoint{2.076575in}{0.749138in}}%
\pgfpathlineto{\pgfqpoint{2.076575in}{0.759313in}}%
\pgfpathlineto{\pgfqpoint{2.076972in}{0.728788in}}%
\pgfpathlineto{\pgfqpoint{2.077567in}{0.728788in}}%
\pgfpathlineto{\pgfqpoint{2.077964in}{0.728788in}}%
\pgfpathlineto{\pgfqpoint{2.077964in}{0.749138in}}%
\pgfpathlineto{\pgfqpoint{2.078957in}{0.738963in}}%
\pgfpathlineto{\pgfqpoint{2.079155in}{0.738963in}}%
\pgfpathlineto{\pgfqpoint{2.079155in}{0.759313in}}%
\pgfpathlineto{\pgfqpoint{2.080148in}{0.738963in}}%
\pgfpathlineto{\pgfqpoint{2.080346in}{0.738963in}}%
\pgfpathlineto{\pgfqpoint{2.080743in}{0.769488in}}%
\pgfpathlineto{\pgfqpoint{2.081140in}{0.728788in}}%
\pgfpathlineto{\pgfqpoint{2.081339in}{0.738963in}}%
\pgfpathlineto{\pgfqpoint{2.082133in}{0.738963in}}%
\pgfpathlineto{\pgfqpoint{2.082728in}{0.759313in}}%
\pgfpathlineto{\pgfqpoint{2.082331in}{0.728788in}}%
\pgfpathlineto{\pgfqpoint{2.083125in}{0.738963in}}%
\pgfpathlineto{\pgfqpoint{2.083324in}{0.738963in}}%
\pgfpathlineto{\pgfqpoint{2.083324in}{0.728788in}}%
\pgfpathlineto{\pgfqpoint{2.083522in}{0.749138in}}%
\pgfpathlineto{\pgfqpoint{2.084316in}{0.738963in}}%
\pgfpathlineto{\pgfqpoint{2.084912in}{0.738963in}}%
\pgfpathlineto{\pgfqpoint{2.084912in}{0.759313in}}%
\pgfpathlineto{\pgfqpoint{2.085110in}{0.728788in}}%
\pgfpathlineto{\pgfqpoint{2.085904in}{0.738963in}}%
\pgfpathlineto{\pgfqpoint{2.086103in}{0.738963in}}%
\pgfpathlineto{\pgfqpoint{2.086103in}{0.728788in}}%
\pgfpathlineto{\pgfqpoint{2.087095in}{0.759313in}}%
\pgfpathlineto{\pgfqpoint{2.087294in}{0.759313in}}%
\pgfpathlineto{\pgfqpoint{2.088088in}{0.728788in}}%
\pgfpathlineto{\pgfqpoint{2.088286in}{0.728788in}}%
\pgfpathlineto{\pgfqpoint{2.088485in}{0.728788in}}%
\pgfpathlineto{\pgfqpoint{2.089080in}{0.749138in}}%
\pgfpathlineto{\pgfqpoint{2.089477in}{0.738963in}}%
\pgfpathlineto{\pgfqpoint{2.089676in}{0.738963in}}%
\pgfpathlineto{\pgfqpoint{2.090073in}{0.769488in}}%
\pgfpathlineto{\pgfqpoint{2.090668in}{0.728788in}}%
\pgfpathlineto{\pgfqpoint{2.091065in}{0.728788in}}%
\pgfpathlineto{\pgfqpoint{2.092058in}{0.769488in}}%
\pgfpathlineto{\pgfqpoint{2.092256in}{0.769488in}}%
\pgfpathlineto{\pgfqpoint{2.092653in}{0.728788in}}%
\pgfpathlineto{\pgfqpoint{2.093249in}{0.728788in}}%
\pgfpathlineto{\pgfqpoint{2.093447in}{0.728788in}}%
\pgfpathlineto{\pgfqpoint{2.093646in}{0.759313in}}%
\pgfpathlineto{\pgfqpoint{2.094440in}{0.749138in}}%
\pgfpathlineto{\pgfqpoint{2.094638in}{0.749138in}}%
\pgfpathlineto{\pgfqpoint{2.095234in}{0.728788in}}%
\pgfpathlineto{\pgfqpoint{2.094837in}{0.769488in}}%
\pgfpathlineto{\pgfqpoint{2.095631in}{0.738963in}}%
\pgfpathlineto{\pgfqpoint{2.095829in}{0.738963in}}%
\pgfpathlineto{\pgfqpoint{2.096425in}{0.769488in}}%
\pgfpathlineto{\pgfqpoint{2.096822in}{0.738963in}}%
\pgfpathlineto{\pgfqpoint{2.097219in}{0.738963in}}%
\pgfpathlineto{\pgfqpoint{2.097219in}{0.728788in}}%
\pgfpathlineto{\pgfqpoint{2.098211in}{0.769488in}}%
\pgfpathlineto{\pgfqpoint{2.098410in}{0.769488in}}%
\pgfpathlineto{\pgfqpoint{2.098410in}{0.728788in}}%
\pgfpathlineto{\pgfqpoint{2.099402in}{0.759313in}}%
\pgfpathlineto{\pgfqpoint{2.099601in}{0.759313in}}%
\pgfpathlineto{\pgfqpoint{2.099601in}{0.728788in}}%
\pgfpathlineto{\pgfqpoint{2.100593in}{0.738963in}}%
\pgfpathlineto{\pgfqpoint{2.100792in}{0.738963in}}%
\pgfpathlineto{\pgfqpoint{2.100792in}{0.728788in}}%
\pgfpathlineto{\pgfqpoint{2.100990in}{0.749138in}}%
\pgfpathlineto{\pgfqpoint{2.101784in}{0.749138in}}%
\pgfpathlineto{\pgfqpoint{2.101983in}{0.749138in}}%
\pgfpathlineto{\pgfqpoint{2.101983in}{0.728788in}}%
\pgfpathlineto{\pgfqpoint{2.102975in}{0.728788in}}%
\pgfpathlineto{\pgfqpoint{2.103174in}{0.728788in}}%
\pgfpathlineto{\pgfqpoint{2.103174in}{0.749138in}}%
\pgfpathlineto{\pgfqpoint{2.104166in}{0.738963in}}%
\pgfpathlineto{\pgfqpoint{2.104563in}{0.738963in}}%
\pgfpathlineto{\pgfqpoint{2.104761in}{0.759313in}}%
\pgfpathlineto{\pgfqpoint{2.105555in}{0.738963in}}%
\pgfpathlineto{\pgfqpoint{2.106151in}{0.738963in}}%
\pgfpathlineto{\pgfqpoint{2.106151in}{0.749138in}}%
\pgfpathlineto{\pgfqpoint{2.106548in}{0.728788in}}%
\pgfpathlineto{\pgfqpoint{2.107143in}{0.728788in}}%
\pgfpathlineto{\pgfqpoint{2.107739in}{0.728788in}}%
\pgfpathlineto{\pgfqpoint{2.107739in}{0.759313in}}%
\pgfpathlineto{\pgfqpoint{2.108731in}{0.749138in}}%
\pgfpathlineto{\pgfqpoint{2.108930in}{0.749138in}}%
\pgfpathlineto{\pgfqpoint{2.109327in}{0.728788in}}%
\pgfpathlineto{\pgfqpoint{2.109128in}{0.769488in}}%
\pgfpathlineto{\pgfqpoint{2.109922in}{0.749138in}}%
\pgfpathlineto{\pgfqpoint{2.110319in}{0.749138in}}%
\pgfpathlineto{\pgfqpoint{2.110319in}{0.769488in}}%
\pgfpathlineto{\pgfqpoint{2.110915in}{0.728788in}}%
\pgfpathlineto{\pgfqpoint{2.111312in}{0.728788in}}%
\pgfpathlineto{\pgfqpoint{2.111510in}{0.728788in}}%
\pgfpathlineto{\pgfqpoint{2.112503in}{0.759313in}}%
\pgfpathlineto{\pgfqpoint{2.112701in}{0.759313in}}%
\pgfpathlineto{\pgfqpoint{2.113297in}{0.728788in}}%
\pgfpathlineto{\pgfqpoint{2.113694in}{0.728788in}}%
\pgfpathlineto{\pgfqpoint{2.114289in}{0.728788in}}%
\pgfpathlineto{\pgfqpoint{2.114289in}{0.769488in}}%
\pgfpathlineto{\pgfqpoint{2.115282in}{0.738963in}}%
\pgfpathlineto{\pgfqpoint{2.115480in}{0.738963in}}%
\pgfpathlineto{\pgfqpoint{2.115480in}{0.759313in}}%
\pgfpathlineto{\pgfqpoint{2.115679in}{0.728788in}}%
\pgfpathlineto{\pgfqpoint{2.116473in}{0.728788in}}%
\pgfpathlineto{\pgfqpoint{2.116671in}{0.728788in}}%
\pgfpathlineto{\pgfqpoint{2.117267in}{0.759313in}}%
\pgfpathlineto{\pgfqpoint{2.117664in}{0.749138in}}%
\pgfpathlineto{\pgfqpoint{2.117862in}{0.749138in}}%
\pgfpathlineto{\pgfqpoint{2.118061in}{0.728788in}}%
\pgfpathlineto{\pgfqpoint{2.118458in}{0.769488in}}%
\pgfpathlineto{\pgfqpoint{2.118855in}{0.749138in}}%
\pgfpathlineto{\pgfqpoint{2.119053in}{0.749138in}}%
\pgfpathlineto{\pgfqpoint{2.119053in}{0.738963in}}%
\pgfpathlineto{\pgfqpoint{2.120046in}{0.759313in}}%
\pgfpathlineto{\pgfqpoint{2.120244in}{0.759313in}}%
\pgfpathlineto{\pgfqpoint{2.120244in}{0.728788in}}%
\pgfpathlineto{\pgfqpoint{2.120443in}{0.769488in}}%
\pgfpathlineto{\pgfqpoint{2.121237in}{0.738963in}}%
\pgfpathlineto{\pgfqpoint{2.121435in}{0.738963in}}%
\pgfpathlineto{\pgfqpoint{2.121435in}{0.728788in}}%
\pgfpathlineto{\pgfqpoint{2.121832in}{0.769488in}}%
\pgfpathlineto{\pgfqpoint{2.122428in}{0.738963in}}%
\pgfpathlineto{\pgfqpoint{2.122626in}{0.738963in}}%
\pgfpathlineto{\pgfqpoint{2.122626in}{0.728788in}}%
\pgfpathlineto{\pgfqpoint{2.122825in}{0.749138in}}%
\pgfpathlineto{\pgfqpoint{2.123619in}{0.728788in}}%
\pgfpathlineto{\pgfqpoint{2.123817in}{0.728788in}}%
\pgfpathlineto{\pgfqpoint{2.124016in}{0.749138in}}%
\pgfpathlineto{\pgfqpoint{2.124810in}{0.749138in}}%
\pgfpathlineto{\pgfqpoint{2.125008in}{0.749138in}}%
\pgfpathlineto{\pgfqpoint{2.125008in}{0.728788in}}%
\pgfpathlineto{\pgfqpoint{2.126001in}{0.749138in}}%
\pgfpathlineto{\pgfqpoint{2.126199in}{0.749138in}}%
\pgfpathlineto{\pgfqpoint{2.126199in}{0.728788in}}%
\pgfpathlineto{\pgfqpoint{2.127192in}{0.738963in}}%
\pgfpathlineto{\pgfqpoint{2.128383in}{0.738963in}}%
\pgfpathlineto{\pgfqpoint{2.128383in}{0.759313in}}%
\pgfpathlineto{\pgfqpoint{2.129375in}{0.749138in}}%
\pgfpathlineto{\pgfqpoint{2.129574in}{0.749138in}}%
\pgfpathlineto{\pgfqpoint{2.129772in}{0.728788in}}%
\pgfpathlineto{\pgfqpoint{2.130566in}{0.749138in}}%
\pgfpathlineto{\pgfqpoint{2.130765in}{0.749138in}}%
\pgfpathlineto{\pgfqpoint{2.130963in}{0.728788in}}%
\pgfpathlineto{\pgfqpoint{2.131162in}{0.779662in}}%
\pgfpathlineto{\pgfqpoint{2.131757in}{0.728788in}}%
\pgfpathlineto{\pgfqpoint{2.131956in}{0.728788in}}%
\pgfpathlineto{\pgfqpoint{2.132750in}{0.759313in}}%
\pgfpathlineto{\pgfqpoint{2.132948in}{0.728788in}}%
\pgfpathlineto{\pgfqpoint{2.133345in}{0.728788in}}%
\pgfpathlineto{\pgfqpoint{2.133544in}{0.749138in}}%
\pgfpathlineto{\pgfqpoint{2.134338in}{0.749138in}}%
\pgfpathlineto{\pgfqpoint{2.134536in}{0.749138in}}%
\pgfpathlineto{\pgfqpoint{2.134536in}{0.738963in}}%
\pgfpathlineto{\pgfqpoint{2.135529in}{0.738963in}}%
\pgfpathlineto{\pgfqpoint{2.135727in}{0.738963in}}%
\pgfpathlineto{\pgfqpoint{2.135727in}{0.728788in}}%
\pgfpathlineto{\pgfqpoint{2.136521in}{0.769488in}}%
\pgfpathlineto{\pgfqpoint{2.136719in}{0.749138in}}%
\pgfpathlineto{\pgfqpoint{2.136918in}{0.749138in}}%
\pgfpathlineto{\pgfqpoint{2.137116in}{0.728788in}}%
\pgfpathlineto{\pgfqpoint{2.137910in}{0.738963in}}%
\pgfpathlineto{\pgfqpoint{2.138704in}{0.738963in}}%
\pgfpathlineto{\pgfqpoint{2.138704in}{0.728788in}}%
\pgfpathlineto{\pgfqpoint{2.139498in}{0.759313in}}%
\pgfpathlineto{\pgfqpoint{2.139697in}{0.728788in}}%
\pgfpathlineto{\pgfqpoint{2.139895in}{0.728788in}}%
\pgfpathlineto{\pgfqpoint{2.140888in}{0.749138in}}%
\pgfpathlineto{\pgfqpoint{2.141086in}{0.749138in}}%
\pgfpathlineto{\pgfqpoint{2.141285in}{0.728788in}}%
\pgfpathlineto{\pgfqpoint{2.142079in}{0.759313in}}%
\pgfpathlineto{\pgfqpoint{2.142277in}{0.759313in}}%
\pgfpathlineto{\pgfqpoint{2.143270in}{0.728788in}}%
\pgfpathlineto{\pgfqpoint{2.143468in}{0.728788in}}%
\pgfpathlineto{\pgfqpoint{2.143468in}{0.759313in}}%
\pgfpathlineto{\pgfqpoint{2.144461in}{0.749138in}}%
\pgfpathlineto{\pgfqpoint{2.144659in}{0.749138in}}%
\pgfpathlineto{\pgfqpoint{2.144659in}{0.759313in}}%
\pgfpathlineto{\pgfqpoint{2.145652in}{0.728788in}}%
\pgfpathlineto{\pgfqpoint{2.146247in}{0.728788in}}%
\pgfpathlineto{\pgfqpoint{2.147240in}{0.749138in}}%
\pgfpathlineto{\pgfqpoint{2.147438in}{0.749138in}}%
\pgfpathlineto{\pgfqpoint{2.147438in}{0.769488in}}%
\pgfpathlineto{\pgfqpoint{2.147637in}{0.728788in}}%
\pgfpathlineto{\pgfqpoint{2.148431in}{0.738963in}}%
\pgfpathlineto{\pgfqpoint{2.148629in}{0.738963in}}%
\pgfpathlineto{\pgfqpoint{2.149423in}{0.759313in}}%
\pgfpathlineto{\pgfqpoint{2.149026in}{0.728788in}}%
\pgfpathlineto{\pgfqpoint{2.149622in}{0.738963in}}%
\pgfpathlineto{\pgfqpoint{2.150217in}{0.738963in}}%
\pgfpathlineto{\pgfqpoint{2.150217in}{0.759313in}}%
\pgfpathlineto{\pgfqpoint{2.150416in}{0.728788in}}%
\pgfpathlineto{\pgfqpoint{2.151210in}{0.738963in}}%
\pgfpathlineto{\pgfqpoint{2.152202in}{0.738963in}}%
\pgfpathlineto{\pgfqpoint{2.152401in}{0.759313in}}%
\pgfpathlineto{\pgfqpoint{2.153195in}{0.738963in}}%
\pgfpathlineto{\pgfqpoint{2.153393in}{0.738963in}}%
\pgfpathlineto{\pgfqpoint{2.153393in}{0.728788in}}%
\pgfpathlineto{\pgfqpoint{2.154386in}{0.738963in}}%
\pgfpathlineto{\pgfqpoint{2.154981in}{0.738963in}}%
\pgfpathlineto{\pgfqpoint{2.154981in}{0.749138in}}%
\pgfpathlineto{\pgfqpoint{2.155180in}{0.728788in}}%
\pgfpathlineto{\pgfqpoint{2.155974in}{0.728788in}}%
\pgfpathlineto{\pgfqpoint{2.156172in}{0.728788in}}%
\pgfpathlineto{\pgfqpoint{2.157165in}{0.749138in}}%
\pgfpathlineto{\pgfqpoint{2.157363in}{0.749138in}}%
\pgfpathlineto{\pgfqpoint{2.157760in}{0.728788in}}%
\pgfpathlineto{\pgfqpoint{2.158356in}{0.728788in}}%
\pgfpathlineto{\pgfqpoint{2.159150in}{0.728788in}}%
\pgfpathlineto{\pgfqpoint{2.159150in}{0.749138in}}%
\pgfpathlineto{\pgfqpoint{2.160142in}{0.728788in}}%
\pgfpathlineto{\pgfqpoint{2.160341in}{0.728788in}}%
\pgfpathlineto{\pgfqpoint{2.160341in}{0.738963in}}%
\pgfpathlineto{\pgfqpoint{2.161333in}{0.728788in}}%
\pgfpathlineto{\pgfqpoint{2.161532in}{0.728788in}}%
\pgfpathlineto{\pgfqpoint{2.161532in}{0.749138in}}%
\pgfpathlineto{\pgfqpoint{2.162524in}{0.738963in}}%
\pgfpathlineto{\pgfqpoint{2.162921in}{0.738963in}}%
\pgfpathlineto{\pgfqpoint{2.163517in}{0.769488in}}%
\pgfpathlineto{\pgfqpoint{2.163914in}{0.738963in}}%
\pgfpathlineto{\pgfqpoint{2.164311in}{0.738963in}}%
\pgfpathlineto{\pgfqpoint{2.164311in}{0.728788in}}%
\pgfpathlineto{\pgfqpoint{2.165303in}{0.738963in}}%
\pgfpathlineto{\pgfqpoint{2.165700in}{0.738963in}}%
\pgfpathlineto{\pgfqpoint{2.165700in}{0.769488in}}%
\pgfpathlineto{\pgfqpoint{2.165899in}{0.728788in}}%
\pgfpathlineto{\pgfqpoint{2.166693in}{0.738963in}}%
\pgfpathlineto{\pgfqpoint{2.166891in}{0.738963in}}%
\pgfpathlineto{\pgfqpoint{2.166891in}{0.728788in}}%
\pgfpathlineto{\pgfqpoint{2.167685in}{0.749138in}}%
\pgfpathlineto{\pgfqpoint{2.167884in}{0.738963in}}%
\pgfpathlineto{\pgfqpoint{2.168281in}{0.738963in}}%
\pgfpathlineto{\pgfqpoint{2.168479in}{0.759313in}}%
\pgfpathlineto{\pgfqpoint{2.169273in}{0.759313in}}%
\pgfpathlineto{\pgfqpoint{2.169471in}{0.759313in}}%
\pgfpathlineto{\pgfqpoint{2.169670in}{0.728788in}}%
\pgfpathlineto{\pgfqpoint{2.170464in}{0.738963in}}%
\pgfpathlineto{\pgfqpoint{2.170662in}{0.738963in}}%
\pgfpathlineto{\pgfqpoint{2.170662in}{0.728788in}}%
\pgfpathlineto{\pgfqpoint{2.171655in}{0.728788in}}%
\pgfpathlineto{\pgfqpoint{2.172052in}{0.728788in}}%
\pgfpathlineto{\pgfqpoint{2.172449in}{0.749138in}}%
\pgfpathlineto{\pgfqpoint{2.173044in}{0.738963in}}%
\pgfpathlineto{\pgfqpoint{2.173243in}{0.738963in}}%
\pgfpathlineto{\pgfqpoint{2.173243in}{0.728788in}}%
\pgfpathlineto{\pgfqpoint{2.174235in}{0.728788in}}%
\pgfpathlineto{\pgfqpoint{2.174434in}{0.728788in}}%
\pgfpathlineto{\pgfqpoint{2.174434in}{0.749138in}}%
\pgfpathlineto{\pgfqpoint{2.175426in}{0.738963in}}%
\pgfpathlineto{\pgfqpoint{2.175625in}{0.738963in}}%
\pgfpathlineto{\pgfqpoint{2.175625in}{0.728788in}}%
\pgfpathlineto{\pgfqpoint{2.176617in}{0.759313in}}%
\pgfpathlineto{\pgfqpoint{2.176816in}{0.759313in}}%
\pgfpathlineto{\pgfqpoint{2.176816in}{0.728788in}}%
\pgfpathlineto{\pgfqpoint{2.177808in}{0.738963in}}%
\pgfpathlineto{\pgfqpoint{2.178007in}{0.738963in}}%
\pgfpathlineto{\pgfqpoint{2.178007in}{0.728788in}}%
\pgfpathlineto{\pgfqpoint{2.178999in}{0.728788in}}%
\pgfpathlineto{\pgfqpoint{2.179396in}{0.728788in}}%
\pgfpathlineto{\pgfqpoint{2.179396in}{0.738963in}}%
\pgfpathlineto{\pgfqpoint{2.180389in}{0.738963in}}%
\pgfpathlineto{\pgfqpoint{2.180587in}{0.738963in}}%
\pgfpathlineto{\pgfqpoint{2.180587in}{0.728788in}}%
\pgfpathlineto{\pgfqpoint{2.180786in}{0.759313in}}%
\pgfpathlineto{\pgfqpoint{2.181580in}{0.738963in}}%
\pgfpathlineto{\pgfqpoint{2.181778in}{0.738963in}}%
\pgfpathlineto{\pgfqpoint{2.181778in}{0.728788in}}%
\pgfpathlineto{\pgfqpoint{2.181977in}{0.759313in}}%
\pgfpathlineto{\pgfqpoint{2.182771in}{0.738963in}}%
\pgfpathlineto{\pgfqpoint{2.182969in}{0.738963in}}%
\pgfpathlineto{\pgfqpoint{2.182969in}{0.728788in}}%
\pgfpathlineto{\pgfqpoint{2.183962in}{0.769488in}}%
\pgfpathlineto{\pgfqpoint{2.184160in}{0.769488in}}%
\pgfpathlineto{\pgfqpoint{2.184359in}{0.728788in}}%
\pgfpathlineto{\pgfqpoint{2.185153in}{0.728788in}}%
\pgfpathlineto{\pgfqpoint{2.185550in}{0.728788in}}%
\pgfpathlineto{\pgfqpoint{2.185550in}{0.749138in}}%
\pgfpathlineto{\pgfqpoint{2.186542in}{0.728788in}}%
\pgfpathlineto{\pgfqpoint{2.186939in}{0.728788in}}%
\pgfpathlineto{\pgfqpoint{2.187535in}{0.749138in}}%
\pgfpathlineto{\pgfqpoint{2.187932in}{0.749138in}}%
\pgfpathlineto{\pgfqpoint{2.188130in}{0.749138in}}%
\pgfpathlineto{\pgfqpoint{2.188726in}{0.728788in}}%
\pgfpathlineto{\pgfqpoint{2.189123in}{0.728788in}}%
\pgfpathlineto{\pgfqpoint{2.189718in}{0.728788in}}%
\pgfpathlineto{\pgfqpoint{2.189718in}{0.749138in}}%
\pgfpathlineto{\pgfqpoint{2.190711in}{0.738963in}}%
\pgfpathlineto{\pgfqpoint{2.190909in}{0.738963in}}%
\pgfpathlineto{\pgfqpoint{2.190909in}{0.728788in}}%
\pgfpathlineto{\pgfqpoint{2.191108in}{0.749138in}}%
\pgfpathlineto{\pgfqpoint{2.191902in}{0.749138in}}%
\pgfpathlineto{\pgfqpoint{2.192100in}{0.749138in}}%
\pgfpathlineto{\pgfqpoint{2.192497in}{0.728788in}}%
\pgfpathlineto{\pgfqpoint{2.193093in}{0.728788in}}%
\pgfpathlineto{\pgfqpoint{2.193291in}{0.728788in}}%
\pgfpathlineto{\pgfqpoint{2.194085in}{0.749138in}}%
\pgfpathlineto{\pgfqpoint{2.194284in}{0.749138in}}%
\pgfpathlineto{\pgfqpoint{2.194482in}{0.749138in}}%
\pgfpathlineto{\pgfqpoint{2.194482in}{0.728788in}}%
\pgfpathlineto{\pgfqpoint{2.195475in}{0.728788in}}%
\pgfpathlineto{\pgfqpoint{2.195872in}{0.728788in}}%
\pgfpathlineto{\pgfqpoint{2.195872in}{0.759313in}}%
\pgfpathlineto{\pgfqpoint{2.196864in}{0.738963in}}%
\pgfpathlineto{\pgfqpoint{2.197063in}{0.738963in}}%
\pgfpathlineto{\pgfqpoint{2.197063in}{0.728788in}}%
\pgfpathlineto{\pgfqpoint{2.197658in}{0.759313in}}%
\pgfpathlineto{\pgfqpoint{2.198055in}{0.728788in}}%
\pgfpathlineto{\pgfqpoint{2.198254in}{0.728788in}}%
\pgfpathlineto{\pgfqpoint{2.198254in}{0.749138in}}%
\pgfpathlineto{\pgfqpoint{2.199246in}{0.738963in}}%
\pgfpathlineto{\pgfqpoint{2.199643in}{0.738963in}}%
\pgfpathlineto{\pgfqpoint{2.199643in}{0.728788in}}%
\pgfpathlineto{\pgfqpoint{2.200636in}{0.738963in}}%
\pgfpathlineto{\pgfqpoint{2.201032in}{0.738963in}}%
\pgfpathlineto{\pgfqpoint{2.201032in}{0.728788in}}%
\pgfpathlineto{\pgfqpoint{2.202025in}{0.728788in}}%
\pgfpathlineto{\pgfqpoint{2.202620in}{0.728788in}}%
\pgfpathlineto{\pgfqpoint{2.203216in}{0.749138in}}%
\pgfpathlineto{\pgfqpoint{2.203613in}{0.738963in}}%
\pgfpathlineto{\pgfqpoint{2.203811in}{0.738963in}}%
\pgfpathlineto{\pgfqpoint{2.203811in}{0.728788in}}%
\pgfpathlineto{\pgfqpoint{2.204010in}{0.759313in}}%
\pgfpathlineto{\pgfqpoint{2.204804in}{0.738963in}}%
\pgfpathlineto{\pgfqpoint{2.205399in}{0.738963in}}%
\pgfpathlineto{\pgfqpoint{2.205399in}{0.728788in}}%
\pgfpathlineto{\pgfqpoint{2.206193in}{0.749138in}}%
\pgfpathlineto{\pgfqpoint{2.206392in}{0.738963in}}%
\pgfpathlineto{\pgfqpoint{2.207186in}{0.738963in}}%
\pgfpathlineto{\pgfqpoint{2.207186in}{0.728788in}}%
\pgfpathlineto{\pgfqpoint{2.207781in}{0.749138in}}%
\pgfpathlineto{\pgfqpoint{2.208178in}{0.738963in}}%
\pgfpathlineto{\pgfqpoint{2.208377in}{0.738963in}}%
\pgfpathlineto{\pgfqpoint{2.208377in}{0.759313in}}%
\pgfpathlineto{\pgfqpoint{2.208575in}{0.728788in}}%
\pgfpathlineto{\pgfqpoint{2.209369in}{0.749138in}}%
\pgfpathlineto{\pgfqpoint{2.209568in}{0.749138in}}%
\pgfpathlineto{\pgfqpoint{2.209766in}{0.728788in}}%
\pgfpathlineto{\pgfqpoint{2.210560in}{0.728788in}}%
\pgfpathlineto{\pgfqpoint{2.211354in}{0.728788in}}%
\pgfpathlineto{\pgfqpoint{2.211354in}{0.749138in}}%
\pgfpathlineto{\pgfqpoint{2.212347in}{0.749138in}}%
\pgfpathlineto{\pgfqpoint{2.212545in}{0.749138in}}%
\pgfpathlineto{\pgfqpoint{2.212545in}{0.728788in}}%
\pgfpathlineto{\pgfqpoint{2.213538in}{0.738963in}}%
\pgfpathlineto{\pgfqpoint{2.213736in}{0.738963in}}%
\pgfpathlineto{\pgfqpoint{2.213736in}{0.728788in}}%
\pgfpathlineto{\pgfqpoint{2.213935in}{0.749138in}}%
\pgfpathlineto{\pgfqpoint{2.214729in}{0.738963in}}%
\pgfpathlineto{\pgfqpoint{2.214927in}{0.738963in}}%
\pgfpathlineto{\pgfqpoint{2.214927in}{0.728788in}}%
\pgfpathlineto{\pgfqpoint{2.215721in}{0.759313in}}%
\pgfpathlineto{\pgfqpoint{2.215920in}{0.749138in}}%
\pgfpathlineto{\pgfqpoint{2.216317in}{0.749138in}}%
\pgfpathlineto{\pgfqpoint{2.216317in}{0.728788in}}%
\pgfpathlineto{\pgfqpoint{2.216515in}{0.759313in}}%
\pgfpathlineto{\pgfqpoint{2.217309in}{0.728788in}}%
\pgfpathlineto{\pgfqpoint{2.217508in}{0.728788in}}%
\pgfpathlineto{\pgfqpoint{2.217508in}{0.749138in}}%
\pgfpathlineto{\pgfqpoint{2.218500in}{0.728788in}}%
\pgfpathlineto{\pgfqpoint{2.218897in}{0.728788in}}%
\pgfpathlineto{\pgfqpoint{2.219096in}{0.749138in}}%
\pgfpathlineto{\pgfqpoint{2.219890in}{0.738963in}}%
\pgfpathlineto{\pgfqpoint{2.220088in}{0.738963in}}%
\pgfpathlineto{\pgfqpoint{2.220088in}{0.728788in}}%
\pgfpathlineto{\pgfqpoint{2.220287in}{0.749138in}}%
\pgfpathlineto{\pgfqpoint{2.221081in}{0.728788in}}%
\pgfpathlineto{\pgfqpoint{2.221676in}{0.728788in}}%
\pgfpathlineto{\pgfqpoint{2.221676in}{0.749138in}}%
\pgfpathlineto{\pgfqpoint{2.222669in}{0.738963in}}%
\pgfpathlineto{\pgfqpoint{2.222867in}{0.738963in}}%
\pgfpathlineto{\pgfqpoint{2.222867in}{0.749138in}}%
\pgfpathlineto{\pgfqpoint{2.223264in}{0.728788in}}%
\pgfpathlineto{\pgfqpoint{2.223860in}{0.728788in}}%
\pgfpathlineto{\pgfqpoint{2.224257in}{0.728788in}}%
\pgfpathlineto{\pgfqpoint{2.224257in}{0.749138in}}%
\pgfpathlineto{\pgfqpoint{2.225249in}{0.738963in}}%
\pgfpathlineto{\pgfqpoint{2.225646in}{0.738963in}}%
\pgfpathlineto{\pgfqpoint{2.225646in}{0.749138in}}%
\pgfpathlineto{\pgfqpoint{2.226043in}{0.728788in}}%
\pgfpathlineto{\pgfqpoint{2.226639in}{0.728788in}}%
\pgfpathlineto{\pgfqpoint{2.227036in}{0.728788in}}%
\pgfpathlineto{\pgfqpoint{2.227830in}{0.749138in}}%
\pgfpathlineto{\pgfqpoint{2.228028in}{0.749138in}}%
\pgfpathlineto{\pgfqpoint{2.228425in}{0.749138in}}%
\pgfpathlineto{\pgfqpoint{2.228425in}{0.728788in}}%
\pgfpathlineto{\pgfqpoint{2.229418in}{0.738963in}}%
\pgfpathlineto{\pgfqpoint{2.229815in}{0.738963in}}%
\pgfpathlineto{\pgfqpoint{2.229815in}{0.728788in}}%
\pgfpathlineto{\pgfqpoint{2.230807in}{0.728788in}}%
\pgfpathlineto{\pgfqpoint{2.231204in}{0.728788in}}%
\pgfpathlineto{\pgfqpoint{2.231204in}{0.749138in}}%
\pgfpathlineto{\pgfqpoint{2.232197in}{0.728788in}}%
\pgfpathlineto{\pgfqpoint{2.232594in}{0.728788in}}%
\pgfpathlineto{\pgfqpoint{2.232594in}{0.738963in}}%
\pgfpathlineto{\pgfqpoint{2.233586in}{0.738963in}}%
\pgfpathlineto{\pgfqpoint{2.234181in}{0.738963in}}%
\pgfpathlineto{\pgfqpoint{2.234181in}{0.728788in}}%
\pgfpathlineto{\pgfqpoint{2.234975in}{0.759313in}}%
\pgfpathlineto{\pgfqpoint{2.235174in}{0.738963in}}%
\pgfpathlineto{\pgfqpoint{2.235372in}{0.738963in}}%
\pgfpathlineto{\pgfqpoint{2.235372in}{0.728788in}}%
\pgfpathlineto{\pgfqpoint{2.236365in}{0.728788in}}%
\pgfpathlineto{\pgfqpoint{2.236563in}{0.728788in}}%
\pgfpathlineto{\pgfqpoint{2.236563in}{0.738963in}}%
\pgfpathlineto{\pgfqpoint{2.237556in}{0.728788in}}%
\pgfpathlineto{\pgfqpoint{2.237754in}{0.728788in}}%
\pgfpathlineto{\pgfqpoint{2.237953in}{0.749138in}}%
\pgfpathlineto{\pgfqpoint{2.238747in}{0.738963in}}%
\pgfpathlineto{\pgfqpoint{2.238945in}{0.738963in}}%
\pgfpathlineto{\pgfqpoint{2.238945in}{0.728788in}}%
\pgfpathlineto{\pgfqpoint{2.239938in}{0.749138in}}%
\pgfpathlineto{\pgfqpoint{2.240136in}{0.749138in}}%
\pgfpathlineto{\pgfqpoint{2.240732in}{0.728788in}}%
\pgfpathlineto{\pgfqpoint{2.241129in}{0.738963in}}%
\pgfpathlineto{\pgfqpoint{2.241327in}{0.738963in}}%
\pgfpathlineto{\pgfqpoint{2.241327in}{0.728788in}}%
\pgfpathlineto{\pgfqpoint{2.242320in}{0.749138in}}%
\pgfpathlineto{\pgfqpoint{2.242518in}{0.749138in}}%
\pgfpathlineto{\pgfqpoint{2.242518in}{0.759313in}}%
\pgfpathlineto{\pgfqpoint{2.242717in}{0.728788in}}%
\pgfpathlineto{\pgfqpoint{2.243511in}{0.738963in}}%
\pgfpathlineto{\pgfqpoint{2.243709in}{0.738963in}}%
\pgfpathlineto{\pgfqpoint{2.243709in}{0.749138in}}%
\pgfpathlineto{\pgfqpoint{2.243908in}{0.728788in}}%
\pgfpathlineto{\pgfqpoint{2.244702in}{0.728788in}}%
\pgfpathlineto{\pgfqpoint{2.244900in}{0.728788in}}%
\pgfpathlineto{\pgfqpoint{2.244900in}{0.738963in}}%
\pgfpathlineto{\pgfqpoint{2.245893in}{0.728788in}}%
\pgfpathlineto{\pgfqpoint{2.246091in}{0.728788in}}%
\pgfpathlineto{\pgfqpoint{2.246091in}{0.759313in}}%
\pgfpathlineto{\pgfqpoint{2.247084in}{0.728788in}}%
\pgfpathlineto{\pgfqpoint{2.247481in}{0.728788in}}%
\pgfpathlineto{\pgfqpoint{2.247679in}{0.749138in}}%
\pgfpathlineto{\pgfqpoint{2.248473in}{0.738963in}}%
\pgfpathlineto{\pgfqpoint{2.248870in}{0.738963in}}%
\pgfpathlineto{\pgfqpoint{2.248870in}{0.728788in}}%
\pgfpathlineto{\pgfqpoint{2.249267in}{0.749138in}}%
\pgfpathlineto{\pgfqpoint{2.249863in}{0.728788in}}%
\pgfpathlineto{\pgfqpoint{2.250458in}{0.728788in}}%
\pgfpathlineto{\pgfqpoint{2.250458in}{0.749138in}}%
\pgfpathlineto{\pgfqpoint{2.251451in}{0.738963in}}%
\pgfpathlineto{\pgfqpoint{2.251649in}{0.738963in}}%
\pgfpathlineto{\pgfqpoint{2.251649in}{0.728788in}}%
\pgfpathlineto{\pgfqpoint{2.252642in}{0.738963in}}%
\pgfpathlineto{\pgfqpoint{2.252840in}{0.738963in}}%
\pgfpathlineto{\pgfqpoint{2.252840in}{0.728788in}}%
\pgfpathlineto{\pgfqpoint{2.253833in}{0.728788in}}%
\pgfpathlineto{\pgfqpoint{2.254031in}{0.728788in}}%
\pgfpathlineto{\pgfqpoint{2.254428in}{0.749138in}}%
\pgfpathlineto{\pgfqpoint{2.255024in}{0.728788in}}%
\pgfpathlineto{\pgfqpoint{2.255818in}{0.728788in}}%
\pgfpathlineto{\pgfqpoint{2.255818in}{0.738963in}}%
\pgfpathlineto{\pgfqpoint{2.256810in}{0.738963in}}%
\pgfpathlineto{\pgfqpoint{2.257009in}{0.738963in}}%
\pgfpathlineto{\pgfqpoint{2.257009in}{0.759313in}}%
\pgfpathlineto{\pgfqpoint{2.257406in}{0.728788in}}%
\pgfpathlineto{\pgfqpoint{2.258001in}{0.728788in}}%
\pgfpathlineto{\pgfqpoint{2.258398in}{0.728788in}}%
\pgfpathlineto{\pgfqpoint{2.258398in}{0.749138in}}%
\pgfpathlineto{\pgfqpoint{2.259391in}{0.738963in}}%
\pgfpathlineto{\pgfqpoint{2.259986in}{0.738963in}}%
\pgfpathlineto{\pgfqpoint{2.259986in}{0.728788in}}%
\pgfpathlineto{\pgfqpoint{2.260979in}{0.728788in}}%
\pgfpathlineto{\pgfqpoint{2.261574in}{0.728788in}}%
\pgfpathlineto{\pgfqpoint{2.261971in}{0.759313in}}%
\pgfpathlineto{\pgfqpoint{2.262567in}{0.728788in}}%
\pgfpathlineto{\pgfqpoint{2.263361in}{0.728788in}}%
\pgfpathlineto{\pgfqpoint{2.263559in}{0.749138in}}%
\pgfpathlineto{\pgfqpoint{2.264353in}{0.738963in}}%
\pgfpathlineto{\pgfqpoint{2.264750in}{0.738963in}}%
\pgfpathlineto{\pgfqpoint{2.264750in}{0.728788in}}%
\pgfpathlineto{\pgfqpoint{2.265742in}{0.738963in}}%
\pgfpathlineto{\pgfqpoint{2.265941in}{0.738963in}}%
\pgfpathlineto{\pgfqpoint{2.265941in}{0.728788in}}%
\pgfpathlineto{\pgfqpoint{2.266338in}{0.749138in}}%
\pgfpathlineto{\pgfqpoint{2.266933in}{0.738963in}}%
\pgfpathlineto{\pgfqpoint{2.267132in}{0.738963in}}%
\pgfpathlineto{\pgfqpoint{2.267132in}{0.749138in}}%
\pgfpathlineto{\pgfqpoint{2.267330in}{0.728788in}}%
\pgfpathlineto{\pgfqpoint{2.268124in}{0.749138in}}%
\pgfpathlineto{\pgfqpoint{2.268323in}{0.749138in}}%
\pgfpathlineto{\pgfqpoint{2.268323in}{0.728788in}}%
\pgfpathlineto{\pgfqpoint{2.269315in}{0.738963in}}%
\pgfpathlineto{\pgfqpoint{2.269514in}{0.738963in}}%
\pgfpathlineto{\pgfqpoint{2.269514in}{0.728788in}}%
\pgfpathlineto{\pgfqpoint{2.270506in}{0.728788in}}%
\pgfpathlineto{\pgfqpoint{2.270705in}{0.728788in}}%
\pgfpathlineto{\pgfqpoint{2.271102in}{0.749138in}}%
\pgfpathlineto{\pgfqpoint{2.271697in}{0.728788in}}%
\pgfpathlineto{\pgfqpoint{2.272491in}{0.728788in}}%
\pgfpathlineto{\pgfqpoint{2.272491in}{0.738963in}}%
\pgfpathlineto{\pgfqpoint{2.273484in}{0.738963in}}%
\pgfpathlineto{\pgfqpoint{2.273682in}{0.738963in}}%
\pgfpathlineto{\pgfqpoint{2.273682in}{0.728788in}}%
\pgfpathlineto{\pgfqpoint{2.274675in}{0.728788in}}%
\pgfpathlineto{\pgfqpoint{2.275270in}{0.728788in}}%
\pgfpathlineto{\pgfqpoint{2.276263in}{0.749138in}}%
\pgfpathlineto{\pgfqpoint{2.276660in}{0.749138in}}%
\pgfpathlineto{\pgfqpoint{2.276660in}{0.728788in}}%
\pgfpathlineto{\pgfqpoint{2.277652in}{0.759313in}}%
\pgfpathlineto{\pgfqpoint{2.277851in}{0.759313in}}%
\pgfpathlineto{\pgfqpoint{2.278049in}{0.728788in}}%
\pgfpathlineto{\pgfqpoint{2.278843in}{0.738963in}}%
\pgfpathlineto{\pgfqpoint{2.279042in}{0.738963in}}%
\pgfpathlineto{\pgfqpoint{2.279042in}{0.749138in}}%
\pgfpathlineto{\pgfqpoint{2.279836in}{0.728788in}}%
\pgfpathlineto{\pgfqpoint{2.280034in}{0.728788in}}%
\pgfpathlineto{\pgfqpoint{2.280233in}{0.728788in}}%
\pgfpathlineto{\pgfqpoint{2.280233in}{0.738963in}}%
\pgfpathlineto{\pgfqpoint{2.281225in}{0.728788in}}%
\pgfpathlineto{\pgfqpoint{2.281424in}{0.728788in}}%
\pgfpathlineto{\pgfqpoint{2.281424in}{0.738963in}}%
\pgfpathlineto{\pgfqpoint{2.282416in}{0.738963in}}%
\pgfpathlineto{\pgfqpoint{2.282615in}{0.738963in}}%
\pgfpathlineto{\pgfqpoint{2.282615in}{0.728788in}}%
\pgfpathlineto{\pgfqpoint{2.283607in}{0.728788in}}%
\pgfpathlineto{\pgfqpoint{2.283806in}{0.728788in}}%
\pgfpathlineto{\pgfqpoint{2.284203in}{0.759313in}}%
\pgfpathlineto{\pgfqpoint{2.284798in}{0.738963in}}%
\pgfpathlineto{\pgfqpoint{2.284997in}{0.738963in}}%
\pgfpathlineto{\pgfqpoint{2.284997in}{0.728788in}}%
\pgfpathlineto{\pgfqpoint{2.285791in}{0.749138in}}%
\pgfpathlineto{\pgfqpoint{2.285989in}{0.738963in}}%
\pgfpathlineto{\pgfqpoint{2.286188in}{0.738963in}}%
\pgfpathlineto{\pgfqpoint{2.286188in}{0.728788in}}%
\pgfpathlineto{\pgfqpoint{2.287180in}{0.738963in}}%
\pgfpathlineto{\pgfqpoint{2.287379in}{0.738963in}}%
\pgfpathlineto{\pgfqpoint{2.287379in}{0.728788in}}%
\pgfpathlineto{\pgfqpoint{2.288371in}{0.749138in}}%
\pgfpathlineto{\pgfqpoint{2.288570in}{0.749138in}}%
\pgfpathlineto{\pgfqpoint{2.288570in}{0.738963in}}%
\pgfpathlineto{\pgfqpoint{2.289562in}{0.738963in}}%
\pgfpathlineto{\pgfqpoint{2.289761in}{0.738963in}}%
\pgfpathlineto{\pgfqpoint{2.289761in}{0.728788in}}%
\pgfpathlineto{\pgfqpoint{2.290753in}{0.728788in}}%
\pgfpathlineto{\pgfqpoint{2.291944in}{0.728788in}}%
\pgfpathlineto{\pgfqpoint{2.291944in}{0.738963in}}%
\pgfpathlineto{\pgfqpoint{2.292937in}{0.738963in}}%
\pgfpathlineto{\pgfqpoint{2.293334in}{0.738963in}}%
\pgfpathlineto{\pgfqpoint{2.293334in}{0.728788in}}%
\pgfpathlineto{\pgfqpoint{2.294326in}{0.728788in}}%
\pgfpathlineto{\pgfqpoint{2.294723in}{0.728788in}}%
\pgfpathlineto{\pgfqpoint{2.294723in}{0.738963in}}%
\pgfpathlineto{\pgfqpoint{2.295716in}{0.738963in}}%
\pgfpathlineto{\pgfqpoint{2.295914in}{0.738963in}}%
\pgfpathlineto{\pgfqpoint{2.295914in}{0.728788in}}%
\pgfpathlineto{\pgfqpoint{2.296510in}{0.749138in}}%
\pgfpathlineto{\pgfqpoint{2.296907in}{0.738963in}}%
\pgfpathlineto{\pgfqpoint{2.297105in}{0.738963in}}%
\pgfpathlineto{\pgfqpoint{2.297105in}{0.749138in}}%
\pgfpathlineto{\pgfqpoint{2.297304in}{0.728788in}}%
\pgfpathlineto{\pgfqpoint{2.298097in}{0.738963in}}%
\pgfpathlineto{\pgfqpoint{2.298296in}{0.738963in}}%
\pgfpathlineto{\pgfqpoint{2.298296in}{0.728788in}}%
\pgfpathlineto{\pgfqpoint{2.299288in}{0.738963in}}%
\pgfpathlineto{\pgfqpoint{2.299487in}{0.738963in}}%
\pgfpathlineto{\pgfqpoint{2.299487in}{0.749138in}}%
\pgfpathlineto{\pgfqpoint{2.299685in}{0.728788in}}%
\pgfpathlineto{\pgfqpoint{2.300479in}{0.728788in}}%
\pgfpathlineto{\pgfqpoint{2.300876in}{0.728788in}}%
\pgfpathlineto{\pgfqpoint{2.300876in}{0.749138in}}%
\pgfpathlineto{\pgfqpoint{2.301869in}{0.738963in}}%
\pgfpathlineto{\pgfqpoint{2.302067in}{0.738963in}}%
\pgfpathlineto{\pgfqpoint{2.302067in}{0.728788in}}%
\pgfpathlineto{\pgfqpoint{2.303060in}{0.728788in}}%
\pgfpathlineto{\pgfqpoint{2.303655in}{0.728788in}}%
\pgfpathlineto{\pgfqpoint{2.303655in}{0.738963in}}%
\pgfpathlineto{\pgfqpoint{2.304648in}{0.728788in}}%
\pgfpathlineto{\pgfqpoint{2.304846in}{0.728788in}}%
\pgfpathlineto{\pgfqpoint{2.305640in}{0.749138in}}%
\pgfpathlineto{\pgfqpoint{2.305839in}{0.738963in}}%
\pgfpathlineto{\pgfqpoint{2.306236in}{0.738963in}}%
\pgfpathlineto{\pgfqpoint{2.306236in}{0.728788in}}%
\pgfpathlineto{\pgfqpoint{2.307228in}{0.728788in}}%
\pgfpathlineto{\pgfqpoint{2.307427in}{0.728788in}}%
\pgfpathlineto{\pgfqpoint{2.307427in}{0.749138in}}%
\pgfpathlineto{\pgfqpoint{2.308419in}{0.728788in}}%
\pgfpathlineto{\pgfqpoint{2.309610in}{0.728788in}}%
\pgfpathlineto{\pgfqpoint{2.310404in}{0.759313in}}%
\pgfpathlineto{\pgfqpoint{2.310603in}{0.728788in}}%
\pgfpathlineto{\pgfqpoint{2.310801in}{0.728788in}}%
\pgfpathlineto{\pgfqpoint{2.311198in}{0.749138in}}%
\pgfpathlineto{\pgfqpoint{2.311794in}{0.728788in}}%
\pgfpathlineto{\pgfqpoint{2.312588in}{0.728788in}}%
\pgfpathlineto{\pgfqpoint{2.312588in}{0.738963in}}%
\pgfpathlineto{\pgfqpoint{2.313580in}{0.738963in}}%
\pgfpathlineto{\pgfqpoint{2.313977in}{0.738963in}}%
\pgfpathlineto{\pgfqpoint{2.313977in}{0.759313in}}%
\pgfpathlineto{\pgfqpoint{2.314176in}{0.728788in}}%
\pgfpathlineto{\pgfqpoint{2.314970in}{0.738963in}}%
\pgfpathlineto{\pgfqpoint{2.315168in}{0.738963in}}%
\pgfpathlineto{\pgfqpoint{2.315168in}{0.728788in}}%
\pgfpathlineto{\pgfqpoint{2.316161in}{0.728788in}}%
\pgfpathlineto{\pgfqpoint{2.316359in}{0.728788in}}%
\pgfpathlineto{\pgfqpoint{2.317153in}{0.759313in}}%
\pgfpathlineto{\pgfqpoint{2.317352in}{0.728788in}}%
\pgfpathlineto{\pgfqpoint{2.317550in}{0.728788in}}%
\pgfpathlineto{\pgfqpoint{2.317749in}{0.749138in}}%
\pgfpathlineto{\pgfqpoint{2.318543in}{0.728788in}}%
\pgfpathlineto{\pgfqpoint{2.318741in}{0.728788in}}%
\pgfpathlineto{\pgfqpoint{2.318741in}{0.738963in}}%
\pgfpathlineto{\pgfqpoint{2.319734in}{0.738963in}}%
\pgfpathlineto{\pgfqpoint{2.319932in}{0.738963in}}%
\pgfpathlineto{\pgfqpoint{2.319932in}{0.728788in}}%
\pgfpathlineto{\pgfqpoint{2.320528in}{0.749138in}}%
\pgfpathlineto{\pgfqpoint{2.320925in}{0.728788in}}%
\pgfpathlineto{\pgfqpoint{2.321123in}{0.728788in}}%
\pgfpathlineto{\pgfqpoint{2.321719in}{0.749138in}}%
\pgfpathlineto{\pgfqpoint{2.322116in}{0.738963in}}%
\pgfpathlineto{\pgfqpoint{2.322513in}{0.738963in}}%
\pgfpathlineto{\pgfqpoint{2.322513in}{0.728788in}}%
\pgfpathlineto{\pgfqpoint{2.323505in}{0.749138in}}%
\pgfpathlineto{\pgfqpoint{2.323704in}{0.749138in}}%
\pgfpathlineto{\pgfqpoint{2.323704in}{0.728788in}}%
\pgfpathlineto{\pgfqpoint{2.324498in}{0.759313in}}%
\pgfpathlineto{\pgfqpoint{2.324696in}{0.728788in}}%
\pgfpathlineto{\pgfqpoint{2.324895in}{0.728788in}}%
\pgfpathlineto{\pgfqpoint{2.324895in}{0.738963in}}%
\pgfpathlineto{\pgfqpoint{2.325887in}{0.728788in}}%
\pgfpathlineto{\pgfqpoint{2.327674in}{0.728788in}}%
\pgfpathlineto{\pgfqpoint{2.327674in}{0.738963in}}%
\pgfpathlineto{\pgfqpoint{2.328666in}{0.728788in}}%
\pgfpathlineto{\pgfqpoint{2.329063in}{0.728788in}}%
\pgfpathlineto{\pgfqpoint{2.329857in}{0.749138in}}%
\pgfpathlineto{\pgfqpoint{2.330055in}{0.738963in}}%
\pgfpathlineto{\pgfqpoint{2.330452in}{0.738963in}}%
\pgfpathlineto{\pgfqpoint{2.330452in}{0.728788in}}%
\pgfpathlineto{\pgfqpoint{2.331445in}{0.728788in}}%
\pgfpathlineto{\pgfqpoint{2.331842in}{0.728788in}}%
\pgfpathlineto{\pgfqpoint{2.331842in}{0.738963in}}%
\pgfpathlineto{\pgfqpoint{2.332834in}{0.728788in}}%
\pgfpathlineto{\pgfqpoint{2.333033in}{0.728788in}}%
\pgfpathlineto{\pgfqpoint{2.333033in}{0.749138in}}%
\pgfpathlineto{\pgfqpoint{2.334025in}{0.728788in}}%
\pgfpathlineto{\pgfqpoint{2.334422in}{0.728788in}}%
\pgfpathlineto{\pgfqpoint{2.334422in}{0.738963in}}%
\pgfpathlineto{\pgfqpoint{2.335415in}{0.738963in}}%
\pgfpathlineto{\pgfqpoint{2.335613in}{0.738963in}}%
\pgfpathlineto{\pgfqpoint{2.335613in}{0.728788in}}%
\pgfpathlineto{\pgfqpoint{2.336606in}{0.738963in}}%
\pgfpathlineto{\pgfqpoint{2.336804in}{0.738963in}}%
\pgfpathlineto{\pgfqpoint{2.336804in}{0.728788in}}%
\pgfpathlineto{\pgfqpoint{2.337797in}{0.728788in}}%
\pgfpathlineto{\pgfqpoint{2.338789in}{0.728788in}}%
\pgfpathlineto{\pgfqpoint{2.338789in}{0.738963in}}%
\pgfpathlineto{\pgfqpoint{2.339782in}{0.728788in}}%
\pgfpathlineto{\pgfqpoint{2.340179in}{0.728788in}}%
\pgfpathlineto{\pgfqpoint{2.340179in}{0.738963in}}%
\pgfpathlineto{\pgfqpoint{2.341171in}{0.728788in}}%
\pgfpathlineto{\pgfqpoint{2.342561in}{0.728788in}}%
\pgfpathlineto{\pgfqpoint{2.342561in}{0.759313in}}%
\pgfpathlineto{\pgfqpoint{2.343553in}{0.728788in}}%
\pgfpathlineto{\pgfqpoint{2.343950in}{0.728788in}}%
\pgfpathlineto{\pgfqpoint{2.343950in}{0.738963in}}%
\pgfpathlineto{\pgfqpoint{2.344943in}{0.728788in}}%
\pgfpathlineto{\pgfqpoint{2.345141in}{0.728788in}}%
\pgfpathlineto{\pgfqpoint{2.345141in}{0.738963in}}%
\pgfpathlineto{\pgfqpoint{2.346134in}{0.738963in}}%
\pgfpathlineto{\pgfqpoint{2.346332in}{0.738963in}}%
\pgfpathlineto{\pgfqpoint{2.346332in}{0.728788in}}%
\pgfpathlineto{\pgfqpoint{2.347126in}{0.749138in}}%
\pgfpathlineto{\pgfqpoint{2.347325in}{0.738963in}}%
\pgfpathlineto{\pgfqpoint{2.347523in}{0.738963in}}%
\pgfpathlineto{\pgfqpoint{2.347523in}{0.728788in}}%
\pgfpathlineto{\pgfqpoint{2.348516in}{0.749138in}}%
\pgfpathlineto{\pgfqpoint{2.348714in}{0.749138in}}%
\pgfpathlineto{\pgfqpoint{2.348913in}{0.728788in}}%
\pgfpathlineto{\pgfqpoint{2.349508in}{0.759313in}}%
\pgfpathlineto{\pgfqpoint{2.349707in}{0.728788in}}%
\pgfpathlineto{\pgfqpoint{2.350898in}{0.728788in}}%
\pgfpathlineto{\pgfqpoint{2.351295in}{0.749138in}}%
\pgfpathlineto{\pgfqpoint{2.351890in}{0.728788in}}%
\pgfpathlineto{\pgfqpoint{2.352089in}{0.728788in}}%
\pgfpathlineto{\pgfqpoint{2.352089in}{0.749138in}}%
\pgfpathlineto{\pgfqpoint{2.353081in}{0.738963in}}%
\pgfpathlineto{\pgfqpoint{2.353280in}{0.738963in}}%
\pgfpathlineto{\pgfqpoint{2.353280in}{0.728788in}}%
\pgfpathlineto{\pgfqpoint{2.354272in}{0.728788in}}%
\pgfpathlineto{\pgfqpoint{2.354471in}{0.728788in}}%
\pgfpathlineto{\pgfqpoint{2.354471in}{0.738963in}}%
\pgfpathlineto{\pgfqpoint{2.355463in}{0.738963in}}%
\pgfpathlineto{\pgfqpoint{2.355662in}{0.738963in}}%
\pgfpathlineto{\pgfqpoint{2.355662in}{0.728788in}}%
\pgfpathlineto{\pgfqpoint{2.356654in}{0.728788in}}%
\pgfpathlineto{\pgfqpoint{2.356853in}{0.728788in}}%
\pgfpathlineto{\pgfqpoint{2.356853in}{0.738963in}}%
\pgfpathlineto{\pgfqpoint{2.357845in}{0.728788in}}%
\pgfpathlineto{\pgfqpoint{2.359830in}{0.728788in}}%
\pgfpathlineto{\pgfqpoint{2.359830in}{0.738963in}}%
\pgfpathlineto{\pgfqpoint{2.360823in}{0.728788in}}%
\pgfpathlineto{\pgfqpoint{2.361418in}{0.728788in}}%
\pgfpathlineto{\pgfqpoint{2.362212in}{0.749138in}}%
\pgfpathlineto{\pgfqpoint{2.362410in}{0.728788in}}%
\pgfpathlineto{\pgfqpoint{2.362807in}{0.728788in}}%
\pgfpathlineto{\pgfqpoint{2.362807in}{0.738963in}}%
\pgfpathlineto{\pgfqpoint{2.363800in}{0.728788in}}%
\pgfpathlineto{\pgfqpoint{2.364395in}{0.728788in}}%
\pgfpathlineto{\pgfqpoint{2.364991in}{0.749138in}}%
\pgfpathlineto{\pgfqpoint{2.365388in}{0.728788in}}%
\pgfpathlineto{\pgfqpoint{2.365785in}{0.728788in}}%
\pgfpathlineto{\pgfqpoint{2.365785in}{0.749138in}}%
\pgfpathlineto{\pgfqpoint{2.366777in}{0.749138in}}%
\pgfpathlineto{\pgfqpoint{2.366976in}{0.749138in}}%
\pgfpathlineto{\pgfqpoint{2.366976in}{0.728788in}}%
\pgfpathlineto{\pgfqpoint{2.367968in}{0.728788in}}%
\pgfpathlineto{\pgfqpoint{2.368762in}{0.728788in}}%
\pgfpathlineto{\pgfqpoint{2.368961in}{0.749138in}}%
\pgfpathlineto{\pgfqpoint{2.369755in}{0.728788in}}%
\pgfpathlineto{\pgfqpoint{2.369953in}{0.728788in}}%
\pgfpathlineto{\pgfqpoint{2.369953in}{0.738963in}}%
\pgfpathlineto{\pgfqpoint{2.370946in}{0.728788in}}%
\pgfpathlineto{\pgfqpoint{2.371343in}{0.728788in}}%
\pgfpathlineto{\pgfqpoint{2.371343in}{0.738963in}}%
\pgfpathlineto{\pgfqpoint{2.372335in}{0.738963in}}%
\pgfpathlineto{\pgfqpoint{2.372534in}{0.738963in}}%
\pgfpathlineto{\pgfqpoint{2.372534in}{0.728788in}}%
\pgfpathlineto{\pgfqpoint{2.373526in}{0.728788in}}%
\pgfpathlineto{\pgfqpoint{2.373725in}{0.728788in}}%
\pgfpathlineto{\pgfqpoint{2.373725in}{0.738963in}}%
\pgfpathlineto{\pgfqpoint{2.374717in}{0.728788in}}%
\pgfpathlineto{\pgfqpoint{2.375114in}{0.728788in}}%
\pgfpathlineto{\pgfqpoint{2.375114in}{0.738963in}}%
\pgfpathlineto{\pgfqpoint{2.376107in}{0.728788in}}%
\pgfpathlineto{\pgfqpoint{2.376305in}{0.728788in}}%
\pgfpathlineto{\pgfqpoint{2.376901in}{0.749138in}}%
\pgfpathlineto{\pgfqpoint{2.377298in}{0.738963in}}%
\pgfpathlineto{\pgfqpoint{2.378092in}{0.738963in}}%
\pgfpathlineto{\pgfqpoint{2.378092in}{0.728788in}}%
\pgfpathlineto{\pgfqpoint{2.378290in}{0.749138in}}%
\pgfpathlineto{\pgfqpoint{2.379084in}{0.749138in}}%
\pgfpathlineto{\pgfqpoint{2.379283in}{0.749138in}}%
\pgfpathlineto{\pgfqpoint{2.379283in}{0.728788in}}%
\pgfpathlineto{\pgfqpoint{2.380275in}{0.728788in}}%
\pgfpathlineto{\pgfqpoint{2.380474in}{0.728788in}}%
\pgfpathlineto{\pgfqpoint{2.380474in}{0.738963in}}%
\pgfpathlineto{\pgfqpoint{2.381466in}{0.728788in}}%
\pgfpathlineto{\pgfqpoint{2.381665in}{0.728788in}}%
\pgfpathlineto{\pgfqpoint{2.381665in}{0.738963in}}%
\pgfpathlineto{\pgfqpoint{2.382657in}{0.738963in}}%
\pgfpathlineto{\pgfqpoint{2.383054in}{0.738963in}}%
\pgfpathlineto{\pgfqpoint{2.383054in}{0.759313in}}%
\pgfpathlineto{\pgfqpoint{2.383451in}{0.728788in}}%
\pgfpathlineto{\pgfqpoint{2.384047in}{0.728788in}}%
\pgfpathlineto{\pgfqpoint{2.384245in}{0.728788in}}%
\pgfpathlineto{\pgfqpoint{2.384245in}{0.738963in}}%
\pgfpathlineto{\pgfqpoint{2.385238in}{0.728788in}}%
\pgfpathlineto{\pgfqpoint{2.386627in}{0.728788in}}%
\pgfpathlineto{\pgfqpoint{2.386627in}{0.738963in}}%
\pgfpathlineto{\pgfqpoint{2.387620in}{0.728788in}}%
\pgfpathlineto{\pgfqpoint{2.389009in}{0.728788in}}%
\pgfpathlineto{\pgfqpoint{2.389009in}{0.759313in}}%
\pgfpathlineto{\pgfqpoint{2.390002in}{0.728788in}}%
\pgfpathlineto{\pgfqpoint{2.390399in}{0.728788in}}%
\pgfpathlineto{\pgfqpoint{2.390399in}{0.738963in}}%
\pgfpathlineto{\pgfqpoint{2.391391in}{0.738963in}}%
\pgfpathlineto{\pgfqpoint{2.391788in}{0.738963in}}%
\pgfpathlineto{\pgfqpoint{2.391788in}{0.728788in}}%
\pgfpathlineto{\pgfqpoint{2.392781in}{0.728788in}}%
\pgfpathlineto{\pgfqpoint{2.392979in}{0.728788in}}%
\pgfpathlineto{\pgfqpoint{2.392979in}{0.738963in}}%
\pgfpathlineto{\pgfqpoint{2.393972in}{0.738963in}}%
\pgfpathlineto{\pgfqpoint{2.394170in}{0.738963in}}%
\pgfpathlineto{\pgfqpoint{2.394170in}{0.728788in}}%
\pgfpathlineto{\pgfqpoint{2.394369in}{0.749138in}}%
\pgfpathlineto{\pgfqpoint{2.395162in}{0.728788in}}%
\pgfpathlineto{\pgfqpoint{2.396155in}{0.728788in}}%
\pgfpathlineto{\pgfqpoint{2.396155in}{0.738963in}}%
\pgfpathlineto{\pgfqpoint{2.397147in}{0.738963in}}%
\pgfpathlineto{\pgfqpoint{2.397544in}{0.738963in}}%
\pgfpathlineto{\pgfqpoint{2.397544in}{0.749138in}}%
\pgfpathlineto{\pgfqpoint{2.397743in}{0.728788in}}%
\pgfpathlineto{\pgfqpoint{2.398537in}{0.728788in}}%
\pgfpathlineto{\pgfqpoint{2.398735in}{0.728788in}}%
\pgfpathlineto{\pgfqpoint{2.398735in}{0.738963in}}%
\pgfpathlineto{\pgfqpoint{2.399728in}{0.728788in}}%
\pgfpathlineto{\pgfqpoint{2.400323in}{0.728788in}}%
\pgfpathlineto{\pgfqpoint{2.400323in}{0.738963in}}%
\pgfpathlineto{\pgfqpoint{2.401316in}{0.728788in}}%
\pgfpathlineto{\pgfqpoint{2.402507in}{0.728788in}}%
\pgfpathlineto{\pgfqpoint{2.402507in}{0.738963in}}%
\pgfpathlineto{\pgfqpoint{2.403499in}{0.738963in}}%
\pgfpathlineto{\pgfqpoint{2.404095in}{0.738963in}}%
\pgfpathlineto{\pgfqpoint{2.404095in}{0.728788in}}%
\pgfpathlineto{\pgfqpoint{2.405087in}{0.728788in}}%
\pgfpathlineto{\pgfqpoint{2.405881in}{0.728788in}}%
\pgfpathlineto{\pgfqpoint{2.405881in}{0.749138in}}%
\pgfpathlineto{\pgfqpoint{2.406874in}{0.728788in}}%
\pgfpathlineto{\pgfqpoint{2.407271in}{0.728788in}}%
\pgfpathlineto{\pgfqpoint{2.407668in}{0.759313in}}%
\pgfpathlineto{\pgfqpoint{2.408263in}{0.738963in}}%
\pgfpathlineto{\pgfqpoint{2.408660in}{0.738963in}}%
\pgfpathlineto{\pgfqpoint{2.408660in}{0.728788in}}%
\pgfpathlineto{\pgfqpoint{2.409653in}{0.728788in}}%
\pgfpathlineto{\pgfqpoint{2.410447in}{0.728788in}}%
\pgfpathlineto{\pgfqpoint{2.411042in}{0.749138in}}%
\pgfpathlineto{\pgfqpoint{2.411439in}{0.728788in}}%
\pgfpathlineto{\pgfqpoint{2.411638in}{0.728788in}}%
\pgfpathlineto{\pgfqpoint{2.411638in}{0.738963in}}%
\pgfpathlineto{\pgfqpoint{2.412630in}{0.728788in}}%
\pgfpathlineto{\pgfqpoint{2.413623in}{0.728788in}}%
\pgfpathlineto{\pgfqpoint{2.413623in}{0.738963in}}%
\pgfpathlineto{\pgfqpoint{2.414615in}{0.728788in}}%
\pgfpathlineto{\pgfqpoint{2.415409in}{0.728788in}}%
\pgfpathlineto{\pgfqpoint{2.415409in}{0.759313in}}%
\pgfpathlineto{\pgfqpoint{2.416402in}{0.728788in}}%
\pgfpathlineto{\pgfqpoint{2.416600in}{0.728788in}}%
\pgfpathlineto{\pgfqpoint{2.416600in}{0.738963in}}%
\pgfpathlineto{\pgfqpoint{2.417593in}{0.728788in}}%
\pgfpathlineto{\pgfqpoint{2.417990in}{0.728788in}}%
\pgfpathlineto{\pgfqpoint{2.417990in}{0.738963in}}%
\pgfpathlineto{\pgfqpoint{2.418982in}{0.728788in}}%
\pgfpathlineto{\pgfqpoint{2.419181in}{0.728788in}}%
\pgfpathlineto{\pgfqpoint{2.419181in}{0.738963in}}%
\pgfpathlineto{\pgfqpoint{2.420173in}{0.728788in}}%
\pgfpathlineto{\pgfqpoint{2.420769in}{0.728788in}}%
\pgfpathlineto{\pgfqpoint{2.420769in}{0.738963in}}%
\pgfpathlineto{\pgfqpoint{2.421761in}{0.728788in}}%
\pgfpathlineto{\pgfqpoint{2.421960in}{0.728788in}}%
\pgfpathlineto{\pgfqpoint{2.421960in}{0.738963in}}%
\pgfpathlineto{\pgfqpoint{2.422952in}{0.728788in}}%
\pgfpathlineto{\pgfqpoint{2.424342in}{0.728788in}}%
\pgfpathlineto{\pgfqpoint{2.424342in}{0.738963in}}%
\pgfpathlineto{\pgfqpoint{2.425334in}{0.728788in}}%
\pgfpathlineto{\pgfqpoint{2.426128in}{0.728788in}}%
\pgfpathlineto{\pgfqpoint{2.427120in}{0.769488in}}%
\pgfpathlineto{\pgfqpoint{2.427319in}{0.769488in}}%
\pgfpathlineto{\pgfqpoint{2.427517in}{0.728788in}}%
\pgfpathlineto{\pgfqpoint{2.428311in}{0.728788in}}%
\pgfpathlineto{\pgfqpoint{2.429105in}{0.728788in}}%
\pgfpathlineto{\pgfqpoint{2.429105in}{0.738963in}}%
\pgfpathlineto{\pgfqpoint{2.430098in}{0.738963in}}%
\pgfpathlineto{\pgfqpoint{2.430296in}{0.738963in}}%
\pgfpathlineto{\pgfqpoint{2.430296in}{0.728788in}}%
\pgfpathlineto{\pgfqpoint{2.431289in}{0.728788in}}%
\pgfpathlineto{\pgfqpoint{2.432083in}{0.728788in}}%
\pgfpathlineto{\pgfqpoint{2.432877in}{0.759313in}}%
\pgfpathlineto{\pgfqpoint{2.433075in}{0.728788in}}%
\pgfpathlineto{\pgfqpoint{2.433671in}{0.728788in}}%
\pgfpathlineto{\pgfqpoint{2.434266in}{0.749138in}}%
\pgfpathlineto{\pgfqpoint{2.434663in}{0.728788in}}%
\pgfpathlineto{\pgfqpoint{2.434862in}{0.728788in}}%
\pgfpathlineto{\pgfqpoint{2.434862in}{0.738963in}}%
\pgfpathlineto{\pgfqpoint{2.435854in}{0.728788in}}%
\pgfpathlineto{\pgfqpoint{2.436847in}{0.728788in}}%
\pgfpathlineto{\pgfqpoint{2.437244in}{0.749138in}}%
\pgfpathlineto{\pgfqpoint{2.437839in}{0.728788in}}%
\pgfpathlineto{\pgfqpoint{2.438038in}{0.728788in}}%
\pgfpathlineto{\pgfqpoint{2.438832in}{0.759313in}}%
\pgfpathlineto{\pgfqpoint{2.439030in}{0.728788in}}%
\pgfpathlineto{\pgfqpoint{2.439427in}{0.728788in}}%
\pgfpathlineto{\pgfqpoint{2.439427in}{0.738963in}}%
\pgfpathlineto{\pgfqpoint{2.440420in}{0.728788in}}%
\pgfpathlineto{\pgfqpoint{2.440618in}{0.728788in}}%
\pgfpathlineto{\pgfqpoint{2.440618in}{0.738963in}}%
\pgfpathlineto{\pgfqpoint{2.441611in}{0.738963in}}%
\pgfpathlineto{\pgfqpoint{2.442206in}{0.738963in}}%
\pgfpathlineto{\pgfqpoint{2.442206in}{0.728788in}}%
\pgfpathlineto{\pgfqpoint{2.443199in}{0.728788in}}%
\pgfpathlineto{\pgfqpoint{2.443993in}{0.728788in}}%
\pgfpathlineto{\pgfqpoint{2.443993in}{0.738963in}}%
\pgfpathlineto{\pgfqpoint{2.444985in}{0.738963in}}%
\pgfpathlineto{\pgfqpoint{2.445184in}{0.738963in}}%
\pgfpathlineto{\pgfqpoint{2.445184in}{0.728788in}}%
\pgfpathlineto{\pgfqpoint{2.445779in}{0.749138in}}%
\pgfpathlineto{\pgfqpoint{2.446176in}{0.738963in}}%
\pgfpathlineto{\pgfqpoint{2.446375in}{0.738963in}}%
\pgfpathlineto{\pgfqpoint{2.446375in}{0.728788in}}%
\pgfpathlineto{\pgfqpoint{2.447367in}{0.728788in}}%
\pgfpathlineto{\pgfqpoint{2.448558in}{0.728788in}}%
\pgfpathlineto{\pgfqpoint{2.448558in}{0.738963in}}%
\pgfpathlineto{\pgfqpoint{2.449551in}{0.728788in}}%
\pgfpathlineto{\pgfqpoint{2.449749in}{0.728788in}}%
\pgfpathlineto{\pgfqpoint{2.449749in}{0.738963in}}%
\pgfpathlineto{\pgfqpoint{2.450742in}{0.728788in}}%
\pgfpathlineto{\pgfqpoint{2.451337in}{0.728788in}}%
\pgfpathlineto{\pgfqpoint{2.451337in}{0.738963in}}%
\pgfpathlineto{\pgfqpoint{2.452330in}{0.728788in}}%
\pgfpathlineto{\pgfqpoint{2.452925in}{0.728788in}}%
\pgfpathlineto{\pgfqpoint{2.453124in}{0.749138in}}%
\pgfpathlineto{\pgfqpoint{2.453918in}{0.728788in}}%
\pgfpathlineto{\pgfqpoint{2.454712in}{0.728788in}}%
\pgfpathlineto{\pgfqpoint{2.454712in}{0.738963in}}%
\pgfpathlineto{\pgfqpoint{2.455704in}{0.738963in}}%
\pgfpathlineto{\pgfqpoint{2.455903in}{0.738963in}}%
\pgfpathlineto{\pgfqpoint{2.455903in}{0.728788in}}%
\pgfpathlineto{\pgfqpoint{2.456895in}{0.728788in}}%
\pgfpathlineto{\pgfqpoint{2.457292in}{0.728788in}}%
\pgfpathlineto{\pgfqpoint{2.457292in}{0.738963in}}%
\pgfpathlineto{\pgfqpoint{2.458285in}{0.728788in}}%
\pgfpathlineto{\pgfqpoint{2.458682in}{0.728788in}}%
\pgfpathlineto{\pgfqpoint{2.458682in}{0.749138in}}%
\pgfpathlineto{\pgfqpoint{2.459674in}{0.728788in}}%
\pgfpathlineto{\pgfqpoint{2.460269in}{0.728788in}}%
\pgfpathlineto{\pgfqpoint{2.460269in}{0.738963in}}%
\pgfpathlineto{\pgfqpoint{2.461262in}{0.738963in}}%
\pgfpathlineto{\pgfqpoint{2.461460in}{0.738963in}}%
\pgfpathlineto{\pgfqpoint{2.461460in}{0.728788in}}%
\pgfpathlineto{\pgfqpoint{2.462453in}{0.728788in}}%
\pgfpathlineto{\pgfqpoint{2.462651in}{0.728788in}}%
\pgfpathlineto{\pgfqpoint{2.462651in}{0.738963in}}%
\pgfpathlineto{\pgfqpoint{2.463644in}{0.728788in}}%
\pgfpathlineto{\pgfqpoint{2.464636in}{0.728788in}}%
\pgfpathlineto{\pgfqpoint{2.464636in}{0.738963in}}%
\pgfpathlineto{\pgfqpoint{2.465629in}{0.738963in}}%
\pgfpathlineto{\pgfqpoint{2.465827in}{0.738963in}}%
\pgfpathlineto{\pgfqpoint{2.465827in}{0.728788in}}%
\pgfpathlineto{\pgfqpoint{2.466820in}{0.728788in}}%
\pgfpathlineto{\pgfqpoint{2.467812in}{0.728788in}}%
\pgfpathlineto{\pgfqpoint{2.467812in}{0.738963in}}%
\pgfpathlineto{\pgfqpoint{2.468805in}{0.728788in}}%
\pgfpathlineto{\pgfqpoint{2.470790in}{0.728788in}}%
\pgfpathlineto{\pgfqpoint{2.470790in}{0.738963in}}%
\pgfpathlineto{\pgfqpoint{2.471782in}{0.728788in}}%
\pgfpathlineto{\pgfqpoint{2.471981in}{0.728788in}}%
\pgfpathlineto{\pgfqpoint{2.471981in}{0.738963in}}%
\pgfpathlineto{\pgfqpoint{2.472973in}{0.728788in}}%
\pgfpathlineto{\pgfqpoint{2.473569in}{0.728788in}}%
\pgfpathlineto{\pgfqpoint{2.473569in}{0.738963in}}%
\pgfpathlineto{\pgfqpoint{2.474561in}{0.738963in}}%
\pgfpathlineto{\pgfqpoint{2.474760in}{0.738963in}}%
\pgfpathlineto{\pgfqpoint{2.474760in}{0.728788in}}%
\pgfpathlineto{\pgfqpoint{2.475157in}{0.749138in}}%
\pgfpathlineto{\pgfqpoint{2.475752in}{0.728788in}}%
\pgfpathlineto{\pgfqpoint{2.476745in}{0.728788in}}%
\pgfpathlineto{\pgfqpoint{2.476745in}{0.749138in}}%
\pgfpathlineto{\pgfqpoint{2.477737in}{0.728788in}}%
\pgfpathlineto{\pgfqpoint{2.478333in}{0.728788in}}%
\pgfpathlineto{\pgfqpoint{2.478333in}{0.738963in}}%
\pgfpathlineto{\pgfqpoint{2.479325in}{0.728788in}}%
\pgfpathlineto{\pgfqpoint{2.479921in}{0.728788in}}%
\pgfpathlineto{\pgfqpoint{2.480318in}{0.749138in}}%
\pgfpathlineto{\pgfqpoint{2.480913in}{0.728788in}}%
\pgfpathlineto{\pgfqpoint{2.481707in}{0.728788in}}%
\pgfpathlineto{\pgfqpoint{2.481707in}{0.738963in}}%
\pgfpathlineto{\pgfqpoint{2.482700in}{0.728788in}}%
\pgfpathlineto{\pgfqpoint{2.486471in}{0.728788in}}%
\pgfpathlineto{\pgfqpoint{2.486471in}{0.749138in}}%
\pgfpathlineto{\pgfqpoint{2.487464in}{0.738963in}}%
\pgfpathlineto{\pgfqpoint{2.487662in}{0.738963in}}%
\pgfpathlineto{\pgfqpoint{2.487662in}{0.728788in}}%
\pgfpathlineto{\pgfqpoint{2.488655in}{0.728788in}}%
\pgfpathlineto{\pgfqpoint{2.490441in}{0.728788in}}%
\pgfpathlineto{\pgfqpoint{2.490441in}{0.738963in}}%
\pgfpathlineto{\pgfqpoint{2.491434in}{0.728788in}}%
\pgfpathlineto{\pgfqpoint{2.492426in}{0.728788in}}%
\pgfpathlineto{\pgfqpoint{2.492426in}{0.738963in}}%
\pgfpathlineto{\pgfqpoint{2.493418in}{0.728788in}}%
\pgfpathlineto{\pgfqpoint{2.494411in}{0.728788in}}%
\pgfpathlineto{\pgfqpoint{2.495403in}{0.749138in}}%
\pgfpathlineto{\pgfqpoint{2.495602in}{0.749138in}}%
\pgfpathlineto{\pgfqpoint{2.495800in}{0.728788in}}%
\pgfpathlineto{\pgfqpoint{2.496594in}{0.728788in}}%
\pgfpathlineto{\pgfqpoint{2.496793in}{0.728788in}}%
\pgfpathlineto{\pgfqpoint{2.496793in}{0.738963in}}%
\pgfpathlineto{\pgfqpoint{2.497785in}{0.728788in}}%
\pgfpathlineto{\pgfqpoint{2.498976in}{0.728788in}}%
\pgfpathlineto{\pgfqpoint{2.498976in}{0.738963in}}%
\pgfpathlineto{\pgfqpoint{2.499969in}{0.728788in}}%
\pgfpathlineto{\pgfqpoint{2.500167in}{0.728788in}}%
\pgfpathlineto{\pgfqpoint{2.500167in}{0.738963in}}%
\pgfpathlineto{\pgfqpoint{2.501160in}{0.728788in}}%
\pgfpathlineto{\pgfqpoint{2.502549in}{0.728788in}}%
\pgfpathlineto{\pgfqpoint{2.502549in}{0.738963in}}%
\pgfpathlineto{\pgfqpoint{2.503542in}{0.728788in}}%
\pgfpathlineto{\pgfqpoint{2.503939in}{0.728788in}}%
\pgfpathlineto{\pgfqpoint{2.504137in}{0.749138in}}%
\pgfpathlineto{\pgfqpoint{2.504931in}{0.749138in}}%
\pgfpathlineto{\pgfqpoint{2.505130in}{0.749138in}}%
\pgfpathlineto{\pgfqpoint{2.505130in}{0.728788in}}%
\pgfpathlineto{\pgfqpoint{2.506122in}{0.728788in}}%
\pgfpathlineto{\pgfqpoint{2.506321in}{0.728788in}}%
\pgfpathlineto{\pgfqpoint{2.506321in}{0.738963in}}%
\pgfpathlineto{\pgfqpoint{2.507313in}{0.728788in}}%
\pgfpathlineto{\pgfqpoint{2.507512in}{0.728788in}}%
\pgfpathlineto{\pgfqpoint{2.507512in}{0.738963in}}%
\pgfpathlineto{\pgfqpoint{2.508504in}{0.738963in}}%
\pgfpathlineto{\pgfqpoint{2.508703in}{0.738963in}}%
\pgfpathlineto{\pgfqpoint{2.508703in}{0.728788in}}%
\pgfpathlineto{\pgfqpoint{2.509695in}{0.738963in}}%
\pgfpathlineto{\pgfqpoint{2.509894in}{0.738963in}}%
\pgfpathlineto{\pgfqpoint{2.509894in}{0.728788in}}%
\pgfpathlineto{\pgfqpoint{2.510886in}{0.728788in}}%
\pgfpathlineto{\pgfqpoint{2.512673in}{0.728788in}}%
\pgfpathlineto{\pgfqpoint{2.512673in}{0.738963in}}%
\pgfpathlineto{\pgfqpoint{2.513665in}{0.728788in}}%
\pgfpathlineto{\pgfqpoint{2.516246in}{0.728788in}}%
\pgfpathlineto{\pgfqpoint{2.516246in}{0.738963in}}%
\pgfpathlineto{\pgfqpoint{2.517238in}{0.738963in}}%
\pgfpathlineto{\pgfqpoint{2.517437in}{0.738963in}}%
\pgfpathlineto{\pgfqpoint{2.517437in}{0.728788in}}%
\pgfpathlineto{\pgfqpoint{2.518429in}{0.728788in}}%
\pgfpathlineto{\pgfqpoint{2.519620in}{0.728788in}}%
\pgfpathlineto{\pgfqpoint{2.519620in}{0.738963in}}%
\pgfpathlineto{\pgfqpoint{2.520613in}{0.728788in}}%
\pgfpathlineto{\pgfqpoint{2.521605in}{0.728788in}}%
\pgfpathlineto{\pgfqpoint{2.521605in}{0.738963in}}%
\pgfpathlineto{\pgfqpoint{2.522598in}{0.728788in}}%
\pgfpathlineto{\pgfqpoint{2.522995in}{0.728788in}}%
\pgfpathlineto{\pgfqpoint{2.522995in}{0.738963in}}%
\pgfpathlineto{\pgfqpoint{2.523987in}{0.738963in}}%
\pgfpathlineto{\pgfqpoint{2.524384in}{0.738963in}}%
\pgfpathlineto{\pgfqpoint{2.524384in}{0.728788in}}%
\pgfpathlineto{\pgfqpoint{2.525376in}{0.738963in}}%
\pgfpathlineto{\pgfqpoint{2.525575in}{0.738963in}}%
\pgfpathlineto{\pgfqpoint{2.525575in}{0.728788in}}%
\pgfpathlineto{\pgfqpoint{2.526567in}{0.728788in}}%
\pgfpathlineto{\pgfqpoint{2.526964in}{0.728788in}}%
\pgfpathlineto{\pgfqpoint{2.526964in}{0.738963in}}%
\pgfpathlineto{\pgfqpoint{2.527957in}{0.728788in}}%
\pgfpathlineto{\pgfqpoint{2.528552in}{0.728788in}}%
\pgfpathlineto{\pgfqpoint{2.528552in}{0.738963in}}%
\pgfpathlineto{\pgfqpoint{2.529545in}{0.728788in}}%
\pgfpathlineto{\pgfqpoint{2.531133in}{0.728788in}}%
\pgfpathlineto{\pgfqpoint{2.531133in}{0.738963in}}%
\pgfpathlineto{\pgfqpoint{2.532125in}{0.728788in}}%
\pgfpathlineto{\pgfqpoint{2.532324in}{0.728788in}}%
\pgfpathlineto{\pgfqpoint{2.532324in}{0.738963in}}%
\pgfpathlineto{\pgfqpoint{2.533316in}{0.738963in}}%
\pgfpathlineto{\pgfqpoint{2.533515in}{0.738963in}}%
\pgfpathlineto{\pgfqpoint{2.533515in}{0.728788in}}%
\pgfpathlineto{\pgfqpoint{2.534507in}{0.728788in}}%
\pgfpathlineto{\pgfqpoint{2.536492in}{0.728788in}}%
\pgfpathlineto{\pgfqpoint{2.536492in}{0.749138in}}%
\pgfpathlineto{\pgfqpoint{2.537485in}{0.728788in}}%
\pgfpathlineto{\pgfqpoint{2.540661in}{0.728788in}}%
\pgfpathlineto{\pgfqpoint{2.540661in}{0.738963in}}%
\pgfpathlineto{\pgfqpoint{2.541653in}{0.738963in}}%
\pgfpathlineto{\pgfqpoint{2.541852in}{0.738963in}}%
\pgfpathlineto{\pgfqpoint{2.541852in}{0.728788in}}%
\pgfpathlineto{\pgfqpoint{2.542844in}{0.728788in}}%
\pgfpathlineto{\pgfqpoint{2.543638in}{0.728788in}}%
\pgfpathlineto{\pgfqpoint{2.543638in}{0.738963in}}%
\pgfpathlineto{\pgfqpoint{2.544631in}{0.728788in}}%
\pgfpathlineto{\pgfqpoint{2.545226in}{0.728788in}}%
\pgfpathlineto{\pgfqpoint{2.545226in}{0.738963in}}%
\pgfpathlineto{\pgfqpoint{2.546219in}{0.728788in}}%
\pgfpathlineto{\pgfqpoint{2.547410in}{0.728788in}}%
\pgfpathlineto{\pgfqpoint{2.547410in}{0.738963in}}%
\pgfpathlineto{\pgfqpoint{2.548402in}{0.728788in}}%
\pgfpathlineto{\pgfqpoint{2.549196in}{0.728788in}}%
\pgfpathlineto{\pgfqpoint{2.549196in}{0.738963in}}%
\pgfpathlineto{\pgfqpoint{2.550189in}{0.728788in}}%
\pgfpathlineto{\pgfqpoint{2.550387in}{0.728788in}}%
\pgfpathlineto{\pgfqpoint{2.550387in}{0.738963in}}%
\pgfpathlineto{\pgfqpoint{2.551380in}{0.728788in}}%
\pgfpathlineto{\pgfqpoint{2.551578in}{0.728788in}}%
\pgfpathlineto{\pgfqpoint{2.552372in}{0.749138in}}%
\pgfpathlineto{\pgfqpoint{2.552571in}{0.738963in}}%
\pgfpathlineto{\pgfqpoint{2.552769in}{0.738963in}}%
\pgfpathlineto{\pgfqpoint{2.552769in}{0.728788in}}%
\pgfpathlineto{\pgfqpoint{2.553762in}{0.728788in}}%
\pgfpathlineto{\pgfqpoint{2.555151in}{0.728788in}}%
\pgfpathlineto{\pgfqpoint{2.555151in}{0.738963in}}%
\pgfpathlineto{\pgfqpoint{2.556143in}{0.728788in}}%
\pgfpathlineto{\pgfqpoint{2.558724in}{0.728788in}}%
\pgfpathlineto{\pgfqpoint{2.558724in}{0.738963in}}%
\pgfpathlineto{\pgfqpoint{2.559716in}{0.738963in}}%
\pgfpathlineto{\pgfqpoint{2.559915in}{0.738963in}}%
\pgfpathlineto{\pgfqpoint{2.559915in}{0.728788in}}%
\pgfpathlineto{\pgfqpoint{2.560907in}{0.728788in}}%
\pgfpathlineto{\pgfqpoint{2.562495in}{0.728788in}}%
\pgfpathlineto{\pgfqpoint{2.562495in}{0.738963in}}%
\pgfpathlineto{\pgfqpoint{2.563488in}{0.738963in}}%
\pgfpathlineto{\pgfqpoint{2.563686in}{0.738963in}}%
\pgfpathlineto{\pgfqpoint{2.563686in}{0.728788in}}%
\pgfpathlineto{\pgfqpoint{2.564679in}{0.738963in}}%
\pgfpathlineto{\pgfqpoint{2.564877in}{0.738963in}}%
\pgfpathlineto{\pgfqpoint{2.564877in}{0.728788in}}%
\pgfpathlineto{\pgfqpoint{2.565870in}{0.728788in}}%
\pgfpathlineto{\pgfqpoint{2.567259in}{0.728788in}}%
\pgfpathlineto{\pgfqpoint{2.567259in}{0.749138in}}%
\pgfpathlineto{\pgfqpoint{2.568252in}{0.728788in}}%
\pgfpathlineto{\pgfqpoint{2.568450in}{0.728788in}}%
\pgfpathlineto{\pgfqpoint{2.568450in}{0.738963in}}%
\pgfpathlineto{\pgfqpoint{2.569443in}{0.728788in}}%
\pgfpathlineto{\pgfqpoint{2.571825in}{0.728788in}}%
\pgfpathlineto{\pgfqpoint{2.571825in}{0.738963in}}%
\pgfpathlineto{\pgfqpoint{2.572817in}{0.728788in}}%
\pgfpathlineto{\pgfqpoint{2.573016in}{0.728788in}}%
\pgfpathlineto{\pgfqpoint{2.573016in}{0.738963in}}%
\pgfpathlineto{\pgfqpoint{2.574008in}{0.728788in}}%
\pgfpathlineto{\pgfqpoint{2.574207in}{0.728788in}}%
\pgfpathlineto{\pgfqpoint{2.574207in}{0.738963in}}%
\pgfpathlineto{\pgfqpoint{2.575199in}{0.728788in}}%
\pgfpathlineto{\pgfqpoint{2.578177in}{0.728788in}}%
\pgfpathlineto{\pgfqpoint{2.578177in}{0.738963in}}%
\pgfpathlineto{\pgfqpoint{2.579169in}{0.728788in}}%
\pgfpathlineto{\pgfqpoint{2.583735in}{0.728788in}}%
\pgfpathlineto{\pgfqpoint{2.583735in}{0.738963in}}%
\pgfpathlineto{\pgfqpoint{2.584727in}{0.728788in}}%
\pgfpathlineto{\pgfqpoint{2.587308in}{0.728788in}}%
\pgfpathlineto{\pgfqpoint{2.588300in}{0.749138in}}%
\pgfpathlineto{\pgfqpoint{2.588498in}{0.749138in}}%
\pgfpathlineto{\pgfqpoint{2.588498in}{0.728788in}}%
\pgfpathlineto{\pgfqpoint{2.589491in}{0.728788in}}%
\pgfpathlineto{\pgfqpoint{2.590086in}{0.728788in}}%
\pgfpathlineto{\pgfqpoint{2.590086in}{0.738963in}}%
\pgfpathlineto{\pgfqpoint{2.591079in}{0.728788in}}%
\pgfpathlineto{\pgfqpoint{2.591674in}{0.728788in}}%
\pgfpathlineto{\pgfqpoint{2.591674in}{0.738963in}}%
\pgfpathlineto{\pgfqpoint{2.592667in}{0.728788in}}%
\pgfpathlineto{\pgfqpoint{2.593262in}{0.728788in}}%
\pgfpathlineto{\pgfqpoint{2.593262in}{0.738963in}}%
\pgfpathlineto{\pgfqpoint{2.594255in}{0.738963in}}%
\pgfpathlineto{\pgfqpoint{2.594453in}{0.738963in}}%
\pgfpathlineto{\pgfqpoint{2.594453in}{0.728788in}}%
\pgfpathlineto{\pgfqpoint{2.595446in}{0.728788in}}%
\pgfpathlineto{\pgfqpoint{2.598026in}{0.728788in}}%
\pgfpathlineto{\pgfqpoint{2.598026in}{0.738963in}}%
\pgfpathlineto{\pgfqpoint{2.599019in}{0.728788in}}%
\pgfpathlineto{\pgfqpoint{2.600210in}{0.728788in}}%
\pgfpathlineto{\pgfqpoint{2.600210in}{0.738963in}}%
\pgfpathlineto{\pgfqpoint{2.601202in}{0.728788in}}%
\pgfpathlineto{\pgfqpoint{2.601996in}{0.728788in}}%
\pgfpathlineto{\pgfqpoint{2.601996in}{0.738963in}}%
\pgfpathlineto{\pgfqpoint{2.602989in}{0.728788in}}%
\pgfpathlineto{\pgfqpoint{2.604180in}{0.728788in}}%
\pgfpathlineto{\pgfqpoint{2.604180in}{0.749138in}}%
\pgfpathlineto{\pgfqpoint{2.605172in}{0.738963in}}%
\pgfpathlineto{\pgfqpoint{2.605371in}{0.738963in}}%
\pgfpathlineto{\pgfqpoint{2.605371in}{0.728788in}}%
\pgfpathlineto{\pgfqpoint{2.606363in}{0.728788in}}%
\pgfpathlineto{\pgfqpoint{2.606959in}{0.728788in}}%
\pgfpathlineto{\pgfqpoint{2.606959in}{0.738963in}}%
\pgfpathlineto{\pgfqpoint{2.607951in}{0.738963in}}%
\pgfpathlineto{\pgfqpoint{2.608348in}{0.738963in}}%
\pgfpathlineto{\pgfqpoint{2.608348in}{0.728788in}}%
\pgfpathlineto{\pgfqpoint{2.609341in}{0.738963in}}%
\pgfpathlineto{\pgfqpoint{2.609539in}{0.738963in}}%
\pgfpathlineto{\pgfqpoint{2.609539in}{0.728788in}}%
\pgfpathlineto{\pgfqpoint{2.610532in}{0.728788in}}%
\pgfpathlineto{\pgfqpoint{2.611723in}{0.728788in}}%
\pgfpathlineto{\pgfqpoint{2.611723in}{0.738963in}}%
\pgfpathlineto{\pgfqpoint{2.612715in}{0.728788in}}%
\pgfpathlineto{\pgfqpoint{2.612914in}{0.728788in}}%
\pgfpathlineto{\pgfqpoint{2.612914in}{0.738963in}}%
\pgfpathlineto{\pgfqpoint{2.613906in}{0.728788in}}%
\pgfpathlineto{\pgfqpoint{2.617082in}{0.728788in}}%
\pgfpathlineto{\pgfqpoint{2.617082in}{0.738963in}}%
\pgfpathlineto{\pgfqpoint{2.618075in}{0.728788in}}%
\pgfpathlineto{\pgfqpoint{2.618472in}{0.728788in}}%
\pgfpathlineto{\pgfqpoint{2.618472in}{0.738963in}}%
\pgfpathlineto{\pgfqpoint{2.619464in}{0.728788in}}%
\pgfpathlineto{\pgfqpoint{2.620060in}{0.728788in}}%
\pgfpathlineto{\pgfqpoint{2.620060in}{0.738963in}}%
\pgfpathlineto{\pgfqpoint{2.621052in}{0.728788in}}%
\pgfpathlineto{\pgfqpoint{2.625419in}{0.728788in}}%
\pgfpathlineto{\pgfqpoint{2.625419in}{0.738963in}}%
\pgfpathlineto{\pgfqpoint{2.626411in}{0.728788in}}%
\pgfpathlineto{\pgfqpoint{2.626808in}{0.728788in}}%
\pgfpathlineto{\pgfqpoint{2.626808in}{0.738963in}}%
\pgfpathlineto{\pgfqpoint{2.627801in}{0.728788in}}%
\pgfpathlineto{\pgfqpoint{2.628793in}{0.728788in}}%
\pgfpathlineto{\pgfqpoint{2.628793in}{0.738963in}}%
\pgfpathlineto{\pgfqpoint{2.629786in}{0.728788in}}%
\pgfpathlineto{\pgfqpoint{2.630778in}{0.728788in}}%
\pgfpathlineto{\pgfqpoint{2.630778in}{0.738963in}}%
\pgfpathlineto{\pgfqpoint{2.631771in}{0.728788in}}%
\pgfpathlineto{\pgfqpoint{2.637527in}{0.728788in}}%
\pgfpathlineto{\pgfqpoint{2.637527in}{0.738963in}}%
\pgfpathlineto{\pgfqpoint{2.638520in}{0.728788in}}%
\pgfpathlineto{\pgfqpoint{2.642093in}{0.728788in}}%
\pgfpathlineto{\pgfqpoint{2.642093in}{0.738963in}}%
\pgfpathlineto{\pgfqpoint{2.643085in}{0.728788in}}%
\pgfpathlineto{\pgfqpoint{2.644276in}{0.728788in}}%
\pgfpathlineto{\pgfqpoint{2.644276in}{0.738963in}}%
\pgfpathlineto{\pgfqpoint{2.645269in}{0.728788in}}%
\pgfpathlineto{\pgfqpoint{2.648048in}{0.728788in}}%
\pgfpathlineto{\pgfqpoint{2.648048in}{0.738963in}}%
\pgfpathlineto{\pgfqpoint{2.649040in}{0.728788in}}%
\pgfpathlineto{\pgfqpoint{2.649239in}{0.728788in}}%
\pgfpathlineto{\pgfqpoint{2.649239in}{0.738963in}}%
\pgfpathlineto{\pgfqpoint{2.650231in}{0.728788in}}%
\pgfpathlineto{\pgfqpoint{2.650827in}{0.728788in}}%
\pgfpathlineto{\pgfqpoint{2.650827in}{0.738963in}}%
\pgfpathlineto{\pgfqpoint{2.651819in}{0.728788in}}%
\pgfpathlineto{\pgfqpoint{2.652812in}{0.728788in}}%
\pgfpathlineto{\pgfqpoint{2.652812in}{0.738963in}}%
\pgfpathlineto{\pgfqpoint{2.653804in}{0.738963in}}%
\pgfpathlineto{\pgfqpoint{2.654002in}{0.738963in}}%
\pgfpathlineto{\pgfqpoint{2.654002in}{0.728788in}}%
\pgfpathlineto{\pgfqpoint{2.654995in}{0.728788in}}%
\pgfpathlineto{\pgfqpoint{2.655392in}{0.728788in}}%
\pgfpathlineto{\pgfqpoint{2.655392in}{0.749138in}}%
\pgfpathlineto{\pgfqpoint{2.656384in}{0.728788in}}%
\pgfpathlineto{\pgfqpoint{2.656781in}{0.728788in}}%
\pgfpathlineto{\pgfqpoint{2.656781in}{0.738963in}}%
\pgfpathlineto{\pgfqpoint{2.657774in}{0.728788in}}%
\pgfpathlineto{\pgfqpoint{2.658766in}{0.728788in}}%
\pgfpathlineto{\pgfqpoint{2.658766in}{0.738963in}}%
\pgfpathlineto{\pgfqpoint{2.659759in}{0.728788in}}%
\pgfpathlineto{\pgfqpoint{2.673257in}{0.728788in}}%
\pgfpathlineto{\pgfqpoint{2.673257in}{0.738963in}}%
\pgfpathlineto{\pgfqpoint{2.674249in}{0.728788in}}%
\pgfpathlineto{\pgfqpoint{2.676433in}{0.728788in}}%
\pgfpathlineto{\pgfqpoint{2.676433in}{0.738963in}}%
\pgfpathlineto{\pgfqpoint{2.677425in}{0.728788in}}%
\pgfpathlineto{\pgfqpoint{2.690923in}{0.728788in}}%
\pgfpathlineto{\pgfqpoint{2.690923in}{0.738963in}}%
\pgfpathlineto{\pgfqpoint{2.691915in}{0.728788in}}%
\pgfpathlineto{\pgfqpoint{2.707994in}{0.728788in}}%
\pgfpathlineto{\pgfqpoint{2.707994in}{0.738963in}}%
\pgfpathlineto{\pgfqpoint{2.708986in}{0.728788in}}%
\pgfpathlineto{\pgfqpoint{2.709780in}{0.728788in}}%
\pgfpathlineto{\pgfqpoint{2.709780in}{0.738963in}}%
\pgfpathlineto{\pgfqpoint{2.710773in}{0.728788in}}%
\pgfpathlineto{\pgfqpoint{2.711368in}{0.728788in}}%
\pgfpathlineto{\pgfqpoint{2.711368in}{0.738963in}}%
\pgfpathlineto{\pgfqpoint{2.712361in}{0.728788in}}%
\pgfpathlineto{\pgfqpoint{2.715735in}{0.728788in}}%
\pgfpathlineto{\pgfqpoint{2.715735in}{0.738963in}}%
\pgfpathlineto{\pgfqpoint{2.716728in}{0.728788in}}%
\pgfpathlineto{\pgfqpoint{2.716926in}{0.728788in}}%
\pgfpathlineto{\pgfqpoint{2.716926in}{0.738963in}}%
\pgfpathlineto{\pgfqpoint{2.717918in}{0.728788in}}%
\pgfpathlineto{\pgfqpoint{2.720499in}{0.728788in}}%
\pgfpathlineto{\pgfqpoint{2.720499in}{0.738963in}}%
\pgfpathlineto{\pgfqpoint{2.721491in}{0.728788in}}%
\pgfpathlineto{\pgfqpoint{2.724270in}{0.728788in}}%
\pgfpathlineto{\pgfqpoint{2.724270in}{0.738963in}}%
\pgfpathlineto{\pgfqpoint{2.725263in}{0.728788in}}%
\pgfpathlineto{\pgfqpoint{2.730622in}{0.728788in}}%
\pgfpathlineto{\pgfqpoint{2.730622in}{0.738963in}}%
\pgfpathlineto{\pgfqpoint{2.731615in}{0.728788in}}%
\pgfpathlineto{\pgfqpoint{2.737371in}{0.728788in}}%
\pgfpathlineto{\pgfqpoint{2.737371in}{0.738963in}}%
\pgfpathlineto{\pgfqpoint{2.738364in}{0.728788in}}%
\pgfpathlineto{\pgfqpoint{2.742929in}{0.728788in}}%
\pgfpathlineto{\pgfqpoint{2.742929in}{0.738963in}}%
\pgfpathlineto{\pgfqpoint{2.743922in}{0.728788in}}%
\pgfpathlineto{\pgfqpoint{2.745907in}{0.728788in}}%
\pgfpathlineto{\pgfqpoint{2.745907in}{0.738963in}}%
\pgfpathlineto{\pgfqpoint{2.746899in}{0.728788in}}%
\pgfpathlineto{\pgfqpoint{2.747693in}{0.728788in}}%
\pgfpathlineto{\pgfqpoint{2.747693in}{0.738963in}}%
\pgfpathlineto{\pgfqpoint{2.748686in}{0.728788in}}%
\pgfpathlineto{\pgfqpoint{2.749281in}{0.728788in}}%
\pgfpathlineto{\pgfqpoint{2.749281in}{0.738963in}}%
\pgfpathlineto{\pgfqpoint{2.750273in}{0.728788in}}%
\pgfpathlineto{\pgfqpoint{2.755037in}{0.728788in}}%
\pgfpathlineto{\pgfqpoint{2.755037in}{0.738963in}}%
\pgfpathlineto{\pgfqpoint{2.756030in}{0.728788in}}%
\pgfpathlineto{\pgfqpoint{2.761786in}{0.728788in}}%
\pgfpathlineto{\pgfqpoint{2.761786in}{0.738963in}}%
\pgfpathlineto{\pgfqpoint{2.762779in}{0.728788in}}%
\pgfpathlineto{\pgfqpoint{2.764764in}{0.728788in}}%
\pgfpathlineto{\pgfqpoint{2.764764in}{0.738963in}}%
\pgfpathlineto{\pgfqpoint{2.765756in}{0.728788in}}%
\pgfpathlineto{\pgfqpoint{2.767543in}{0.728788in}}%
\pgfpathlineto{\pgfqpoint{2.767543in}{0.738963in}}%
\pgfpathlineto{\pgfqpoint{2.768535in}{0.728788in}}%
\pgfpathlineto{\pgfqpoint{2.771116in}{0.728788in}}%
\pgfpathlineto{\pgfqpoint{2.771116in}{0.738963in}}%
\pgfpathlineto{\pgfqpoint{2.772108in}{0.728788in}}%
\pgfpathlineto{\pgfqpoint{2.786598in}{0.728788in}}%
\pgfpathlineto{\pgfqpoint{2.786598in}{0.738963in}}%
\pgfpathlineto{\pgfqpoint{2.787591in}{0.728788in}}%
\pgfpathlineto{\pgfqpoint{2.787789in}{0.728788in}}%
\pgfpathlineto{\pgfqpoint{2.787789in}{0.738963in}}%
\pgfpathlineto{\pgfqpoint{2.788782in}{0.728788in}}%
\pgfpathlineto{\pgfqpoint{2.791759in}{0.728788in}}%
\pgfpathlineto{\pgfqpoint{2.791759in}{0.738963in}}%
\pgfpathlineto{\pgfqpoint{2.792752in}{0.728788in}}%
\pgfpathlineto{\pgfqpoint{2.794538in}{0.728788in}}%
\pgfpathlineto{\pgfqpoint{2.794538in}{0.738963in}}%
\pgfpathlineto{\pgfqpoint{2.795531in}{0.728788in}}%
\pgfpathlineto{\pgfqpoint{2.800096in}{0.728788in}}%
\pgfpathlineto{\pgfqpoint{2.800096in}{0.738963in}}%
\pgfpathlineto{\pgfqpoint{2.801089in}{0.728788in}}%
\pgfpathlineto{\pgfqpoint{2.802081in}{0.728788in}}%
\pgfpathlineto{\pgfqpoint{2.802081in}{0.738963in}}%
\pgfpathlineto{\pgfqpoint{2.803074in}{0.728788in}}%
\pgfpathlineto{\pgfqpoint{2.808036in}{0.728788in}}%
\pgfpathlineto{\pgfqpoint{2.808036in}{0.738963in}}%
\pgfpathlineto{\pgfqpoint{2.809029in}{0.728788in}}%
\pgfpathlineto{\pgfqpoint{2.829275in}{0.728788in}}%
\pgfpathlineto{\pgfqpoint{2.829275in}{0.738963in}}%
\pgfpathlineto{\pgfqpoint{2.830268in}{0.728788in}}%
\pgfpathlineto{\pgfqpoint{2.837215in}{0.728788in}}%
\pgfpathlineto{\pgfqpoint{2.837215in}{0.738963in}}%
\pgfpathlineto{\pgfqpoint{2.838208in}{0.728788in}}%
\pgfpathlineto{\pgfqpoint{2.844758in}{0.728788in}}%
\pgfpathlineto{\pgfqpoint{2.844758in}{0.738963in}}%
\pgfpathlineto{\pgfqpoint{2.845751in}{0.728788in}}%
\pgfpathlineto{\pgfqpoint{2.846743in}{0.728788in}}%
\pgfpathlineto{\pgfqpoint{2.846743in}{0.738963in}}%
\pgfpathlineto{\pgfqpoint{2.847735in}{0.728788in}}%
\pgfpathlineto{\pgfqpoint{2.853095in}{0.728788in}}%
\pgfpathlineto{\pgfqpoint{2.853095in}{0.738963in}}%
\pgfpathlineto{\pgfqpoint{2.854087in}{0.728788in}}%
\pgfpathlineto{\pgfqpoint{2.854484in}{0.728788in}}%
\pgfpathlineto{\pgfqpoint{2.854484in}{0.738963in}}%
\pgfpathlineto{\pgfqpoint{2.855477in}{0.728788in}}%
\pgfpathlineto{\pgfqpoint{2.856668in}{0.728788in}}%
\pgfpathlineto{\pgfqpoint{2.856668in}{0.738963in}}%
\pgfpathlineto{\pgfqpoint{2.857660in}{0.728788in}}%
\pgfpathlineto{\pgfqpoint{2.860241in}{0.728788in}}%
\pgfpathlineto{\pgfqpoint{2.860241in}{0.738963in}}%
\pgfpathlineto{\pgfqpoint{2.861233in}{0.728788in}}%
\pgfpathlineto{\pgfqpoint{2.868181in}{0.728788in}}%
\pgfpathlineto{\pgfqpoint{2.868181in}{0.738963in}}%
\pgfpathlineto{\pgfqpoint{2.869173in}{0.728788in}}%
\pgfpathlineto{\pgfqpoint{2.870166in}{0.728788in}}%
\pgfpathlineto{\pgfqpoint{2.870166in}{0.738963in}}%
\pgfpathlineto{\pgfqpoint{2.871158in}{0.728788in}}%
\pgfpathlineto{\pgfqpoint{2.873143in}{0.728788in}}%
\pgfpathlineto{\pgfqpoint{2.873143in}{0.738963in}}%
\pgfpathlineto{\pgfqpoint{2.874136in}{0.728788in}}%
\pgfpathlineto{\pgfqpoint{2.876121in}{0.728788in}}%
\pgfpathlineto{\pgfqpoint{2.876121in}{0.738963in}}%
\pgfpathlineto{\pgfqpoint{2.877113in}{0.728788in}}%
\pgfpathlineto{\pgfqpoint{2.877510in}{0.728788in}}%
\pgfpathlineto{\pgfqpoint{2.877510in}{0.738963in}}%
\pgfpathlineto{\pgfqpoint{2.878503in}{0.728788in}}%
\pgfpathlineto{\pgfqpoint{2.893985in}{0.728788in}}%
\pgfpathlineto{\pgfqpoint{2.893985in}{0.738963in}}%
\pgfpathlineto{\pgfqpoint{2.894978in}{0.728788in}}%
\pgfpathlineto{\pgfqpoint{2.899345in}{0.728788in}}%
\pgfpathlineto{\pgfqpoint{2.899345in}{0.738963in}}%
\pgfpathlineto{\pgfqpoint{2.900337in}{0.728788in}}%
\pgfpathlineto{\pgfqpoint{2.910659in}{0.728788in}}%
\pgfpathlineto{\pgfqpoint{2.910659in}{0.738963in}}%
\pgfpathlineto{\pgfqpoint{2.911651in}{0.728788in}}%
\pgfpathlineto{\pgfqpoint{2.927333in}{0.728788in}}%
\pgfpathlineto{\pgfqpoint{2.927333in}{0.738963in}}%
\pgfpathlineto{\pgfqpoint{2.928325in}{0.728788in}}%
\pgfpathlineto{\pgfqpoint{2.930310in}{0.728788in}}%
\pgfpathlineto{\pgfqpoint{2.930310in}{0.738963in}}%
\pgfpathlineto{\pgfqpoint{2.931303in}{0.728788in}}%
\pgfpathlineto{\pgfqpoint{2.932494in}{0.728788in}}%
\pgfpathlineto{\pgfqpoint{2.932494in}{0.738963in}}%
\pgfpathlineto{\pgfqpoint{2.933486in}{0.728788in}}%
\pgfpathlineto{\pgfqpoint{2.938250in}{0.728788in}}%
\pgfpathlineto{\pgfqpoint{2.938250in}{0.738963in}}%
\pgfpathlineto{\pgfqpoint{2.939243in}{0.728788in}}%
\pgfpathlineto{\pgfqpoint{2.947976in}{0.728788in}}%
\pgfpathlineto{\pgfqpoint{2.947976in}{0.738963in}}%
\pgfpathlineto{\pgfqpoint{2.948969in}{0.728788in}}%
\pgfpathlineto{\pgfqpoint{2.949763in}{0.728788in}}%
\pgfpathlineto{\pgfqpoint{2.949763in}{0.738963in}}%
\pgfpathlineto{\pgfqpoint{2.950755in}{0.728788in}}%
\pgfpathlineto{\pgfqpoint{2.955916in}{0.728788in}}%
\pgfpathlineto{\pgfqpoint{2.955916in}{0.738963in}}%
\pgfpathlineto{\pgfqpoint{2.956909in}{0.728788in}}%
\pgfpathlineto{\pgfqpoint{2.961871in}{0.728788in}}%
\pgfpathlineto{\pgfqpoint{2.961871in}{0.738963in}}%
\pgfpathlineto{\pgfqpoint{2.962864in}{0.728788in}}%
\pgfpathlineto{\pgfqpoint{2.973384in}{0.728788in}}%
\pgfpathlineto{\pgfqpoint{2.973384in}{0.738963in}}%
\pgfpathlineto{\pgfqpoint{2.974377in}{0.728788in}}%
\pgfpathlineto{\pgfqpoint{3.004747in}{0.728788in}}%
\pgfpathlineto{\pgfqpoint{3.004747in}{0.738963in}}%
\pgfpathlineto{\pgfqpoint{3.005739in}{0.728788in}}%
\pgfpathlineto{\pgfqpoint{3.027375in}{0.728788in}}%
\pgfpathlineto{\pgfqpoint{3.027375in}{0.738963in}}%
\pgfpathlineto{\pgfqpoint{3.028368in}{0.728788in}}%
\pgfpathlineto{\pgfqpoint{3.029559in}{0.728788in}}%
\pgfpathlineto{\pgfqpoint{3.029559in}{0.738963in}}%
\pgfpathlineto{\pgfqpoint{3.030551in}{0.728788in}}%
\pgfpathlineto{\pgfqpoint{3.042659in}{0.728788in}}%
\pgfpathlineto{\pgfqpoint{3.042659in}{0.738963in}}%
\pgfpathlineto{\pgfqpoint{3.043652in}{0.728788in}}%
\pgfpathlineto{\pgfqpoint{3.054371in}{0.728788in}}%
\pgfpathlineto{\pgfqpoint{3.054371in}{0.738963in}}%
\pgfpathlineto{\pgfqpoint{3.055363in}{0.728788in}}%
\pgfpathlineto{\pgfqpoint{3.058738in}{0.728788in}}%
\pgfpathlineto{\pgfqpoint{3.058738in}{0.738963in}}%
\pgfpathlineto{\pgfqpoint{3.059730in}{0.728788in}}%
\pgfpathlineto{\pgfqpoint{3.061120in}{0.728788in}}%
\pgfpathlineto{\pgfqpoint{3.061120in}{0.738963in}}%
\pgfpathlineto{\pgfqpoint{3.062112in}{0.728788in}}%
\pgfpathlineto{\pgfqpoint{3.088909in}{0.728788in}}%
\pgfpathlineto{\pgfqpoint{3.088909in}{0.738963in}}%
\pgfpathlineto{\pgfqpoint{3.089902in}{0.728788in}}%
\pgfpathlineto{\pgfqpoint{3.090299in}{0.728788in}}%
\pgfpathlineto{\pgfqpoint{3.090299in}{0.738963in}}%
\pgfpathlineto{\pgfqpoint{3.091291in}{0.728788in}}%
\pgfpathlineto{\pgfqpoint{3.103995in}{0.728788in}}%
\pgfpathlineto{\pgfqpoint{3.103995in}{0.738963in}}%
\pgfpathlineto{\pgfqpoint{3.104988in}{0.728788in}}%
\pgfpathlineto{\pgfqpoint{3.114714in}{0.728788in}}%
\pgfpathlineto{\pgfqpoint{3.114714in}{0.738963in}}%
\pgfpathlineto{\pgfqpoint{3.115706in}{0.728788in}}%
\pgfpathlineto{\pgfqpoint{3.140518in}{0.728788in}}%
\pgfpathlineto{\pgfqpoint{3.140518in}{0.738963in}}%
\pgfpathlineto{\pgfqpoint{3.141511in}{0.728788in}}%
\pgfpathlineto{\pgfqpoint{3.151436in}{0.728788in}}%
\pgfpathlineto{\pgfqpoint{3.151436in}{0.749138in}}%
\pgfpathlineto{\pgfqpoint{3.152428in}{0.728788in}}%
\pgfpathlineto{\pgfqpoint{3.156597in}{0.728788in}}%
\pgfpathlineto{\pgfqpoint{3.156597in}{0.738963in}}%
\pgfpathlineto{\pgfqpoint{3.157589in}{0.728788in}}%
\pgfpathlineto{\pgfqpoint{3.185776in}{0.728788in}}%
\pgfpathlineto{\pgfqpoint{3.185776in}{0.738963in}}%
\pgfpathlineto{\pgfqpoint{3.186768in}{0.728788in}}%
\pgfpathlineto{\pgfqpoint{3.202449in}{0.728788in}}%
\pgfpathlineto{\pgfqpoint{3.202449in}{0.738963in}}%
\pgfpathlineto{\pgfqpoint{3.203442in}{0.728788in}}%
\pgfpathlineto{\pgfqpoint{3.208603in}{0.728788in}}%
\pgfpathlineto{\pgfqpoint{3.208603in}{0.738963in}}%
\pgfpathlineto{\pgfqpoint{3.209595in}{0.728788in}}%
\pgfpathlineto{\pgfqpoint{3.266763in}{0.728788in}}%
\pgfpathlineto{\pgfqpoint{3.266763in}{0.738963in}}%
\pgfpathlineto{\pgfqpoint{3.267755in}{0.728788in}}%
\pgfpathlineto{\pgfqpoint{3.914259in}{0.728788in}}%
\pgfpathlineto{\pgfqpoint{3.914259in}{0.728788in}}%
\pgfusepath{stroke}%
\end{pgfscope}%
\begin{pgfscope}%
\pgfpathrectangle{\pgfqpoint{0.752778in}{0.582778in}}{\pgfqpoint{4.048611in}{3.212222in}}%
\pgfusepath{clip}%
\pgfsetrectcap%
\pgfsetroundjoin%
\pgfsetlinewidth{1.505625pt}%
\definecolor{currentstroke}{rgb}{1.000000,0.498039,0.054902}%
\pgfsetstrokecolor{currentstroke}%
\pgfsetdash{}{0pt}%
\pgfpathmoveto{\pgfqpoint{0.957921in}{0.749425in}}%
\pgfpathlineto{\pgfqpoint{0.958164in}{0.733947in}}%
\pgfpathlineto{\pgfqpoint{0.958896in}{0.733947in}}%
\pgfpathlineto{\pgfqpoint{0.959140in}{0.733947in}}%
\pgfpathlineto{\pgfqpoint{0.959140in}{0.770063in}}%
\pgfpathlineto{\pgfqpoint{0.960116in}{0.744266in}}%
\pgfpathlineto{\pgfqpoint{0.960360in}{0.744266in}}%
\pgfpathlineto{\pgfqpoint{0.960604in}{0.764903in}}%
\pgfpathlineto{\pgfqpoint{0.961092in}{0.733947in}}%
\pgfpathlineto{\pgfqpoint{0.961336in}{0.744266in}}%
\pgfpathlineto{\pgfqpoint{0.961580in}{0.744266in}}%
\pgfpathlineto{\pgfqpoint{0.961824in}{0.764903in}}%
\pgfpathlineto{\pgfqpoint{0.962068in}{0.733947in}}%
\pgfpathlineto{\pgfqpoint{0.962556in}{0.749425in}}%
\pgfpathlineto{\pgfqpoint{0.962800in}{0.749425in}}%
\pgfpathlineto{\pgfqpoint{0.963044in}{0.739107in}}%
\pgfpathlineto{\pgfqpoint{0.963776in}{0.744266in}}%
\pgfpathlineto{\pgfqpoint{0.964020in}{0.744266in}}%
\pgfpathlineto{\pgfqpoint{0.964264in}{0.770063in}}%
\pgfpathlineto{\pgfqpoint{0.964995in}{0.733947in}}%
\pgfpathlineto{\pgfqpoint{0.965483in}{0.733947in}}%
\pgfpathlineto{\pgfqpoint{0.966215in}{0.749425in}}%
\pgfpathlineto{\pgfqpoint{0.966459in}{0.744266in}}%
\pgfpathlineto{\pgfqpoint{0.966947in}{0.744266in}}%
\pgfpathlineto{\pgfqpoint{0.967435in}{0.759744in}}%
\pgfpathlineto{\pgfqpoint{0.967923in}{0.754585in}}%
\pgfpathlineto{\pgfqpoint{0.968167in}{0.754585in}}%
\pgfpathlineto{\pgfqpoint{0.968655in}{0.733947in}}%
\pgfpathlineto{\pgfqpoint{0.969143in}{0.744266in}}%
\pgfpathlineto{\pgfqpoint{0.969875in}{0.744266in}}%
\pgfpathlineto{\pgfqpoint{0.970607in}{0.733947in}}%
\pgfpathlineto{\pgfqpoint{0.970119in}{0.759744in}}%
\pgfpathlineto{\pgfqpoint{0.970851in}{0.733947in}}%
\pgfpathlineto{\pgfqpoint{0.971095in}{0.733947in}}%
\pgfpathlineto{\pgfqpoint{0.971095in}{0.754585in}}%
\pgfpathlineto{\pgfqpoint{0.972070in}{0.739107in}}%
\pgfpathlineto{\pgfqpoint{0.972314in}{0.739107in}}%
\pgfpathlineto{\pgfqpoint{0.972314in}{0.749425in}}%
\pgfpathlineto{\pgfqpoint{0.973290in}{0.728788in}}%
\pgfpathlineto{\pgfqpoint{0.973534in}{0.728788in}}%
\pgfpathlineto{\pgfqpoint{0.974022in}{0.759744in}}%
\pgfpathlineto{\pgfqpoint{0.974510in}{0.744266in}}%
\pgfpathlineto{\pgfqpoint{0.974754in}{0.744266in}}%
\pgfpathlineto{\pgfqpoint{0.974998in}{0.739107in}}%
\pgfpathlineto{\pgfqpoint{0.975730in}{0.770063in}}%
\pgfpathlineto{\pgfqpoint{0.975974in}{0.770063in}}%
\pgfpathlineto{\pgfqpoint{0.976218in}{0.733947in}}%
\pgfpathlineto{\pgfqpoint{0.976950in}{0.739107in}}%
\pgfpathlineto{\pgfqpoint{0.977194in}{0.739107in}}%
\pgfpathlineto{\pgfqpoint{0.977925in}{0.759744in}}%
\pgfpathlineto{\pgfqpoint{0.978169in}{0.749425in}}%
\pgfpathlineto{\pgfqpoint{0.978657in}{0.749425in}}%
\pgfpathlineto{\pgfqpoint{0.979145in}{0.733947in}}%
\pgfpathlineto{\pgfqpoint{0.979389in}{0.754585in}}%
\pgfpathlineto{\pgfqpoint{0.979633in}{0.754585in}}%
\pgfpathlineto{\pgfqpoint{0.980121in}{0.754585in}}%
\pgfpathlineto{\pgfqpoint{0.980365in}{0.733947in}}%
\pgfpathlineto{\pgfqpoint{0.980609in}{0.759744in}}%
\pgfpathlineto{\pgfqpoint{0.981097in}{0.749425in}}%
\pgfpathlineto{\pgfqpoint{0.981341in}{0.749425in}}%
\pgfpathlineto{\pgfqpoint{0.981341in}{0.764903in}}%
\pgfpathlineto{\pgfqpoint{0.982073in}{0.739107in}}%
\pgfpathlineto{\pgfqpoint{0.982317in}{0.749425in}}%
\pgfpathlineto{\pgfqpoint{0.982561in}{0.749425in}}%
\pgfpathlineto{\pgfqpoint{0.982561in}{0.739107in}}%
\pgfpathlineto{\pgfqpoint{0.983537in}{0.759744in}}%
\pgfpathlineto{\pgfqpoint{0.983781in}{0.759744in}}%
\pgfpathlineto{\pgfqpoint{0.984025in}{0.764903in}}%
\pgfpathlineto{\pgfqpoint{0.984756in}{0.744266in}}%
\pgfpathlineto{\pgfqpoint{0.985000in}{0.744266in}}%
\pgfpathlineto{\pgfqpoint{0.985000in}{0.739107in}}%
\pgfpathlineto{\pgfqpoint{0.985732in}{0.790700in}}%
\pgfpathlineto{\pgfqpoint{0.985976in}{0.775222in}}%
\pgfpathlineto{\pgfqpoint{0.986220in}{0.775222in}}%
\pgfpathlineto{\pgfqpoint{0.986464in}{0.749425in}}%
\pgfpathlineto{\pgfqpoint{0.987196in}{0.754585in}}%
\pgfpathlineto{\pgfqpoint{0.987684in}{0.754585in}}%
\pgfpathlineto{\pgfqpoint{0.988172in}{0.733947in}}%
\pgfpathlineto{\pgfqpoint{0.988416in}{0.764903in}}%
\pgfpathlineto{\pgfqpoint{0.988660in}{0.759744in}}%
\pgfpathlineto{\pgfqpoint{0.988904in}{0.759744in}}%
\pgfpathlineto{\pgfqpoint{0.989636in}{0.790700in}}%
\pgfpathlineto{\pgfqpoint{0.989392in}{0.749425in}}%
\pgfpathlineto{\pgfqpoint{0.989880in}{0.759744in}}%
\pgfpathlineto{\pgfqpoint{0.990124in}{0.759744in}}%
\pgfpathlineto{\pgfqpoint{0.990368in}{0.739107in}}%
\pgfpathlineto{\pgfqpoint{0.991099in}{0.770063in}}%
\pgfpathlineto{\pgfqpoint{0.991343in}{0.770063in}}%
\pgfpathlineto{\pgfqpoint{0.991587in}{0.749425in}}%
\pgfpathlineto{\pgfqpoint{0.992319in}{0.780382in}}%
\pgfpathlineto{\pgfqpoint{0.992563in}{0.780382in}}%
\pgfpathlineto{\pgfqpoint{0.993295in}{0.733947in}}%
\pgfpathlineto{\pgfqpoint{0.993539in}{0.764903in}}%
\pgfpathlineto{\pgfqpoint{0.993783in}{0.764903in}}%
\pgfpathlineto{\pgfqpoint{0.994027in}{0.749425in}}%
\pgfpathlineto{\pgfqpoint{0.994271in}{0.770063in}}%
\pgfpathlineto{\pgfqpoint{0.994759in}{0.754585in}}%
\pgfpathlineto{\pgfqpoint{0.995003in}{0.754585in}}%
\pgfpathlineto{\pgfqpoint{0.995491in}{0.744266in}}%
\pgfpathlineto{\pgfqpoint{0.995247in}{0.775222in}}%
\pgfpathlineto{\pgfqpoint{0.995979in}{0.759744in}}%
\pgfpathlineto{\pgfqpoint{0.996223in}{0.759744in}}%
\pgfpathlineto{\pgfqpoint{0.996223in}{0.785541in}}%
\pgfpathlineto{\pgfqpoint{0.996467in}{0.744266in}}%
\pgfpathlineto{\pgfqpoint{0.997199in}{0.749425in}}%
\pgfpathlineto{\pgfqpoint{0.997442in}{0.749425in}}%
\pgfpathlineto{\pgfqpoint{0.997442in}{0.785541in}}%
\pgfpathlineto{\pgfqpoint{0.998418in}{0.770063in}}%
\pgfpathlineto{\pgfqpoint{0.998662in}{0.770063in}}%
\pgfpathlineto{\pgfqpoint{0.998662in}{0.739107in}}%
\pgfpathlineto{\pgfqpoint{0.998906in}{0.780382in}}%
\pgfpathlineto{\pgfqpoint{0.999638in}{0.775222in}}%
\pgfpathlineto{\pgfqpoint{0.999882in}{0.775222in}}%
\pgfpathlineto{\pgfqpoint{0.999882in}{0.790700in}}%
\pgfpathlineto{\pgfqpoint{1.000370in}{0.754585in}}%
\pgfpathlineto{\pgfqpoint{1.000858in}{0.754585in}}%
\pgfpathlineto{\pgfqpoint{1.001102in}{0.754585in}}%
\pgfpathlineto{\pgfqpoint{1.001590in}{0.801019in}}%
\pgfpathlineto{\pgfqpoint{1.002078in}{0.754585in}}%
\pgfpathlineto{\pgfqpoint{1.002322in}{0.754585in}}%
\pgfpathlineto{\pgfqpoint{1.002566in}{0.775222in}}%
\pgfpathlineto{\pgfqpoint{1.003298in}{0.744266in}}%
\pgfpathlineto{\pgfqpoint{1.003542in}{0.744266in}}%
\pgfpathlineto{\pgfqpoint{1.003786in}{0.780382in}}%
\pgfpathlineto{\pgfqpoint{1.004517in}{0.759744in}}%
\pgfpathlineto{\pgfqpoint{1.004761in}{0.759744in}}%
\pgfpathlineto{\pgfqpoint{1.004761in}{0.790700in}}%
\pgfpathlineto{\pgfqpoint{1.005737in}{0.764903in}}%
\pgfpathlineto{\pgfqpoint{1.005981in}{0.764903in}}%
\pgfpathlineto{\pgfqpoint{1.006713in}{0.801019in}}%
\pgfpathlineto{\pgfqpoint{1.006957in}{0.764903in}}%
\pgfpathlineto{\pgfqpoint{1.007201in}{0.764903in}}%
\pgfpathlineto{\pgfqpoint{1.007933in}{0.790700in}}%
\pgfpathlineto{\pgfqpoint{1.008177in}{0.780382in}}%
\pgfpathlineto{\pgfqpoint{1.008665in}{0.780382in}}%
\pgfpathlineto{\pgfqpoint{1.008665in}{0.759744in}}%
\pgfpathlineto{\pgfqpoint{1.009153in}{0.801019in}}%
\pgfpathlineto{\pgfqpoint{1.009641in}{0.801019in}}%
\pgfpathlineto{\pgfqpoint{1.009885in}{0.801019in}}%
\pgfpathlineto{\pgfqpoint{1.010860in}{0.770063in}}%
\pgfpathlineto{\pgfqpoint{1.011104in}{0.770063in}}%
\pgfpathlineto{\pgfqpoint{1.011104in}{0.764903in}}%
\pgfpathlineto{\pgfqpoint{1.012080in}{0.837135in}}%
\pgfpathlineto{\pgfqpoint{1.012324in}{0.837135in}}%
\pgfpathlineto{\pgfqpoint{1.012324in}{0.775222in}}%
\pgfpathlineto{\pgfqpoint{1.013300in}{0.775222in}}%
\pgfpathlineto{\pgfqpoint{1.013544in}{0.775222in}}%
\pgfpathlineto{\pgfqpoint{1.014032in}{0.816497in}}%
\pgfpathlineto{\pgfqpoint{1.014520in}{0.775222in}}%
\pgfpathlineto{\pgfqpoint{1.014764in}{0.775222in}}%
\pgfpathlineto{\pgfqpoint{1.015008in}{0.806178in}}%
\pgfpathlineto{\pgfqpoint{1.015740in}{0.780382in}}%
\pgfpathlineto{\pgfqpoint{1.015984in}{0.780382in}}%
\pgfpathlineto{\pgfqpoint{1.015984in}{0.770063in}}%
\pgfpathlineto{\pgfqpoint{1.016716in}{0.816497in}}%
\pgfpathlineto{\pgfqpoint{1.016960in}{0.795860in}}%
\pgfpathlineto{\pgfqpoint{1.017203in}{0.795860in}}%
\pgfpathlineto{\pgfqpoint{1.017203in}{0.806178in}}%
\pgfpathlineto{\pgfqpoint{1.017691in}{0.790700in}}%
\pgfpathlineto{\pgfqpoint{1.018179in}{0.801019in}}%
\pgfpathlineto{\pgfqpoint{1.018423in}{0.801019in}}%
\pgfpathlineto{\pgfqpoint{1.018667in}{0.775222in}}%
\pgfpathlineto{\pgfqpoint{1.019399in}{0.821656in}}%
\pgfpathlineto{\pgfqpoint{1.019643in}{0.821656in}}%
\pgfpathlineto{\pgfqpoint{1.020375in}{0.857772in}}%
\pgfpathlineto{\pgfqpoint{1.020619in}{0.821656in}}%
\pgfpathlineto{\pgfqpoint{1.020863in}{0.821656in}}%
\pgfpathlineto{\pgfqpoint{1.021107in}{0.795860in}}%
\pgfpathlineto{\pgfqpoint{1.021595in}{0.837135in}}%
\pgfpathlineto{\pgfqpoint{1.021839in}{0.816497in}}%
\pgfpathlineto{\pgfqpoint{1.022083in}{0.816497in}}%
\pgfpathlineto{\pgfqpoint{1.022083in}{0.801019in}}%
\pgfpathlineto{\pgfqpoint{1.023059in}{0.826816in}}%
\pgfpathlineto{\pgfqpoint{1.023303in}{0.826816in}}%
\pgfpathlineto{\pgfqpoint{1.023303in}{0.842294in}}%
\pgfpathlineto{\pgfqpoint{1.023546in}{0.770063in}}%
\pgfpathlineto{\pgfqpoint{1.024278in}{0.826816in}}%
\pgfpathlineto{\pgfqpoint{1.024522in}{0.826816in}}%
\pgfpathlineto{\pgfqpoint{1.024766in}{0.770063in}}%
\pgfpathlineto{\pgfqpoint{1.025498in}{0.801019in}}%
\pgfpathlineto{\pgfqpoint{1.026230in}{0.801019in}}%
\pgfpathlineto{\pgfqpoint{1.026474in}{0.857772in}}%
\pgfpathlineto{\pgfqpoint{1.027206in}{0.816497in}}%
\pgfpathlineto{\pgfqpoint{1.027450in}{0.816497in}}%
\pgfpathlineto{\pgfqpoint{1.027694in}{0.857772in}}%
\pgfpathlineto{\pgfqpoint{1.028182in}{0.775222in}}%
\pgfpathlineto{\pgfqpoint{1.028426in}{0.816497in}}%
\pgfpathlineto{\pgfqpoint{1.028670in}{0.816497in}}%
\pgfpathlineto{\pgfqpoint{1.028670in}{0.806178in}}%
\pgfpathlineto{\pgfqpoint{1.028914in}{0.868091in}}%
\pgfpathlineto{\pgfqpoint{1.029646in}{0.837135in}}%
\pgfpathlineto{\pgfqpoint{1.029890in}{0.837135in}}%
\pgfpathlineto{\pgfqpoint{1.030377in}{0.868091in}}%
\pgfpathlineto{\pgfqpoint{1.030133in}{0.831975in}}%
\pgfpathlineto{\pgfqpoint{1.030865in}{0.847453in}}%
\pgfpathlineto{\pgfqpoint{1.031109in}{0.847453in}}%
\pgfpathlineto{\pgfqpoint{1.031597in}{0.785541in}}%
\pgfpathlineto{\pgfqpoint{1.032085in}{0.873250in}}%
\pgfpathlineto{\pgfqpoint{1.032329in}{0.873250in}}%
\pgfpathlineto{\pgfqpoint{1.032329in}{0.780382in}}%
\pgfpathlineto{\pgfqpoint{1.033305in}{0.806178in}}%
\pgfpathlineto{\pgfqpoint{1.033549in}{0.806178in}}%
\pgfpathlineto{\pgfqpoint{1.034037in}{0.847453in}}%
\pgfpathlineto{\pgfqpoint{1.034525in}{0.831975in}}%
\pgfpathlineto{\pgfqpoint{1.034769in}{0.831975in}}%
\pgfpathlineto{\pgfqpoint{1.034769in}{0.811338in}}%
\pgfpathlineto{\pgfqpoint{1.035501in}{0.842294in}}%
\pgfpathlineto{\pgfqpoint{1.035745in}{0.811338in}}%
\pgfpathlineto{\pgfqpoint{1.035989in}{0.811338in}}%
\pgfpathlineto{\pgfqpoint{1.036964in}{0.868091in}}%
\pgfpathlineto{\pgfqpoint{1.037208in}{0.868091in}}%
\pgfpathlineto{\pgfqpoint{1.037452in}{0.904206in}}%
\pgfpathlineto{\pgfqpoint{1.038184in}{0.826816in}}%
\pgfpathlineto{\pgfqpoint{1.038428in}{0.826816in}}%
\pgfpathlineto{\pgfqpoint{1.039160in}{0.873250in}}%
\pgfpathlineto{\pgfqpoint{1.039404in}{0.816497in}}%
\pgfpathlineto{\pgfqpoint{1.039648in}{0.816497in}}%
\pgfpathlineto{\pgfqpoint{1.040136in}{0.878410in}}%
\pgfpathlineto{\pgfqpoint{1.040624in}{0.816497in}}%
\pgfpathlineto{\pgfqpoint{1.040868in}{0.816497in}}%
\pgfpathlineto{\pgfqpoint{1.041600in}{0.873250in}}%
\pgfpathlineto{\pgfqpoint{1.041844in}{0.831975in}}%
\pgfpathlineto{\pgfqpoint{1.042088in}{0.831975in}}%
\pgfpathlineto{\pgfqpoint{1.042088in}{0.878410in}}%
\pgfpathlineto{\pgfqpoint{1.042576in}{0.826816in}}%
\pgfpathlineto{\pgfqpoint{1.043064in}{0.857772in}}%
\pgfpathlineto{\pgfqpoint{1.043307in}{0.857772in}}%
\pgfpathlineto{\pgfqpoint{1.044039in}{0.888728in}}%
\pgfpathlineto{\pgfqpoint{1.044283in}{0.868091in}}%
\pgfpathlineto{\pgfqpoint{1.044527in}{0.868091in}}%
\pgfpathlineto{\pgfqpoint{1.044527in}{0.852613in}}%
\pgfpathlineto{\pgfqpoint{1.045015in}{0.883569in}}%
\pgfpathlineto{\pgfqpoint{1.045503in}{0.868091in}}%
\pgfpathlineto{\pgfqpoint{1.045991in}{0.868091in}}%
\pgfpathlineto{\pgfqpoint{1.045991in}{0.852613in}}%
\pgfpathlineto{\pgfqpoint{1.046235in}{0.924844in}}%
\pgfpathlineto{\pgfqpoint{1.046967in}{0.878410in}}%
\pgfpathlineto{\pgfqpoint{1.047211in}{0.878410in}}%
\pgfpathlineto{\pgfqpoint{1.047943in}{0.945481in}}%
\pgfpathlineto{\pgfqpoint{1.048187in}{0.847453in}}%
\pgfpathlineto{\pgfqpoint{1.048431in}{0.847453in}}%
\pgfpathlineto{\pgfqpoint{1.049163in}{0.899047in}}%
\pgfpathlineto{\pgfqpoint{1.049407in}{0.857772in}}%
\pgfpathlineto{\pgfqpoint{1.049651in}{0.857772in}}%
\pgfpathlineto{\pgfqpoint{1.050382in}{0.919684in}}%
\pgfpathlineto{\pgfqpoint{1.050626in}{0.888728in}}%
\pgfpathlineto{\pgfqpoint{1.050870in}{0.888728in}}%
\pgfpathlineto{\pgfqpoint{1.051358in}{0.919684in}}%
\pgfpathlineto{\pgfqpoint{1.051602in}{0.831975in}}%
\pgfpathlineto{\pgfqpoint{1.051846in}{0.888728in}}%
\pgfpathlineto{\pgfqpoint{1.052334in}{0.888728in}}%
\pgfpathlineto{\pgfqpoint{1.053066in}{0.909366in}}%
\pgfpathlineto{\pgfqpoint{1.053310in}{0.904206in}}%
\pgfpathlineto{\pgfqpoint{1.053554in}{0.904206in}}%
\pgfpathlineto{\pgfqpoint{1.053554in}{0.852613in}}%
\pgfpathlineto{\pgfqpoint{1.054530in}{0.904206in}}%
\pgfpathlineto{\pgfqpoint{1.054774in}{0.904206in}}%
\pgfpathlineto{\pgfqpoint{1.055506in}{0.919684in}}%
\pgfpathlineto{\pgfqpoint{1.055018in}{0.873250in}}%
\pgfpathlineto{\pgfqpoint{1.055750in}{0.893888in}}%
\pgfpathlineto{\pgfqpoint{1.055994in}{0.893888in}}%
\pgfpathlineto{\pgfqpoint{1.055994in}{0.837135in}}%
\pgfpathlineto{\pgfqpoint{1.056481in}{0.971278in}}%
\pgfpathlineto{\pgfqpoint{1.056969in}{0.919684in}}%
\pgfpathlineto{\pgfqpoint{1.057213in}{0.919684in}}%
\pgfpathlineto{\pgfqpoint{1.057213in}{0.950641in}}%
\pgfpathlineto{\pgfqpoint{1.058189in}{0.883569in}}%
\pgfpathlineto{\pgfqpoint{1.058433in}{0.883569in}}%
\pgfpathlineto{\pgfqpoint{1.058433in}{0.857772in}}%
\pgfpathlineto{\pgfqpoint{1.058677in}{0.945481in}}%
\pgfpathlineto{\pgfqpoint{1.059409in}{0.904206in}}%
\pgfpathlineto{\pgfqpoint{1.059897in}{0.904206in}}%
\pgfpathlineto{\pgfqpoint{1.059897in}{0.888728in}}%
\pgfpathlineto{\pgfqpoint{1.060141in}{0.919684in}}%
\pgfpathlineto{\pgfqpoint{1.060873in}{0.888728in}}%
\pgfpathlineto{\pgfqpoint{1.061117in}{0.888728in}}%
\pgfpathlineto{\pgfqpoint{1.061605in}{0.873250in}}%
\pgfpathlineto{\pgfqpoint{1.061849in}{0.914525in}}%
\pgfpathlineto{\pgfqpoint{1.062093in}{0.899047in}}%
\pgfpathlineto{\pgfqpoint{1.062337in}{0.899047in}}%
\pgfpathlineto{\pgfqpoint{1.062824in}{0.893888in}}%
\pgfpathlineto{\pgfqpoint{1.063312in}{0.955800in}}%
\pgfpathlineto{\pgfqpoint{1.063556in}{0.955800in}}%
\pgfpathlineto{\pgfqpoint{1.064288in}{0.857772in}}%
\pgfpathlineto{\pgfqpoint{1.064532in}{0.899047in}}%
\pgfpathlineto{\pgfqpoint{1.064776in}{0.899047in}}%
\pgfpathlineto{\pgfqpoint{1.065020in}{0.945481in}}%
\pgfpathlineto{\pgfqpoint{1.065508in}{0.888728in}}%
\pgfpathlineto{\pgfqpoint{1.065752in}{0.924844in}}%
\pgfpathlineto{\pgfqpoint{1.065996in}{0.924844in}}%
\pgfpathlineto{\pgfqpoint{1.065996in}{0.940322in}}%
\pgfpathlineto{\pgfqpoint{1.066240in}{0.919684in}}%
\pgfpathlineto{\pgfqpoint{1.066972in}{0.924844in}}%
\pgfpathlineto{\pgfqpoint{1.067216in}{0.924844in}}%
\pgfpathlineto{\pgfqpoint{1.067704in}{0.873250in}}%
\pgfpathlineto{\pgfqpoint{1.068192in}{0.950641in}}%
\pgfpathlineto{\pgfqpoint{1.068436in}{0.950641in}}%
\pgfpathlineto{\pgfqpoint{1.069168in}{0.997075in}}%
\pgfpathlineto{\pgfqpoint{1.069411in}{0.893888in}}%
\pgfpathlineto{\pgfqpoint{1.069655in}{0.893888in}}%
\pgfpathlineto{\pgfqpoint{1.069899in}{0.960959in}}%
\pgfpathlineto{\pgfqpoint{1.070631in}{0.883569in}}%
\pgfpathlineto{\pgfqpoint{1.070875in}{0.883569in}}%
\pgfpathlineto{\pgfqpoint{1.070875in}{0.960959in}}%
\pgfpathlineto{\pgfqpoint{1.071851in}{0.899047in}}%
\pgfpathlineto{\pgfqpoint{1.072095in}{0.899047in}}%
\pgfpathlineto{\pgfqpoint{1.072827in}{0.950641in}}%
\pgfpathlineto{\pgfqpoint{1.073071in}{0.950641in}}%
\pgfpathlineto{\pgfqpoint{1.073315in}{0.950641in}}%
\pgfpathlineto{\pgfqpoint{1.073315in}{0.971278in}}%
\pgfpathlineto{\pgfqpoint{1.073559in}{0.888728in}}%
\pgfpathlineto{\pgfqpoint{1.074291in}{0.955800in}}%
\pgfpathlineto{\pgfqpoint{1.074535in}{0.955800in}}%
\pgfpathlineto{\pgfqpoint{1.075267in}{1.017712in}}%
\pgfpathlineto{\pgfqpoint{1.075511in}{0.914525in}}%
\pgfpathlineto{\pgfqpoint{1.075755in}{0.914525in}}%
\pgfpathlineto{\pgfqpoint{1.076242in}{0.966119in}}%
\pgfpathlineto{\pgfqpoint{1.076730in}{0.852613in}}%
\pgfpathlineto{\pgfqpoint{1.076974in}{0.852613in}}%
\pgfpathlineto{\pgfqpoint{1.077218in}{0.986756in}}%
\pgfpathlineto{\pgfqpoint{1.077950in}{0.930003in}}%
\pgfpathlineto{\pgfqpoint{1.078194in}{0.930003in}}%
\pgfpathlineto{\pgfqpoint{1.078438in}{0.914525in}}%
\pgfpathlineto{\pgfqpoint{1.078926in}{0.976438in}}%
\pgfpathlineto{\pgfqpoint{1.079170in}{0.976438in}}%
\pgfpathlineto{\pgfqpoint{1.079414in}{0.976438in}}%
\pgfpathlineto{\pgfqpoint{1.079414in}{0.997075in}}%
\pgfpathlineto{\pgfqpoint{1.080146in}{0.904206in}}%
\pgfpathlineto{\pgfqpoint{1.080390in}{0.919684in}}%
\pgfpathlineto{\pgfqpoint{1.080634in}{0.919684in}}%
\pgfpathlineto{\pgfqpoint{1.080634in}{0.997075in}}%
\pgfpathlineto{\pgfqpoint{1.081610in}{0.991916in}}%
\pgfpathlineto{\pgfqpoint{1.081854in}{0.991916in}}%
\pgfpathlineto{\pgfqpoint{1.081854in}{1.002234in}}%
\pgfpathlineto{\pgfqpoint{1.082098in}{0.924844in}}%
\pgfpathlineto{\pgfqpoint{1.082829in}{0.981597in}}%
\pgfpathlineto{\pgfqpoint{1.083073in}{0.981597in}}%
\pgfpathlineto{\pgfqpoint{1.083317in}{1.007394in}}%
\pgfpathlineto{\pgfqpoint{1.084049in}{0.955800in}}%
\pgfpathlineto{\pgfqpoint{1.084293in}{0.955800in}}%
\pgfpathlineto{\pgfqpoint{1.084293in}{0.997075in}}%
\pgfpathlineto{\pgfqpoint{1.085269in}{0.991916in}}%
\pgfpathlineto{\pgfqpoint{1.085513in}{0.991916in}}%
\pgfpathlineto{\pgfqpoint{1.085757in}{0.919684in}}%
\pgfpathlineto{\pgfqpoint{1.086489in}{1.022872in}}%
\pgfpathlineto{\pgfqpoint{1.086733in}{1.022872in}}%
\pgfpathlineto{\pgfqpoint{1.087709in}{0.914525in}}%
\pgfpathlineto{\pgfqpoint{1.087953in}{0.914525in}}%
\pgfpathlineto{\pgfqpoint{1.088197in}{1.033191in}}%
\pgfpathlineto{\pgfqpoint{1.088929in}{1.007394in}}%
\pgfpathlineto{\pgfqpoint{1.089172in}{1.007394in}}%
\pgfpathlineto{\pgfqpoint{1.089172in}{1.028031in}}%
\pgfpathlineto{\pgfqpoint{1.089416in}{0.950641in}}%
\pgfpathlineto{\pgfqpoint{1.090148in}{0.971278in}}%
\pgfpathlineto{\pgfqpoint{1.090392in}{0.971278in}}%
\pgfpathlineto{\pgfqpoint{1.090392in}{1.002234in}}%
\pgfpathlineto{\pgfqpoint{1.090880in}{0.966119in}}%
\pgfpathlineto{\pgfqpoint{1.091368in}{0.971278in}}%
\pgfpathlineto{\pgfqpoint{1.091612in}{0.971278in}}%
\pgfpathlineto{\pgfqpoint{1.092344in}{1.038350in}}%
\pgfpathlineto{\pgfqpoint{1.091856in}{0.924844in}}%
\pgfpathlineto{\pgfqpoint{1.092588in}{1.012553in}}%
\pgfpathlineto{\pgfqpoint{1.092832in}{1.012553in}}%
\pgfpathlineto{\pgfqpoint{1.093076in}{0.924844in}}%
\pgfpathlineto{\pgfqpoint{1.093808in}{1.033191in}}%
\pgfpathlineto{\pgfqpoint{1.094052in}{1.033191in}}%
\pgfpathlineto{\pgfqpoint{1.094296in}{0.935163in}}%
\pgfpathlineto{\pgfqpoint{1.095028in}{0.976438in}}%
\pgfpathlineto{\pgfqpoint{1.095272in}{0.976438in}}%
\pgfpathlineto{\pgfqpoint{1.095272in}{0.930003in}}%
\pgfpathlineto{\pgfqpoint{1.095759in}{1.007394in}}%
\pgfpathlineto{\pgfqpoint{1.096247in}{0.991916in}}%
\pgfpathlineto{\pgfqpoint{1.096491in}{0.991916in}}%
\pgfpathlineto{\pgfqpoint{1.096491in}{0.945481in}}%
\pgfpathlineto{\pgfqpoint{1.097223in}{1.043509in}}%
\pgfpathlineto{\pgfqpoint{1.097467in}{1.007394in}}%
\pgfpathlineto{\pgfqpoint{1.097711in}{1.007394in}}%
\pgfpathlineto{\pgfqpoint{1.097711in}{0.966119in}}%
\pgfpathlineto{\pgfqpoint{1.097955in}{1.053828in}}%
\pgfpathlineto{\pgfqpoint{1.098687in}{0.966119in}}%
\pgfpathlineto{\pgfqpoint{1.098931in}{0.966119in}}%
\pgfpathlineto{\pgfqpoint{1.099419in}{1.074466in}}%
\pgfpathlineto{\pgfqpoint{1.099907in}{1.007394in}}%
\pgfpathlineto{\pgfqpoint{1.100151in}{1.007394in}}%
\pgfpathlineto{\pgfqpoint{1.100151in}{0.930003in}}%
\pgfpathlineto{\pgfqpoint{1.100395in}{1.022872in}}%
\pgfpathlineto{\pgfqpoint{1.101127in}{1.012553in}}%
\pgfpathlineto{\pgfqpoint{1.101371in}{1.012553in}}%
\pgfpathlineto{\pgfqpoint{1.101859in}{1.022872in}}%
\pgfpathlineto{\pgfqpoint{1.102346in}{0.960959in}}%
\pgfpathlineto{\pgfqpoint{1.102590in}{0.960959in}}%
\pgfpathlineto{\pgfqpoint{1.102590in}{1.058987in}}%
\pgfpathlineto{\pgfqpoint{1.103078in}{0.950641in}}%
\pgfpathlineto{\pgfqpoint{1.103566in}{1.007394in}}%
\pgfpathlineto{\pgfqpoint{1.103810in}{1.007394in}}%
\pgfpathlineto{\pgfqpoint{1.103810in}{0.955800in}}%
\pgfpathlineto{\pgfqpoint{1.104054in}{1.028031in}}%
\pgfpathlineto{\pgfqpoint{1.104786in}{0.960959in}}%
\pgfpathlineto{\pgfqpoint{1.105030in}{0.960959in}}%
\pgfpathlineto{\pgfqpoint{1.105030in}{1.012553in}}%
\pgfpathlineto{\pgfqpoint{1.105762in}{0.940322in}}%
\pgfpathlineto{\pgfqpoint{1.106006in}{0.955800in}}%
\pgfpathlineto{\pgfqpoint{1.106250in}{0.955800in}}%
\pgfpathlineto{\pgfqpoint{1.106250in}{0.940322in}}%
\pgfpathlineto{\pgfqpoint{1.106494in}{0.997075in}}%
\pgfpathlineto{\pgfqpoint{1.107226in}{0.997075in}}%
\pgfpathlineto{\pgfqpoint{1.107470in}{0.997075in}}%
\pgfpathlineto{\pgfqpoint{1.107470in}{1.038350in}}%
\pgfpathlineto{\pgfqpoint{1.108202in}{0.986756in}}%
\pgfpathlineto{\pgfqpoint{1.108446in}{0.991916in}}%
\pgfpathlineto{\pgfqpoint{1.108689in}{0.991916in}}%
\pgfpathlineto{\pgfqpoint{1.108933in}{1.074466in}}%
\pgfpathlineto{\pgfqpoint{1.109177in}{0.986756in}}%
\pgfpathlineto{\pgfqpoint{1.109665in}{0.991916in}}%
\pgfpathlineto{\pgfqpoint{1.109909in}{0.991916in}}%
\pgfpathlineto{\pgfqpoint{1.109909in}{0.966119in}}%
\pgfpathlineto{\pgfqpoint{1.110397in}{1.007394in}}%
\pgfpathlineto{\pgfqpoint{1.110885in}{0.997075in}}%
\pgfpathlineto{\pgfqpoint{1.111129in}{0.997075in}}%
\pgfpathlineto{\pgfqpoint{1.111373in}{0.971278in}}%
\pgfpathlineto{\pgfqpoint{1.111861in}{1.089944in}}%
\pgfpathlineto{\pgfqpoint{1.112105in}{0.986756in}}%
\pgfpathlineto{\pgfqpoint{1.112349in}{0.986756in}}%
\pgfpathlineto{\pgfqpoint{1.112349in}{0.981597in}}%
\pgfpathlineto{\pgfqpoint{1.113325in}{1.064147in}}%
\pgfpathlineto{\pgfqpoint{1.113569in}{1.064147in}}%
\pgfpathlineto{\pgfqpoint{1.113813in}{0.971278in}}%
\pgfpathlineto{\pgfqpoint{1.114545in}{1.007394in}}%
\pgfpathlineto{\pgfqpoint{1.114789in}{1.007394in}}%
\pgfpathlineto{\pgfqpoint{1.115520in}{1.028031in}}%
\pgfpathlineto{\pgfqpoint{1.115764in}{0.945481in}}%
\pgfpathlineto{\pgfqpoint{1.116008in}{0.945481in}}%
\pgfpathlineto{\pgfqpoint{1.116740in}{1.028031in}}%
\pgfpathlineto{\pgfqpoint{1.116984in}{1.007394in}}%
\pgfpathlineto{\pgfqpoint{1.117228in}{1.007394in}}%
\pgfpathlineto{\pgfqpoint{1.117228in}{1.033191in}}%
\pgfpathlineto{\pgfqpoint{1.117716in}{0.966119in}}%
\pgfpathlineto{\pgfqpoint{1.118204in}{1.002234in}}%
\pgfpathlineto{\pgfqpoint{1.118448in}{1.002234in}}%
\pgfpathlineto{\pgfqpoint{1.118448in}{0.971278in}}%
\pgfpathlineto{\pgfqpoint{1.119180in}{1.038350in}}%
\pgfpathlineto{\pgfqpoint{1.119424in}{1.002234in}}%
\pgfpathlineto{\pgfqpoint{1.119668in}{1.002234in}}%
\pgfpathlineto{\pgfqpoint{1.120156in}{0.976438in}}%
\pgfpathlineto{\pgfqpoint{1.120400in}{1.048669in}}%
\pgfpathlineto{\pgfqpoint{1.120644in}{1.007394in}}%
\pgfpathlineto{\pgfqpoint{1.120888in}{1.007394in}}%
\pgfpathlineto{\pgfqpoint{1.120888in}{1.095103in}}%
\pgfpathlineto{\pgfqpoint{1.121620in}{0.986756in}}%
\pgfpathlineto{\pgfqpoint{1.121863in}{1.002234in}}%
\pgfpathlineto{\pgfqpoint{1.122107in}{1.002234in}}%
\pgfpathlineto{\pgfqpoint{1.122839in}{1.115740in}}%
\pgfpathlineto{\pgfqpoint{1.123083in}{1.038350in}}%
\pgfpathlineto{\pgfqpoint{1.123327in}{1.038350in}}%
\pgfpathlineto{\pgfqpoint{1.123327in}{0.904206in}}%
\pgfpathlineto{\pgfqpoint{1.124059in}{1.058987in}}%
\pgfpathlineto{\pgfqpoint{1.124303in}{1.028031in}}%
\pgfpathlineto{\pgfqpoint{1.124547in}{1.028031in}}%
\pgfpathlineto{\pgfqpoint{1.125035in}{0.960959in}}%
\pgfpathlineto{\pgfqpoint{1.125279in}{1.033191in}}%
\pgfpathlineto{\pgfqpoint{1.125523in}{0.960959in}}%
\pgfpathlineto{\pgfqpoint{1.125767in}{0.960959in}}%
\pgfpathlineto{\pgfqpoint{1.126499in}{0.945481in}}%
\pgfpathlineto{\pgfqpoint{1.126743in}{1.084784in}}%
\pgfpathlineto{\pgfqpoint{1.126987in}{1.084784in}}%
\pgfpathlineto{\pgfqpoint{1.127231in}{0.976438in}}%
\pgfpathlineto{\pgfqpoint{1.127963in}{1.058987in}}%
\pgfpathlineto{\pgfqpoint{1.128206in}{1.058987in}}%
\pgfpathlineto{\pgfqpoint{1.128938in}{0.966119in}}%
\pgfpathlineto{\pgfqpoint{1.129182in}{1.012553in}}%
\pgfpathlineto{\pgfqpoint{1.129426in}{1.012553in}}%
\pgfpathlineto{\pgfqpoint{1.129426in}{0.924844in}}%
\pgfpathlineto{\pgfqpoint{1.129670in}{1.033191in}}%
\pgfpathlineto{\pgfqpoint{1.130402in}{0.924844in}}%
\pgfpathlineto{\pgfqpoint{1.130646in}{0.924844in}}%
\pgfpathlineto{\pgfqpoint{1.131622in}{1.053828in}}%
\pgfpathlineto{\pgfqpoint{1.131866in}{1.053828in}}%
\pgfpathlineto{\pgfqpoint{1.131866in}{1.095103in}}%
\pgfpathlineto{\pgfqpoint{1.132598in}{0.945481in}}%
\pgfpathlineto{\pgfqpoint{1.132842in}{0.971278in}}%
\pgfpathlineto{\pgfqpoint{1.133086in}{0.971278in}}%
\pgfpathlineto{\pgfqpoint{1.133330in}{1.043509in}}%
\pgfpathlineto{\pgfqpoint{1.134062in}{1.033191in}}%
\pgfpathlineto{\pgfqpoint{1.134306in}{1.033191in}}%
\pgfpathlineto{\pgfqpoint{1.135037in}{0.986756in}}%
\pgfpathlineto{\pgfqpoint{1.134550in}{1.038350in}}%
\pgfpathlineto{\pgfqpoint{1.135281in}{1.007394in}}%
\pgfpathlineto{\pgfqpoint{1.135525in}{1.007394in}}%
\pgfpathlineto{\pgfqpoint{1.135769in}{0.960959in}}%
\pgfpathlineto{\pgfqpoint{1.136501in}{1.043509in}}%
\pgfpathlineto{\pgfqpoint{1.136745in}{1.043509in}}%
\pgfpathlineto{\pgfqpoint{1.136745in}{0.981597in}}%
\pgfpathlineto{\pgfqpoint{1.137721in}{1.038350in}}%
\pgfpathlineto{\pgfqpoint{1.137965in}{1.038350in}}%
\pgfpathlineto{\pgfqpoint{1.138697in}{0.960959in}}%
\pgfpathlineto{\pgfqpoint{1.138941in}{0.960959in}}%
\pgfpathlineto{\pgfqpoint{1.139185in}{0.960959in}}%
\pgfpathlineto{\pgfqpoint{1.139429in}{1.105422in}}%
\pgfpathlineto{\pgfqpoint{1.140161in}{1.064147in}}%
\pgfpathlineto{\pgfqpoint{1.140405in}{1.064147in}}%
\pgfpathlineto{\pgfqpoint{1.140649in}{0.940322in}}%
\pgfpathlineto{\pgfqpoint{1.141380in}{1.038350in}}%
\pgfpathlineto{\pgfqpoint{1.141624in}{1.038350in}}%
\pgfpathlineto{\pgfqpoint{1.141868in}{0.976438in}}%
\pgfpathlineto{\pgfqpoint{1.142112in}{1.043509in}}%
\pgfpathlineto{\pgfqpoint{1.142600in}{0.997075in}}%
\pgfpathlineto{\pgfqpoint{1.142844in}{0.997075in}}%
\pgfpathlineto{\pgfqpoint{1.143088in}{1.079625in}}%
\pgfpathlineto{\pgfqpoint{1.143820in}{0.930003in}}%
\pgfpathlineto{\pgfqpoint{1.144064in}{0.930003in}}%
\pgfpathlineto{\pgfqpoint{1.144796in}{1.058987in}}%
\pgfpathlineto{\pgfqpoint{1.145040in}{1.038350in}}%
\pgfpathlineto{\pgfqpoint{1.145284in}{1.038350in}}%
\pgfpathlineto{\pgfqpoint{1.145284in}{1.048669in}}%
\pgfpathlineto{\pgfqpoint{1.145528in}{0.924844in}}%
\pgfpathlineto{\pgfqpoint{1.146260in}{0.971278in}}%
\pgfpathlineto{\pgfqpoint{1.146504in}{0.971278in}}%
\pgfpathlineto{\pgfqpoint{1.146992in}{1.058987in}}%
\pgfpathlineto{\pgfqpoint{1.147236in}{0.950641in}}%
\pgfpathlineto{\pgfqpoint{1.147480in}{1.053828in}}%
\pgfpathlineto{\pgfqpoint{1.147724in}{1.053828in}}%
\pgfpathlineto{\pgfqpoint{1.147724in}{0.960959in}}%
\pgfpathlineto{\pgfqpoint{1.148699in}{0.981597in}}%
\pgfpathlineto{\pgfqpoint{1.148943in}{0.981597in}}%
\pgfpathlineto{\pgfqpoint{1.149187in}{1.022872in}}%
\pgfpathlineto{\pgfqpoint{1.149919in}{1.017712in}}%
\pgfpathlineto{\pgfqpoint{1.150163in}{1.017712in}}%
\pgfpathlineto{\pgfqpoint{1.150651in}{0.930003in}}%
\pgfpathlineto{\pgfqpoint{1.151139in}{0.976438in}}%
\pgfpathlineto{\pgfqpoint{1.151383in}{0.976438in}}%
\pgfpathlineto{\pgfqpoint{1.151383in}{1.007394in}}%
\pgfpathlineto{\pgfqpoint{1.151627in}{0.971278in}}%
\pgfpathlineto{\pgfqpoint{1.152359in}{1.002234in}}%
\pgfpathlineto{\pgfqpoint{1.152603in}{1.002234in}}%
\pgfpathlineto{\pgfqpoint{1.152847in}{1.033191in}}%
\pgfpathlineto{\pgfqpoint{1.153579in}{0.950641in}}%
\pgfpathlineto{\pgfqpoint{1.153823in}{0.950641in}}%
\pgfpathlineto{\pgfqpoint{1.153823in}{1.033191in}}%
\pgfpathlineto{\pgfqpoint{1.154311in}{0.930003in}}%
\pgfpathlineto{\pgfqpoint{1.154798in}{0.981597in}}%
\pgfpathlineto{\pgfqpoint{1.155042in}{0.981597in}}%
\pgfpathlineto{\pgfqpoint{1.155530in}{0.955800in}}%
\pgfpathlineto{\pgfqpoint{1.155774in}{1.069306in}}%
\pgfpathlineto{\pgfqpoint{1.156018in}{0.997075in}}%
\pgfpathlineto{\pgfqpoint{1.156506in}{0.997075in}}%
\pgfpathlineto{\pgfqpoint{1.156506in}{0.966119in}}%
\pgfpathlineto{\pgfqpoint{1.157238in}{1.084784in}}%
\pgfpathlineto{\pgfqpoint{1.157482in}{1.033191in}}%
\pgfpathlineto{\pgfqpoint{1.157726in}{1.033191in}}%
\pgfpathlineto{\pgfqpoint{1.158458in}{0.960959in}}%
\pgfpathlineto{\pgfqpoint{1.158214in}{1.048669in}}%
\pgfpathlineto{\pgfqpoint{1.158702in}{1.017712in}}%
\pgfpathlineto{\pgfqpoint{1.158946in}{1.017712in}}%
\pgfpathlineto{\pgfqpoint{1.159190in}{0.981597in}}%
\pgfpathlineto{\pgfqpoint{1.159922in}{1.038350in}}%
\pgfpathlineto{\pgfqpoint{1.160166in}{1.038350in}}%
\pgfpathlineto{\pgfqpoint{1.160166in}{0.960959in}}%
\pgfpathlineto{\pgfqpoint{1.161141in}{1.048669in}}%
\pgfpathlineto{\pgfqpoint{1.161385in}{1.048669in}}%
\pgfpathlineto{\pgfqpoint{1.162117in}{0.945481in}}%
\pgfpathlineto{\pgfqpoint{1.161873in}{1.058987in}}%
\pgfpathlineto{\pgfqpoint{1.162361in}{1.012553in}}%
\pgfpathlineto{\pgfqpoint{1.162605in}{1.012553in}}%
\pgfpathlineto{\pgfqpoint{1.162849in}{1.038350in}}%
\pgfpathlineto{\pgfqpoint{1.163337in}{0.991916in}}%
\pgfpathlineto{\pgfqpoint{1.163581in}{1.033191in}}%
\pgfpathlineto{\pgfqpoint{1.163825in}{1.033191in}}%
\pgfpathlineto{\pgfqpoint{1.163825in}{1.048669in}}%
\pgfpathlineto{\pgfqpoint{1.164801in}{0.955800in}}%
\pgfpathlineto{\pgfqpoint{1.165045in}{0.955800in}}%
\pgfpathlineto{\pgfqpoint{1.165533in}{0.986756in}}%
\pgfpathlineto{\pgfqpoint{1.166021in}{0.930003in}}%
\pgfpathlineto{\pgfqpoint{1.166265in}{0.930003in}}%
\pgfpathlineto{\pgfqpoint{1.167241in}{1.017712in}}%
\pgfpathlineto{\pgfqpoint{1.167484in}{1.017712in}}%
\pgfpathlineto{\pgfqpoint{1.167728in}{0.971278in}}%
\pgfpathlineto{\pgfqpoint{1.168460in}{1.100262in}}%
\pgfpathlineto{\pgfqpoint{1.168704in}{1.100262in}}%
\pgfpathlineto{\pgfqpoint{1.168704in}{0.991916in}}%
\pgfpathlineto{\pgfqpoint{1.169680in}{1.089944in}}%
\pgfpathlineto{\pgfqpoint{1.169924in}{1.089944in}}%
\pgfpathlineto{\pgfqpoint{1.169924in}{0.950641in}}%
\pgfpathlineto{\pgfqpoint{1.170900in}{1.022872in}}%
\pgfpathlineto{\pgfqpoint{1.171144in}{1.022872in}}%
\pgfpathlineto{\pgfqpoint{1.171876in}{0.909366in}}%
\pgfpathlineto{\pgfqpoint{1.171632in}{1.038350in}}%
\pgfpathlineto{\pgfqpoint{1.172120in}{1.007394in}}%
\pgfpathlineto{\pgfqpoint{1.172364in}{1.007394in}}%
\pgfpathlineto{\pgfqpoint{1.172364in}{0.935163in}}%
\pgfpathlineto{\pgfqpoint{1.173340in}{1.033191in}}%
\pgfpathlineto{\pgfqpoint{1.173584in}{1.033191in}}%
\pgfpathlineto{\pgfqpoint{1.173828in}{0.966119in}}%
\pgfpathlineto{\pgfqpoint{1.174071in}{1.074466in}}%
\pgfpathlineto{\pgfqpoint{1.174559in}{1.022872in}}%
\pgfpathlineto{\pgfqpoint{1.174803in}{1.022872in}}%
\pgfpathlineto{\pgfqpoint{1.175291in}{0.914525in}}%
\pgfpathlineto{\pgfqpoint{1.175779in}{0.986756in}}%
\pgfpathlineto{\pgfqpoint{1.176023in}{0.986756in}}%
\pgfpathlineto{\pgfqpoint{1.176023in}{0.971278in}}%
\pgfpathlineto{\pgfqpoint{1.176755in}{1.007394in}}%
\pgfpathlineto{\pgfqpoint{1.176999in}{0.981597in}}%
\pgfpathlineto{\pgfqpoint{1.177243in}{0.981597in}}%
\pgfpathlineto{\pgfqpoint{1.177243in}{0.935163in}}%
\pgfpathlineto{\pgfqpoint{1.178219in}{1.043509in}}%
\pgfpathlineto{\pgfqpoint{1.178463in}{1.043509in}}%
\pgfpathlineto{\pgfqpoint{1.178463in}{1.053828in}}%
\pgfpathlineto{\pgfqpoint{1.179439in}{0.924844in}}%
\pgfpathlineto{\pgfqpoint{1.179683in}{0.924844in}}%
\pgfpathlineto{\pgfqpoint{1.180658in}{1.064147in}}%
\pgfpathlineto{\pgfqpoint{1.180902in}{1.064147in}}%
\pgfpathlineto{\pgfqpoint{1.181390in}{0.940322in}}%
\pgfpathlineto{\pgfqpoint{1.181878in}{0.971278in}}%
\pgfpathlineto{\pgfqpoint{1.182122in}{0.971278in}}%
\pgfpathlineto{\pgfqpoint{1.182122in}{1.022872in}}%
\pgfpathlineto{\pgfqpoint{1.183098in}{0.945481in}}%
\pgfpathlineto{\pgfqpoint{1.183342in}{0.945481in}}%
\pgfpathlineto{\pgfqpoint{1.183342in}{0.940322in}}%
\pgfpathlineto{\pgfqpoint{1.183830in}{1.022872in}}%
\pgfpathlineto{\pgfqpoint{1.184318in}{0.971278in}}%
\pgfpathlineto{\pgfqpoint{1.184562in}{0.971278in}}%
\pgfpathlineto{\pgfqpoint{1.185050in}{1.022872in}}%
\pgfpathlineto{\pgfqpoint{1.185538in}{0.960959in}}%
\pgfpathlineto{\pgfqpoint{1.185782in}{0.960959in}}%
\pgfpathlineto{\pgfqpoint{1.186514in}{1.048669in}}%
\pgfpathlineto{\pgfqpoint{1.186758in}{0.960959in}}%
\pgfpathlineto{\pgfqpoint{1.187002in}{0.960959in}}%
\pgfpathlineto{\pgfqpoint{1.187245in}{1.022872in}}%
\pgfpathlineto{\pgfqpoint{1.187977in}{0.930003in}}%
\pgfpathlineto{\pgfqpoint{1.188221in}{0.930003in}}%
\pgfpathlineto{\pgfqpoint{1.188221in}{1.028031in}}%
\pgfpathlineto{\pgfqpoint{1.189197in}{1.007394in}}%
\pgfpathlineto{\pgfqpoint{1.189441in}{1.007394in}}%
\pgfpathlineto{\pgfqpoint{1.189685in}{0.930003in}}%
\pgfpathlineto{\pgfqpoint{1.190417in}{1.017712in}}%
\pgfpathlineto{\pgfqpoint{1.190661in}{1.017712in}}%
\pgfpathlineto{\pgfqpoint{1.190905in}{0.950641in}}%
\pgfpathlineto{\pgfqpoint{1.191637in}{0.981597in}}%
\pgfpathlineto{\pgfqpoint{1.191881in}{0.981597in}}%
\pgfpathlineto{\pgfqpoint{1.192613in}{1.017712in}}%
\pgfpathlineto{\pgfqpoint{1.192857in}{0.930003in}}%
\pgfpathlineto{\pgfqpoint{1.193101in}{0.930003in}}%
\pgfpathlineto{\pgfqpoint{1.193101in}{1.074466in}}%
\pgfpathlineto{\pgfqpoint{1.194076in}{1.053828in}}%
\pgfpathlineto{\pgfqpoint{1.194320in}{1.053828in}}%
\pgfpathlineto{\pgfqpoint{1.194320in}{0.945481in}}%
\pgfpathlineto{\pgfqpoint{1.195296in}{0.997075in}}%
\pgfpathlineto{\pgfqpoint{1.195540in}{0.997075in}}%
\pgfpathlineto{\pgfqpoint{1.196028in}{1.043509in}}%
\pgfpathlineto{\pgfqpoint{1.196516in}{0.981597in}}%
\pgfpathlineto{\pgfqpoint{1.196760in}{0.981597in}}%
\pgfpathlineto{\pgfqpoint{1.197004in}{0.976438in}}%
\pgfpathlineto{\pgfqpoint{1.197736in}{1.084784in}}%
\pgfpathlineto{\pgfqpoint{1.197980in}{1.084784in}}%
\pgfpathlineto{\pgfqpoint{1.198468in}{0.945481in}}%
\pgfpathlineto{\pgfqpoint{1.198956in}{0.971278in}}%
\pgfpathlineto{\pgfqpoint{1.199200in}{0.971278in}}%
\pgfpathlineto{\pgfqpoint{1.199688in}{0.930003in}}%
\pgfpathlineto{\pgfqpoint{1.199444in}{1.007394in}}%
\pgfpathlineto{\pgfqpoint{1.200175in}{0.976438in}}%
\pgfpathlineto{\pgfqpoint{1.200419in}{0.976438in}}%
\pgfpathlineto{\pgfqpoint{1.200419in}{0.945481in}}%
\pgfpathlineto{\pgfqpoint{1.201151in}{1.064147in}}%
\pgfpathlineto{\pgfqpoint{1.201395in}{0.976438in}}%
\pgfpathlineto{\pgfqpoint{1.201639in}{0.976438in}}%
\pgfpathlineto{\pgfqpoint{1.202127in}{1.074466in}}%
\pgfpathlineto{\pgfqpoint{1.202615in}{0.930003in}}%
\pgfpathlineto{\pgfqpoint{1.202859in}{0.930003in}}%
\pgfpathlineto{\pgfqpoint{1.203835in}{1.053828in}}%
\pgfpathlineto{\pgfqpoint{1.204079in}{1.053828in}}%
\pgfpathlineto{\pgfqpoint{1.204079in}{0.971278in}}%
\pgfpathlineto{\pgfqpoint{1.205055in}{0.997075in}}%
\pgfpathlineto{\pgfqpoint{1.205299in}{0.997075in}}%
\pgfpathlineto{\pgfqpoint{1.205299in}{0.935163in}}%
\pgfpathlineto{\pgfqpoint{1.205787in}{1.017712in}}%
\pgfpathlineto{\pgfqpoint{1.206275in}{0.981597in}}%
\pgfpathlineto{\pgfqpoint{1.206519in}{0.981597in}}%
\pgfpathlineto{\pgfqpoint{1.206519in}{0.966119in}}%
\pgfpathlineto{\pgfqpoint{1.207494in}{1.079625in}}%
\pgfpathlineto{\pgfqpoint{1.207738in}{1.079625in}}%
\pgfpathlineto{\pgfqpoint{1.208226in}{0.930003in}}%
\pgfpathlineto{\pgfqpoint{1.208714in}{0.960959in}}%
\pgfpathlineto{\pgfqpoint{1.208958in}{0.960959in}}%
\pgfpathlineto{\pgfqpoint{1.209202in}{1.053828in}}%
\pgfpathlineto{\pgfqpoint{1.209934in}{0.930003in}}%
\pgfpathlineto{\pgfqpoint{1.210178in}{0.930003in}}%
\pgfpathlineto{\pgfqpoint{1.210910in}{1.007394in}}%
\pgfpathlineto{\pgfqpoint{1.211154in}{0.950641in}}%
\pgfpathlineto{\pgfqpoint{1.211398in}{0.950641in}}%
\pgfpathlineto{\pgfqpoint{1.211398in}{0.914525in}}%
\pgfpathlineto{\pgfqpoint{1.212374in}{0.976438in}}%
\pgfpathlineto{\pgfqpoint{1.212618in}{0.976438in}}%
\pgfpathlineto{\pgfqpoint{1.212862in}{1.028031in}}%
\pgfpathlineto{\pgfqpoint{1.213349in}{0.971278in}}%
\pgfpathlineto{\pgfqpoint{1.213593in}{0.971278in}}%
\pgfpathlineto{\pgfqpoint{1.214081in}{0.971278in}}%
\pgfpathlineto{\pgfqpoint{1.214569in}{1.074466in}}%
\pgfpathlineto{\pgfqpoint{1.214325in}{0.930003in}}%
\pgfpathlineto{\pgfqpoint{1.215057in}{0.997075in}}%
\pgfpathlineto{\pgfqpoint{1.215301in}{0.997075in}}%
\pgfpathlineto{\pgfqpoint{1.216033in}{0.919684in}}%
\pgfpathlineto{\pgfqpoint{1.216277in}{0.991916in}}%
\pgfpathlineto{\pgfqpoint{1.216521in}{0.991916in}}%
\pgfpathlineto{\pgfqpoint{1.216765in}{1.002234in}}%
\pgfpathlineto{\pgfqpoint{1.217253in}{0.945481in}}%
\pgfpathlineto{\pgfqpoint{1.217497in}{0.981597in}}%
\pgfpathlineto{\pgfqpoint{1.217741in}{0.981597in}}%
\pgfpathlineto{\pgfqpoint{1.218473in}{1.048669in}}%
\pgfpathlineto{\pgfqpoint{1.217985in}{0.914525in}}%
\pgfpathlineto{\pgfqpoint{1.218717in}{1.007394in}}%
\pgfpathlineto{\pgfqpoint{1.218961in}{1.007394in}}%
\pgfpathlineto{\pgfqpoint{1.219205in}{0.904206in}}%
\pgfpathlineto{\pgfqpoint{1.219936in}{0.950641in}}%
\pgfpathlineto{\pgfqpoint{1.220180in}{0.950641in}}%
\pgfpathlineto{\pgfqpoint{1.220424in}{1.017712in}}%
\pgfpathlineto{\pgfqpoint{1.221156in}{1.007394in}}%
\pgfpathlineto{\pgfqpoint{1.221400in}{1.007394in}}%
\pgfpathlineto{\pgfqpoint{1.222132in}{0.914525in}}%
\pgfpathlineto{\pgfqpoint{1.222376in}{0.971278in}}%
\pgfpathlineto{\pgfqpoint{1.222620in}{0.971278in}}%
\pgfpathlineto{\pgfqpoint{1.222864in}{0.997075in}}%
\pgfpathlineto{\pgfqpoint{1.222864in}{0.924844in}}%
\pgfpathlineto{\pgfqpoint{1.223596in}{0.981597in}}%
\pgfpathlineto{\pgfqpoint{1.223840in}{0.981597in}}%
\pgfpathlineto{\pgfqpoint{1.224572in}{0.986756in}}%
\pgfpathlineto{\pgfqpoint{1.224816in}{0.930003in}}%
\pgfpathlineto{\pgfqpoint{1.225060in}{0.930003in}}%
\pgfpathlineto{\pgfqpoint{1.225792in}{1.017712in}}%
\pgfpathlineto{\pgfqpoint{1.226036in}{0.914525in}}%
\pgfpathlineto{\pgfqpoint{1.226280in}{0.914525in}}%
\pgfpathlineto{\pgfqpoint{1.226523in}{1.043509in}}%
\pgfpathlineto{\pgfqpoint{1.227255in}{0.981597in}}%
\pgfpathlineto{\pgfqpoint{1.227499in}{0.981597in}}%
\pgfpathlineto{\pgfqpoint{1.227499in}{1.017712in}}%
\pgfpathlineto{\pgfqpoint{1.227743in}{0.945481in}}%
\pgfpathlineto{\pgfqpoint{1.228475in}{1.007394in}}%
\pgfpathlineto{\pgfqpoint{1.228719in}{1.007394in}}%
\pgfpathlineto{\pgfqpoint{1.228963in}{0.945481in}}%
\pgfpathlineto{\pgfqpoint{1.229207in}{1.012553in}}%
\pgfpathlineto{\pgfqpoint{1.229695in}{0.981597in}}%
\pgfpathlineto{\pgfqpoint{1.229939in}{0.981597in}}%
\pgfpathlineto{\pgfqpoint{1.230671in}{1.022872in}}%
\pgfpathlineto{\pgfqpoint{1.230915in}{0.950641in}}%
\pgfpathlineto{\pgfqpoint{1.231159in}{0.950641in}}%
\pgfpathlineto{\pgfqpoint{1.231403in}{1.017712in}}%
\pgfpathlineto{\pgfqpoint{1.232135in}{0.991916in}}%
\pgfpathlineto{\pgfqpoint{1.232379in}{0.991916in}}%
\pgfpathlineto{\pgfqpoint{1.232623in}{0.899047in}}%
\pgfpathlineto{\pgfqpoint{1.233110in}{1.017712in}}%
\pgfpathlineto{\pgfqpoint{1.233354in}{0.945481in}}%
\pgfpathlineto{\pgfqpoint{1.233598in}{0.945481in}}%
\pgfpathlineto{\pgfqpoint{1.233598in}{0.935163in}}%
\pgfpathlineto{\pgfqpoint{1.233842in}{0.981597in}}%
\pgfpathlineto{\pgfqpoint{1.234574in}{0.950641in}}%
\pgfpathlineto{\pgfqpoint{1.234818in}{0.950641in}}%
\pgfpathlineto{\pgfqpoint{1.235306in}{0.940322in}}%
\pgfpathlineto{\pgfqpoint{1.235794in}{1.017712in}}%
\pgfpathlineto{\pgfqpoint{1.236038in}{1.017712in}}%
\pgfpathlineto{\pgfqpoint{1.236282in}{0.945481in}}%
\pgfpathlineto{\pgfqpoint{1.236770in}{1.022872in}}%
\pgfpathlineto{\pgfqpoint{1.237014in}{0.986756in}}%
\pgfpathlineto{\pgfqpoint{1.237258in}{0.986756in}}%
\pgfpathlineto{\pgfqpoint{1.237258in}{0.893888in}}%
\pgfpathlineto{\pgfqpoint{1.237746in}{1.022872in}}%
\pgfpathlineto{\pgfqpoint{1.238234in}{0.955800in}}%
\pgfpathlineto{\pgfqpoint{1.238478in}{0.955800in}}%
\pgfpathlineto{\pgfqpoint{1.238478in}{1.038350in}}%
\pgfpathlineto{\pgfqpoint{1.238722in}{0.924844in}}%
\pgfpathlineto{\pgfqpoint{1.239453in}{0.997075in}}%
\pgfpathlineto{\pgfqpoint{1.239697in}{0.997075in}}%
\pgfpathlineto{\pgfqpoint{1.239697in}{0.955800in}}%
\pgfpathlineto{\pgfqpoint{1.240673in}{1.043509in}}%
\pgfpathlineto{\pgfqpoint{1.240917in}{1.043509in}}%
\pgfpathlineto{\pgfqpoint{1.240917in}{0.997075in}}%
\pgfpathlineto{\pgfqpoint{1.241893in}{1.002234in}}%
\pgfpathlineto{\pgfqpoint{1.242137in}{1.002234in}}%
\pgfpathlineto{\pgfqpoint{1.242869in}{0.899047in}}%
\pgfpathlineto{\pgfqpoint{1.242625in}{1.079625in}}%
\pgfpathlineto{\pgfqpoint{1.243113in}{0.899047in}}%
\pgfpathlineto{\pgfqpoint{1.243357in}{0.899047in}}%
\pgfpathlineto{\pgfqpoint{1.244333in}{1.007394in}}%
\pgfpathlineto{\pgfqpoint{1.244577in}{1.007394in}}%
\pgfpathlineto{\pgfqpoint{1.245065in}{0.919684in}}%
\pgfpathlineto{\pgfqpoint{1.245309in}{1.064147in}}%
\pgfpathlineto{\pgfqpoint{1.245553in}{1.017712in}}%
\pgfpathlineto{\pgfqpoint{1.245797in}{1.017712in}}%
\pgfpathlineto{\pgfqpoint{1.246284in}{0.930003in}}%
\pgfpathlineto{\pgfqpoint{1.246772in}{1.048669in}}%
\pgfpathlineto{\pgfqpoint{1.247016in}{1.048669in}}%
\pgfpathlineto{\pgfqpoint{1.247992in}{0.909366in}}%
\pgfpathlineto{\pgfqpoint{1.248236in}{0.909366in}}%
\pgfpathlineto{\pgfqpoint{1.248724in}{1.002234in}}%
\pgfpathlineto{\pgfqpoint{1.249212in}{0.986756in}}%
\pgfpathlineto{\pgfqpoint{1.249456in}{0.986756in}}%
\pgfpathlineto{\pgfqpoint{1.249700in}{0.940322in}}%
\pgfpathlineto{\pgfqpoint{1.249944in}{1.033191in}}%
\pgfpathlineto{\pgfqpoint{1.250432in}{0.986756in}}%
\pgfpathlineto{\pgfqpoint{1.250676in}{0.986756in}}%
\pgfpathlineto{\pgfqpoint{1.250676in}{1.017712in}}%
\pgfpathlineto{\pgfqpoint{1.251652in}{0.930003in}}%
\pgfpathlineto{\pgfqpoint{1.251896in}{0.930003in}}%
\pgfpathlineto{\pgfqpoint{1.252384in}{1.022872in}}%
\pgfpathlineto{\pgfqpoint{1.252871in}{0.971278in}}%
\pgfpathlineto{\pgfqpoint{1.253115in}{0.971278in}}%
\pgfpathlineto{\pgfqpoint{1.253115in}{0.935163in}}%
\pgfpathlineto{\pgfqpoint{1.253603in}{1.038350in}}%
\pgfpathlineto{\pgfqpoint{1.254091in}{0.991916in}}%
\pgfpathlineto{\pgfqpoint{1.254335in}{0.991916in}}%
\pgfpathlineto{\pgfqpoint{1.254579in}{1.074466in}}%
\pgfpathlineto{\pgfqpoint{1.255311in}{0.971278in}}%
\pgfpathlineto{\pgfqpoint{1.255555in}{0.971278in}}%
\pgfpathlineto{\pgfqpoint{1.256531in}{1.095103in}}%
\pgfpathlineto{\pgfqpoint{1.256775in}{1.095103in}}%
\pgfpathlineto{\pgfqpoint{1.257507in}{0.950641in}}%
\pgfpathlineto{\pgfqpoint{1.257751in}{1.069306in}}%
\pgfpathlineto{\pgfqpoint{1.257995in}{1.069306in}}%
\pgfpathlineto{\pgfqpoint{1.257995in}{0.997075in}}%
\pgfpathlineto{\pgfqpoint{1.258971in}{1.074466in}}%
\pgfpathlineto{\pgfqpoint{1.259214in}{1.074466in}}%
\pgfpathlineto{\pgfqpoint{1.259214in}{0.966119in}}%
\pgfpathlineto{\pgfqpoint{1.260190in}{1.002234in}}%
\pgfpathlineto{\pgfqpoint{1.260434in}{1.002234in}}%
\pgfpathlineto{\pgfqpoint{1.260678in}{1.048669in}}%
\pgfpathlineto{\pgfqpoint{1.261410in}{0.986756in}}%
\pgfpathlineto{\pgfqpoint{1.261654in}{0.986756in}}%
\pgfpathlineto{\pgfqpoint{1.262386in}{0.971278in}}%
\pgfpathlineto{\pgfqpoint{1.262142in}{1.058987in}}%
\pgfpathlineto{\pgfqpoint{1.262630in}{1.012553in}}%
\pgfpathlineto{\pgfqpoint{1.262874in}{1.012553in}}%
\pgfpathlineto{\pgfqpoint{1.263118in}{0.966119in}}%
\pgfpathlineto{\pgfqpoint{1.263362in}{1.022872in}}%
\pgfpathlineto{\pgfqpoint{1.263850in}{1.007394in}}%
\pgfpathlineto{\pgfqpoint{1.264094in}{1.007394in}}%
\pgfpathlineto{\pgfqpoint{1.264338in}{1.038350in}}%
\pgfpathlineto{\pgfqpoint{1.264582in}{0.986756in}}%
\pgfpathlineto{\pgfqpoint{1.265070in}{1.022872in}}%
\pgfpathlineto{\pgfqpoint{1.265314in}{1.022872in}}%
\pgfpathlineto{\pgfqpoint{1.265314in}{0.960959in}}%
\pgfpathlineto{\pgfqpoint{1.266289in}{1.002234in}}%
\pgfpathlineto{\pgfqpoint{1.266777in}{1.002234in}}%
\pgfpathlineto{\pgfqpoint{1.267265in}{0.966119in}}%
\pgfpathlineto{\pgfqpoint{1.267509in}{1.007394in}}%
\pgfpathlineto{\pgfqpoint{1.267753in}{0.981597in}}%
\pgfpathlineto{\pgfqpoint{1.267997in}{0.981597in}}%
\pgfpathlineto{\pgfqpoint{1.267997in}{0.966119in}}%
\pgfpathlineto{\pgfqpoint{1.268485in}{1.079625in}}%
\pgfpathlineto{\pgfqpoint{1.268973in}{0.997075in}}%
\pgfpathlineto{\pgfqpoint{1.269217in}{0.997075in}}%
\pgfpathlineto{\pgfqpoint{1.269217in}{1.022872in}}%
\pgfpathlineto{\pgfqpoint{1.269461in}{0.971278in}}%
\pgfpathlineto{\pgfqpoint{1.270193in}{0.986756in}}%
\pgfpathlineto{\pgfqpoint{1.270437in}{0.986756in}}%
\pgfpathlineto{\pgfqpoint{1.270681in}{1.084784in}}%
\pgfpathlineto{\pgfqpoint{1.271169in}{0.960959in}}%
\pgfpathlineto{\pgfqpoint{1.271413in}{1.064147in}}%
\pgfpathlineto{\pgfqpoint{1.271657in}{1.064147in}}%
\pgfpathlineto{\pgfqpoint{1.272144in}{0.945481in}}%
\pgfpathlineto{\pgfqpoint{1.272632in}{1.028031in}}%
\pgfpathlineto{\pgfqpoint{1.272876in}{1.028031in}}%
\pgfpathlineto{\pgfqpoint{1.272876in}{1.043509in}}%
\pgfpathlineto{\pgfqpoint{1.273120in}{0.991916in}}%
\pgfpathlineto{\pgfqpoint{1.273852in}{1.043509in}}%
\pgfpathlineto{\pgfqpoint{1.274096in}{1.043509in}}%
\pgfpathlineto{\pgfqpoint{1.274096in}{0.991916in}}%
\pgfpathlineto{\pgfqpoint{1.274584in}{1.074466in}}%
\pgfpathlineto{\pgfqpoint{1.275072in}{1.038350in}}%
\pgfpathlineto{\pgfqpoint{1.275316in}{1.038350in}}%
\pgfpathlineto{\pgfqpoint{1.275316in}{1.069306in}}%
\pgfpathlineto{\pgfqpoint{1.275560in}{0.955800in}}%
\pgfpathlineto{\pgfqpoint{1.276292in}{0.981597in}}%
\pgfpathlineto{\pgfqpoint{1.276536in}{0.981597in}}%
\pgfpathlineto{\pgfqpoint{1.276780in}{1.058987in}}%
\pgfpathlineto{\pgfqpoint{1.277512in}{0.986756in}}%
\pgfpathlineto{\pgfqpoint{1.277756in}{0.986756in}}%
\pgfpathlineto{\pgfqpoint{1.278488in}{1.043509in}}%
\pgfpathlineto{\pgfqpoint{1.278731in}{0.986756in}}%
\pgfpathlineto{\pgfqpoint{1.278975in}{0.986756in}}%
\pgfpathlineto{\pgfqpoint{1.279951in}{1.131219in}}%
\pgfpathlineto{\pgfqpoint{1.280195in}{1.131219in}}%
\pgfpathlineto{\pgfqpoint{1.280195in}{1.167334in}}%
\pgfpathlineto{\pgfqpoint{1.281171in}{0.997075in}}%
\pgfpathlineto{\pgfqpoint{1.281415in}{0.997075in}}%
\pgfpathlineto{\pgfqpoint{1.281415in}{0.981597in}}%
\pgfpathlineto{\pgfqpoint{1.281903in}{1.074466in}}%
\pgfpathlineto{\pgfqpoint{1.282391in}{1.043509in}}%
\pgfpathlineto{\pgfqpoint{1.282635in}{1.043509in}}%
\pgfpathlineto{\pgfqpoint{1.283123in}{1.224087in}}%
\pgfpathlineto{\pgfqpoint{1.283611in}{1.115740in}}%
\pgfpathlineto{\pgfqpoint{1.283855in}{1.115740in}}%
\pgfpathlineto{\pgfqpoint{1.284587in}{1.053828in}}%
\pgfpathlineto{\pgfqpoint{1.284831in}{1.053828in}}%
\pgfpathlineto{\pgfqpoint{1.285075in}{1.053828in}}%
\pgfpathlineto{\pgfqpoint{1.285075in}{1.038350in}}%
\pgfpathlineto{\pgfqpoint{1.286050in}{1.136378in}}%
\pgfpathlineto{\pgfqpoint{1.286294in}{1.136378in}}%
\pgfpathlineto{\pgfqpoint{1.286294in}{1.048669in}}%
\pgfpathlineto{\pgfqpoint{1.286538in}{1.177653in}}%
\pgfpathlineto{\pgfqpoint{1.287270in}{1.146697in}}%
\pgfpathlineto{\pgfqpoint{1.287514in}{1.146697in}}%
\pgfpathlineto{\pgfqpoint{1.288246in}{1.028031in}}%
\pgfpathlineto{\pgfqpoint{1.288490in}{1.151856in}}%
\pgfpathlineto{\pgfqpoint{1.288734in}{1.151856in}}%
\pgfpathlineto{\pgfqpoint{1.288978in}{1.028031in}}%
\pgfpathlineto{\pgfqpoint{1.289466in}{1.162175in}}%
\pgfpathlineto{\pgfqpoint{1.289710in}{1.141537in}}%
\pgfpathlineto{\pgfqpoint{1.289954in}{1.141537in}}%
\pgfpathlineto{\pgfqpoint{1.290198in}{1.203450in}}%
\pgfpathlineto{\pgfqpoint{1.290442in}{1.095103in}}%
\pgfpathlineto{\pgfqpoint{1.290930in}{1.151856in}}%
\pgfpathlineto{\pgfqpoint{1.291174in}{1.151856in}}%
\pgfpathlineto{\pgfqpoint{1.291418in}{1.105422in}}%
\pgfpathlineto{\pgfqpoint{1.291905in}{1.167334in}}%
\pgfpathlineto{\pgfqpoint{1.292149in}{1.146697in}}%
\pgfpathlineto{\pgfqpoint{1.292393in}{1.146697in}}%
\pgfpathlineto{\pgfqpoint{1.292637in}{1.126059in}}%
\pgfpathlineto{\pgfqpoint{1.293125in}{1.203450in}}%
\pgfpathlineto{\pgfqpoint{1.293369in}{1.151856in}}%
\pgfpathlineto{\pgfqpoint{1.293613in}{1.151856in}}%
\pgfpathlineto{\pgfqpoint{1.294345in}{1.234406in}}%
\pgfpathlineto{\pgfqpoint{1.294589in}{1.100262in}}%
\pgfpathlineto{\pgfqpoint{1.294833in}{1.100262in}}%
\pgfpathlineto{\pgfqpoint{1.294833in}{1.084784in}}%
\pgfpathlineto{\pgfqpoint{1.295565in}{1.306637in}}%
\pgfpathlineto{\pgfqpoint{1.295809in}{1.162175in}}%
\pgfpathlineto{\pgfqpoint{1.296053in}{1.162175in}}%
\pgfpathlineto{\pgfqpoint{1.296785in}{1.316956in}}%
\pgfpathlineto{\pgfqpoint{1.296541in}{1.151856in}}%
\pgfpathlineto{\pgfqpoint{1.297029in}{1.198290in}}%
\pgfpathlineto{\pgfqpoint{1.297273in}{1.198290in}}%
\pgfpathlineto{\pgfqpoint{1.297273in}{1.126059in}}%
\pgfpathlineto{\pgfqpoint{1.297517in}{1.275681in}}%
\pgfpathlineto{\pgfqpoint{1.298249in}{1.255043in}}%
\pgfpathlineto{\pgfqpoint{1.298492in}{1.255043in}}%
\pgfpathlineto{\pgfqpoint{1.298736in}{1.177653in}}%
\pgfpathlineto{\pgfqpoint{1.299468in}{1.270521in}}%
\pgfpathlineto{\pgfqpoint{1.299712in}{1.270521in}}%
\pgfpathlineto{\pgfqpoint{1.300444in}{1.213768in}}%
\pgfpathlineto{\pgfqpoint{1.300688in}{1.275681in}}%
\pgfpathlineto{\pgfqpoint{1.300932in}{1.275681in}}%
\pgfpathlineto{\pgfqpoint{1.300932in}{1.213768in}}%
\pgfpathlineto{\pgfqpoint{1.301908in}{1.378868in}}%
\pgfpathlineto{\pgfqpoint{1.302152in}{1.378868in}}%
\pgfpathlineto{\pgfqpoint{1.302152in}{1.234406in}}%
\pgfpathlineto{\pgfqpoint{1.303128in}{1.363390in}}%
\pgfpathlineto{\pgfqpoint{1.303372in}{1.363390in}}%
\pgfpathlineto{\pgfqpoint{1.303372in}{1.162175in}}%
\pgfpathlineto{\pgfqpoint{1.304348in}{1.322115in}}%
\pgfpathlineto{\pgfqpoint{1.304592in}{1.322115in}}%
\pgfpathlineto{\pgfqpoint{1.304592in}{1.172493in}}%
\pgfpathlineto{\pgfqpoint{1.305079in}{1.332434in}}%
\pgfpathlineto{\pgfqpoint{1.305567in}{1.280840in}}%
\pgfpathlineto{\pgfqpoint{1.305811in}{1.280840in}}%
\pgfpathlineto{\pgfqpoint{1.306543in}{1.420143in}}%
\pgfpathlineto{\pgfqpoint{1.306787in}{1.342753in}}%
\pgfpathlineto{\pgfqpoint{1.307031in}{1.342753in}}%
\pgfpathlineto{\pgfqpoint{1.307763in}{1.249884in}}%
\pgfpathlineto{\pgfqpoint{1.308007in}{1.451099in}}%
\pgfpathlineto{\pgfqpoint{1.308251in}{1.451099in}}%
\pgfpathlineto{\pgfqpoint{1.308251in}{1.249884in}}%
\pgfpathlineto{\pgfqpoint{1.309227in}{1.332434in}}%
\pgfpathlineto{\pgfqpoint{1.309471in}{1.332434in}}%
\pgfpathlineto{\pgfqpoint{1.309471in}{1.306637in}}%
\pgfpathlineto{\pgfqpoint{1.309959in}{1.451099in}}%
\pgfpathlineto{\pgfqpoint{1.310447in}{1.373709in}}%
\pgfpathlineto{\pgfqpoint{1.310691in}{1.373709in}}%
\pgfpathlineto{\pgfqpoint{1.310691in}{1.368549in}}%
\pgfpathlineto{\pgfqpoint{1.310935in}{1.461418in}}%
\pgfpathlineto{\pgfqpoint{1.311666in}{1.389187in}}%
\pgfpathlineto{\pgfqpoint{1.311910in}{1.389187in}}%
\pgfpathlineto{\pgfqpoint{1.311910in}{1.353071in}}%
\pgfpathlineto{\pgfqpoint{1.312398in}{1.440781in}}%
\pgfpathlineto{\pgfqpoint{1.312886in}{1.440781in}}%
\pgfpathlineto{\pgfqpoint{1.313130in}{1.440781in}}%
\pgfpathlineto{\pgfqpoint{1.313130in}{1.451099in}}%
\pgfpathlineto{\pgfqpoint{1.313618in}{1.389187in}}%
\pgfpathlineto{\pgfqpoint{1.314106in}{1.420143in}}%
\pgfpathlineto{\pgfqpoint{1.314350in}{1.420143in}}%
\pgfpathlineto{\pgfqpoint{1.314594in}{1.476896in}}%
\pgfpathlineto{\pgfqpoint{1.315326in}{1.301478in}}%
\pgfpathlineto{\pgfqpoint{1.315570in}{1.301478in}}%
\pgfpathlineto{\pgfqpoint{1.316058in}{1.492374in}}%
\pgfpathlineto{\pgfqpoint{1.316546in}{1.471737in}}%
\pgfpathlineto{\pgfqpoint{1.316790in}{1.471737in}}%
\pgfpathlineto{\pgfqpoint{1.316790in}{1.523330in}}%
\pgfpathlineto{\pgfqpoint{1.317034in}{1.420143in}}%
\pgfpathlineto{\pgfqpoint{1.317766in}{1.513012in}}%
\pgfpathlineto{\pgfqpoint{1.318009in}{1.513012in}}%
\pgfpathlineto{\pgfqpoint{1.318497in}{1.368549in}}%
\pgfpathlineto{\pgfqpoint{1.318253in}{1.585243in}}%
\pgfpathlineto{\pgfqpoint{1.318985in}{1.523330in}}%
\pgfpathlineto{\pgfqpoint{1.319229in}{1.523330in}}%
\pgfpathlineto{\pgfqpoint{1.319229in}{1.440781in}}%
\pgfpathlineto{\pgfqpoint{1.320205in}{1.574924in}}%
\pgfpathlineto{\pgfqpoint{1.320449in}{1.574924in}}%
\pgfpathlineto{\pgfqpoint{1.320449in}{1.636837in}}%
\pgfpathlineto{\pgfqpoint{1.320693in}{1.409824in}}%
\pgfpathlineto{\pgfqpoint{1.321425in}{1.528490in}}%
\pgfpathlineto{\pgfqpoint{1.321669in}{1.528490in}}%
\pgfpathlineto{\pgfqpoint{1.322157in}{1.404665in}}%
\pgfpathlineto{\pgfqpoint{1.321913in}{1.559446in}}%
\pgfpathlineto{\pgfqpoint{1.322645in}{1.513012in}}%
\pgfpathlineto{\pgfqpoint{1.322889in}{1.513012in}}%
\pgfpathlineto{\pgfqpoint{1.323621in}{1.600721in}}%
\pgfpathlineto{\pgfqpoint{1.323133in}{1.456259in}}%
\pgfpathlineto{\pgfqpoint{1.323865in}{1.585243in}}%
\pgfpathlineto{\pgfqpoint{1.324109in}{1.585243in}}%
\pgfpathlineto{\pgfqpoint{1.324109in}{1.616199in}}%
\pgfpathlineto{\pgfqpoint{1.325084in}{1.538809in}}%
\pgfpathlineto{\pgfqpoint{1.325328in}{1.538809in}}%
\pgfpathlineto{\pgfqpoint{1.326060in}{1.729705in}}%
\pgfpathlineto{\pgfqpoint{1.326304in}{1.569765in}}%
\pgfpathlineto{\pgfqpoint{1.326548in}{1.569765in}}%
\pgfpathlineto{\pgfqpoint{1.327036in}{1.538809in}}%
\pgfpathlineto{\pgfqpoint{1.327524in}{1.765821in}}%
\pgfpathlineto{\pgfqpoint{1.327768in}{1.765821in}}%
\pgfpathlineto{\pgfqpoint{1.327768in}{1.518171in}}%
\pgfpathlineto{\pgfqpoint{1.328744in}{1.657474in}}%
\pgfpathlineto{\pgfqpoint{1.328988in}{1.657474in}}%
\pgfpathlineto{\pgfqpoint{1.329232in}{1.683271in}}%
\pgfpathlineto{\pgfqpoint{1.329964in}{1.543968in}}%
\pgfpathlineto{\pgfqpoint{1.330208in}{1.543968in}}%
\pgfpathlineto{\pgfqpoint{1.330452in}{1.745183in}}%
\pgfpathlineto{\pgfqpoint{1.331183in}{1.580084in}}%
\pgfpathlineto{\pgfqpoint{1.331427in}{1.580084in}}%
\pgfpathlineto{\pgfqpoint{1.331915in}{1.781299in}}%
\pgfpathlineto{\pgfqpoint{1.332403in}{1.729705in}}%
\pgfpathlineto{\pgfqpoint{1.332647in}{1.729705in}}%
\pgfpathlineto{\pgfqpoint{1.333379in}{1.693590in}}%
\pgfpathlineto{\pgfqpoint{1.332891in}{1.796777in}}%
\pgfpathlineto{\pgfqpoint{1.333623in}{1.796777in}}%
\pgfpathlineto{\pgfqpoint{1.333867in}{1.796777in}}%
\pgfpathlineto{\pgfqpoint{1.334599in}{1.672952in}}%
\pgfpathlineto{\pgfqpoint{1.334111in}{1.910283in}}%
\pgfpathlineto{\pgfqpoint{1.334843in}{1.770980in}}%
\pgfpathlineto{\pgfqpoint{1.335087in}{1.770980in}}%
\pgfpathlineto{\pgfqpoint{1.335575in}{1.858689in}}%
\pgfpathlineto{\pgfqpoint{1.336063in}{1.796777in}}%
\pgfpathlineto{\pgfqpoint{1.336307in}{1.796777in}}%
\pgfpathlineto{\pgfqpoint{1.336307in}{1.714227in}}%
\pgfpathlineto{\pgfqpoint{1.337283in}{1.961877in}}%
\pgfpathlineto{\pgfqpoint{1.337526in}{1.961877in}}%
\pgfpathlineto{\pgfqpoint{1.337770in}{1.724546in}}%
\pgfpathlineto{\pgfqpoint{1.338502in}{1.951558in}}%
\pgfpathlineto{\pgfqpoint{1.338746in}{1.951558in}}%
\pgfpathlineto{\pgfqpoint{1.339478in}{1.781299in}}%
\pgfpathlineto{\pgfqpoint{1.339722in}{1.796777in}}%
\pgfpathlineto{\pgfqpoint{1.339966in}{1.796777in}}%
\pgfpathlineto{\pgfqpoint{1.340454in}{2.054745in}}%
\pgfpathlineto{\pgfqpoint{1.340942in}{1.791618in}}%
\pgfpathlineto{\pgfqpoint{1.341186in}{1.791618in}}%
\pgfpathlineto{\pgfqpoint{1.341430in}{2.018630in}}%
\pgfpathlineto{\pgfqpoint{1.342162in}{1.941239in}}%
\pgfpathlineto{\pgfqpoint{1.342406in}{1.941239in}}%
\pgfpathlineto{\pgfqpoint{1.342406in}{1.874167in}}%
\pgfpathlineto{\pgfqpoint{1.343382in}{2.147614in}}%
\pgfpathlineto{\pgfqpoint{1.343626in}{2.147614in}}%
\pgfpathlineto{\pgfqpoint{1.343626in}{1.941239in}}%
\pgfpathlineto{\pgfqpoint{1.344601in}{2.039267in}}%
\pgfpathlineto{\pgfqpoint{1.344845in}{2.039267in}}%
\pgfpathlineto{\pgfqpoint{1.345089in}{1.879327in}}%
\pgfpathlineto{\pgfqpoint{1.345577in}{2.194048in}}%
\pgfpathlineto{\pgfqpoint{1.345821in}{2.168251in}}%
\pgfpathlineto{\pgfqpoint{1.346065in}{2.168251in}}%
\pgfpathlineto{\pgfqpoint{1.346065in}{2.173411in}}%
\pgfpathlineto{\pgfqpoint{1.347041in}{1.961877in}}%
\pgfpathlineto{\pgfqpoint{1.347285in}{1.961877in}}%
\pgfpathlineto{\pgfqpoint{1.348017in}{2.188889in}}%
\pgfpathlineto{\pgfqpoint{1.348261in}{2.008311in}}%
\pgfpathlineto{\pgfqpoint{1.348505in}{2.008311in}}%
\pgfpathlineto{\pgfqpoint{1.348993in}{2.214686in}}%
\pgfpathlineto{\pgfqpoint{1.349481in}{2.116658in}}%
\pgfpathlineto{\pgfqpoint{1.349725in}{2.116658in}}%
\pgfpathlineto{\pgfqpoint{1.350457in}{1.941239in}}%
\pgfpathlineto{\pgfqpoint{1.350700in}{2.183730in}}%
\pgfpathlineto{\pgfqpoint{1.350944in}{2.183730in}}%
\pgfpathlineto{\pgfqpoint{1.351188in}{2.008311in}}%
\pgfpathlineto{\pgfqpoint{1.351920in}{2.183730in}}%
\pgfpathlineto{\pgfqpoint{1.352164in}{2.183730in}}%
\pgfpathlineto{\pgfqpoint{1.352164in}{2.044427in}}%
\pgfpathlineto{\pgfqpoint{1.353140in}{2.271439in}}%
\pgfpathlineto{\pgfqpoint{1.353384in}{2.271439in}}%
\pgfpathlineto{\pgfqpoint{1.353384in}{2.353989in}}%
\pgfpathlineto{\pgfqpoint{1.353628in}{2.085702in}}%
\pgfpathlineto{\pgfqpoint{1.354360in}{2.245642in}}%
\pgfpathlineto{\pgfqpoint{1.354604in}{2.245642in}}%
\pgfpathlineto{\pgfqpoint{1.355336in}{2.333351in}}%
\pgfpathlineto{\pgfqpoint{1.354848in}{2.121817in}}%
\pgfpathlineto{\pgfqpoint{1.355580in}{2.225004in}}%
\pgfpathlineto{\pgfqpoint{1.355824in}{2.225004in}}%
\pgfpathlineto{\pgfqpoint{1.355824in}{2.163092in}}%
\pgfpathlineto{\pgfqpoint{1.356556in}{2.292076in}}%
\pgfpathlineto{\pgfqpoint{1.356800in}{2.230164in}}%
\pgfpathlineto{\pgfqpoint{1.357044in}{2.230164in}}%
\pgfpathlineto{\pgfqpoint{1.357287in}{2.400423in}}%
\pgfpathlineto{\pgfqpoint{1.357531in}{2.225004in}}%
\pgfpathlineto{\pgfqpoint{1.358019in}{2.384945in}}%
\pgfpathlineto{\pgfqpoint{1.358263in}{2.384945in}}%
\pgfpathlineto{\pgfqpoint{1.358507in}{2.194048in}}%
\pgfpathlineto{\pgfqpoint{1.358995in}{2.446857in}}%
\pgfpathlineto{\pgfqpoint{1.359239in}{2.343670in}}%
\pgfpathlineto{\pgfqpoint{1.359483in}{2.343670in}}%
\pgfpathlineto{\pgfqpoint{1.360215in}{2.452017in}}%
\pgfpathlineto{\pgfqpoint{1.359971in}{2.250801in}}%
\pgfpathlineto{\pgfqpoint{1.360459in}{2.359148in}}%
\pgfpathlineto{\pgfqpoint{1.360703in}{2.359148in}}%
\pgfpathlineto{\pgfqpoint{1.360947in}{2.482973in}}%
\pgfpathlineto{\pgfqpoint{1.361191in}{2.183730in}}%
\pgfpathlineto{\pgfqpoint{1.361679in}{2.457176in}}%
\pgfpathlineto{\pgfqpoint{1.361923in}{2.457176in}}%
\pgfpathlineto{\pgfqpoint{1.362411in}{2.297236in}}%
\pgfpathlineto{\pgfqpoint{1.362899in}{2.395264in}}%
\pgfpathlineto{\pgfqpoint{1.363143in}{2.395264in}}%
\pgfpathlineto{\pgfqpoint{1.363631in}{2.534567in}}%
\pgfpathlineto{\pgfqpoint{1.364118in}{2.364307in}}%
\pgfpathlineto{\pgfqpoint{1.364362in}{2.364307in}}%
\pgfpathlineto{\pgfqpoint{1.364606in}{2.581001in}}%
\pgfpathlineto{\pgfqpoint{1.365338in}{2.410742in}}%
\pgfpathlineto{\pgfqpoint{1.365582in}{2.410742in}}%
\pgfpathlineto{\pgfqpoint{1.366314in}{2.570682in}}%
\pgfpathlineto{\pgfqpoint{1.366070in}{2.240483in}}%
\pgfpathlineto{\pgfqpoint{1.366558in}{2.359148in}}%
\pgfpathlineto{\pgfqpoint{1.366802in}{2.359148in}}%
\pgfpathlineto{\pgfqpoint{1.367534in}{2.255961in}}%
\pgfpathlineto{\pgfqpoint{1.367778in}{2.565523in}}%
\pgfpathlineto{\pgfqpoint{1.368022in}{2.565523in}}%
\pgfpathlineto{\pgfqpoint{1.368754in}{2.663551in}}%
\pgfpathlineto{\pgfqpoint{1.368998in}{2.436539in}}%
\pgfpathlineto{\pgfqpoint{1.369242in}{2.436539in}}%
\pgfpathlineto{\pgfqpoint{1.369486in}{2.570682in}}%
\pgfpathlineto{\pgfqpoint{1.370218in}{2.534567in}}%
\pgfpathlineto{\pgfqpoint{1.370461in}{2.534567in}}%
\pgfpathlineto{\pgfqpoint{1.370705in}{2.467495in}}%
\pgfpathlineto{\pgfqpoint{1.371193in}{2.813172in}}%
\pgfpathlineto{\pgfqpoint{1.371437in}{2.482973in}}%
\pgfpathlineto{\pgfqpoint{1.371681in}{2.482973in}}%
\pgfpathlineto{\pgfqpoint{1.372169in}{2.704826in}}%
\pgfpathlineto{\pgfqpoint{1.372657in}{2.519088in}}%
\pgfpathlineto{\pgfqpoint{1.372901in}{2.519088in}}%
\pgfpathlineto{\pgfqpoint{1.373145in}{2.808013in}}%
\pgfpathlineto{\pgfqpoint{1.373877in}{2.596479in}}%
\pgfpathlineto{\pgfqpoint{1.374121in}{2.596479in}}%
\pgfpathlineto{\pgfqpoint{1.374121in}{2.895722in}}%
\pgfpathlineto{\pgfqpoint{1.374365in}{2.529407in}}%
\pgfpathlineto{\pgfqpoint{1.375097in}{2.797694in}}%
\pgfpathlineto{\pgfqpoint{1.375341in}{2.797694in}}%
\pgfpathlineto{\pgfqpoint{1.375585in}{2.493292in}}%
\pgfpathlineto{\pgfqpoint{1.376317in}{2.771897in}}%
\pgfpathlineto{\pgfqpoint{1.376561in}{2.771897in}}%
\pgfpathlineto{\pgfqpoint{1.376804in}{2.560363in}}%
\pgfpathlineto{\pgfqpoint{1.377048in}{2.849288in}}%
\pgfpathlineto{\pgfqpoint{1.377536in}{2.699666in}}%
\pgfpathlineto{\pgfqpoint{1.377780in}{2.699666in}}%
\pgfpathlineto{\pgfqpoint{1.378512in}{2.869925in}}%
\pgfpathlineto{\pgfqpoint{1.378024in}{2.689348in}}%
\pgfpathlineto{\pgfqpoint{1.378756in}{2.823491in}}%
\pgfpathlineto{\pgfqpoint{1.379000in}{2.823491in}}%
\pgfpathlineto{\pgfqpoint{1.379732in}{2.648073in}}%
\pgfpathlineto{\pgfqpoint{1.379976in}{2.911200in}}%
\pgfpathlineto{\pgfqpoint{1.380220in}{2.911200in}}%
\pgfpathlineto{\pgfqpoint{1.380708in}{2.740941in}}%
\pgfpathlineto{\pgfqpoint{1.381196in}{2.849288in}}%
\pgfpathlineto{\pgfqpoint{1.381440in}{2.849288in}}%
\pgfpathlineto{\pgfqpoint{1.381684in}{2.617116in}}%
\pgfpathlineto{\pgfqpoint{1.381928in}{2.890563in}}%
\pgfpathlineto{\pgfqpoint{1.382416in}{2.715144in}}%
\pgfpathlineto{\pgfqpoint{1.382660in}{2.715144in}}%
\pgfpathlineto{\pgfqpoint{1.382904in}{2.921519in}}%
\pgfpathlineto{\pgfqpoint{1.383148in}{2.648073in}}%
\pgfpathlineto{\pgfqpoint{1.383635in}{2.885404in}}%
\pgfpathlineto{\pgfqpoint{1.383879in}{2.885404in}}%
\pgfpathlineto{\pgfqpoint{1.384123in}{2.715144in}}%
\pgfpathlineto{\pgfqpoint{1.384611in}{2.983431in}}%
\pgfpathlineto{\pgfqpoint{1.384855in}{2.967953in}}%
\pgfpathlineto{\pgfqpoint{1.385099in}{2.967953in}}%
\pgfpathlineto{\pgfqpoint{1.385099in}{2.854447in}}%
\pgfpathlineto{\pgfqpoint{1.385343in}{3.133053in}}%
\pgfpathlineto{\pgfqpoint{1.386075in}{3.096938in}}%
\pgfpathlineto{\pgfqpoint{1.386319in}{3.096938in}}%
\pgfpathlineto{\pgfqpoint{1.386319in}{2.730622in}}%
\pgfpathlineto{\pgfqpoint{1.387295in}{2.916360in}}%
\pgfpathlineto{\pgfqpoint{1.387539in}{2.916360in}}%
\pgfpathlineto{\pgfqpoint{1.387539in}{3.117575in}}%
\pgfpathlineto{\pgfqpoint{1.388515in}{3.065981in}}%
\pgfpathlineto{\pgfqpoint{1.388759in}{3.065981in}}%
\pgfpathlineto{\pgfqpoint{1.388759in}{2.735782in}}%
\pgfpathlineto{\pgfqpoint{1.389735in}{2.761579in}}%
\pgfpathlineto{\pgfqpoint{1.389978in}{2.761579in}}%
\pgfpathlineto{\pgfqpoint{1.390954in}{3.334268in}}%
\pgfpathlineto{\pgfqpoint{1.391198in}{3.334268in}}%
\pgfpathlineto{\pgfqpoint{1.391198in}{2.936997in}}%
\pgfpathlineto{\pgfqpoint{1.392174in}{3.153691in}}%
\pgfpathlineto{\pgfqpoint{1.392418in}{3.153691in}}%
\pgfpathlineto{\pgfqpoint{1.392906in}{2.864766in}}%
\pgfpathlineto{\pgfqpoint{1.393394in}{3.024706in}}%
\pgfpathlineto{\pgfqpoint{1.393638in}{3.024706in}}%
\pgfpathlineto{\pgfqpoint{1.393882in}{3.174328in}}%
\pgfpathlineto{\pgfqpoint{1.394370in}{2.988591in}}%
\pgfpathlineto{\pgfqpoint{1.394614in}{3.086619in}}%
\pgfpathlineto{\pgfqpoint{1.394858in}{3.086619in}}%
\pgfpathlineto{\pgfqpoint{1.394858in}{2.916360in}}%
\pgfpathlineto{\pgfqpoint{1.395346in}{3.138213in}}%
\pgfpathlineto{\pgfqpoint{1.395834in}{2.988591in}}%
\pgfpathlineto{\pgfqpoint{1.396078in}{2.988591in}}%
\pgfpathlineto{\pgfqpoint{1.396565in}{3.148531in}}%
\pgfpathlineto{\pgfqpoint{1.396322in}{2.973113in}}%
\pgfpathlineto{\pgfqpoint{1.397053in}{3.029866in}}%
\pgfpathlineto{\pgfqpoint{1.397297in}{3.029866in}}%
\pgfpathlineto{\pgfqpoint{1.397785in}{3.210444in}}%
\pgfpathlineto{\pgfqpoint{1.397541in}{3.004069in}}%
\pgfpathlineto{\pgfqpoint{1.398273in}{3.194966in}}%
\pgfpathlineto{\pgfqpoint{1.398517in}{3.194966in}}%
\pgfpathlineto{\pgfqpoint{1.399249in}{3.282675in}}%
\pgfpathlineto{\pgfqpoint{1.399493in}{2.962794in}}%
\pgfpathlineto{\pgfqpoint{1.399737in}{2.962794in}}%
\pgfpathlineto{\pgfqpoint{1.399981in}{3.231081in}}%
\pgfpathlineto{\pgfqpoint{1.400713in}{3.040185in}}%
\pgfpathlineto{\pgfqpoint{1.400957in}{3.040185in}}%
\pgfpathlineto{\pgfqpoint{1.400957in}{3.231081in}}%
\pgfpathlineto{\pgfqpoint{1.401201in}{2.973113in}}%
\pgfpathlineto{\pgfqpoint{1.401933in}{3.035025in}}%
\pgfpathlineto{\pgfqpoint{1.402177in}{3.035025in}}%
\pgfpathlineto{\pgfqpoint{1.403152in}{3.277515in}}%
\pgfpathlineto{\pgfqpoint{1.403396in}{3.277515in}}%
\pgfpathlineto{\pgfqpoint{1.403884in}{2.875085in}}%
\pgfpathlineto{\pgfqpoint{1.404372in}{3.035025in}}%
\pgfpathlineto{\pgfqpoint{1.404616in}{3.035025in}}%
\pgfpathlineto{\pgfqpoint{1.405592in}{3.272356in}}%
\pgfpathlineto{\pgfqpoint{1.405836in}{3.272356in}}%
\pgfpathlineto{\pgfqpoint{1.406080in}{3.463253in}}%
\pgfpathlineto{\pgfqpoint{1.406812in}{3.009228in}}%
\pgfpathlineto{\pgfqpoint{1.407056in}{3.009228in}}%
\pgfpathlineto{\pgfqpoint{1.407300in}{3.344587in}}%
\pgfpathlineto{\pgfqpoint{1.408032in}{3.251719in}}%
\pgfpathlineto{\pgfqpoint{1.408276in}{3.251719in}}%
\pgfpathlineto{\pgfqpoint{1.408276in}{3.241400in}}%
\pgfpathlineto{\pgfqpoint{1.409252in}{3.483890in}}%
\pgfpathlineto{\pgfqpoint{1.409495in}{3.483890in}}%
\pgfpathlineto{\pgfqpoint{1.409739in}{3.277515in}}%
\pgfpathlineto{\pgfqpoint{1.410471in}{3.344587in}}%
\pgfpathlineto{\pgfqpoint{1.410715in}{3.344587in}}%
\pgfpathlineto{\pgfqpoint{1.411691in}{3.189806in}}%
\pgfpathlineto{\pgfqpoint{1.411935in}{3.189806in}}%
\pgfpathlineto{\pgfqpoint{1.412911in}{3.458093in}}%
\pgfpathlineto{\pgfqpoint{1.413155in}{3.458093in}}%
\pgfpathlineto{\pgfqpoint{1.413155in}{3.184647in}}%
\pgfpathlineto{\pgfqpoint{1.414131in}{3.360065in}}%
\pgfpathlineto{\pgfqpoint{1.414375in}{3.360065in}}%
\pgfpathlineto{\pgfqpoint{1.415351in}{3.220762in}}%
\pgfpathlineto{\pgfqpoint{1.415595in}{3.220762in}}%
\pgfpathlineto{\pgfqpoint{1.415595in}{3.169169in}}%
\pgfpathlineto{\pgfqpoint{1.416082in}{3.556121in}}%
\pgfpathlineto{\pgfqpoint{1.416570in}{3.354906in}}%
\pgfpathlineto{\pgfqpoint{1.416814in}{3.354906in}}%
\pgfpathlineto{\pgfqpoint{1.417302in}{3.251719in}}%
\pgfpathlineto{\pgfqpoint{1.417546in}{3.421978in}}%
\pgfpathlineto{\pgfqpoint{1.417790in}{3.344587in}}%
\pgfpathlineto{\pgfqpoint{1.418034in}{3.344587in}}%
\pgfpathlineto{\pgfqpoint{1.418766in}{3.148531in}}%
\pgfpathlineto{\pgfqpoint{1.419010in}{3.344587in}}%
\pgfpathlineto{\pgfqpoint{1.419254in}{3.344587in}}%
\pgfpathlineto{\pgfqpoint{1.419986in}{3.184647in}}%
\pgfpathlineto{\pgfqpoint{1.420230in}{3.468412in}}%
\pgfpathlineto{\pgfqpoint{1.420474in}{3.468412in}}%
\pgfpathlineto{\pgfqpoint{1.420474in}{3.241400in}}%
\pgfpathlineto{\pgfqpoint{1.421450in}{3.401340in}}%
\pgfpathlineto{\pgfqpoint{1.421694in}{3.401340in}}%
\pgfpathlineto{\pgfqpoint{1.421694in}{3.514846in}}%
\pgfpathlineto{\pgfqpoint{1.422182in}{3.272356in}}%
\pgfpathlineto{\pgfqpoint{1.422669in}{3.458093in}}%
\pgfpathlineto{\pgfqpoint{1.422913in}{3.458093in}}%
\pgfpathlineto{\pgfqpoint{1.423157in}{3.169169in}}%
\pgfpathlineto{\pgfqpoint{1.423401in}{3.540643in}}%
\pgfpathlineto{\pgfqpoint{1.423889in}{3.396181in}}%
\pgfpathlineto{\pgfqpoint{1.424133in}{3.396181in}}%
\pgfpathlineto{\pgfqpoint{1.424133in}{3.458093in}}%
\pgfpathlineto{\pgfqpoint{1.424865in}{3.308472in}}%
\pgfpathlineto{\pgfqpoint{1.425109in}{3.323950in}}%
\pgfpathlineto{\pgfqpoint{1.425353in}{3.323950in}}%
\pgfpathlineto{\pgfqpoint{1.425353in}{3.303312in}}%
\pgfpathlineto{\pgfqpoint{1.426329in}{3.648990in}}%
\pgfpathlineto{\pgfqpoint{1.426573in}{3.648990in}}%
\pgfpathlineto{\pgfqpoint{1.426817in}{3.236240in}}%
\pgfpathlineto{\pgfqpoint{1.427549in}{3.354906in}}%
\pgfpathlineto{\pgfqpoint{1.427793in}{3.354906in}}%
\pgfpathlineto{\pgfqpoint{1.428525in}{3.478731in}}%
\pgfpathlineto{\pgfqpoint{1.428769in}{3.360065in}}%
\pgfpathlineto{\pgfqpoint{1.429013in}{3.360065in}}%
\pgfpathlineto{\pgfqpoint{1.429744in}{3.148531in}}%
\pgfpathlineto{\pgfqpoint{1.429988in}{3.406500in}}%
\pgfpathlineto{\pgfqpoint{1.430232in}{3.406500in}}%
\pgfpathlineto{\pgfqpoint{1.430720in}{3.184647in}}%
\pgfpathlineto{\pgfqpoint{1.431208in}{3.437456in}}%
\pgfpathlineto{\pgfqpoint{1.431452in}{3.437456in}}%
\pgfpathlineto{\pgfqpoint{1.431452in}{3.612874in}}%
\pgfpathlineto{\pgfqpoint{1.432428in}{3.169169in}}%
\pgfpathlineto{\pgfqpoint{1.432672in}{3.169169in}}%
\pgfpathlineto{\pgfqpoint{1.432916in}{3.437456in}}%
\pgfpathlineto{\pgfqpoint{1.433648in}{3.096938in}}%
\pgfpathlineto{\pgfqpoint{1.433892in}{3.096938in}}%
\pgfpathlineto{\pgfqpoint{1.434136in}{3.483890in}}%
\pgfpathlineto{\pgfqpoint{1.434868in}{3.225922in}}%
\pgfpathlineto{\pgfqpoint{1.435112in}{3.225922in}}%
\pgfpathlineto{\pgfqpoint{1.435112in}{2.993750in}}%
\pgfpathlineto{\pgfqpoint{1.435843in}{3.489050in}}%
\pgfpathlineto{\pgfqpoint{1.436087in}{3.396181in}}%
\pgfpathlineto{\pgfqpoint{1.436331in}{3.396181in}}%
\pgfpathlineto{\pgfqpoint{1.436575in}{3.329109in}}%
\pgfpathlineto{\pgfqpoint{1.437063in}{3.478731in}}%
\pgfpathlineto{\pgfqpoint{1.437307in}{3.401340in}}%
\pgfpathlineto{\pgfqpoint{1.437551in}{3.401340in}}%
\pgfpathlineto{\pgfqpoint{1.437551in}{3.277515in}}%
\pgfpathlineto{\pgfqpoint{1.438283in}{3.458093in}}%
\pgfpathlineto{\pgfqpoint{1.438527in}{3.437456in}}%
\pgfpathlineto{\pgfqpoint{1.438771in}{3.437456in}}%
\pgfpathlineto{\pgfqpoint{1.439503in}{3.133053in}}%
\pgfpathlineto{\pgfqpoint{1.439747in}{3.344587in}}%
\pgfpathlineto{\pgfqpoint{1.439991in}{3.344587in}}%
\pgfpathlineto{\pgfqpoint{1.440723in}{3.489050in}}%
\pgfpathlineto{\pgfqpoint{1.440967in}{3.225922in}}%
\pgfpathlineto{\pgfqpoint{1.441211in}{3.225922in}}%
\pgfpathlineto{\pgfqpoint{1.441455in}{3.385862in}}%
\pgfpathlineto{\pgfqpoint{1.442186in}{3.127894in}}%
\pgfpathlineto{\pgfqpoint{1.442430in}{3.127894in}}%
\pgfpathlineto{\pgfqpoint{1.443162in}{3.411659in}}%
\pgfpathlineto{\pgfqpoint{1.443406in}{3.081459in}}%
\pgfpathlineto{\pgfqpoint{1.443650in}{3.081459in}}%
\pgfpathlineto{\pgfqpoint{1.443894in}{3.014388in}}%
\pgfpathlineto{\pgfqpoint{1.444626in}{3.262037in}}%
\pgfpathlineto{\pgfqpoint{1.444870in}{3.262037in}}%
\pgfpathlineto{\pgfqpoint{1.444870in}{3.499368in}}%
\pgfpathlineto{\pgfqpoint{1.445114in}{3.096938in}}%
\pgfpathlineto{\pgfqpoint{1.445846in}{3.303312in}}%
\pgfpathlineto{\pgfqpoint{1.446334in}{3.303312in}}%
\pgfpathlineto{\pgfqpoint{1.446578in}{3.050503in}}%
\pgfpathlineto{\pgfqpoint{1.447310in}{3.292994in}}%
\pgfpathlineto{\pgfqpoint{1.447554in}{3.292994in}}%
\pgfpathlineto{\pgfqpoint{1.448042in}{3.040185in}}%
\pgfpathlineto{\pgfqpoint{1.448530in}{3.050503in}}%
\pgfpathlineto{\pgfqpoint{1.448773in}{3.050503in}}%
\pgfpathlineto{\pgfqpoint{1.449505in}{3.385862in}}%
\pgfpathlineto{\pgfqpoint{1.449749in}{3.194966in}}%
\pgfpathlineto{\pgfqpoint{1.449993in}{3.194966in}}%
\pgfpathlineto{\pgfqpoint{1.450481in}{3.360065in}}%
\pgfpathlineto{\pgfqpoint{1.450725in}{3.096938in}}%
\pgfpathlineto{\pgfqpoint{1.450969in}{3.339428in}}%
\pgfpathlineto{\pgfqpoint{1.451213in}{3.339428in}}%
\pgfpathlineto{\pgfqpoint{1.451701in}{3.416818in}}%
\pgfpathlineto{\pgfqpoint{1.452189in}{3.035025in}}%
\pgfpathlineto{\pgfqpoint{1.452433in}{3.035025in}}%
\pgfpathlineto{\pgfqpoint{1.452921in}{3.004069in}}%
\pgfpathlineto{\pgfqpoint{1.452677in}{3.292994in}}%
\pgfpathlineto{\pgfqpoint{1.453409in}{3.215603in}}%
\pgfpathlineto{\pgfqpoint{1.453653in}{3.215603in}}%
\pgfpathlineto{\pgfqpoint{1.453653in}{2.957635in}}%
\pgfpathlineto{\pgfqpoint{1.454629in}{3.009228in}}%
\pgfpathlineto{\pgfqpoint{1.454873in}{3.009228in}}%
\pgfpathlineto{\pgfqpoint{1.455604in}{3.241400in}}%
\pgfpathlineto{\pgfqpoint{1.455848in}{3.231081in}}%
\pgfpathlineto{\pgfqpoint{1.456092in}{3.231081in}}%
\pgfpathlineto{\pgfqpoint{1.456824in}{2.921519in}}%
\pgfpathlineto{\pgfqpoint{1.457068in}{3.117575in}}%
\pgfpathlineto{\pgfqpoint{1.457312in}{3.117575in}}%
\pgfpathlineto{\pgfqpoint{1.458288in}{2.849288in}}%
\pgfpathlineto{\pgfqpoint{1.458532in}{2.849288in}}%
\pgfpathlineto{\pgfqpoint{1.458532in}{3.174328in}}%
\pgfpathlineto{\pgfqpoint{1.459508in}{2.911200in}}%
\pgfpathlineto{\pgfqpoint{1.459752in}{2.911200in}}%
\pgfpathlineto{\pgfqpoint{1.459752in}{3.210444in}}%
\pgfpathlineto{\pgfqpoint{1.460728in}{2.983431in}}%
\pgfpathlineto{\pgfqpoint{1.460972in}{2.983431in}}%
\pgfpathlineto{\pgfqpoint{1.461460in}{3.200125in}}%
\pgfpathlineto{\pgfqpoint{1.461947in}{2.936997in}}%
\pgfpathlineto{\pgfqpoint{1.462191in}{2.936997in}}%
\pgfpathlineto{\pgfqpoint{1.462435in}{3.205284in}}%
\pgfpathlineto{\pgfqpoint{1.462679in}{2.792535in}}%
\pgfpathlineto{\pgfqpoint{1.463167in}{2.890563in}}%
\pgfpathlineto{\pgfqpoint{1.463411in}{2.890563in}}%
\pgfpathlineto{\pgfqpoint{1.463899in}{3.009228in}}%
\pgfpathlineto{\pgfqpoint{1.464387in}{2.808013in}}%
\pgfpathlineto{\pgfqpoint{1.464631in}{2.808013in}}%
\pgfpathlineto{\pgfqpoint{1.465119in}{3.189806in}}%
\pgfpathlineto{\pgfqpoint{1.465363in}{2.756419in}}%
\pgfpathlineto{\pgfqpoint{1.465607in}{2.854447in}}%
\pgfpathlineto{\pgfqpoint{1.465851in}{2.854447in}}%
\pgfpathlineto{\pgfqpoint{1.466583in}{3.019547in}}%
\pgfpathlineto{\pgfqpoint{1.466339in}{2.833810in}}%
\pgfpathlineto{\pgfqpoint{1.466827in}{2.838969in}}%
\pgfpathlineto{\pgfqpoint{1.467071in}{2.838969in}}%
\pgfpathlineto{\pgfqpoint{1.467071in}{2.936997in}}%
\pgfpathlineto{\pgfqpoint{1.467315in}{2.735782in}}%
\pgfpathlineto{\pgfqpoint{1.468047in}{2.797694in}}%
\pgfpathlineto{\pgfqpoint{1.468291in}{2.797694in}}%
\pgfpathlineto{\pgfqpoint{1.469022in}{2.983431in}}%
\pgfpathlineto{\pgfqpoint{1.468778in}{2.642913in}}%
\pgfpathlineto{\pgfqpoint{1.469266in}{2.844129in}}%
\pgfpathlineto{\pgfqpoint{1.469510in}{2.844129in}}%
\pgfpathlineto{\pgfqpoint{1.470486in}{2.916360in}}%
\pgfpathlineto{\pgfqpoint{1.470730in}{2.916360in}}%
\pgfpathlineto{\pgfqpoint{1.471462in}{2.771897in}}%
\pgfpathlineto{\pgfqpoint{1.471218in}{3.060822in}}%
\pgfpathlineto{\pgfqpoint{1.471706in}{2.983431in}}%
\pgfpathlineto{\pgfqpoint{1.471950in}{2.983431in}}%
\pgfpathlineto{\pgfqpoint{1.472194in}{2.673869in}}%
\pgfpathlineto{\pgfqpoint{1.472926in}{2.962794in}}%
\pgfpathlineto{\pgfqpoint{1.473170in}{2.962794in}}%
\pgfpathlineto{\pgfqpoint{1.473902in}{2.591320in}}%
\pgfpathlineto{\pgfqpoint{1.473414in}{2.993750in}}%
\pgfpathlineto{\pgfqpoint{1.474146in}{2.715144in}}%
\pgfpathlineto{\pgfqpoint{1.474390in}{2.715144in}}%
\pgfpathlineto{\pgfqpoint{1.474390in}{2.792535in}}%
\pgfpathlineto{\pgfqpoint{1.475365in}{2.586160in}}%
\pgfpathlineto{\pgfqpoint{1.475609in}{2.586160in}}%
\pgfpathlineto{\pgfqpoint{1.475853in}{2.890563in}}%
\pgfpathlineto{\pgfqpoint{1.476097in}{2.472654in}}%
\pgfpathlineto{\pgfqpoint{1.476585in}{2.751260in}}%
\pgfpathlineto{\pgfqpoint{1.476829in}{2.751260in}}%
\pgfpathlineto{\pgfqpoint{1.477805in}{2.591320in}}%
\pgfpathlineto{\pgfqpoint{1.478049in}{2.591320in}}%
\pgfpathlineto{\pgfqpoint{1.478537in}{2.673869in}}%
\pgfpathlineto{\pgfqpoint{1.478781in}{2.467495in}}%
\pgfpathlineto{\pgfqpoint{1.479025in}{2.544885in}}%
\pgfpathlineto{\pgfqpoint{1.479269in}{2.544885in}}%
\pgfpathlineto{\pgfqpoint{1.479757in}{2.653232in}}%
\pgfpathlineto{\pgfqpoint{1.480245in}{2.348829in}}%
\pgfpathlineto{\pgfqpoint{1.480489in}{2.348829in}}%
\pgfpathlineto{\pgfqpoint{1.480489in}{2.673869in}}%
\pgfpathlineto{\pgfqpoint{1.481221in}{2.297236in}}%
\pgfpathlineto{\pgfqpoint{1.481464in}{2.508770in}}%
\pgfpathlineto{\pgfqpoint{1.481708in}{2.508770in}}%
\pgfpathlineto{\pgfqpoint{1.482440in}{2.364307in}}%
\pgfpathlineto{\pgfqpoint{1.482684in}{2.581001in}}%
\pgfpathlineto{\pgfqpoint{1.482928in}{2.581001in}}%
\pgfpathlineto{\pgfqpoint{1.482928in}{2.601638in}}%
\pgfpathlineto{\pgfqpoint{1.483660in}{2.379785in}}%
\pgfpathlineto{\pgfqpoint{1.483904in}{2.524248in}}%
\pgfpathlineto{\pgfqpoint{1.484148in}{2.524248in}}%
\pgfpathlineto{\pgfqpoint{1.484392in}{2.405582in}}%
\pgfpathlineto{\pgfqpoint{1.484636in}{2.601638in}}%
\pgfpathlineto{\pgfqpoint{1.485124in}{2.457176in}}%
\pgfpathlineto{\pgfqpoint{1.485368in}{2.457176in}}%
\pgfpathlineto{\pgfqpoint{1.485368in}{2.668710in}}%
\pgfpathlineto{\pgfqpoint{1.486344in}{2.364307in}}%
\pgfpathlineto{\pgfqpoint{1.486588in}{2.364307in}}%
\pgfpathlineto{\pgfqpoint{1.486588in}{2.255961in}}%
\pgfpathlineto{\pgfqpoint{1.487564in}{2.565523in}}%
\pgfpathlineto{\pgfqpoint{1.487808in}{2.565523in}}%
\pgfpathlineto{\pgfqpoint{1.487808in}{2.343670in}}%
\pgfpathlineto{\pgfqpoint{1.488783in}{2.482973in}}%
\pgfpathlineto{\pgfqpoint{1.489027in}{2.482973in}}%
\pgfpathlineto{\pgfqpoint{1.489515in}{2.152773in}}%
\pgfpathlineto{\pgfqpoint{1.490003in}{2.302395in}}%
\pgfpathlineto{\pgfqpoint{1.490247in}{2.302395in}}%
\pgfpathlineto{\pgfqpoint{1.490247in}{2.462335in}}%
\pgfpathlineto{\pgfqpoint{1.491223in}{2.188889in}}%
\pgfpathlineto{\pgfqpoint{1.491467in}{2.188889in}}%
\pgfpathlineto{\pgfqpoint{1.491711in}{2.426220in}}%
\pgfpathlineto{\pgfqpoint{1.492443in}{2.410742in}}%
\pgfpathlineto{\pgfqpoint{1.492687in}{2.410742in}}%
\pgfpathlineto{\pgfqpoint{1.492931in}{2.173411in}}%
\pgfpathlineto{\pgfqpoint{1.493663in}{2.292076in}}%
\pgfpathlineto{\pgfqpoint{1.493907in}{2.292076in}}%
\pgfpathlineto{\pgfqpoint{1.493907in}{2.312714in}}%
\pgfpathlineto{\pgfqpoint{1.494882in}{2.209526in}}%
\pgfpathlineto{\pgfqpoint{1.495126in}{2.209526in}}%
\pgfpathlineto{\pgfqpoint{1.495126in}{2.250801in}}%
\pgfpathlineto{\pgfqpoint{1.495858in}{2.132136in}}%
\pgfpathlineto{\pgfqpoint{1.496102in}{2.163092in}}%
\pgfpathlineto{\pgfqpoint{1.496346in}{2.163092in}}%
\pgfpathlineto{\pgfqpoint{1.496590in}{2.312714in}}%
\pgfpathlineto{\pgfqpoint{1.496590in}{2.096020in}}%
\pgfpathlineto{\pgfqpoint{1.497322in}{2.126976in}}%
\pgfpathlineto{\pgfqpoint{1.497566in}{2.126976in}}%
\pgfpathlineto{\pgfqpoint{1.497810in}{2.255961in}}%
\pgfpathlineto{\pgfqpoint{1.497810in}{2.028948in}}%
\pgfpathlineto{\pgfqpoint{1.498542in}{2.209526in}}%
\pgfpathlineto{\pgfqpoint{1.498786in}{2.209526in}}%
\pgfpathlineto{\pgfqpoint{1.499274in}{2.044427in}}%
\pgfpathlineto{\pgfqpoint{1.499518in}{2.369467in}}%
\pgfpathlineto{\pgfqpoint{1.499762in}{2.173411in}}%
\pgfpathlineto{\pgfqpoint{1.500006in}{2.173411in}}%
\pgfpathlineto{\pgfqpoint{1.500006in}{2.008311in}}%
\pgfpathlineto{\pgfqpoint{1.500982in}{2.183730in}}%
\pgfpathlineto{\pgfqpoint{1.501225in}{2.183730in}}%
\pgfpathlineto{\pgfqpoint{1.501713in}{2.281758in}}%
\pgfpathlineto{\pgfqpoint{1.502201in}{2.018630in}}%
\pgfpathlineto{\pgfqpoint{1.502445in}{2.018630in}}%
\pgfpathlineto{\pgfqpoint{1.502445in}{2.173411in}}%
\pgfpathlineto{\pgfqpoint{1.502933in}{1.992833in}}%
\pgfpathlineto{\pgfqpoint{1.503421in}{2.152773in}}%
\pgfpathlineto{\pgfqpoint{1.503665in}{2.152773in}}%
\pgfpathlineto{\pgfqpoint{1.503665in}{2.173411in}}%
\pgfpathlineto{\pgfqpoint{1.503909in}{1.915442in}}%
\pgfpathlineto{\pgfqpoint{1.504641in}{1.920602in}}%
\pgfpathlineto{\pgfqpoint{1.504885in}{1.920602in}}%
\pgfpathlineto{\pgfqpoint{1.504885in}{1.899964in}}%
\pgfpathlineto{\pgfqpoint{1.505129in}{2.147614in}}%
\pgfpathlineto{\pgfqpoint{1.505861in}{2.028948in}}%
\pgfpathlineto{\pgfqpoint{1.506105in}{2.028948in}}%
\pgfpathlineto{\pgfqpoint{1.506593in}{2.183730in}}%
\pgfpathlineto{\pgfqpoint{1.506349in}{1.930921in}}%
\pgfpathlineto{\pgfqpoint{1.507081in}{1.946399in}}%
\pgfpathlineto{\pgfqpoint{1.507325in}{1.946399in}}%
\pgfpathlineto{\pgfqpoint{1.507569in}{1.817414in}}%
\pgfpathlineto{\pgfqpoint{1.508300in}{1.961877in}}%
\pgfpathlineto{\pgfqpoint{1.508544in}{1.961877in}}%
\pgfpathlineto{\pgfqpoint{1.508544in}{1.884486in}}%
\pgfpathlineto{\pgfqpoint{1.508788in}{2.034108in}}%
\pgfpathlineto{\pgfqpoint{1.509520in}{1.997992in}}%
\pgfpathlineto{\pgfqpoint{1.509764in}{1.997992in}}%
\pgfpathlineto{\pgfqpoint{1.510008in}{2.101180in}}%
\pgfpathlineto{\pgfqpoint{1.510740in}{1.693590in}}%
\pgfpathlineto{\pgfqpoint{1.510984in}{1.693590in}}%
\pgfpathlineto{\pgfqpoint{1.511960in}{1.915442in}}%
\pgfpathlineto{\pgfqpoint{1.512204in}{1.915442in}}%
\pgfpathlineto{\pgfqpoint{1.512936in}{1.946399in}}%
\pgfpathlineto{\pgfqpoint{1.513180in}{1.724546in}}%
\pgfpathlineto{\pgfqpoint{1.513424in}{1.724546in}}%
\pgfpathlineto{\pgfqpoint{1.513668in}{1.967036in}}%
\pgfpathlineto{\pgfqpoint{1.514399in}{1.776139in}}%
\pgfpathlineto{\pgfqpoint{1.514643in}{1.776139in}}%
\pgfpathlineto{\pgfqpoint{1.515131in}{1.930921in}}%
\pgfpathlineto{\pgfqpoint{1.514887in}{1.724546in}}%
\pgfpathlineto{\pgfqpoint{1.515619in}{1.807096in}}%
\pgfpathlineto{\pgfqpoint{1.515863in}{1.807096in}}%
\pgfpathlineto{\pgfqpoint{1.516595in}{1.683271in}}%
\pgfpathlineto{\pgfqpoint{1.516107in}{1.889646in}}%
\pgfpathlineto{\pgfqpoint{1.516839in}{1.740024in}}%
\pgfpathlineto{\pgfqpoint{1.517083in}{1.740024in}}%
\pgfpathlineto{\pgfqpoint{1.517815in}{1.770980in}}%
\pgfpathlineto{\pgfqpoint{1.517327in}{1.662633in}}%
\pgfpathlineto{\pgfqpoint{1.518059in}{1.709068in}}%
\pgfpathlineto{\pgfqpoint{1.518303in}{1.709068in}}%
\pgfpathlineto{\pgfqpoint{1.519035in}{1.616199in}}%
\pgfpathlineto{\pgfqpoint{1.519279in}{1.858689in}}%
\pgfpathlineto{\pgfqpoint{1.519523in}{1.858689in}}%
\pgfpathlineto{\pgfqpoint{1.520499in}{1.549127in}}%
\pgfpathlineto{\pgfqpoint{1.520742in}{1.549127in}}%
\pgfpathlineto{\pgfqpoint{1.520742in}{1.822574in}}%
\pgfpathlineto{\pgfqpoint{1.521230in}{1.543968in}}%
\pgfpathlineto{\pgfqpoint{1.521718in}{1.605880in}}%
\pgfpathlineto{\pgfqpoint{1.521962in}{1.605880in}}%
\pgfpathlineto{\pgfqpoint{1.521962in}{1.709068in}}%
\pgfpathlineto{\pgfqpoint{1.522694in}{1.600721in}}%
\pgfpathlineto{\pgfqpoint{1.522938in}{1.693590in}}%
\pgfpathlineto{\pgfqpoint{1.523182in}{1.693590in}}%
\pgfpathlineto{\pgfqpoint{1.523914in}{1.600721in}}%
\pgfpathlineto{\pgfqpoint{1.524158in}{1.600721in}}%
\pgfpathlineto{\pgfqpoint{1.524402in}{1.600721in}}%
\pgfpathlineto{\pgfqpoint{1.524402in}{1.456259in}}%
\pgfpathlineto{\pgfqpoint{1.524646in}{1.621358in}}%
\pgfpathlineto{\pgfqpoint{1.525378in}{1.605880in}}%
\pgfpathlineto{\pgfqpoint{1.525622in}{1.605880in}}%
\pgfpathlineto{\pgfqpoint{1.525622in}{1.626518in}}%
\pgfpathlineto{\pgfqpoint{1.526598in}{1.420143in}}%
\pgfpathlineto{\pgfqpoint{1.526842in}{1.420143in}}%
\pgfpathlineto{\pgfqpoint{1.527817in}{1.559446in}}%
\pgfpathlineto{\pgfqpoint{1.528061in}{1.559446in}}%
\pgfpathlineto{\pgfqpoint{1.528305in}{1.466577in}}%
\pgfpathlineto{\pgfqpoint{1.528549in}{1.564605in}}%
\pgfpathlineto{\pgfqpoint{1.529037in}{1.482056in}}%
\pgfpathlineto{\pgfqpoint{1.529281in}{1.482056in}}%
\pgfpathlineto{\pgfqpoint{1.529281in}{1.564605in}}%
\pgfpathlineto{\pgfqpoint{1.530013in}{1.461418in}}%
\pgfpathlineto{\pgfqpoint{1.530257in}{1.564605in}}%
\pgfpathlineto{\pgfqpoint{1.530501in}{1.564605in}}%
\pgfpathlineto{\pgfqpoint{1.531233in}{1.420143in}}%
\pgfpathlineto{\pgfqpoint{1.531477in}{1.616199in}}%
\pgfpathlineto{\pgfqpoint{1.531721in}{1.616199in}}%
\pgfpathlineto{\pgfqpoint{1.532453in}{1.327275in}}%
\pgfpathlineto{\pgfqpoint{1.532697in}{1.430462in}}%
\pgfpathlineto{\pgfqpoint{1.532941in}{1.430462in}}%
\pgfpathlineto{\pgfqpoint{1.533429in}{1.549127in}}%
\pgfpathlineto{\pgfqpoint{1.533916in}{1.384028in}}%
\pgfpathlineto{\pgfqpoint{1.534160in}{1.384028in}}%
\pgfpathlineto{\pgfqpoint{1.534648in}{1.497534in}}%
\pgfpathlineto{\pgfqpoint{1.535136in}{1.451099in}}%
\pgfpathlineto{\pgfqpoint{1.535380in}{1.451099in}}%
\pgfpathlineto{\pgfqpoint{1.535380in}{1.327275in}}%
\pgfpathlineto{\pgfqpoint{1.535868in}{1.513012in}}%
\pgfpathlineto{\pgfqpoint{1.536356in}{1.337593in}}%
\pgfpathlineto{\pgfqpoint{1.536600in}{1.337593in}}%
\pgfpathlineto{\pgfqpoint{1.537576in}{1.425302in}}%
\pgfpathlineto{\pgfqpoint{1.537820in}{1.425302in}}%
\pgfpathlineto{\pgfqpoint{1.538552in}{1.306637in}}%
\pgfpathlineto{\pgfqpoint{1.538796in}{1.430462in}}%
\pgfpathlineto{\pgfqpoint{1.539040in}{1.430462in}}%
\pgfpathlineto{\pgfqpoint{1.540016in}{1.260203in}}%
\pgfpathlineto{\pgfqpoint{1.540260in}{1.260203in}}%
\pgfpathlineto{\pgfqpoint{1.540747in}{1.255043in}}%
\pgfpathlineto{\pgfqpoint{1.541235in}{1.358231in}}%
\pgfpathlineto{\pgfqpoint{1.541479in}{1.358231in}}%
\pgfpathlineto{\pgfqpoint{1.541967in}{1.255043in}}%
\pgfpathlineto{\pgfqpoint{1.542455in}{1.322115in}}%
\pgfpathlineto{\pgfqpoint{1.542699in}{1.322115in}}%
\pgfpathlineto{\pgfqpoint{1.543675in}{1.198290in}}%
\pgfpathlineto{\pgfqpoint{1.543919in}{1.198290in}}%
\pgfpathlineto{\pgfqpoint{1.543919in}{1.322115in}}%
\pgfpathlineto{\pgfqpoint{1.544895in}{1.249884in}}%
\pgfpathlineto{\pgfqpoint{1.545139in}{1.249884in}}%
\pgfpathlineto{\pgfqpoint{1.545871in}{1.327275in}}%
\pgfpathlineto{\pgfqpoint{1.545627in}{1.218928in}}%
\pgfpathlineto{\pgfqpoint{1.546115in}{1.234406in}}%
\pgfpathlineto{\pgfqpoint{1.546359in}{1.234406in}}%
\pgfpathlineto{\pgfqpoint{1.546846in}{1.141537in}}%
\pgfpathlineto{\pgfqpoint{1.546603in}{1.260203in}}%
\pgfpathlineto{\pgfqpoint{1.547334in}{1.260203in}}%
\pgfpathlineto{\pgfqpoint{1.547578in}{1.260203in}}%
\pgfpathlineto{\pgfqpoint{1.547822in}{1.126059in}}%
\pgfpathlineto{\pgfqpoint{1.548554in}{1.275681in}}%
\pgfpathlineto{\pgfqpoint{1.548798in}{1.275681in}}%
\pgfpathlineto{\pgfqpoint{1.549286in}{1.167334in}}%
\pgfpathlineto{\pgfqpoint{1.549774in}{1.208609in}}%
\pgfpathlineto{\pgfqpoint{1.550018in}{1.208609in}}%
\pgfpathlineto{\pgfqpoint{1.550262in}{1.100262in}}%
\pgfpathlineto{\pgfqpoint{1.550750in}{1.244725in}}%
\pgfpathlineto{\pgfqpoint{1.550994in}{1.193131in}}%
\pgfpathlineto{\pgfqpoint{1.551238in}{1.193131in}}%
\pgfpathlineto{\pgfqpoint{1.551970in}{1.265362in}}%
\pgfpathlineto{\pgfqpoint{1.552214in}{1.131219in}}%
\pgfpathlineto{\pgfqpoint{1.552458in}{1.131219in}}%
\pgfpathlineto{\pgfqpoint{1.552458in}{1.079625in}}%
\pgfpathlineto{\pgfqpoint{1.553433in}{1.239565in}}%
\pgfpathlineto{\pgfqpoint{1.553677in}{1.239565in}}%
\pgfpathlineto{\pgfqpoint{1.554409in}{1.105422in}}%
\pgfpathlineto{\pgfqpoint{1.554653in}{1.157015in}}%
\pgfpathlineto{\pgfqpoint{1.554897in}{1.157015in}}%
\pgfpathlineto{\pgfqpoint{1.554897in}{1.177653in}}%
\pgfpathlineto{\pgfqpoint{1.555873in}{1.064147in}}%
\pgfpathlineto{\pgfqpoint{1.556117in}{1.064147in}}%
\pgfpathlineto{\pgfqpoint{1.556849in}{1.022872in}}%
\pgfpathlineto{\pgfqpoint{1.557093in}{1.177653in}}%
\pgfpathlineto{\pgfqpoint{1.557337in}{1.177653in}}%
\pgfpathlineto{\pgfqpoint{1.558069in}{1.017712in}}%
\pgfpathlineto{\pgfqpoint{1.558313in}{1.064147in}}%
\pgfpathlineto{\pgfqpoint{1.558557in}{1.064147in}}%
\pgfpathlineto{\pgfqpoint{1.559533in}{1.172493in}}%
\pgfpathlineto{\pgfqpoint{1.559777in}{1.172493in}}%
\pgfpathlineto{\pgfqpoint{1.559777in}{1.012553in}}%
\pgfpathlineto{\pgfqpoint{1.560752in}{1.053828in}}%
\pgfpathlineto{\pgfqpoint{1.560996in}{1.053828in}}%
\pgfpathlineto{\pgfqpoint{1.560996in}{1.017712in}}%
\pgfpathlineto{\pgfqpoint{1.561484in}{1.105422in}}%
\pgfpathlineto{\pgfqpoint{1.561972in}{1.022872in}}%
\pgfpathlineto{\pgfqpoint{1.562216in}{1.022872in}}%
\pgfpathlineto{\pgfqpoint{1.562460in}{0.966119in}}%
\pgfpathlineto{\pgfqpoint{1.562704in}{1.038350in}}%
\pgfpathlineto{\pgfqpoint{1.563192in}{0.997075in}}%
\pgfpathlineto{\pgfqpoint{1.563436in}{0.997075in}}%
\pgfpathlineto{\pgfqpoint{1.564168in}{0.976438in}}%
\pgfpathlineto{\pgfqpoint{1.564412in}{1.089944in}}%
\pgfpathlineto{\pgfqpoint{1.564656in}{1.089944in}}%
\pgfpathlineto{\pgfqpoint{1.565144in}{0.997075in}}%
\pgfpathlineto{\pgfqpoint{1.565632in}{1.002234in}}%
\pgfpathlineto{\pgfqpoint{1.565876in}{1.002234in}}%
\pgfpathlineto{\pgfqpoint{1.566607in}{1.028031in}}%
\pgfpathlineto{\pgfqpoint{1.566851in}{0.914525in}}%
\pgfpathlineto{\pgfqpoint{1.567095in}{0.914525in}}%
\pgfpathlineto{\pgfqpoint{1.567827in}{0.991916in}}%
\pgfpathlineto{\pgfqpoint{1.568071in}{0.991916in}}%
\pgfpathlineto{\pgfqpoint{1.568315in}{0.991916in}}%
\pgfpathlineto{\pgfqpoint{1.568315in}{1.064147in}}%
\pgfpathlineto{\pgfqpoint{1.568559in}{0.945481in}}%
\pgfpathlineto{\pgfqpoint{1.569291in}{1.033191in}}%
\pgfpathlineto{\pgfqpoint{1.569535in}{1.033191in}}%
\pgfpathlineto{\pgfqpoint{1.570511in}{0.909366in}}%
\pgfpathlineto{\pgfqpoint{1.570755in}{0.909366in}}%
\pgfpathlineto{\pgfqpoint{1.571487in}{0.997075in}}%
\pgfpathlineto{\pgfqpoint{1.571731in}{0.940322in}}%
\pgfpathlineto{\pgfqpoint{1.571975in}{0.940322in}}%
\pgfpathlineto{\pgfqpoint{1.571975in}{0.919684in}}%
\pgfpathlineto{\pgfqpoint{1.572707in}{0.991916in}}%
\pgfpathlineto{\pgfqpoint{1.572951in}{0.981597in}}%
\pgfpathlineto{\pgfqpoint{1.573194in}{0.981597in}}%
\pgfpathlineto{\pgfqpoint{1.573194in}{0.904206in}}%
\pgfpathlineto{\pgfqpoint{1.573438in}{0.986756in}}%
\pgfpathlineto{\pgfqpoint{1.574170in}{0.909366in}}%
\pgfpathlineto{\pgfqpoint{1.574414in}{0.909366in}}%
\pgfpathlineto{\pgfqpoint{1.574414in}{0.960959in}}%
\pgfpathlineto{\pgfqpoint{1.575390in}{0.935163in}}%
\pgfpathlineto{\pgfqpoint{1.575634in}{0.935163in}}%
\pgfpathlineto{\pgfqpoint{1.575634in}{0.899047in}}%
\pgfpathlineto{\pgfqpoint{1.576366in}{0.966119in}}%
\pgfpathlineto{\pgfqpoint{1.576610in}{0.940322in}}%
\pgfpathlineto{\pgfqpoint{1.576854in}{0.940322in}}%
\pgfpathlineto{\pgfqpoint{1.577098in}{0.852613in}}%
\pgfpathlineto{\pgfqpoint{1.577830in}{0.878410in}}%
\pgfpathlineto{\pgfqpoint{1.578074in}{0.878410in}}%
\pgfpathlineto{\pgfqpoint{1.578074in}{0.868091in}}%
\pgfpathlineto{\pgfqpoint{1.578318in}{0.919684in}}%
\pgfpathlineto{\pgfqpoint{1.579050in}{0.919684in}}%
\pgfpathlineto{\pgfqpoint{1.579294in}{0.919684in}}%
\pgfpathlineto{\pgfqpoint{1.579294in}{0.878410in}}%
\pgfpathlineto{\pgfqpoint{1.579781in}{0.935163in}}%
\pgfpathlineto{\pgfqpoint{1.580269in}{0.930003in}}%
\pgfpathlineto{\pgfqpoint{1.580513in}{0.930003in}}%
\pgfpathlineto{\pgfqpoint{1.581489in}{0.847453in}}%
\pgfpathlineto{\pgfqpoint{1.581733in}{0.847453in}}%
\pgfpathlineto{\pgfqpoint{1.581977in}{0.960959in}}%
\pgfpathlineto{\pgfqpoint{1.582709in}{0.852613in}}%
\pgfpathlineto{\pgfqpoint{1.582953in}{0.852613in}}%
\pgfpathlineto{\pgfqpoint{1.582953in}{0.837135in}}%
\pgfpathlineto{\pgfqpoint{1.583441in}{0.893888in}}%
\pgfpathlineto{\pgfqpoint{1.583929in}{0.852613in}}%
\pgfpathlineto{\pgfqpoint{1.584173in}{0.852613in}}%
\pgfpathlineto{\pgfqpoint{1.584173in}{0.821656in}}%
\pgfpathlineto{\pgfqpoint{1.585149in}{0.883569in}}%
\pgfpathlineto{\pgfqpoint{1.585393in}{0.883569in}}%
\pgfpathlineto{\pgfqpoint{1.585393in}{0.919684in}}%
\pgfpathlineto{\pgfqpoint{1.586368in}{0.816497in}}%
\pgfpathlineto{\pgfqpoint{1.586612in}{0.816497in}}%
\pgfpathlineto{\pgfqpoint{1.587100in}{0.924844in}}%
\pgfpathlineto{\pgfqpoint{1.587588in}{0.837135in}}%
\pgfpathlineto{\pgfqpoint{1.587832in}{0.837135in}}%
\pgfpathlineto{\pgfqpoint{1.588076in}{0.893888in}}%
\pgfpathlineto{\pgfqpoint{1.588564in}{0.801019in}}%
\pgfpathlineto{\pgfqpoint{1.588808in}{0.873250in}}%
\pgfpathlineto{\pgfqpoint{1.589052in}{0.873250in}}%
\pgfpathlineto{\pgfqpoint{1.589784in}{0.893888in}}%
\pgfpathlineto{\pgfqpoint{1.590028in}{0.837135in}}%
\pgfpathlineto{\pgfqpoint{1.590272in}{0.837135in}}%
\pgfpathlineto{\pgfqpoint{1.590760in}{0.878410in}}%
\pgfpathlineto{\pgfqpoint{1.591248in}{0.826816in}}%
\pgfpathlineto{\pgfqpoint{1.591492in}{0.826816in}}%
\pgfpathlineto{\pgfqpoint{1.591492in}{0.847453in}}%
\pgfpathlineto{\pgfqpoint{1.592224in}{0.806178in}}%
\pgfpathlineto{\pgfqpoint{1.592468in}{0.816497in}}%
\pgfpathlineto{\pgfqpoint{1.592711in}{0.816497in}}%
\pgfpathlineto{\pgfqpoint{1.592711in}{0.873250in}}%
\pgfpathlineto{\pgfqpoint{1.592955in}{0.806178in}}%
\pgfpathlineto{\pgfqpoint{1.593687in}{0.847453in}}%
\pgfpathlineto{\pgfqpoint{1.594175in}{0.847453in}}%
\pgfpathlineto{\pgfqpoint{1.594175in}{0.795860in}}%
\pgfpathlineto{\pgfqpoint{1.594663in}{0.862931in}}%
\pgfpathlineto{\pgfqpoint{1.595151in}{0.826816in}}%
\pgfpathlineto{\pgfqpoint{1.595395in}{0.826816in}}%
\pgfpathlineto{\pgfqpoint{1.596127in}{0.873250in}}%
\pgfpathlineto{\pgfqpoint{1.596371in}{0.785541in}}%
\pgfpathlineto{\pgfqpoint{1.596615in}{0.785541in}}%
\pgfpathlineto{\pgfqpoint{1.596859in}{0.847453in}}%
\pgfpathlineto{\pgfqpoint{1.597591in}{0.837135in}}%
\pgfpathlineto{\pgfqpoint{1.598079in}{0.837135in}}%
\pgfpathlineto{\pgfqpoint{1.598079in}{0.826816in}}%
\pgfpathlineto{\pgfqpoint{1.598323in}{0.888728in}}%
\pgfpathlineto{\pgfqpoint{1.599055in}{0.842294in}}%
\pgfpathlineto{\pgfqpoint{1.599298in}{0.842294in}}%
\pgfpathlineto{\pgfqpoint{1.599298in}{0.801019in}}%
\pgfpathlineto{\pgfqpoint{1.600274in}{0.842294in}}%
\pgfpathlineto{\pgfqpoint{1.600518in}{0.842294in}}%
\pgfpathlineto{\pgfqpoint{1.601006in}{0.785541in}}%
\pgfpathlineto{\pgfqpoint{1.601494in}{0.806178in}}%
\pgfpathlineto{\pgfqpoint{1.601738in}{0.806178in}}%
\pgfpathlineto{\pgfqpoint{1.601738in}{0.826816in}}%
\pgfpathlineto{\pgfqpoint{1.602226in}{0.801019in}}%
\pgfpathlineto{\pgfqpoint{1.602714in}{0.801019in}}%
\pgfpathlineto{\pgfqpoint{1.603690in}{0.801019in}}%
\pgfpathlineto{\pgfqpoint{1.603690in}{0.816497in}}%
\pgfpathlineto{\pgfqpoint{1.603934in}{0.770063in}}%
\pgfpathlineto{\pgfqpoint{1.604666in}{0.806178in}}%
\pgfpathlineto{\pgfqpoint{1.604910in}{0.806178in}}%
\pgfpathlineto{\pgfqpoint{1.604910in}{0.790700in}}%
\pgfpathlineto{\pgfqpoint{1.605885in}{0.826816in}}%
\pgfpathlineto{\pgfqpoint{1.606129in}{0.826816in}}%
\pgfpathlineto{\pgfqpoint{1.607105in}{0.775222in}}%
\pgfpathlineto{\pgfqpoint{1.607349in}{0.775222in}}%
\pgfpathlineto{\pgfqpoint{1.607593in}{0.806178in}}%
\pgfpathlineto{\pgfqpoint{1.608325in}{0.795860in}}%
\pgfpathlineto{\pgfqpoint{1.608569in}{0.795860in}}%
\pgfpathlineto{\pgfqpoint{1.608813in}{0.764903in}}%
\pgfpathlineto{\pgfqpoint{1.609545in}{0.770063in}}%
\pgfpathlineto{\pgfqpoint{1.609789in}{0.770063in}}%
\pgfpathlineto{\pgfqpoint{1.610033in}{0.806178in}}%
\pgfpathlineto{\pgfqpoint{1.610765in}{0.801019in}}%
\pgfpathlineto{\pgfqpoint{1.611009in}{0.801019in}}%
\pgfpathlineto{\pgfqpoint{1.611009in}{0.775222in}}%
\pgfpathlineto{\pgfqpoint{1.611253in}{0.816497in}}%
\pgfpathlineto{\pgfqpoint{1.611985in}{0.801019in}}%
\pgfpathlineto{\pgfqpoint{1.612472in}{0.801019in}}%
\pgfpathlineto{\pgfqpoint{1.613204in}{0.775222in}}%
\pgfpathlineto{\pgfqpoint{1.612716in}{0.811338in}}%
\pgfpathlineto{\pgfqpoint{1.613448in}{0.806178in}}%
\pgfpathlineto{\pgfqpoint{1.613692in}{0.806178in}}%
\pgfpathlineto{\pgfqpoint{1.614180in}{0.759744in}}%
\pgfpathlineto{\pgfqpoint{1.614668in}{0.775222in}}%
\pgfpathlineto{\pgfqpoint{1.614912in}{0.775222in}}%
\pgfpathlineto{\pgfqpoint{1.614912in}{0.785541in}}%
\pgfpathlineto{\pgfqpoint{1.615156in}{0.764903in}}%
\pgfpathlineto{\pgfqpoint{1.615888in}{0.785541in}}%
\pgfpathlineto{\pgfqpoint{1.616132in}{0.785541in}}%
\pgfpathlineto{\pgfqpoint{1.616132in}{0.775222in}}%
\pgfpathlineto{\pgfqpoint{1.616620in}{0.790700in}}%
\pgfpathlineto{\pgfqpoint{1.617108in}{0.790700in}}%
\pgfpathlineto{\pgfqpoint{1.617352in}{0.790700in}}%
\pgfpathlineto{\pgfqpoint{1.618084in}{0.749425in}}%
\pgfpathlineto{\pgfqpoint{1.618328in}{0.775222in}}%
\pgfpathlineto{\pgfqpoint{1.618572in}{0.775222in}}%
\pgfpathlineto{\pgfqpoint{1.619547in}{0.744266in}}%
\pgfpathlineto{\pgfqpoint{1.619791in}{0.744266in}}%
\pgfpathlineto{\pgfqpoint{1.620035in}{0.790700in}}%
\pgfpathlineto{\pgfqpoint{1.620767in}{0.754585in}}%
\pgfpathlineto{\pgfqpoint{1.621011in}{0.754585in}}%
\pgfpathlineto{\pgfqpoint{1.621011in}{0.785541in}}%
\pgfpathlineto{\pgfqpoint{1.621499in}{0.749425in}}%
\pgfpathlineto{\pgfqpoint{1.621987in}{0.759744in}}%
\pgfpathlineto{\pgfqpoint{1.622719in}{0.759744in}}%
\pgfpathlineto{\pgfqpoint{1.623207in}{0.785541in}}%
\pgfpathlineto{\pgfqpoint{1.623451in}{0.754585in}}%
\pgfpathlineto{\pgfqpoint{1.623695in}{0.754585in}}%
\pgfpathlineto{\pgfqpoint{1.623939in}{0.754585in}}%
\pgfpathlineto{\pgfqpoint{1.624671in}{0.811338in}}%
\pgfpathlineto{\pgfqpoint{1.624183in}{0.744266in}}%
\pgfpathlineto{\pgfqpoint{1.624915in}{0.759744in}}%
\pgfpathlineto{\pgfqpoint{1.625159in}{0.759744in}}%
\pgfpathlineto{\pgfqpoint{1.625159in}{0.728788in}}%
\pgfpathlineto{\pgfqpoint{1.625646in}{0.775222in}}%
\pgfpathlineto{\pgfqpoint{1.626134in}{0.759744in}}%
\pgfpathlineto{\pgfqpoint{1.626378in}{0.759744in}}%
\pgfpathlineto{\pgfqpoint{1.626378in}{0.744266in}}%
\pgfpathlineto{\pgfqpoint{1.626866in}{0.785541in}}%
\pgfpathlineto{\pgfqpoint{1.627354in}{0.749425in}}%
\pgfpathlineto{\pgfqpoint{1.627598in}{0.749425in}}%
\pgfpathlineto{\pgfqpoint{1.628330in}{0.795860in}}%
\pgfpathlineto{\pgfqpoint{1.628574in}{0.744266in}}%
\pgfpathlineto{\pgfqpoint{1.628818in}{0.744266in}}%
\pgfpathlineto{\pgfqpoint{1.629306in}{0.785541in}}%
\pgfpathlineto{\pgfqpoint{1.629794in}{0.759744in}}%
\pgfpathlineto{\pgfqpoint{1.630038in}{0.759744in}}%
\pgfpathlineto{\pgfqpoint{1.630038in}{0.754585in}}%
\pgfpathlineto{\pgfqpoint{1.631014in}{0.770063in}}%
\pgfpathlineto{\pgfqpoint{1.631258in}{0.770063in}}%
\pgfpathlineto{\pgfqpoint{1.631502in}{0.739107in}}%
\pgfpathlineto{\pgfqpoint{1.632233in}{0.739107in}}%
\pgfpathlineto{\pgfqpoint{1.632721in}{0.739107in}}%
\pgfpathlineto{\pgfqpoint{1.633453in}{0.770063in}}%
\pgfpathlineto{\pgfqpoint{1.632965in}{0.733947in}}%
\pgfpathlineto{\pgfqpoint{1.633697in}{0.754585in}}%
\pgfpathlineto{\pgfqpoint{1.634429in}{0.754585in}}%
\pgfpathlineto{\pgfqpoint{1.634429in}{0.739107in}}%
\pgfpathlineto{\pgfqpoint{1.634673in}{0.759744in}}%
\pgfpathlineto{\pgfqpoint{1.635405in}{0.749425in}}%
\pgfpathlineto{\pgfqpoint{1.635893in}{0.749425in}}%
\pgfpathlineto{\pgfqpoint{1.636137in}{0.775222in}}%
\pgfpathlineto{\pgfqpoint{1.636869in}{0.759744in}}%
\pgfpathlineto{\pgfqpoint{1.637113in}{0.759744in}}%
\pgfpathlineto{\pgfqpoint{1.638089in}{0.728788in}}%
\pgfpathlineto{\pgfqpoint{1.638333in}{0.728788in}}%
\pgfpathlineto{\pgfqpoint{1.639064in}{0.759744in}}%
\pgfpathlineto{\pgfqpoint{1.639308in}{0.739107in}}%
\pgfpathlineto{\pgfqpoint{1.639552in}{0.739107in}}%
\pgfpathlineto{\pgfqpoint{1.639552in}{0.759744in}}%
\pgfpathlineto{\pgfqpoint{1.640528in}{0.749425in}}%
\pgfpathlineto{\pgfqpoint{1.640772in}{0.749425in}}%
\pgfpathlineto{\pgfqpoint{1.640772in}{0.739107in}}%
\pgfpathlineto{\pgfqpoint{1.641016in}{0.764903in}}%
\pgfpathlineto{\pgfqpoint{1.641748in}{0.739107in}}%
\pgfpathlineto{\pgfqpoint{1.641992in}{0.739107in}}%
\pgfpathlineto{\pgfqpoint{1.642724in}{0.770063in}}%
\pgfpathlineto{\pgfqpoint{1.642968in}{0.739107in}}%
\pgfpathlineto{\pgfqpoint{1.643212in}{0.739107in}}%
\pgfpathlineto{\pgfqpoint{1.643212in}{0.733947in}}%
\pgfpathlineto{\pgfqpoint{1.644188in}{0.764903in}}%
\pgfpathlineto{\pgfqpoint{1.644432in}{0.764903in}}%
\pgfpathlineto{\pgfqpoint{1.644432in}{0.739107in}}%
\pgfpathlineto{\pgfqpoint{1.645407in}{0.744266in}}%
\pgfpathlineto{\pgfqpoint{1.645895in}{0.744266in}}%
\pgfpathlineto{\pgfqpoint{1.646383in}{0.754585in}}%
\pgfpathlineto{\pgfqpoint{1.646627in}{0.739107in}}%
\pgfpathlineto{\pgfqpoint{1.646871in}{0.744266in}}%
\pgfpathlineto{\pgfqpoint{1.647115in}{0.744266in}}%
\pgfpathlineto{\pgfqpoint{1.647359in}{0.733947in}}%
\pgfpathlineto{\pgfqpoint{1.647603in}{0.759744in}}%
\pgfpathlineto{\pgfqpoint{1.648091in}{0.744266in}}%
\pgfpathlineto{\pgfqpoint{1.648335in}{0.744266in}}%
\pgfpathlineto{\pgfqpoint{1.648823in}{0.749425in}}%
\pgfpathlineto{\pgfqpoint{1.649311in}{0.733947in}}%
\pgfpathlineto{\pgfqpoint{1.649555in}{0.733947in}}%
\pgfpathlineto{\pgfqpoint{1.650043in}{0.749425in}}%
\pgfpathlineto{\pgfqpoint{1.650531in}{0.739107in}}%
\pgfpathlineto{\pgfqpoint{1.651019in}{0.739107in}}%
\pgfpathlineto{\pgfqpoint{1.651750in}{0.759744in}}%
\pgfpathlineto{\pgfqpoint{1.651263in}{0.733947in}}%
\pgfpathlineto{\pgfqpoint{1.651994in}{0.744266in}}%
\pgfpathlineto{\pgfqpoint{1.652238in}{0.744266in}}%
\pgfpathlineto{\pgfqpoint{1.652970in}{0.733947in}}%
\pgfpathlineto{\pgfqpoint{1.653214in}{0.759744in}}%
\pgfpathlineto{\pgfqpoint{1.653458in}{0.759744in}}%
\pgfpathlineto{\pgfqpoint{1.653702in}{0.744266in}}%
\pgfpathlineto{\pgfqpoint{1.654434in}{0.759744in}}%
\pgfpathlineto{\pgfqpoint{1.654678in}{0.759744in}}%
\pgfpathlineto{\pgfqpoint{1.655166in}{0.733947in}}%
\pgfpathlineto{\pgfqpoint{1.655654in}{0.733947in}}%
\pgfpathlineto{\pgfqpoint{1.655898in}{0.733947in}}%
\pgfpathlineto{\pgfqpoint{1.655898in}{0.754585in}}%
\pgfpathlineto{\pgfqpoint{1.656874in}{0.728788in}}%
\pgfpathlineto{\pgfqpoint{1.657118in}{0.728788in}}%
\pgfpathlineto{\pgfqpoint{1.658093in}{0.759744in}}%
\pgfpathlineto{\pgfqpoint{1.658337in}{0.759744in}}%
\pgfpathlineto{\pgfqpoint{1.658825in}{0.733947in}}%
\pgfpathlineto{\pgfqpoint{1.659313in}{0.744266in}}%
\pgfpathlineto{\pgfqpoint{1.659557in}{0.744266in}}%
\pgfpathlineto{\pgfqpoint{1.659801in}{0.728788in}}%
\pgfpathlineto{\pgfqpoint{1.660289in}{0.764903in}}%
\pgfpathlineto{\pgfqpoint{1.660533in}{0.759744in}}%
\pgfpathlineto{\pgfqpoint{1.660777in}{0.759744in}}%
\pgfpathlineto{\pgfqpoint{1.661265in}{0.739107in}}%
\pgfpathlineto{\pgfqpoint{1.661753in}{0.754585in}}%
\pgfpathlineto{\pgfqpoint{1.661997in}{0.754585in}}%
\pgfpathlineto{\pgfqpoint{1.661997in}{0.733947in}}%
\pgfpathlineto{\pgfqpoint{1.662973in}{0.739107in}}%
\pgfpathlineto{\pgfqpoint{1.663461in}{0.739107in}}%
\pgfpathlineto{\pgfqpoint{1.663705in}{0.764903in}}%
\pgfpathlineto{\pgfqpoint{1.663949in}{0.733947in}}%
\pgfpathlineto{\pgfqpoint{1.664437in}{0.739107in}}%
\pgfpathlineto{\pgfqpoint{1.664680in}{0.739107in}}%
\pgfpathlineto{\pgfqpoint{1.664680in}{0.744266in}}%
\pgfpathlineto{\pgfqpoint{1.665412in}{0.733947in}}%
\pgfpathlineto{\pgfqpoint{1.665656in}{0.744266in}}%
\pgfpathlineto{\pgfqpoint{1.666144in}{0.744266in}}%
\pgfpathlineto{\pgfqpoint{1.666144in}{0.739107in}}%
\pgfpathlineto{\pgfqpoint{1.667120in}{0.754585in}}%
\pgfpathlineto{\pgfqpoint{1.667364in}{0.754585in}}%
\pgfpathlineto{\pgfqpoint{1.667852in}{0.728788in}}%
\pgfpathlineto{\pgfqpoint{1.668340in}{0.733947in}}%
\pgfpathlineto{\pgfqpoint{1.669316in}{0.733947in}}%
\pgfpathlineto{\pgfqpoint{1.670048in}{0.749425in}}%
\pgfpathlineto{\pgfqpoint{1.669804in}{0.728788in}}%
\pgfpathlineto{\pgfqpoint{1.670292in}{0.749425in}}%
\pgfpathlineto{\pgfqpoint{1.670780in}{0.749425in}}%
\pgfpathlineto{\pgfqpoint{1.670780in}{0.733947in}}%
\pgfpathlineto{\pgfqpoint{1.671024in}{0.754585in}}%
\pgfpathlineto{\pgfqpoint{1.671755in}{0.733947in}}%
\pgfpathlineto{\pgfqpoint{1.671999in}{0.733947in}}%
\pgfpathlineto{\pgfqpoint{1.672731in}{0.759744in}}%
\pgfpathlineto{\pgfqpoint{1.672975in}{0.754585in}}%
\pgfpathlineto{\pgfqpoint{1.673219in}{0.754585in}}%
\pgfpathlineto{\pgfqpoint{1.673219in}{0.733947in}}%
\pgfpathlineto{\pgfqpoint{1.674195in}{0.749425in}}%
\pgfpathlineto{\pgfqpoint{1.674439in}{0.749425in}}%
\pgfpathlineto{\pgfqpoint{1.674439in}{0.744266in}}%
\pgfpathlineto{\pgfqpoint{1.674683in}{0.754585in}}%
\pgfpathlineto{\pgfqpoint{1.675415in}{0.749425in}}%
\pgfpathlineto{\pgfqpoint{1.675659in}{0.749425in}}%
\pgfpathlineto{\pgfqpoint{1.676391in}{0.733947in}}%
\pgfpathlineto{\pgfqpoint{1.675903in}{0.754585in}}%
\pgfpathlineto{\pgfqpoint{1.676635in}{0.744266in}}%
\pgfpathlineto{\pgfqpoint{1.676879in}{0.744266in}}%
\pgfpathlineto{\pgfqpoint{1.676879in}{0.754585in}}%
\pgfpathlineto{\pgfqpoint{1.677123in}{0.739107in}}%
\pgfpathlineto{\pgfqpoint{1.677854in}{0.739107in}}%
\pgfpathlineto{\pgfqpoint{1.678098in}{0.739107in}}%
\pgfpathlineto{\pgfqpoint{1.678098in}{0.749425in}}%
\pgfpathlineto{\pgfqpoint{1.678586in}{0.733947in}}%
\pgfpathlineto{\pgfqpoint{1.679074in}{0.739107in}}%
\pgfpathlineto{\pgfqpoint{1.679562in}{0.739107in}}%
\pgfpathlineto{\pgfqpoint{1.679562in}{0.744266in}}%
\pgfpathlineto{\pgfqpoint{1.680294in}{0.733947in}}%
\pgfpathlineto{\pgfqpoint{1.680538in}{0.733947in}}%
\pgfpathlineto{\pgfqpoint{1.680782in}{0.733947in}}%
\pgfpathlineto{\pgfqpoint{1.681514in}{0.749425in}}%
\pgfpathlineto{\pgfqpoint{1.681758in}{0.739107in}}%
\pgfpathlineto{\pgfqpoint{1.682002in}{0.739107in}}%
\pgfpathlineto{\pgfqpoint{1.682002in}{0.770063in}}%
\pgfpathlineto{\pgfqpoint{1.682978in}{0.728788in}}%
\pgfpathlineto{\pgfqpoint{1.683222in}{0.728788in}}%
\pgfpathlineto{\pgfqpoint{1.683954in}{0.759744in}}%
\pgfpathlineto{\pgfqpoint{1.684198in}{0.744266in}}%
\pgfpathlineto{\pgfqpoint{1.684441in}{0.744266in}}%
\pgfpathlineto{\pgfqpoint{1.684441in}{0.775222in}}%
\pgfpathlineto{\pgfqpoint{1.684685in}{0.739107in}}%
\pgfpathlineto{\pgfqpoint{1.685417in}{0.744266in}}%
\pgfpathlineto{\pgfqpoint{1.685905in}{0.744266in}}%
\pgfpathlineto{\pgfqpoint{1.685905in}{0.728788in}}%
\pgfpathlineto{\pgfqpoint{1.686881in}{0.749425in}}%
\pgfpathlineto{\pgfqpoint{1.687125in}{0.749425in}}%
\pgfpathlineto{\pgfqpoint{1.687125in}{0.739107in}}%
\pgfpathlineto{\pgfqpoint{1.687857in}{0.770063in}}%
\pgfpathlineto{\pgfqpoint{1.688101in}{0.739107in}}%
\pgfpathlineto{\pgfqpoint{1.688345in}{0.739107in}}%
\pgfpathlineto{\pgfqpoint{1.688345in}{0.728788in}}%
\pgfpathlineto{\pgfqpoint{1.688833in}{0.749425in}}%
\pgfpathlineto{\pgfqpoint{1.689321in}{0.728788in}}%
\pgfpathlineto{\pgfqpoint{1.689565in}{0.728788in}}%
\pgfpathlineto{\pgfqpoint{1.690297in}{0.759744in}}%
\pgfpathlineto{\pgfqpoint{1.690541in}{0.744266in}}%
\pgfpathlineto{\pgfqpoint{1.691028in}{0.744266in}}%
\pgfpathlineto{\pgfqpoint{1.691028in}{0.739107in}}%
\pgfpathlineto{\pgfqpoint{1.691760in}{0.754585in}}%
\pgfpathlineto{\pgfqpoint{1.692004in}{0.739107in}}%
\pgfpathlineto{\pgfqpoint{1.692492in}{0.739107in}}%
\pgfpathlineto{\pgfqpoint{1.692736in}{0.728788in}}%
\pgfpathlineto{\pgfqpoint{1.692980in}{0.744266in}}%
\pgfpathlineto{\pgfqpoint{1.693468in}{0.739107in}}%
\pgfpathlineto{\pgfqpoint{1.693712in}{0.739107in}}%
\pgfpathlineto{\pgfqpoint{1.693712in}{0.728788in}}%
\pgfpathlineto{\pgfqpoint{1.694688in}{0.749425in}}%
\pgfpathlineto{\pgfqpoint{1.694932in}{0.749425in}}%
\pgfpathlineto{\pgfqpoint{1.694932in}{0.733947in}}%
\pgfpathlineto{\pgfqpoint{1.695908in}{0.744266in}}%
\pgfpathlineto{\pgfqpoint{1.696152in}{0.744266in}}%
\pgfpathlineto{\pgfqpoint{1.696640in}{0.728788in}}%
\pgfpathlineto{\pgfqpoint{1.697128in}{0.749425in}}%
\pgfpathlineto{\pgfqpoint{1.697371in}{0.749425in}}%
\pgfpathlineto{\pgfqpoint{1.697859in}{0.728788in}}%
\pgfpathlineto{\pgfqpoint{1.698347in}{0.728788in}}%
\pgfpathlineto{\pgfqpoint{1.698591in}{0.728788in}}%
\pgfpathlineto{\pgfqpoint{1.698835in}{0.744266in}}%
\pgfpathlineto{\pgfqpoint{1.699567in}{0.744266in}}%
\pgfpathlineto{\pgfqpoint{1.699811in}{0.744266in}}%
\pgfpathlineto{\pgfqpoint{1.699811in}{0.733947in}}%
\pgfpathlineto{\pgfqpoint{1.700543in}{0.754585in}}%
\pgfpathlineto{\pgfqpoint{1.700787in}{0.733947in}}%
\pgfpathlineto{\pgfqpoint{1.701275in}{0.733947in}}%
\pgfpathlineto{\pgfqpoint{1.701275in}{0.744266in}}%
\pgfpathlineto{\pgfqpoint{1.701763in}{0.728788in}}%
\pgfpathlineto{\pgfqpoint{1.702251in}{0.733947in}}%
\pgfpathlineto{\pgfqpoint{1.702495in}{0.733947in}}%
\pgfpathlineto{\pgfqpoint{1.702495in}{0.728788in}}%
\pgfpathlineto{\pgfqpoint{1.702983in}{0.749425in}}%
\pgfpathlineto{\pgfqpoint{1.703471in}{0.744266in}}%
\pgfpathlineto{\pgfqpoint{1.703715in}{0.744266in}}%
\pgfpathlineto{\pgfqpoint{1.704202in}{0.728788in}}%
\pgfpathlineto{\pgfqpoint{1.704690in}{0.739107in}}%
\pgfpathlineto{\pgfqpoint{1.704934in}{0.739107in}}%
\pgfpathlineto{\pgfqpoint{1.704934in}{0.733947in}}%
\pgfpathlineto{\pgfqpoint{1.705910in}{0.739107in}}%
\pgfpathlineto{\pgfqpoint{1.706154in}{0.739107in}}%
\pgfpathlineto{\pgfqpoint{1.706154in}{0.733947in}}%
\pgfpathlineto{\pgfqpoint{1.706886in}{0.744266in}}%
\pgfpathlineto{\pgfqpoint{1.707130in}{0.739107in}}%
\pgfpathlineto{\pgfqpoint{1.707374in}{0.739107in}}%
\pgfpathlineto{\pgfqpoint{1.707374in}{0.733947in}}%
\pgfpathlineto{\pgfqpoint{1.708106in}{0.744266in}}%
\pgfpathlineto{\pgfqpoint{1.708350in}{0.744266in}}%
\pgfpathlineto{\pgfqpoint{1.708838in}{0.744266in}}%
\pgfpathlineto{\pgfqpoint{1.708838in}{0.728788in}}%
\pgfpathlineto{\pgfqpoint{1.709814in}{0.728788in}}%
\pgfpathlineto{\pgfqpoint{1.710058in}{0.728788in}}%
\pgfpathlineto{\pgfqpoint{1.710789in}{0.744266in}}%
\pgfpathlineto{\pgfqpoint{1.711033in}{0.739107in}}%
\pgfpathlineto{\pgfqpoint{1.711277in}{0.739107in}}%
\pgfpathlineto{\pgfqpoint{1.711277in}{0.728788in}}%
\pgfpathlineto{\pgfqpoint{1.712253in}{0.739107in}}%
\pgfpathlineto{\pgfqpoint{1.712497in}{0.739107in}}%
\pgfpathlineto{\pgfqpoint{1.712497in}{0.728788in}}%
\pgfpathlineto{\pgfqpoint{1.713473in}{0.739107in}}%
\pgfpathlineto{\pgfqpoint{1.713717in}{0.739107in}}%
\pgfpathlineto{\pgfqpoint{1.714449in}{0.728788in}}%
\pgfpathlineto{\pgfqpoint{1.713961in}{0.754585in}}%
\pgfpathlineto{\pgfqpoint{1.714693in}{0.733947in}}%
\pgfpathlineto{\pgfqpoint{1.714937in}{0.733947in}}%
\pgfpathlineto{\pgfqpoint{1.715181in}{0.749425in}}%
\pgfpathlineto{\pgfqpoint{1.715913in}{0.733947in}}%
\pgfpathlineto{\pgfqpoint{1.716157in}{0.733947in}}%
\pgfpathlineto{\pgfqpoint{1.716157in}{0.744266in}}%
\pgfpathlineto{\pgfqpoint{1.717132in}{0.733947in}}%
\pgfpathlineto{\pgfqpoint{1.717376in}{0.733947in}}%
\pgfpathlineto{\pgfqpoint{1.717376in}{0.744266in}}%
\pgfpathlineto{\pgfqpoint{1.717864in}{0.728788in}}%
\pgfpathlineto{\pgfqpoint{1.718352in}{0.733947in}}%
\pgfpathlineto{\pgfqpoint{1.718596in}{0.733947in}}%
\pgfpathlineto{\pgfqpoint{1.718596in}{0.749425in}}%
\pgfpathlineto{\pgfqpoint{1.719572in}{0.733947in}}%
\pgfpathlineto{\pgfqpoint{1.719816in}{0.733947in}}%
\pgfpathlineto{\pgfqpoint{1.720792in}{0.759744in}}%
\pgfpathlineto{\pgfqpoint{1.721280in}{0.759744in}}%
\pgfpathlineto{\pgfqpoint{1.721280in}{0.733947in}}%
\pgfpathlineto{\pgfqpoint{1.722256in}{0.744266in}}%
\pgfpathlineto{\pgfqpoint{1.722500in}{0.744266in}}%
\pgfpathlineto{\pgfqpoint{1.723475in}{0.728788in}}%
\pgfpathlineto{\pgfqpoint{1.723719in}{0.728788in}}%
\pgfpathlineto{\pgfqpoint{1.723719in}{0.754585in}}%
\pgfpathlineto{\pgfqpoint{1.724695in}{0.739107in}}%
\pgfpathlineto{\pgfqpoint{1.724939in}{0.739107in}}%
\pgfpathlineto{\pgfqpoint{1.724939in}{0.728788in}}%
\pgfpathlineto{\pgfqpoint{1.725915in}{0.754585in}}%
\pgfpathlineto{\pgfqpoint{1.726159in}{0.754585in}}%
\pgfpathlineto{\pgfqpoint{1.726647in}{0.733947in}}%
\pgfpathlineto{\pgfqpoint{1.727135in}{0.744266in}}%
\pgfpathlineto{\pgfqpoint{1.727379in}{0.744266in}}%
\pgfpathlineto{\pgfqpoint{1.727379in}{0.749425in}}%
\pgfpathlineto{\pgfqpoint{1.727867in}{0.728788in}}%
\pgfpathlineto{\pgfqpoint{1.728355in}{0.733947in}}%
\pgfpathlineto{\pgfqpoint{1.728843in}{0.733947in}}%
\pgfpathlineto{\pgfqpoint{1.728843in}{0.728788in}}%
\pgfpathlineto{\pgfqpoint{1.729331in}{0.739107in}}%
\pgfpathlineto{\pgfqpoint{1.729819in}{0.733947in}}%
\pgfpathlineto{\pgfqpoint{1.730306in}{0.733947in}}%
\pgfpathlineto{\pgfqpoint{1.730794in}{0.754585in}}%
\pgfpathlineto{\pgfqpoint{1.731282in}{0.728788in}}%
\pgfpathlineto{\pgfqpoint{1.731526in}{0.728788in}}%
\pgfpathlineto{\pgfqpoint{1.732014in}{0.744266in}}%
\pgfpathlineto{\pgfqpoint{1.732502in}{0.733947in}}%
\pgfpathlineto{\pgfqpoint{1.732746in}{0.733947in}}%
\pgfpathlineto{\pgfqpoint{1.732746in}{0.728788in}}%
\pgfpathlineto{\pgfqpoint{1.732990in}{0.744266in}}%
\pgfpathlineto{\pgfqpoint{1.733722in}{0.739107in}}%
\pgfpathlineto{\pgfqpoint{1.734454in}{0.739107in}}%
\pgfpathlineto{\pgfqpoint{1.734454in}{0.733947in}}%
\pgfpathlineto{\pgfqpoint{1.735430in}{0.759744in}}%
\pgfpathlineto{\pgfqpoint{1.735674in}{0.759744in}}%
\pgfpathlineto{\pgfqpoint{1.735918in}{0.728788in}}%
\pgfpathlineto{\pgfqpoint{1.736649in}{0.733947in}}%
\pgfpathlineto{\pgfqpoint{1.737137in}{0.733947in}}%
\pgfpathlineto{\pgfqpoint{1.737381in}{0.744266in}}%
\pgfpathlineto{\pgfqpoint{1.738113in}{0.728788in}}%
\pgfpathlineto{\pgfqpoint{1.738357in}{0.728788in}}%
\pgfpathlineto{\pgfqpoint{1.738357in}{0.749425in}}%
\pgfpathlineto{\pgfqpoint{1.739333in}{0.733947in}}%
\pgfpathlineto{\pgfqpoint{1.739577in}{0.733947in}}%
\pgfpathlineto{\pgfqpoint{1.740309in}{0.754585in}}%
\pgfpathlineto{\pgfqpoint{1.740553in}{0.733947in}}%
\pgfpathlineto{\pgfqpoint{1.740797in}{0.733947in}}%
\pgfpathlineto{\pgfqpoint{1.740797in}{0.728788in}}%
\pgfpathlineto{\pgfqpoint{1.741041in}{0.739107in}}%
\pgfpathlineto{\pgfqpoint{1.741773in}{0.739107in}}%
\pgfpathlineto{\pgfqpoint{1.742017in}{0.739107in}}%
\pgfpathlineto{\pgfqpoint{1.742017in}{0.733947in}}%
\pgfpathlineto{\pgfqpoint{1.742261in}{0.744266in}}%
\pgfpathlineto{\pgfqpoint{1.742993in}{0.744266in}}%
\pgfpathlineto{\pgfqpoint{1.743236in}{0.744266in}}%
\pgfpathlineto{\pgfqpoint{1.743236in}{0.728788in}}%
\pgfpathlineto{\pgfqpoint{1.744212in}{0.733947in}}%
\pgfpathlineto{\pgfqpoint{1.744456in}{0.733947in}}%
\pgfpathlineto{\pgfqpoint{1.744456in}{0.744266in}}%
\pgfpathlineto{\pgfqpoint{1.745432in}{0.733947in}}%
\pgfpathlineto{\pgfqpoint{1.745676in}{0.733947in}}%
\pgfpathlineto{\pgfqpoint{1.745676in}{0.754585in}}%
\pgfpathlineto{\pgfqpoint{1.746164in}{0.728788in}}%
\pgfpathlineto{\pgfqpoint{1.746652in}{0.749425in}}%
\pgfpathlineto{\pgfqpoint{1.746896in}{0.749425in}}%
\pgfpathlineto{\pgfqpoint{1.746896in}{0.733947in}}%
\pgfpathlineto{\pgfqpoint{1.747872in}{0.739107in}}%
\pgfpathlineto{\pgfqpoint{1.748604in}{0.739107in}}%
\pgfpathlineto{\pgfqpoint{1.748604in}{0.728788in}}%
\pgfpathlineto{\pgfqpoint{1.749092in}{0.754585in}}%
\pgfpathlineto{\pgfqpoint{1.749580in}{0.754585in}}%
\pgfpathlineto{\pgfqpoint{1.749823in}{0.754585in}}%
\pgfpathlineto{\pgfqpoint{1.750311in}{0.728788in}}%
\pgfpathlineto{\pgfqpoint{1.750799in}{0.733947in}}%
\pgfpathlineto{\pgfqpoint{1.751043in}{0.733947in}}%
\pgfpathlineto{\pgfqpoint{1.751287in}{0.749425in}}%
\pgfpathlineto{\pgfqpoint{1.752019in}{0.739107in}}%
\pgfpathlineto{\pgfqpoint{1.752995in}{0.739107in}}%
\pgfpathlineto{\pgfqpoint{1.752995in}{0.744266in}}%
\pgfpathlineto{\pgfqpoint{1.753971in}{0.733947in}}%
\pgfpathlineto{\pgfqpoint{1.754215in}{0.733947in}}%
\pgfpathlineto{\pgfqpoint{1.754215in}{0.728788in}}%
\pgfpathlineto{\pgfqpoint{1.754703in}{0.739107in}}%
\pgfpathlineto{\pgfqpoint{1.755191in}{0.739107in}}%
\pgfpathlineto{\pgfqpoint{1.755435in}{0.739107in}}%
\pgfpathlineto{\pgfqpoint{1.755923in}{0.728788in}}%
\pgfpathlineto{\pgfqpoint{1.756410in}{0.759744in}}%
\pgfpathlineto{\pgfqpoint{1.756654in}{0.759744in}}%
\pgfpathlineto{\pgfqpoint{1.757386in}{0.728788in}}%
\pgfpathlineto{\pgfqpoint{1.757630in}{0.739107in}}%
\pgfpathlineto{\pgfqpoint{1.758118in}{0.739107in}}%
\pgfpathlineto{\pgfqpoint{1.758118in}{0.749425in}}%
\pgfpathlineto{\pgfqpoint{1.759094in}{0.739107in}}%
\pgfpathlineto{\pgfqpoint{1.759338in}{0.739107in}}%
\pgfpathlineto{\pgfqpoint{1.759582in}{0.759744in}}%
\pgfpathlineto{\pgfqpoint{1.760314in}{0.728788in}}%
\pgfpathlineto{\pgfqpoint{1.760558in}{0.728788in}}%
\pgfpathlineto{\pgfqpoint{1.760558in}{0.739107in}}%
\pgfpathlineto{\pgfqpoint{1.761534in}{0.728788in}}%
\pgfpathlineto{\pgfqpoint{1.761778in}{0.728788in}}%
\pgfpathlineto{\pgfqpoint{1.762753in}{0.744266in}}%
\pgfpathlineto{\pgfqpoint{1.762997in}{0.744266in}}%
\pgfpathlineto{\pgfqpoint{1.762997in}{0.733947in}}%
\pgfpathlineto{\pgfqpoint{1.763485in}{0.754585in}}%
\pgfpathlineto{\pgfqpoint{1.763973in}{0.733947in}}%
\pgfpathlineto{\pgfqpoint{1.764217in}{0.733947in}}%
\pgfpathlineto{\pgfqpoint{1.764217in}{0.739107in}}%
\pgfpathlineto{\pgfqpoint{1.764461in}{0.728788in}}%
\pgfpathlineto{\pgfqpoint{1.765193in}{0.739107in}}%
\pgfpathlineto{\pgfqpoint{1.765437in}{0.739107in}}%
\pgfpathlineto{\pgfqpoint{1.765925in}{0.749425in}}%
\pgfpathlineto{\pgfqpoint{1.765681in}{0.733947in}}%
\pgfpathlineto{\pgfqpoint{1.766413in}{0.744266in}}%
\pgfpathlineto{\pgfqpoint{1.766657in}{0.744266in}}%
\pgfpathlineto{\pgfqpoint{1.766901in}{0.728788in}}%
\pgfpathlineto{\pgfqpoint{1.767145in}{0.749425in}}%
\pgfpathlineto{\pgfqpoint{1.767633in}{0.733947in}}%
\pgfpathlineto{\pgfqpoint{1.767877in}{0.733947in}}%
\pgfpathlineto{\pgfqpoint{1.768609in}{0.759744in}}%
\pgfpathlineto{\pgfqpoint{1.768853in}{0.733947in}}%
\pgfpathlineto{\pgfqpoint{1.769340in}{0.733947in}}%
\pgfpathlineto{\pgfqpoint{1.769584in}{0.744266in}}%
\pgfpathlineto{\pgfqpoint{1.769828in}{0.728788in}}%
\pgfpathlineto{\pgfqpoint{1.770316in}{0.744266in}}%
\pgfpathlineto{\pgfqpoint{1.770560in}{0.744266in}}%
\pgfpathlineto{\pgfqpoint{1.770804in}{0.728788in}}%
\pgfpathlineto{\pgfqpoint{1.771536in}{0.744266in}}%
\pgfpathlineto{\pgfqpoint{1.771780in}{0.744266in}}%
\pgfpathlineto{\pgfqpoint{1.771780in}{0.749425in}}%
\pgfpathlineto{\pgfqpoint{1.772512in}{0.728788in}}%
\pgfpathlineto{\pgfqpoint{1.772756in}{0.733947in}}%
\pgfpathlineto{\pgfqpoint{1.773000in}{0.733947in}}%
\pgfpathlineto{\pgfqpoint{1.773488in}{0.749425in}}%
\pgfpathlineto{\pgfqpoint{1.773976in}{0.733947in}}%
\pgfpathlineto{\pgfqpoint{1.774220in}{0.733947in}}%
\pgfpathlineto{\pgfqpoint{1.774708in}{0.744266in}}%
\pgfpathlineto{\pgfqpoint{1.775196in}{0.733947in}}%
\pgfpathlineto{\pgfqpoint{1.775440in}{0.733947in}}%
\pgfpathlineto{\pgfqpoint{1.775440in}{0.728788in}}%
\pgfpathlineto{\pgfqpoint{1.775684in}{0.744266in}}%
\pgfpathlineto{\pgfqpoint{1.776415in}{0.744266in}}%
\pgfpathlineto{\pgfqpoint{1.776659in}{0.744266in}}%
\pgfpathlineto{\pgfqpoint{1.777635in}{0.728788in}}%
\pgfpathlineto{\pgfqpoint{1.777879in}{0.728788in}}%
\pgfpathlineto{\pgfqpoint{1.778123in}{0.744266in}}%
\pgfpathlineto{\pgfqpoint{1.778855in}{0.744266in}}%
\pgfpathlineto{\pgfqpoint{1.779099in}{0.744266in}}%
\pgfpathlineto{\pgfqpoint{1.779587in}{0.749425in}}%
\pgfpathlineto{\pgfqpoint{1.780075in}{0.733947in}}%
\pgfpathlineto{\pgfqpoint{1.780319in}{0.733947in}}%
\pgfpathlineto{\pgfqpoint{1.780319in}{0.754585in}}%
\pgfpathlineto{\pgfqpoint{1.781295in}{0.733947in}}%
\pgfpathlineto{\pgfqpoint{1.781539in}{0.733947in}}%
\pgfpathlineto{\pgfqpoint{1.782027in}{0.749425in}}%
\pgfpathlineto{\pgfqpoint{1.782514in}{0.733947in}}%
\pgfpathlineto{\pgfqpoint{1.782758in}{0.733947in}}%
\pgfpathlineto{\pgfqpoint{1.782758in}{0.754585in}}%
\pgfpathlineto{\pgfqpoint{1.783490in}{0.728788in}}%
\pgfpathlineto{\pgfqpoint{1.783734in}{0.744266in}}%
\pgfpathlineto{\pgfqpoint{1.783978in}{0.744266in}}%
\pgfpathlineto{\pgfqpoint{1.784710in}{0.733947in}}%
\pgfpathlineto{\pgfqpoint{1.784954in}{0.744266in}}%
\pgfpathlineto{\pgfqpoint{1.785198in}{0.744266in}}%
\pgfpathlineto{\pgfqpoint{1.785198in}{0.728788in}}%
\pgfpathlineto{\pgfqpoint{1.785442in}{0.749425in}}%
\pgfpathlineto{\pgfqpoint{1.786174in}{0.744266in}}%
\pgfpathlineto{\pgfqpoint{1.786418in}{0.744266in}}%
\pgfpathlineto{\pgfqpoint{1.786418in}{0.733947in}}%
\pgfpathlineto{\pgfqpoint{1.787394in}{0.733947in}}%
\pgfpathlineto{\pgfqpoint{1.787638in}{0.733947in}}%
\pgfpathlineto{\pgfqpoint{1.787638in}{0.744266in}}%
\pgfpathlineto{\pgfqpoint{1.788614in}{0.733947in}}%
\pgfpathlineto{\pgfqpoint{1.788858in}{0.733947in}}%
\pgfpathlineto{\pgfqpoint{1.789101in}{0.749425in}}%
\pgfpathlineto{\pgfqpoint{1.789345in}{0.728788in}}%
\pgfpathlineto{\pgfqpoint{1.789833in}{0.739107in}}%
\pgfpathlineto{\pgfqpoint{1.790565in}{0.739107in}}%
\pgfpathlineto{\pgfqpoint{1.790565in}{0.744266in}}%
\pgfpathlineto{\pgfqpoint{1.790809in}{0.733947in}}%
\pgfpathlineto{\pgfqpoint{1.791541in}{0.739107in}}%
\pgfpathlineto{\pgfqpoint{1.791785in}{0.739107in}}%
\pgfpathlineto{\pgfqpoint{1.792517in}{0.728788in}}%
\pgfpathlineto{\pgfqpoint{1.792761in}{0.754585in}}%
\pgfpathlineto{\pgfqpoint{1.793005in}{0.754585in}}%
\pgfpathlineto{\pgfqpoint{1.793493in}{0.733947in}}%
\pgfpathlineto{\pgfqpoint{1.793981in}{0.739107in}}%
\pgfpathlineto{\pgfqpoint{1.794469in}{0.739107in}}%
\pgfpathlineto{\pgfqpoint{1.794469in}{0.733947in}}%
\pgfpathlineto{\pgfqpoint{1.794957in}{0.749425in}}%
\pgfpathlineto{\pgfqpoint{1.795444in}{0.739107in}}%
\pgfpathlineto{\pgfqpoint{1.796420in}{0.739107in}}%
\pgfpathlineto{\pgfqpoint{1.796420in}{0.733947in}}%
\pgfpathlineto{\pgfqpoint{1.797152in}{0.744266in}}%
\pgfpathlineto{\pgfqpoint{1.797396in}{0.744266in}}%
\pgfpathlineto{\pgfqpoint{1.797640in}{0.744266in}}%
\pgfpathlineto{\pgfqpoint{1.797640in}{0.739107in}}%
\pgfpathlineto{\pgfqpoint{1.798616in}{0.739107in}}%
\pgfpathlineto{\pgfqpoint{1.798860in}{0.739107in}}%
\pgfpathlineto{\pgfqpoint{1.799104in}{0.728788in}}%
\pgfpathlineto{\pgfqpoint{1.799836in}{0.728788in}}%
\pgfpathlineto{\pgfqpoint{1.800080in}{0.728788in}}%
\pgfpathlineto{\pgfqpoint{1.800812in}{0.744266in}}%
\pgfpathlineto{\pgfqpoint{1.801056in}{0.728788in}}%
\pgfpathlineto{\pgfqpoint{1.801300in}{0.728788in}}%
\pgfpathlineto{\pgfqpoint{1.801788in}{0.744266in}}%
\pgfpathlineto{\pgfqpoint{1.802275in}{0.744266in}}%
\pgfpathlineto{\pgfqpoint{1.802519in}{0.744266in}}%
\pgfpathlineto{\pgfqpoint{1.803251in}{0.728788in}}%
\pgfpathlineto{\pgfqpoint{1.802763in}{0.749425in}}%
\pgfpathlineto{\pgfqpoint{1.803495in}{0.733947in}}%
\pgfpathlineto{\pgfqpoint{1.803739in}{0.733947in}}%
\pgfpathlineto{\pgfqpoint{1.803739in}{0.749425in}}%
\pgfpathlineto{\pgfqpoint{1.804715in}{0.739107in}}%
\pgfpathlineto{\pgfqpoint{1.805203in}{0.739107in}}%
\pgfpathlineto{\pgfqpoint{1.805203in}{0.733947in}}%
\pgfpathlineto{\pgfqpoint{1.805447in}{0.749425in}}%
\pgfpathlineto{\pgfqpoint{1.806179in}{0.744266in}}%
\pgfpathlineto{\pgfqpoint{1.806423in}{0.744266in}}%
\pgfpathlineto{\pgfqpoint{1.806667in}{0.733947in}}%
\pgfpathlineto{\pgfqpoint{1.807399in}{0.739107in}}%
\pgfpathlineto{\pgfqpoint{1.807643in}{0.739107in}}%
\pgfpathlineto{\pgfqpoint{1.807643in}{0.744266in}}%
\pgfpathlineto{\pgfqpoint{1.808131in}{0.733947in}}%
\pgfpathlineto{\pgfqpoint{1.808618in}{0.744266in}}%
\pgfpathlineto{\pgfqpoint{1.808862in}{0.744266in}}%
\pgfpathlineto{\pgfqpoint{1.808862in}{0.728788in}}%
\pgfpathlineto{\pgfqpoint{1.809106in}{0.754585in}}%
\pgfpathlineto{\pgfqpoint{1.809838in}{0.744266in}}%
\pgfpathlineto{\pgfqpoint{1.810082in}{0.744266in}}%
\pgfpathlineto{\pgfqpoint{1.810326in}{0.728788in}}%
\pgfpathlineto{\pgfqpoint{1.811058in}{0.728788in}}%
\pgfpathlineto{\pgfqpoint{1.811546in}{0.728788in}}%
\pgfpathlineto{\pgfqpoint{1.812522in}{0.744266in}}%
\pgfpathlineto{\pgfqpoint{1.813010in}{0.744266in}}%
\pgfpathlineto{\pgfqpoint{1.813010in}{0.733947in}}%
\pgfpathlineto{\pgfqpoint{1.813498in}{0.754585in}}%
\pgfpathlineto{\pgfqpoint{1.813986in}{0.739107in}}%
\pgfpathlineto{\pgfqpoint{1.814474in}{0.739107in}}%
\pgfpathlineto{\pgfqpoint{1.814718in}{0.733947in}}%
\pgfpathlineto{\pgfqpoint{1.815449in}{0.749425in}}%
\pgfpathlineto{\pgfqpoint{1.815693in}{0.749425in}}%
\pgfpathlineto{\pgfqpoint{1.815937in}{0.728788in}}%
\pgfpathlineto{\pgfqpoint{1.816181in}{0.764903in}}%
\pgfpathlineto{\pgfqpoint{1.816669in}{0.749425in}}%
\pgfpathlineto{\pgfqpoint{1.816913in}{0.749425in}}%
\pgfpathlineto{\pgfqpoint{1.816913in}{0.728788in}}%
\pgfpathlineto{\pgfqpoint{1.817889in}{0.739107in}}%
\pgfpathlineto{\pgfqpoint{1.818133in}{0.739107in}}%
\pgfpathlineto{\pgfqpoint{1.818133in}{0.728788in}}%
\pgfpathlineto{\pgfqpoint{1.819109in}{0.739107in}}%
\pgfpathlineto{\pgfqpoint{1.819597in}{0.739107in}}%
\pgfpathlineto{\pgfqpoint{1.819597in}{0.744266in}}%
\pgfpathlineto{\pgfqpoint{1.820085in}{0.728788in}}%
\pgfpathlineto{\pgfqpoint{1.820573in}{0.744266in}}%
\pgfpathlineto{\pgfqpoint{1.820817in}{0.744266in}}%
\pgfpathlineto{\pgfqpoint{1.821792in}{0.728788in}}%
\pgfpathlineto{\pgfqpoint{1.822036in}{0.728788in}}%
\pgfpathlineto{\pgfqpoint{1.822280in}{0.744266in}}%
\pgfpathlineto{\pgfqpoint{1.823012in}{0.728788in}}%
\pgfpathlineto{\pgfqpoint{1.823256in}{0.728788in}}%
\pgfpathlineto{\pgfqpoint{1.823500in}{0.775222in}}%
\pgfpathlineto{\pgfqpoint{1.824232in}{0.744266in}}%
\pgfpathlineto{\pgfqpoint{1.824476in}{0.744266in}}%
\pgfpathlineto{\pgfqpoint{1.824476in}{0.733947in}}%
\pgfpathlineto{\pgfqpoint{1.824720in}{0.749425in}}%
\pgfpathlineto{\pgfqpoint{1.825452in}{0.733947in}}%
\pgfpathlineto{\pgfqpoint{1.825696in}{0.733947in}}%
\pgfpathlineto{\pgfqpoint{1.825696in}{0.728788in}}%
\pgfpathlineto{\pgfqpoint{1.825940in}{0.739107in}}%
\pgfpathlineto{\pgfqpoint{1.826672in}{0.739107in}}%
\pgfpathlineto{\pgfqpoint{1.827160in}{0.739107in}}%
\pgfpathlineto{\pgfqpoint{1.827160in}{0.749425in}}%
\pgfpathlineto{\pgfqpoint{1.827404in}{0.733947in}}%
\pgfpathlineto{\pgfqpoint{1.828135in}{0.739107in}}%
\pgfpathlineto{\pgfqpoint{1.828379in}{0.739107in}}%
\pgfpathlineto{\pgfqpoint{1.828623in}{0.728788in}}%
\pgfpathlineto{\pgfqpoint{1.829355in}{0.733947in}}%
\pgfpathlineto{\pgfqpoint{1.829599in}{0.733947in}}%
\pgfpathlineto{\pgfqpoint{1.830575in}{0.754585in}}%
\pgfpathlineto{\pgfqpoint{1.830819in}{0.754585in}}%
\pgfpathlineto{\pgfqpoint{1.831307in}{0.733947in}}%
\pgfpathlineto{\pgfqpoint{1.831795in}{0.733947in}}%
\pgfpathlineto{\pgfqpoint{1.832039in}{0.733947in}}%
\pgfpathlineto{\pgfqpoint{1.832771in}{0.744266in}}%
\pgfpathlineto{\pgfqpoint{1.833015in}{0.728788in}}%
\pgfpathlineto{\pgfqpoint{1.833747in}{0.728788in}}%
\pgfpathlineto{\pgfqpoint{1.834235in}{0.749425in}}%
\pgfpathlineto{\pgfqpoint{1.834722in}{0.733947in}}%
\pgfpathlineto{\pgfqpoint{1.834966in}{0.733947in}}%
\pgfpathlineto{\pgfqpoint{1.834966in}{0.744266in}}%
\pgfpathlineto{\pgfqpoint{1.835942in}{0.744266in}}%
\pgfpathlineto{\pgfqpoint{1.836186in}{0.744266in}}%
\pgfpathlineto{\pgfqpoint{1.836186in}{0.749425in}}%
\pgfpathlineto{\pgfqpoint{1.837162in}{0.728788in}}%
\pgfpathlineto{\pgfqpoint{1.837406in}{0.728788in}}%
\pgfpathlineto{\pgfqpoint{1.837406in}{0.744266in}}%
\pgfpathlineto{\pgfqpoint{1.838382in}{0.733947in}}%
\pgfpathlineto{\pgfqpoint{1.838870in}{0.733947in}}%
\pgfpathlineto{\pgfqpoint{1.838870in}{0.728788in}}%
\pgfpathlineto{\pgfqpoint{1.839114in}{0.754585in}}%
\pgfpathlineto{\pgfqpoint{1.839846in}{0.739107in}}%
\pgfpathlineto{\pgfqpoint{1.840334in}{0.739107in}}%
\pgfpathlineto{\pgfqpoint{1.840334in}{0.749425in}}%
\pgfpathlineto{\pgfqpoint{1.841309in}{0.728788in}}%
\pgfpathlineto{\pgfqpoint{1.841553in}{0.728788in}}%
\pgfpathlineto{\pgfqpoint{1.841553in}{0.739107in}}%
\pgfpathlineto{\pgfqpoint{1.842529in}{0.733947in}}%
\pgfpathlineto{\pgfqpoint{1.842773in}{0.733947in}}%
\pgfpathlineto{\pgfqpoint{1.843261in}{0.728788in}}%
\pgfpathlineto{\pgfqpoint{1.843749in}{0.744266in}}%
\pgfpathlineto{\pgfqpoint{1.844237in}{0.744266in}}%
\pgfpathlineto{\pgfqpoint{1.844237in}{0.728788in}}%
\pgfpathlineto{\pgfqpoint{1.845213in}{0.733947in}}%
\pgfpathlineto{\pgfqpoint{1.845945in}{0.733947in}}%
\pgfpathlineto{\pgfqpoint{1.845945in}{0.744266in}}%
\pgfpathlineto{\pgfqpoint{1.846189in}{0.728788in}}%
\pgfpathlineto{\pgfqpoint{1.846921in}{0.744266in}}%
\pgfpathlineto{\pgfqpoint{1.847409in}{0.744266in}}%
\pgfpathlineto{\pgfqpoint{1.847896in}{0.728788in}}%
\pgfpathlineto{\pgfqpoint{1.848384in}{0.744266in}}%
\pgfpathlineto{\pgfqpoint{1.848628in}{0.744266in}}%
\pgfpathlineto{\pgfqpoint{1.849360in}{0.728788in}}%
\pgfpathlineto{\pgfqpoint{1.849604in}{0.733947in}}%
\pgfpathlineto{\pgfqpoint{1.849848in}{0.733947in}}%
\pgfpathlineto{\pgfqpoint{1.849848in}{0.754585in}}%
\pgfpathlineto{\pgfqpoint{1.850824in}{0.728788in}}%
\pgfpathlineto{\pgfqpoint{1.851068in}{0.728788in}}%
\pgfpathlineto{\pgfqpoint{1.851068in}{0.733947in}}%
\pgfpathlineto{\pgfqpoint{1.852044in}{0.728788in}}%
\pgfpathlineto{\pgfqpoint{1.852288in}{0.728788in}}%
\pgfpathlineto{\pgfqpoint{1.853264in}{0.744266in}}%
\pgfpathlineto{\pgfqpoint{1.853508in}{0.744266in}}%
\pgfpathlineto{\pgfqpoint{1.853508in}{0.749425in}}%
\pgfpathlineto{\pgfqpoint{1.854483in}{0.728788in}}%
\pgfpathlineto{\pgfqpoint{1.854727in}{0.728788in}}%
\pgfpathlineto{\pgfqpoint{1.854727in}{0.744266in}}%
\pgfpathlineto{\pgfqpoint{1.855703in}{0.733947in}}%
\pgfpathlineto{\pgfqpoint{1.855947in}{0.733947in}}%
\pgfpathlineto{\pgfqpoint{1.856435in}{0.749425in}}%
\pgfpathlineto{\pgfqpoint{1.856191in}{0.728788in}}%
\pgfpathlineto{\pgfqpoint{1.856923in}{0.728788in}}%
\pgfpathlineto{\pgfqpoint{1.857167in}{0.728788in}}%
\pgfpathlineto{\pgfqpoint{1.857899in}{0.754585in}}%
\pgfpathlineto{\pgfqpoint{1.858143in}{0.733947in}}%
\pgfpathlineto{\pgfqpoint{1.858387in}{0.733947in}}%
\pgfpathlineto{\pgfqpoint{1.858387in}{0.728788in}}%
\pgfpathlineto{\pgfqpoint{1.858631in}{0.739107in}}%
\pgfpathlineto{\pgfqpoint{1.859363in}{0.739107in}}%
\pgfpathlineto{\pgfqpoint{1.859851in}{0.739107in}}%
\pgfpathlineto{\pgfqpoint{1.860827in}{0.728788in}}%
\pgfpathlineto{\pgfqpoint{1.861070in}{0.728788in}}%
\pgfpathlineto{\pgfqpoint{1.861314in}{0.749425in}}%
\pgfpathlineto{\pgfqpoint{1.862046in}{0.728788in}}%
\pgfpathlineto{\pgfqpoint{1.862290in}{0.728788in}}%
\pgfpathlineto{\pgfqpoint{1.862290in}{0.749425in}}%
\pgfpathlineto{\pgfqpoint{1.863266in}{0.749425in}}%
\pgfpathlineto{\pgfqpoint{1.863510in}{0.749425in}}%
\pgfpathlineto{\pgfqpoint{1.863510in}{0.733947in}}%
\pgfpathlineto{\pgfqpoint{1.864486in}{0.733947in}}%
\pgfpathlineto{\pgfqpoint{1.864730in}{0.733947in}}%
\pgfpathlineto{\pgfqpoint{1.864730in}{0.754585in}}%
\pgfpathlineto{\pgfqpoint{1.865706in}{0.749425in}}%
\pgfpathlineto{\pgfqpoint{1.865950in}{0.749425in}}%
\pgfpathlineto{\pgfqpoint{1.865950in}{0.728788in}}%
\pgfpathlineto{\pgfqpoint{1.866926in}{0.733947in}}%
\pgfpathlineto{\pgfqpoint{1.867170in}{0.733947in}}%
\pgfpathlineto{\pgfqpoint{1.867170in}{0.744266in}}%
\pgfpathlineto{\pgfqpoint{1.867901in}{0.728788in}}%
\pgfpathlineto{\pgfqpoint{1.868145in}{0.739107in}}%
\pgfpathlineto{\pgfqpoint{1.868389in}{0.739107in}}%
\pgfpathlineto{\pgfqpoint{1.868389in}{0.728788in}}%
\pgfpathlineto{\pgfqpoint{1.869365in}{0.744266in}}%
\pgfpathlineto{\pgfqpoint{1.869609in}{0.744266in}}%
\pgfpathlineto{\pgfqpoint{1.869609in}{0.733947in}}%
\pgfpathlineto{\pgfqpoint{1.870585in}{0.739107in}}%
\pgfpathlineto{\pgfqpoint{1.870829in}{0.739107in}}%
\pgfpathlineto{\pgfqpoint{1.870829in}{0.744266in}}%
\pgfpathlineto{\pgfqpoint{1.871805in}{0.728788in}}%
\pgfpathlineto{\pgfqpoint{1.872049in}{0.728788in}}%
\pgfpathlineto{\pgfqpoint{1.872293in}{0.744266in}}%
\pgfpathlineto{\pgfqpoint{1.873025in}{0.744266in}}%
\pgfpathlineto{\pgfqpoint{1.873269in}{0.744266in}}%
\pgfpathlineto{\pgfqpoint{1.873269in}{0.728788in}}%
\pgfpathlineto{\pgfqpoint{1.874244in}{0.739107in}}%
\pgfpathlineto{\pgfqpoint{1.874488in}{0.739107in}}%
\pgfpathlineto{\pgfqpoint{1.874976in}{0.744266in}}%
\pgfpathlineto{\pgfqpoint{1.875464in}{0.728788in}}%
\pgfpathlineto{\pgfqpoint{1.875708in}{0.728788in}}%
\pgfpathlineto{\pgfqpoint{1.875708in}{0.749425in}}%
\pgfpathlineto{\pgfqpoint{1.876684in}{0.733947in}}%
\pgfpathlineto{\pgfqpoint{1.876928in}{0.733947in}}%
\pgfpathlineto{\pgfqpoint{1.876928in}{0.739107in}}%
\pgfpathlineto{\pgfqpoint{1.877172in}{0.728788in}}%
\pgfpathlineto{\pgfqpoint{1.877904in}{0.739107in}}%
\pgfpathlineto{\pgfqpoint{1.878392in}{0.739107in}}%
\pgfpathlineto{\pgfqpoint{1.879368in}{0.754585in}}%
\pgfpathlineto{\pgfqpoint{1.879612in}{0.754585in}}%
\pgfpathlineto{\pgfqpoint{1.879612in}{0.759744in}}%
\pgfpathlineto{\pgfqpoint{1.880100in}{0.733947in}}%
\pgfpathlineto{\pgfqpoint{1.880587in}{0.739107in}}%
\pgfpathlineto{\pgfqpoint{1.880831in}{0.739107in}}%
\pgfpathlineto{\pgfqpoint{1.881075in}{0.728788in}}%
\pgfpathlineto{\pgfqpoint{1.881319in}{0.749425in}}%
\pgfpathlineto{\pgfqpoint{1.881807in}{0.733947in}}%
\pgfpathlineto{\pgfqpoint{1.882051in}{0.733947in}}%
\pgfpathlineto{\pgfqpoint{1.882051in}{0.728788in}}%
\pgfpathlineto{\pgfqpoint{1.882295in}{0.744266in}}%
\pgfpathlineto{\pgfqpoint{1.883027in}{0.728788in}}%
\pgfpathlineto{\pgfqpoint{1.883271in}{0.728788in}}%
\pgfpathlineto{\pgfqpoint{1.884247in}{0.749425in}}%
\pgfpathlineto{\pgfqpoint{1.884491in}{0.749425in}}%
\pgfpathlineto{\pgfqpoint{1.884491in}{0.728788in}}%
\pgfpathlineto{\pgfqpoint{1.885467in}{0.739107in}}%
\pgfpathlineto{\pgfqpoint{1.885711in}{0.739107in}}%
\pgfpathlineto{\pgfqpoint{1.885955in}{0.728788in}}%
\pgfpathlineto{\pgfqpoint{1.886199in}{0.744266in}}%
\pgfpathlineto{\pgfqpoint{1.886687in}{0.744266in}}%
\pgfpathlineto{\pgfqpoint{1.886931in}{0.744266in}}%
\pgfpathlineto{\pgfqpoint{1.886931in}{0.728788in}}%
\pgfpathlineto{\pgfqpoint{1.887906in}{0.739107in}}%
\pgfpathlineto{\pgfqpoint{1.888150in}{0.739107in}}%
\pgfpathlineto{\pgfqpoint{1.888150in}{0.749425in}}%
\pgfpathlineto{\pgfqpoint{1.888638in}{0.733947in}}%
\pgfpathlineto{\pgfqpoint{1.889126in}{0.749425in}}%
\pgfpathlineto{\pgfqpoint{1.889370in}{0.749425in}}%
\pgfpathlineto{\pgfqpoint{1.889614in}{0.754585in}}%
\pgfpathlineto{\pgfqpoint{1.890346in}{0.728788in}}%
\pgfpathlineto{\pgfqpoint{1.890590in}{0.728788in}}%
\pgfpathlineto{\pgfqpoint{1.891322in}{0.754585in}}%
\pgfpathlineto{\pgfqpoint{1.891566in}{0.733947in}}%
\pgfpathlineto{\pgfqpoint{1.891810in}{0.733947in}}%
\pgfpathlineto{\pgfqpoint{1.891810in}{0.728788in}}%
\pgfpathlineto{\pgfqpoint{1.892298in}{0.744266in}}%
\pgfpathlineto{\pgfqpoint{1.892786in}{0.744266in}}%
\pgfpathlineto{\pgfqpoint{1.893030in}{0.744266in}}%
\pgfpathlineto{\pgfqpoint{1.893030in}{0.728788in}}%
\pgfpathlineto{\pgfqpoint{1.894005in}{0.733947in}}%
\pgfpathlineto{\pgfqpoint{1.894249in}{0.733947in}}%
\pgfpathlineto{\pgfqpoint{1.894249in}{0.728788in}}%
\pgfpathlineto{\pgfqpoint{1.894493in}{0.739107in}}%
\pgfpathlineto{\pgfqpoint{1.895225in}{0.733947in}}%
\pgfpathlineto{\pgfqpoint{1.895469in}{0.733947in}}%
\pgfpathlineto{\pgfqpoint{1.895713in}{0.754585in}}%
\pgfpathlineto{\pgfqpoint{1.896445in}{0.744266in}}%
\pgfpathlineto{\pgfqpoint{1.896689in}{0.744266in}}%
\pgfpathlineto{\pgfqpoint{1.896689in}{0.733947in}}%
\pgfpathlineto{\pgfqpoint{1.897665in}{0.733947in}}%
\pgfpathlineto{\pgfqpoint{1.897909in}{0.733947in}}%
\pgfpathlineto{\pgfqpoint{1.897909in}{0.739107in}}%
\pgfpathlineto{\pgfqpoint{1.898153in}{0.728788in}}%
\pgfpathlineto{\pgfqpoint{1.898885in}{0.733947in}}%
\pgfpathlineto{\pgfqpoint{1.899129in}{0.733947in}}%
\pgfpathlineto{\pgfqpoint{1.899129in}{0.728788in}}%
\pgfpathlineto{\pgfqpoint{1.899617in}{0.739107in}}%
\pgfpathlineto{\pgfqpoint{1.900104in}{0.728788in}}%
\pgfpathlineto{\pgfqpoint{1.900348in}{0.728788in}}%
\pgfpathlineto{\pgfqpoint{1.901324in}{0.739107in}}%
\pgfpathlineto{\pgfqpoint{1.901568in}{0.739107in}}%
\pgfpathlineto{\pgfqpoint{1.901568in}{0.749425in}}%
\pgfpathlineto{\pgfqpoint{1.902056in}{0.733947in}}%
\pgfpathlineto{\pgfqpoint{1.902544in}{0.744266in}}%
\pgfpathlineto{\pgfqpoint{1.902788in}{0.744266in}}%
\pgfpathlineto{\pgfqpoint{1.902788in}{0.749425in}}%
\pgfpathlineto{\pgfqpoint{1.903764in}{0.733947in}}%
\pgfpathlineto{\pgfqpoint{1.904008in}{0.733947in}}%
\pgfpathlineto{\pgfqpoint{1.904008in}{0.728788in}}%
\pgfpathlineto{\pgfqpoint{1.904252in}{0.739107in}}%
\pgfpathlineto{\pgfqpoint{1.904984in}{0.728788in}}%
\pgfpathlineto{\pgfqpoint{1.905228in}{0.728788in}}%
\pgfpathlineto{\pgfqpoint{1.905228in}{0.744266in}}%
\pgfpathlineto{\pgfqpoint{1.906204in}{0.728788in}}%
\pgfpathlineto{\pgfqpoint{1.906448in}{0.728788in}}%
\pgfpathlineto{\pgfqpoint{1.906448in}{0.754585in}}%
\pgfpathlineto{\pgfqpoint{1.907423in}{0.733947in}}%
\pgfpathlineto{\pgfqpoint{1.907667in}{0.733947in}}%
\pgfpathlineto{\pgfqpoint{1.907667in}{0.749425in}}%
\pgfpathlineto{\pgfqpoint{1.907911in}{0.728788in}}%
\pgfpathlineto{\pgfqpoint{1.908643in}{0.733947in}}%
\pgfpathlineto{\pgfqpoint{1.909131in}{0.733947in}}%
\pgfpathlineto{\pgfqpoint{1.909131in}{0.728788in}}%
\pgfpathlineto{\pgfqpoint{1.909375in}{0.739107in}}%
\pgfpathlineto{\pgfqpoint{1.910107in}{0.739107in}}%
\pgfpathlineto{\pgfqpoint{1.910351in}{0.739107in}}%
\pgfpathlineto{\pgfqpoint{1.910595in}{0.728788in}}%
\pgfpathlineto{\pgfqpoint{1.911327in}{0.733947in}}%
\pgfpathlineto{\pgfqpoint{1.911571in}{0.733947in}}%
\pgfpathlineto{\pgfqpoint{1.912547in}{0.759744in}}%
\pgfpathlineto{\pgfqpoint{1.912791in}{0.759744in}}%
\pgfpathlineto{\pgfqpoint{1.913035in}{0.728788in}}%
\pgfpathlineto{\pgfqpoint{1.913766in}{0.744266in}}%
\pgfpathlineto{\pgfqpoint{1.914010in}{0.744266in}}%
\pgfpathlineto{\pgfqpoint{1.914010in}{0.759744in}}%
\pgfpathlineto{\pgfqpoint{1.914254in}{0.739107in}}%
\pgfpathlineto{\pgfqpoint{1.914986in}{0.739107in}}%
\pgfpathlineto{\pgfqpoint{1.915230in}{0.739107in}}%
\pgfpathlineto{\pgfqpoint{1.915718in}{0.728788in}}%
\pgfpathlineto{\pgfqpoint{1.916206in}{0.728788in}}%
\pgfpathlineto{\pgfqpoint{1.916450in}{0.728788in}}%
\pgfpathlineto{\pgfqpoint{1.917426in}{0.739107in}}%
\pgfpathlineto{\pgfqpoint{1.917914in}{0.739107in}}%
\pgfpathlineto{\pgfqpoint{1.918158in}{0.728788in}}%
\pgfpathlineto{\pgfqpoint{1.918890in}{0.749425in}}%
\pgfpathlineto{\pgfqpoint{1.919134in}{0.749425in}}%
\pgfpathlineto{\pgfqpoint{1.919134in}{0.728788in}}%
\pgfpathlineto{\pgfqpoint{1.920109in}{0.739107in}}%
\pgfpathlineto{\pgfqpoint{1.920597in}{0.739107in}}%
\pgfpathlineto{\pgfqpoint{1.920597in}{0.733947in}}%
\pgfpathlineto{\pgfqpoint{1.920841in}{0.764903in}}%
\pgfpathlineto{\pgfqpoint{1.921573in}{0.739107in}}%
\pgfpathlineto{\pgfqpoint{1.921817in}{0.739107in}}%
\pgfpathlineto{\pgfqpoint{1.921817in}{0.728788in}}%
\pgfpathlineto{\pgfqpoint{1.922061in}{0.744266in}}%
\pgfpathlineto{\pgfqpoint{1.922793in}{0.739107in}}%
\pgfpathlineto{\pgfqpoint{1.923037in}{0.739107in}}%
\pgfpathlineto{\pgfqpoint{1.923037in}{0.749425in}}%
\pgfpathlineto{\pgfqpoint{1.923281in}{0.728788in}}%
\pgfpathlineto{\pgfqpoint{1.924013in}{0.739107in}}%
\pgfpathlineto{\pgfqpoint{1.924257in}{0.739107in}}%
\pgfpathlineto{\pgfqpoint{1.924989in}{0.749425in}}%
\pgfpathlineto{\pgfqpoint{1.925233in}{0.728788in}}%
\pgfpathlineto{\pgfqpoint{1.925477in}{0.728788in}}%
\pgfpathlineto{\pgfqpoint{1.925721in}{0.744266in}}%
\pgfpathlineto{\pgfqpoint{1.926452in}{0.728788in}}%
\pgfpathlineto{\pgfqpoint{1.926696in}{0.728788in}}%
\pgfpathlineto{\pgfqpoint{1.927428in}{0.749425in}}%
\pgfpathlineto{\pgfqpoint{1.927672in}{0.739107in}}%
\pgfpathlineto{\pgfqpoint{1.927916in}{0.739107in}}%
\pgfpathlineto{\pgfqpoint{1.927916in}{0.754585in}}%
\pgfpathlineto{\pgfqpoint{1.928892in}{0.728788in}}%
\pgfpathlineto{\pgfqpoint{1.929136in}{0.728788in}}%
\pgfpathlineto{\pgfqpoint{1.929380in}{0.744266in}}%
\pgfpathlineto{\pgfqpoint{1.930112in}{0.728788in}}%
\pgfpathlineto{\pgfqpoint{1.930600in}{0.728788in}}%
\pgfpathlineto{\pgfqpoint{1.931332in}{0.744266in}}%
\pgfpathlineto{\pgfqpoint{1.931576in}{0.733947in}}%
\pgfpathlineto{\pgfqpoint{1.931820in}{0.733947in}}%
\pgfpathlineto{\pgfqpoint{1.932552in}{0.744266in}}%
\pgfpathlineto{\pgfqpoint{1.932795in}{0.744266in}}%
\pgfpathlineto{\pgfqpoint{1.933039in}{0.744266in}}%
\pgfpathlineto{\pgfqpoint{1.934015in}{0.728788in}}%
\pgfpathlineto{\pgfqpoint{1.934259in}{0.728788in}}%
\pgfpathlineto{\pgfqpoint{1.935235in}{0.744266in}}%
\pgfpathlineto{\pgfqpoint{1.935479in}{0.744266in}}%
\pgfpathlineto{\pgfqpoint{1.935479in}{0.728788in}}%
\pgfpathlineto{\pgfqpoint{1.936455in}{0.744266in}}%
\pgfpathlineto{\pgfqpoint{1.936699in}{0.744266in}}%
\pgfpathlineto{\pgfqpoint{1.936943in}{0.728788in}}%
\pgfpathlineto{\pgfqpoint{1.937675in}{0.733947in}}%
\pgfpathlineto{\pgfqpoint{1.937919in}{0.733947in}}%
\pgfpathlineto{\pgfqpoint{1.938895in}{0.749425in}}%
\pgfpathlineto{\pgfqpoint{1.939139in}{0.749425in}}%
\pgfpathlineto{\pgfqpoint{1.939139in}{0.728788in}}%
\pgfpathlineto{\pgfqpoint{1.940114in}{0.739107in}}%
\pgfpathlineto{\pgfqpoint{1.940358in}{0.739107in}}%
\pgfpathlineto{\pgfqpoint{1.940358in}{0.728788in}}%
\pgfpathlineto{\pgfqpoint{1.941090in}{0.744266in}}%
\pgfpathlineto{\pgfqpoint{1.941334in}{0.739107in}}%
\pgfpathlineto{\pgfqpoint{1.941578in}{0.739107in}}%
\pgfpathlineto{\pgfqpoint{1.941822in}{0.728788in}}%
\pgfpathlineto{\pgfqpoint{1.942066in}{0.749425in}}%
\pgfpathlineto{\pgfqpoint{1.942554in}{0.749425in}}%
\pgfpathlineto{\pgfqpoint{1.942798in}{0.749425in}}%
\pgfpathlineto{\pgfqpoint{1.943042in}{0.754585in}}%
\pgfpathlineto{\pgfqpoint{1.943774in}{0.733947in}}%
\pgfpathlineto{\pgfqpoint{1.944262in}{0.733947in}}%
\pgfpathlineto{\pgfqpoint{1.944506in}{0.754585in}}%
\pgfpathlineto{\pgfqpoint{1.945238in}{0.733947in}}%
\pgfpathlineto{\pgfqpoint{1.945482in}{0.733947in}}%
\pgfpathlineto{\pgfqpoint{1.945482in}{0.728788in}}%
\pgfpathlineto{\pgfqpoint{1.945726in}{0.754585in}}%
\pgfpathlineto{\pgfqpoint{1.946457in}{0.749425in}}%
\pgfpathlineto{\pgfqpoint{1.946701in}{0.749425in}}%
\pgfpathlineto{\pgfqpoint{1.947189in}{0.733947in}}%
\pgfpathlineto{\pgfqpoint{1.947677in}{0.733947in}}%
\pgfpathlineto{\pgfqpoint{1.947921in}{0.733947in}}%
\pgfpathlineto{\pgfqpoint{1.948653in}{0.728788in}}%
\pgfpathlineto{\pgfqpoint{1.948897in}{0.759744in}}%
\pgfpathlineto{\pgfqpoint{1.949141in}{0.759744in}}%
\pgfpathlineto{\pgfqpoint{1.950117in}{0.728788in}}%
\pgfpathlineto{\pgfqpoint{1.950361in}{0.728788in}}%
\pgfpathlineto{\pgfqpoint{1.951093in}{0.749425in}}%
\pgfpathlineto{\pgfqpoint{1.951337in}{0.739107in}}%
\pgfpathlineto{\pgfqpoint{1.951825in}{0.739107in}}%
\pgfpathlineto{\pgfqpoint{1.951825in}{0.728788in}}%
\pgfpathlineto{\pgfqpoint{1.952069in}{0.759744in}}%
\pgfpathlineto{\pgfqpoint{1.952800in}{0.749425in}}%
\pgfpathlineto{\pgfqpoint{1.953044in}{0.749425in}}%
\pgfpathlineto{\pgfqpoint{1.953288in}{0.733947in}}%
\pgfpathlineto{\pgfqpoint{1.954020in}{0.749425in}}%
\pgfpathlineto{\pgfqpoint{1.954264in}{0.749425in}}%
\pgfpathlineto{\pgfqpoint{1.954996in}{0.728788in}}%
\pgfpathlineto{\pgfqpoint{1.955240in}{0.733947in}}%
\pgfpathlineto{\pgfqpoint{1.955484in}{0.733947in}}%
\pgfpathlineto{\pgfqpoint{1.956216in}{0.744266in}}%
\pgfpathlineto{\pgfqpoint{1.956460in}{0.744266in}}%
\pgfpathlineto{\pgfqpoint{1.956704in}{0.744266in}}%
\pgfpathlineto{\pgfqpoint{1.956704in}{0.733947in}}%
\pgfpathlineto{\pgfqpoint{1.957680in}{0.739107in}}%
\pgfpathlineto{\pgfqpoint{1.958168in}{0.739107in}}%
\pgfpathlineto{\pgfqpoint{1.958168in}{0.749425in}}%
\pgfpathlineto{\pgfqpoint{1.958656in}{0.733947in}}%
\pgfpathlineto{\pgfqpoint{1.959143in}{0.733947in}}%
\pgfpathlineto{\pgfqpoint{1.959387in}{0.733947in}}%
\pgfpathlineto{\pgfqpoint{1.959631in}{0.728788in}}%
\pgfpathlineto{\pgfqpoint{1.960363in}{0.749425in}}%
\pgfpathlineto{\pgfqpoint{1.960607in}{0.749425in}}%
\pgfpathlineto{\pgfqpoint{1.960851in}{0.728788in}}%
\pgfpathlineto{\pgfqpoint{1.961339in}{0.754585in}}%
\pgfpathlineto{\pgfqpoint{1.961583in}{0.728788in}}%
\pgfpathlineto{\pgfqpoint{1.961827in}{0.728788in}}%
\pgfpathlineto{\pgfqpoint{1.962071in}{0.749425in}}%
\pgfpathlineto{\pgfqpoint{1.962803in}{0.733947in}}%
\pgfpathlineto{\pgfqpoint{1.963047in}{0.733947in}}%
\pgfpathlineto{\pgfqpoint{1.963291in}{0.744266in}}%
\pgfpathlineto{\pgfqpoint{1.963779in}{0.728788in}}%
\pgfpathlineto{\pgfqpoint{1.964023in}{0.739107in}}%
\pgfpathlineto{\pgfqpoint{1.964267in}{0.739107in}}%
\pgfpathlineto{\pgfqpoint{1.964267in}{0.749425in}}%
\pgfpathlineto{\pgfqpoint{1.964999in}{0.733947in}}%
\pgfpathlineto{\pgfqpoint{1.965243in}{0.739107in}}%
\pgfpathlineto{\pgfqpoint{1.965487in}{0.739107in}}%
\pgfpathlineto{\pgfqpoint{1.966462in}{0.770063in}}%
\pgfpathlineto{\pgfqpoint{1.966706in}{0.770063in}}%
\pgfpathlineto{\pgfqpoint{1.966706in}{0.728788in}}%
\pgfpathlineto{\pgfqpoint{1.967682in}{0.733947in}}%
\pgfpathlineto{\pgfqpoint{1.968170in}{0.733947in}}%
\pgfpathlineto{\pgfqpoint{1.968170in}{0.744266in}}%
\pgfpathlineto{\pgfqpoint{1.968414in}{0.728788in}}%
\pgfpathlineto{\pgfqpoint{1.969146in}{0.739107in}}%
\pgfpathlineto{\pgfqpoint{1.969390in}{0.739107in}}%
\pgfpathlineto{\pgfqpoint{1.969634in}{0.728788in}}%
\pgfpathlineto{\pgfqpoint{1.970366in}{0.733947in}}%
\pgfpathlineto{\pgfqpoint{1.970610in}{0.733947in}}%
\pgfpathlineto{\pgfqpoint{1.971098in}{0.749425in}}%
\pgfpathlineto{\pgfqpoint{1.971586in}{0.739107in}}%
\pgfpathlineto{\pgfqpoint{1.971830in}{0.739107in}}%
\pgfpathlineto{\pgfqpoint{1.971830in}{0.733947in}}%
\pgfpathlineto{\pgfqpoint{1.972805in}{0.754585in}}%
\pgfpathlineto{\pgfqpoint{1.973049in}{0.754585in}}%
\pgfpathlineto{\pgfqpoint{1.973537in}{0.733947in}}%
\pgfpathlineto{\pgfqpoint{1.974025in}{0.733947in}}%
\pgfpathlineto{\pgfqpoint{1.974513in}{0.733947in}}%
\pgfpathlineto{\pgfqpoint{1.974513in}{0.728788in}}%
\pgfpathlineto{\pgfqpoint{1.975489in}{0.754585in}}%
\pgfpathlineto{\pgfqpoint{1.975733in}{0.754585in}}%
\pgfpathlineto{\pgfqpoint{1.975977in}{0.733947in}}%
\pgfpathlineto{\pgfqpoint{1.976709in}{0.733947in}}%
\pgfpathlineto{\pgfqpoint{1.976953in}{0.733947in}}%
\pgfpathlineto{\pgfqpoint{1.976953in}{0.728788in}}%
\pgfpathlineto{\pgfqpoint{1.977929in}{0.744266in}}%
\pgfpathlineto{\pgfqpoint{1.978173in}{0.744266in}}%
\pgfpathlineto{\pgfqpoint{1.978173in}{0.728788in}}%
\pgfpathlineto{\pgfqpoint{1.979148in}{0.744266in}}%
\pgfpathlineto{\pgfqpoint{1.979392in}{0.744266in}}%
\pgfpathlineto{\pgfqpoint{1.979636in}{0.733947in}}%
\pgfpathlineto{\pgfqpoint{1.980124in}{0.754585in}}%
\pgfpathlineto{\pgfqpoint{1.980368in}{0.744266in}}%
\pgfpathlineto{\pgfqpoint{1.980856in}{0.744266in}}%
\pgfpathlineto{\pgfqpoint{1.980856in}{0.728788in}}%
\pgfpathlineto{\pgfqpoint{1.981832in}{0.739107in}}%
\pgfpathlineto{\pgfqpoint{1.982076in}{0.739107in}}%
\pgfpathlineto{\pgfqpoint{1.982076in}{0.749425in}}%
\pgfpathlineto{\pgfqpoint{1.982320in}{0.728788in}}%
\pgfpathlineto{\pgfqpoint{1.983052in}{0.728788in}}%
\pgfpathlineto{\pgfqpoint{1.983296in}{0.728788in}}%
\pgfpathlineto{\pgfqpoint{1.983784in}{0.739107in}}%
\pgfpathlineto{\pgfqpoint{1.984272in}{0.739107in}}%
\pgfpathlineto{\pgfqpoint{1.984760in}{0.739107in}}%
\pgfpathlineto{\pgfqpoint{1.985491in}{0.749425in}}%
\pgfpathlineto{\pgfqpoint{1.985735in}{0.728788in}}%
\pgfpathlineto{\pgfqpoint{1.985979in}{0.728788in}}%
\pgfpathlineto{\pgfqpoint{1.986711in}{0.744266in}}%
\pgfpathlineto{\pgfqpoint{1.986955in}{0.733947in}}%
\pgfpathlineto{\pgfqpoint{1.987443in}{0.733947in}}%
\pgfpathlineto{\pgfqpoint{1.987443in}{0.728788in}}%
\pgfpathlineto{\pgfqpoint{1.987931in}{0.744266in}}%
\pgfpathlineto{\pgfqpoint{1.988419in}{0.739107in}}%
\pgfpathlineto{\pgfqpoint{1.988663in}{0.739107in}}%
\pgfpathlineto{\pgfqpoint{1.988663in}{0.728788in}}%
\pgfpathlineto{\pgfqpoint{1.989395in}{0.754585in}}%
\pgfpathlineto{\pgfqpoint{1.989639in}{0.739107in}}%
\pgfpathlineto{\pgfqpoint{1.989883in}{0.739107in}}%
\pgfpathlineto{\pgfqpoint{1.989883in}{0.728788in}}%
\pgfpathlineto{\pgfqpoint{1.990371in}{0.744266in}}%
\pgfpathlineto{\pgfqpoint{1.990859in}{0.728788in}}%
\pgfpathlineto{\pgfqpoint{1.991103in}{0.728788in}}%
\pgfpathlineto{\pgfqpoint{1.991103in}{0.739107in}}%
\pgfpathlineto{\pgfqpoint{1.992078in}{0.739107in}}%
\pgfpathlineto{\pgfqpoint{1.992322in}{0.739107in}}%
\pgfpathlineto{\pgfqpoint{1.993298in}{0.728788in}}%
\pgfpathlineto{\pgfqpoint{1.993542in}{0.728788in}}%
\pgfpathlineto{\pgfqpoint{1.994274in}{0.749425in}}%
\pgfpathlineto{\pgfqpoint{1.994518in}{0.728788in}}%
\pgfpathlineto{\pgfqpoint{1.994762in}{0.728788in}}%
\pgfpathlineto{\pgfqpoint{1.994762in}{0.749425in}}%
\pgfpathlineto{\pgfqpoint{1.995738in}{0.733947in}}%
\pgfpathlineto{\pgfqpoint{1.996226in}{0.733947in}}%
\pgfpathlineto{\pgfqpoint{1.996470in}{0.744266in}}%
\pgfpathlineto{\pgfqpoint{1.997202in}{0.739107in}}%
\pgfpathlineto{\pgfqpoint{1.997446in}{0.739107in}}%
\pgfpathlineto{\pgfqpoint{1.997446in}{0.728788in}}%
\pgfpathlineto{\pgfqpoint{1.997690in}{0.744266in}}%
\pgfpathlineto{\pgfqpoint{1.998421in}{0.728788in}}%
\pgfpathlineto{\pgfqpoint{1.999397in}{0.728788in}}%
\pgfpathlineto{\pgfqpoint{2.000373in}{0.744266in}}%
\pgfpathlineto{\pgfqpoint{2.000861in}{0.744266in}}%
\pgfpathlineto{\pgfqpoint{2.001105in}{0.728788in}}%
\pgfpathlineto{\pgfqpoint{2.001837in}{0.733947in}}%
\pgfpathlineto{\pgfqpoint{2.002081in}{0.733947in}}%
\pgfpathlineto{\pgfqpoint{2.002081in}{0.749425in}}%
\pgfpathlineto{\pgfqpoint{2.002813in}{0.728788in}}%
\pgfpathlineto{\pgfqpoint{2.003057in}{0.744266in}}%
\pgfpathlineto{\pgfqpoint{2.003301in}{0.744266in}}%
\pgfpathlineto{\pgfqpoint{2.003301in}{0.728788in}}%
\pgfpathlineto{\pgfqpoint{2.004277in}{0.733947in}}%
\pgfpathlineto{\pgfqpoint{2.004521in}{0.733947in}}%
\pgfpathlineto{\pgfqpoint{2.004521in}{0.728788in}}%
\pgfpathlineto{\pgfqpoint{2.004764in}{0.739107in}}%
\pgfpathlineto{\pgfqpoint{2.005496in}{0.739107in}}%
\pgfpathlineto{\pgfqpoint{2.005740in}{0.739107in}}%
\pgfpathlineto{\pgfqpoint{2.005984in}{0.728788in}}%
\pgfpathlineto{\pgfqpoint{2.006716in}{0.733947in}}%
\pgfpathlineto{\pgfqpoint{2.007204in}{0.733947in}}%
\pgfpathlineto{\pgfqpoint{2.007448in}{0.744266in}}%
\pgfpathlineto{\pgfqpoint{2.007936in}{0.728788in}}%
\pgfpathlineto{\pgfqpoint{2.008180in}{0.728788in}}%
\pgfpathlineto{\pgfqpoint{2.008424in}{0.728788in}}%
\pgfpathlineto{\pgfqpoint{2.008424in}{0.739107in}}%
\pgfpathlineto{\pgfqpoint{2.009400in}{0.739107in}}%
\pgfpathlineto{\pgfqpoint{2.009644in}{0.739107in}}%
\pgfpathlineto{\pgfqpoint{2.009888in}{0.728788in}}%
\pgfpathlineto{\pgfqpoint{2.010620in}{0.733947in}}%
\pgfpathlineto{\pgfqpoint{2.010864in}{0.733947in}}%
\pgfpathlineto{\pgfqpoint{2.010864in}{0.739107in}}%
\pgfpathlineto{\pgfqpoint{2.011839in}{0.728788in}}%
\pgfpathlineto{\pgfqpoint{2.012083in}{0.728788in}}%
\pgfpathlineto{\pgfqpoint{2.012815in}{0.749425in}}%
\pgfpathlineto{\pgfqpoint{2.013059in}{0.728788in}}%
\pgfpathlineto{\pgfqpoint{2.013303in}{0.728788in}}%
\pgfpathlineto{\pgfqpoint{2.014035in}{0.754585in}}%
\pgfpathlineto{\pgfqpoint{2.014279in}{0.733947in}}%
\pgfpathlineto{\pgfqpoint{2.014523in}{0.733947in}}%
\pgfpathlineto{\pgfqpoint{2.014523in}{0.739107in}}%
\pgfpathlineto{\pgfqpoint{2.015499in}{0.739107in}}%
\pgfpathlineto{\pgfqpoint{2.015743in}{0.739107in}}%
\pgfpathlineto{\pgfqpoint{2.016475in}{0.728788in}}%
\pgfpathlineto{\pgfqpoint{2.015987in}{0.744266in}}%
\pgfpathlineto{\pgfqpoint{2.016719in}{0.728788in}}%
\pgfpathlineto{\pgfqpoint{2.016963in}{0.728788in}}%
\pgfpathlineto{\pgfqpoint{2.017207in}{0.744266in}}%
\pgfpathlineto{\pgfqpoint{2.017938in}{0.739107in}}%
\pgfpathlineto{\pgfqpoint{2.018182in}{0.739107in}}%
\pgfpathlineto{\pgfqpoint{2.018182in}{0.728788in}}%
\pgfpathlineto{\pgfqpoint{2.018426in}{0.744266in}}%
\pgfpathlineto{\pgfqpoint{2.019158in}{0.733947in}}%
\pgfpathlineto{\pgfqpoint{2.019646in}{0.733947in}}%
\pgfpathlineto{\pgfqpoint{2.019646in}{0.728788in}}%
\pgfpathlineto{\pgfqpoint{2.020134in}{0.739107in}}%
\pgfpathlineto{\pgfqpoint{2.020622in}{0.728788in}}%
\pgfpathlineto{\pgfqpoint{2.020866in}{0.728788in}}%
\pgfpathlineto{\pgfqpoint{2.021598in}{0.739107in}}%
\pgfpathlineto{\pgfqpoint{2.021842in}{0.733947in}}%
\pgfpathlineto{\pgfqpoint{2.022086in}{0.733947in}}%
\pgfpathlineto{\pgfqpoint{2.022086in}{0.749425in}}%
\pgfpathlineto{\pgfqpoint{2.023062in}{0.739107in}}%
\pgfpathlineto{\pgfqpoint{2.023306in}{0.739107in}}%
\pgfpathlineto{\pgfqpoint{2.023794in}{0.728788in}}%
\pgfpathlineto{\pgfqpoint{2.023550in}{0.764903in}}%
\pgfpathlineto{\pgfqpoint{2.024282in}{0.728788in}}%
\pgfpathlineto{\pgfqpoint{2.024525in}{0.728788in}}%
\pgfpathlineto{\pgfqpoint{2.024525in}{0.749425in}}%
\pgfpathlineto{\pgfqpoint{2.025501in}{0.739107in}}%
\pgfpathlineto{\pgfqpoint{2.025745in}{0.739107in}}%
\pgfpathlineto{\pgfqpoint{2.025745in}{0.728788in}}%
\pgfpathlineto{\pgfqpoint{2.026721in}{0.728788in}}%
\pgfpathlineto{\pgfqpoint{2.027453in}{0.728788in}}%
\pgfpathlineto{\pgfqpoint{2.027453in}{0.739107in}}%
\pgfpathlineto{\pgfqpoint{2.028429in}{0.739107in}}%
\pgfpathlineto{\pgfqpoint{2.028917in}{0.739107in}}%
\pgfpathlineto{\pgfqpoint{2.028917in}{0.728788in}}%
\pgfpathlineto{\pgfqpoint{2.029893in}{0.733947in}}%
\pgfpathlineto{\pgfqpoint{2.030137in}{0.733947in}}%
\pgfpathlineto{\pgfqpoint{2.030137in}{0.728788in}}%
\pgfpathlineto{\pgfqpoint{2.030625in}{0.744266in}}%
\pgfpathlineto{\pgfqpoint{2.031112in}{0.739107in}}%
\pgfpathlineto{\pgfqpoint{2.031356in}{0.739107in}}%
\pgfpathlineto{\pgfqpoint{2.031844in}{0.728788in}}%
\pgfpathlineto{\pgfqpoint{2.031600in}{0.744266in}}%
\pgfpathlineto{\pgfqpoint{2.032332in}{0.744266in}}%
\pgfpathlineto{\pgfqpoint{2.032576in}{0.744266in}}%
\pgfpathlineto{\pgfqpoint{2.032576in}{0.728788in}}%
\pgfpathlineto{\pgfqpoint{2.032820in}{0.749425in}}%
\pgfpathlineto{\pgfqpoint{2.033552in}{0.749425in}}%
\pgfpathlineto{\pgfqpoint{2.033796in}{0.749425in}}%
\pgfpathlineto{\pgfqpoint{2.033796in}{0.754585in}}%
\pgfpathlineto{\pgfqpoint{2.034284in}{0.728788in}}%
\pgfpathlineto{\pgfqpoint{2.034772in}{0.728788in}}%
\pgfpathlineto{\pgfqpoint{2.035016in}{0.728788in}}%
\pgfpathlineto{\pgfqpoint{2.035992in}{0.759744in}}%
\pgfpathlineto{\pgfqpoint{2.036236in}{0.759744in}}%
\pgfpathlineto{\pgfqpoint{2.036236in}{0.728788in}}%
\pgfpathlineto{\pgfqpoint{2.037212in}{0.733947in}}%
\pgfpathlineto{\pgfqpoint{2.037699in}{0.733947in}}%
\pgfpathlineto{\pgfqpoint{2.037943in}{0.744266in}}%
\pgfpathlineto{\pgfqpoint{2.038187in}{0.728788in}}%
\pgfpathlineto{\pgfqpoint{2.038675in}{0.739107in}}%
\pgfpathlineto{\pgfqpoint{2.038919in}{0.739107in}}%
\pgfpathlineto{\pgfqpoint{2.039407in}{0.728788in}}%
\pgfpathlineto{\pgfqpoint{2.039651in}{0.744266in}}%
\pgfpathlineto{\pgfqpoint{2.039895in}{0.728788in}}%
\pgfpathlineto{\pgfqpoint{2.040383in}{0.728788in}}%
\pgfpathlineto{\pgfqpoint{2.040383in}{0.744266in}}%
\pgfpathlineto{\pgfqpoint{2.041359in}{0.728788in}}%
\pgfpathlineto{\pgfqpoint{2.041847in}{0.728788in}}%
\pgfpathlineto{\pgfqpoint{2.042091in}{0.744266in}}%
\pgfpathlineto{\pgfqpoint{2.042823in}{0.733947in}}%
\pgfpathlineto{\pgfqpoint{2.043555in}{0.733947in}}%
\pgfpathlineto{\pgfqpoint{2.043555in}{0.728788in}}%
\pgfpathlineto{\pgfqpoint{2.044530in}{0.749425in}}%
\pgfpathlineto{\pgfqpoint{2.044774in}{0.749425in}}%
\pgfpathlineto{\pgfqpoint{2.044774in}{0.733947in}}%
\pgfpathlineto{\pgfqpoint{2.045750in}{0.733947in}}%
\pgfpathlineto{\pgfqpoint{2.045994in}{0.733947in}}%
\pgfpathlineto{\pgfqpoint{2.046726in}{0.744266in}}%
\pgfpathlineto{\pgfqpoint{2.046970in}{0.739107in}}%
\pgfpathlineto{\pgfqpoint{2.047702in}{0.739107in}}%
\pgfpathlineto{\pgfqpoint{2.047946in}{0.728788in}}%
\pgfpathlineto{\pgfqpoint{2.048434in}{0.744266in}}%
\pgfpathlineto{\pgfqpoint{2.048678in}{0.733947in}}%
\pgfpathlineto{\pgfqpoint{2.048922in}{0.733947in}}%
\pgfpathlineto{\pgfqpoint{2.048922in}{0.739107in}}%
\pgfpathlineto{\pgfqpoint{2.049898in}{0.733947in}}%
\pgfpathlineto{\pgfqpoint{2.050142in}{0.733947in}}%
\pgfpathlineto{\pgfqpoint{2.050142in}{0.728788in}}%
\pgfpathlineto{\pgfqpoint{2.050629in}{0.744266in}}%
\pgfpathlineto{\pgfqpoint{2.051117in}{0.739107in}}%
\pgfpathlineto{\pgfqpoint{2.051361in}{0.739107in}}%
\pgfpathlineto{\pgfqpoint{2.051361in}{0.728788in}}%
\pgfpathlineto{\pgfqpoint{2.052093in}{0.744266in}}%
\pgfpathlineto{\pgfqpoint{2.052337in}{0.728788in}}%
\pgfpathlineto{\pgfqpoint{2.052581in}{0.728788in}}%
\pgfpathlineto{\pgfqpoint{2.053557in}{0.749425in}}%
\pgfpathlineto{\pgfqpoint{2.053801in}{0.749425in}}%
\pgfpathlineto{\pgfqpoint{2.054045in}{0.733947in}}%
\pgfpathlineto{\pgfqpoint{2.054289in}{0.759744in}}%
\pgfpathlineto{\pgfqpoint{2.054777in}{0.739107in}}%
\pgfpathlineto{\pgfqpoint{2.055021in}{0.739107in}}%
\pgfpathlineto{\pgfqpoint{2.055509in}{0.744266in}}%
\pgfpathlineto{\pgfqpoint{2.055997in}{0.728788in}}%
\pgfpathlineto{\pgfqpoint{2.056485in}{0.728788in}}%
\pgfpathlineto{\pgfqpoint{2.056973in}{0.744266in}}%
\pgfpathlineto{\pgfqpoint{2.057460in}{0.733947in}}%
\pgfpathlineto{\pgfqpoint{2.057948in}{0.733947in}}%
\pgfpathlineto{\pgfqpoint{2.057948in}{0.728788in}}%
\pgfpathlineto{\pgfqpoint{2.058436in}{0.739107in}}%
\pgfpathlineto{\pgfqpoint{2.058924in}{0.733947in}}%
\pgfpathlineto{\pgfqpoint{2.059412in}{0.733947in}}%
\pgfpathlineto{\pgfqpoint{2.059412in}{0.754585in}}%
\pgfpathlineto{\pgfqpoint{2.060388in}{0.749425in}}%
\pgfpathlineto{\pgfqpoint{2.060876in}{0.749425in}}%
\pgfpathlineto{\pgfqpoint{2.060876in}{0.733947in}}%
\pgfpathlineto{\pgfqpoint{2.061852in}{0.744266in}}%
\pgfpathlineto{\pgfqpoint{2.062096in}{0.744266in}}%
\pgfpathlineto{\pgfqpoint{2.062096in}{0.728788in}}%
\pgfpathlineto{\pgfqpoint{2.063072in}{0.733947in}}%
\pgfpathlineto{\pgfqpoint{2.063316in}{0.733947in}}%
\pgfpathlineto{\pgfqpoint{2.064047in}{0.749425in}}%
\pgfpathlineto{\pgfqpoint{2.064291in}{0.733947in}}%
\pgfpathlineto{\pgfqpoint{2.064535in}{0.733947in}}%
\pgfpathlineto{\pgfqpoint{2.064779in}{0.728788in}}%
\pgfpathlineto{\pgfqpoint{2.065511in}{0.744266in}}%
\pgfpathlineto{\pgfqpoint{2.065755in}{0.744266in}}%
\pgfpathlineto{\pgfqpoint{2.066487in}{0.733947in}}%
\pgfpathlineto{\pgfqpoint{2.066731in}{0.739107in}}%
\pgfpathlineto{\pgfqpoint{2.066975in}{0.739107in}}%
\pgfpathlineto{\pgfqpoint{2.067219in}{0.728788in}}%
\pgfpathlineto{\pgfqpoint{2.067463in}{0.744266in}}%
\pgfpathlineto{\pgfqpoint{2.067951in}{0.728788in}}%
\pgfpathlineto{\pgfqpoint{2.068195in}{0.728788in}}%
\pgfpathlineto{\pgfqpoint{2.068439in}{0.739107in}}%
\pgfpathlineto{\pgfqpoint{2.069171in}{0.728788in}}%
\pgfpathlineto{\pgfqpoint{2.069659in}{0.728788in}}%
\pgfpathlineto{\pgfqpoint{2.069659in}{0.739107in}}%
\pgfpathlineto{\pgfqpoint{2.070634in}{0.728788in}}%
\pgfpathlineto{\pgfqpoint{2.071122in}{0.728788in}}%
\pgfpathlineto{\pgfqpoint{2.072098in}{0.744266in}}%
\pgfpathlineto{\pgfqpoint{2.072342in}{0.744266in}}%
\pgfpathlineto{\pgfqpoint{2.072342in}{0.728788in}}%
\pgfpathlineto{\pgfqpoint{2.073318in}{0.744266in}}%
\pgfpathlineto{\pgfqpoint{2.073562in}{0.744266in}}%
\pgfpathlineto{\pgfqpoint{2.074050in}{0.733947in}}%
\pgfpathlineto{\pgfqpoint{2.074538in}{0.739107in}}%
\pgfpathlineto{\pgfqpoint{2.074782in}{0.739107in}}%
\pgfpathlineto{\pgfqpoint{2.074782in}{0.728788in}}%
\pgfpathlineto{\pgfqpoint{2.075758in}{0.733947in}}%
\pgfpathlineto{\pgfqpoint{2.076002in}{0.733947in}}%
\pgfpathlineto{\pgfqpoint{2.076002in}{0.728788in}}%
\pgfpathlineto{\pgfqpoint{2.076246in}{0.744266in}}%
\pgfpathlineto{\pgfqpoint{2.076977in}{0.739107in}}%
\pgfpathlineto{\pgfqpoint{2.077221in}{0.739107in}}%
\pgfpathlineto{\pgfqpoint{2.077221in}{0.744266in}}%
\pgfpathlineto{\pgfqpoint{2.077709in}{0.733947in}}%
\pgfpathlineto{\pgfqpoint{2.078197in}{0.744266in}}%
\pgfpathlineto{\pgfqpoint{2.078441in}{0.744266in}}%
\pgfpathlineto{\pgfqpoint{2.078929in}{0.728788in}}%
\pgfpathlineto{\pgfqpoint{2.079417in}{0.728788in}}%
\pgfpathlineto{\pgfqpoint{2.079661in}{0.728788in}}%
\pgfpathlineto{\pgfqpoint{2.079661in}{0.744266in}}%
\pgfpathlineto{\pgfqpoint{2.080637in}{0.728788in}}%
\pgfpathlineto{\pgfqpoint{2.080881in}{0.728788in}}%
\pgfpathlineto{\pgfqpoint{2.081857in}{0.744266in}}%
\pgfpathlineto{\pgfqpoint{2.082345in}{0.744266in}}%
\pgfpathlineto{\pgfqpoint{2.082345in}{0.728788in}}%
\pgfpathlineto{\pgfqpoint{2.083320in}{0.733947in}}%
\pgfpathlineto{\pgfqpoint{2.083808in}{0.733947in}}%
\pgfpathlineto{\pgfqpoint{2.083808in}{0.739107in}}%
\pgfpathlineto{\pgfqpoint{2.084784in}{0.739107in}}%
\pgfpathlineto{\pgfqpoint{2.085028in}{0.739107in}}%
\pgfpathlineto{\pgfqpoint{2.085028in}{0.728788in}}%
\pgfpathlineto{\pgfqpoint{2.085272in}{0.749425in}}%
\pgfpathlineto{\pgfqpoint{2.086004in}{0.728788in}}%
\pgfpathlineto{\pgfqpoint{2.086248in}{0.728788in}}%
\pgfpathlineto{\pgfqpoint{2.086736in}{0.739107in}}%
\pgfpathlineto{\pgfqpoint{2.087224in}{0.733947in}}%
\pgfpathlineto{\pgfqpoint{2.087712in}{0.733947in}}%
\pgfpathlineto{\pgfqpoint{2.088688in}{0.759744in}}%
\pgfpathlineto{\pgfqpoint{2.088932in}{0.759744in}}%
\pgfpathlineto{\pgfqpoint{2.089420in}{0.728788in}}%
\pgfpathlineto{\pgfqpoint{2.089907in}{0.728788in}}%
\pgfpathlineto{\pgfqpoint{2.090151in}{0.728788in}}%
\pgfpathlineto{\pgfqpoint{2.091127in}{0.749425in}}%
\pgfpathlineto{\pgfqpoint{2.091371in}{0.749425in}}%
\pgfpathlineto{\pgfqpoint{2.091615in}{0.728788in}}%
\pgfpathlineto{\pgfqpoint{2.092103in}{0.754585in}}%
\pgfpathlineto{\pgfqpoint{2.092347in}{0.739107in}}%
\pgfpathlineto{\pgfqpoint{2.092591in}{0.739107in}}%
\pgfpathlineto{\pgfqpoint{2.092591in}{0.733947in}}%
\pgfpathlineto{\pgfqpoint{2.093323in}{0.759744in}}%
\pgfpathlineto{\pgfqpoint{2.093567in}{0.733947in}}%
\pgfpathlineto{\pgfqpoint{2.093811in}{0.733947in}}%
\pgfpathlineto{\pgfqpoint{2.093811in}{0.728788in}}%
\pgfpathlineto{\pgfqpoint{2.094787in}{0.733947in}}%
\pgfpathlineto{\pgfqpoint{2.095031in}{0.733947in}}%
\pgfpathlineto{\pgfqpoint{2.095031in}{0.728788in}}%
\pgfpathlineto{\pgfqpoint{2.096007in}{0.744266in}}%
\pgfpathlineto{\pgfqpoint{2.096251in}{0.744266in}}%
\pgfpathlineto{\pgfqpoint{2.096738in}{0.728788in}}%
\pgfpathlineto{\pgfqpoint{2.097226in}{0.733947in}}%
\pgfpathlineto{\pgfqpoint{2.097714in}{0.733947in}}%
\pgfpathlineto{\pgfqpoint{2.097714in}{0.728788in}}%
\pgfpathlineto{\pgfqpoint{2.098446in}{0.744266in}}%
\pgfpathlineto{\pgfqpoint{2.098690in}{0.733947in}}%
\pgfpathlineto{\pgfqpoint{2.098934in}{0.733947in}}%
\pgfpathlineto{\pgfqpoint{2.098934in}{0.739107in}}%
\pgfpathlineto{\pgfqpoint{2.099178in}{0.728788in}}%
\pgfpathlineto{\pgfqpoint{2.099910in}{0.728788in}}%
\pgfpathlineto{\pgfqpoint{2.100398in}{0.728788in}}%
\pgfpathlineto{\pgfqpoint{2.100886in}{0.754585in}}%
\pgfpathlineto{\pgfqpoint{2.101374in}{0.739107in}}%
\pgfpathlineto{\pgfqpoint{2.101618in}{0.739107in}}%
\pgfpathlineto{\pgfqpoint{2.101618in}{0.728788in}}%
\pgfpathlineto{\pgfqpoint{2.102350in}{0.749425in}}%
\pgfpathlineto{\pgfqpoint{2.102594in}{0.733947in}}%
\pgfpathlineto{\pgfqpoint{2.102838in}{0.733947in}}%
\pgfpathlineto{\pgfqpoint{2.103569in}{0.749425in}}%
\pgfpathlineto{\pgfqpoint{2.103813in}{0.733947in}}%
\pgfpathlineto{\pgfqpoint{2.104057in}{0.733947in}}%
\pgfpathlineto{\pgfqpoint{2.104057in}{0.739107in}}%
\pgfpathlineto{\pgfqpoint{2.105033in}{0.739107in}}%
\pgfpathlineto{\pgfqpoint{2.105277in}{0.739107in}}%
\pgfpathlineto{\pgfqpoint{2.105765in}{0.728788in}}%
\pgfpathlineto{\pgfqpoint{2.106009in}{0.744266in}}%
\pgfpathlineto{\pgfqpoint{2.106253in}{0.728788in}}%
\pgfpathlineto{\pgfqpoint{2.106497in}{0.728788in}}%
\pgfpathlineto{\pgfqpoint{2.106741in}{0.739107in}}%
\pgfpathlineto{\pgfqpoint{2.107473in}{0.733947in}}%
\pgfpathlineto{\pgfqpoint{2.107717in}{0.733947in}}%
\pgfpathlineto{\pgfqpoint{2.107961in}{0.749425in}}%
\pgfpathlineto{\pgfqpoint{2.108693in}{0.733947in}}%
\pgfpathlineto{\pgfqpoint{2.110156in}{0.733947in}}%
\pgfpathlineto{\pgfqpoint{2.110156in}{0.739107in}}%
\pgfpathlineto{\pgfqpoint{2.110400in}{0.728788in}}%
\pgfpathlineto{\pgfqpoint{2.111132in}{0.733947in}}%
\pgfpathlineto{\pgfqpoint{2.111620in}{0.733947in}}%
\pgfpathlineto{\pgfqpoint{2.111620in}{0.728788in}}%
\pgfpathlineto{\pgfqpoint{2.111864in}{0.744266in}}%
\pgfpathlineto{\pgfqpoint{2.112596in}{0.733947in}}%
\pgfpathlineto{\pgfqpoint{2.112840in}{0.733947in}}%
\pgfpathlineto{\pgfqpoint{2.112840in}{0.728788in}}%
\pgfpathlineto{\pgfqpoint{2.113328in}{0.739107in}}%
\pgfpathlineto{\pgfqpoint{2.113816in}{0.728788in}}%
\pgfpathlineto{\pgfqpoint{2.114060in}{0.728788in}}%
\pgfpathlineto{\pgfqpoint{2.114060in}{0.744266in}}%
\pgfpathlineto{\pgfqpoint{2.115036in}{0.728788in}}%
\pgfpathlineto{\pgfqpoint{2.115280in}{0.728788in}}%
\pgfpathlineto{\pgfqpoint{2.115280in}{0.749425in}}%
\pgfpathlineto{\pgfqpoint{2.116255in}{0.739107in}}%
\pgfpathlineto{\pgfqpoint{2.116743in}{0.739107in}}%
\pgfpathlineto{\pgfqpoint{2.116743in}{0.744266in}}%
\pgfpathlineto{\pgfqpoint{2.116987in}{0.733947in}}%
\pgfpathlineto{\pgfqpoint{2.117719in}{0.744266in}}%
\pgfpathlineto{\pgfqpoint{2.117963in}{0.744266in}}%
\pgfpathlineto{\pgfqpoint{2.118451in}{0.728788in}}%
\pgfpathlineto{\pgfqpoint{2.118939in}{0.728788in}}%
\pgfpathlineto{\pgfqpoint{2.119183in}{0.728788in}}%
\pgfpathlineto{\pgfqpoint{2.119915in}{0.744266in}}%
\pgfpathlineto{\pgfqpoint{2.120159in}{0.733947in}}%
\pgfpathlineto{\pgfqpoint{2.120403in}{0.733947in}}%
\pgfpathlineto{\pgfqpoint{2.121135in}{0.749425in}}%
\pgfpathlineto{\pgfqpoint{2.120891in}{0.728788in}}%
\pgfpathlineto{\pgfqpoint{2.121379in}{0.739107in}}%
\pgfpathlineto{\pgfqpoint{2.121867in}{0.739107in}}%
\pgfpathlineto{\pgfqpoint{2.121867in}{0.728788in}}%
\pgfpathlineto{\pgfqpoint{2.122842in}{0.733947in}}%
\pgfpathlineto{\pgfqpoint{2.123086in}{0.733947in}}%
\pgfpathlineto{\pgfqpoint{2.123086in}{0.728788in}}%
\pgfpathlineto{\pgfqpoint{2.123574in}{0.749425in}}%
\pgfpathlineto{\pgfqpoint{2.124062in}{0.728788in}}%
\pgfpathlineto{\pgfqpoint{2.125038in}{0.728788in}}%
\pgfpathlineto{\pgfqpoint{2.125526in}{0.739107in}}%
\pgfpathlineto{\pgfqpoint{2.126014in}{0.733947in}}%
\pgfpathlineto{\pgfqpoint{2.126502in}{0.733947in}}%
\pgfpathlineto{\pgfqpoint{2.126502in}{0.744266in}}%
\pgfpathlineto{\pgfqpoint{2.127234in}{0.728788in}}%
\pgfpathlineto{\pgfqpoint{2.127478in}{0.739107in}}%
\pgfpathlineto{\pgfqpoint{2.127722in}{0.739107in}}%
\pgfpathlineto{\pgfqpoint{2.127722in}{0.728788in}}%
\pgfpathlineto{\pgfqpoint{2.127966in}{0.744266in}}%
\pgfpathlineto{\pgfqpoint{2.128698in}{0.739107in}}%
\pgfpathlineto{\pgfqpoint{2.129185in}{0.739107in}}%
\pgfpathlineto{\pgfqpoint{2.129673in}{0.728788in}}%
\pgfpathlineto{\pgfqpoint{2.130161in}{0.728788in}}%
\pgfpathlineto{\pgfqpoint{2.130405in}{0.728788in}}%
\pgfpathlineto{\pgfqpoint{2.130405in}{0.739107in}}%
\pgfpathlineto{\pgfqpoint{2.131381in}{0.733947in}}%
\pgfpathlineto{\pgfqpoint{2.131625in}{0.733947in}}%
\pgfpathlineto{\pgfqpoint{2.131625in}{0.728788in}}%
\pgfpathlineto{\pgfqpoint{2.132601in}{0.739107in}}%
\pgfpathlineto{\pgfqpoint{2.132845in}{0.739107in}}%
\pgfpathlineto{\pgfqpoint{2.132845in}{0.728788in}}%
\pgfpathlineto{\pgfqpoint{2.133089in}{0.744266in}}%
\pgfpathlineto{\pgfqpoint{2.133821in}{0.733947in}}%
\pgfpathlineto{\pgfqpoint{2.134309in}{0.733947in}}%
\pgfpathlineto{\pgfqpoint{2.134309in}{0.728788in}}%
\pgfpathlineto{\pgfqpoint{2.135285in}{0.744266in}}%
\pgfpathlineto{\pgfqpoint{2.135529in}{0.744266in}}%
\pgfpathlineto{\pgfqpoint{2.135529in}{0.733947in}}%
\pgfpathlineto{\pgfqpoint{2.136504in}{0.733947in}}%
\pgfpathlineto{\pgfqpoint{2.136748in}{0.733947in}}%
\pgfpathlineto{\pgfqpoint{2.136992in}{0.744266in}}%
\pgfpathlineto{\pgfqpoint{2.137724in}{0.728788in}}%
\pgfpathlineto{\pgfqpoint{2.137968in}{0.728788in}}%
\pgfpathlineto{\pgfqpoint{2.137968in}{0.744266in}}%
\pgfpathlineto{\pgfqpoint{2.138944in}{0.739107in}}%
\pgfpathlineto{\pgfqpoint{2.139188in}{0.739107in}}%
\pgfpathlineto{\pgfqpoint{2.139920in}{0.728788in}}%
\pgfpathlineto{\pgfqpoint{2.140164in}{0.744266in}}%
\pgfpathlineto{\pgfqpoint{2.140408in}{0.744266in}}%
\pgfpathlineto{\pgfqpoint{2.140408in}{0.728788in}}%
\pgfpathlineto{\pgfqpoint{2.141384in}{0.733947in}}%
\pgfpathlineto{\pgfqpoint{2.141628in}{0.733947in}}%
\pgfpathlineto{\pgfqpoint{2.141628in}{0.744266in}}%
\pgfpathlineto{\pgfqpoint{2.142603in}{0.733947in}}%
\pgfpathlineto{\pgfqpoint{2.142847in}{0.733947in}}%
\pgfpathlineto{\pgfqpoint{2.142847in}{0.744266in}}%
\pgfpathlineto{\pgfqpoint{2.143579in}{0.728788in}}%
\pgfpathlineto{\pgfqpoint{2.143823in}{0.728788in}}%
\pgfpathlineto{\pgfqpoint{2.144311in}{0.728788in}}%
\pgfpathlineto{\pgfqpoint{2.144311in}{0.739107in}}%
\pgfpathlineto{\pgfqpoint{2.145287in}{0.728788in}}%
\pgfpathlineto{\pgfqpoint{2.146263in}{0.728788in}}%
\pgfpathlineto{\pgfqpoint{2.146263in}{0.739107in}}%
\pgfpathlineto{\pgfqpoint{2.147239in}{0.739107in}}%
\pgfpathlineto{\pgfqpoint{2.147483in}{0.739107in}}%
\pgfpathlineto{\pgfqpoint{2.147483in}{0.733947in}}%
\pgfpathlineto{\pgfqpoint{2.148459in}{0.739107in}}%
\pgfpathlineto{\pgfqpoint{2.148702in}{0.739107in}}%
\pgfpathlineto{\pgfqpoint{2.149434in}{0.728788in}}%
\pgfpathlineto{\pgfqpoint{2.149678in}{0.733947in}}%
\pgfpathlineto{\pgfqpoint{2.150410in}{0.733947in}}%
\pgfpathlineto{\pgfqpoint{2.150410in}{0.739107in}}%
\pgfpathlineto{\pgfqpoint{2.150898in}{0.728788in}}%
\pgfpathlineto{\pgfqpoint{2.151386in}{0.733947in}}%
\pgfpathlineto{\pgfqpoint{2.151630in}{0.733947in}}%
\pgfpathlineto{\pgfqpoint{2.151630in}{0.739107in}}%
\pgfpathlineto{\pgfqpoint{2.152606in}{0.733947in}}%
\pgfpathlineto{\pgfqpoint{2.152850in}{0.733947in}}%
\pgfpathlineto{\pgfqpoint{2.152850in}{0.728788in}}%
\pgfpathlineto{\pgfqpoint{2.153094in}{0.749425in}}%
\pgfpathlineto{\pgfqpoint{2.153826in}{0.728788in}}%
\pgfpathlineto{\pgfqpoint{2.154314in}{0.728788in}}%
\pgfpathlineto{\pgfqpoint{2.155046in}{0.749425in}}%
\pgfpathlineto{\pgfqpoint{2.155289in}{0.733947in}}%
\pgfpathlineto{\pgfqpoint{2.155533in}{0.733947in}}%
\pgfpathlineto{\pgfqpoint{2.155533in}{0.728788in}}%
\pgfpathlineto{\pgfqpoint{2.156509in}{0.728788in}}%
\pgfpathlineto{\pgfqpoint{2.156753in}{0.728788in}}%
\pgfpathlineto{\pgfqpoint{2.156753in}{0.739107in}}%
\pgfpathlineto{\pgfqpoint{2.157729in}{0.739107in}}%
\pgfpathlineto{\pgfqpoint{2.158461in}{0.739107in}}%
\pgfpathlineto{\pgfqpoint{2.158461in}{0.744266in}}%
\pgfpathlineto{\pgfqpoint{2.158705in}{0.728788in}}%
\pgfpathlineto{\pgfqpoint{2.159437in}{0.733947in}}%
\pgfpathlineto{\pgfqpoint{2.160169in}{0.733947in}}%
\pgfpathlineto{\pgfqpoint{2.160169in}{0.739107in}}%
\pgfpathlineto{\pgfqpoint{2.160657in}{0.728788in}}%
\pgfpathlineto{\pgfqpoint{2.161145in}{0.733947in}}%
\pgfpathlineto{\pgfqpoint{2.161389in}{0.733947in}}%
\pgfpathlineto{\pgfqpoint{2.161389in}{0.728788in}}%
\pgfpathlineto{\pgfqpoint{2.162120in}{0.744266in}}%
\pgfpathlineto{\pgfqpoint{2.162364in}{0.739107in}}%
\pgfpathlineto{\pgfqpoint{2.162608in}{0.739107in}}%
\pgfpathlineto{\pgfqpoint{2.162608in}{0.728788in}}%
\pgfpathlineto{\pgfqpoint{2.163584in}{0.744266in}}%
\pgfpathlineto{\pgfqpoint{2.163828in}{0.744266in}}%
\pgfpathlineto{\pgfqpoint{2.163828in}{0.733947in}}%
\pgfpathlineto{\pgfqpoint{2.164804in}{0.733947in}}%
\pgfpathlineto{\pgfqpoint{2.165780in}{0.733947in}}%
\pgfpathlineto{\pgfqpoint{2.166024in}{0.749425in}}%
\pgfpathlineto{\pgfqpoint{2.166756in}{0.733947in}}%
\pgfpathlineto{\pgfqpoint{2.167244in}{0.733947in}}%
\pgfpathlineto{\pgfqpoint{2.167244in}{0.728788in}}%
\pgfpathlineto{\pgfqpoint{2.168220in}{0.733947in}}%
\pgfpathlineto{\pgfqpoint{2.168463in}{0.733947in}}%
\pgfpathlineto{\pgfqpoint{2.168463in}{0.749425in}}%
\pgfpathlineto{\pgfqpoint{2.169439in}{0.733947in}}%
\pgfpathlineto{\pgfqpoint{2.169683in}{0.733947in}}%
\pgfpathlineto{\pgfqpoint{2.169683in}{0.739107in}}%
\pgfpathlineto{\pgfqpoint{2.170171in}{0.728788in}}%
\pgfpathlineto{\pgfqpoint{2.170659in}{0.733947in}}%
\pgfpathlineto{\pgfqpoint{2.171147in}{0.733947in}}%
\pgfpathlineto{\pgfqpoint{2.171147in}{0.728788in}}%
\pgfpathlineto{\pgfqpoint{2.171879in}{0.739107in}}%
\pgfpathlineto{\pgfqpoint{2.172123in}{0.739107in}}%
\pgfpathlineto{\pgfqpoint{2.172367in}{0.739107in}}%
\pgfpathlineto{\pgfqpoint{2.172367in}{0.728788in}}%
\pgfpathlineto{\pgfqpoint{2.173343in}{0.733947in}}%
\pgfpathlineto{\pgfqpoint{2.173831in}{0.733947in}}%
\pgfpathlineto{\pgfqpoint{2.173831in}{0.744266in}}%
\pgfpathlineto{\pgfqpoint{2.174807in}{0.733947in}}%
\pgfpathlineto{\pgfqpoint{2.175050in}{0.733947in}}%
\pgfpathlineto{\pgfqpoint{2.175294in}{0.744266in}}%
\pgfpathlineto{\pgfqpoint{2.176026in}{0.739107in}}%
\pgfpathlineto{\pgfqpoint{2.176270in}{0.739107in}}%
\pgfpathlineto{\pgfqpoint{2.176270in}{0.728788in}}%
\pgfpathlineto{\pgfqpoint{2.177246in}{0.739107in}}%
\pgfpathlineto{\pgfqpoint{2.177490in}{0.739107in}}%
\pgfpathlineto{\pgfqpoint{2.177490in}{0.733947in}}%
\pgfpathlineto{\pgfqpoint{2.178466in}{0.733947in}}%
\pgfpathlineto{\pgfqpoint{2.179198in}{0.733947in}}%
\pgfpathlineto{\pgfqpoint{2.179198in}{0.728788in}}%
\pgfpathlineto{\pgfqpoint{2.179442in}{0.739107in}}%
\pgfpathlineto{\pgfqpoint{2.180174in}{0.728788in}}%
\pgfpathlineto{\pgfqpoint{2.180418in}{0.728788in}}%
\pgfpathlineto{\pgfqpoint{2.180906in}{0.744266in}}%
\pgfpathlineto{\pgfqpoint{2.181393in}{0.733947in}}%
\pgfpathlineto{\pgfqpoint{2.181637in}{0.733947in}}%
\pgfpathlineto{\pgfqpoint{2.181637in}{0.728788in}}%
\pgfpathlineto{\pgfqpoint{2.182613in}{0.744266in}}%
\pgfpathlineto{\pgfqpoint{2.182857in}{0.744266in}}%
\pgfpathlineto{\pgfqpoint{2.183345in}{0.728788in}}%
\pgfpathlineto{\pgfqpoint{2.183833in}{0.733947in}}%
\pgfpathlineto{\pgfqpoint{2.184321in}{0.733947in}}%
\pgfpathlineto{\pgfqpoint{2.184321in}{0.728788in}}%
\pgfpathlineto{\pgfqpoint{2.184809in}{0.739107in}}%
\pgfpathlineto{\pgfqpoint{2.185297in}{0.733947in}}%
\pgfpathlineto{\pgfqpoint{2.185785in}{0.733947in}}%
\pgfpathlineto{\pgfqpoint{2.185785in}{0.744266in}}%
\pgfpathlineto{\pgfqpoint{2.186517in}{0.728788in}}%
\pgfpathlineto{\pgfqpoint{2.186761in}{0.733947in}}%
\pgfpathlineto{\pgfqpoint{2.187005in}{0.733947in}}%
\pgfpathlineto{\pgfqpoint{2.187005in}{0.728788in}}%
\pgfpathlineto{\pgfqpoint{2.187980in}{0.733947in}}%
\pgfpathlineto{\pgfqpoint{2.188224in}{0.733947in}}%
\pgfpathlineto{\pgfqpoint{2.188224in}{0.739107in}}%
\pgfpathlineto{\pgfqpoint{2.188712in}{0.728788in}}%
\pgfpathlineto{\pgfqpoint{2.189200in}{0.728788in}}%
\pgfpathlineto{\pgfqpoint{2.189444in}{0.728788in}}%
\pgfpathlineto{\pgfqpoint{2.190176in}{0.739107in}}%
\pgfpathlineto{\pgfqpoint{2.190420in}{0.728788in}}%
\pgfpathlineto{\pgfqpoint{2.190664in}{0.728788in}}%
\pgfpathlineto{\pgfqpoint{2.190664in}{0.733947in}}%
\pgfpathlineto{\pgfqpoint{2.191640in}{0.733947in}}%
\pgfpathlineto{\pgfqpoint{2.191884in}{0.733947in}}%
\pgfpathlineto{\pgfqpoint{2.191884in}{0.728788in}}%
\pgfpathlineto{\pgfqpoint{2.192860in}{0.728788in}}%
\pgfpathlineto{\pgfqpoint{2.193348in}{0.728788in}}%
\pgfpathlineto{\pgfqpoint{2.194324in}{0.744266in}}%
\pgfpathlineto{\pgfqpoint{2.194567in}{0.744266in}}%
\pgfpathlineto{\pgfqpoint{2.194567in}{0.728788in}}%
\pgfpathlineto{\pgfqpoint{2.195543in}{0.733947in}}%
\pgfpathlineto{\pgfqpoint{2.195787in}{0.733947in}}%
\pgfpathlineto{\pgfqpoint{2.195787in}{0.728788in}}%
\pgfpathlineto{\pgfqpoint{2.196031in}{0.749425in}}%
\pgfpathlineto{\pgfqpoint{2.196763in}{0.733947in}}%
\pgfpathlineto{\pgfqpoint{2.197007in}{0.733947in}}%
\pgfpathlineto{\pgfqpoint{2.197007in}{0.739107in}}%
\pgfpathlineto{\pgfqpoint{2.197739in}{0.728788in}}%
\pgfpathlineto{\pgfqpoint{2.197983in}{0.728788in}}%
\pgfpathlineto{\pgfqpoint{2.198471in}{0.728788in}}%
\pgfpathlineto{\pgfqpoint{2.198959in}{0.744266in}}%
\pgfpathlineto{\pgfqpoint{2.199447in}{0.728788in}}%
\pgfpathlineto{\pgfqpoint{2.200179in}{0.728788in}}%
\pgfpathlineto{\pgfqpoint{2.201154in}{0.739107in}}%
\pgfpathlineto{\pgfqpoint{2.201398in}{0.739107in}}%
\pgfpathlineto{\pgfqpoint{2.202130in}{0.728788in}}%
\pgfpathlineto{\pgfqpoint{2.202374in}{0.728788in}}%
\pgfpathlineto{\pgfqpoint{2.202618in}{0.728788in}}%
\pgfpathlineto{\pgfqpoint{2.202618in}{0.739107in}}%
\pgfpathlineto{\pgfqpoint{2.203594in}{0.733947in}}%
\pgfpathlineto{\pgfqpoint{2.204082in}{0.733947in}}%
\pgfpathlineto{\pgfqpoint{2.204082in}{0.739107in}}%
\pgfpathlineto{\pgfqpoint{2.204570in}{0.728788in}}%
\pgfpathlineto{\pgfqpoint{2.205058in}{0.739107in}}%
\pgfpathlineto{\pgfqpoint{2.205302in}{0.739107in}}%
\pgfpathlineto{\pgfqpoint{2.205302in}{0.728788in}}%
\pgfpathlineto{\pgfqpoint{2.205546in}{0.744266in}}%
\pgfpathlineto{\pgfqpoint{2.206278in}{0.728788in}}%
\pgfpathlineto{\pgfqpoint{2.206522in}{0.728788in}}%
\pgfpathlineto{\pgfqpoint{2.206522in}{0.739107in}}%
\pgfpathlineto{\pgfqpoint{2.207498in}{0.728788in}}%
\pgfpathlineto{\pgfqpoint{2.207741in}{0.728788in}}%
\pgfpathlineto{\pgfqpoint{2.208229in}{0.739107in}}%
\pgfpathlineto{\pgfqpoint{2.208717in}{0.733947in}}%
\pgfpathlineto{\pgfqpoint{2.208961in}{0.733947in}}%
\pgfpathlineto{\pgfqpoint{2.209205in}{0.744266in}}%
\pgfpathlineto{\pgfqpoint{2.209937in}{0.733947in}}%
\pgfpathlineto{\pgfqpoint{2.210181in}{0.733947in}}%
\pgfpathlineto{\pgfqpoint{2.210425in}{0.744266in}}%
\pgfpathlineto{\pgfqpoint{2.210669in}{0.728788in}}%
\pgfpathlineto{\pgfqpoint{2.211157in}{0.728788in}}%
\pgfpathlineto{\pgfqpoint{2.211401in}{0.728788in}}%
\pgfpathlineto{\pgfqpoint{2.211401in}{0.739107in}}%
\pgfpathlineto{\pgfqpoint{2.212377in}{0.728788in}}%
\pgfpathlineto{\pgfqpoint{2.212621in}{0.728788in}}%
\pgfpathlineto{\pgfqpoint{2.212865in}{0.739107in}}%
\pgfpathlineto{\pgfqpoint{2.213597in}{0.728788in}}%
\pgfpathlineto{\pgfqpoint{2.213841in}{0.728788in}}%
\pgfpathlineto{\pgfqpoint{2.214816in}{0.739107in}}%
\pgfpathlineto{\pgfqpoint{2.215060in}{0.739107in}}%
\pgfpathlineto{\pgfqpoint{2.215060in}{0.728788in}}%
\pgfpathlineto{\pgfqpoint{2.216036in}{0.733947in}}%
\pgfpathlineto{\pgfqpoint{2.216280in}{0.733947in}}%
\pgfpathlineto{\pgfqpoint{2.216280in}{0.739107in}}%
\pgfpathlineto{\pgfqpoint{2.217012in}{0.728788in}}%
\pgfpathlineto{\pgfqpoint{2.217256in}{0.733947in}}%
\pgfpathlineto{\pgfqpoint{2.217500in}{0.733947in}}%
\pgfpathlineto{\pgfqpoint{2.217500in}{0.728788in}}%
\pgfpathlineto{\pgfqpoint{2.218476in}{0.733947in}}%
\pgfpathlineto{\pgfqpoint{2.218720in}{0.733947in}}%
\pgfpathlineto{\pgfqpoint{2.218720in}{0.728788in}}%
\pgfpathlineto{\pgfqpoint{2.219696in}{0.728788in}}%
\pgfpathlineto{\pgfqpoint{2.219940in}{0.728788in}}%
\pgfpathlineto{\pgfqpoint{2.220428in}{0.739107in}}%
\pgfpathlineto{\pgfqpoint{2.220915in}{0.728788in}}%
\pgfpathlineto{\pgfqpoint{2.221159in}{0.728788in}}%
\pgfpathlineto{\pgfqpoint{2.221159in}{0.733947in}}%
\pgfpathlineto{\pgfqpoint{2.222135in}{0.733947in}}%
\pgfpathlineto{\pgfqpoint{2.222867in}{0.733947in}}%
\pgfpathlineto{\pgfqpoint{2.222867in}{0.728788in}}%
\pgfpathlineto{\pgfqpoint{2.223599in}{0.739107in}}%
\pgfpathlineto{\pgfqpoint{2.223843in}{0.728788in}}%
\pgfpathlineto{\pgfqpoint{2.224331in}{0.728788in}}%
\pgfpathlineto{\pgfqpoint{2.224331in}{0.733947in}}%
\pgfpathlineto{\pgfqpoint{2.225307in}{0.728788in}}%
\pgfpathlineto{\pgfqpoint{2.225551in}{0.728788in}}%
\pgfpathlineto{\pgfqpoint{2.225551in}{0.733947in}}%
\pgfpathlineto{\pgfqpoint{2.226527in}{0.728788in}}%
\pgfpathlineto{\pgfqpoint{2.226771in}{0.728788in}}%
\pgfpathlineto{\pgfqpoint{2.226771in}{0.739107in}}%
\pgfpathlineto{\pgfqpoint{2.227746in}{0.728788in}}%
\pgfpathlineto{\pgfqpoint{2.228478in}{0.728788in}}%
\pgfpathlineto{\pgfqpoint{2.228478in}{0.739107in}}%
\pgfpathlineto{\pgfqpoint{2.229454in}{0.728788in}}%
\pgfpathlineto{\pgfqpoint{2.229698in}{0.728788in}}%
\pgfpathlineto{\pgfqpoint{2.229698in}{0.739107in}}%
\pgfpathlineto{\pgfqpoint{2.230674in}{0.733947in}}%
\pgfpathlineto{\pgfqpoint{2.231406in}{0.733947in}}%
\pgfpathlineto{\pgfqpoint{2.231406in}{0.728788in}}%
\pgfpathlineto{\pgfqpoint{2.232382in}{0.733947in}}%
\pgfpathlineto{\pgfqpoint{2.233845in}{0.733947in}}%
\pgfpathlineto{\pgfqpoint{2.233845in}{0.728788in}}%
\pgfpathlineto{\pgfqpoint{2.234089in}{0.739107in}}%
\pgfpathlineto{\pgfqpoint{2.234821in}{0.733947in}}%
\pgfpathlineto{\pgfqpoint{2.235065in}{0.733947in}}%
\pgfpathlineto{\pgfqpoint{2.235065in}{0.728788in}}%
\pgfpathlineto{\pgfqpoint{2.235553in}{0.739107in}}%
\pgfpathlineto{\pgfqpoint{2.236041in}{0.733947in}}%
\pgfpathlineto{\pgfqpoint{2.236285in}{0.733947in}}%
\pgfpathlineto{\pgfqpoint{2.236285in}{0.728788in}}%
\pgfpathlineto{\pgfqpoint{2.236773in}{0.739107in}}%
\pgfpathlineto{\pgfqpoint{2.237261in}{0.728788in}}%
\pgfpathlineto{\pgfqpoint{2.237505in}{0.728788in}}%
\pgfpathlineto{\pgfqpoint{2.237505in}{0.733947in}}%
\pgfpathlineto{\pgfqpoint{2.238481in}{0.728788in}}%
\pgfpathlineto{\pgfqpoint{2.239457in}{0.728788in}}%
\pgfpathlineto{\pgfqpoint{2.239457in}{0.739107in}}%
\pgfpathlineto{\pgfqpoint{2.240432in}{0.733947in}}%
\pgfpathlineto{\pgfqpoint{2.240920in}{0.733947in}}%
\pgfpathlineto{\pgfqpoint{2.240920in}{0.728788in}}%
\pgfpathlineto{\pgfqpoint{2.241896in}{0.733947in}}%
\pgfpathlineto{\pgfqpoint{2.242140in}{0.733947in}}%
\pgfpathlineto{\pgfqpoint{2.242140in}{0.728788in}}%
\pgfpathlineto{\pgfqpoint{2.242628in}{0.744266in}}%
\pgfpathlineto{\pgfqpoint{2.243116in}{0.728788in}}%
\pgfpathlineto{\pgfqpoint{2.244092in}{0.728788in}}%
\pgfpathlineto{\pgfqpoint{2.244092in}{0.739107in}}%
\pgfpathlineto{\pgfqpoint{2.245068in}{0.728788in}}%
\pgfpathlineto{\pgfqpoint{2.245312in}{0.728788in}}%
\pgfpathlineto{\pgfqpoint{2.245312in}{0.739107in}}%
\pgfpathlineto{\pgfqpoint{2.246288in}{0.733947in}}%
\pgfpathlineto{\pgfqpoint{2.247263in}{0.733947in}}%
\pgfpathlineto{\pgfqpoint{2.247263in}{0.739107in}}%
\pgfpathlineto{\pgfqpoint{2.247507in}{0.728788in}}%
\pgfpathlineto{\pgfqpoint{2.248239in}{0.733947in}}%
\pgfpathlineto{\pgfqpoint{2.248483in}{0.733947in}}%
\pgfpathlineto{\pgfqpoint{2.248483in}{0.728788in}}%
\pgfpathlineto{\pgfqpoint{2.249459in}{0.728788in}}%
\pgfpathlineto{\pgfqpoint{2.250923in}{0.728788in}}%
\pgfpathlineto{\pgfqpoint{2.250923in}{0.733947in}}%
\pgfpathlineto{\pgfqpoint{2.251899in}{0.728788in}}%
\pgfpathlineto{\pgfqpoint{2.252387in}{0.728788in}}%
\pgfpathlineto{\pgfqpoint{2.253119in}{0.739107in}}%
\pgfpathlineto{\pgfqpoint{2.253362in}{0.728788in}}%
\pgfpathlineto{\pgfqpoint{2.253850in}{0.728788in}}%
\pgfpathlineto{\pgfqpoint{2.253850in}{0.733947in}}%
\pgfpathlineto{\pgfqpoint{2.254826in}{0.728788in}}%
\pgfpathlineto{\pgfqpoint{2.255070in}{0.728788in}}%
\pgfpathlineto{\pgfqpoint{2.256046in}{0.744266in}}%
\pgfpathlineto{\pgfqpoint{2.256290in}{0.744266in}}%
\pgfpathlineto{\pgfqpoint{2.256290in}{0.728788in}}%
\pgfpathlineto{\pgfqpoint{2.257266in}{0.728788in}}%
\pgfpathlineto{\pgfqpoint{2.257510in}{0.728788in}}%
\pgfpathlineto{\pgfqpoint{2.257510in}{0.733947in}}%
\pgfpathlineto{\pgfqpoint{2.258486in}{0.728788in}}%
\pgfpathlineto{\pgfqpoint{2.258730in}{0.728788in}}%
\pgfpathlineto{\pgfqpoint{2.259462in}{0.739107in}}%
\pgfpathlineto{\pgfqpoint{2.259706in}{0.728788in}}%
\pgfpathlineto{\pgfqpoint{2.260437in}{0.728788in}}%
\pgfpathlineto{\pgfqpoint{2.261413in}{0.749425in}}%
\pgfpathlineto{\pgfqpoint{2.261657in}{0.749425in}}%
\pgfpathlineto{\pgfqpoint{2.261901in}{0.728788in}}%
\pgfpathlineto{\pgfqpoint{2.262633in}{0.733947in}}%
\pgfpathlineto{\pgfqpoint{2.263609in}{0.733947in}}%
\pgfpathlineto{\pgfqpoint{2.263609in}{0.728788in}}%
\pgfpathlineto{\pgfqpoint{2.264585in}{0.728788in}}%
\pgfpathlineto{\pgfqpoint{2.265073in}{0.728788in}}%
\pgfpathlineto{\pgfqpoint{2.265073in}{0.739107in}}%
\pgfpathlineto{\pgfqpoint{2.266049in}{0.733947in}}%
\pgfpathlineto{\pgfqpoint{2.266293in}{0.733947in}}%
\pgfpathlineto{\pgfqpoint{2.266293in}{0.739107in}}%
\pgfpathlineto{\pgfqpoint{2.266536in}{0.728788in}}%
\pgfpathlineto{\pgfqpoint{2.267268in}{0.728788in}}%
\pgfpathlineto{\pgfqpoint{2.268732in}{0.728788in}}%
\pgfpathlineto{\pgfqpoint{2.268732in}{0.733947in}}%
\pgfpathlineto{\pgfqpoint{2.269708in}{0.733947in}}%
\pgfpathlineto{\pgfqpoint{2.269952in}{0.733947in}}%
\pgfpathlineto{\pgfqpoint{2.269952in}{0.728788in}}%
\pgfpathlineto{\pgfqpoint{2.270928in}{0.728788in}}%
\pgfpathlineto{\pgfqpoint{2.271172in}{0.728788in}}%
\pgfpathlineto{\pgfqpoint{2.271172in}{0.733947in}}%
\pgfpathlineto{\pgfqpoint{2.272148in}{0.728788in}}%
\pgfpathlineto{\pgfqpoint{2.272636in}{0.728788in}}%
\pgfpathlineto{\pgfqpoint{2.272636in}{0.733947in}}%
\pgfpathlineto{\pgfqpoint{2.273611in}{0.728788in}}%
\pgfpathlineto{\pgfqpoint{2.273855in}{0.728788in}}%
\pgfpathlineto{\pgfqpoint{2.273855in}{0.733947in}}%
\pgfpathlineto{\pgfqpoint{2.274831in}{0.733947in}}%
\pgfpathlineto{\pgfqpoint{2.275807in}{0.733947in}}%
\pgfpathlineto{\pgfqpoint{2.275807in}{0.728788in}}%
\pgfpathlineto{\pgfqpoint{2.276783in}{0.733947in}}%
\pgfpathlineto{\pgfqpoint{2.277027in}{0.733947in}}%
\pgfpathlineto{\pgfqpoint{2.277027in}{0.728788in}}%
\pgfpathlineto{\pgfqpoint{2.278003in}{0.733947in}}%
\pgfpathlineto{\pgfqpoint{2.278247in}{0.733947in}}%
\pgfpathlineto{\pgfqpoint{2.278247in}{0.728788in}}%
\pgfpathlineto{\pgfqpoint{2.278735in}{0.739107in}}%
\pgfpathlineto{\pgfqpoint{2.279223in}{0.728788in}}%
\pgfpathlineto{\pgfqpoint{2.279954in}{0.728788in}}%
\pgfpathlineto{\pgfqpoint{2.279954in}{0.733947in}}%
\pgfpathlineto{\pgfqpoint{2.280930in}{0.728788in}}%
\pgfpathlineto{\pgfqpoint{2.281174in}{0.728788in}}%
\pgfpathlineto{\pgfqpoint{2.281174in}{0.733947in}}%
\pgfpathlineto{\pgfqpoint{2.282150in}{0.728788in}}%
\pgfpathlineto{\pgfqpoint{2.282638in}{0.728788in}}%
\pgfpathlineto{\pgfqpoint{2.282638in}{0.733947in}}%
\pgfpathlineto{\pgfqpoint{2.283614in}{0.728788in}}%
\pgfpathlineto{\pgfqpoint{2.283858in}{0.728788in}}%
\pgfpathlineto{\pgfqpoint{2.283858in}{0.733947in}}%
\pgfpathlineto{\pgfqpoint{2.284834in}{0.728788in}}%
\pgfpathlineto{\pgfqpoint{2.285566in}{0.728788in}}%
\pgfpathlineto{\pgfqpoint{2.285566in}{0.733947in}}%
\pgfpathlineto{\pgfqpoint{2.286541in}{0.733947in}}%
\pgfpathlineto{\pgfqpoint{2.286785in}{0.733947in}}%
\pgfpathlineto{\pgfqpoint{2.286785in}{0.728788in}}%
\pgfpathlineto{\pgfqpoint{2.287761in}{0.728788in}}%
\pgfpathlineto{\pgfqpoint{2.288493in}{0.728788in}}%
\pgfpathlineto{\pgfqpoint{2.288493in}{0.733947in}}%
\pgfpathlineto{\pgfqpoint{2.289469in}{0.728788in}}%
\pgfpathlineto{\pgfqpoint{2.290201in}{0.728788in}}%
\pgfpathlineto{\pgfqpoint{2.290201in}{0.733947in}}%
\pgfpathlineto{\pgfqpoint{2.291177in}{0.733947in}}%
\pgfpathlineto{\pgfqpoint{2.291421in}{0.733947in}}%
\pgfpathlineto{\pgfqpoint{2.291421in}{0.728788in}}%
\pgfpathlineto{\pgfqpoint{2.291665in}{0.739107in}}%
\pgfpathlineto{\pgfqpoint{2.292397in}{0.728788in}}%
\pgfpathlineto{\pgfqpoint{2.293128in}{0.728788in}}%
\pgfpathlineto{\pgfqpoint{2.293128in}{0.733947in}}%
\pgfpathlineto{\pgfqpoint{2.294104in}{0.728788in}}%
\pgfpathlineto{\pgfqpoint{2.294348in}{0.728788in}}%
\pgfpathlineto{\pgfqpoint{2.294348in}{0.733947in}}%
\pgfpathlineto{\pgfqpoint{2.295324in}{0.728788in}}%
\pgfpathlineto{\pgfqpoint{2.295568in}{0.728788in}}%
\pgfpathlineto{\pgfqpoint{2.296056in}{0.739107in}}%
\pgfpathlineto{\pgfqpoint{2.296544in}{0.728788in}}%
\pgfpathlineto{\pgfqpoint{2.296788in}{0.728788in}}%
\pgfpathlineto{\pgfqpoint{2.297276in}{0.744266in}}%
\pgfpathlineto{\pgfqpoint{2.297764in}{0.728788in}}%
\pgfpathlineto{\pgfqpoint{2.299715in}{0.728788in}}%
\pgfpathlineto{\pgfqpoint{2.299715in}{0.733947in}}%
\pgfpathlineto{\pgfqpoint{2.300691in}{0.728788in}}%
\pgfpathlineto{\pgfqpoint{2.301911in}{0.728788in}}%
\pgfpathlineto{\pgfqpoint{2.301911in}{0.733947in}}%
\pgfpathlineto{\pgfqpoint{2.302887in}{0.728788in}}%
\pgfpathlineto{\pgfqpoint{2.303131in}{0.728788in}}%
\pgfpathlineto{\pgfqpoint{2.303619in}{0.739107in}}%
\pgfpathlineto{\pgfqpoint{2.304107in}{0.728788in}}%
\pgfpathlineto{\pgfqpoint{2.304595in}{0.728788in}}%
\pgfpathlineto{\pgfqpoint{2.304839in}{0.744266in}}%
\pgfpathlineto{\pgfqpoint{2.305571in}{0.728788in}}%
\pgfpathlineto{\pgfqpoint{2.306058in}{0.728788in}}%
\pgfpathlineto{\pgfqpoint{2.306058in}{0.733947in}}%
\pgfpathlineto{\pgfqpoint{2.307034in}{0.728788in}}%
\pgfpathlineto{\pgfqpoint{2.307278in}{0.728788in}}%
\pgfpathlineto{\pgfqpoint{2.307278in}{0.733947in}}%
\pgfpathlineto{\pgfqpoint{2.308254in}{0.728788in}}%
\pgfpathlineto{\pgfqpoint{2.308498in}{0.728788in}}%
\pgfpathlineto{\pgfqpoint{2.308498in}{0.733947in}}%
\pgfpathlineto{\pgfqpoint{2.309474in}{0.728788in}}%
\pgfpathlineto{\pgfqpoint{2.310206in}{0.728788in}}%
\pgfpathlineto{\pgfqpoint{2.310206in}{0.733947in}}%
\pgfpathlineto{\pgfqpoint{2.311182in}{0.728788in}}%
\pgfpathlineto{\pgfqpoint{2.311426in}{0.728788in}}%
\pgfpathlineto{\pgfqpoint{2.311426in}{0.733947in}}%
\pgfpathlineto{\pgfqpoint{2.312401in}{0.728788in}}%
\pgfpathlineto{\pgfqpoint{2.314353in}{0.728788in}}%
\pgfpathlineto{\pgfqpoint{2.314353in}{0.739107in}}%
\pgfpathlineto{\pgfqpoint{2.315329in}{0.728788in}}%
\pgfpathlineto{\pgfqpoint{2.318501in}{0.728788in}}%
\pgfpathlineto{\pgfqpoint{2.319232in}{0.739107in}}%
\pgfpathlineto{\pgfqpoint{2.319476in}{0.728788in}}%
\pgfpathlineto{\pgfqpoint{2.320208in}{0.728788in}}%
\pgfpathlineto{\pgfqpoint{2.320208in}{0.733947in}}%
\pgfpathlineto{\pgfqpoint{2.321184in}{0.728788in}}%
\pgfpathlineto{\pgfqpoint{2.321672in}{0.728788in}}%
\pgfpathlineto{\pgfqpoint{2.321672in}{0.733947in}}%
\pgfpathlineto{\pgfqpoint{2.322648in}{0.728788in}}%
\pgfpathlineto{\pgfqpoint{2.324112in}{0.728788in}}%
\pgfpathlineto{\pgfqpoint{2.324112in}{0.733947in}}%
\pgfpathlineto{\pgfqpoint{2.325088in}{0.728788in}}%
\pgfpathlineto{\pgfqpoint{2.326307in}{0.728788in}}%
\pgfpathlineto{\pgfqpoint{2.326307in}{0.733947in}}%
\pgfpathlineto{\pgfqpoint{2.327283in}{0.728788in}}%
\pgfpathlineto{\pgfqpoint{2.327527in}{0.728788in}}%
\pgfpathlineto{\pgfqpoint{2.327527in}{0.733947in}}%
\pgfpathlineto{\pgfqpoint{2.328503in}{0.728788in}}%
\pgfpathlineto{\pgfqpoint{2.328747in}{0.728788in}}%
\pgfpathlineto{\pgfqpoint{2.328747in}{0.733947in}}%
\pgfpathlineto{\pgfqpoint{2.329723in}{0.728788in}}%
\pgfpathlineto{\pgfqpoint{2.329967in}{0.728788in}}%
\pgfpathlineto{\pgfqpoint{2.329967in}{0.733947in}}%
\pgfpathlineto{\pgfqpoint{2.330943in}{0.733947in}}%
\pgfpathlineto{\pgfqpoint{2.331918in}{0.733947in}}%
\pgfpathlineto{\pgfqpoint{2.331918in}{0.728788in}}%
\pgfpathlineto{\pgfqpoint{2.332894in}{0.728788in}}%
\pgfpathlineto{\pgfqpoint{2.334358in}{0.728788in}}%
\pgfpathlineto{\pgfqpoint{2.334358in}{0.733947in}}%
\pgfpathlineto{\pgfqpoint{2.335334in}{0.728788in}}%
\pgfpathlineto{\pgfqpoint{2.335578in}{0.728788in}}%
\pgfpathlineto{\pgfqpoint{2.335578in}{0.733947in}}%
\pgfpathlineto{\pgfqpoint{2.336554in}{0.728788in}}%
\pgfpathlineto{\pgfqpoint{2.338262in}{0.728788in}}%
\pgfpathlineto{\pgfqpoint{2.338262in}{0.739107in}}%
\pgfpathlineto{\pgfqpoint{2.339237in}{0.728788in}}%
\pgfpathlineto{\pgfqpoint{2.340701in}{0.728788in}}%
\pgfpathlineto{\pgfqpoint{2.340701in}{0.733947in}}%
\pgfpathlineto{\pgfqpoint{2.341677in}{0.728788in}}%
\pgfpathlineto{\pgfqpoint{2.344605in}{0.728788in}}%
\pgfpathlineto{\pgfqpoint{2.345580in}{0.739107in}}%
\pgfpathlineto{\pgfqpoint{2.345824in}{0.739107in}}%
\pgfpathlineto{\pgfqpoint{2.345824in}{0.728788in}}%
\pgfpathlineto{\pgfqpoint{2.346800in}{0.728788in}}%
\pgfpathlineto{\pgfqpoint{2.347288in}{0.728788in}}%
\pgfpathlineto{\pgfqpoint{2.347288in}{0.739107in}}%
\pgfpathlineto{\pgfqpoint{2.348264in}{0.728788in}}%
\pgfpathlineto{\pgfqpoint{2.348508in}{0.728788in}}%
\pgfpathlineto{\pgfqpoint{2.348508in}{0.733947in}}%
\pgfpathlineto{\pgfqpoint{2.349484in}{0.728788in}}%
\pgfpathlineto{\pgfqpoint{2.350460in}{0.728788in}}%
\pgfpathlineto{\pgfqpoint{2.350460in}{0.733947in}}%
\pgfpathlineto{\pgfqpoint{2.351436in}{0.728788in}}%
\pgfpathlineto{\pgfqpoint{2.351679in}{0.728788in}}%
\pgfpathlineto{\pgfqpoint{2.351679in}{0.733947in}}%
\pgfpathlineto{\pgfqpoint{2.352655in}{0.728788in}}%
\pgfpathlineto{\pgfqpoint{2.353875in}{0.728788in}}%
\pgfpathlineto{\pgfqpoint{2.353875in}{0.733947in}}%
\pgfpathlineto{\pgfqpoint{2.354851in}{0.728788in}}%
\pgfpathlineto{\pgfqpoint{2.355095in}{0.728788in}}%
\pgfpathlineto{\pgfqpoint{2.355095in}{0.733947in}}%
\pgfpathlineto{\pgfqpoint{2.356071in}{0.728788in}}%
\pgfpathlineto{\pgfqpoint{2.356803in}{0.728788in}}%
\pgfpathlineto{\pgfqpoint{2.356803in}{0.733947in}}%
\pgfpathlineto{\pgfqpoint{2.357779in}{0.728788in}}%
\pgfpathlineto{\pgfqpoint{2.358998in}{0.728788in}}%
\pgfpathlineto{\pgfqpoint{2.359242in}{0.739107in}}%
\pgfpathlineto{\pgfqpoint{2.359974in}{0.728788in}}%
\pgfpathlineto{\pgfqpoint{2.361194in}{0.728788in}}%
\pgfpathlineto{\pgfqpoint{2.361194in}{0.733947in}}%
\pgfpathlineto{\pgfqpoint{2.362170in}{0.733947in}}%
\pgfpathlineto{\pgfqpoint{2.362414in}{0.733947in}}%
\pgfpathlineto{\pgfqpoint{2.362414in}{0.728788in}}%
\pgfpathlineto{\pgfqpoint{2.363390in}{0.728788in}}%
\pgfpathlineto{\pgfqpoint{2.363634in}{0.728788in}}%
\pgfpathlineto{\pgfqpoint{2.363634in}{0.733947in}}%
\pgfpathlineto{\pgfqpoint{2.364609in}{0.728788in}}%
\pgfpathlineto{\pgfqpoint{2.364853in}{0.728788in}}%
\pgfpathlineto{\pgfqpoint{2.364853in}{0.733947in}}%
\pgfpathlineto{\pgfqpoint{2.365829in}{0.728788in}}%
\pgfpathlineto{\pgfqpoint{2.366561in}{0.728788in}}%
\pgfpathlineto{\pgfqpoint{2.366561in}{0.733947in}}%
\pgfpathlineto{\pgfqpoint{2.367537in}{0.733947in}}%
\pgfpathlineto{\pgfqpoint{2.367781in}{0.733947in}}%
\pgfpathlineto{\pgfqpoint{2.367781in}{0.728788in}}%
\pgfpathlineto{\pgfqpoint{2.368757in}{0.728788in}}%
\pgfpathlineto{\pgfqpoint{2.371196in}{0.728788in}}%
\pgfpathlineto{\pgfqpoint{2.371196in}{0.733947in}}%
\pgfpathlineto{\pgfqpoint{2.372172in}{0.728788in}}%
\pgfpathlineto{\pgfqpoint{2.372660in}{0.728788in}}%
\pgfpathlineto{\pgfqpoint{2.372660in}{0.733947in}}%
\pgfpathlineto{\pgfqpoint{2.373636in}{0.728788in}}%
\pgfpathlineto{\pgfqpoint{2.374368in}{0.728788in}}%
\pgfpathlineto{\pgfqpoint{2.374368in}{0.739107in}}%
\pgfpathlineto{\pgfqpoint{2.375344in}{0.728788in}}%
\pgfpathlineto{\pgfqpoint{2.375588in}{0.728788in}}%
\pgfpathlineto{\pgfqpoint{2.375588in}{0.733947in}}%
\pgfpathlineto{\pgfqpoint{2.376564in}{0.728788in}}%
\pgfpathlineto{\pgfqpoint{2.381931in}{0.728788in}}%
\pgfpathlineto{\pgfqpoint{2.381931in}{0.739107in}}%
\pgfpathlineto{\pgfqpoint{2.382907in}{0.728788in}}%
\pgfpathlineto{\pgfqpoint{2.385102in}{0.728788in}}%
\pgfpathlineto{\pgfqpoint{2.385102in}{0.733947in}}%
\pgfpathlineto{\pgfqpoint{2.386078in}{0.728788in}}%
\pgfpathlineto{\pgfqpoint{2.387054in}{0.728788in}}%
\pgfpathlineto{\pgfqpoint{2.387054in}{0.733947in}}%
\pgfpathlineto{\pgfqpoint{2.388030in}{0.728788in}}%
\pgfpathlineto{\pgfqpoint{2.389250in}{0.728788in}}%
\pgfpathlineto{\pgfqpoint{2.389250in}{0.733947in}}%
\pgfpathlineto{\pgfqpoint{2.390226in}{0.728788in}}%
\pgfpathlineto{\pgfqpoint{2.390957in}{0.728788in}}%
\pgfpathlineto{\pgfqpoint{2.390957in}{0.733947in}}%
\pgfpathlineto{\pgfqpoint{2.391933in}{0.733947in}}%
\pgfpathlineto{\pgfqpoint{2.392177in}{0.733947in}}%
\pgfpathlineto{\pgfqpoint{2.392177in}{0.728788in}}%
\pgfpathlineto{\pgfqpoint{2.393153in}{0.728788in}}%
\pgfpathlineto{\pgfqpoint{2.393885in}{0.728788in}}%
\pgfpathlineto{\pgfqpoint{2.393885in}{0.733947in}}%
\pgfpathlineto{\pgfqpoint{2.394861in}{0.728788in}}%
\pgfpathlineto{\pgfqpoint{2.395837in}{0.728788in}}%
\pgfpathlineto{\pgfqpoint{2.395837in}{0.733947in}}%
\pgfpathlineto{\pgfqpoint{2.396813in}{0.728788in}}%
\pgfpathlineto{\pgfqpoint{2.397300in}{0.728788in}}%
\pgfpathlineto{\pgfqpoint{2.397544in}{0.739107in}}%
\pgfpathlineto{\pgfqpoint{2.398276in}{0.728788in}}%
\pgfpathlineto{\pgfqpoint{2.401204in}{0.728788in}}%
\pgfpathlineto{\pgfqpoint{2.401204in}{0.739107in}}%
\pgfpathlineto{\pgfqpoint{2.402180in}{0.733947in}}%
\pgfpathlineto{\pgfqpoint{2.402668in}{0.733947in}}%
\pgfpathlineto{\pgfqpoint{2.402668in}{0.728788in}}%
\pgfpathlineto{\pgfqpoint{2.403644in}{0.728788in}}%
\pgfpathlineto{\pgfqpoint{2.404375in}{0.728788in}}%
\pgfpathlineto{\pgfqpoint{2.404863in}{0.739107in}}%
\pgfpathlineto{\pgfqpoint{2.405351in}{0.728788in}}%
\pgfpathlineto{\pgfqpoint{2.405595in}{0.728788in}}%
\pgfpathlineto{\pgfqpoint{2.405595in}{0.733947in}}%
\pgfpathlineto{\pgfqpoint{2.406571in}{0.728788in}}%
\pgfpathlineto{\pgfqpoint{2.406815in}{0.728788in}}%
\pgfpathlineto{\pgfqpoint{2.407059in}{0.739107in}}%
\pgfpathlineto{\pgfqpoint{2.407791in}{0.733947in}}%
\pgfpathlineto{\pgfqpoint{2.408035in}{0.733947in}}%
\pgfpathlineto{\pgfqpoint{2.408035in}{0.728788in}}%
\pgfpathlineto{\pgfqpoint{2.409011in}{0.733947in}}%
\pgfpathlineto{\pgfqpoint{2.409255in}{0.733947in}}%
\pgfpathlineto{\pgfqpoint{2.409255in}{0.728788in}}%
\pgfpathlineto{\pgfqpoint{2.410231in}{0.733947in}}%
\pgfpathlineto{\pgfqpoint{2.410474in}{0.733947in}}%
\pgfpathlineto{\pgfqpoint{2.410474in}{0.728788in}}%
\pgfpathlineto{\pgfqpoint{2.410718in}{0.739107in}}%
\pgfpathlineto{\pgfqpoint{2.411450in}{0.728788in}}%
\pgfpathlineto{\pgfqpoint{2.412182in}{0.728788in}}%
\pgfpathlineto{\pgfqpoint{2.412182in}{0.733947in}}%
\pgfpathlineto{\pgfqpoint{2.413158in}{0.728788in}}%
\pgfpathlineto{\pgfqpoint{2.414866in}{0.728788in}}%
\pgfpathlineto{\pgfqpoint{2.414866in}{0.733947in}}%
\pgfpathlineto{\pgfqpoint{2.415842in}{0.728788in}}%
\pgfpathlineto{\pgfqpoint{2.418525in}{0.728788in}}%
\pgfpathlineto{\pgfqpoint{2.418525in}{0.733947in}}%
\pgfpathlineto{\pgfqpoint{2.419501in}{0.728788in}}%
\pgfpathlineto{\pgfqpoint{2.419989in}{0.728788in}}%
\pgfpathlineto{\pgfqpoint{2.419989in}{0.733947in}}%
\pgfpathlineto{\pgfqpoint{2.420965in}{0.728788in}}%
\pgfpathlineto{\pgfqpoint{2.422429in}{0.728788in}}%
\pgfpathlineto{\pgfqpoint{2.423161in}{0.739107in}}%
\pgfpathlineto{\pgfqpoint{2.423405in}{0.728788in}}%
\pgfpathlineto{\pgfqpoint{2.426576in}{0.728788in}}%
\pgfpathlineto{\pgfqpoint{2.426820in}{0.739107in}}%
\pgfpathlineto{\pgfqpoint{2.427552in}{0.728788in}}%
\pgfpathlineto{\pgfqpoint{2.428284in}{0.728788in}}%
\pgfpathlineto{\pgfqpoint{2.428284in}{0.739107in}}%
\pgfpathlineto{\pgfqpoint{2.429260in}{0.728788in}}%
\pgfpathlineto{\pgfqpoint{2.429504in}{0.728788in}}%
\pgfpathlineto{\pgfqpoint{2.429504in}{0.739107in}}%
\pgfpathlineto{\pgfqpoint{2.430479in}{0.728788in}}%
\pgfpathlineto{\pgfqpoint{2.432675in}{0.728788in}}%
\pgfpathlineto{\pgfqpoint{2.432675in}{0.733947in}}%
\pgfpathlineto{\pgfqpoint{2.433651in}{0.733947in}}%
\pgfpathlineto{\pgfqpoint{2.433895in}{0.733947in}}%
\pgfpathlineto{\pgfqpoint{2.433895in}{0.728788in}}%
\pgfpathlineto{\pgfqpoint{2.434871in}{0.728788in}}%
\pgfpathlineto{\pgfqpoint{2.435847in}{0.728788in}}%
\pgfpathlineto{\pgfqpoint{2.435847in}{0.739107in}}%
\pgfpathlineto{\pgfqpoint{2.436822in}{0.728788in}}%
\pgfpathlineto{\pgfqpoint{2.437554in}{0.728788in}}%
\pgfpathlineto{\pgfqpoint{2.437554in}{0.733947in}}%
\pgfpathlineto{\pgfqpoint{2.438530in}{0.728788in}}%
\pgfpathlineto{\pgfqpoint{2.439262in}{0.728788in}}%
\pgfpathlineto{\pgfqpoint{2.439262in}{0.733947in}}%
\pgfpathlineto{\pgfqpoint{2.440238in}{0.733947in}}%
\pgfpathlineto{\pgfqpoint{2.440482in}{0.733947in}}%
\pgfpathlineto{\pgfqpoint{2.440482in}{0.728788in}}%
\pgfpathlineto{\pgfqpoint{2.441214in}{0.739107in}}%
\pgfpathlineto{\pgfqpoint{2.441458in}{0.728788in}}%
\pgfpathlineto{\pgfqpoint{2.442922in}{0.728788in}}%
\pgfpathlineto{\pgfqpoint{2.442922in}{0.733947in}}%
\pgfpathlineto{\pgfqpoint{2.443897in}{0.733947in}}%
\pgfpathlineto{\pgfqpoint{2.444141in}{0.733947in}}%
\pgfpathlineto{\pgfqpoint{2.444141in}{0.728788in}}%
\pgfpathlineto{\pgfqpoint{2.445117in}{0.728788in}}%
\pgfpathlineto{\pgfqpoint{2.445605in}{0.728788in}}%
\pgfpathlineto{\pgfqpoint{2.445605in}{0.733947in}}%
\pgfpathlineto{\pgfqpoint{2.446581in}{0.728788in}}%
\pgfpathlineto{\pgfqpoint{2.447069in}{0.728788in}}%
\pgfpathlineto{\pgfqpoint{2.447069in}{0.739107in}}%
\pgfpathlineto{\pgfqpoint{2.448045in}{0.728788in}}%
\pgfpathlineto{\pgfqpoint{2.448289in}{0.728788in}}%
\pgfpathlineto{\pgfqpoint{2.449021in}{0.739107in}}%
\pgfpathlineto{\pgfqpoint{2.449265in}{0.728788in}}%
\pgfpathlineto{\pgfqpoint{2.449996in}{0.728788in}}%
\pgfpathlineto{\pgfqpoint{2.449996in}{0.739107in}}%
\pgfpathlineto{\pgfqpoint{2.450972in}{0.728788in}}%
\pgfpathlineto{\pgfqpoint{2.451704in}{0.728788in}}%
\pgfpathlineto{\pgfqpoint{2.451704in}{0.733947in}}%
\pgfpathlineto{\pgfqpoint{2.452680in}{0.728788in}}%
\pgfpathlineto{\pgfqpoint{2.454144in}{0.728788in}}%
\pgfpathlineto{\pgfqpoint{2.454632in}{0.739107in}}%
\pgfpathlineto{\pgfqpoint{2.455120in}{0.733947in}}%
\pgfpathlineto{\pgfqpoint{2.455364in}{0.733947in}}%
\pgfpathlineto{\pgfqpoint{2.455364in}{0.728788in}}%
\pgfpathlineto{\pgfqpoint{2.456339in}{0.728788in}}%
\pgfpathlineto{\pgfqpoint{2.457315in}{0.728788in}}%
\pgfpathlineto{\pgfqpoint{2.457315in}{0.739107in}}%
\pgfpathlineto{\pgfqpoint{2.458291in}{0.728788in}}%
\pgfpathlineto{\pgfqpoint{2.459511in}{0.728788in}}%
\pgfpathlineto{\pgfqpoint{2.459511in}{0.739107in}}%
\pgfpathlineto{\pgfqpoint{2.460487in}{0.728788in}}%
\pgfpathlineto{\pgfqpoint{2.460731in}{0.728788in}}%
\pgfpathlineto{\pgfqpoint{2.460731in}{0.733947in}}%
\pgfpathlineto{\pgfqpoint{2.461707in}{0.733947in}}%
\pgfpathlineto{\pgfqpoint{2.462195in}{0.733947in}}%
\pgfpathlineto{\pgfqpoint{2.462195in}{0.728788in}}%
\pgfpathlineto{\pgfqpoint{2.463170in}{0.733947in}}%
\pgfpathlineto{\pgfqpoint{2.463658in}{0.733947in}}%
\pgfpathlineto{\pgfqpoint{2.463658in}{0.728788in}}%
\pgfpathlineto{\pgfqpoint{2.464634in}{0.728788in}}%
\pgfpathlineto{\pgfqpoint{2.466830in}{0.728788in}}%
\pgfpathlineto{\pgfqpoint{2.466830in}{0.733947in}}%
\pgfpathlineto{\pgfqpoint{2.467806in}{0.728788in}}%
\pgfpathlineto{\pgfqpoint{2.468782in}{0.728788in}}%
\pgfpathlineto{\pgfqpoint{2.468782in}{0.733947in}}%
\pgfpathlineto{\pgfqpoint{2.469757in}{0.728788in}}%
\pgfpathlineto{\pgfqpoint{2.470489in}{0.728788in}}%
\pgfpathlineto{\pgfqpoint{2.470489in}{0.733947in}}%
\pgfpathlineto{\pgfqpoint{2.471465in}{0.728788in}}%
\pgfpathlineto{\pgfqpoint{2.473417in}{0.728788in}}%
\pgfpathlineto{\pgfqpoint{2.473417in}{0.733947in}}%
\pgfpathlineto{\pgfqpoint{2.474393in}{0.728788in}}%
\pgfpathlineto{\pgfqpoint{2.474881in}{0.728788in}}%
\pgfpathlineto{\pgfqpoint{2.474881in}{0.733947in}}%
\pgfpathlineto{\pgfqpoint{2.475856in}{0.728788in}}%
\pgfpathlineto{\pgfqpoint{2.476832in}{0.728788in}}%
\pgfpathlineto{\pgfqpoint{2.477564in}{0.739107in}}%
\pgfpathlineto{\pgfqpoint{2.477808in}{0.728788in}}%
\pgfpathlineto{\pgfqpoint{2.478052in}{0.728788in}}%
\pgfpathlineto{\pgfqpoint{2.478052in}{0.733947in}}%
\pgfpathlineto{\pgfqpoint{2.479028in}{0.733947in}}%
\pgfpathlineto{\pgfqpoint{2.479272in}{0.733947in}}%
\pgfpathlineto{\pgfqpoint{2.479272in}{0.728788in}}%
\pgfpathlineto{\pgfqpoint{2.480248in}{0.728788in}}%
\pgfpathlineto{\pgfqpoint{2.481468in}{0.728788in}}%
\pgfpathlineto{\pgfqpoint{2.481468in}{0.733947in}}%
\pgfpathlineto{\pgfqpoint{2.482443in}{0.728788in}}%
\pgfpathlineto{\pgfqpoint{2.482931in}{0.728788in}}%
\pgfpathlineto{\pgfqpoint{2.482931in}{0.733947in}}%
\pgfpathlineto{\pgfqpoint{2.483907in}{0.728788in}}%
\pgfpathlineto{\pgfqpoint{2.485615in}{0.728788in}}%
\pgfpathlineto{\pgfqpoint{2.486347in}{0.739107in}}%
\pgfpathlineto{\pgfqpoint{2.486591in}{0.733947in}}%
\pgfpathlineto{\pgfqpoint{2.486835in}{0.733947in}}%
\pgfpathlineto{\pgfqpoint{2.486835in}{0.728788in}}%
\pgfpathlineto{\pgfqpoint{2.487811in}{0.728788in}}%
\pgfpathlineto{\pgfqpoint{2.489274in}{0.728788in}}%
\pgfpathlineto{\pgfqpoint{2.489274in}{0.733947in}}%
\pgfpathlineto{\pgfqpoint{2.490250in}{0.728788in}}%
\pgfpathlineto{\pgfqpoint{2.490738in}{0.728788in}}%
\pgfpathlineto{\pgfqpoint{2.490738in}{0.733947in}}%
\pgfpathlineto{\pgfqpoint{2.491714in}{0.728788in}}%
\pgfpathlineto{\pgfqpoint{2.492934in}{0.728788in}}%
\pgfpathlineto{\pgfqpoint{2.492934in}{0.739107in}}%
\pgfpathlineto{\pgfqpoint{2.493910in}{0.728788in}}%
\pgfpathlineto{\pgfqpoint{2.495617in}{0.728788in}}%
\pgfpathlineto{\pgfqpoint{2.495617in}{0.733947in}}%
\pgfpathlineto{\pgfqpoint{2.496593in}{0.728788in}}%
\pgfpathlineto{\pgfqpoint{2.497325in}{0.728788in}}%
\pgfpathlineto{\pgfqpoint{2.497325in}{0.733947in}}%
\pgfpathlineto{\pgfqpoint{2.498301in}{0.733947in}}%
\pgfpathlineto{\pgfqpoint{2.498545in}{0.733947in}}%
\pgfpathlineto{\pgfqpoint{2.498545in}{0.728788in}}%
\pgfpathlineto{\pgfqpoint{2.499521in}{0.728788in}}%
\pgfpathlineto{\pgfqpoint{2.500741in}{0.728788in}}%
\pgfpathlineto{\pgfqpoint{2.500741in}{0.733947in}}%
\pgfpathlineto{\pgfqpoint{2.501717in}{0.728788in}}%
\pgfpathlineto{\pgfqpoint{2.502936in}{0.728788in}}%
\pgfpathlineto{\pgfqpoint{2.502936in}{0.733947in}}%
\pgfpathlineto{\pgfqpoint{2.503912in}{0.733947in}}%
\pgfpathlineto{\pgfqpoint{2.504156in}{0.733947in}}%
\pgfpathlineto{\pgfqpoint{2.504156in}{0.739107in}}%
\pgfpathlineto{\pgfqpoint{2.504400in}{0.728788in}}%
\pgfpathlineto{\pgfqpoint{2.505132in}{0.728788in}}%
\pgfpathlineto{\pgfqpoint{2.506352in}{0.728788in}}%
\pgfpathlineto{\pgfqpoint{2.506352in}{0.733947in}}%
\pgfpathlineto{\pgfqpoint{2.507328in}{0.733947in}}%
\pgfpathlineto{\pgfqpoint{2.507572in}{0.733947in}}%
\pgfpathlineto{\pgfqpoint{2.507572in}{0.728788in}}%
\pgfpathlineto{\pgfqpoint{2.508547in}{0.728788in}}%
\pgfpathlineto{\pgfqpoint{2.509523in}{0.728788in}}%
\pgfpathlineto{\pgfqpoint{2.509523in}{0.733947in}}%
\pgfpathlineto{\pgfqpoint{2.510499in}{0.728788in}}%
\pgfpathlineto{\pgfqpoint{2.511719in}{0.728788in}}%
\pgfpathlineto{\pgfqpoint{2.511719in}{0.733947in}}%
\pgfpathlineto{\pgfqpoint{2.512695in}{0.728788in}}%
\pgfpathlineto{\pgfqpoint{2.515378in}{0.728788in}}%
\pgfpathlineto{\pgfqpoint{2.515866in}{0.739107in}}%
\pgfpathlineto{\pgfqpoint{2.516354in}{0.728788in}}%
\pgfpathlineto{\pgfqpoint{2.517818in}{0.728788in}}%
\pgfpathlineto{\pgfqpoint{2.517818in}{0.733947in}}%
\pgfpathlineto{\pgfqpoint{2.518794in}{0.728788in}}%
\pgfpathlineto{\pgfqpoint{2.519282in}{0.728788in}}%
\pgfpathlineto{\pgfqpoint{2.519282in}{0.739107in}}%
\pgfpathlineto{\pgfqpoint{2.520258in}{0.728788in}}%
\pgfpathlineto{\pgfqpoint{2.522697in}{0.728788in}}%
\pgfpathlineto{\pgfqpoint{2.522697in}{0.733947in}}%
\pgfpathlineto{\pgfqpoint{2.523673in}{0.728788in}}%
\pgfpathlineto{\pgfqpoint{2.524161in}{0.728788in}}%
\pgfpathlineto{\pgfqpoint{2.524161in}{0.733947in}}%
\pgfpathlineto{\pgfqpoint{2.525137in}{0.728788in}}%
\pgfpathlineto{\pgfqpoint{2.525625in}{0.728788in}}%
\pgfpathlineto{\pgfqpoint{2.525625in}{0.733947in}}%
\pgfpathlineto{\pgfqpoint{2.526601in}{0.728788in}}%
\pgfpathlineto{\pgfqpoint{2.527333in}{0.728788in}}%
\pgfpathlineto{\pgfqpoint{2.527333in}{0.733947in}}%
\pgfpathlineto{\pgfqpoint{2.528308in}{0.733947in}}%
\pgfpathlineto{\pgfqpoint{2.528552in}{0.733947in}}%
\pgfpathlineto{\pgfqpoint{2.528552in}{0.728788in}}%
\pgfpathlineto{\pgfqpoint{2.529528in}{0.728788in}}%
\pgfpathlineto{\pgfqpoint{2.529772in}{0.728788in}}%
\pgfpathlineto{\pgfqpoint{2.529772in}{0.733947in}}%
\pgfpathlineto{\pgfqpoint{2.530748in}{0.733947in}}%
\pgfpathlineto{\pgfqpoint{2.530992in}{0.733947in}}%
\pgfpathlineto{\pgfqpoint{2.530992in}{0.728788in}}%
\pgfpathlineto{\pgfqpoint{2.531968in}{0.728788in}}%
\pgfpathlineto{\pgfqpoint{2.532212in}{0.728788in}}%
\pgfpathlineto{\pgfqpoint{2.532212in}{0.733947in}}%
\pgfpathlineto{\pgfqpoint{2.533188in}{0.728788in}}%
\pgfpathlineto{\pgfqpoint{2.533432in}{0.728788in}}%
\pgfpathlineto{\pgfqpoint{2.533432in}{0.733947in}}%
\pgfpathlineto{\pgfqpoint{2.534408in}{0.728788in}}%
\pgfpathlineto{\pgfqpoint{2.535871in}{0.728788in}}%
\pgfpathlineto{\pgfqpoint{2.535871in}{0.733947in}}%
\pgfpathlineto{\pgfqpoint{2.536847in}{0.728788in}}%
\pgfpathlineto{\pgfqpoint{2.538311in}{0.728788in}}%
\pgfpathlineto{\pgfqpoint{2.538311in}{0.733947in}}%
\pgfpathlineto{\pgfqpoint{2.539287in}{0.728788in}}%
\pgfpathlineto{\pgfqpoint{2.541482in}{0.728788in}}%
\pgfpathlineto{\pgfqpoint{2.541482in}{0.733947in}}%
\pgfpathlineto{\pgfqpoint{2.542458in}{0.728788in}}%
\pgfpathlineto{\pgfqpoint{2.543434in}{0.728788in}}%
\pgfpathlineto{\pgfqpoint{2.543434in}{0.733947in}}%
\pgfpathlineto{\pgfqpoint{2.544410in}{0.728788in}}%
\pgfpathlineto{\pgfqpoint{2.544654in}{0.728788in}}%
\pgfpathlineto{\pgfqpoint{2.544654in}{0.733947in}}%
\pgfpathlineto{\pgfqpoint{2.545630in}{0.728788in}}%
\pgfpathlineto{\pgfqpoint{2.547094in}{0.728788in}}%
\pgfpathlineto{\pgfqpoint{2.547094in}{0.733947in}}%
\pgfpathlineto{\pgfqpoint{2.548069in}{0.728788in}}%
\pgfpathlineto{\pgfqpoint{2.552949in}{0.728788in}}%
\pgfpathlineto{\pgfqpoint{2.552949in}{0.733947in}}%
\pgfpathlineto{\pgfqpoint{2.553925in}{0.728788in}}%
\pgfpathlineto{\pgfqpoint{2.557584in}{0.728788in}}%
\pgfpathlineto{\pgfqpoint{2.557584in}{0.733947in}}%
\pgfpathlineto{\pgfqpoint{2.558560in}{0.728788in}}%
\pgfpathlineto{\pgfqpoint{2.558804in}{0.728788in}}%
\pgfpathlineto{\pgfqpoint{2.558804in}{0.733947in}}%
\pgfpathlineto{\pgfqpoint{2.559780in}{0.728788in}}%
\pgfpathlineto{\pgfqpoint{2.561487in}{0.728788in}}%
\pgfpathlineto{\pgfqpoint{2.561487in}{0.733947in}}%
\pgfpathlineto{\pgfqpoint{2.562463in}{0.728788in}}%
\pgfpathlineto{\pgfqpoint{2.563927in}{0.728788in}}%
\pgfpathlineto{\pgfqpoint{2.563927in}{0.733947in}}%
\pgfpathlineto{\pgfqpoint{2.564903in}{0.728788in}}%
\pgfpathlineto{\pgfqpoint{2.565391in}{0.728788in}}%
\pgfpathlineto{\pgfqpoint{2.565391in}{0.733947in}}%
\pgfpathlineto{\pgfqpoint{2.566367in}{0.728788in}}%
\pgfpathlineto{\pgfqpoint{2.567830in}{0.728788in}}%
\pgfpathlineto{\pgfqpoint{2.567830in}{0.733947in}}%
\pgfpathlineto{\pgfqpoint{2.568806in}{0.728788in}}%
\pgfpathlineto{\pgfqpoint{2.569050in}{0.728788in}}%
\pgfpathlineto{\pgfqpoint{2.569050in}{0.739107in}}%
\pgfpathlineto{\pgfqpoint{2.570026in}{0.733947in}}%
\pgfpathlineto{\pgfqpoint{2.570270in}{0.733947in}}%
\pgfpathlineto{\pgfqpoint{2.570270in}{0.728788in}}%
\pgfpathlineto{\pgfqpoint{2.571246in}{0.728788in}}%
\pgfpathlineto{\pgfqpoint{2.575149in}{0.728788in}}%
\pgfpathlineto{\pgfqpoint{2.575149in}{0.733947in}}%
\pgfpathlineto{\pgfqpoint{2.576125in}{0.728788in}}%
\pgfpathlineto{\pgfqpoint{2.577345in}{0.728788in}}%
\pgfpathlineto{\pgfqpoint{2.577345in}{0.733947in}}%
\pgfpathlineto{\pgfqpoint{2.578321in}{0.728788in}}%
\pgfpathlineto{\pgfqpoint{2.580516in}{0.728788in}}%
\pgfpathlineto{\pgfqpoint{2.580516in}{0.733947in}}%
\pgfpathlineto{\pgfqpoint{2.581492in}{0.728788in}}%
\pgfpathlineto{\pgfqpoint{2.581980in}{0.728788in}}%
\pgfpathlineto{\pgfqpoint{2.581980in}{0.733947in}}%
\pgfpathlineto{\pgfqpoint{2.582956in}{0.733947in}}%
\pgfpathlineto{\pgfqpoint{2.583200in}{0.733947in}}%
\pgfpathlineto{\pgfqpoint{2.583200in}{0.728788in}}%
\pgfpathlineto{\pgfqpoint{2.584176in}{0.728788in}}%
\pgfpathlineto{\pgfqpoint{2.587591in}{0.728788in}}%
\pgfpathlineto{\pgfqpoint{2.587591in}{0.733947in}}%
\pgfpathlineto{\pgfqpoint{2.588567in}{0.728788in}}%
\pgfpathlineto{\pgfqpoint{2.589055in}{0.728788in}}%
\pgfpathlineto{\pgfqpoint{2.589055in}{0.733947in}}%
\pgfpathlineto{\pgfqpoint{2.590031in}{0.728788in}}%
\pgfpathlineto{\pgfqpoint{2.591495in}{0.728788in}}%
\pgfpathlineto{\pgfqpoint{2.591495in}{0.733947in}}%
\pgfpathlineto{\pgfqpoint{2.592471in}{0.728788in}}%
\pgfpathlineto{\pgfqpoint{2.593203in}{0.728788in}}%
\pgfpathlineto{\pgfqpoint{2.593203in}{0.733947in}}%
\pgfpathlineto{\pgfqpoint{2.594178in}{0.728788in}}%
\pgfpathlineto{\pgfqpoint{2.595154in}{0.728788in}}%
\pgfpathlineto{\pgfqpoint{2.595154in}{0.733947in}}%
\pgfpathlineto{\pgfqpoint{2.596130in}{0.728788in}}%
\pgfpathlineto{\pgfqpoint{2.600033in}{0.728788in}}%
\pgfpathlineto{\pgfqpoint{2.600033in}{0.733947in}}%
\pgfpathlineto{\pgfqpoint{2.601009in}{0.728788in}}%
\pgfpathlineto{\pgfqpoint{2.603449in}{0.728788in}}%
\pgfpathlineto{\pgfqpoint{2.603449in}{0.733947in}}%
\pgfpathlineto{\pgfqpoint{2.604425in}{0.728788in}}%
\pgfpathlineto{\pgfqpoint{2.611012in}{0.728788in}}%
\pgfpathlineto{\pgfqpoint{2.611012in}{0.733947in}}%
\pgfpathlineto{\pgfqpoint{2.611988in}{0.728788in}}%
\pgfpathlineto{\pgfqpoint{2.613695in}{0.728788in}}%
\pgfpathlineto{\pgfqpoint{2.613695in}{0.733947in}}%
\pgfpathlineto{\pgfqpoint{2.614671in}{0.728788in}}%
\pgfpathlineto{\pgfqpoint{2.617355in}{0.728788in}}%
\pgfpathlineto{\pgfqpoint{2.617355in}{0.733947in}}%
\pgfpathlineto{\pgfqpoint{2.618331in}{0.728788in}}%
\pgfpathlineto{\pgfqpoint{2.635164in}{0.728788in}}%
\pgfpathlineto{\pgfqpoint{2.635164in}{0.733947in}}%
\pgfpathlineto{\pgfqpoint{2.636140in}{0.728788in}}%
\pgfpathlineto{\pgfqpoint{2.636628in}{0.728788in}}%
\pgfpathlineto{\pgfqpoint{2.636628in}{0.733947in}}%
\pgfpathlineto{\pgfqpoint{2.637604in}{0.728788in}}%
\pgfpathlineto{\pgfqpoint{2.641507in}{0.728788in}}%
\pgfpathlineto{\pgfqpoint{2.641507in}{0.733947in}}%
\pgfpathlineto{\pgfqpoint{2.642483in}{0.728788in}}%
\pgfpathlineto{\pgfqpoint{2.643703in}{0.728788in}}%
\pgfpathlineto{\pgfqpoint{2.643703in}{0.733947in}}%
\pgfpathlineto{\pgfqpoint{2.644679in}{0.728788in}}%
\pgfpathlineto{\pgfqpoint{2.648094in}{0.728788in}}%
\pgfpathlineto{\pgfqpoint{2.648094in}{0.733947in}}%
\pgfpathlineto{\pgfqpoint{2.649070in}{0.728788in}}%
\pgfpathlineto{\pgfqpoint{2.654681in}{0.728788in}}%
\pgfpathlineto{\pgfqpoint{2.654681in}{0.733947in}}%
\pgfpathlineto{\pgfqpoint{2.655657in}{0.728788in}}%
\pgfpathlineto{\pgfqpoint{2.655901in}{0.728788in}}%
\pgfpathlineto{\pgfqpoint{2.655901in}{0.739107in}}%
\pgfpathlineto{\pgfqpoint{2.656877in}{0.728788in}}%
\pgfpathlineto{\pgfqpoint{2.660292in}{0.728788in}}%
\pgfpathlineto{\pgfqpoint{2.660292in}{0.733947in}}%
\pgfpathlineto{\pgfqpoint{2.661268in}{0.728788in}}%
\pgfpathlineto{\pgfqpoint{2.664928in}{0.728788in}}%
\pgfpathlineto{\pgfqpoint{2.664928in}{0.733947in}}%
\pgfpathlineto{\pgfqpoint{2.665903in}{0.728788in}}%
\pgfpathlineto{\pgfqpoint{2.668587in}{0.728788in}}%
\pgfpathlineto{\pgfqpoint{2.668587in}{0.733947in}}%
\pgfpathlineto{\pgfqpoint{2.669563in}{0.728788in}}%
\pgfpathlineto{\pgfqpoint{2.671515in}{0.728788in}}%
\pgfpathlineto{\pgfqpoint{2.671515in}{0.733947in}}%
\pgfpathlineto{\pgfqpoint{2.672490in}{0.728788in}}%
\pgfpathlineto{\pgfqpoint{2.681029in}{0.728788in}}%
\pgfpathlineto{\pgfqpoint{2.681029in}{0.733947in}}%
\pgfpathlineto{\pgfqpoint{2.682005in}{0.728788in}}%
\pgfpathlineto{\pgfqpoint{2.682737in}{0.728788in}}%
\pgfpathlineto{\pgfqpoint{2.682737in}{0.733947in}}%
\pgfpathlineto{\pgfqpoint{2.683713in}{0.728788in}}%
\pgfpathlineto{\pgfqpoint{2.688836in}{0.728788in}}%
\pgfpathlineto{\pgfqpoint{2.688836in}{0.733947in}}%
\pgfpathlineto{\pgfqpoint{2.689812in}{0.728788in}}%
\pgfpathlineto{\pgfqpoint{2.692007in}{0.728788in}}%
\pgfpathlineto{\pgfqpoint{2.692007in}{0.733947in}}%
\pgfpathlineto{\pgfqpoint{2.692983in}{0.728788in}}%
\pgfpathlineto{\pgfqpoint{2.693959in}{0.728788in}}%
\pgfpathlineto{\pgfqpoint{2.693959in}{0.733947in}}%
\pgfpathlineto{\pgfqpoint{2.694935in}{0.728788in}}%
\pgfpathlineto{\pgfqpoint{2.698107in}{0.728788in}}%
\pgfpathlineto{\pgfqpoint{2.698107in}{0.733947in}}%
\pgfpathlineto{\pgfqpoint{2.699082in}{0.728788in}}%
\pgfpathlineto{\pgfqpoint{2.705669in}{0.728788in}}%
\pgfpathlineto{\pgfqpoint{2.705669in}{0.733947in}}%
\pgfpathlineto{\pgfqpoint{2.706645in}{0.728788in}}%
\pgfpathlineto{\pgfqpoint{2.707377in}{0.728788in}}%
\pgfpathlineto{\pgfqpoint{2.707377in}{0.733947in}}%
\pgfpathlineto{\pgfqpoint{2.708353in}{0.728788in}}%
\pgfpathlineto{\pgfqpoint{2.717136in}{0.728788in}}%
\pgfpathlineto{\pgfqpoint{2.717136in}{0.733947in}}%
\pgfpathlineto{\pgfqpoint{2.718111in}{0.728788in}}%
\pgfpathlineto{\pgfqpoint{2.718599in}{0.728788in}}%
\pgfpathlineto{\pgfqpoint{2.718599in}{0.733947in}}%
\pgfpathlineto{\pgfqpoint{2.719575in}{0.728788in}}%
\pgfpathlineto{\pgfqpoint{2.721039in}{0.728788in}}%
\pgfpathlineto{\pgfqpoint{2.721039in}{0.733947in}}%
\pgfpathlineto{\pgfqpoint{2.722015in}{0.728788in}}%
\pgfpathlineto{\pgfqpoint{2.722747in}{0.728788in}}%
\pgfpathlineto{\pgfqpoint{2.722747in}{0.733947in}}%
\pgfpathlineto{\pgfqpoint{2.723723in}{0.728788in}}%
\pgfpathlineto{\pgfqpoint{2.724211in}{0.728788in}}%
\pgfpathlineto{\pgfqpoint{2.724211in}{0.733947in}}%
\pgfpathlineto{\pgfqpoint{2.725186in}{0.728788in}}%
\pgfpathlineto{\pgfqpoint{2.725918in}{0.728788in}}%
\pgfpathlineto{\pgfqpoint{2.725918in}{0.733947in}}%
\pgfpathlineto{\pgfqpoint{2.726894in}{0.728788in}}%
\pgfpathlineto{\pgfqpoint{2.727138in}{0.728788in}}%
\pgfpathlineto{\pgfqpoint{2.727138in}{0.739107in}}%
\pgfpathlineto{\pgfqpoint{2.728114in}{0.728788in}}%
\pgfpathlineto{\pgfqpoint{2.735921in}{0.728788in}}%
\pgfpathlineto{\pgfqpoint{2.735921in}{0.733947in}}%
\pgfpathlineto{\pgfqpoint{2.736897in}{0.728788in}}%
\pgfpathlineto{\pgfqpoint{2.737628in}{0.728788in}}%
\pgfpathlineto{\pgfqpoint{2.737628in}{0.733947in}}%
\pgfpathlineto{\pgfqpoint{2.738604in}{0.728788in}}%
\pgfpathlineto{\pgfqpoint{2.742752in}{0.728788in}}%
\pgfpathlineto{\pgfqpoint{2.742752in}{0.733947in}}%
\pgfpathlineto{\pgfqpoint{2.743728in}{0.728788in}}%
\pgfpathlineto{\pgfqpoint{2.748851in}{0.728788in}}%
\pgfpathlineto{\pgfqpoint{2.748851in}{0.739107in}}%
\pgfpathlineto{\pgfqpoint{2.749827in}{0.728788in}}%
\pgfpathlineto{\pgfqpoint{2.753730in}{0.728788in}}%
\pgfpathlineto{\pgfqpoint{2.753730in}{0.733947in}}%
\pgfpathlineto{\pgfqpoint{2.754706in}{0.728788in}}%
\pgfpathlineto{\pgfqpoint{2.762025in}{0.728788in}}%
\pgfpathlineto{\pgfqpoint{2.762025in}{0.733947in}}%
\pgfpathlineto{\pgfqpoint{2.763001in}{0.728788in}}%
\pgfpathlineto{\pgfqpoint{2.768856in}{0.728788in}}%
\pgfpathlineto{\pgfqpoint{2.768856in}{0.733947in}}%
\pgfpathlineto{\pgfqpoint{2.769832in}{0.728788in}}%
\pgfpathlineto{\pgfqpoint{2.774467in}{0.728788in}}%
\pgfpathlineto{\pgfqpoint{2.774467in}{0.733947in}}%
\pgfpathlineto{\pgfqpoint{2.775443in}{0.728788in}}%
\pgfpathlineto{\pgfqpoint{2.776906in}{0.728788in}}%
\pgfpathlineto{\pgfqpoint{2.776906in}{0.733947in}}%
\pgfpathlineto{\pgfqpoint{2.777882in}{0.728788in}}%
\pgfpathlineto{\pgfqpoint{2.784469in}{0.728788in}}%
\pgfpathlineto{\pgfqpoint{2.784469in}{0.733947in}}%
\pgfpathlineto{\pgfqpoint{2.785445in}{0.728788in}}%
\pgfpathlineto{\pgfqpoint{2.788373in}{0.728788in}}%
\pgfpathlineto{\pgfqpoint{2.788373in}{0.733947in}}%
\pgfpathlineto{\pgfqpoint{2.789349in}{0.728788in}}%
\pgfpathlineto{\pgfqpoint{2.789593in}{0.728788in}}%
\pgfpathlineto{\pgfqpoint{2.789593in}{0.733947in}}%
\pgfpathlineto{\pgfqpoint{2.790568in}{0.728788in}}%
\pgfpathlineto{\pgfqpoint{2.792032in}{0.728788in}}%
\pgfpathlineto{\pgfqpoint{2.792032in}{0.733947in}}%
\pgfpathlineto{\pgfqpoint{2.793008in}{0.728788in}}%
\pgfpathlineto{\pgfqpoint{2.795692in}{0.728788in}}%
\pgfpathlineto{\pgfqpoint{2.795692in}{0.733947in}}%
\pgfpathlineto{\pgfqpoint{2.796667in}{0.728788in}}%
\pgfpathlineto{\pgfqpoint{2.811549in}{0.728788in}}%
\pgfpathlineto{\pgfqpoint{2.811549in}{0.733947in}}%
\pgfpathlineto{\pgfqpoint{2.812525in}{0.728788in}}%
\pgfpathlineto{\pgfqpoint{2.813501in}{0.728788in}}%
\pgfpathlineto{\pgfqpoint{2.813501in}{0.733947in}}%
\pgfpathlineto{\pgfqpoint{2.814477in}{0.728788in}}%
\pgfpathlineto{\pgfqpoint{2.819600in}{0.728788in}}%
\pgfpathlineto{\pgfqpoint{2.819600in}{0.733947in}}%
\pgfpathlineto{\pgfqpoint{2.820576in}{0.728788in}}%
\pgfpathlineto{\pgfqpoint{2.821796in}{0.728788in}}%
\pgfpathlineto{\pgfqpoint{2.821796in}{0.733947in}}%
\pgfpathlineto{\pgfqpoint{2.822771in}{0.728788in}}%
\pgfpathlineto{\pgfqpoint{2.823747in}{0.728788in}}%
\pgfpathlineto{\pgfqpoint{2.823747in}{0.733947in}}%
\pgfpathlineto{\pgfqpoint{2.824723in}{0.728788in}}%
\pgfpathlineto{\pgfqpoint{2.831310in}{0.728788in}}%
\pgfpathlineto{\pgfqpoint{2.831310in}{0.733947in}}%
\pgfpathlineto{\pgfqpoint{2.832286in}{0.728788in}}%
\pgfpathlineto{\pgfqpoint{2.832774in}{0.728788in}}%
\pgfpathlineto{\pgfqpoint{2.832774in}{0.733947in}}%
\pgfpathlineto{\pgfqpoint{2.833750in}{0.728788in}}%
\pgfpathlineto{\pgfqpoint{2.838629in}{0.728788in}}%
\pgfpathlineto{\pgfqpoint{2.838629in}{0.733947in}}%
\pgfpathlineto{\pgfqpoint{2.839605in}{0.728788in}}%
\pgfpathlineto{\pgfqpoint{2.840825in}{0.728788in}}%
\pgfpathlineto{\pgfqpoint{2.840825in}{0.733947in}}%
\pgfpathlineto{\pgfqpoint{2.841801in}{0.728788in}}%
\pgfpathlineto{\pgfqpoint{2.854731in}{0.728788in}}%
\pgfpathlineto{\pgfqpoint{2.854731in}{0.733947in}}%
\pgfpathlineto{\pgfqpoint{2.855706in}{0.728788in}}%
\pgfpathlineto{\pgfqpoint{2.859366in}{0.728788in}}%
\pgfpathlineto{\pgfqpoint{2.859366in}{0.733947in}}%
\pgfpathlineto{\pgfqpoint{2.860342in}{0.728788in}}%
\pgfpathlineto{\pgfqpoint{2.863025in}{0.728788in}}%
\pgfpathlineto{\pgfqpoint{2.863025in}{0.733947in}}%
\pgfpathlineto{\pgfqpoint{2.864001in}{0.728788in}}%
\pgfpathlineto{\pgfqpoint{2.877175in}{0.728788in}}%
\pgfpathlineto{\pgfqpoint{2.877175in}{0.733947in}}%
\pgfpathlineto{\pgfqpoint{2.878151in}{0.728788in}}%
\pgfpathlineto{\pgfqpoint{2.883762in}{0.728788in}}%
\pgfpathlineto{\pgfqpoint{2.883762in}{0.733947in}}%
\pgfpathlineto{\pgfqpoint{2.884738in}{0.728788in}}%
\pgfpathlineto{\pgfqpoint{2.888641in}{0.728788in}}%
\pgfpathlineto{\pgfqpoint{2.888641in}{0.733947in}}%
\pgfpathlineto{\pgfqpoint{2.889617in}{0.728788in}}%
\pgfpathlineto{\pgfqpoint{2.906695in}{0.728788in}}%
\pgfpathlineto{\pgfqpoint{2.906695in}{0.733947in}}%
\pgfpathlineto{\pgfqpoint{2.907670in}{0.728788in}}%
\pgfpathlineto{\pgfqpoint{2.908646in}{0.728788in}}%
\pgfpathlineto{\pgfqpoint{2.908646in}{0.733947in}}%
\pgfpathlineto{\pgfqpoint{2.909622in}{0.728788in}}%
\pgfpathlineto{\pgfqpoint{2.916697in}{0.728788in}}%
\pgfpathlineto{\pgfqpoint{2.916697in}{0.733947in}}%
\pgfpathlineto{\pgfqpoint{2.917673in}{0.728788in}}%
\pgfpathlineto{\pgfqpoint{2.918893in}{0.728788in}}%
\pgfpathlineto{\pgfqpoint{2.918893in}{0.733947in}}%
\pgfpathlineto{\pgfqpoint{2.919869in}{0.728788in}}%
\pgfpathlineto{\pgfqpoint{2.927919in}{0.728788in}}%
\pgfpathlineto{\pgfqpoint{2.927919in}{0.733947in}}%
\pgfpathlineto{\pgfqpoint{2.928895in}{0.728788in}}%
\pgfpathlineto{\pgfqpoint{2.930359in}{0.728788in}}%
\pgfpathlineto{\pgfqpoint{2.930359in}{0.733947in}}%
\pgfpathlineto{\pgfqpoint{2.931335in}{0.728788in}}%
\pgfpathlineto{\pgfqpoint{2.936946in}{0.728788in}}%
\pgfpathlineto{\pgfqpoint{2.936946in}{0.733947in}}%
\pgfpathlineto{\pgfqpoint{2.937922in}{0.728788in}}%
\pgfpathlineto{\pgfqpoint{2.939630in}{0.728788in}}%
\pgfpathlineto{\pgfqpoint{2.939630in}{0.733947in}}%
\pgfpathlineto{\pgfqpoint{2.940605in}{0.728788in}}%
\pgfpathlineto{\pgfqpoint{2.948168in}{0.728788in}}%
\pgfpathlineto{\pgfqpoint{2.948168in}{0.733947in}}%
\pgfpathlineto{\pgfqpoint{2.949144in}{0.728788in}}%
\pgfpathlineto{\pgfqpoint{2.949388in}{0.728788in}}%
\pgfpathlineto{\pgfqpoint{2.949388in}{0.733947in}}%
\pgfpathlineto{\pgfqpoint{2.950364in}{0.728788in}}%
\pgfpathlineto{\pgfqpoint{2.956707in}{0.728788in}}%
\pgfpathlineto{\pgfqpoint{2.956707in}{0.733947in}}%
\pgfpathlineto{\pgfqpoint{2.957683in}{0.728788in}}%
\pgfpathlineto{\pgfqpoint{2.962562in}{0.728788in}}%
\pgfpathlineto{\pgfqpoint{2.962562in}{0.733947in}}%
\pgfpathlineto{\pgfqpoint{2.963538in}{0.728788in}}%
\pgfpathlineto{\pgfqpoint{2.973052in}{0.728788in}}%
\pgfpathlineto{\pgfqpoint{2.973052in}{0.733947in}}%
\pgfpathlineto{\pgfqpoint{2.974028in}{0.728788in}}%
\pgfpathlineto{\pgfqpoint{2.977444in}{0.728788in}}%
\pgfpathlineto{\pgfqpoint{2.977444in}{0.733947in}}%
\pgfpathlineto{\pgfqpoint{2.978420in}{0.728788in}}%
\pgfpathlineto{\pgfqpoint{2.997693in}{0.728788in}}%
\pgfpathlineto{\pgfqpoint{2.997693in}{0.733947in}}%
\pgfpathlineto{\pgfqpoint{2.998669in}{0.728788in}}%
\pgfpathlineto{\pgfqpoint{3.000132in}{0.728788in}}%
\pgfpathlineto{\pgfqpoint{3.000132in}{0.733947in}}%
\pgfpathlineto{\pgfqpoint{3.001108in}{0.728788in}}%
\pgfpathlineto{\pgfqpoint{3.004524in}{0.728788in}}%
\pgfpathlineto{\pgfqpoint{3.004524in}{0.733947in}}%
\pgfpathlineto{\pgfqpoint{3.005500in}{0.728788in}}%
\pgfpathlineto{\pgfqpoint{3.013062in}{0.728788in}}%
\pgfpathlineto{\pgfqpoint{3.013062in}{0.733947in}}%
\pgfpathlineto{\pgfqpoint{3.014038in}{0.728788in}}%
\pgfpathlineto{\pgfqpoint{3.021601in}{0.728788in}}%
\pgfpathlineto{\pgfqpoint{3.021601in}{0.733947in}}%
\pgfpathlineto{\pgfqpoint{3.022577in}{0.728788in}}%
\pgfpathlineto{\pgfqpoint{3.045753in}{0.728788in}}%
\pgfpathlineto{\pgfqpoint{3.045753in}{0.733947in}}%
\pgfpathlineto{\pgfqpoint{3.046729in}{0.728788in}}%
\pgfpathlineto{\pgfqpoint{3.047949in}{0.728788in}}%
\pgfpathlineto{\pgfqpoint{3.047949in}{0.733947in}}%
\pgfpathlineto{\pgfqpoint{3.048925in}{0.728788in}}%
\pgfpathlineto{\pgfqpoint{3.064051in}{0.728788in}}%
\pgfpathlineto{\pgfqpoint{3.064051in}{0.733947in}}%
\pgfpathlineto{\pgfqpoint{3.065026in}{0.728788in}}%
\pgfpathlineto{\pgfqpoint{3.093082in}{0.728788in}}%
\pgfpathlineto{\pgfqpoint{3.093082in}{0.733947in}}%
\pgfpathlineto{\pgfqpoint{3.094058in}{0.728788in}}%
\pgfpathlineto{\pgfqpoint{3.119674in}{0.728788in}}%
\pgfpathlineto{\pgfqpoint{3.119674in}{0.733947in}}%
\pgfpathlineto{\pgfqpoint{3.120650in}{0.728788in}}%
\pgfpathlineto{\pgfqpoint{3.169930in}{0.728788in}}%
\pgfpathlineto{\pgfqpoint{3.169930in}{0.733947in}}%
\pgfpathlineto{\pgfqpoint{3.170906in}{0.728788in}}%
\pgfpathlineto{\pgfqpoint{3.291424in}{0.728788in}}%
\pgfpathlineto{\pgfqpoint{3.291424in}{0.733947in}}%
\pgfpathlineto{\pgfqpoint{3.292400in}{0.728788in}}%
\pgfpathlineto{\pgfqpoint{4.617361in}{0.728788in}}%
\pgfpathlineto{\pgfqpoint{4.617361in}{0.728788in}}%
\pgfusepath{stroke}%
\end{pgfscope}%
\begin{pgfscope}%
\pgfsetrectcap%
\pgfsetmiterjoin%
\pgfsetlinewidth{0.803000pt}%
\definecolor{currentstroke}{rgb}{0.000000,0.000000,0.000000}%
\pgfsetstrokecolor{currentstroke}%
\pgfsetdash{}{0pt}%
\pgfpathmoveto{\pgfqpoint{0.752778in}{0.582778in}}%
\pgfpathlineto{\pgfqpoint{0.752778in}{3.795000in}}%
\pgfusepath{stroke}%
\end{pgfscope}%
\begin{pgfscope}%
\pgfsetrectcap%
\pgfsetmiterjoin%
\pgfsetlinewidth{0.803000pt}%
\definecolor{currentstroke}{rgb}{0.000000,0.000000,0.000000}%
\pgfsetstrokecolor{currentstroke}%
\pgfsetdash{}{0pt}%
\pgfpathmoveto{\pgfqpoint{4.801389in}{0.582778in}}%
\pgfpathlineto{\pgfqpoint{4.801389in}{3.795000in}}%
\pgfusepath{stroke}%
\end{pgfscope}%
\begin{pgfscope}%
\pgfsetrectcap%
\pgfsetmiterjoin%
\pgfsetlinewidth{0.803000pt}%
\definecolor{currentstroke}{rgb}{0.000000,0.000000,0.000000}%
\pgfsetstrokecolor{currentstroke}%
\pgfsetdash{}{0pt}%
\pgfpathmoveto{\pgfqpoint{0.752778in}{0.582778in}}%
\pgfpathlineto{\pgfqpoint{4.801389in}{0.582778in}}%
\pgfusepath{stroke}%
\end{pgfscope}%
\begin{pgfscope}%
\pgfsetrectcap%
\pgfsetmiterjoin%
\pgfsetlinewidth{0.803000pt}%
\definecolor{currentstroke}{rgb}{0.000000,0.000000,0.000000}%
\pgfsetstrokecolor{currentstroke}%
\pgfsetdash{}{0pt}%
\pgfpathmoveto{\pgfqpoint{0.752778in}{3.795000in}}%
\pgfpathlineto{\pgfqpoint{4.801389in}{3.795000in}}%
\pgfusepath{stroke}%
\end{pgfscope}%
\end{pgfpicture}%
\makeatother%
\endgroup%

  \caption{Z\"ahlraten der beiden detektoren in Abhängigkeit der Energie.}
  \label{fig:calibration-mid_over_energy}
\end{figure}

Die gew\"ahlten Energieintervalle sind
in~\ref{fig:calibration-mid_over_energy} eingezeichenet und richten
sich nach der Halbwertsbreite der prominentesten Peaks. Da in
Detektor A ein PM ausgetauscht wurde zeigt die Energiekurve andere
Charakteristiken als die von B. Interessanter weise liegen die beiden
h\"ochsten Peaks nicht aufeinander (Kalibrierungsproblem/Skalierung).

\begin{align}
  R_A &= [2800, 3600] \hat{=} [582.2, 725.4]\,\si{\kilo\electronvolt}
  \\
  R_B &= [1600, 2300] \hat{=} [452.0, 606.0]\,\si{\kilo\electronvolt}
\end{align}

Bei der Bestimmung des Zeitintervals wurde analog durch auftragen der
Ereignisszahl \"uber die Kan\"ale vorgegangen
(\ref{fig:calibration-time_range}). Dabei wurde die Intervalbreite
etwas gr\"o\ss{}er als die die Halbwertsbreite gew\"ahlt um eine gute
Z\"ahlrate zu gew\"ahrleisten.

\begin{figure}[h]\centering
  %% Creator: Matplotlib, PGF backend
%%
%% To include the figure in your LaTeX document, write
%%   \input{<filename>.pgf}
%%
%% Make sure the required packages are loaded in your preamble
%%   \usepackage{pgf}
%%
%% Figures using additional raster images can only be included by \input if
%% they are in the same directory as the main LaTeX file. For loading figures
%% from other directories you can use the `import` package
%%   \usepackage{import}
%% and then include the figures with
%%   \import{<path to file>}{<filename>.pgf}
%%
%% Matplotlib used the following preamble
%%   \usepackage{fontspec}
%%
\begingroup%
\makeatletter%
\begin{pgfpicture}%
\pgfpathrectangle{\pgfpointorigin}{\pgfqpoint{5.000000in}{4.000000in}}%
\pgfusepath{use as bounding box, clip}%
\begin{pgfscope}%
\pgfsetbuttcap%
\pgfsetmiterjoin%
\definecolor{currentfill}{rgb}{1.000000,1.000000,1.000000}%
\pgfsetfillcolor{currentfill}%
\pgfsetlinewidth{0.000000pt}%
\definecolor{currentstroke}{rgb}{1.000000,1.000000,1.000000}%
\pgfsetstrokecolor{currentstroke}%
\pgfsetdash{}{0pt}%
\pgfpathmoveto{\pgfqpoint{0.000000in}{0.000000in}}%
\pgfpathlineto{\pgfqpoint{5.000000in}{0.000000in}}%
\pgfpathlineto{\pgfqpoint{5.000000in}{4.000000in}}%
\pgfpathlineto{\pgfqpoint{0.000000in}{4.000000in}}%
\pgfpathclose%
\pgfusepath{fill}%
\end{pgfscope}%
\begin{pgfscope}%
\pgfsetbuttcap%
\pgfsetmiterjoin%
\definecolor{currentfill}{rgb}{1.000000,1.000000,1.000000}%
\pgfsetfillcolor{currentfill}%
\pgfsetlinewidth{0.000000pt}%
\definecolor{currentstroke}{rgb}{0.000000,0.000000,0.000000}%
\pgfsetstrokecolor{currentstroke}%
\pgfsetstrokeopacity{0.000000}%
\pgfsetdash{}{0pt}%
\pgfpathmoveto{\pgfqpoint{0.781944in}{0.552778in}}%
\pgfpathlineto{\pgfqpoint{4.672917in}{0.552778in}}%
\pgfpathlineto{\pgfqpoint{4.672917in}{3.801389in}}%
\pgfpathlineto{\pgfqpoint{0.781944in}{3.801389in}}%
\pgfpathclose%
\pgfusepath{fill}%
\end{pgfscope}%
\begin{pgfscope}%
\pgfpathrectangle{\pgfqpoint{0.781944in}{0.552778in}}{\pgfqpoint{3.890972in}{3.248611in}}%
\pgfusepath{clip}%
\pgfsetrectcap%
\pgfsetroundjoin%
\pgfsetlinewidth{0.803000pt}%
\definecolor{currentstroke}{rgb}{0.690196,0.690196,0.690196}%
\pgfsetstrokecolor{currentstroke}%
\pgfsetstrokeopacity{0.800000}%
\pgfsetdash{}{0pt}%
\pgfpathmoveto{\pgfqpoint{1.169874in}{0.552778in}}%
\pgfpathlineto{\pgfqpoint{1.169874in}{3.801389in}}%
\pgfusepath{stroke}%
\end{pgfscope}%
\begin{pgfscope}%
\pgfsetbuttcap%
\pgfsetroundjoin%
\definecolor{currentfill}{rgb}{0.000000,0.000000,0.000000}%
\pgfsetfillcolor{currentfill}%
\pgfsetlinewidth{0.803000pt}%
\definecolor{currentstroke}{rgb}{0.000000,0.000000,0.000000}%
\pgfsetstrokecolor{currentstroke}%
\pgfsetdash{}{0pt}%
\pgfsys@defobject{currentmarker}{\pgfqpoint{0.000000in}{-0.048611in}}{\pgfqpoint{0.000000in}{0.000000in}}{%
\pgfpathmoveto{\pgfqpoint{0.000000in}{0.000000in}}%
\pgfpathlineto{\pgfqpoint{0.000000in}{-0.048611in}}%
\pgfusepath{stroke,fill}%
}%
\begin{pgfscope}%
\pgfsys@transformshift{1.169874in}{0.552778in}%
\pgfsys@useobject{currentmarker}{}%
\end{pgfscope}%
\end{pgfscope}%
\begin{pgfscope}%
\pgfsetbuttcap%
\pgfsetroundjoin%
\definecolor{currentfill}{rgb}{0.000000,0.000000,0.000000}%
\pgfsetfillcolor{currentfill}%
\pgfsetlinewidth{0.803000pt}%
\definecolor{currentstroke}{rgb}{0.000000,0.000000,0.000000}%
\pgfsetstrokecolor{currentstroke}%
\pgfsetdash{}{0pt}%
\pgfsys@defobject{currentmarker}{\pgfqpoint{0.000000in}{0.000000in}}{\pgfqpoint{0.000000in}{0.048611in}}{%
\pgfpathmoveto{\pgfqpoint{0.000000in}{0.000000in}}%
\pgfpathlineto{\pgfqpoint{0.000000in}{0.048611in}}%
\pgfusepath{stroke,fill}%
}%
\begin{pgfscope}%
\pgfsys@transformshift{1.169874in}{3.801389in}%
\pgfsys@useobject{currentmarker}{}%
\end{pgfscope}%
\end{pgfscope}%
\begin{pgfscope}%
\definecolor{textcolor}{rgb}{0.000000,0.000000,0.000000}%
\pgfsetstrokecolor{textcolor}%
\pgfsetfillcolor{textcolor}%
\pgftext[x=1.169874in,y=0.455556in,,top]{\color{textcolor}\rmfamily\fontsize{10.000000}{12.000000}\selectfont 300}%
\end{pgfscope}%
\begin{pgfscope}%
\pgfpathrectangle{\pgfqpoint{0.781944in}{0.552778in}}{\pgfqpoint{3.890972in}{3.248611in}}%
\pgfusepath{clip}%
\pgfsetrectcap%
\pgfsetroundjoin%
\pgfsetlinewidth{0.803000pt}%
\definecolor{currentstroke}{rgb}{0.690196,0.690196,0.690196}%
\pgfsetstrokecolor{currentstroke}%
\pgfsetstrokeopacity{0.800000}%
\pgfsetdash{}{0pt}%
\pgfpathmoveto{\pgfqpoint{1.559101in}{0.552778in}}%
\pgfpathlineto{\pgfqpoint{1.559101in}{3.801389in}}%
\pgfusepath{stroke}%
\end{pgfscope}%
\begin{pgfscope}%
\pgfsetbuttcap%
\pgfsetroundjoin%
\definecolor{currentfill}{rgb}{0.000000,0.000000,0.000000}%
\pgfsetfillcolor{currentfill}%
\pgfsetlinewidth{0.803000pt}%
\definecolor{currentstroke}{rgb}{0.000000,0.000000,0.000000}%
\pgfsetstrokecolor{currentstroke}%
\pgfsetdash{}{0pt}%
\pgfsys@defobject{currentmarker}{\pgfqpoint{0.000000in}{-0.048611in}}{\pgfqpoint{0.000000in}{0.000000in}}{%
\pgfpathmoveto{\pgfqpoint{0.000000in}{0.000000in}}%
\pgfpathlineto{\pgfqpoint{0.000000in}{-0.048611in}}%
\pgfusepath{stroke,fill}%
}%
\begin{pgfscope}%
\pgfsys@transformshift{1.559101in}{0.552778in}%
\pgfsys@useobject{currentmarker}{}%
\end{pgfscope}%
\end{pgfscope}%
\begin{pgfscope}%
\pgfsetbuttcap%
\pgfsetroundjoin%
\definecolor{currentfill}{rgb}{0.000000,0.000000,0.000000}%
\pgfsetfillcolor{currentfill}%
\pgfsetlinewidth{0.803000pt}%
\definecolor{currentstroke}{rgb}{0.000000,0.000000,0.000000}%
\pgfsetstrokecolor{currentstroke}%
\pgfsetdash{}{0pt}%
\pgfsys@defobject{currentmarker}{\pgfqpoint{0.000000in}{0.000000in}}{\pgfqpoint{0.000000in}{0.048611in}}{%
\pgfpathmoveto{\pgfqpoint{0.000000in}{0.000000in}}%
\pgfpathlineto{\pgfqpoint{0.000000in}{0.048611in}}%
\pgfusepath{stroke,fill}%
}%
\begin{pgfscope}%
\pgfsys@transformshift{1.559101in}{3.801389in}%
\pgfsys@useobject{currentmarker}{}%
\end{pgfscope}%
\end{pgfscope}%
\begin{pgfscope}%
\definecolor{textcolor}{rgb}{0.000000,0.000000,0.000000}%
\pgfsetstrokecolor{textcolor}%
\pgfsetfillcolor{textcolor}%
\pgftext[x=1.559101in,y=0.455556in,,top]{\color{textcolor}\rmfamily\fontsize{10.000000}{12.000000}\selectfont 600}%
\end{pgfscope}%
\begin{pgfscope}%
\pgfpathrectangle{\pgfqpoint{0.781944in}{0.552778in}}{\pgfqpoint{3.890972in}{3.248611in}}%
\pgfusepath{clip}%
\pgfsetrectcap%
\pgfsetroundjoin%
\pgfsetlinewidth{0.803000pt}%
\definecolor{currentstroke}{rgb}{0.690196,0.690196,0.690196}%
\pgfsetstrokecolor{currentstroke}%
\pgfsetstrokeopacity{0.800000}%
\pgfsetdash{}{0pt}%
\pgfpathmoveto{\pgfqpoint{1.948328in}{0.552778in}}%
\pgfpathlineto{\pgfqpoint{1.948328in}{3.801389in}}%
\pgfusepath{stroke}%
\end{pgfscope}%
\begin{pgfscope}%
\pgfsetbuttcap%
\pgfsetroundjoin%
\definecolor{currentfill}{rgb}{0.000000,0.000000,0.000000}%
\pgfsetfillcolor{currentfill}%
\pgfsetlinewidth{0.803000pt}%
\definecolor{currentstroke}{rgb}{0.000000,0.000000,0.000000}%
\pgfsetstrokecolor{currentstroke}%
\pgfsetdash{}{0pt}%
\pgfsys@defobject{currentmarker}{\pgfqpoint{0.000000in}{-0.048611in}}{\pgfqpoint{0.000000in}{0.000000in}}{%
\pgfpathmoveto{\pgfqpoint{0.000000in}{0.000000in}}%
\pgfpathlineto{\pgfqpoint{0.000000in}{-0.048611in}}%
\pgfusepath{stroke,fill}%
}%
\begin{pgfscope}%
\pgfsys@transformshift{1.948328in}{0.552778in}%
\pgfsys@useobject{currentmarker}{}%
\end{pgfscope}%
\end{pgfscope}%
\begin{pgfscope}%
\pgfsetbuttcap%
\pgfsetroundjoin%
\definecolor{currentfill}{rgb}{0.000000,0.000000,0.000000}%
\pgfsetfillcolor{currentfill}%
\pgfsetlinewidth{0.803000pt}%
\definecolor{currentstroke}{rgb}{0.000000,0.000000,0.000000}%
\pgfsetstrokecolor{currentstroke}%
\pgfsetdash{}{0pt}%
\pgfsys@defobject{currentmarker}{\pgfqpoint{0.000000in}{0.000000in}}{\pgfqpoint{0.000000in}{0.048611in}}{%
\pgfpathmoveto{\pgfqpoint{0.000000in}{0.000000in}}%
\pgfpathlineto{\pgfqpoint{0.000000in}{0.048611in}}%
\pgfusepath{stroke,fill}%
}%
\begin{pgfscope}%
\pgfsys@transformshift{1.948328in}{3.801389in}%
\pgfsys@useobject{currentmarker}{}%
\end{pgfscope}%
\end{pgfscope}%
\begin{pgfscope}%
\definecolor{textcolor}{rgb}{0.000000,0.000000,0.000000}%
\pgfsetstrokecolor{textcolor}%
\pgfsetfillcolor{textcolor}%
\pgftext[x=1.948328in,y=0.455556in,,top]{\color{textcolor}\rmfamily\fontsize{10.000000}{12.000000}\selectfont 900}%
\end{pgfscope}%
\begin{pgfscope}%
\pgfpathrectangle{\pgfqpoint{0.781944in}{0.552778in}}{\pgfqpoint{3.890972in}{3.248611in}}%
\pgfusepath{clip}%
\pgfsetrectcap%
\pgfsetroundjoin%
\pgfsetlinewidth{0.803000pt}%
\definecolor{currentstroke}{rgb}{0.690196,0.690196,0.690196}%
\pgfsetstrokecolor{currentstroke}%
\pgfsetstrokeopacity{0.800000}%
\pgfsetdash{}{0pt}%
\pgfpathmoveto{\pgfqpoint{2.337555in}{0.552778in}}%
\pgfpathlineto{\pgfqpoint{2.337555in}{3.801389in}}%
\pgfusepath{stroke}%
\end{pgfscope}%
\begin{pgfscope}%
\pgfsetbuttcap%
\pgfsetroundjoin%
\definecolor{currentfill}{rgb}{0.000000,0.000000,0.000000}%
\pgfsetfillcolor{currentfill}%
\pgfsetlinewidth{0.803000pt}%
\definecolor{currentstroke}{rgb}{0.000000,0.000000,0.000000}%
\pgfsetstrokecolor{currentstroke}%
\pgfsetdash{}{0pt}%
\pgfsys@defobject{currentmarker}{\pgfqpoint{0.000000in}{-0.048611in}}{\pgfqpoint{0.000000in}{0.000000in}}{%
\pgfpathmoveto{\pgfqpoint{0.000000in}{0.000000in}}%
\pgfpathlineto{\pgfqpoint{0.000000in}{-0.048611in}}%
\pgfusepath{stroke,fill}%
}%
\begin{pgfscope}%
\pgfsys@transformshift{2.337555in}{0.552778in}%
\pgfsys@useobject{currentmarker}{}%
\end{pgfscope}%
\end{pgfscope}%
\begin{pgfscope}%
\pgfsetbuttcap%
\pgfsetroundjoin%
\definecolor{currentfill}{rgb}{0.000000,0.000000,0.000000}%
\pgfsetfillcolor{currentfill}%
\pgfsetlinewidth{0.803000pt}%
\definecolor{currentstroke}{rgb}{0.000000,0.000000,0.000000}%
\pgfsetstrokecolor{currentstroke}%
\pgfsetdash{}{0pt}%
\pgfsys@defobject{currentmarker}{\pgfqpoint{0.000000in}{0.000000in}}{\pgfqpoint{0.000000in}{0.048611in}}{%
\pgfpathmoveto{\pgfqpoint{0.000000in}{0.000000in}}%
\pgfpathlineto{\pgfqpoint{0.000000in}{0.048611in}}%
\pgfusepath{stroke,fill}%
}%
\begin{pgfscope}%
\pgfsys@transformshift{2.337555in}{3.801389in}%
\pgfsys@useobject{currentmarker}{}%
\end{pgfscope}%
\end{pgfscope}%
\begin{pgfscope}%
\definecolor{textcolor}{rgb}{0.000000,0.000000,0.000000}%
\pgfsetstrokecolor{textcolor}%
\pgfsetfillcolor{textcolor}%
\pgftext[x=2.337555in,y=0.455556in,,top]{\color{textcolor}\rmfamily\fontsize{10.000000}{12.000000}\selectfont 1200}%
\end{pgfscope}%
\begin{pgfscope}%
\pgfpathrectangle{\pgfqpoint{0.781944in}{0.552778in}}{\pgfqpoint{3.890972in}{3.248611in}}%
\pgfusepath{clip}%
\pgfsetrectcap%
\pgfsetroundjoin%
\pgfsetlinewidth{0.803000pt}%
\definecolor{currentstroke}{rgb}{0.690196,0.690196,0.690196}%
\pgfsetstrokecolor{currentstroke}%
\pgfsetstrokeopacity{0.800000}%
\pgfsetdash{}{0pt}%
\pgfpathmoveto{\pgfqpoint{2.726782in}{0.552778in}}%
\pgfpathlineto{\pgfqpoint{2.726782in}{3.801389in}}%
\pgfusepath{stroke}%
\end{pgfscope}%
\begin{pgfscope}%
\pgfsetbuttcap%
\pgfsetroundjoin%
\definecolor{currentfill}{rgb}{0.000000,0.000000,0.000000}%
\pgfsetfillcolor{currentfill}%
\pgfsetlinewidth{0.803000pt}%
\definecolor{currentstroke}{rgb}{0.000000,0.000000,0.000000}%
\pgfsetstrokecolor{currentstroke}%
\pgfsetdash{}{0pt}%
\pgfsys@defobject{currentmarker}{\pgfqpoint{0.000000in}{-0.048611in}}{\pgfqpoint{0.000000in}{0.000000in}}{%
\pgfpathmoveto{\pgfqpoint{0.000000in}{0.000000in}}%
\pgfpathlineto{\pgfqpoint{0.000000in}{-0.048611in}}%
\pgfusepath{stroke,fill}%
}%
\begin{pgfscope}%
\pgfsys@transformshift{2.726782in}{0.552778in}%
\pgfsys@useobject{currentmarker}{}%
\end{pgfscope}%
\end{pgfscope}%
\begin{pgfscope}%
\pgfsetbuttcap%
\pgfsetroundjoin%
\definecolor{currentfill}{rgb}{0.000000,0.000000,0.000000}%
\pgfsetfillcolor{currentfill}%
\pgfsetlinewidth{0.803000pt}%
\definecolor{currentstroke}{rgb}{0.000000,0.000000,0.000000}%
\pgfsetstrokecolor{currentstroke}%
\pgfsetdash{}{0pt}%
\pgfsys@defobject{currentmarker}{\pgfqpoint{0.000000in}{0.000000in}}{\pgfqpoint{0.000000in}{0.048611in}}{%
\pgfpathmoveto{\pgfqpoint{0.000000in}{0.000000in}}%
\pgfpathlineto{\pgfqpoint{0.000000in}{0.048611in}}%
\pgfusepath{stroke,fill}%
}%
\begin{pgfscope}%
\pgfsys@transformshift{2.726782in}{3.801389in}%
\pgfsys@useobject{currentmarker}{}%
\end{pgfscope}%
\end{pgfscope}%
\begin{pgfscope}%
\definecolor{textcolor}{rgb}{0.000000,0.000000,0.000000}%
\pgfsetstrokecolor{textcolor}%
\pgfsetfillcolor{textcolor}%
\pgftext[x=2.726782in,y=0.455556in,,top]{\color{textcolor}\rmfamily\fontsize{10.000000}{12.000000}\selectfont 1500}%
\end{pgfscope}%
\begin{pgfscope}%
\pgfpathrectangle{\pgfqpoint{0.781944in}{0.552778in}}{\pgfqpoint{3.890972in}{3.248611in}}%
\pgfusepath{clip}%
\pgfsetrectcap%
\pgfsetroundjoin%
\pgfsetlinewidth{0.803000pt}%
\definecolor{currentstroke}{rgb}{0.690196,0.690196,0.690196}%
\pgfsetstrokecolor{currentstroke}%
\pgfsetstrokeopacity{0.800000}%
\pgfsetdash{}{0pt}%
\pgfpathmoveto{\pgfqpoint{3.116009in}{0.552778in}}%
\pgfpathlineto{\pgfqpoint{3.116009in}{3.801389in}}%
\pgfusepath{stroke}%
\end{pgfscope}%
\begin{pgfscope}%
\pgfsetbuttcap%
\pgfsetroundjoin%
\definecolor{currentfill}{rgb}{0.000000,0.000000,0.000000}%
\pgfsetfillcolor{currentfill}%
\pgfsetlinewidth{0.803000pt}%
\definecolor{currentstroke}{rgb}{0.000000,0.000000,0.000000}%
\pgfsetstrokecolor{currentstroke}%
\pgfsetdash{}{0pt}%
\pgfsys@defobject{currentmarker}{\pgfqpoint{0.000000in}{-0.048611in}}{\pgfqpoint{0.000000in}{0.000000in}}{%
\pgfpathmoveto{\pgfqpoint{0.000000in}{0.000000in}}%
\pgfpathlineto{\pgfqpoint{0.000000in}{-0.048611in}}%
\pgfusepath{stroke,fill}%
}%
\begin{pgfscope}%
\pgfsys@transformshift{3.116009in}{0.552778in}%
\pgfsys@useobject{currentmarker}{}%
\end{pgfscope}%
\end{pgfscope}%
\begin{pgfscope}%
\pgfsetbuttcap%
\pgfsetroundjoin%
\definecolor{currentfill}{rgb}{0.000000,0.000000,0.000000}%
\pgfsetfillcolor{currentfill}%
\pgfsetlinewidth{0.803000pt}%
\definecolor{currentstroke}{rgb}{0.000000,0.000000,0.000000}%
\pgfsetstrokecolor{currentstroke}%
\pgfsetdash{}{0pt}%
\pgfsys@defobject{currentmarker}{\pgfqpoint{0.000000in}{0.000000in}}{\pgfqpoint{0.000000in}{0.048611in}}{%
\pgfpathmoveto{\pgfqpoint{0.000000in}{0.000000in}}%
\pgfpathlineto{\pgfqpoint{0.000000in}{0.048611in}}%
\pgfusepath{stroke,fill}%
}%
\begin{pgfscope}%
\pgfsys@transformshift{3.116009in}{3.801389in}%
\pgfsys@useobject{currentmarker}{}%
\end{pgfscope}%
\end{pgfscope}%
\begin{pgfscope}%
\definecolor{textcolor}{rgb}{0.000000,0.000000,0.000000}%
\pgfsetstrokecolor{textcolor}%
\pgfsetfillcolor{textcolor}%
\pgftext[x=3.116009in,y=0.455556in,,top]{\color{textcolor}\rmfamily\fontsize{10.000000}{12.000000}\selectfont 1800}%
\end{pgfscope}%
\begin{pgfscope}%
\pgfpathrectangle{\pgfqpoint{0.781944in}{0.552778in}}{\pgfqpoint{3.890972in}{3.248611in}}%
\pgfusepath{clip}%
\pgfsetrectcap%
\pgfsetroundjoin%
\pgfsetlinewidth{0.803000pt}%
\definecolor{currentstroke}{rgb}{0.690196,0.690196,0.690196}%
\pgfsetstrokecolor{currentstroke}%
\pgfsetstrokeopacity{0.800000}%
\pgfsetdash{}{0pt}%
\pgfpathmoveto{\pgfqpoint{3.505236in}{0.552778in}}%
\pgfpathlineto{\pgfqpoint{3.505236in}{3.801389in}}%
\pgfusepath{stroke}%
\end{pgfscope}%
\begin{pgfscope}%
\pgfsetbuttcap%
\pgfsetroundjoin%
\definecolor{currentfill}{rgb}{0.000000,0.000000,0.000000}%
\pgfsetfillcolor{currentfill}%
\pgfsetlinewidth{0.803000pt}%
\definecolor{currentstroke}{rgb}{0.000000,0.000000,0.000000}%
\pgfsetstrokecolor{currentstroke}%
\pgfsetdash{}{0pt}%
\pgfsys@defobject{currentmarker}{\pgfqpoint{0.000000in}{-0.048611in}}{\pgfqpoint{0.000000in}{0.000000in}}{%
\pgfpathmoveto{\pgfqpoint{0.000000in}{0.000000in}}%
\pgfpathlineto{\pgfqpoint{0.000000in}{-0.048611in}}%
\pgfusepath{stroke,fill}%
}%
\begin{pgfscope}%
\pgfsys@transformshift{3.505236in}{0.552778in}%
\pgfsys@useobject{currentmarker}{}%
\end{pgfscope}%
\end{pgfscope}%
\begin{pgfscope}%
\pgfsetbuttcap%
\pgfsetroundjoin%
\definecolor{currentfill}{rgb}{0.000000,0.000000,0.000000}%
\pgfsetfillcolor{currentfill}%
\pgfsetlinewidth{0.803000pt}%
\definecolor{currentstroke}{rgb}{0.000000,0.000000,0.000000}%
\pgfsetstrokecolor{currentstroke}%
\pgfsetdash{}{0pt}%
\pgfsys@defobject{currentmarker}{\pgfqpoint{0.000000in}{0.000000in}}{\pgfqpoint{0.000000in}{0.048611in}}{%
\pgfpathmoveto{\pgfqpoint{0.000000in}{0.000000in}}%
\pgfpathlineto{\pgfqpoint{0.000000in}{0.048611in}}%
\pgfusepath{stroke,fill}%
}%
\begin{pgfscope}%
\pgfsys@transformshift{3.505236in}{3.801389in}%
\pgfsys@useobject{currentmarker}{}%
\end{pgfscope}%
\end{pgfscope}%
\begin{pgfscope}%
\definecolor{textcolor}{rgb}{0.000000,0.000000,0.000000}%
\pgfsetstrokecolor{textcolor}%
\pgfsetfillcolor{textcolor}%
\pgftext[x=3.505236in,y=0.455556in,,top]{\color{textcolor}\rmfamily\fontsize{10.000000}{12.000000}\selectfont 2100}%
\end{pgfscope}%
\begin{pgfscope}%
\pgfpathrectangle{\pgfqpoint{0.781944in}{0.552778in}}{\pgfqpoint{3.890972in}{3.248611in}}%
\pgfusepath{clip}%
\pgfsetrectcap%
\pgfsetroundjoin%
\pgfsetlinewidth{0.803000pt}%
\definecolor{currentstroke}{rgb}{0.690196,0.690196,0.690196}%
\pgfsetstrokecolor{currentstroke}%
\pgfsetstrokeopacity{0.800000}%
\pgfsetdash{}{0pt}%
\pgfpathmoveto{\pgfqpoint{3.894463in}{0.552778in}}%
\pgfpathlineto{\pgfqpoint{3.894463in}{3.801389in}}%
\pgfusepath{stroke}%
\end{pgfscope}%
\begin{pgfscope}%
\pgfsetbuttcap%
\pgfsetroundjoin%
\definecolor{currentfill}{rgb}{0.000000,0.000000,0.000000}%
\pgfsetfillcolor{currentfill}%
\pgfsetlinewidth{0.803000pt}%
\definecolor{currentstroke}{rgb}{0.000000,0.000000,0.000000}%
\pgfsetstrokecolor{currentstroke}%
\pgfsetdash{}{0pt}%
\pgfsys@defobject{currentmarker}{\pgfqpoint{0.000000in}{-0.048611in}}{\pgfqpoint{0.000000in}{0.000000in}}{%
\pgfpathmoveto{\pgfqpoint{0.000000in}{0.000000in}}%
\pgfpathlineto{\pgfqpoint{0.000000in}{-0.048611in}}%
\pgfusepath{stroke,fill}%
}%
\begin{pgfscope}%
\pgfsys@transformshift{3.894463in}{0.552778in}%
\pgfsys@useobject{currentmarker}{}%
\end{pgfscope}%
\end{pgfscope}%
\begin{pgfscope}%
\pgfsetbuttcap%
\pgfsetroundjoin%
\definecolor{currentfill}{rgb}{0.000000,0.000000,0.000000}%
\pgfsetfillcolor{currentfill}%
\pgfsetlinewidth{0.803000pt}%
\definecolor{currentstroke}{rgb}{0.000000,0.000000,0.000000}%
\pgfsetstrokecolor{currentstroke}%
\pgfsetdash{}{0pt}%
\pgfsys@defobject{currentmarker}{\pgfqpoint{0.000000in}{0.000000in}}{\pgfqpoint{0.000000in}{0.048611in}}{%
\pgfpathmoveto{\pgfqpoint{0.000000in}{0.000000in}}%
\pgfpathlineto{\pgfqpoint{0.000000in}{0.048611in}}%
\pgfusepath{stroke,fill}%
}%
\begin{pgfscope}%
\pgfsys@transformshift{3.894463in}{3.801389in}%
\pgfsys@useobject{currentmarker}{}%
\end{pgfscope}%
\end{pgfscope}%
\begin{pgfscope}%
\definecolor{textcolor}{rgb}{0.000000,0.000000,0.000000}%
\pgfsetstrokecolor{textcolor}%
\pgfsetfillcolor{textcolor}%
\pgftext[x=3.894463in,y=0.455556in,,top]{\color{textcolor}\rmfamily\fontsize{10.000000}{12.000000}\selectfont 2400}%
\end{pgfscope}%
\begin{pgfscope}%
\pgfpathrectangle{\pgfqpoint{0.781944in}{0.552778in}}{\pgfqpoint{3.890972in}{3.248611in}}%
\pgfusepath{clip}%
\pgfsetrectcap%
\pgfsetroundjoin%
\pgfsetlinewidth{0.803000pt}%
\definecolor{currentstroke}{rgb}{0.690196,0.690196,0.690196}%
\pgfsetstrokecolor{currentstroke}%
\pgfsetstrokeopacity{0.800000}%
\pgfsetdash{}{0pt}%
\pgfpathmoveto{\pgfqpoint{4.283690in}{0.552778in}}%
\pgfpathlineto{\pgfqpoint{4.283690in}{3.801389in}}%
\pgfusepath{stroke}%
\end{pgfscope}%
\begin{pgfscope}%
\pgfsetbuttcap%
\pgfsetroundjoin%
\definecolor{currentfill}{rgb}{0.000000,0.000000,0.000000}%
\pgfsetfillcolor{currentfill}%
\pgfsetlinewidth{0.803000pt}%
\definecolor{currentstroke}{rgb}{0.000000,0.000000,0.000000}%
\pgfsetstrokecolor{currentstroke}%
\pgfsetdash{}{0pt}%
\pgfsys@defobject{currentmarker}{\pgfqpoint{0.000000in}{-0.048611in}}{\pgfqpoint{0.000000in}{0.000000in}}{%
\pgfpathmoveto{\pgfqpoint{0.000000in}{0.000000in}}%
\pgfpathlineto{\pgfqpoint{0.000000in}{-0.048611in}}%
\pgfusepath{stroke,fill}%
}%
\begin{pgfscope}%
\pgfsys@transformshift{4.283690in}{0.552778in}%
\pgfsys@useobject{currentmarker}{}%
\end{pgfscope}%
\end{pgfscope}%
\begin{pgfscope}%
\pgfsetbuttcap%
\pgfsetroundjoin%
\definecolor{currentfill}{rgb}{0.000000,0.000000,0.000000}%
\pgfsetfillcolor{currentfill}%
\pgfsetlinewidth{0.803000pt}%
\definecolor{currentstroke}{rgb}{0.000000,0.000000,0.000000}%
\pgfsetstrokecolor{currentstroke}%
\pgfsetdash{}{0pt}%
\pgfsys@defobject{currentmarker}{\pgfqpoint{0.000000in}{0.000000in}}{\pgfqpoint{0.000000in}{0.048611in}}{%
\pgfpathmoveto{\pgfqpoint{0.000000in}{0.000000in}}%
\pgfpathlineto{\pgfqpoint{0.000000in}{0.048611in}}%
\pgfusepath{stroke,fill}%
}%
\begin{pgfscope}%
\pgfsys@transformshift{4.283690in}{3.801389in}%
\pgfsys@useobject{currentmarker}{}%
\end{pgfscope}%
\end{pgfscope}%
\begin{pgfscope}%
\definecolor{textcolor}{rgb}{0.000000,0.000000,0.000000}%
\pgfsetstrokecolor{textcolor}%
\pgfsetfillcolor{textcolor}%
\pgftext[x=4.283690in,y=0.455556in,,top]{\color{textcolor}\rmfamily\fontsize{10.000000}{12.000000}\selectfont 2700}%
\end{pgfscope}%
\begin{pgfscope}%
\pgfpathrectangle{\pgfqpoint{0.781944in}{0.552778in}}{\pgfqpoint{3.890972in}{3.248611in}}%
\pgfusepath{clip}%
\pgfsetrectcap%
\pgfsetroundjoin%
\pgfsetlinewidth{0.803000pt}%
\definecolor{currentstroke}{rgb}{0.690196,0.690196,0.690196}%
\pgfsetstrokecolor{currentstroke}%
\pgfsetstrokeopacity{0.800000}%
\pgfsetdash{}{0pt}%
\pgfpathmoveto{\pgfqpoint{4.672917in}{0.552778in}}%
\pgfpathlineto{\pgfqpoint{4.672917in}{3.801389in}}%
\pgfusepath{stroke}%
\end{pgfscope}%
\begin{pgfscope}%
\pgfsetbuttcap%
\pgfsetroundjoin%
\definecolor{currentfill}{rgb}{0.000000,0.000000,0.000000}%
\pgfsetfillcolor{currentfill}%
\pgfsetlinewidth{0.803000pt}%
\definecolor{currentstroke}{rgb}{0.000000,0.000000,0.000000}%
\pgfsetstrokecolor{currentstroke}%
\pgfsetdash{}{0pt}%
\pgfsys@defobject{currentmarker}{\pgfqpoint{0.000000in}{-0.048611in}}{\pgfqpoint{0.000000in}{0.000000in}}{%
\pgfpathmoveto{\pgfqpoint{0.000000in}{0.000000in}}%
\pgfpathlineto{\pgfqpoint{0.000000in}{-0.048611in}}%
\pgfusepath{stroke,fill}%
}%
\begin{pgfscope}%
\pgfsys@transformshift{4.672917in}{0.552778in}%
\pgfsys@useobject{currentmarker}{}%
\end{pgfscope}%
\end{pgfscope}%
\begin{pgfscope}%
\pgfsetbuttcap%
\pgfsetroundjoin%
\definecolor{currentfill}{rgb}{0.000000,0.000000,0.000000}%
\pgfsetfillcolor{currentfill}%
\pgfsetlinewidth{0.803000pt}%
\definecolor{currentstroke}{rgb}{0.000000,0.000000,0.000000}%
\pgfsetstrokecolor{currentstroke}%
\pgfsetdash{}{0pt}%
\pgfsys@defobject{currentmarker}{\pgfqpoint{0.000000in}{0.000000in}}{\pgfqpoint{0.000000in}{0.048611in}}{%
\pgfpathmoveto{\pgfqpoint{0.000000in}{0.000000in}}%
\pgfpathlineto{\pgfqpoint{0.000000in}{0.048611in}}%
\pgfusepath{stroke,fill}%
}%
\begin{pgfscope}%
\pgfsys@transformshift{4.672917in}{3.801389in}%
\pgfsys@useobject{currentmarker}{}%
\end{pgfscope}%
\end{pgfscope}%
\begin{pgfscope}%
\definecolor{textcolor}{rgb}{0.000000,0.000000,0.000000}%
\pgfsetstrokecolor{textcolor}%
\pgfsetfillcolor{textcolor}%
\pgftext[x=4.672917in,y=0.455556in,,top]{\color{textcolor}\rmfamily\fontsize{10.000000}{12.000000}\selectfont 3000}%
\end{pgfscope}%
\begin{pgfscope}%
\pgfpathrectangle{\pgfqpoint{0.781944in}{0.552778in}}{\pgfqpoint{3.890972in}{3.248611in}}%
\pgfusepath{clip}%
\pgfsetrectcap%
\pgfsetroundjoin%
\pgfsetlinewidth{0.803000pt}%
\definecolor{currentstroke}{rgb}{0.690196,0.690196,0.690196}%
\pgfsetstrokecolor{currentstroke}%
\pgfsetstrokeopacity{0.300000}%
\pgfsetdash{}{0pt}%
\pgfpathmoveto{\pgfqpoint{0.819570in}{0.552778in}}%
\pgfpathlineto{\pgfqpoint{0.819570in}{3.801389in}}%
\pgfusepath{stroke}%
\end{pgfscope}%
\begin{pgfscope}%
\pgfsetbuttcap%
\pgfsetroundjoin%
\definecolor{currentfill}{rgb}{0.000000,0.000000,0.000000}%
\pgfsetfillcolor{currentfill}%
\pgfsetlinewidth{0.602250pt}%
\definecolor{currentstroke}{rgb}{0.000000,0.000000,0.000000}%
\pgfsetstrokecolor{currentstroke}%
\pgfsetdash{}{0pt}%
\pgfsys@defobject{currentmarker}{\pgfqpoint{0.000000in}{-0.027778in}}{\pgfqpoint{0.000000in}{0.000000in}}{%
\pgfpathmoveto{\pgfqpoint{0.000000in}{0.000000in}}%
\pgfpathlineto{\pgfqpoint{0.000000in}{-0.027778in}}%
\pgfusepath{stroke,fill}%
}%
\begin{pgfscope}%
\pgfsys@transformshift{0.819570in}{0.552778in}%
\pgfsys@useobject{currentmarker}{}%
\end{pgfscope}%
\end{pgfscope}%
\begin{pgfscope}%
\pgfsetbuttcap%
\pgfsetroundjoin%
\definecolor{currentfill}{rgb}{0.000000,0.000000,0.000000}%
\pgfsetfillcolor{currentfill}%
\pgfsetlinewidth{0.602250pt}%
\definecolor{currentstroke}{rgb}{0.000000,0.000000,0.000000}%
\pgfsetstrokecolor{currentstroke}%
\pgfsetdash{}{0pt}%
\pgfsys@defobject{currentmarker}{\pgfqpoint{0.000000in}{0.000000in}}{\pgfqpoint{0.000000in}{0.027778in}}{%
\pgfpathmoveto{\pgfqpoint{0.000000in}{0.000000in}}%
\pgfpathlineto{\pgfqpoint{0.000000in}{0.027778in}}%
\pgfusepath{stroke,fill}%
}%
\begin{pgfscope}%
\pgfsys@transformshift{0.819570in}{3.801389in}%
\pgfsys@useobject{currentmarker}{}%
\end{pgfscope}%
\end{pgfscope}%
\begin{pgfscope}%
\pgfpathrectangle{\pgfqpoint{0.781944in}{0.552778in}}{\pgfqpoint{3.890972in}{3.248611in}}%
\pgfusepath{clip}%
\pgfsetrectcap%
\pgfsetroundjoin%
\pgfsetlinewidth{0.803000pt}%
\definecolor{currentstroke}{rgb}{0.690196,0.690196,0.690196}%
\pgfsetstrokecolor{currentstroke}%
\pgfsetstrokeopacity{0.300000}%
\pgfsetdash{}{0pt}%
\pgfpathmoveto{\pgfqpoint{0.858492in}{0.552778in}}%
\pgfpathlineto{\pgfqpoint{0.858492in}{3.801389in}}%
\pgfusepath{stroke}%
\end{pgfscope}%
\begin{pgfscope}%
\pgfsetbuttcap%
\pgfsetroundjoin%
\definecolor{currentfill}{rgb}{0.000000,0.000000,0.000000}%
\pgfsetfillcolor{currentfill}%
\pgfsetlinewidth{0.602250pt}%
\definecolor{currentstroke}{rgb}{0.000000,0.000000,0.000000}%
\pgfsetstrokecolor{currentstroke}%
\pgfsetdash{}{0pt}%
\pgfsys@defobject{currentmarker}{\pgfqpoint{0.000000in}{-0.027778in}}{\pgfqpoint{0.000000in}{0.000000in}}{%
\pgfpathmoveto{\pgfqpoint{0.000000in}{0.000000in}}%
\pgfpathlineto{\pgfqpoint{0.000000in}{-0.027778in}}%
\pgfusepath{stroke,fill}%
}%
\begin{pgfscope}%
\pgfsys@transformshift{0.858492in}{0.552778in}%
\pgfsys@useobject{currentmarker}{}%
\end{pgfscope}%
\end{pgfscope}%
\begin{pgfscope}%
\pgfsetbuttcap%
\pgfsetroundjoin%
\definecolor{currentfill}{rgb}{0.000000,0.000000,0.000000}%
\pgfsetfillcolor{currentfill}%
\pgfsetlinewidth{0.602250pt}%
\definecolor{currentstroke}{rgb}{0.000000,0.000000,0.000000}%
\pgfsetstrokecolor{currentstroke}%
\pgfsetdash{}{0pt}%
\pgfsys@defobject{currentmarker}{\pgfqpoint{0.000000in}{0.000000in}}{\pgfqpoint{0.000000in}{0.027778in}}{%
\pgfpathmoveto{\pgfqpoint{0.000000in}{0.000000in}}%
\pgfpathlineto{\pgfqpoint{0.000000in}{0.027778in}}%
\pgfusepath{stroke,fill}%
}%
\begin{pgfscope}%
\pgfsys@transformshift{0.858492in}{3.801389in}%
\pgfsys@useobject{currentmarker}{}%
\end{pgfscope}%
\end{pgfscope}%
\begin{pgfscope}%
\pgfpathrectangle{\pgfqpoint{0.781944in}{0.552778in}}{\pgfqpoint{3.890972in}{3.248611in}}%
\pgfusepath{clip}%
\pgfsetrectcap%
\pgfsetroundjoin%
\pgfsetlinewidth{0.803000pt}%
\definecolor{currentstroke}{rgb}{0.690196,0.690196,0.690196}%
\pgfsetstrokecolor{currentstroke}%
\pgfsetstrokeopacity{0.300000}%
\pgfsetdash{}{0pt}%
\pgfpathmoveto{\pgfqpoint{0.897415in}{0.552778in}}%
\pgfpathlineto{\pgfqpoint{0.897415in}{3.801389in}}%
\pgfusepath{stroke}%
\end{pgfscope}%
\begin{pgfscope}%
\pgfsetbuttcap%
\pgfsetroundjoin%
\definecolor{currentfill}{rgb}{0.000000,0.000000,0.000000}%
\pgfsetfillcolor{currentfill}%
\pgfsetlinewidth{0.602250pt}%
\definecolor{currentstroke}{rgb}{0.000000,0.000000,0.000000}%
\pgfsetstrokecolor{currentstroke}%
\pgfsetdash{}{0pt}%
\pgfsys@defobject{currentmarker}{\pgfqpoint{0.000000in}{-0.027778in}}{\pgfqpoint{0.000000in}{0.000000in}}{%
\pgfpathmoveto{\pgfqpoint{0.000000in}{0.000000in}}%
\pgfpathlineto{\pgfqpoint{0.000000in}{-0.027778in}}%
\pgfusepath{stroke,fill}%
}%
\begin{pgfscope}%
\pgfsys@transformshift{0.897415in}{0.552778in}%
\pgfsys@useobject{currentmarker}{}%
\end{pgfscope}%
\end{pgfscope}%
\begin{pgfscope}%
\pgfsetbuttcap%
\pgfsetroundjoin%
\definecolor{currentfill}{rgb}{0.000000,0.000000,0.000000}%
\pgfsetfillcolor{currentfill}%
\pgfsetlinewidth{0.602250pt}%
\definecolor{currentstroke}{rgb}{0.000000,0.000000,0.000000}%
\pgfsetstrokecolor{currentstroke}%
\pgfsetdash{}{0pt}%
\pgfsys@defobject{currentmarker}{\pgfqpoint{0.000000in}{0.000000in}}{\pgfqpoint{0.000000in}{0.027778in}}{%
\pgfpathmoveto{\pgfqpoint{0.000000in}{0.000000in}}%
\pgfpathlineto{\pgfqpoint{0.000000in}{0.027778in}}%
\pgfusepath{stroke,fill}%
}%
\begin{pgfscope}%
\pgfsys@transformshift{0.897415in}{3.801389in}%
\pgfsys@useobject{currentmarker}{}%
\end{pgfscope}%
\end{pgfscope}%
\begin{pgfscope}%
\pgfpathrectangle{\pgfqpoint{0.781944in}{0.552778in}}{\pgfqpoint{3.890972in}{3.248611in}}%
\pgfusepath{clip}%
\pgfsetrectcap%
\pgfsetroundjoin%
\pgfsetlinewidth{0.803000pt}%
\definecolor{currentstroke}{rgb}{0.690196,0.690196,0.690196}%
\pgfsetstrokecolor{currentstroke}%
\pgfsetstrokeopacity{0.300000}%
\pgfsetdash{}{0pt}%
\pgfpathmoveto{\pgfqpoint{0.936338in}{0.552778in}}%
\pgfpathlineto{\pgfqpoint{0.936338in}{3.801389in}}%
\pgfusepath{stroke}%
\end{pgfscope}%
\begin{pgfscope}%
\pgfsetbuttcap%
\pgfsetroundjoin%
\definecolor{currentfill}{rgb}{0.000000,0.000000,0.000000}%
\pgfsetfillcolor{currentfill}%
\pgfsetlinewidth{0.602250pt}%
\definecolor{currentstroke}{rgb}{0.000000,0.000000,0.000000}%
\pgfsetstrokecolor{currentstroke}%
\pgfsetdash{}{0pt}%
\pgfsys@defobject{currentmarker}{\pgfqpoint{0.000000in}{-0.027778in}}{\pgfqpoint{0.000000in}{0.000000in}}{%
\pgfpathmoveto{\pgfqpoint{0.000000in}{0.000000in}}%
\pgfpathlineto{\pgfqpoint{0.000000in}{-0.027778in}}%
\pgfusepath{stroke,fill}%
}%
\begin{pgfscope}%
\pgfsys@transformshift{0.936338in}{0.552778in}%
\pgfsys@useobject{currentmarker}{}%
\end{pgfscope}%
\end{pgfscope}%
\begin{pgfscope}%
\pgfsetbuttcap%
\pgfsetroundjoin%
\definecolor{currentfill}{rgb}{0.000000,0.000000,0.000000}%
\pgfsetfillcolor{currentfill}%
\pgfsetlinewidth{0.602250pt}%
\definecolor{currentstroke}{rgb}{0.000000,0.000000,0.000000}%
\pgfsetstrokecolor{currentstroke}%
\pgfsetdash{}{0pt}%
\pgfsys@defobject{currentmarker}{\pgfqpoint{0.000000in}{0.000000in}}{\pgfqpoint{0.000000in}{0.027778in}}{%
\pgfpathmoveto{\pgfqpoint{0.000000in}{0.000000in}}%
\pgfpathlineto{\pgfqpoint{0.000000in}{0.027778in}}%
\pgfusepath{stroke,fill}%
}%
\begin{pgfscope}%
\pgfsys@transformshift{0.936338in}{3.801389in}%
\pgfsys@useobject{currentmarker}{}%
\end{pgfscope}%
\end{pgfscope}%
\begin{pgfscope}%
\pgfpathrectangle{\pgfqpoint{0.781944in}{0.552778in}}{\pgfqpoint{3.890972in}{3.248611in}}%
\pgfusepath{clip}%
\pgfsetrectcap%
\pgfsetroundjoin%
\pgfsetlinewidth{0.803000pt}%
\definecolor{currentstroke}{rgb}{0.690196,0.690196,0.690196}%
\pgfsetstrokecolor{currentstroke}%
\pgfsetstrokeopacity{0.300000}%
\pgfsetdash{}{0pt}%
\pgfpathmoveto{\pgfqpoint{0.975261in}{0.552778in}}%
\pgfpathlineto{\pgfqpoint{0.975261in}{3.801389in}}%
\pgfusepath{stroke}%
\end{pgfscope}%
\begin{pgfscope}%
\pgfsetbuttcap%
\pgfsetroundjoin%
\definecolor{currentfill}{rgb}{0.000000,0.000000,0.000000}%
\pgfsetfillcolor{currentfill}%
\pgfsetlinewidth{0.602250pt}%
\definecolor{currentstroke}{rgb}{0.000000,0.000000,0.000000}%
\pgfsetstrokecolor{currentstroke}%
\pgfsetdash{}{0pt}%
\pgfsys@defobject{currentmarker}{\pgfqpoint{0.000000in}{-0.027778in}}{\pgfqpoint{0.000000in}{0.000000in}}{%
\pgfpathmoveto{\pgfqpoint{0.000000in}{0.000000in}}%
\pgfpathlineto{\pgfqpoint{0.000000in}{-0.027778in}}%
\pgfusepath{stroke,fill}%
}%
\begin{pgfscope}%
\pgfsys@transformshift{0.975261in}{0.552778in}%
\pgfsys@useobject{currentmarker}{}%
\end{pgfscope}%
\end{pgfscope}%
\begin{pgfscope}%
\pgfsetbuttcap%
\pgfsetroundjoin%
\definecolor{currentfill}{rgb}{0.000000,0.000000,0.000000}%
\pgfsetfillcolor{currentfill}%
\pgfsetlinewidth{0.602250pt}%
\definecolor{currentstroke}{rgb}{0.000000,0.000000,0.000000}%
\pgfsetstrokecolor{currentstroke}%
\pgfsetdash{}{0pt}%
\pgfsys@defobject{currentmarker}{\pgfqpoint{0.000000in}{0.000000in}}{\pgfqpoint{0.000000in}{0.027778in}}{%
\pgfpathmoveto{\pgfqpoint{0.000000in}{0.000000in}}%
\pgfpathlineto{\pgfqpoint{0.000000in}{0.027778in}}%
\pgfusepath{stroke,fill}%
}%
\begin{pgfscope}%
\pgfsys@transformshift{0.975261in}{3.801389in}%
\pgfsys@useobject{currentmarker}{}%
\end{pgfscope}%
\end{pgfscope}%
\begin{pgfscope}%
\pgfpathrectangle{\pgfqpoint{0.781944in}{0.552778in}}{\pgfqpoint{3.890972in}{3.248611in}}%
\pgfusepath{clip}%
\pgfsetrectcap%
\pgfsetroundjoin%
\pgfsetlinewidth{0.803000pt}%
\definecolor{currentstroke}{rgb}{0.690196,0.690196,0.690196}%
\pgfsetstrokecolor{currentstroke}%
\pgfsetstrokeopacity{0.300000}%
\pgfsetdash{}{0pt}%
\pgfpathmoveto{\pgfqpoint{1.014183in}{0.552778in}}%
\pgfpathlineto{\pgfqpoint{1.014183in}{3.801389in}}%
\pgfusepath{stroke}%
\end{pgfscope}%
\begin{pgfscope}%
\pgfsetbuttcap%
\pgfsetroundjoin%
\definecolor{currentfill}{rgb}{0.000000,0.000000,0.000000}%
\pgfsetfillcolor{currentfill}%
\pgfsetlinewidth{0.602250pt}%
\definecolor{currentstroke}{rgb}{0.000000,0.000000,0.000000}%
\pgfsetstrokecolor{currentstroke}%
\pgfsetdash{}{0pt}%
\pgfsys@defobject{currentmarker}{\pgfqpoint{0.000000in}{-0.027778in}}{\pgfqpoint{0.000000in}{0.000000in}}{%
\pgfpathmoveto{\pgfqpoint{0.000000in}{0.000000in}}%
\pgfpathlineto{\pgfqpoint{0.000000in}{-0.027778in}}%
\pgfusepath{stroke,fill}%
}%
\begin{pgfscope}%
\pgfsys@transformshift{1.014183in}{0.552778in}%
\pgfsys@useobject{currentmarker}{}%
\end{pgfscope}%
\end{pgfscope}%
\begin{pgfscope}%
\pgfsetbuttcap%
\pgfsetroundjoin%
\definecolor{currentfill}{rgb}{0.000000,0.000000,0.000000}%
\pgfsetfillcolor{currentfill}%
\pgfsetlinewidth{0.602250pt}%
\definecolor{currentstroke}{rgb}{0.000000,0.000000,0.000000}%
\pgfsetstrokecolor{currentstroke}%
\pgfsetdash{}{0pt}%
\pgfsys@defobject{currentmarker}{\pgfqpoint{0.000000in}{0.000000in}}{\pgfqpoint{0.000000in}{0.027778in}}{%
\pgfpathmoveto{\pgfqpoint{0.000000in}{0.000000in}}%
\pgfpathlineto{\pgfqpoint{0.000000in}{0.027778in}}%
\pgfusepath{stroke,fill}%
}%
\begin{pgfscope}%
\pgfsys@transformshift{1.014183in}{3.801389in}%
\pgfsys@useobject{currentmarker}{}%
\end{pgfscope}%
\end{pgfscope}%
\begin{pgfscope}%
\pgfpathrectangle{\pgfqpoint{0.781944in}{0.552778in}}{\pgfqpoint{3.890972in}{3.248611in}}%
\pgfusepath{clip}%
\pgfsetrectcap%
\pgfsetroundjoin%
\pgfsetlinewidth{0.803000pt}%
\definecolor{currentstroke}{rgb}{0.690196,0.690196,0.690196}%
\pgfsetstrokecolor{currentstroke}%
\pgfsetstrokeopacity{0.300000}%
\pgfsetdash{}{0pt}%
\pgfpathmoveto{\pgfqpoint{1.053106in}{0.552778in}}%
\pgfpathlineto{\pgfqpoint{1.053106in}{3.801389in}}%
\pgfusepath{stroke}%
\end{pgfscope}%
\begin{pgfscope}%
\pgfsetbuttcap%
\pgfsetroundjoin%
\definecolor{currentfill}{rgb}{0.000000,0.000000,0.000000}%
\pgfsetfillcolor{currentfill}%
\pgfsetlinewidth{0.602250pt}%
\definecolor{currentstroke}{rgb}{0.000000,0.000000,0.000000}%
\pgfsetstrokecolor{currentstroke}%
\pgfsetdash{}{0pt}%
\pgfsys@defobject{currentmarker}{\pgfqpoint{0.000000in}{-0.027778in}}{\pgfqpoint{0.000000in}{0.000000in}}{%
\pgfpathmoveto{\pgfqpoint{0.000000in}{0.000000in}}%
\pgfpathlineto{\pgfqpoint{0.000000in}{-0.027778in}}%
\pgfusepath{stroke,fill}%
}%
\begin{pgfscope}%
\pgfsys@transformshift{1.053106in}{0.552778in}%
\pgfsys@useobject{currentmarker}{}%
\end{pgfscope}%
\end{pgfscope}%
\begin{pgfscope}%
\pgfsetbuttcap%
\pgfsetroundjoin%
\definecolor{currentfill}{rgb}{0.000000,0.000000,0.000000}%
\pgfsetfillcolor{currentfill}%
\pgfsetlinewidth{0.602250pt}%
\definecolor{currentstroke}{rgb}{0.000000,0.000000,0.000000}%
\pgfsetstrokecolor{currentstroke}%
\pgfsetdash{}{0pt}%
\pgfsys@defobject{currentmarker}{\pgfqpoint{0.000000in}{0.000000in}}{\pgfqpoint{0.000000in}{0.027778in}}{%
\pgfpathmoveto{\pgfqpoint{0.000000in}{0.000000in}}%
\pgfpathlineto{\pgfqpoint{0.000000in}{0.027778in}}%
\pgfusepath{stroke,fill}%
}%
\begin{pgfscope}%
\pgfsys@transformshift{1.053106in}{3.801389in}%
\pgfsys@useobject{currentmarker}{}%
\end{pgfscope}%
\end{pgfscope}%
\begin{pgfscope}%
\pgfpathrectangle{\pgfqpoint{0.781944in}{0.552778in}}{\pgfqpoint{3.890972in}{3.248611in}}%
\pgfusepath{clip}%
\pgfsetrectcap%
\pgfsetroundjoin%
\pgfsetlinewidth{0.803000pt}%
\definecolor{currentstroke}{rgb}{0.690196,0.690196,0.690196}%
\pgfsetstrokecolor{currentstroke}%
\pgfsetstrokeopacity{0.300000}%
\pgfsetdash{}{0pt}%
\pgfpathmoveto{\pgfqpoint{1.092029in}{0.552778in}}%
\pgfpathlineto{\pgfqpoint{1.092029in}{3.801389in}}%
\pgfusepath{stroke}%
\end{pgfscope}%
\begin{pgfscope}%
\pgfsetbuttcap%
\pgfsetroundjoin%
\definecolor{currentfill}{rgb}{0.000000,0.000000,0.000000}%
\pgfsetfillcolor{currentfill}%
\pgfsetlinewidth{0.602250pt}%
\definecolor{currentstroke}{rgb}{0.000000,0.000000,0.000000}%
\pgfsetstrokecolor{currentstroke}%
\pgfsetdash{}{0pt}%
\pgfsys@defobject{currentmarker}{\pgfqpoint{0.000000in}{-0.027778in}}{\pgfqpoint{0.000000in}{0.000000in}}{%
\pgfpathmoveto{\pgfqpoint{0.000000in}{0.000000in}}%
\pgfpathlineto{\pgfqpoint{0.000000in}{-0.027778in}}%
\pgfusepath{stroke,fill}%
}%
\begin{pgfscope}%
\pgfsys@transformshift{1.092029in}{0.552778in}%
\pgfsys@useobject{currentmarker}{}%
\end{pgfscope}%
\end{pgfscope}%
\begin{pgfscope}%
\pgfsetbuttcap%
\pgfsetroundjoin%
\definecolor{currentfill}{rgb}{0.000000,0.000000,0.000000}%
\pgfsetfillcolor{currentfill}%
\pgfsetlinewidth{0.602250pt}%
\definecolor{currentstroke}{rgb}{0.000000,0.000000,0.000000}%
\pgfsetstrokecolor{currentstroke}%
\pgfsetdash{}{0pt}%
\pgfsys@defobject{currentmarker}{\pgfqpoint{0.000000in}{0.000000in}}{\pgfqpoint{0.000000in}{0.027778in}}{%
\pgfpathmoveto{\pgfqpoint{0.000000in}{0.000000in}}%
\pgfpathlineto{\pgfqpoint{0.000000in}{0.027778in}}%
\pgfusepath{stroke,fill}%
}%
\begin{pgfscope}%
\pgfsys@transformshift{1.092029in}{3.801389in}%
\pgfsys@useobject{currentmarker}{}%
\end{pgfscope}%
\end{pgfscope}%
\begin{pgfscope}%
\pgfpathrectangle{\pgfqpoint{0.781944in}{0.552778in}}{\pgfqpoint{3.890972in}{3.248611in}}%
\pgfusepath{clip}%
\pgfsetrectcap%
\pgfsetroundjoin%
\pgfsetlinewidth{0.803000pt}%
\definecolor{currentstroke}{rgb}{0.690196,0.690196,0.690196}%
\pgfsetstrokecolor{currentstroke}%
\pgfsetstrokeopacity{0.300000}%
\pgfsetdash{}{0pt}%
\pgfpathmoveto{\pgfqpoint{1.130951in}{0.552778in}}%
\pgfpathlineto{\pgfqpoint{1.130951in}{3.801389in}}%
\pgfusepath{stroke}%
\end{pgfscope}%
\begin{pgfscope}%
\pgfsetbuttcap%
\pgfsetroundjoin%
\definecolor{currentfill}{rgb}{0.000000,0.000000,0.000000}%
\pgfsetfillcolor{currentfill}%
\pgfsetlinewidth{0.602250pt}%
\definecolor{currentstroke}{rgb}{0.000000,0.000000,0.000000}%
\pgfsetstrokecolor{currentstroke}%
\pgfsetdash{}{0pt}%
\pgfsys@defobject{currentmarker}{\pgfqpoint{0.000000in}{-0.027778in}}{\pgfqpoint{0.000000in}{0.000000in}}{%
\pgfpathmoveto{\pgfqpoint{0.000000in}{0.000000in}}%
\pgfpathlineto{\pgfqpoint{0.000000in}{-0.027778in}}%
\pgfusepath{stroke,fill}%
}%
\begin{pgfscope}%
\pgfsys@transformshift{1.130951in}{0.552778in}%
\pgfsys@useobject{currentmarker}{}%
\end{pgfscope}%
\end{pgfscope}%
\begin{pgfscope}%
\pgfsetbuttcap%
\pgfsetroundjoin%
\definecolor{currentfill}{rgb}{0.000000,0.000000,0.000000}%
\pgfsetfillcolor{currentfill}%
\pgfsetlinewidth{0.602250pt}%
\definecolor{currentstroke}{rgb}{0.000000,0.000000,0.000000}%
\pgfsetstrokecolor{currentstroke}%
\pgfsetdash{}{0pt}%
\pgfsys@defobject{currentmarker}{\pgfqpoint{0.000000in}{0.000000in}}{\pgfqpoint{0.000000in}{0.027778in}}{%
\pgfpathmoveto{\pgfqpoint{0.000000in}{0.000000in}}%
\pgfpathlineto{\pgfqpoint{0.000000in}{0.027778in}}%
\pgfusepath{stroke,fill}%
}%
\begin{pgfscope}%
\pgfsys@transformshift{1.130951in}{3.801389in}%
\pgfsys@useobject{currentmarker}{}%
\end{pgfscope}%
\end{pgfscope}%
\begin{pgfscope}%
\pgfpathrectangle{\pgfqpoint{0.781944in}{0.552778in}}{\pgfqpoint{3.890972in}{3.248611in}}%
\pgfusepath{clip}%
\pgfsetrectcap%
\pgfsetroundjoin%
\pgfsetlinewidth{0.803000pt}%
\definecolor{currentstroke}{rgb}{0.690196,0.690196,0.690196}%
\pgfsetstrokecolor{currentstroke}%
\pgfsetstrokeopacity{0.300000}%
\pgfsetdash{}{0pt}%
\pgfpathmoveto{\pgfqpoint{1.208797in}{0.552778in}}%
\pgfpathlineto{\pgfqpoint{1.208797in}{3.801389in}}%
\pgfusepath{stroke}%
\end{pgfscope}%
\begin{pgfscope}%
\pgfsetbuttcap%
\pgfsetroundjoin%
\definecolor{currentfill}{rgb}{0.000000,0.000000,0.000000}%
\pgfsetfillcolor{currentfill}%
\pgfsetlinewidth{0.602250pt}%
\definecolor{currentstroke}{rgb}{0.000000,0.000000,0.000000}%
\pgfsetstrokecolor{currentstroke}%
\pgfsetdash{}{0pt}%
\pgfsys@defobject{currentmarker}{\pgfqpoint{0.000000in}{-0.027778in}}{\pgfqpoint{0.000000in}{0.000000in}}{%
\pgfpathmoveto{\pgfqpoint{0.000000in}{0.000000in}}%
\pgfpathlineto{\pgfqpoint{0.000000in}{-0.027778in}}%
\pgfusepath{stroke,fill}%
}%
\begin{pgfscope}%
\pgfsys@transformshift{1.208797in}{0.552778in}%
\pgfsys@useobject{currentmarker}{}%
\end{pgfscope}%
\end{pgfscope}%
\begin{pgfscope}%
\pgfsetbuttcap%
\pgfsetroundjoin%
\definecolor{currentfill}{rgb}{0.000000,0.000000,0.000000}%
\pgfsetfillcolor{currentfill}%
\pgfsetlinewidth{0.602250pt}%
\definecolor{currentstroke}{rgb}{0.000000,0.000000,0.000000}%
\pgfsetstrokecolor{currentstroke}%
\pgfsetdash{}{0pt}%
\pgfsys@defobject{currentmarker}{\pgfqpoint{0.000000in}{0.000000in}}{\pgfqpoint{0.000000in}{0.027778in}}{%
\pgfpathmoveto{\pgfqpoint{0.000000in}{0.000000in}}%
\pgfpathlineto{\pgfqpoint{0.000000in}{0.027778in}}%
\pgfusepath{stroke,fill}%
}%
\begin{pgfscope}%
\pgfsys@transformshift{1.208797in}{3.801389in}%
\pgfsys@useobject{currentmarker}{}%
\end{pgfscope}%
\end{pgfscope}%
\begin{pgfscope}%
\pgfpathrectangle{\pgfqpoint{0.781944in}{0.552778in}}{\pgfqpoint{3.890972in}{3.248611in}}%
\pgfusepath{clip}%
\pgfsetrectcap%
\pgfsetroundjoin%
\pgfsetlinewidth{0.803000pt}%
\definecolor{currentstroke}{rgb}{0.690196,0.690196,0.690196}%
\pgfsetstrokecolor{currentstroke}%
\pgfsetstrokeopacity{0.300000}%
\pgfsetdash{}{0pt}%
\pgfpathmoveto{\pgfqpoint{1.247719in}{0.552778in}}%
\pgfpathlineto{\pgfqpoint{1.247719in}{3.801389in}}%
\pgfusepath{stroke}%
\end{pgfscope}%
\begin{pgfscope}%
\pgfsetbuttcap%
\pgfsetroundjoin%
\definecolor{currentfill}{rgb}{0.000000,0.000000,0.000000}%
\pgfsetfillcolor{currentfill}%
\pgfsetlinewidth{0.602250pt}%
\definecolor{currentstroke}{rgb}{0.000000,0.000000,0.000000}%
\pgfsetstrokecolor{currentstroke}%
\pgfsetdash{}{0pt}%
\pgfsys@defobject{currentmarker}{\pgfqpoint{0.000000in}{-0.027778in}}{\pgfqpoint{0.000000in}{0.000000in}}{%
\pgfpathmoveto{\pgfqpoint{0.000000in}{0.000000in}}%
\pgfpathlineto{\pgfqpoint{0.000000in}{-0.027778in}}%
\pgfusepath{stroke,fill}%
}%
\begin{pgfscope}%
\pgfsys@transformshift{1.247719in}{0.552778in}%
\pgfsys@useobject{currentmarker}{}%
\end{pgfscope}%
\end{pgfscope}%
\begin{pgfscope}%
\pgfsetbuttcap%
\pgfsetroundjoin%
\definecolor{currentfill}{rgb}{0.000000,0.000000,0.000000}%
\pgfsetfillcolor{currentfill}%
\pgfsetlinewidth{0.602250pt}%
\definecolor{currentstroke}{rgb}{0.000000,0.000000,0.000000}%
\pgfsetstrokecolor{currentstroke}%
\pgfsetdash{}{0pt}%
\pgfsys@defobject{currentmarker}{\pgfqpoint{0.000000in}{0.000000in}}{\pgfqpoint{0.000000in}{0.027778in}}{%
\pgfpathmoveto{\pgfqpoint{0.000000in}{0.000000in}}%
\pgfpathlineto{\pgfqpoint{0.000000in}{0.027778in}}%
\pgfusepath{stroke,fill}%
}%
\begin{pgfscope}%
\pgfsys@transformshift{1.247719in}{3.801389in}%
\pgfsys@useobject{currentmarker}{}%
\end{pgfscope}%
\end{pgfscope}%
\begin{pgfscope}%
\pgfpathrectangle{\pgfqpoint{0.781944in}{0.552778in}}{\pgfqpoint{3.890972in}{3.248611in}}%
\pgfusepath{clip}%
\pgfsetrectcap%
\pgfsetroundjoin%
\pgfsetlinewidth{0.803000pt}%
\definecolor{currentstroke}{rgb}{0.690196,0.690196,0.690196}%
\pgfsetstrokecolor{currentstroke}%
\pgfsetstrokeopacity{0.300000}%
\pgfsetdash{}{0pt}%
\pgfpathmoveto{\pgfqpoint{1.286642in}{0.552778in}}%
\pgfpathlineto{\pgfqpoint{1.286642in}{3.801389in}}%
\pgfusepath{stroke}%
\end{pgfscope}%
\begin{pgfscope}%
\pgfsetbuttcap%
\pgfsetroundjoin%
\definecolor{currentfill}{rgb}{0.000000,0.000000,0.000000}%
\pgfsetfillcolor{currentfill}%
\pgfsetlinewidth{0.602250pt}%
\definecolor{currentstroke}{rgb}{0.000000,0.000000,0.000000}%
\pgfsetstrokecolor{currentstroke}%
\pgfsetdash{}{0pt}%
\pgfsys@defobject{currentmarker}{\pgfqpoint{0.000000in}{-0.027778in}}{\pgfqpoint{0.000000in}{0.000000in}}{%
\pgfpathmoveto{\pgfqpoint{0.000000in}{0.000000in}}%
\pgfpathlineto{\pgfqpoint{0.000000in}{-0.027778in}}%
\pgfusepath{stroke,fill}%
}%
\begin{pgfscope}%
\pgfsys@transformshift{1.286642in}{0.552778in}%
\pgfsys@useobject{currentmarker}{}%
\end{pgfscope}%
\end{pgfscope}%
\begin{pgfscope}%
\pgfsetbuttcap%
\pgfsetroundjoin%
\definecolor{currentfill}{rgb}{0.000000,0.000000,0.000000}%
\pgfsetfillcolor{currentfill}%
\pgfsetlinewidth{0.602250pt}%
\definecolor{currentstroke}{rgb}{0.000000,0.000000,0.000000}%
\pgfsetstrokecolor{currentstroke}%
\pgfsetdash{}{0pt}%
\pgfsys@defobject{currentmarker}{\pgfqpoint{0.000000in}{0.000000in}}{\pgfqpoint{0.000000in}{0.027778in}}{%
\pgfpathmoveto{\pgfqpoint{0.000000in}{0.000000in}}%
\pgfpathlineto{\pgfqpoint{0.000000in}{0.027778in}}%
\pgfusepath{stroke,fill}%
}%
\begin{pgfscope}%
\pgfsys@transformshift{1.286642in}{3.801389in}%
\pgfsys@useobject{currentmarker}{}%
\end{pgfscope}%
\end{pgfscope}%
\begin{pgfscope}%
\pgfpathrectangle{\pgfqpoint{0.781944in}{0.552778in}}{\pgfqpoint{3.890972in}{3.248611in}}%
\pgfusepath{clip}%
\pgfsetrectcap%
\pgfsetroundjoin%
\pgfsetlinewidth{0.803000pt}%
\definecolor{currentstroke}{rgb}{0.690196,0.690196,0.690196}%
\pgfsetstrokecolor{currentstroke}%
\pgfsetstrokeopacity{0.300000}%
\pgfsetdash{}{0pt}%
\pgfpathmoveto{\pgfqpoint{1.325565in}{0.552778in}}%
\pgfpathlineto{\pgfqpoint{1.325565in}{3.801389in}}%
\pgfusepath{stroke}%
\end{pgfscope}%
\begin{pgfscope}%
\pgfsetbuttcap%
\pgfsetroundjoin%
\definecolor{currentfill}{rgb}{0.000000,0.000000,0.000000}%
\pgfsetfillcolor{currentfill}%
\pgfsetlinewidth{0.602250pt}%
\definecolor{currentstroke}{rgb}{0.000000,0.000000,0.000000}%
\pgfsetstrokecolor{currentstroke}%
\pgfsetdash{}{0pt}%
\pgfsys@defobject{currentmarker}{\pgfqpoint{0.000000in}{-0.027778in}}{\pgfqpoint{0.000000in}{0.000000in}}{%
\pgfpathmoveto{\pgfqpoint{0.000000in}{0.000000in}}%
\pgfpathlineto{\pgfqpoint{0.000000in}{-0.027778in}}%
\pgfusepath{stroke,fill}%
}%
\begin{pgfscope}%
\pgfsys@transformshift{1.325565in}{0.552778in}%
\pgfsys@useobject{currentmarker}{}%
\end{pgfscope}%
\end{pgfscope}%
\begin{pgfscope}%
\pgfsetbuttcap%
\pgfsetroundjoin%
\definecolor{currentfill}{rgb}{0.000000,0.000000,0.000000}%
\pgfsetfillcolor{currentfill}%
\pgfsetlinewidth{0.602250pt}%
\definecolor{currentstroke}{rgb}{0.000000,0.000000,0.000000}%
\pgfsetstrokecolor{currentstroke}%
\pgfsetdash{}{0pt}%
\pgfsys@defobject{currentmarker}{\pgfqpoint{0.000000in}{0.000000in}}{\pgfqpoint{0.000000in}{0.027778in}}{%
\pgfpathmoveto{\pgfqpoint{0.000000in}{0.000000in}}%
\pgfpathlineto{\pgfqpoint{0.000000in}{0.027778in}}%
\pgfusepath{stroke,fill}%
}%
\begin{pgfscope}%
\pgfsys@transformshift{1.325565in}{3.801389in}%
\pgfsys@useobject{currentmarker}{}%
\end{pgfscope}%
\end{pgfscope}%
\begin{pgfscope}%
\pgfpathrectangle{\pgfqpoint{0.781944in}{0.552778in}}{\pgfqpoint{3.890972in}{3.248611in}}%
\pgfusepath{clip}%
\pgfsetrectcap%
\pgfsetroundjoin%
\pgfsetlinewidth{0.803000pt}%
\definecolor{currentstroke}{rgb}{0.690196,0.690196,0.690196}%
\pgfsetstrokecolor{currentstroke}%
\pgfsetstrokeopacity{0.300000}%
\pgfsetdash{}{0pt}%
\pgfpathmoveto{\pgfqpoint{1.364487in}{0.552778in}}%
\pgfpathlineto{\pgfqpoint{1.364487in}{3.801389in}}%
\pgfusepath{stroke}%
\end{pgfscope}%
\begin{pgfscope}%
\pgfsetbuttcap%
\pgfsetroundjoin%
\definecolor{currentfill}{rgb}{0.000000,0.000000,0.000000}%
\pgfsetfillcolor{currentfill}%
\pgfsetlinewidth{0.602250pt}%
\definecolor{currentstroke}{rgb}{0.000000,0.000000,0.000000}%
\pgfsetstrokecolor{currentstroke}%
\pgfsetdash{}{0pt}%
\pgfsys@defobject{currentmarker}{\pgfqpoint{0.000000in}{-0.027778in}}{\pgfqpoint{0.000000in}{0.000000in}}{%
\pgfpathmoveto{\pgfqpoint{0.000000in}{0.000000in}}%
\pgfpathlineto{\pgfqpoint{0.000000in}{-0.027778in}}%
\pgfusepath{stroke,fill}%
}%
\begin{pgfscope}%
\pgfsys@transformshift{1.364487in}{0.552778in}%
\pgfsys@useobject{currentmarker}{}%
\end{pgfscope}%
\end{pgfscope}%
\begin{pgfscope}%
\pgfsetbuttcap%
\pgfsetroundjoin%
\definecolor{currentfill}{rgb}{0.000000,0.000000,0.000000}%
\pgfsetfillcolor{currentfill}%
\pgfsetlinewidth{0.602250pt}%
\definecolor{currentstroke}{rgb}{0.000000,0.000000,0.000000}%
\pgfsetstrokecolor{currentstroke}%
\pgfsetdash{}{0pt}%
\pgfsys@defobject{currentmarker}{\pgfqpoint{0.000000in}{0.000000in}}{\pgfqpoint{0.000000in}{0.027778in}}{%
\pgfpathmoveto{\pgfqpoint{0.000000in}{0.000000in}}%
\pgfpathlineto{\pgfqpoint{0.000000in}{0.027778in}}%
\pgfusepath{stroke,fill}%
}%
\begin{pgfscope}%
\pgfsys@transformshift{1.364487in}{3.801389in}%
\pgfsys@useobject{currentmarker}{}%
\end{pgfscope}%
\end{pgfscope}%
\begin{pgfscope}%
\pgfpathrectangle{\pgfqpoint{0.781944in}{0.552778in}}{\pgfqpoint{3.890972in}{3.248611in}}%
\pgfusepath{clip}%
\pgfsetrectcap%
\pgfsetroundjoin%
\pgfsetlinewidth{0.803000pt}%
\definecolor{currentstroke}{rgb}{0.690196,0.690196,0.690196}%
\pgfsetstrokecolor{currentstroke}%
\pgfsetstrokeopacity{0.300000}%
\pgfsetdash{}{0pt}%
\pgfpathmoveto{\pgfqpoint{1.403410in}{0.552778in}}%
\pgfpathlineto{\pgfqpoint{1.403410in}{3.801389in}}%
\pgfusepath{stroke}%
\end{pgfscope}%
\begin{pgfscope}%
\pgfsetbuttcap%
\pgfsetroundjoin%
\definecolor{currentfill}{rgb}{0.000000,0.000000,0.000000}%
\pgfsetfillcolor{currentfill}%
\pgfsetlinewidth{0.602250pt}%
\definecolor{currentstroke}{rgb}{0.000000,0.000000,0.000000}%
\pgfsetstrokecolor{currentstroke}%
\pgfsetdash{}{0pt}%
\pgfsys@defobject{currentmarker}{\pgfqpoint{0.000000in}{-0.027778in}}{\pgfqpoint{0.000000in}{0.000000in}}{%
\pgfpathmoveto{\pgfqpoint{0.000000in}{0.000000in}}%
\pgfpathlineto{\pgfqpoint{0.000000in}{-0.027778in}}%
\pgfusepath{stroke,fill}%
}%
\begin{pgfscope}%
\pgfsys@transformshift{1.403410in}{0.552778in}%
\pgfsys@useobject{currentmarker}{}%
\end{pgfscope}%
\end{pgfscope}%
\begin{pgfscope}%
\pgfsetbuttcap%
\pgfsetroundjoin%
\definecolor{currentfill}{rgb}{0.000000,0.000000,0.000000}%
\pgfsetfillcolor{currentfill}%
\pgfsetlinewidth{0.602250pt}%
\definecolor{currentstroke}{rgb}{0.000000,0.000000,0.000000}%
\pgfsetstrokecolor{currentstroke}%
\pgfsetdash{}{0pt}%
\pgfsys@defobject{currentmarker}{\pgfqpoint{0.000000in}{0.000000in}}{\pgfqpoint{0.000000in}{0.027778in}}{%
\pgfpathmoveto{\pgfqpoint{0.000000in}{0.000000in}}%
\pgfpathlineto{\pgfqpoint{0.000000in}{0.027778in}}%
\pgfusepath{stroke,fill}%
}%
\begin{pgfscope}%
\pgfsys@transformshift{1.403410in}{3.801389in}%
\pgfsys@useobject{currentmarker}{}%
\end{pgfscope}%
\end{pgfscope}%
\begin{pgfscope}%
\pgfpathrectangle{\pgfqpoint{0.781944in}{0.552778in}}{\pgfqpoint{3.890972in}{3.248611in}}%
\pgfusepath{clip}%
\pgfsetrectcap%
\pgfsetroundjoin%
\pgfsetlinewidth{0.803000pt}%
\definecolor{currentstroke}{rgb}{0.690196,0.690196,0.690196}%
\pgfsetstrokecolor{currentstroke}%
\pgfsetstrokeopacity{0.300000}%
\pgfsetdash{}{0pt}%
\pgfpathmoveto{\pgfqpoint{1.442333in}{0.552778in}}%
\pgfpathlineto{\pgfqpoint{1.442333in}{3.801389in}}%
\pgfusepath{stroke}%
\end{pgfscope}%
\begin{pgfscope}%
\pgfsetbuttcap%
\pgfsetroundjoin%
\definecolor{currentfill}{rgb}{0.000000,0.000000,0.000000}%
\pgfsetfillcolor{currentfill}%
\pgfsetlinewidth{0.602250pt}%
\definecolor{currentstroke}{rgb}{0.000000,0.000000,0.000000}%
\pgfsetstrokecolor{currentstroke}%
\pgfsetdash{}{0pt}%
\pgfsys@defobject{currentmarker}{\pgfqpoint{0.000000in}{-0.027778in}}{\pgfqpoint{0.000000in}{0.000000in}}{%
\pgfpathmoveto{\pgfqpoint{0.000000in}{0.000000in}}%
\pgfpathlineto{\pgfqpoint{0.000000in}{-0.027778in}}%
\pgfusepath{stroke,fill}%
}%
\begin{pgfscope}%
\pgfsys@transformshift{1.442333in}{0.552778in}%
\pgfsys@useobject{currentmarker}{}%
\end{pgfscope}%
\end{pgfscope}%
\begin{pgfscope}%
\pgfsetbuttcap%
\pgfsetroundjoin%
\definecolor{currentfill}{rgb}{0.000000,0.000000,0.000000}%
\pgfsetfillcolor{currentfill}%
\pgfsetlinewidth{0.602250pt}%
\definecolor{currentstroke}{rgb}{0.000000,0.000000,0.000000}%
\pgfsetstrokecolor{currentstroke}%
\pgfsetdash{}{0pt}%
\pgfsys@defobject{currentmarker}{\pgfqpoint{0.000000in}{0.000000in}}{\pgfqpoint{0.000000in}{0.027778in}}{%
\pgfpathmoveto{\pgfqpoint{0.000000in}{0.000000in}}%
\pgfpathlineto{\pgfqpoint{0.000000in}{0.027778in}}%
\pgfusepath{stroke,fill}%
}%
\begin{pgfscope}%
\pgfsys@transformshift{1.442333in}{3.801389in}%
\pgfsys@useobject{currentmarker}{}%
\end{pgfscope}%
\end{pgfscope}%
\begin{pgfscope}%
\pgfpathrectangle{\pgfqpoint{0.781944in}{0.552778in}}{\pgfqpoint{3.890972in}{3.248611in}}%
\pgfusepath{clip}%
\pgfsetrectcap%
\pgfsetroundjoin%
\pgfsetlinewidth{0.803000pt}%
\definecolor{currentstroke}{rgb}{0.690196,0.690196,0.690196}%
\pgfsetstrokecolor{currentstroke}%
\pgfsetstrokeopacity{0.300000}%
\pgfsetdash{}{0pt}%
\pgfpathmoveto{\pgfqpoint{1.481256in}{0.552778in}}%
\pgfpathlineto{\pgfqpoint{1.481256in}{3.801389in}}%
\pgfusepath{stroke}%
\end{pgfscope}%
\begin{pgfscope}%
\pgfsetbuttcap%
\pgfsetroundjoin%
\definecolor{currentfill}{rgb}{0.000000,0.000000,0.000000}%
\pgfsetfillcolor{currentfill}%
\pgfsetlinewidth{0.602250pt}%
\definecolor{currentstroke}{rgb}{0.000000,0.000000,0.000000}%
\pgfsetstrokecolor{currentstroke}%
\pgfsetdash{}{0pt}%
\pgfsys@defobject{currentmarker}{\pgfqpoint{0.000000in}{-0.027778in}}{\pgfqpoint{0.000000in}{0.000000in}}{%
\pgfpathmoveto{\pgfqpoint{0.000000in}{0.000000in}}%
\pgfpathlineto{\pgfqpoint{0.000000in}{-0.027778in}}%
\pgfusepath{stroke,fill}%
}%
\begin{pgfscope}%
\pgfsys@transformshift{1.481256in}{0.552778in}%
\pgfsys@useobject{currentmarker}{}%
\end{pgfscope}%
\end{pgfscope}%
\begin{pgfscope}%
\pgfsetbuttcap%
\pgfsetroundjoin%
\definecolor{currentfill}{rgb}{0.000000,0.000000,0.000000}%
\pgfsetfillcolor{currentfill}%
\pgfsetlinewidth{0.602250pt}%
\definecolor{currentstroke}{rgb}{0.000000,0.000000,0.000000}%
\pgfsetstrokecolor{currentstroke}%
\pgfsetdash{}{0pt}%
\pgfsys@defobject{currentmarker}{\pgfqpoint{0.000000in}{0.000000in}}{\pgfqpoint{0.000000in}{0.027778in}}{%
\pgfpathmoveto{\pgfqpoint{0.000000in}{0.000000in}}%
\pgfpathlineto{\pgfqpoint{0.000000in}{0.027778in}}%
\pgfusepath{stroke,fill}%
}%
\begin{pgfscope}%
\pgfsys@transformshift{1.481256in}{3.801389in}%
\pgfsys@useobject{currentmarker}{}%
\end{pgfscope}%
\end{pgfscope}%
\begin{pgfscope}%
\pgfpathrectangle{\pgfqpoint{0.781944in}{0.552778in}}{\pgfqpoint{3.890972in}{3.248611in}}%
\pgfusepath{clip}%
\pgfsetrectcap%
\pgfsetroundjoin%
\pgfsetlinewidth{0.803000pt}%
\definecolor{currentstroke}{rgb}{0.690196,0.690196,0.690196}%
\pgfsetstrokecolor{currentstroke}%
\pgfsetstrokeopacity{0.300000}%
\pgfsetdash{}{0pt}%
\pgfpathmoveto{\pgfqpoint{1.520178in}{0.552778in}}%
\pgfpathlineto{\pgfqpoint{1.520178in}{3.801389in}}%
\pgfusepath{stroke}%
\end{pgfscope}%
\begin{pgfscope}%
\pgfsetbuttcap%
\pgfsetroundjoin%
\definecolor{currentfill}{rgb}{0.000000,0.000000,0.000000}%
\pgfsetfillcolor{currentfill}%
\pgfsetlinewidth{0.602250pt}%
\definecolor{currentstroke}{rgb}{0.000000,0.000000,0.000000}%
\pgfsetstrokecolor{currentstroke}%
\pgfsetdash{}{0pt}%
\pgfsys@defobject{currentmarker}{\pgfqpoint{0.000000in}{-0.027778in}}{\pgfqpoint{0.000000in}{0.000000in}}{%
\pgfpathmoveto{\pgfqpoint{0.000000in}{0.000000in}}%
\pgfpathlineto{\pgfqpoint{0.000000in}{-0.027778in}}%
\pgfusepath{stroke,fill}%
}%
\begin{pgfscope}%
\pgfsys@transformshift{1.520178in}{0.552778in}%
\pgfsys@useobject{currentmarker}{}%
\end{pgfscope}%
\end{pgfscope}%
\begin{pgfscope}%
\pgfsetbuttcap%
\pgfsetroundjoin%
\definecolor{currentfill}{rgb}{0.000000,0.000000,0.000000}%
\pgfsetfillcolor{currentfill}%
\pgfsetlinewidth{0.602250pt}%
\definecolor{currentstroke}{rgb}{0.000000,0.000000,0.000000}%
\pgfsetstrokecolor{currentstroke}%
\pgfsetdash{}{0pt}%
\pgfsys@defobject{currentmarker}{\pgfqpoint{0.000000in}{0.000000in}}{\pgfqpoint{0.000000in}{0.027778in}}{%
\pgfpathmoveto{\pgfqpoint{0.000000in}{0.000000in}}%
\pgfpathlineto{\pgfqpoint{0.000000in}{0.027778in}}%
\pgfusepath{stroke,fill}%
}%
\begin{pgfscope}%
\pgfsys@transformshift{1.520178in}{3.801389in}%
\pgfsys@useobject{currentmarker}{}%
\end{pgfscope}%
\end{pgfscope}%
\begin{pgfscope}%
\pgfpathrectangle{\pgfqpoint{0.781944in}{0.552778in}}{\pgfqpoint{3.890972in}{3.248611in}}%
\pgfusepath{clip}%
\pgfsetrectcap%
\pgfsetroundjoin%
\pgfsetlinewidth{0.803000pt}%
\definecolor{currentstroke}{rgb}{0.690196,0.690196,0.690196}%
\pgfsetstrokecolor{currentstroke}%
\pgfsetstrokeopacity{0.300000}%
\pgfsetdash{}{0pt}%
\pgfpathmoveto{\pgfqpoint{1.598024in}{0.552778in}}%
\pgfpathlineto{\pgfqpoint{1.598024in}{3.801389in}}%
\pgfusepath{stroke}%
\end{pgfscope}%
\begin{pgfscope}%
\pgfsetbuttcap%
\pgfsetroundjoin%
\definecolor{currentfill}{rgb}{0.000000,0.000000,0.000000}%
\pgfsetfillcolor{currentfill}%
\pgfsetlinewidth{0.602250pt}%
\definecolor{currentstroke}{rgb}{0.000000,0.000000,0.000000}%
\pgfsetstrokecolor{currentstroke}%
\pgfsetdash{}{0pt}%
\pgfsys@defobject{currentmarker}{\pgfqpoint{0.000000in}{-0.027778in}}{\pgfqpoint{0.000000in}{0.000000in}}{%
\pgfpathmoveto{\pgfqpoint{0.000000in}{0.000000in}}%
\pgfpathlineto{\pgfqpoint{0.000000in}{-0.027778in}}%
\pgfusepath{stroke,fill}%
}%
\begin{pgfscope}%
\pgfsys@transformshift{1.598024in}{0.552778in}%
\pgfsys@useobject{currentmarker}{}%
\end{pgfscope}%
\end{pgfscope}%
\begin{pgfscope}%
\pgfsetbuttcap%
\pgfsetroundjoin%
\definecolor{currentfill}{rgb}{0.000000,0.000000,0.000000}%
\pgfsetfillcolor{currentfill}%
\pgfsetlinewidth{0.602250pt}%
\definecolor{currentstroke}{rgb}{0.000000,0.000000,0.000000}%
\pgfsetstrokecolor{currentstroke}%
\pgfsetdash{}{0pt}%
\pgfsys@defobject{currentmarker}{\pgfqpoint{0.000000in}{0.000000in}}{\pgfqpoint{0.000000in}{0.027778in}}{%
\pgfpathmoveto{\pgfqpoint{0.000000in}{0.000000in}}%
\pgfpathlineto{\pgfqpoint{0.000000in}{0.027778in}}%
\pgfusepath{stroke,fill}%
}%
\begin{pgfscope}%
\pgfsys@transformshift{1.598024in}{3.801389in}%
\pgfsys@useobject{currentmarker}{}%
\end{pgfscope}%
\end{pgfscope}%
\begin{pgfscope}%
\pgfpathrectangle{\pgfqpoint{0.781944in}{0.552778in}}{\pgfqpoint{3.890972in}{3.248611in}}%
\pgfusepath{clip}%
\pgfsetrectcap%
\pgfsetroundjoin%
\pgfsetlinewidth{0.803000pt}%
\definecolor{currentstroke}{rgb}{0.690196,0.690196,0.690196}%
\pgfsetstrokecolor{currentstroke}%
\pgfsetstrokeopacity{0.300000}%
\pgfsetdash{}{0pt}%
\pgfpathmoveto{\pgfqpoint{1.636946in}{0.552778in}}%
\pgfpathlineto{\pgfqpoint{1.636946in}{3.801389in}}%
\pgfusepath{stroke}%
\end{pgfscope}%
\begin{pgfscope}%
\pgfsetbuttcap%
\pgfsetroundjoin%
\definecolor{currentfill}{rgb}{0.000000,0.000000,0.000000}%
\pgfsetfillcolor{currentfill}%
\pgfsetlinewidth{0.602250pt}%
\definecolor{currentstroke}{rgb}{0.000000,0.000000,0.000000}%
\pgfsetstrokecolor{currentstroke}%
\pgfsetdash{}{0pt}%
\pgfsys@defobject{currentmarker}{\pgfqpoint{0.000000in}{-0.027778in}}{\pgfqpoint{0.000000in}{0.000000in}}{%
\pgfpathmoveto{\pgfqpoint{0.000000in}{0.000000in}}%
\pgfpathlineto{\pgfqpoint{0.000000in}{-0.027778in}}%
\pgfusepath{stroke,fill}%
}%
\begin{pgfscope}%
\pgfsys@transformshift{1.636946in}{0.552778in}%
\pgfsys@useobject{currentmarker}{}%
\end{pgfscope}%
\end{pgfscope}%
\begin{pgfscope}%
\pgfsetbuttcap%
\pgfsetroundjoin%
\definecolor{currentfill}{rgb}{0.000000,0.000000,0.000000}%
\pgfsetfillcolor{currentfill}%
\pgfsetlinewidth{0.602250pt}%
\definecolor{currentstroke}{rgb}{0.000000,0.000000,0.000000}%
\pgfsetstrokecolor{currentstroke}%
\pgfsetdash{}{0pt}%
\pgfsys@defobject{currentmarker}{\pgfqpoint{0.000000in}{0.000000in}}{\pgfqpoint{0.000000in}{0.027778in}}{%
\pgfpathmoveto{\pgfqpoint{0.000000in}{0.000000in}}%
\pgfpathlineto{\pgfqpoint{0.000000in}{0.027778in}}%
\pgfusepath{stroke,fill}%
}%
\begin{pgfscope}%
\pgfsys@transformshift{1.636946in}{3.801389in}%
\pgfsys@useobject{currentmarker}{}%
\end{pgfscope}%
\end{pgfscope}%
\begin{pgfscope}%
\pgfpathrectangle{\pgfqpoint{0.781944in}{0.552778in}}{\pgfqpoint{3.890972in}{3.248611in}}%
\pgfusepath{clip}%
\pgfsetrectcap%
\pgfsetroundjoin%
\pgfsetlinewidth{0.803000pt}%
\definecolor{currentstroke}{rgb}{0.690196,0.690196,0.690196}%
\pgfsetstrokecolor{currentstroke}%
\pgfsetstrokeopacity{0.300000}%
\pgfsetdash{}{0pt}%
\pgfpathmoveto{\pgfqpoint{1.675869in}{0.552778in}}%
\pgfpathlineto{\pgfqpoint{1.675869in}{3.801389in}}%
\pgfusepath{stroke}%
\end{pgfscope}%
\begin{pgfscope}%
\pgfsetbuttcap%
\pgfsetroundjoin%
\definecolor{currentfill}{rgb}{0.000000,0.000000,0.000000}%
\pgfsetfillcolor{currentfill}%
\pgfsetlinewidth{0.602250pt}%
\definecolor{currentstroke}{rgb}{0.000000,0.000000,0.000000}%
\pgfsetstrokecolor{currentstroke}%
\pgfsetdash{}{0pt}%
\pgfsys@defobject{currentmarker}{\pgfqpoint{0.000000in}{-0.027778in}}{\pgfqpoint{0.000000in}{0.000000in}}{%
\pgfpathmoveto{\pgfqpoint{0.000000in}{0.000000in}}%
\pgfpathlineto{\pgfqpoint{0.000000in}{-0.027778in}}%
\pgfusepath{stroke,fill}%
}%
\begin{pgfscope}%
\pgfsys@transformshift{1.675869in}{0.552778in}%
\pgfsys@useobject{currentmarker}{}%
\end{pgfscope}%
\end{pgfscope}%
\begin{pgfscope}%
\pgfsetbuttcap%
\pgfsetroundjoin%
\definecolor{currentfill}{rgb}{0.000000,0.000000,0.000000}%
\pgfsetfillcolor{currentfill}%
\pgfsetlinewidth{0.602250pt}%
\definecolor{currentstroke}{rgb}{0.000000,0.000000,0.000000}%
\pgfsetstrokecolor{currentstroke}%
\pgfsetdash{}{0pt}%
\pgfsys@defobject{currentmarker}{\pgfqpoint{0.000000in}{0.000000in}}{\pgfqpoint{0.000000in}{0.027778in}}{%
\pgfpathmoveto{\pgfqpoint{0.000000in}{0.000000in}}%
\pgfpathlineto{\pgfqpoint{0.000000in}{0.027778in}}%
\pgfusepath{stroke,fill}%
}%
\begin{pgfscope}%
\pgfsys@transformshift{1.675869in}{3.801389in}%
\pgfsys@useobject{currentmarker}{}%
\end{pgfscope}%
\end{pgfscope}%
\begin{pgfscope}%
\pgfpathrectangle{\pgfqpoint{0.781944in}{0.552778in}}{\pgfqpoint{3.890972in}{3.248611in}}%
\pgfusepath{clip}%
\pgfsetrectcap%
\pgfsetroundjoin%
\pgfsetlinewidth{0.803000pt}%
\definecolor{currentstroke}{rgb}{0.690196,0.690196,0.690196}%
\pgfsetstrokecolor{currentstroke}%
\pgfsetstrokeopacity{0.300000}%
\pgfsetdash{}{0pt}%
\pgfpathmoveto{\pgfqpoint{1.714792in}{0.552778in}}%
\pgfpathlineto{\pgfqpoint{1.714792in}{3.801389in}}%
\pgfusepath{stroke}%
\end{pgfscope}%
\begin{pgfscope}%
\pgfsetbuttcap%
\pgfsetroundjoin%
\definecolor{currentfill}{rgb}{0.000000,0.000000,0.000000}%
\pgfsetfillcolor{currentfill}%
\pgfsetlinewidth{0.602250pt}%
\definecolor{currentstroke}{rgb}{0.000000,0.000000,0.000000}%
\pgfsetstrokecolor{currentstroke}%
\pgfsetdash{}{0pt}%
\pgfsys@defobject{currentmarker}{\pgfqpoint{0.000000in}{-0.027778in}}{\pgfqpoint{0.000000in}{0.000000in}}{%
\pgfpathmoveto{\pgfqpoint{0.000000in}{0.000000in}}%
\pgfpathlineto{\pgfqpoint{0.000000in}{-0.027778in}}%
\pgfusepath{stroke,fill}%
}%
\begin{pgfscope}%
\pgfsys@transformshift{1.714792in}{0.552778in}%
\pgfsys@useobject{currentmarker}{}%
\end{pgfscope}%
\end{pgfscope}%
\begin{pgfscope}%
\pgfsetbuttcap%
\pgfsetroundjoin%
\definecolor{currentfill}{rgb}{0.000000,0.000000,0.000000}%
\pgfsetfillcolor{currentfill}%
\pgfsetlinewidth{0.602250pt}%
\definecolor{currentstroke}{rgb}{0.000000,0.000000,0.000000}%
\pgfsetstrokecolor{currentstroke}%
\pgfsetdash{}{0pt}%
\pgfsys@defobject{currentmarker}{\pgfqpoint{0.000000in}{0.000000in}}{\pgfqpoint{0.000000in}{0.027778in}}{%
\pgfpathmoveto{\pgfqpoint{0.000000in}{0.000000in}}%
\pgfpathlineto{\pgfqpoint{0.000000in}{0.027778in}}%
\pgfusepath{stroke,fill}%
}%
\begin{pgfscope}%
\pgfsys@transformshift{1.714792in}{3.801389in}%
\pgfsys@useobject{currentmarker}{}%
\end{pgfscope}%
\end{pgfscope}%
\begin{pgfscope}%
\pgfpathrectangle{\pgfqpoint{0.781944in}{0.552778in}}{\pgfqpoint{3.890972in}{3.248611in}}%
\pgfusepath{clip}%
\pgfsetrectcap%
\pgfsetroundjoin%
\pgfsetlinewidth{0.803000pt}%
\definecolor{currentstroke}{rgb}{0.690196,0.690196,0.690196}%
\pgfsetstrokecolor{currentstroke}%
\pgfsetstrokeopacity{0.300000}%
\pgfsetdash{}{0pt}%
\pgfpathmoveto{\pgfqpoint{1.753714in}{0.552778in}}%
\pgfpathlineto{\pgfqpoint{1.753714in}{3.801389in}}%
\pgfusepath{stroke}%
\end{pgfscope}%
\begin{pgfscope}%
\pgfsetbuttcap%
\pgfsetroundjoin%
\definecolor{currentfill}{rgb}{0.000000,0.000000,0.000000}%
\pgfsetfillcolor{currentfill}%
\pgfsetlinewidth{0.602250pt}%
\definecolor{currentstroke}{rgb}{0.000000,0.000000,0.000000}%
\pgfsetstrokecolor{currentstroke}%
\pgfsetdash{}{0pt}%
\pgfsys@defobject{currentmarker}{\pgfqpoint{0.000000in}{-0.027778in}}{\pgfqpoint{0.000000in}{0.000000in}}{%
\pgfpathmoveto{\pgfqpoint{0.000000in}{0.000000in}}%
\pgfpathlineto{\pgfqpoint{0.000000in}{-0.027778in}}%
\pgfusepath{stroke,fill}%
}%
\begin{pgfscope}%
\pgfsys@transformshift{1.753714in}{0.552778in}%
\pgfsys@useobject{currentmarker}{}%
\end{pgfscope}%
\end{pgfscope}%
\begin{pgfscope}%
\pgfsetbuttcap%
\pgfsetroundjoin%
\definecolor{currentfill}{rgb}{0.000000,0.000000,0.000000}%
\pgfsetfillcolor{currentfill}%
\pgfsetlinewidth{0.602250pt}%
\definecolor{currentstroke}{rgb}{0.000000,0.000000,0.000000}%
\pgfsetstrokecolor{currentstroke}%
\pgfsetdash{}{0pt}%
\pgfsys@defobject{currentmarker}{\pgfqpoint{0.000000in}{0.000000in}}{\pgfqpoint{0.000000in}{0.027778in}}{%
\pgfpathmoveto{\pgfqpoint{0.000000in}{0.000000in}}%
\pgfpathlineto{\pgfqpoint{0.000000in}{0.027778in}}%
\pgfusepath{stroke,fill}%
}%
\begin{pgfscope}%
\pgfsys@transformshift{1.753714in}{3.801389in}%
\pgfsys@useobject{currentmarker}{}%
\end{pgfscope}%
\end{pgfscope}%
\begin{pgfscope}%
\pgfpathrectangle{\pgfqpoint{0.781944in}{0.552778in}}{\pgfqpoint{3.890972in}{3.248611in}}%
\pgfusepath{clip}%
\pgfsetrectcap%
\pgfsetroundjoin%
\pgfsetlinewidth{0.803000pt}%
\definecolor{currentstroke}{rgb}{0.690196,0.690196,0.690196}%
\pgfsetstrokecolor{currentstroke}%
\pgfsetstrokeopacity{0.300000}%
\pgfsetdash{}{0pt}%
\pgfpathmoveto{\pgfqpoint{1.792637in}{0.552778in}}%
\pgfpathlineto{\pgfqpoint{1.792637in}{3.801389in}}%
\pgfusepath{stroke}%
\end{pgfscope}%
\begin{pgfscope}%
\pgfsetbuttcap%
\pgfsetroundjoin%
\definecolor{currentfill}{rgb}{0.000000,0.000000,0.000000}%
\pgfsetfillcolor{currentfill}%
\pgfsetlinewidth{0.602250pt}%
\definecolor{currentstroke}{rgb}{0.000000,0.000000,0.000000}%
\pgfsetstrokecolor{currentstroke}%
\pgfsetdash{}{0pt}%
\pgfsys@defobject{currentmarker}{\pgfqpoint{0.000000in}{-0.027778in}}{\pgfqpoint{0.000000in}{0.000000in}}{%
\pgfpathmoveto{\pgfqpoint{0.000000in}{0.000000in}}%
\pgfpathlineto{\pgfqpoint{0.000000in}{-0.027778in}}%
\pgfusepath{stroke,fill}%
}%
\begin{pgfscope}%
\pgfsys@transformshift{1.792637in}{0.552778in}%
\pgfsys@useobject{currentmarker}{}%
\end{pgfscope}%
\end{pgfscope}%
\begin{pgfscope}%
\pgfsetbuttcap%
\pgfsetroundjoin%
\definecolor{currentfill}{rgb}{0.000000,0.000000,0.000000}%
\pgfsetfillcolor{currentfill}%
\pgfsetlinewidth{0.602250pt}%
\definecolor{currentstroke}{rgb}{0.000000,0.000000,0.000000}%
\pgfsetstrokecolor{currentstroke}%
\pgfsetdash{}{0pt}%
\pgfsys@defobject{currentmarker}{\pgfqpoint{0.000000in}{0.000000in}}{\pgfqpoint{0.000000in}{0.027778in}}{%
\pgfpathmoveto{\pgfqpoint{0.000000in}{0.000000in}}%
\pgfpathlineto{\pgfqpoint{0.000000in}{0.027778in}}%
\pgfusepath{stroke,fill}%
}%
\begin{pgfscope}%
\pgfsys@transformshift{1.792637in}{3.801389in}%
\pgfsys@useobject{currentmarker}{}%
\end{pgfscope}%
\end{pgfscope}%
\begin{pgfscope}%
\pgfpathrectangle{\pgfqpoint{0.781944in}{0.552778in}}{\pgfqpoint{3.890972in}{3.248611in}}%
\pgfusepath{clip}%
\pgfsetrectcap%
\pgfsetroundjoin%
\pgfsetlinewidth{0.803000pt}%
\definecolor{currentstroke}{rgb}{0.690196,0.690196,0.690196}%
\pgfsetstrokecolor{currentstroke}%
\pgfsetstrokeopacity{0.300000}%
\pgfsetdash{}{0pt}%
\pgfpathmoveto{\pgfqpoint{1.831560in}{0.552778in}}%
\pgfpathlineto{\pgfqpoint{1.831560in}{3.801389in}}%
\pgfusepath{stroke}%
\end{pgfscope}%
\begin{pgfscope}%
\pgfsetbuttcap%
\pgfsetroundjoin%
\definecolor{currentfill}{rgb}{0.000000,0.000000,0.000000}%
\pgfsetfillcolor{currentfill}%
\pgfsetlinewidth{0.602250pt}%
\definecolor{currentstroke}{rgb}{0.000000,0.000000,0.000000}%
\pgfsetstrokecolor{currentstroke}%
\pgfsetdash{}{0pt}%
\pgfsys@defobject{currentmarker}{\pgfqpoint{0.000000in}{-0.027778in}}{\pgfqpoint{0.000000in}{0.000000in}}{%
\pgfpathmoveto{\pgfqpoint{0.000000in}{0.000000in}}%
\pgfpathlineto{\pgfqpoint{0.000000in}{-0.027778in}}%
\pgfusepath{stroke,fill}%
}%
\begin{pgfscope}%
\pgfsys@transformshift{1.831560in}{0.552778in}%
\pgfsys@useobject{currentmarker}{}%
\end{pgfscope}%
\end{pgfscope}%
\begin{pgfscope}%
\pgfsetbuttcap%
\pgfsetroundjoin%
\definecolor{currentfill}{rgb}{0.000000,0.000000,0.000000}%
\pgfsetfillcolor{currentfill}%
\pgfsetlinewidth{0.602250pt}%
\definecolor{currentstroke}{rgb}{0.000000,0.000000,0.000000}%
\pgfsetstrokecolor{currentstroke}%
\pgfsetdash{}{0pt}%
\pgfsys@defobject{currentmarker}{\pgfqpoint{0.000000in}{0.000000in}}{\pgfqpoint{0.000000in}{0.027778in}}{%
\pgfpathmoveto{\pgfqpoint{0.000000in}{0.000000in}}%
\pgfpathlineto{\pgfqpoint{0.000000in}{0.027778in}}%
\pgfusepath{stroke,fill}%
}%
\begin{pgfscope}%
\pgfsys@transformshift{1.831560in}{3.801389in}%
\pgfsys@useobject{currentmarker}{}%
\end{pgfscope}%
\end{pgfscope}%
\begin{pgfscope}%
\pgfpathrectangle{\pgfqpoint{0.781944in}{0.552778in}}{\pgfqpoint{3.890972in}{3.248611in}}%
\pgfusepath{clip}%
\pgfsetrectcap%
\pgfsetroundjoin%
\pgfsetlinewidth{0.803000pt}%
\definecolor{currentstroke}{rgb}{0.690196,0.690196,0.690196}%
\pgfsetstrokecolor{currentstroke}%
\pgfsetstrokeopacity{0.300000}%
\pgfsetdash{}{0pt}%
\pgfpathmoveto{\pgfqpoint{1.870483in}{0.552778in}}%
\pgfpathlineto{\pgfqpoint{1.870483in}{3.801389in}}%
\pgfusepath{stroke}%
\end{pgfscope}%
\begin{pgfscope}%
\pgfsetbuttcap%
\pgfsetroundjoin%
\definecolor{currentfill}{rgb}{0.000000,0.000000,0.000000}%
\pgfsetfillcolor{currentfill}%
\pgfsetlinewidth{0.602250pt}%
\definecolor{currentstroke}{rgb}{0.000000,0.000000,0.000000}%
\pgfsetstrokecolor{currentstroke}%
\pgfsetdash{}{0pt}%
\pgfsys@defobject{currentmarker}{\pgfqpoint{0.000000in}{-0.027778in}}{\pgfqpoint{0.000000in}{0.000000in}}{%
\pgfpathmoveto{\pgfqpoint{0.000000in}{0.000000in}}%
\pgfpathlineto{\pgfqpoint{0.000000in}{-0.027778in}}%
\pgfusepath{stroke,fill}%
}%
\begin{pgfscope}%
\pgfsys@transformshift{1.870483in}{0.552778in}%
\pgfsys@useobject{currentmarker}{}%
\end{pgfscope}%
\end{pgfscope}%
\begin{pgfscope}%
\pgfsetbuttcap%
\pgfsetroundjoin%
\definecolor{currentfill}{rgb}{0.000000,0.000000,0.000000}%
\pgfsetfillcolor{currentfill}%
\pgfsetlinewidth{0.602250pt}%
\definecolor{currentstroke}{rgb}{0.000000,0.000000,0.000000}%
\pgfsetstrokecolor{currentstroke}%
\pgfsetdash{}{0pt}%
\pgfsys@defobject{currentmarker}{\pgfqpoint{0.000000in}{0.000000in}}{\pgfqpoint{0.000000in}{0.027778in}}{%
\pgfpathmoveto{\pgfqpoint{0.000000in}{0.000000in}}%
\pgfpathlineto{\pgfqpoint{0.000000in}{0.027778in}}%
\pgfusepath{stroke,fill}%
}%
\begin{pgfscope}%
\pgfsys@transformshift{1.870483in}{3.801389in}%
\pgfsys@useobject{currentmarker}{}%
\end{pgfscope}%
\end{pgfscope}%
\begin{pgfscope}%
\pgfpathrectangle{\pgfqpoint{0.781944in}{0.552778in}}{\pgfqpoint{3.890972in}{3.248611in}}%
\pgfusepath{clip}%
\pgfsetrectcap%
\pgfsetroundjoin%
\pgfsetlinewidth{0.803000pt}%
\definecolor{currentstroke}{rgb}{0.690196,0.690196,0.690196}%
\pgfsetstrokecolor{currentstroke}%
\pgfsetstrokeopacity{0.300000}%
\pgfsetdash{}{0pt}%
\pgfpathmoveto{\pgfqpoint{1.909405in}{0.552778in}}%
\pgfpathlineto{\pgfqpoint{1.909405in}{3.801389in}}%
\pgfusepath{stroke}%
\end{pgfscope}%
\begin{pgfscope}%
\pgfsetbuttcap%
\pgfsetroundjoin%
\definecolor{currentfill}{rgb}{0.000000,0.000000,0.000000}%
\pgfsetfillcolor{currentfill}%
\pgfsetlinewidth{0.602250pt}%
\definecolor{currentstroke}{rgb}{0.000000,0.000000,0.000000}%
\pgfsetstrokecolor{currentstroke}%
\pgfsetdash{}{0pt}%
\pgfsys@defobject{currentmarker}{\pgfqpoint{0.000000in}{-0.027778in}}{\pgfqpoint{0.000000in}{0.000000in}}{%
\pgfpathmoveto{\pgfqpoint{0.000000in}{0.000000in}}%
\pgfpathlineto{\pgfqpoint{0.000000in}{-0.027778in}}%
\pgfusepath{stroke,fill}%
}%
\begin{pgfscope}%
\pgfsys@transformshift{1.909405in}{0.552778in}%
\pgfsys@useobject{currentmarker}{}%
\end{pgfscope}%
\end{pgfscope}%
\begin{pgfscope}%
\pgfsetbuttcap%
\pgfsetroundjoin%
\definecolor{currentfill}{rgb}{0.000000,0.000000,0.000000}%
\pgfsetfillcolor{currentfill}%
\pgfsetlinewidth{0.602250pt}%
\definecolor{currentstroke}{rgb}{0.000000,0.000000,0.000000}%
\pgfsetstrokecolor{currentstroke}%
\pgfsetdash{}{0pt}%
\pgfsys@defobject{currentmarker}{\pgfqpoint{0.000000in}{0.000000in}}{\pgfqpoint{0.000000in}{0.027778in}}{%
\pgfpathmoveto{\pgfqpoint{0.000000in}{0.000000in}}%
\pgfpathlineto{\pgfqpoint{0.000000in}{0.027778in}}%
\pgfusepath{stroke,fill}%
}%
\begin{pgfscope}%
\pgfsys@transformshift{1.909405in}{3.801389in}%
\pgfsys@useobject{currentmarker}{}%
\end{pgfscope}%
\end{pgfscope}%
\begin{pgfscope}%
\pgfpathrectangle{\pgfqpoint{0.781944in}{0.552778in}}{\pgfqpoint{3.890972in}{3.248611in}}%
\pgfusepath{clip}%
\pgfsetrectcap%
\pgfsetroundjoin%
\pgfsetlinewidth{0.803000pt}%
\definecolor{currentstroke}{rgb}{0.690196,0.690196,0.690196}%
\pgfsetstrokecolor{currentstroke}%
\pgfsetstrokeopacity{0.300000}%
\pgfsetdash{}{0pt}%
\pgfpathmoveto{\pgfqpoint{1.987251in}{0.552778in}}%
\pgfpathlineto{\pgfqpoint{1.987251in}{3.801389in}}%
\pgfusepath{stroke}%
\end{pgfscope}%
\begin{pgfscope}%
\pgfsetbuttcap%
\pgfsetroundjoin%
\definecolor{currentfill}{rgb}{0.000000,0.000000,0.000000}%
\pgfsetfillcolor{currentfill}%
\pgfsetlinewidth{0.602250pt}%
\definecolor{currentstroke}{rgb}{0.000000,0.000000,0.000000}%
\pgfsetstrokecolor{currentstroke}%
\pgfsetdash{}{0pt}%
\pgfsys@defobject{currentmarker}{\pgfqpoint{0.000000in}{-0.027778in}}{\pgfqpoint{0.000000in}{0.000000in}}{%
\pgfpathmoveto{\pgfqpoint{0.000000in}{0.000000in}}%
\pgfpathlineto{\pgfqpoint{0.000000in}{-0.027778in}}%
\pgfusepath{stroke,fill}%
}%
\begin{pgfscope}%
\pgfsys@transformshift{1.987251in}{0.552778in}%
\pgfsys@useobject{currentmarker}{}%
\end{pgfscope}%
\end{pgfscope}%
\begin{pgfscope}%
\pgfsetbuttcap%
\pgfsetroundjoin%
\definecolor{currentfill}{rgb}{0.000000,0.000000,0.000000}%
\pgfsetfillcolor{currentfill}%
\pgfsetlinewidth{0.602250pt}%
\definecolor{currentstroke}{rgb}{0.000000,0.000000,0.000000}%
\pgfsetstrokecolor{currentstroke}%
\pgfsetdash{}{0pt}%
\pgfsys@defobject{currentmarker}{\pgfqpoint{0.000000in}{0.000000in}}{\pgfqpoint{0.000000in}{0.027778in}}{%
\pgfpathmoveto{\pgfqpoint{0.000000in}{0.000000in}}%
\pgfpathlineto{\pgfqpoint{0.000000in}{0.027778in}}%
\pgfusepath{stroke,fill}%
}%
\begin{pgfscope}%
\pgfsys@transformshift{1.987251in}{3.801389in}%
\pgfsys@useobject{currentmarker}{}%
\end{pgfscope}%
\end{pgfscope}%
\begin{pgfscope}%
\pgfpathrectangle{\pgfqpoint{0.781944in}{0.552778in}}{\pgfqpoint{3.890972in}{3.248611in}}%
\pgfusepath{clip}%
\pgfsetrectcap%
\pgfsetroundjoin%
\pgfsetlinewidth{0.803000pt}%
\definecolor{currentstroke}{rgb}{0.690196,0.690196,0.690196}%
\pgfsetstrokecolor{currentstroke}%
\pgfsetstrokeopacity{0.300000}%
\pgfsetdash{}{0pt}%
\pgfpathmoveto{\pgfqpoint{2.026173in}{0.552778in}}%
\pgfpathlineto{\pgfqpoint{2.026173in}{3.801389in}}%
\pgfusepath{stroke}%
\end{pgfscope}%
\begin{pgfscope}%
\pgfsetbuttcap%
\pgfsetroundjoin%
\definecolor{currentfill}{rgb}{0.000000,0.000000,0.000000}%
\pgfsetfillcolor{currentfill}%
\pgfsetlinewidth{0.602250pt}%
\definecolor{currentstroke}{rgb}{0.000000,0.000000,0.000000}%
\pgfsetstrokecolor{currentstroke}%
\pgfsetdash{}{0pt}%
\pgfsys@defobject{currentmarker}{\pgfqpoint{0.000000in}{-0.027778in}}{\pgfqpoint{0.000000in}{0.000000in}}{%
\pgfpathmoveto{\pgfqpoint{0.000000in}{0.000000in}}%
\pgfpathlineto{\pgfqpoint{0.000000in}{-0.027778in}}%
\pgfusepath{stroke,fill}%
}%
\begin{pgfscope}%
\pgfsys@transformshift{2.026173in}{0.552778in}%
\pgfsys@useobject{currentmarker}{}%
\end{pgfscope}%
\end{pgfscope}%
\begin{pgfscope}%
\pgfsetbuttcap%
\pgfsetroundjoin%
\definecolor{currentfill}{rgb}{0.000000,0.000000,0.000000}%
\pgfsetfillcolor{currentfill}%
\pgfsetlinewidth{0.602250pt}%
\definecolor{currentstroke}{rgb}{0.000000,0.000000,0.000000}%
\pgfsetstrokecolor{currentstroke}%
\pgfsetdash{}{0pt}%
\pgfsys@defobject{currentmarker}{\pgfqpoint{0.000000in}{0.000000in}}{\pgfqpoint{0.000000in}{0.027778in}}{%
\pgfpathmoveto{\pgfqpoint{0.000000in}{0.000000in}}%
\pgfpathlineto{\pgfqpoint{0.000000in}{0.027778in}}%
\pgfusepath{stroke,fill}%
}%
\begin{pgfscope}%
\pgfsys@transformshift{2.026173in}{3.801389in}%
\pgfsys@useobject{currentmarker}{}%
\end{pgfscope}%
\end{pgfscope}%
\begin{pgfscope}%
\pgfpathrectangle{\pgfqpoint{0.781944in}{0.552778in}}{\pgfqpoint{3.890972in}{3.248611in}}%
\pgfusepath{clip}%
\pgfsetrectcap%
\pgfsetroundjoin%
\pgfsetlinewidth{0.803000pt}%
\definecolor{currentstroke}{rgb}{0.690196,0.690196,0.690196}%
\pgfsetstrokecolor{currentstroke}%
\pgfsetstrokeopacity{0.300000}%
\pgfsetdash{}{0pt}%
\pgfpathmoveto{\pgfqpoint{2.065096in}{0.552778in}}%
\pgfpathlineto{\pgfqpoint{2.065096in}{3.801389in}}%
\pgfusepath{stroke}%
\end{pgfscope}%
\begin{pgfscope}%
\pgfsetbuttcap%
\pgfsetroundjoin%
\definecolor{currentfill}{rgb}{0.000000,0.000000,0.000000}%
\pgfsetfillcolor{currentfill}%
\pgfsetlinewidth{0.602250pt}%
\definecolor{currentstroke}{rgb}{0.000000,0.000000,0.000000}%
\pgfsetstrokecolor{currentstroke}%
\pgfsetdash{}{0pt}%
\pgfsys@defobject{currentmarker}{\pgfqpoint{0.000000in}{-0.027778in}}{\pgfqpoint{0.000000in}{0.000000in}}{%
\pgfpathmoveto{\pgfqpoint{0.000000in}{0.000000in}}%
\pgfpathlineto{\pgfqpoint{0.000000in}{-0.027778in}}%
\pgfusepath{stroke,fill}%
}%
\begin{pgfscope}%
\pgfsys@transformshift{2.065096in}{0.552778in}%
\pgfsys@useobject{currentmarker}{}%
\end{pgfscope}%
\end{pgfscope}%
\begin{pgfscope}%
\pgfsetbuttcap%
\pgfsetroundjoin%
\definecolor{currentfill}{rgb}{0.000000,0.000000,0.000000}%
\pgfsetfillcolor{currentfill}%
\pgfsetlinewidth{0.602250pt}%
\definecolor{currentstroke}{rgb}{0.000000,0.000000,0.000000}%
\pgfsetstrokecolor{currentstroke}%
\pgfsetdash{}{0pt}%
\pgfsys@defobject{currentmarker}{\pgfqpoint{0.000000in}{0.000000in}}{\pgfqpoint{0.000000in}{0.027778in}}{%
\pgfpathmoveto{\pgfqpoint{0.000000in}{0.000000in}}%
\pgfpathlineto{\pgfqpoint{0.000000in}{0.027778in}}%
\pgfusepath{stroke,fill}%
}%
\begin{pgfscope}%
\pgfsys@transformshift{2.065096in}{3.801389in}%
\pgfsys@useobject{currentmarker}{}%
\end{pgfscope}%
\end{pgfscope}%
\begin{pgfscope}%
\pgfpathrectangle{\pgfqpoint{0.781944in}{0.552778in}}{\pgfqpoint{3.890972in}{3.248611in}}%
\pgfusepath{clip}%
\pgfsetrectcap%
\pgfsetroundjoin%
\pgfsetlinewidth{0.803000pt}%
\definecolor{currentstroke}{rgb}{0.690196,0.690196,0.690196}%
\pgfsetstrokecolor{currentstroke}%
\pgfsetstrokeopacity{0.300000}%
\pgfsetdash{}{0pt}%
\pgfpathmoveto{\pgfqpoint{2.104019in}{0.552778in}}%
\pgfpathlineto{\pgfqpoint{2.104019in}{3.801389in}}%
\pgfusepath{stroke}%
\end{pgfscope}%
\begin{pgfscope}%
\pgfsetbuttcap%
\pgfsetroundjoin%
\definecolor{currentfill}{rgb}{0.000000,0.000000,0.000000}%
\pgfsetfillcolor{currentfill}%
\pgfsetlinewidth{0.602250pt}%
\definecolor{currentstroke}{rgb}{0.000000,0.000000,0.000000}%
\pgfsetstrokecolor{currentstroke}%
\pgfsetdash{}{0pt}%
\pgfsys@defobject{currentmarker}{\pgfqpoint{0.000000in}{-0.027778in}}{\pgfqpoint{0.000000in}{0.000000in}}{%
\pgfpathmoveto{\pgfqpoint{0.000000in}{0.000000in}}%
\pgfpathlineto{\pgfqpoint{0.000000in}{-0.027778in}}%
\pgfusepath{stroke,fill}%
}%
\begin{pgfscope}%
\pgfsys@transformshift{2.104019in}{0.552778in}%
\pgfsys@useobject{currentmarker}{}%
\end{pgfscope}%
\end{pgfscope}%
\begin{pgfscope}%
\pgfsetbuttcap%
\pgfsetroundjoin%
\definecolor{currentfill}{rgb}{0.000000,0.000000,0.000000}%
\pgfsetfillcolor{currentfill}%
\pgfsetlinewidth{0.602250pt}%
\definecolor{currentstroke}{rgb}{0.000000,0.000000,0.000000}%
\pgfsetstrokecolor{currentstroke}%
\pgfsetdash{}{0pt}%
\pgfsys@defobject{currentmarker}{\pgfqpoint{0.000000in}{0.000000in}}{\pgfqpoint{0.000000in}{0.027778in}}{%
\pgfpathmoveto{\pgfqpoint{0.000000in}{0.000000in}}%
\pgfpathlineto{\pgfqpoint{0.000000in}{0.027778in}}%
\pgfusepath{stroke,fill}%
}%
\begin{pgfscope}%
\pgfsys@transformshift{2.104019in}{3.801389in}%
\pgfsys@useobject{currentmarker}{}%
\end{pgfscope}%
\end{pgfscope}%
\begin{pgfscope}%
\pgfpathrectangle{\pgfqpoint{0.781944in}{0.552778in}}{\pgfqpoint{3.890972in}{3.248611in}}%
\pgfusepath{clip}%
\pgfsetrectcap%
\pgfsetroundjoin%
\pgfsetlinewidth{0.803000pt}%
\definecolor{currentstroke}{rgb}{0.690196,0.690196,0.690196}%
\pgfsetstrokecolor{currentstroke}%
\pgfsetstrokeopacity{0.300000}%
\pgfsetdash{}{0pt}%
\pgfpathmoveto{\pgfqpoint{2.142941in}{0.552778in}}%
\pgfpathlineto{\pgfqpoint{2.142941in}{3.801389in}}%
\pgfusepath{stroke}%
\end{pgfscope}%
\begin{pgfscope}%
\pgfsetbuttcap%
\pgfsetroundjoin%
\definecolor{currentfill}{rgb}{0.000000,0.000000,0.000000}%
\pgfsetfillcolor{currentfill}%
\pgfsetlinewidth{0.602250pt}%
\definecolor{currentstroke}{rgb}{0.000000,0.000000,0.000000}%
\pgfsetstrokecolor{currentstroke}%
\pgfsetdash{}{0pt}%
\pgfsys@defobject{currentmarker}{\pgfqpoint{0.000000in}{-0.027778in}}{\pgfqpoint{0.000000in}{0.000000in}}{%
\pgfpathmoveto{\pgfqpoint{0.000000in}{0.000000in}}%
\pgfpathlineto{\pgfqpoint{0.000000in}{-0.027778in}}%
\pgfusepath{stroke,fill}%
}%
\begin{pgfscope}%
\pgfsys@transformshift{2.142941in}{0.552778in}%
\pgfsys@useobject{currentmarker}{}%
\end{pgfscope}%
\end{pgfscope}%
\begin{pgfscope}%
\pgfsetbuttcap%
\pgfsetroundjoin%
\definecolor{currentfill}{rgb}{0.000000,0.000000,0.000000}%
\pgfsetfillcolor{currentfill}%
\pgfsetlinewidth{0.602250pt}%
\definecolor{currentstroke}{rgb}{0.000000,0.000000,0.000000}%
\pgfsetstrokecolor{currentstroke}%
\pgfsetdash{}{0pt}%
\pgfsys@defobject{currentmarker}{\pgfqpoint{0.000000in}{0.000000in}}{\pgfqpoint{0.000000in}{0.027778in}}{%
\pgfpathmoveto{\pgfqpoint{0.000000in}{0.000000in}}%
\pgfpathlineto{\pgfqpoint{0.000000in}{0.027778in}}%
\pgfusepath{stroke,fill}%
}%
\begin{pgfscope}%
\pgfsys@transformshift{2.142941in}{3.801389in}%
\pgfsys@useobject{currentmarker}{}%
\end{pgfscope}%
\end{pgfscope}%
\begin{pgfscope}%
\pgfpathrectangle{\pgfqpoint{0.781944in}{0.552778in}}{\pgfqpoint{3.890972in}{3.248611in}}%
\pgfusepath{clip}%
\pgfsetrectcap%
\pgfsetroundjoin%
\pgfsetlinewidth{0.803000pt}%
\definecolor{currentstroke}{rgb}{0.690196,0.690196,0.690196}%
\pgfsetstrokecolor{currentstroke}%
\pgfsetstrokeopacity{0.300000}%
\pgfsetdash{}{0pt}%
\pgfpathmoveto{\pgfqpoint{2.181864in}{0.552778in}}%
\pgfpathlineto{\pgfqpoint{2.181864in}{3.801389in}}%
\pgfusepath{stroke}%
\end{pgfscope}%
\begin{pgfscope}%
\pgfsetbuttcap%
\pgfsetroundjoin%
\definecolor{currentfill}{rgb}{0.000000,0.000000,0.000000}%
\pgfsetfillcolor{currentfill}%
\pgfsetlinewidth{0.602250pt}%
\definecolor{currentstroke}{rgb}{0.000000,0.000000,0.000000}%
\pgfsetstrokecolor{currentstroke}%
\pgfsetdash{}{0pt}%
\pgfsys@defobject{currentmarker}{\pgfqpoint{0.000000in}{-0.027778in}}{\pgfqpoint{0.000000in}{0.000000in}}{%
\pgfpathmoveto{\pgfqpoint{0.000000in}{0.000000in}}%
\pgfpathlineto{\pgfqpoint{0.000000in}{-0.027778in}}%
\pgfusepath{stroke,fill}%
}%
\begin{pgfscope}%
\pgfsys@transformshift{2.181864in}{0.552778in}%
\pgfsys@useobject{currentmarker}{}%
\end{pgfscope}%
\end{pgfscope}%
\begin{pgfscope}%
\pgfsetbuttcap%
\pgfsetroundjoin%
\definecolor{currentfill}{rgb}{0.000000,0.000000,0.000000}%
\pgfsetfillcolor{currentfill}%
\pgfsetlinewidth{0.602250pt}%
\definecolor{currentstroke}{rgb}{0.000000,0.000000,0.000000}%
\pgfsetstrokecolor{currentstroke}%
\pgfsetdash{}{0pt}%
\pgfsys@defobject{currentmarker}{\pgfqpoint{0.000000in}{0.000000in}}{\pgfqpoint{0.000000in}{0.027778in}}{%
\pgfpathmoveto{\pgfqpoint{0.000000in}{0.000000in}}%
\pgfpathlineto{\pgfqpoint{0.000000in}{0.027778in}}%
\pgfusepath{stroke,fill}%
}%
\begin{pgfscope}%
\pgfsys@transformshift{2.181864in}{3.801389in}%
\pgfsys@useobject{currentmarker}{}%
\end{pgfscope}%
\end{pgfscope}%
\begin{pgfscope}%
\pgfpathrectangle{\pgfqpoint{0.781944in}{0.552778in}}{\pgfqpoint{3.890972in}{3.248611in}}%
\pgfusepath{clip}%
\pgfsetrectcap%
\pgfsetroundjoin%
\pgfsetlinewidth{0.803000pt}%
\definecolor{currentstroke}{rgb}{0.690196,0.690196,0.690196}%
\pgfsetstrokecolor{currentstroke}%
\pgfsetstrokeopacity{0.300000}%
\pgfsetdash{}{0pt}%
\pgfpathmoveto{\pgfqpoint{2.220787in}{0.552778in}}%
\pgfpathlineto{\pgfqpoint{2.220787in}{3.801389in}}%
\pgfusepath{stroke}%
\end{pgfscope}%
\begin{pgfscope}%
\pgfsetbuttcap%
\pgfsetroundjoin%
\definecolor{currentfill}{rgb}{0.000000,0.000000,0.000000}%
\pgfsetfillcolor{currentfill}%
\pgfsetlinewidth{0.602250pt}%
\definecolor{currentstroke}{rgb}{0.000000,0.000000,0.000000}%
\pgfsetstrokecolor{currentstroke}%
\pgfsetdash{}{0pt}%
\pgfsys@defobject{currentmarker}{\pgfqpoint{0.000000in}{-0.027778in}}{\pgfqpoint{0.000000in}{0.000000in}}{%
\pgfpathmoveto{\pgfqpoint{0.000000in}{0.000000in}}%
\pgfpathlineto{\pgfqpoint{0.000000in}{-0.027778in}}%
\pgfusepath{stroke,fill}%
}%
\begin{pgfscope}%
\pgfsys@transformshift{2.220787in}{0.552778in}%
\pgfsys@useobject{currentmarker}{}%
\end{pgfscope}%
\end{pgfscope}%
\begin{pgfscope}%
\pgfsetbuttcap%
\pgfsetroundjoin%
\definecolor{currentfill}{rgb}{0.000000,0.000000,0.000000}%
\pgfsetfillcolor{currentfill}%
\pgfsetlinewidth{0.602250pt}%
\definecolor{currentstroke}{rgb}{0.000000,0.000000,0.000000}%
\pgfsetstrokecolor{currentstroke}%
\pgfsetdash{}{0pt}%
\pgfsys@defobject{currentmarker}{\pgfqpoint{0.000000in}{0.000000in}}{\pgfqpoint{0.000000in}{0.027778in}}{%
\pgfpathmoveto{\pgfqpoint{0.000000in}{0.000000in}}%
\pgfpathlineto{\pgfqpoint{0.000000in}{0.027778in}}%
\pgfusepath{stroke,fill}%
}%
\begin{pgfscope}%
\pgfsys@transformshift{2.220787in}{3.801389in}%
\pgfsys@useobject{currentmarker}{}%
\end{pgfscope}%
\end{pgfscope}%
\begin{pgfscope}%
\pgfpathrectangle{\pgfqpoint{0.781944in}{0.552778in}}{\pgfqpoint{3.890972in}{3.248611in}}%
\pgfusepath{clip}%
\pgfsetrectcap%
\pgfsetroundjoin%
\pgfsetlinewidth{0.803000pt}%
\definecolor{currentstroke}{rgb}{0.690196,0.690196,0.690196}%
\pgfsetstrokecolor{currentstroke}%
\pgfsetstrokeopacity{0.300000}%
\pgfsetdash{}{0pt}%
\pgfpathmoveto{\pgfqpoint{2.259709in}{0.552778in}}%
\pgfpathlineto{\pgfqpoint{2.259709in}{3.801389in}}%
\pgfusepath{stroke}%
\end{pgfscope}%
\begin{pgfscope}%
\pgfsetbuttcap%
\pgfsetroundjoin%
\definecolor{currentfill}{rgb}{0.000000,0.000000,0.000000}%
\pgfsetfillcolor{currentfill}%
\pgfsetlinewidth{0.602250pt}%
\definecolor{currentstroke}{rgb}{0.000000,0.000000,0.000000}%
\pgfsetstrokecolor{currentstroke}%
\pgfsetdash{}{0pt}%
\pgfsys@defobject{currentmarker}{\pgfqpoint{0.000000in}{-0.027778in}}{\pgfqpoint{0.000000in}{0.000000in}}{%
\pgfpathmoveto{\pgfqpoint{0.000000in}{0.000000in}}%
\pgfpathlineto{\pgfqpoint{0.000000in}{-0.027778in}}%
\pgfusepath{stroke,fill}%
}%
\begin{pgfscope}%
\pgfsys@transformshift{2.259709in}{0.552778in}%
\pgfsys@useobject{currentmarker}{}%
\end{pgfscope}%
\end{pgfscope}%
\begin{pgfscope}%
\pgfsetbuttcap%
\pgfsetroundjoin%
\definecolor{currentfill}{rgb}{0.000000,0.000000,0.000000}%
\pgfsetfillcolor{currentfill}%
\pgfsetlinewidth{0.602250pt}%
\definecolor{currentstroke}{rgb}{0.000000,0.000000,0.000000}%
\pgfsetstrokecolor{currentstroke}%
\pgfsetdash{}{0pt}%
\pgfsys@defobject{currentmarker}{\pgfqpoint{0.000000in}{0.000000in}}{\pgfqpoint{0.000000in}{0.027778in}}{%
\pgfpathmoveto{\pgfqpoint{0.000000in}{0.000000in}}%
\pgfpathlineto{\pgfqpoint{0.000000in}{0.027778in}}%
\pgfusepath{stroke,fill}%
}%
\begin{pgfscope}%
\pgfsys@transformshift{2.259709in}{3.801389in}%
\pgfsys@useobject{currentmarker}{}%
\end{pgfscope}%
\end{pgfscope}%
\begin{pgfscope}%
\pgfpathrectangle{\pgfqpoint{0.781944in}{0.552778in}}{\pgfqpoint{3.890972in}{3.248611in}}%
\pgfusepath{clip}%
\pgfsetrectcap%
\pgfsetroundjoin%
\pgfsetlinewidth{0.803000pt}%
\definecolor{currentstroke}{rgb}{0.690196,0.690196,0.690196}%
\pgfsetstrokecolor{currentstroke}%
\pgfsetstrokeopacity{0.300000}%
\pgfsetdash{}{0pt}%
\pgfpathmoveto{\pgfqpoint{2.298632in}{0.552778in}}%
\pgfpathlineto{\pgfqpoint{2.298632in}{3.801389in}}%
\pgfusepath{stroke}%
\end{pgfscope}%
\begin{pgfscope}%
\pgfsetbuttcap%
\pgfsetroundjoin%
\definecolor{currentfill}{rgb}{0.000000,0.000000,0.000000}%
\pgfsetfillcolor{currentfill}%
\pgfsetlinewidth{0.602250pt}%
\definecolor{currentstroke}{rgb}{0.000000,0.000000,0.000000}%
\pgfsetstrokecolor{currentstroke}%
\pgfsetdash{}{0pt}%
\pgfsys@defobject{currentmarker}{\pgfqpoint{0.000000in}{-0.027778in}}{\pgfqpoint{0.000000in}{0.000000in}}{%
\pgfpathmoveto{\pgfqpoint{0.000000in}{0.000000in}}%
\pgfpathlineto{\pgfqpoint{0.000000in}{-0.027778in}}%
\pgfusepath{stroke,fill}%
}%
\begin{pgfscope}%
\pgfsys@transformshift{2.298632in}{0.552778in}%
\pgfsys@useobject{currentmarker}{}%
\end{pgfscope}%
\end{pgfscope}%
\begin{pgfscope}%
\pgfsetbuttcap%
\pgfsetroundjoin%
\definecolor{currentfill}{rgb}{0.000000,0.000000,0.000000}%
\pgfsetfillcolor{currentfill}%
\pgfsetlinewidth{0.602250pt}%
\definecolor{currentstroke}{rgb}{0.000000,0.000000,0.000000}%
\pgfsetstrokecolor{currentstroke}%
\pgfsetdash{}{0pt}%
\pgfsys@defobject{currentmarker}{\pgfqpoint{0.000000in}{0.000000in}}{\pgfqpoint{0.000000in}{0.027778in}}{%
\pgfpathmoveto{\pgfqpoint{0.000000in}{0.000000in}}%
\pgfpathlineto{\pgfqpoint{0.000000in}{0.027778in}}%
\pgfusepath{stroke,fill}%
}%
\begin{pgfscope}%
\pgfsys@transformshift{2.298632in}{3.801389in}%
\pgfsys@useobject{currentmarker}{}%
\end{pgfscope}%
\end{pgfscope}%
\begin{pgfscope}%
\pgfpathrectangle{\pgfqpoint{0.781944in}{0.552778in}}{\pgfqpoint{3.890972in}{3.248611in}}%
\pgfusepath{clip}%
\pgfsetrectcap%
\pgfsetroundjoin%
\pgfsetlinewidth{0.803000pt}%
\definecolor{currentstroke}{rgb}{0.690196,0.690196,0.690196}%
\pgfsetstrokecolor{currentstroke}%
\pgfsetstrokeopacity{0.300000}%
\pgfsetdash{}{0pt}%
\pgfpathmoveto{\pgfqpoint{2.376478in}{0.552778in}}%
\pgfpathlineto{\pgfqpoint{2.376478in}{3.801389in}}%
\pgfusepath{stroke}%
\end{pgfscope}%
\begin{pgfscope}%
\pgfsetbuttcap%
\pgfsetroundjoin%
\definecolor{currentfill}{rgb}{0.000000,0.000000,0.000000}%
\pgfsetfillcolor{currentfill}%
\pgfsetlinewidth{0.602250pt}%
\definecolor{currentstroke}{rgb}{0.000000,0.000000,0.000000}%
\pgfsetstrokecolor{currentstroke}%
\pgfsetdash{}{0pt}%
\pgfsys@defobject{currentmarker}{\pgfqpoint{0.000000in}{-0.027778in}}{\pgfqpoint{0.000000in}{0.000000in}}{%
\pgfpathmoveto{\pgfqpoint{0.000000in}{0.000000in}}%
\pgfpathlineto{\pgfqpoint{0.000000in}{-0.027778in}}%
\pgfusepath{stroke,fill}%
}%
\begin{pgfscope}%
\pgfsys@transformshift{2.376478in}{0.552778in}%
\pgfsys@useobject{currentmarker}{}%
\end{pgfscope}%
\end{pgfscope}%
\begin{pgfscope}%
\pgfsetbuttcap%
\pgfsetroundjoin%
\definecolor{currentfill}{rgb}{0.000000,0.000000,0.000000}%
\pgfsetfillcolor{currentfill}%
\pgfsetlinewidth{0.602250pt}%
\definecolor{currentstroke}{rgb}{0.000000,0.000000,0.000000}%
\pgfsetstrokecolor{currentstroke}%
\pgfsetdash{}{0pt}%
\pgfsys@defobject{currentmarker}{\pgfqpoint{0.000000in}{0.000000in}}{\pgfqpoint{0.000000in}{0.027778in}}{%
\pgfpathmoveto{\pgfqpoint{0.000000in}{0.000000in}}%
\pgfpathlineto{\pgfqpoint{0.000000in}{0.027778in}}%
\pgfusepath{stroke,fill}%
}%
\begin{pgfscope}%
\pgfsys@transformshift{2.376478in}{3.801389in}%
\pgfsys@useobject{currentmarker}{}%
\end{pgfscope}%
\end{pgfscope}%
\begin{pgfscope}%
\pgfpathrectangle{\pgfqpoint{0.781944in}{0.552778in}}{\pgfqpoint{3.890972in}{3.248611in}}%
\pgfusepath{clip}%
\pgfsetrectcap%
\pgfsetroundjoin%
\pgfsetlinewidth{0.803000pt}%
\definecolor{currentstroke}{rgb}{0.690196,0.690196,0.690196}%
\pgfsetstrokecolor{currentstroke}%
\pgfsetstrokeopacity{0.300000}%
\pgfsetdash{}{0pt}%
\pgfpathmoveto{\pgfqpoint{2.415400in}{0.552778in}}%
\pgfpathlineto{\pgfqpoint{2.415400in}{3.801389in}}%
\pgfusepath{stroke}%
\end{pgfscope}%
\begin{pgfscope}%
\pgfsetbuttcap%
\pgfsetroundjoin%
\definecolor{currentfill}{rgb}{0.000000,0.000000,0.000000}%
\pgfsetfillcolor{currentfill}%
\pgfsetlinewidth{0.602250pt}%
\definecolor{currentstroke}{rgb}{0.000000,0.000000,0.000000}%
\pgfsetstrokecolor{currentstroke}%
\pgfsetdash{}{0pt}%
\pgfsys@defobject{currentmarker}{\pgfqpoint{0.000000in}{-0.027778in}}{\pgfqpoint{0.000000in}{0.000000in}}{%
\pgfpathmoveto{\pgfqpoint{0.000000in}{0.000000in}}%
\pgfpathlineto{\pgfqpoint{0.000000in}{-0.027778in}}%
\pgfusepath{stroke,fill}%
}%
\begin{pgfscope}%
\pgfsys@transformshift{2.415400in}{0.552778in}%
\pgfsys@useobject{currentmarker}{}%
\end{pgfscope}%
\end{pgfscope}%
\begin{pgfscope}%
\pgfsetbuttcap%
\pgfsetroundjoin%
\definecolor{currentfill}{rgb}{0.000000,0.000000,0.000000}%
\pgfsetfillcolor{currentfill}%
\pgfsetlinewidth{0.602250pt}%
\definecolor{currentstroke}{rgb}{0.000000,0.000000,0.000000}%
\pgfsetstrokecolor{currentstroke}%
\pgfsetdash{}{0pt}%
\pgfsys@defobject{currentmarker}{\pgfqpoint{0.000000in}{0.000000in}}{\pgfqpoint{0.000000in}{0.027778in}}{%
\pgfpathmoveto{\pgfqpoint{0.000000in}{0.000000in}}%
\pgfpathlineto{\pgfqpoint{0.000000in}{0.027778in}}%
\pgfusepath{stroke,fill}%
}%
\begin{pgfscope}%
\pgfsys@transformshift{2.415400in}{3.801389in}%
\pgfsys@useobject{currentmarker}{}%
\end{pgfscope}%
\end{pgfscope}%
\begin{pgfscope}%
\pgfpathrectangle{\pgfqpoint{0.781944in}{0.552778in}}{\pgfqpoint{3.890972in}{3.248611in}}%
\pgfusepath{clip}%
\pgfsetrectcap%
\pgfsetroundjoin%
\pgfsetlinewidth{0.803000pt}%
\definecolor{currentstroke}{rgb}{0.690196,0.690196,0.690196}%
\pgfsetstrokecolor{currentstroke}%
\pgfsetstrokeopacity{0.300000}%
\pgfsetdash{}{0pt}%
\pgfpathmoveto{\pgfqpoint{2.454323in}{0.552778in}}%
\pgfpathlineto{\pgfqpoint{2.454323in}{3.801389in}}%
\pgfusepath{stroke}%
\end{pgfscope}%
\begin{pgfscope}%
\pgfsetbuttcap%
\pgfsetroundjoin%
\definecolor{currentfill}{rgb}{0.000000,0.000000,0.000000}%
\pgfsetfillcolor{currentfill}%
\pgfsetlinewidth{0.602250pt}%
\definecolor{currentstroke}{rgb}{0.000000,0.000000,0.000000}%
\pgfsetstrokecolor{currentstroke}%
\pgfsetdash{}{0pt}%
\pgfsys@defobject{currentmarker}{\pgfqpoint{0.000000in}{-0.027778in}}{\pgfqpoint{0.000000in}{0.000000in}}{%
\pgfpathmoveto{\pgfqpoint{0.000000in}{0.000000in}}%
\pgfpathlineto{\pgfqpoint{0.000000in}{-0.027778in}}%
\pgfusepath{stroke,fill}%
}%
\begin{pgfscope}%
\pgfsys@transformshift{2.454323in}{0.552778in}%
\pgfsys@useobject{currentmarker}{}%
\end{pgfscope}%
\end{pgfscope}%
\begin{pgfscope}%
\pgfsetbuttcap%
\pgfsetroundjoin%
\definecolor{currentfill}{rgb}{0.000000,0.000000,0.000000}%
\pgfsetfillcolor{currentfill}%
\pgfsetlinewidth{0.602250pt}%
\definecolor{currentstroke}{rgb}{0.000000,0.000000,0.000000}%
\pgfsetstrokecolor{currentstroke}%
\pgfsetdash{}{0pt}%
\pgfsys@defobject{currentmarker}{\pgfqpoint{0.000000in}{0.000000in}}{\pgfqpoint{0.000000in}{0.027778in}}{%
\pgfpathmoveto{\pgfqpoint{0.000000in}{0.000000in}}%
\pgfpathlineto{\pgfqpoint{0.000000in}{0.027778in}}%
\pgfusepath{stroke,fill}%
}%
\begin{pgfscope}%
\pgfsys@transformshift{2.454323in}{3.801389in}%
\pgfsys@useobject{currentmarker}{}%
\end{pgfscope}%
\end{pgfscope}%
\begin{pgfscope}%
\pgfpathrectangle{\pgfqpoint{0.781944in}{0.552778in}}{\pgfqpoint{3.890972in}{3.248611in}}%
\pgfusepath{clip}%
\pgfsetrectcap%
\pgfsetroundjoin%
\pgfsetlinewidth{0.803000pt}%
\definecolor{currentstroke}{rgb}{0.690196,0.690196,0.690196}%
\pgfsetstrokecolor{currentstroke}%
\pgfsetstrokeopacity{0.300000}%
\pgfsetdash{}{0pt}%
\pgfpathmoveto{\pgfqpoint{2.493246in}{0.552778in}}%
\pgfpathlineto{\pgfqpoint{2.493246in}{3.801389in}}%
\pgfusepath{stroke}%
\end{pgfscope}%
\begin{pgfscope}%
\pgfsetbuttcap%
\pgfsetroundjoin%
\definecolor{currentfill}{rgb}{0.000000,0.000000,0.000000}%
\pgfsetfillcolor{currentfill}%
\pgfsetlinewidth{0.602250pt}%
\definecolor{currentstroke}{rgb}{0.000000,0.000000,0.000000}%
\pgfsetstrokecolor{currentstroke}%
\pgfsetdash{}{0pt}%
\pgfsys@defobject{currentmarker}{\pgfqpoint{0.000000in}{-0.027778in}}{\pgfqpoint{0.000000in}{0.000000in}}{%
\pgfpathmoveto{\pgfqpoint{0.000000in}{0.000000in}}%
\pgfpathlineto{\pgfqpoint{0.000000in}{-0.027778in}}%
\pgfusepath{stroke,fill}%
}%
\begin{pgfscope}%
\pgfsys@transformshift{2.493246in}{0.552778in}%
\pgfsys@useobject{currentmarker}{}%
\end{pgfscope}%
\end{pgfscope}%
\begin{pgfscope}%
\pgfsetbuttcap%
\pgfsetroundjoin%
\definecolor{currentfill}{rgb}{0.000000,0.000000,0.000000}%
\pgfsetfillcolor{currentfill}%
\pgfsetlinewidth{0.602250pt}%
\definecolor{currentstroke}{rgb}{0.000000,0.000000,0.000000}%
\pgfsetstrokecolor{currentstroke}%
\pgfsetdash{}{0pt}%
\pgfsys@defobject{currentmarker}{\pgfqpoint{0.000000in}{0.000000in}}{\pgfqpoint{0.000000in}{0.027778in}}{%
\pgfpathmoveto{\pgfqpoint{0.000000in}{0.000000in}}%
\pgfpathlineto{\pgfqpoint{0.000000in}{0.027778in}}%
\pgfusepath{stroke,fill}%
}%
\begin{pgfscope}%
\pgfsys@transformshift{2.493246in}{3.801389in}%
\pgfsys@useobject{currentmarker}{}%
\end{pgfscope}%
\end{pgfscope}%
\begin{pgfscope}%
\pgfpathrectangle{\pgfqpoint{0.781944in}{0.552778in}}{\pgfqpoint{3.890972in}{3.248611in}}%
\pgfusepath{clip}%
\pgfsetrectcap%
\pgfsetroundjoin%
\pgfsetlinewidth{0.803000pt}%
\definecolor{currentstroke}{rgb}{0.690196,0.690196,0.690196}%
\pgfsetstrokecolor{currentstroke}%
\pgfsetstrokeopacity{0.300000}%
\pgfsetdash{}{0pt}%
\pgfpathmoveto{\pgfqpoint{2.532168in}{0.552778in}}%
\pgfpathlineto{\pgfqpoint{2.532168in}{3.801389in}}%
\pgfusepath{stroke}%
\end{pgfscope}%
\begin{pgfscope}%
\pgfsetbuttcap%
\pgfsetroundjoin%
\definecolor{currentfill}{rgb}{0.000000,0.000000,0.000000}%
\pgfsetfillcolor{currentfill}%
\pgfsetlinewidth{0.602250pt}%
\definecolor{currentstroke}{rgb}{0.000000,0.000000,0.000000}%
\pgfsetstrokecolor{currentstroke}%
\pgfsetdash{}{0pt}%
\pgfsys@defobject{currentmarker}{\pgfqpoint{0.000000in}{-0.027778in}}{\pgfqpoint{0.000000in}{0.000000in}}{%
\pgfpathmoveto{\pgfqpoint{0.000000in}{0.000000in}}%
\pgfpathlineto{\pgfqpoint{0.000000in}{-0.027778in}}%
\pgfusepath{stroke,fill}%
}%
\begin{pgfscope}%
\pgfsys@transformshift{2.532168in}{0.552778in}%
\pgfsys@useobject{currentmarker}{}%
\end{pgfscope}%
\end{pgfscope}%
\begin{pgfscope}%
\pgfsetbuttcap%
\pgfsetroundjoin%
\definecolor{currentfill}{rgb}{0.000000,0.000000,0.000000}%
\pgfsetfillcolor{currentfill}%
\pgfsetlinewidth{0.602250pt}%
\definecolor{currentstroke}{rgb}{0.000000,0.000000,0.000000}%
\pgfsetstrokecolor{currentstroke}%
\pgfsetdash{}{0pt}%
\pgfsys@defobject{currentmarker}{\pgfqpoint{0.000000in}{0.000000in}}{\pgfqpoint{0.000000in}{0.027778in}}{%
\pgfpathmoveto{\pgfqpoint{0.000000in}{0.000000in}}%
\pgfpathlineto{\pgfqpoint{0.000000in}{0.027778in}}%
\pgfusepath{stroke,fill}%
}%
\begin{pgfscope}%
\pgfsys@transformshift{2.532168in}{3.801389in}%
\pgfsys@useobject{currentmarker}{}%
\end{pgfscope}%
\end{pgfscope}%
\begin{pgfscope}%
\pgfpathrectangle{\pgfqpoint{0.781944in}{0.552778in}}{\pgfqpoint{3.890972in}{3.248611in}}%
\pgfusepath{clip}%
\pgfsetrectcap%
\pgfsetroundjoin%
\pgfsetlinewidth{0.803000pt}%
\definecolor{currentstroke}{rgb}{0.690196,0.690196,0.690196}%
\pgfsetstrokecolor{currentstroke}%
\pgfsetstrokeopacity{0.300000}%
\pgfsetdash{}{0pt}%
\pgfpathmoveto{\pgfqpoint{2.571091in}{0.552778in}}%
\pgfpathlineto{\pgfqpoint{2.571091in}{3.801389in}}%
\pgfusepath{stroke}%
\end{pgfscope}%
\begin{pgfscope}%
\pgfsetbuttcap%
\pgfsetroundjoin%
\definecolor{currentfill}{rgb}{0.000000,0.000000,0.000000}%
\pgfsetfillcolor{currentfill}%
\pgfsetlinewidth{0.602250pt}%
\definecolor{currentstroke}{rgb}{0.000000,0.000000,0.000000}%
\pgfsetstrokecolor{currentstroke}%
\pgfsetdash{}{0pt}%
\pgfsys@defobject{currentmarker}{\pgfqpoint{0.000000in}{-0.027778in}}{\pgfqpoint{0.000000in}{0.000000in}}{%
\pgfpathmoveto{\pgfqpoint{0.000000in}{0.000000in}}%
\pgfpathlineto{\pgfqpoint{0.000000in}{-0.027778in}}%
\pgfusepath{stroke,fill}%
}%
\begin{pgfscope}%
\pgfsys@transformshift{2.571091in}{0.552778in}%
\pgfsys@useobject{currentmarker}{}%
\end{pgfscope}%
\end{pgfscope}%
\begin{pgfscope}%
\pgfsetbuttcap%
\pgfsetroundjoin%
\definecolor{currentfill}{rgb}{0.000000,0.000000,0.000000}%
\pgfsetfillcolor{currentfill}%
\pgfsetlinewidth{0.602250pt}%
\definecolor{currentstroke}{rgb}{0.000000,0.000000,0.000000}%
\pgfsetstrokecolor{currentstroke}%
\pgfsetdash{}{0pt}%
\pgfsys@defobject{currentmarker}{\pgfqpoint{0.000000in}{0.000000in}}{\pgfqpoint{0.000000in}{0.027778in}}{%
\pgfpathmoveto{\pgfqpoint{0.000000in}{0.000000in}}%
\pgfpathlineto{\pgfqpoint{0.000000in}{0.027778in}}%
\pgfusepath{stroke,fill}%
}%
\begin{pgfscope}%
\pgfsys@transformshift{2.571091in}{3.801389in}%
\pgfsys@useobject{currentmarker}{}%
\end{pgfscope}%
\end{pgfscope}%
\begin{pgfscope}%
\pgfpathrectangle{\pgfqpoint{0.781944in}{0.552778in}}{\pgfqpoint{3.890972in}{3.248611in}}%
\pgfusepath{clip}%
\pgfsetrectcap%
\pgfsetroundjoin%
\pgfsetlinewidth{0.803000pt}%
\definecolor{currentstroke}{rgb}{0.690196,0.690196,0.690196}%
\pgfsetstrokecolor{currentstroke}%
\pgfsetstrokeopacity{0.300000}%
\pgfsetdash{}{0pt}%
\pgfpathmoveto{\pgfqpoint{2.610014in}{0.552778in}}%
\pgfpathlineto{\pgfqpoint{2.610014in}{3.801389in}}%
\pgfusepath{stroke}%
\end{pgfscope}%
\begin{pgfscope}%
\pgfsetbuttcap%
\pgfsetroundjoin%
\definecolor{currentfill}{rgb}{0.000000,0.000000,0.000000}%
\pgfsetfillcolor{currentfill}%
\pgfsetlinewidth{0.602250pt}%
\definecolor{currentstroke}{rgb}{0.000000,0.000000,0.000000}%
\pgfsetstrokecolor{currentstroke}%
\pgfsetdash{}{0pt}%
\pgfsys@defobject{currentmarker}{\pgfqpoint{0.000000in}{-0.027778in}}{\pgfqpoint{0.000000in}{0.000000in}}{%
\pgfpathmoveto{\pgfqpoint{0.000000in}{0.000000in}}%
\pgfpathlineto{\pgfqpoint{0.000000in}{-0.027778in}}%
\pgfusepath{stroke,fill}%
}%
\begin{pgfscope}%
\pgfsys@transformshift{2.610014in}{0.552778in}%
\pgfsys@useobject{currentmarker}{}%
\end{pgfscope}%
\end{pgfscope}%
\begin{pgfscope}%
\pgfsetbuttcap%
\pgfsetroundjoin%
\definecolor{currentfill}{rgb}{0.000000,0.000000,0.000000}%
\pgfsetfillcolor{currentfill}%
\pgfsetlinewidth{0.602250pt}%
\definecolor{currentstroke}{rgb}{0.000000,0.000000,0.000000}%
\pgfsetstrokecolor{currentstroke}%
\pgfsetdash{}{0pt}%
\pgfsys@defobject{currentmarker}{\pgfqpoint{0.000000in}{0.000000in}}{\pgfqpoint{0.000000in}{0.027778in}}{%
\pgfpathmoveto{\pgfqpoint{0.000000in}{0.000000in}}%
\pgfpathlineto{\pgfqpoint{0.000000in}{0.027778in}}%
\pgfusepath{stroke,fill}%
}%
\begin{pgfscope}%
\pgfsys@transformshift{2.610014in}{3.801389in}%
\pgfsys@useobject{currentmarker}{}%
\end{pgfscope}%
\end{pgfscope}%
\begin{pgfscope}%
\pgfpathrectangle{\pgfqpoint{0.781944in}{0.552778in}}{\pgfqpoint{3.890972in}{3.248611in}}%
\pgfusepath{clip}%
\pgfsetrectcap%
\pgfsetroundjoin%
\pgfsetlinewidth{0.803000pt}%
\definecolor{currentstroke}{rgb}{0.690196,0.690196,0.690196}%
\pgfsetstrokecolor{currentstroke}%
\pgfsetstrokeopacity{0.300000}%
\pgfsetdash{}{0pt}%
\pgfpathmoveto{\pgfqpoint{2.648936in}{0.552778in}}%
\pgfpathlineto{\pgfqpoint{2.648936in}{3.801389in}}%
\pgfusepath{stroke}%
\end{pgfscope}%
\begin{pgfscope}%
\pgfsetbuttcap%
\pgfsetroundjoin%
\definecolor{currentfill}{rgb}{0.000000,0.000000,0.000000}%
\pgfsetfillcolor{currentfill}%
\pgfsetlinewidth{0.602250pt}%
\definecolor{currentstroke}{rgb}{0.000000,0.000000,0.000000}%
\pgfsetstrokecolor{currentstroke}%
\pgfsetdash{}{0pt}%
\pgfsys@defobject{currentmarker}{\pgfqpoint{0.000000in}{-0.027778in}}{\pgfqpoint{0.000000in}{0.000000in}}{%
\pgfpathmoveto{\pgfqpoint{0.000000in}{0.000000in}}%
\pgfpathlineto{\pgfqpoint{0.000000in}{-0.027778in}}%
\pgfusepath{stroke,fill}%
}%
\begin{pgfscope}%
\pgfsys@transformshift{2.648936in}{0.552778in}%
\pgfsys@useobject{currentmarker}{}%
\end{pgfscope}%
\end{pgfscope}%
\begin{pgfscope}%
\pgfsetbuttcap%
\pgfsetroundjoin%
\definecolor{currentfill}{rgb}{0.000000,0.000000,0.000000}%
\pgfsetfillcolor{currentfill}%
\pgfsetlinewidth{0.602250pt}%
\definecolor{currentstroke}{rgb}{0.000000,0.000000,0.000000}%
\pgfsetstrokecolor{currentstroke}%
\pgfsetdash{}{0pt}%
\pgfsys@defobject{currentmarker}{\pgfqpoint{0.000000in}{0.000000in}}{\pgfqpoint{0.000000in}{0.027778in}}{%
\pgfpathmoveto{\pgfqpoint{0.000000in}{0.000000in}}%
\pgfpathlineto{\pgfqpoint{0.000000in}{0.027778in}}%
\pgfusepath{stroke,fill}%
}%
\begin{pgfscope}%
\pgfsys@transformshift{2.648936in}{3.801389in}%
\pgfsys@useobject{currentmarker}{}%
\end{pgfscope}%
\end{pgfscope}%
\begin{pgfscope}%
\pgfpathrectangle{\pgfqpoint{0.781944in}{0.552778in}}{\pgfqpoint{3.890972in}{3.248611in}}%
\pgfusepath{clip}%
\pgfsetrectcap%
\pgfsetroundjoin%
\pgfsetlinewidth{0.803000pt}%
\definecolor{currentstroke}{rgb}{0.690196,0.690196,0.690196}%
\pgfsetstrokecolor{currentstroke}%
\pgfsetstrokeopacity{0.300000}%
\pgfsetdash{}{0pt}%
\pgfpathmoveto{\pgfqpoint{2.687859in}{0.552778in}}%
\pgfpathlineto{\pgfqpoint{2.687859in}{3.801389in}}%
\pgfusepath{stroke}%
\end{pgfscope}%
\begin{pgfscope}%
\pgfsetbuttcap%
\pgfsetroundjoin%
\definecolor{currentfill}{rgb}{0.000000,0.000000,0.000000}%
\pgfsetfillcolor{currentfill}%
\pgfsetlinewidth{0.602250pt}%
\definecolor{currentstroke}{rgb}{0.000000,0.000000,0.000000}%
\pgfsetstrokecolor{currentstroke}%
\pgfsetdash{}{0pt}%
\pgfsys@defobject{currentmarker}{\pgfqpoint{0.000000in}{-0.027778in}}{\pgfqpoint{0.000000in}{0.000000in}}{%
\pgfpathmoveto{\pgfqpoint{0.000000in}{0.000000in}}%
\pgfpathlineto{\pgfqpoint{0.000000in}{-0.027778in}}%
\pgfusepath{stroke,fill}%
}%
\begin{pgfscope}%
\pgfsys@transformshift{2.687859in}{0.552778in}%
\pgfsys@useobject{currentmarker}{}%
\end{pgfscope}%
\end{pgfscope}%
\begin{pgfscope}%
\pgfsetbuttcap%
\pgfsetroundjoin%
\definecolor{currentfill}{rgb}{0.000000,0.000000,0.000000}%
\pgfsetfillcolor{currentfill}%
\pgfsetlinewidth{0.602250pt}%
\definecolor{currentstroke}{rgb}{0.000000,0.000000,0.000000}%
\pgfsetstrokecolor{currentstroke}%
\pgfsetdash{}{0pt}%
\pgfsys@defobject{currentmarker}{\pgfqpoint{0.000000in}{0.000000in}}{\pgfqpoint{0.000000in}{0.027778in}}{%
\pgfpathmoveto{\pgfqpoint{0.000000in}{0.000000in}}%
\pgfpathlineto{\pgfqpoint{0.000000in}{0.027778in}}%
\pgfusepath{stroke,fill}%
}%
\begin{pgfscope}%
\pgfsys@transformshift{2.687859in}{3.801389in}%
\pgfsys@useobject{currentmarker}{}%
\end{pgfscope}%
\end{pgfscope}%
\begin{pgfscope}%
\pgfpathrectangle{\pgfqpoint{0.781944in}{0.552778in}}{\pgfqpoint{3.890972in}{3.248611in}}%
\pgfusepath{clip}%
\pgfsetrectcap%
\pgfsetroundjoin%
\pgfsetlinewidth{0.803000pt}%
\definecolor{currentstroke}{rgb}{0.690196,0.690196,0.690196}%
\pgfsetstrokecolor{currentstroke}%
\pgfsetstrokeopacity{0.300000}%
\pgfsetdash{}{0pt}%
\pgfpathmoveto{\pgfqpoint{2.765705in}{0.552778in}}%
\pgfpathlineto{\pgfqpoint{2.765705in}{3.801389in}}%
\pgfusepath{stroke}%
\end{pgfscope}%
\begin{pgfscope}%
\pgfsetbuttcap%
\pgfsetroundjoin%
\definecolor{currentfill}{rgb}{0.000000,0.000000,0.000000}%
\pgfsetfillcolor{currentfill}%
\pgfsetlinewidth{0.602250pt}%
\definecolor{currentstroke}{rgb}{0.000000,0.000000,0.000000}%
\pgfsetstrokecolor{currentstroke}%
\pgfsetdash{}{0pt}%
\pgfsys@defobject{currentmarker}{\pgfqpoint{0.000000in}{-0.027778in}}{\pgfqpoint{0.000000in}{0.000000in}}{%
\pgfpathmoveto{\pgfqpoint{0.000000in}{0.000000in}}%
\pgfpathlineto{\pgfqpoint{0.000000in}{-0.027778in}}%
\pgfusepath{stroke,fill}%
}%
\begin{pgfscope}%
\pgfsys@transformshift{2.765705in}{0.552778in}%
\pgfsys@useobject{currentmarker}{}%
\end{pgfscope}%
\end{pgfscope}%
\begin{pgfscope}%
\pgfsetbuttcap%
\pgfsetroundjoin%
\definecolor{currentfill}{rgb}{0.000000,0.000000,0.000000}%
\pgfsetfillcolor{currentfill}%
\pgfsetlinewidth{0.602250pt}%
\definecolor{currentstroke}{rgb}{0.000000,0.000000,0.000000}%
\pgfsetstrokecolor{currentstroke}%
\pgfsetdash{}{0pt}%
\pgfsys@defobject{currentmarker}{\pgfqpoint{0.000000in}{0.000000in}}{\pgfqpoint{0.000000in}{0.027778in}}{%
\pgfpathmoveto{\pgfqpoint{0.000000in}{0.000000in}}%
\pgfpathlineto{\pgfqpoint{0.000000in}{0.027778in}}%
\pgfusepath{stroke,fill}%
}%
\begin{pgfscope}%
\pgfsys@transformshift{2.765705in}{3.801389in}%
\pgfsys@useobject{currentmarker}{}%
\end{pgfscope}%
\end{pgfscope}%
\begin{pgfscope}%
\pgfpathrectangle{\pgfqpoint{0.781944in}{0.552778in}}{\pgfqpoint{3.890972in}{3.248611in}}%
\pgfusepath{clip}%
\pgfsetrectcap%
\pgfsetroundjoin%
\pgfsetlinewidth{0.803000pt}%
\definecolor{currentstroke}{rgb}{0.690196,0.690196,0.690196}%
\pgfsetstrokecolor{currentstroke}%
\pgfsetstrokeopacity{0.300000}%
\pgfsetdash{}{0pt}%
\pgfpathmoveto{\pgfqpoint{2.804627in}{0.552778in}}%
\pgfpathlineto{\pgfqpoint{2.804627in}{3.801389in}}%
\pgfusepath{stroke}%
\end{pgfscope}%
\begin{pgfscope}%
\pgfsetbuttcap%
\pgfsetroundjoin%
\definecolor{currentfill}{rgb}{0.000000,0.000000,0.000000}%
\pgfsetfillcolor{currentfill}%
\pgfsetlinewidth{0.602250pt}%
\definecolor{currentstroke}{rgb}{0.000000,0.000000,0.000000}%
\pgfsetstrokecolor{currentstroke}%
\pgfsetdash{}{0pt}%
\pgfsys@defobject{currentmarker}{\pgfqpoint{0.000000in}{-0.027778in}}{\pgfqpoint{0.000000in}{0.000000in}}{%
\pgfpathmoveto{\pgfqpoint{0.000000in}{0.000000in}}%
\pgfpathlineto{\pgfqpoint{0.000000in}{-0.027778in}}%
\pgfusepath{stroke,fill}%
}%
\begin{pgfscope}%
\pgfsys@transformshift{2.804627in}{0.552778in}%
\pgfsys@useobject{currentmarker}{}%
\end{pgfscope}%
\end{pgfscope}%
\begin{pgfscope}%
\pgfsetbuttcap%
\pgfsetroundjoin%
\definecolor{currentfill}{rgb}{0.000000,0.000000,0.000000}%
\pgfsetfillcolor{currentfill}%
\pgfsetlinewidth{0.602250pt}%
\definecolor{currentstroke}{rgb}{0.000000,0.000000,0.000000}%
\pgfsetstrokecolor{currentstroke}%
\pgfsetdash{}{0pt}%
\pgfsys@defobject{currentmarker}{\pgfqpoint{0.000000in}{0.000000in}}{\pgfqpoint{0.000000in}{0.027778in}}{%
\pgfpathmoveto{\pgfqpoint{0.000000in}{0.000000in}}%
\pgfpathlineto{\pgfqpoint{0.000000in}{0.027778in}}%
\pgfusepath{stroke,fill}%
}%
\begin{pgfscope}%
\pgfsys@transformshift{2.804627in}{3.801389in}%
\pgfsys@useobject{currentmarker}{}%
\end{pgfscope}%
\end{pgfscope}%
\begin{pgfscope}%
\pgfpathrectangle{\pgfqpoint{0.781944in}{0.552778in}}{\pgfqpoint{3.890972in}{3.248611in}}%
\pgfusepath{clip}%
\pgfsetrectcap%
\pgfsetroundjoin%
\pgfsetlinewidth{0.803000pt}%
\definecolor{currentstroke}{rgb}{0.690196,0.690196,0.690196}%
\pgfsetstrokecolor{currentstroke}%
\pgfsetstrokeopacity{0.300000}%
\pgfsetdash{}{0pt}%
\pgfpathmoveto{\pgfqpoint{2.843550in}{0.552778in}}%
\pgfpathlineto{\pgfqpoint{2.843550in}{3.801389in}}%
\pgfusepath{stroke}%
\end{pgfscope}%
\begin{pgfscope}%
\pgfsetbuttcap%
\pgfsetroundjoin%
\definecolor{currentfill}{rgb}{0.000000,0.000000,0.000000}%
\pgfsetfillcolor{currentfill}%
\pgfsetlinewidth{0.602250pt}%
\definecolor{currentstroke}{rgb}{0.000000,0.000000,0.000000}%
\pgfsetstrokecolor{currentstroke}%
\pgfsetdash{}{0pt}%
\pgfsys@defobject{currentmarker}{\pgfqpoint{0.000000in}{-0.027778in}}{\pgfqpoint{0.000000in}{0.000000in}}{%
\pgfpathmoveto{\pgfqpoint{0.000000in}{0.000000in}}%
\pgfpathlineto{\pgfqpoint{0.000000in}{-0.027778in}}%
\pgfusepath{stroke,fill}%
}%
\begin{pgfscope}%
\pgfsys@transformshift{2.843550in}{0.552778in}%
\pgfsys@useobject{currentmarker}{}%
\end{pgfscope}%
\end{pgfscope}%
\begin{pgfscope}%
\pgfsetbuttcap%
\pgfsetroundjoin%
\definecolor{currentfill}{rgb}{0.000000,0.000000,0.000000}%
\pgfsetfillcolor{currentfill}%
\pgfsetlinewidth{0.602250pt}%
\definecolor{currentstroke}{rgb}{0.000000,0.000000,0.000000}%
\pgfsetstrokecolor{currentstroke}%
\pgfsetdash{}{0pt}%
\pgfsys@defobject{currentmarker}{\pgfqpoint{0.000000in}{0.000000in}}{\pgfqpoint{0.000000in}{0.027778in}}{%
\pgfpathmoveto{\pgfqpoint{0.000000in}{0.000000in}}%
\pgfpathlineto{\pgfqpoint{0.000000in}{0.027778in}}%
\pgfusepath{stroke,fill}%
}%
\begin{pgfscope}%
\pgfsys@transformshift{2.843550in}{3.801389in}%
\pgfsys@useobject{currentmarker}{}%
\end{pgfscope}%
\end{pgfscope}%
\begin{pgfscope}%
\pgfpathrectangle{\pgfqpoint{0.781944in}{0.552778in}}{\pgfqpoint{3.890972in}{3.248611in}}%
\pgfusepath{clip}%
\pgfsetrectcap%
\pgfsetroundjoin%
\pgfsetlinewidth{0.803000pt}%
\definecolor{currentstroke}{rgb}{0.690196,0.690196,0.690196}%
\pgfsetstrokecolor{currentstroke}%
\pgfsetstrokeopacity{0.300000}%
\pgfsetdash{}{0pt}%
\pgfpathmoveto{\pgfqpoint{2.882473in}{0.552778in}}%
\pgfpathlineto{\pgfqpoint{2.882473in}{3.801389in}}%
\pgfusepath{stroke}%
\end{pgfscope}%
\begin{pgfscope}%
\pgfsetbuttcap%
\pgfsetroundjoin%
\definecolor{currentfill}{rgb}{0.000000,0.000000,0.000000}%
\pgfsetfillcolor{currentfill}%
\pgfsetlinewidth{0.602250pt}%
\definecolor{currentstroke}{rgb}{0.000000,0.000000,0.000000}%
\pgfsetstrokecolor{currentstroke}%
\pgfsetdash{}{0pt}%
\pgfsys@defobject{currentmarker}{\pgfqpoint{0.000000in}{-0.027778in}}{\pgfqpoint{0.000000in}{0.000000in}}{%
\pgfpathmoveto{\pgfqpoint{0.000000in}{0.000000in}}%
\pgfpathlineto{\pgfqpoint{0.000000in}{-0.027778in}}%
\pgfusepath{stroke,fill}%
}%
\begin{pgfscope}%
\pgfsys@transformshift{2.882473in}{0.552778in}%
\pgfsys@useobject{currentmarker}{}%
\end{pgfscope}%
\end{pgfscope}%
\begin{pgfscope}%
\pgfsetbuttcap%
\pgfsetroundjoin%
\definecolor{currentfill}{rgb}{0.000000,0.000000,0.000000}%
\pgfsetfillcolor{currentfill}%
\pgfsetlinewidth{0.602250pt}%
\definecolor{currentstroke}{rgb}{0.000000,0.000000,0.000000}%
\pgfsetstrokecolor{currentstroke}%
\pgfsetdash{}{0pt}%
\pgfsys@defobject{currentmarker}{\pgfqpoint{0.000000in}{0.000000in}}{\pgfqpoint{0.000000in}{0.027778in}}{%
\pgfpathmoveto{\pgfqpoint{0.000000in}{0.000000in}}%
\pgfpathlineto{\pgfqpoint{0.000000in}{0.027778in}}%
\pgfusepath{stroke,fill}%
}%
\begin{pgfscope}%
\pgfsys@transformshift{2.882473in}{3.801389in}%
\pgfsys@useobject{currentmarker}{}%
\end{pgfscope}%
\end{pgfscope}%
\begin{pgfscope}%
\pgfpathrectangle{\pgfqpoint{0.781944in}{0.552778in}}{\pgfqpoint{3.890972in}{3.248611in}}%
\pgfusepath{clip}%
\pgfsetrectcap%
\pgfsetroundjoin%
\pgfsetlinewidth{0.803000pt}%
\definecolor{currentstroke}{rgb}{0.690196,0.690196,0.690196}%
\pgfsetstrokecolor{currentstroke}%
\pgfsetstrokeopacity{0.300000}%
\pgfsetdash{}{0pt}%
\pgfpathmoveto{\pgfqpoint{2.921395in}{0.552778in}}%
\pgfpathlineto{\pgfqpoint{2.921395in}{3.801389in}}%
\pgfusepath{stroke}%
\end{pgfscope}%
\begin{pgfscope}%
\pgfsetbuttcap%
\pgfsetroundjoin%
\definecolor{currentfill}{rgb}{0.000000,0.000000,0.000000}%
\pgfsetfillcolor{currentfill}%
\pgfsetlinewidth{0.602250pt}%
\definecolor{currentstroke}{rgb}{0.000000,0.000000,0.000000}%
\pgfsetstrokecolor{currentstroke}%
\pgfsetdash{}{0pt}%
\pgfsys@defobject{currentmarker}{\pgfqpoint{0.000000in}{-0.027778in}}{\pgfqpoint{0.000000in}{0.000000in}}{%
\pgfpathmoveto{\pgfqpoint{0.000000in}{0.000000in}}%
\pgfpathlineto{\pgfqpoint{0.000000in}{-0.027778in}}%
\pgfusepath{stroke,fill}%
}%
\begin{pgfscope}%
\pgfsys@transformshift{2.921395in}{0.552778in}%
\pgfsys@useobject{currentmarker}{}%
\end{pgfscope}%
\end{pgfscope}%
\begin{pgfscope}%
\pgfsetbuttcap%
\pgfsetroundjoin%
\definecolor{currentfill}{rgb}{0.000000,0.000000,0.000000}%
\pgfsetfillcolor{currentfill}%
\pgfsetlinewidth{0.602250pt}%
\definecolor{currentstroke}{rgb}{0.000000,0.000000,0.000000}%
\pgfsetstrokecolor{currentstroke}%
\pgfsetdash{}{0pt}%
\pgfsys@defobject{currentmarker}{\pgfqpoint{0.000000in}{0.000000in}}{\pgfqpoint{0.000000in}{0.027778in}}{%
\pgfpathmoveto{\pgfqpoint{0.000000in}{0.000000in}}%
\pgfpathlineto{\pgfqpoint{0.000000in}{0.027778in}}%
\pgfusepath{stroke,fill}%
}%
\begin{pgfscope}%
\pgfsys@transformshift{2.921395in}{3.801389in}%
\pgfsys@useobject{currentmarker}{}%
\end{pgfscope}%
\end{pgfscope}%
\begin{pgfscope}%
\pgfpathrectangle{\pgfqpoint{0.781944in}{0.552778in}}{\pgfqpoint{3.890972in}{3.248611in}}%
\pgfusepath{clip}%
\pgfsetrectcap%
\pgfsetroundjoin%
\pgfsetlinewidth{0.803000pt}%
\definecolor{currentstroke}{rgb}{0.690196,0.690196,0.690196}%
\pgfsetstrokecolor{currentstroke}%
\pgfsetstrokeopacity{0.300000}%
\pgfsetdash{}{0pt}%
\pgfpathmoveto{\pgfqpoint{2.960318in}{0.552778in}}%
\pgfpathlineto{\pgfqpoint{2.960318in}{3.801389in}}%
\pgfusepath{stroke}%
\end{pgfscope}%
\begin{pgfscope}%
\pgfsetbuttcap%
\pgfsetroundjoin%
\definecolor{currentfill}{rgb}{0.000000,0.000000,0.000000}%
\pgfsetfillcolor{currentfill}%
\pgfsetlinewidth{0.602250pt}%
\definecolor{currentstroke}{rgb}{0.000000,0.000000,0.000000}%
\pgfsetstrokecolor{currentstroke}%
\pgfsetdash{}{0pt}%
\pgfsys@defobject{currentmarker}{\pgfqpoint{0.000000in}{-0.027778in}}{\pgfqpoint{0.000000in}{0.000000in}}{%
\pgfpathmoveto{\pgfqpoint{0.000000in}{0.000000in}}%
\pgfpathlineto{\pgfqpoint{0.000000in}{-0.027778in}}%
\pgfusepath{stroke,fill}%
}%
\begin{pgfscope}%
\pgfsys@transformshift{2.960318in}{0.552778in}%
\pgfsys@useobject{currentmarker}{}%
\end{pgfscope}%
\end{pgfscope}%
\begin{pgfscope}%
\pgfsetbuttcap%
\pgfsetroundjoin%
\definecolor{currentfill}{rgb}{0.000000,0.000000,0.000000}%
\pgfsetfillcolor{currentfill}%
\pgfsetlinewidth{0.602250pt}%
\definecolor{currentstroke}{rgb}{0.000000,0.000000,0.000000}%
\pgfsetstrokecolor{currentstroke}%
\pgfsetdash{}{0pt}%
\pgfsys@defobject{currentmarker}{\pgfqpoint{0.000000in}{0.000000in}}{\pgfqpoint{0.000000in}{0.027778in}}{%
\pgfpathmoveto{\pgfqpoint{0.000000in}{0.000000in}}%
\pgfpathlineto{\pgfqpoint{0.000000in}{0.027778in}}%
\pgfusepath{stroke,fill}%
}%
\begin{pgfscope}%
\pgfsys@transformshift{2.960318in}{3.801389in}%
\pgfsys@useobject{currentmarker}{}%
\end{pgfscope}%
\end{pgfscope}%
\begin{pgfscope}%
\pgfpathrectangle{\pgfqpoint{0.781944in}{0.552778in}}{\pgfqpoint{3.890972in}{3.248611in}}%
\pgfusepath{clip}%
\pgfsetrectcap%
\pgfsetroundjoin%
\pgfsetlinewidth{0.803000pt}%
\definecolor{currentstroke}{rgb}{0.690196,0.690196,0.690196}%
\pgfsetstrokecolor{currentstroke}%
\pgfsetstrokeopacity{0.300000}%
\pgfsetdash{}{0pt}%
\pgfpathmoveto{\pgfqpoint{2.999241in}{0.552778in}}%
\pgfpathlineto{\pgfqpoint{2.999241in}{3.801389in}}%
\pgfusepath{stroke}%
\end{pgfscope}%
\begin{pgfscope}%
\pgfsetbuttcap%
\pgfsetroundjoin%
\definecolor{currentfill}{rgb}{0.000000,0.000000,0.000000}%
\pgfsetfillcolor{currentfill}%
\pgfsetlinewidth{0.602250pt}%
\definecolor{currentstroke}{rgb}{0.000000,0.000000,0.000000}%
\pgfsetstrokecolor{currentstroke}%
\pgfsetdash{}{0pt}%
\pgfsys@defobject{currentmarker}{\pgfqpoint{0.000000in}{-0.027778in}}{\pgfqpoint{0.000000in}{0.000000in}}{%
\pgfpathmoveto{\pgfqpoint{0.000000in}{0.000000in}}%
\pgfpathlineto{\pgfqpoint{0.000000in}{-0.027778in}}%
\pgfusepath{stroke,fill}%
}%
\begin{pgfscope}%
\pgfsys@transformshift{2.999241in}{0.552778in}%
\pgfsys@useobject{currentmarker}{}%
\end{pgfscope}%
\end{pgfscope}%
\begin{pgfscope}%
\pgfsetbuttcap%
\pgfsetroundjoin%
\definecolor{currentfill}{rgb}{0.000000,0.000000,0.000000}%
\pgfsetfillcolor{currentfill}%
\pgfsetlinewidth{0.602250pt}%
\definecolor{currentstroke}{rgb}{0.000000,0.000000,0.000000}%
\pgfsetstrokecolor{currentstroke}%
\pgfsetdash{}{0pt}%
\pgfsys@defobject{currentmarker}{\pgfqpoint{0.000000in}{0.000000in}}{\pgfqpoint{0.000000in}{0.027778in}}{%
\pgfpathmoveto{\pgfqpoint{0.000000in}{0.000000in}}%
\pgfpathlineto{\pgfqpoint{0.000000in}{0.027778in}}%
\pgfusepath{stroke,fill}%
}%
\begin{pgfscope}%
\pgfsys@transformshift{2.999241in}{3.801389in}%
\pgfsys@useobject{currentmarker}{}%
\end{pgfscope}%
\end{pgfscope}%
\begin{pgfscope}%
\pgfpathrectangle{\pgfqpoint{0.781944in}{0.552778in}}{\pgfqpoint{3.890972in}{3.248611in}}%
\pgfusepath{clip}%
\pgfsetrectcap%
\pgfsetroundjoin%
\pgfsetlinewidth{0.803000pt}%
\definecolor{currentstroke}{rgb}{0.690196,0.690196,0.690196}%
\pgfsetstrokecolor{currentstroke}%
\pgfsetstrokeopacity{0.300000}%
\pgfsetdash{}{0pt}%
\pgfpathmoveto{\pgfqpoint{3.038163in}{0.552778in}}%
\pgfpathlineto{\pgfqpoint{3.038163in}{3.801389in}}%
\pgfusepath{stroke}%
\end{pgfscope}%
\begin{pgfscope}%
\pgfsetbuttcap%
\pgfsetroundjoin%
\definecolor{currentfill}{rgb}{0.000000,0.000000,0.000000}%
\pgfsetfillcolor{currentfill}%
\pgfsetlinewidth{0.602250pt}%
\definecolor{currentstroke}{rgb}{0.000000,0.000000,0.000000}%
\pgfsetstrokecolor{currentstroke}%
\pgfsetdash{}{0pt}%
\pgfsys@defobject{currentmarker}{\pgfqpoint{0.000000in}{-0.027778in}}{\pgfqpoint{0.000000in}{0.000000in}}{%
\pgfpathmoveto{\pgfqpoint{0.000000in}{0.000000in}}%
\pgfpathlineto{\pgfqpoint{0.000000in}{-0.027778in}}%
\pgfusepath{stroke,fill}%
}%
\begin{pgfscope}%
\pgfsys@transformshift{3.038163in}{0.552778in}%
\pgfsys@useobject{currentmarker}{}%
\end{pgfscope}%
\end{pgfscope}%
\begin{pgfscope}%
\pgfsetbuttcap%
\pgfsetroundjoin%
\definecolor{currentfill}{rgb}{0.000000,0.000000,0.000000}%
\pgfsetfillcolor{currentfill}%
\pgfsetlinewidth{0.602250pt}%
\definecolor{currentstroke}{rgb}{0.000000,0.000000,0.000000}%
\pgfsetstrokecolor{currentstroke}%
\pgfsetdash{}{0pt}%
\pgfsys@defobject{currentmarker}{\pgfqpoint{0.000000in}{0.000000in}}{\pgfqpoint{0.000000in}{0.027778in}}{%
\pgfpathmoveto{\pgfqpoint{0.000000in}{0.000000in}}%
\pgfpathlineto{\pgfqpoint{0.000000in}{0.027778in}}%
\pgfusepath{stroke,fill}%
}%
\begin{pgfscope}%
\pgfsys@transformshift{3.038163in}{3.801389in}%
\pgfsys@useobject{currentmarker}{}%
\end{pgfscope}%
\end{pgfscope}%
\begin{pgfscope}%
\pgfpathrectangle{\pgfqpoint{0.781944in}{0.552778in}}{\pgfqpoint{3.890972in}{3.248611in}}%
\pgfusepath{clip}%
\pgfsetrectcap%
\pgfsetroundjoin%
\pgfsetlinewidth{0.803000pt}%
\definecolor{currentstroke}{rgb}{0.690196,0.690196,0.690196}%
\pgfsetstrokecolor{currentstroke}%
\pgfsetstrokeopacity{0.300000}%
\pgfsetdash{}{0pt}%
\pgfpathmoveto{\pgfqpoint{3.077086in}{0.552778in}}%
\pgfpathlineto{\pgfqpoint{3.077086in}{3.801389in}}%
\pgfusepath{stroke}%
\end{pgfscope}%
\begin{pgfscope}%
\pgfsetbuttcap%
\pgfsetroundjoin%
\definecolor{currentfill}{rgb}{0.000000,0.000000,0.000000}%
\pgfsetfillcolor{currentfill}%
\pgfsetlinewidth{0.602250pt}%
\definecolor{currentstroke}{rgb}{0.000000,0.000000,0.000000}%
\pgfsetstrokecolor{currentstroke}%
\pgfsetdash{}{0pt}%
\pgfsys@defobject{currentmarker}{\pgfqpoint{0.000000in}{-0.027778in}}{\pgfqpoint{0.000000in}{0.000000in}}{%
\pgfpathmoveto{\pgfqpoint{0.000000in}{0.000000in}}%
\pgfpathlineto{\pgfqpoint{0.000000in}{-0.027778in}}%
\pgfusepath{stroke,fill}%
}%
\begin{pgfscope}%
\pgfsys@transformshift{3.077086in}{0.552778in}%
\pgfsys@useobject{currentmarker}{}%
\end{pgfscope}%
\end{pgfscope}%
\begin{pgfscope}%
\pgfsetbuttcap%
\pgfsetroundjoin%
\definecolor{currentfill}{rgb}{0.000000,0.000000,0.000000}%
\pgfsetfillcolor{currentfill}%
\pgfsetlinewidth{0.602250pt}%
\definecolor{currentstroke}{rgb}{0.000000,0.000000,0.000000}%
\pgfsetstrokecolor{currentstroke}%
\pgfsetdash{}{0pt}%
\pgfsys@defobject{currentmarker}{\pgfqpoint{0.000000in}{0.000000in}}{\pgfqpoint{0.000000in}{0.027778in}}{%
\pgfpathmoveto{\pgfqpoint{0.000000in}{0.000000in}}%
\pgfpathlineto{\pgfqpoint{0.000000in}{0.027778in}}%
\pgfusepath{stroke,fill}%
}%
\begin{pgfscope}%
\pgfsys@transformshift{3.077086in}{3.801389in}%
\pgfsys@useobject{currentmarker}{}%
\end{pgfscope}%
\end{pgfscope}%
\begin{pgfscope}%
\pgfpathrectangle{\pgfqpoint{0.781944in}{0.552778in}}{\pgfqpoint{3.890972in}{3.248611in}}%
\pgfusepath{clip}%
\pgfsetrectcap%
\pgfsetroundjoin%
\pgfsetlinewidth{0.803000pt}%
\definecolor{currentstroke}{rgb}{0.690196,0.690196,0.690196}%
\pgfsetstrokecolor{currentstroke}%
\pgfsetstrokeopacity{0.300000}%
\pgfsetdash{}{0pt}%
\pgfpathmoveto{\pgfqpoint{3.154932in}{0.552778in}}%
\pgfpathlineto{\pgfqpoint{3.154932in}{3.801389in}}%
\pgfusepath{stroke}%
\end{pgfscope}%
\begin{pgfscope}%
\pgfsetbuttcap%
\pgfsetroundjoin%
\definecolor{currentfill}{rgb}{0.000000,0.000000,0.000000}%
\pgfsetfillcolor{currentfill}%
\pgfsetlinewidth{0.602250pt}%
\definecolor{currentstroke}{rgb}{0.000000,0.000000,0.000000}%
\pgfsetstrokecolor{currentstroke}%
\pgfsetdash{}{0pt}%
\pgfsys@defobject{currentmarker}{\pgfqpoint{0.000000in}{-0.027778in}}{\pgfqpoint{0.000000in}{0.000000in}}{%
\pgfpathmoveto{\pgfqpoint{0.000000in}{0.000000in}}%
\pgfpathlineto{\pgfqpoint{0.000000in}{-0.027778in}}%
\pgfusepath{stroke,fill}%
}%
\begin{pgfscope}%
\pgfsys@transformshift{3.154932in}{0.552778in}%
\pgfsys@useobject{currentmarker}{}%
\end{pgfscope}%
\end{pgfscope}%
\begin{pgfscope}%
\pgfsetbuttcap%
\pgfsetroundjoin%
\definecolor{currentfill}{rgb}{0.000000,0.000000,0.000000}%
\pgfsetfillcolor{currentfill}%
\pgfsetlinewidth{0.602250pt}%
\definecolor{currentstroke}{rgb}{0.000000,0.000000,0.000000}%
\pgfsetstrokecolor{currentstroke}%
\pgfsetdash{}{0pt}%
\pgfsys@defobject{currentmarker}{\pgfqpoint{0.000000in}{0.000000in}}{\pgfqpoint{0.000000in}{0.027778in}}{%
\pgfpathmoveto{\pgfqpoint{0.000000in}{0.000000in}}%
\pgfpathlineto{\pgfqpoint{0.000000in}{0.027778in}}%
\pgfusepath{stroke,fill}%
}%
\begin{pgfscope}%
\pgfsys@transformshift{3.154932in}{3.801389in}%
\pgfsys@useobject{currentmarker}{}%
\end{pgfscope}%
\end{pgfscope}%
\begin{pgfscope}%
\pgfpathrectangle{\pgfqpoint{0.781944in}{0.552778in}}{\pgfqpoint{3.890972in}{3.248611in}}%
\pgfusepath{clip}%
\pgfsetrectcap%
\pgfsetroundjoin%
\pgfsetlinewidth{0.803000pt}%
\definecolor{currentstroke}{rgb}{0.690196,0.690196,0.690196}%
\pgfsetstrokecolor{currentstroke}%
\pgfsetstrokeopacity{0.300000}%
\pgfsetdash{}{0pt}%
\pgfpathmoveto{\pgfqpoint{3.193854in}{0.552778in}}%
\pgfpathlineto{\pgfqpoint{3.193854in}{3.801389in}}%
\pgfusepath{stroke}%
\end{pgfscope}%
\begin{pgfscope}%
\pgfsetbuttcap%
\pgfsetroundjoin%
\definecolor{currentfill}{rgb}{0.000000,0.000000,0.000000}%
\pgfsetfillcolor{currentfill}%
\pgfsetlinewidth{0.602250pt}%
\definecolor{currentstroke}{rgb}{0.000000,0.000000,0.000000}%
\pgfsetstrokecolor{currentstroke}%
\pgfsetdash{}{0pt}%
\pgfsys@defobject{currentmarker}{\pgfqpoint{0.000000in}{-0.027778in}}{\pgfqpoint{0.000000in}{0.000000in}}{%
\pgfpathmoveto{\pgfqpoint{0.000000in}{0.000000in}}%
\pgfpathlineto{\pgfqpoint{0.000000in}{-0.027778in}}%
\pgfusepath{stroke,fill}%
}%
\begin{pgfscope}%
\pgfsys@transformshift{3.193854in}{0.552778in}%
\pgfsys@useobject{currentmarker}{}%
\end{pgfscope}%
\end{pgfscope}%
\begin{pgfscope}%
\pgfsetbuttcap%
\pgfsetroundjoin%
\definecolor{currentfill}{rgb}{0.000000,0.000000,0.000000}%
\pgfsetfillcolor{currentfill}%
\pgfsetlinewidth{0.602250pt}%
\definecolor{currentstroke}{rgb}{0.000000,0.000000,0.000000}%
\pgfsetstrokecolor{currentstroke}%
\pgfsetdash{}{0pt}%
\pgfsys@defobject{currentmarker}{\pgfqpoint{0.000000in}{0.000000in}}{\pgfqpoint{0.000000in}{0.027778in}}{%
\pgfpathmoveto{\pgfqpoint{0.000000in}{0.000000in}}%
\pgfpathlineto{\pgfqpoint{0.000000in}{0.027778in}}%
\pgfusepath{stroke,fill}%
}%
\begin{pgfscope}%
\pgfsys@transformshift{3.193854in}{3.801389in}%
\pgfsys@useobject{currentmarker}{}%
\end{pgfscope}%
\end{pgfscope}%
\begin{pgfscope}%
\pgfpathrectangle{\pgfqpoint{0.781944in}{0.552778in}}{\pgfqpoint{3.890972in}{3.248611in}}%
\pgfusepath{clip}%
\pgfsetrectcap%
\pgfsetroundjoin%
\pgfsetlinewidth{0.803000pt}%
\definecolor{currentstroke}{rgb}{0.690196,0.690196,0.690196}%
\pgfsetstrokecolor{currentstroke}%
\pgfsetstrokeopacity{0.300000}%
\pgfsetdash{}{0pt}%
\pgfpathmoveto{\pgfqpoint{3.232777in}{0.552778in}}%
\pgfpathlineto{\pgfqpoint{3.232777in}{3.801389in}}%
\pgfusepath{stroke}%
\end{pgfscope}%
\begin{pgfscope}%
\pgfsetbuttcap%
\pgfsetroundjoin%
\definecolor{currentfill}{rgb}{0.000000,0.000000,0.000000}%
\pgfsetfillcolor{currentfill}%
\pgfsetlinewidth{0.602250pt}%
\definecolor{currentstroke}{rgb}{0.000000,0.000000,0.000000}%
\pgfsetstrokecolor{currentstroke}%
\pgfsetdash{}{0pt}%
\pgfsys@defobject{currentmarker}{\pgfqpoint{0.000000in}{-0.027778in}}{\pgfqpoint{0.000000in}{0.000000in}}{%
\pgfpathmoveto{\pgfqpoint{0.000000in}{0.000000in}}%
\pgfpathlineto{\pgfqpoint{0.000000in}{-0.027778in}}%
\pgfusepath{stroke,fill}%
}%
\begin{pgfscope}%
\pgfsys@transformshift{3.232777in}{0.552778in}%
\pgfsys@useobject{currentmarker}{}%
\end{pgfscope}%
\end{pgfscope}%
\begin{pgfscope}%
\pgfsetbuttcap%
\pgfsetroundjoin%
\definecolor{currentfill}{rgb}{0.000000,0.000000,0.000000}%
\pgfsetfillcolor{currentfill}%
\pgfsetlinewidth{0.602250pt}%
\definecolor{currentstroke}{rgb}{0.000000,0.000000,0.000000}%
\pgfsetstrokecolor{currentstroke}%
\pgfsetdash{}{0pt}%
\pgfsys@defobject{currentmarker}{\pgfqpoint{0.000000in}{0.000000in}}{\pgfqpoint{0.000000in}{0.027778in}}{%
\pgfpathmoveto{\pgfqpoint{0.000000in}{0.000000in}}%
\pgfpathlineto{\pgfqpoint{0.000000in}{0.027778in}}%
\pgfusepath{stroke,fill}%
}%
\begin{pgfscope}%
\pgfsys@transformshift{3.232777in}{3.801389in}%
\pgfsys@useobject{currentmarker}{}%
\end{pgfscope}%
\end{pgfscope}%
\begin{pgfscope}%
\pgfpathrectangle{\pgfqpoint{0.781944in}{0.552778in}}{\pgfqpoint{3.890972in}{3.248611in}}%
\pgfusepath{clip}%
\pgfsetrectcap%
\pgfsetroundjoin%
\pgfsetlinewidth{0.803000pt}%
\definecolor{currentstroke}{rgb}{0.690196,0.690196,0.690196}%
\pgfsetstrokecolor{currentstroke}%
\pgfsetstrokeopacity{0.300000}%
\pgfsetdash{}{0pt}%
\pgfpathmoveto{\pgfqpoint{3.271700in}{0.552778in}}%
\pgfpathlineto{\pgfqpoint{3.271700in}{3.801389in}}%
\pgfusepath{stroke}%
\end{pgfscope}%
\begin{pgfscope}%
\pgfsetbuttcap%
\pgfsetroundjoin%
\definecolor{currentfill}{rgb}{0.000000,0.000000,0.000000}%
\pgfsetfillcolor{currentfill}%
\pgfsetlinewidth{0.602250pt}%
\definecolor{currentstroke}{rgb}{0.000000,0.000000,0.000000}%
\pgfsetstrokecolor{currentstroke}%
\pgfsetdash{}{0pt}%
\pgfsys@defobject{currentmarker}{\pgfqpoint{0.000000in}{-0.027778in}}{\pgfqpoint{0.000000in}{0.000000in}}{%
\pgfpathmoveto{\pgfqpoint{0.000000in}{0.000000in}}%
\pgfpathlineto{\pgfqpoint{0.000000in}{-0.027778in}}%
\pgfusepath{stroke,fill}%
}%
\begin{pgfscope}%
\pgfsys@transformshift{3.271700in}{0.552778in}%
\pgfsys@useobject{currentmarker}{}%
\end{pgfscope}%
\end{pgfscope}%
\begin{pgfscope}%
\pgfsetbuttcap%
\pgfsetroundjoin%
\definecolor{currentfill}{rgb}{0.000000,0.000000,0.000000}%
\pgfsetfillcolor{currentfill}%
\pgfsetlinewidth{0.602250pt}%
\definecolor{currentstroke}{rgb}{0.000000,0.000000,0.000000}%
\pgfsetstrokecolor{currentstroke}%
\pgfsetdash{}{0pt}%
\pgfsys@defobject{currentmarker}{\pgfqpoint{0.000000in}{0.000000in}}{\pgfqpoint{0.000000in}{0.027778in}}{%
\pgfpathmoveto{\pgfqpoint{0.000000in}{0.000000in}}%
\pgfpathlineto{\pgfqpoint{0.000000in}{0.027778in}}%
\pgfusepath{stroke,fill}%
}%
\begin{pgfscope}%
\pgfsys@transformshift{3.271700in}{3.801389in}%
\pgfsys@useobject{currentmarker}{}%
\end{pgfscope}%
\end{pgfscope}%
\begin{pgfscope}%
\pgfpathrectangle{\pgfqpoint{0.781944in}{0.552778in}}{\pgfqpoint{3.890972in}{3.248611in}}%
\pgfusepath{clip}%
\pgfsetrectcap%
\pgfsetroundjoin%
\pgfsetlinewidth{0.803000pt}%
\definecolor{currentstroke}{rgb}{0.690196,0.690196,0.690196}%
\pgfsetstrokecolor{currentstroke}%
\pgfsetstrokeopacity{0.300000}%
\pgfsetdash{}{0pt}%
\pgfpathmoveto{\pgfqpoint{3.310622in}{0.552778in}}%
\pgfpathlineto{\pgfqpoint{3.310622in}{3.801389in}}%
\pgfusepath{stroke}%
\end{pgfscope}%
\begin{pgfscope}%
\pgfsetbuttcap%
\pgfsetroundjoin%
\definecolor{currentfill}{rgb}{0.000000,0.000000,0.000000}%
\pgfsetfillcolor{currentfill}%
\pgfsetlinewidth{0.602250pt}%
\definecolor{currentstroke}{rgb}{0.000000,0.000000,0.000000}%
\pgfsetstrokecolor{currentstroke}%
\pgfsetdash{}{0pt}%
\pgfsys@defobject{currentmarker}{\pgfqpoint{0.000000in}{-0.027778in}}{\pgfqpoint{0.000000in}{0.000000in}}{%
\pgfpathmoveto{\pgfqpoint{0.000000in}{0.000000in}}%
\pgfpathlineto{\pgfqpoint{0.000000in}{-0.027778in}}%
\pgfusepath{stroke,fill}%
}%
\begin{pgfscope}%
\pgfsys@transformshift{3.310622in}{0.552778in}%
\pgfsys@useobject{currentmarker}{}%
\end{pgfscope}%
\end{pgfscope}%
\begin{pgfscope}%
\pgfsetbuttcap%
\pgfsetroundjoin%
\definecolor{currentfill}{rgb}{0.000000,0.000000,0.000000}%
\pgfsetfillcolor{currentfill}%
\pgfsetlinewidth{0.602250pt}%
\definecolor{currentstroke}{rgb}{0.000000,0.000000,0.000000}%
\pgfsetstrokecolor{currentstroke}%
\pgfsetdash{}{0pt}%
\pgfsys@defobject{currentmarker}{\pgfqpoint{0.000000in}{0.000000in}}{\pgfqpoint{0.000000in}{0.027778in}}{%
\pgfpathmoveto{\pgfqpoint{0.000000in}{0.000000in}}%
\pgfpathlineto{\pgfqpoint{0.000000in}{0.027778in}}%
\pgfusepath{stroke,fill}%
}%
\begin{pgfscope}%
\pgfsys@transformshift{3.310622in}{3.801389in}%
\pgfsys@useobject{currentmarker}{}%
\end{pgfscope}%
\end{pgfscope}%
\begin{pgfscope}%
\pgfpathrectangle{\pgfqpoint{0.781944in}{0.552778in}}{\pgfqpoint{3.890972in}{3.248611in}}%
\pgfusepath{clip}%
\pgfsetrectcap%
\pgfsetroundjoin%
\pgfsetlinewidth{0.803000pt}%
\definecolor{currentstroke}{rgb}{0.690196,0.690196,0.690196}%
\pgfsetstrokecolor{currentstroke}%
\pgfsetstrokeopacity{0.300000}%
\pgfsetdash{}{0pt}%
\pgfpathmoveto{\pgfqpoint{3.349545in}{0.552778in}}%
\pgfpathlineto{\pgfqpoint{3.349545in}{3.801389in}}%
\pgfusepath{stroke}%
\end{pgfscope}%
\begin{pgfscope}%
\pgfsetbuttcap%
\pgfsetroundjoin%
\definecolor{currentfill}{rgb}{0.000000,0.000000,0.000000}%
\pgfsetfillcolor{currentfill}%
\pgfsetlinewidth{0.602250pt}%
\definecolor{currentstroke}{rgb}{0.000000,0.000000,0.000000}%
\pgfsetstrokecolor{currentstroke}%
\pgfsetdash{}{0pt}%
\pgfsys@defobject{currentmarker}{\pgfqpoint{0.000000in}{-0.027778in}}{\pgfqpoint{0.000000in}{0.000000in}}{%
\pgfpathmoveto{\pgfqpoint{0.000000in}{0.000000in}}%
\pgfpathlineto{\pgfqpoint{0.000000in}{-0.027778in}}%
\pgfusepath{stroke,fill}%
}%
\begin{pgfscope}%
\pgfsys@transformshift{3.349545in}{0.552778in}%
\pgfsys@useobject{currentmarker}{}%
\end{pgfscope}%
\end{pgfscope}%
\begin{pgfscope}%
\pgfsetbuttcap%
\pgfsetroundjoin%
\definecolor{currentfill}{rgb}{0.000000,0.000000,0.000000}%
\pgfsetfillcolor{currentfill}%
\pgfsetlinewidth{0.602250pt}%
\definecolor{currentstroke}{rgb}{0.000000,0.000000,0.000000}%
\pgfsetstrokecolor{currentstroke}%
\pgfsetdash{}{0pt}%
\pgfsys@defobject{currentmarker}{\pgfqpoint{0.000000in}{0.000000in}}{\pgfqpoint{0.000000in}{0.027778in}}{%
\pgfpathmoveto{\pgfqpoint{0.000000in}{0.000000in}}%
\pgfpathlineto{\pgfqpoint{0.000000in}{0.027778in}}%
\pgfusepath{stroke,fill}%
}%
\begin{pgfscope}%
\pgfsys@transformshift{3.349545in}{3.801389in}%
\pgfsys@useobject{currentmarker}{}%
\end{pgfscope}%
\end{pgfscope}%
\begin{pgfscope}%
\pgfpathrectangle{\pgfqpoint{0.781944in}{0.552778in}}{\pgfqpoint{3.890972in}{3.248611in}}%
\pgfusepath{clip}%
\pgfsetrectcap%
\pgfsetroundjoin%
\pgfsetlinewidth{0.803000pt}%
\definecolor{currentstroke}{rgb}{0.690196,0.690196,0.690196}%
\pgfsetstrokecolor{currentstroke}%
\pgfsetstrokeopacity{0.300000}%
\pgfsetdash{}{0pt}%
\pgfpathmoveto{\pgfqpoint{3.388468in}{0.552778in}}%
\pgfpathlineto{\pgfqpoint{3.388468in}{3.801389in}}%
\pgfusepath{stroke}%
\end{pgfscope}%
\begin{pgfscope}%
\pgfsetbuttcap%
\pgfsetroundjoin%
\definecolor{currentfill}{rgb}{0.000000,0.000000,0.000000}%
\pgfsetfillcolor{currentfill}%
\pgfsetlinewidth{0.602250pt}%
\definecolor{currentstroke}{rgb}{0.000000,0.000000,0.000000}%
\pgfsetstrokecolor{currentstroke}%
\pgfsetdash{}{0pt}%
\pgfsys@defobject{currentmarker}{\pgfqpoint{0.000000in}{-0.027778in}}{\pgfqpoint{0.000000in}{0.000000in}}{%
\pgfpathmoveto{\pgfqpoint{0.000000in}{0.000000in}}%
\pgfpathlineto{\pgfqpoint{0.000000in}{-0.027778in}}%
\pgfusepath{stroke,fill}%
}%
\begin{pgfscope}%
\pgfsys@transformshift{3.388468in}{0.552778in}%
\pgfsys@useobject{currentmarker}{}%
\end{pgfscope}%
\end{pgfscope}%
\begin{pgfscope}%
\pgfsetbuttcap%
\pgfsetroundjoin%
\definecolor{currentfill}{rgb}{0.000000,0.000000,0.000000}%
\pgfsetfillcolor{currentfill}%
\pgfsetlinewidth{0.602250pt}%
\definecolor{currentstroke}{rgb}{0.000000,0.000000,0.000000}%
\pgfsetstrokecolor{currentstroke}%
\pgfsetdash{}{0pt}%
\pgfsys@defobject{currentmarker}{\pgfqpoint{0.000000in}{0.000000in}}{\pgfqpoint{0.000000in}{0.027778in}}{%
\pgfpathmoveto{\pgfqpoint{0.000000in}{0.000000in}}%
\pgfpathlineto{\pgfqpoint{0.000000in}{0.027778in}}%
\pgfusepath{stroke,fill}%
}%
\begin{pgfscope}%
\pgfsys@transformshift{3.388468in}{3.801389in}%
\pgfsys@useobject{currentmarker}{}%
\end{pgfscope}%
\end{pgfscope}%
\begin{pgfscope}%
\pgfpathrectangle{\pgfqpoint{0.781944in}{0.552778in}}{\pgfqpoint{3.890972in}{3.248611in}}%
\pgfusepath{clip}%
\pgfsetrectcap%
\pgfsetroundjoin%
\pgfsetlinewidth{0.803000pt}%
\definecolor{currentstroke}{rgb}{0.690196,0.690196,0.690196}%
\pgfsetstrokecolor{currentstroke}%
\pgfsetstrokeopacity{0.300000}%
\pgfsetdash{}{0pt}%
\pgfpathmoveto{\pgfqpoint{3.427390in}{0.552778in}}%
\pgfpathlineto{\pgfqpoint{3.427390in}{3.801389in}}%
\pgfusepath{stroke}%
\end{pgfscope}%
\begin{pgfscope}%
\pgfsetbuttcap%
\pgfsetroundjoin%
\definecolor{currentfill}{rgb}{0.000000,0.000000,0.000000}%
\pgfsetfillcolor{currentfill}%
\pgfsetlinewidth{0.602250pt}%
\definecolor{currentstroke}{rgb}{0.000000,0.000000,0.000000}%
\pgfsetstrokecolor{currentstroke}%
\pgfsetdash{}{0pt}%
\pgfsys@defobject{currentmarker}{\pgfqpoint{0.000000in}{-0.027778in}}{\pgfqpoint{0.000000in}{0.000000in}}{%
\pgfpathmoveto{\pgfqpoint{0.000000in}{0.000000in}}%
\pgfpathlineto{\pgfqpoint{0.000000in}{-0.027778in}}%
\pgfusepath{stroke,fill}%
}%
\begin{pgfscope}%
\pgfsys@transformshift{3.427390in}{0.552778in}%
\pgfsys@useobject{currentmarker}{}%
\end{pgfscope}%
\end{pgfscope}%
\begin{pgfscope}%
\pgfsetbuttcap%
\pgfsetroundjoin%
\definecolor{currentfill}{rgb}{0.000000,0.000000,0.000000}%
\pgfsetfillcolor{currentfill}%
\pgfsetlinewidth{0.602250pt}%
\definecolor{currentstroke}{rgb}{0.000000,0.000000,0.000000}%
\pgfsetstrokecolor{currentstroke}%
\pgfsetdash{}{0pt}%
\pgfsys@defobject{currentmarker}{\pgfqpoint{0.000000in}{0.000000in}}{\pgfqpoint{0.000000in}{0.027778in}}{%
\pgfpathmoveto{\pgfqpoint{0.000000in}{0.000000in}}%
\pgfpathlineto{\pgfqpoint{0.000000in}{0.027778in}}%
\pgfusepath{stroke,fill}%
}%
\begin{pgfscope}%
\pgfsys@transformshift{3.427390in}{3.801389in}%
\pgfsys@useobject{currentmarker}{}%
\end{pgfscope}%
\end{pgfscope}%
\begin{pgfscope}%
\pgfpathrectangle{\pgfqpoint{0.781944in}{0.552778in}}{\pgfqpoint{3.890972in}{3.248611in}}%
\pgfusepath{clip}%
\pgfsetrectcap%
\pgfsetroundjoin%
\pgfsetlinewidth{0.803000pt}%
\definecolor{currentstroke}{rgb}{0.690196,0.690196,0.690196}%
\pgfsetstrokecolor{currentstroke}%
\pgfsetstrokeopacity{0.300000}%
\pgfsetdash{}{0pt}%
\pgfpathmoveto{\pgfqpoint{3.466313in}{0.552778in}}%
\pgfpathlineto{\pgfqpoint{3.466313in}{3.801389in}}%
\pgfusepath{stroke}%
\end{pgfscope}%
\begin{pgfscope}%
\pgfsetbuttcap%
\pgfsetroundjoin%
\definecolor{currentfill}{rgb}{0.000000,0.000000,0.000000}%
\pgfsetfillcolor{currentfill}%
\pgfsetlinewidth{0.602250pt}%
\definecolor{currentstroke}{rgb}{0.000000,0.000000,0.000000}%
\pgfsetstrokecolor{currentstroke}%
\pgfsetdash{}{0pt}%
\pgfsys@defobject{currentmarker}{\pgfqpoint{0.000000in}{-0.027778in}}{\pgfqpoint{0.000000in}{0.000000in}}{%
\pgfpathmoveto{\pgfqpoint{0.000000in}{0.000000in}}%
\pgfpathlineto{\pgfqpoint{0.000000in}{-0.027778in}}%
\pgfusepath{stroke,fill}%
}%
\begin{pgfscope}%
\pgfsys@transformshift{3.466313in}{0.552778in}%
\pgfsys@useobject{currentmarker}{}%
\end{pgfscope}%
\end{pgfscope}%
\begin{pgfscope}%
\pgfsetbuttcap%
\pgfsetroundjoin%
\definecolor{currentfill}{rgb}{0.000000,0.000000,0.000000}%
\pgfsetfillcolor{currentfill}%
\pgfsetlinewidth{0.602250pt}%
\definecolor{currentstroke}{rgb}{0.000000,0.000000,0.000000}%
\pgfsetstrokecolor{currentstroke}%
\pgfsetdash{}{0pt}%
\pgfsys@defobject{currentmarker}{\pgfqpoint{0.000000in}{0.000000in}}{\pgfqpoint{0.000000in}{0.027778in}}{%
\pgfpathmoveto{\pgfqpoint{0.000000in}{0.000000in}}%
\pgfpathlineto{\pgfqpoint{0.000000in}{0.027778in}}%
\pgfusepath{stroke,fill}%
}%
\begin{pgfscope}%
\pgfsys@transformshift{3.466313in}{3.801389in}%
\pgfsys@useobject{currentmarker}{}%
\end{pgfscope}%
\end{pgfscope}%
\begin{pgfscope}%
\pgfpathrectangle{\pgfqpoint{0.781944in}{0.552778in}}{\pgfqpoint{3.890972in}{3.248611in}}%
\pgfusepath{clip}%
\pgfsetrectcap%
\pgfsetroundjoin%
\pgfsetlinewidth{0.803000pt}%
\definecolor{currentstroke}{rgb}{0.690196,0.690196,0.690196}%
\pgfsetstrokecolor{currentstroke}%
\pgfsetstrokeopacity{0.300000}%
\pgfsetdash{}{0pt}%
\pgfpathmoveto{\pgfqpoint{3.544158in}{0.552778in}}%
\pgfpathlineto{\pgfqpoint{3.544158in}{3.801389in}}%
\pgfusepath{stroke}%
\end{pgfscope}%
\begin{pgfscope}%
\pgfsetbuttcap%
\pgfsetroundjoin%
\definecolor{currentfill}{rgb}{0.000000,0.000000,0.000000}%
\pgfsetfillcolor{currentfill}%
\pgfsetlinewidth{0.602250pt}%
\definecolor{currentstroke}{rgb}{0.000000,0.000000,0.000000}%
\pgfsetstrokecolor{currentstroke}%
\pgfsetdash{}{0pt}%
\pgfsys@defobject{currentmarker}{\pgfqpoint{0.000000in}{-0.027778in}}{\pgfqpoint{0.000000in}{0.000000in}}{%
\pgfpathmoveto{\pgfqpoint{0.000000in}{0.000000in}}%
\pgfpathlineto{\pgfqpoint{0.000000in}{-0.027778in}}%
\pgfusepath{stroke,fill}%
}%
\begin{pgfscope}%
\pgfsys@transformshift{3.544158in}{0.552778in}%
\pgfsys@useobject{currentmarker}{}%
\end{pgfscope}%
\end{pgfscope}%
\begin{pgfscope}%
\pgfsetbuttcap%
\pgfsetroundjoin%
\definecolor{currentfill}{rgb}{0.000000,0.000000,0.000000}%
\pgfsetfillcolor{currentfill}%
\pgfsetlinewidth{0.602250pt}%
\definecolor{currentstroke}{rgb}{0.000000,0.000000,0.000000}%
\pgfsetstrokecolor{currentstroke}%
\pgfsetdash{}{0pt}%
\pgfsys@defobject{currentmarker}{\pgfqpoint{0.000000in}{0.000000in}}{\pgfqpoint{0.000000in}{0.027778in}}{%
\pgfpathmoveto{\pgfqpoint{0.000000in}{0.000000in}}%
\pgfpathlineto{\pgfqpoint{0.000000in}{0.027778in}}%
\pgfusepath{stroke,fill}%
}%
\begin{pgfscope}%
\pgfsys@transformshift{3.544158in}{3.801389in}%
\pgfsys@useobject{currentmarker}{}%
\end{pgfscope}%
\end{pgfscope}%
\begin{pgfscope}%
\pgfpathrectangle{\pgfqpoint{0.781944in}{0.552778in}}{\pgfqpoint{3.890972in}{3.248611in}}%
\pgfusepath{clip}%
\pgfsetrectcap%
\pgfsetroundjoin%
\pgfsetlinewidth{0.803000pt}%
\definecolor{currentstroke}{rgb}{0.690196,0.690196,0.690196}%
\pgfsetstrokecolor{currentstroke}%
\pgfsetstrokeopacity{0.300000}%
\pgfsetdash{}{0pt}%
\pgfpathmoveto{\pgfqpoint{3.583081in}{0.552778in}}%
\pgfpathlineto{\pgfqpoint{3.583081in}{3.801389in}}%
\pgfusepath{stroke}%
\end{pgfscope}%
\begin{pgfscope}%
\pgfsetbuttcap%
\pgfsetroundjoin%
\definecolor{currentfill}{rgb}{0.000000,0.000000,0.000000}%
\pgfsetfillcolor{currentfill}%
\pgfsetlinewidth{0.602250pt}%
\definecolor{currentstroke}{rgb}{0.000000,0.000000,0.000000}%
\pgfsetstrokecolor{currentstroke}%
\pgfsetdash{}{0pt}%
\pgfsys@defobject{currentmarker}{\pgfqpoint{0.000000in}{-0.027778in}}{\pgfqpoint{0.000000in}{0.000000in}}{%
\pgfpathmoveto{\pgfqpoint{0.000000in}{0.000000in}}%
\pgfpathlineto{\pgfqpoint{0.000000in}{-0.027778in}}%
\pgfusepath{stroke,fill}%
}%
\begin{pgfscope}%
\pgfsys@transformshift{3.583081in}{0.552778in}%
\pgfsys@useobject{currentmarker}{}%
\end{pgfscope}%
\end{pgfscope}%
\begin{pgfscope}%
\pgfsetbuttcap%
\pgfsetroundjoin%
\definecolor{currentfill}{rgb}{0.000000,0.000000,0.000000}%
\pgfsetfillcolor{currentfill}%
\pgfsetlinewidth{0.602250pt}%
\definecolor{currentstroke}{rgb}{0.000000,0.000000,0.000000}%
\pgfsetstrokecolor{currentstroke}%
\pgfsetdash{}{0pt}%
\pgfsys@defobject{currentmarker}{\pgfqpoint{0.000000in}{0.000000in}}{\pgfqpoint{0.000000in}{0.027778in}}{%
\pgfpathmoveto{\pgfqpoint{0.000000in}{0.000000in}}%
\pgfpathlineto{\pgfqpoint{0.000000in}{0.027778in}}%
\pgfusepath{stroke,fill}%
}%
\begin{pgfscope}%
\pgfsys@transformshift{3.583081in}{3.801389in}%
\pgfsys@useobject{currentmarker}{}%
\end{pgfscope}%
\end{pgfscope}%
\begin{pgfscope}%
\pgfpathrectangle{\pgfqpoint{0.781944in}{0.552778in}}{\pgfqpoint{3.890972in}{3.248611in}}%
\pgfusepath{clip}%
\pgfsetrectcap%
\pgfsetroundjoin%
\pgfsetlinewidth{0.803000pt}%
\definecolor{currentstroke}{rgb}{0.690196,0.690196,0.690196}%
\pgfsetstrokecolor{currentstroke}%
\pgfsetstrokeopacity{0.300000}%
\pgfsetdash{}{0pt}%
\pgfpathmoveto{\pgfqpoint{3.622004in}{0.552778in}}%
\pgfpathlineto{\pgfqpoint{3.622004in}{3.801389in}}%
\pgfusepath{stroke}%
\end{pgfscope}%
\begin{pgfscope}%
\pgfsetbuttcap%
\pgfsetroundjoin%
\definecolor{currentfill}{rgb}{0.000000,0.000000,0.000000}%
\pgfsetfillcolor{currentfill}%
\pgfsetlinewidth{0.602250pt}%
\definecolor{currentstroke}{rgb}{0.000000,0.000000,0.000000}%
\pgfsetstrokecolor{currentstroke}%
\pgfsetdash{}{0pt}%
\pgfsys@defobject{currentmarker}{\pgfqpoint{0.000000in}{-0.027778in}}{\pgfqpoint{0.000000in}{0.000000in}}{%
\pgfpathmoveto{\pgfqpoint{0.000000in}{0.000000in}}%
\pgfpathlineto{\pgfqpoint{0.000000in}{-0.027778in}}%
\pgfusepath{stroke,fill}%
}%
\begin{pgfscope}%
\pgfsys@transformshift{3.622004in}{0.552778in}%
\pgfsys@useobject{currentmarker}{}%
\end{pgfscope}%
\end{pgfscope}%
\begin{pgfscope}%
\pgfsetbuttcap%
\pgfsetroundjoin%
\definecolor{currentfill}{rgb}{0.000000,0.000000,0.000000}%
\pgfsetfillcolor{currentfill}%
\pgfsetlinewidth{0.602250pt}%
\definecolor{currentstroke}{rgb}{0.000000,0.000000,0.000000}%
\pgfsetstrokecolor{currentstroke}%
\pgfsetdash{}{0pt}%
\pgfsys@defobject{currentmarker}{\pgfqpoint{0.000000in}{0.000000in}}{\pgfqpoint{0.000000in}{0.027778in}}{%
\pgfpathmoveto{\pgfqpoint{0.000000in}{0.000000in}}%
\pgfpathlineto{\pgfqpoint{0.000000in}{0.027778in}}%
\pgfusepath{stroke,fill}%
}%
\begin{pgfscope}%
\pgfsys@transformshift{3.622004in}{3.801389in}%
\pgfsys@useobject{currentmarker}{}%
\end{pgfscope}%
\end{pgfscope}%
\begin{pgfscope}%
\pgfpathrectangle{\pgfqpoint{0.781944in}{0.552778in}}{\pgfqpoint{3.890972in}{3.248611in}}%
\pgfusepath{clip}%
\pgfsetrectcap%
\pgfsetroundjoin%
\pgfsetlinewidth{0.803000pt}%
\definecolor{currentstroke}{rgb}{0.690196,0.690196,0.690196}%
\pgfsetstrokecolor{currentstroke}%
\pgfsetstrokeopacity{0.300000}%
\pgfsetdash{}{0pt}%
\pgfpathmoveto{\pgfqpoint{3.660927in}{0.552778in}}%
\pgfpathlineto{\pgfqpoint{3.660927in}{3.801389in}}%
\pgfusepath{stroke}%
\end{pgfscope}%
\begin{pgfscope}%
\pgfsetbuttcap%
\pgfsetroundjoin%
\definecolor{currentfill}{rgb}{0.000000,0.000000,0.000000}%
\pgfsetfillcolor{currentfill}%
\pgfsetlinewidth{0.602250pt}%
\definecolor{currentstroke}{rgb}{0.000000,0.000000,0.000000}%
\pgfsetstrokecolor{currentstroke}%
\pgfsetdash{}{0pt}%
\pgfsys@defobject{currentmarker}{\pgfqpoint{0.000000in}{-0.027778in}}{\pgfqpoint{0.000000in}{0.000000in}}{%
\pgfpathmoveto{\pgfqpoint{0.000000in}{0.000000in}}%
\pgfpathlineto{\pgfqpoint{0.000000in}{-0.027778in}}%
\pgfusepath{stroke,fill}%
}%
\begin{pgfscope}%
\pgfsys@transformshift{3.660927in}{0.552778in}%
\pgfsys@useobject{currentmarker}{}%
\end{pgfscope}%
\end{pgfscope}%
\begin{pgfscope}%
\pgfsetbuttcap%
\pgfsetroundjoin%
\definecolor{currentfill}{rgb}{0.000000,0.000000,0.000000}%
\pgfsetfillcolor{currentfill}%
\pgfsetlinewidth{0.602250pt}%
\definecolor{currentstroke}{rgb}{0.000000,0.000000,0.000000}%
\pgfsetstrokecolor{currentstroke}%
\pgfsetdash{}{0pt}%
\pgfsys@defobject{currentmarker}{\pgfqpoint{0.000000in}{0.000000in}}{\pgfqpoint{0.000000in}{0.027778in}}{%
\pgfpathmoveto{\pgfqpoint{0.000000in}{0.000000in}}%
\pgfpathlineto{\pgfqpoint{0.000000in}{0.027778in}}%
\pgfusepath{stroke,fill}%
}%
\begin{pgfscope}%
\pgfsys@transformshift{3.660927in}{3.801389in}%
\pgfsys@useobject{currentmarker}{}%
\end{pgfscope}%
\end{pgfscope}%
\begin{pgfscope}%
\pgfpathrectangle{\pgfqpoint{0.781944in}{0.552778in}}{\pgfqpoint{3.890972in}{3.248611in}}%
\pgfusepath{clip}%
\pgfsetrectcap%
\pgfsetroundjoin%
\pgfsetlinewidth{0.803000pt}%
\definecolor{currentstroke}{rgb}{0.690196,0.690196,0.690196}%
\pgfsetstrokecolor{currentstroke}%
\pgfsetstrokeopacity{0.300000}%
\pgfsetdash{}{0pt}%
\pgfpathmoveto{\pgfqpoint{3.699849in}{0.552778in}}%
\pgfpathlineto{\pgfqpoint{3.699849in}{3.801389in}}%
\pgfusepath{stroke}%
\end{pgfscope}%
\begin{pgfscope}%
\pgfsetbuttcap%
\pgfsetroundjoin%
\definecolor{currentfill}{rgb}{0.000000,0.000000,0.000000}%
\pgfsetfillcolor{currentfill}%
\pgfsetlinewidth{0.602250pt}%
\definecolor{currentstroke}{rgb}{0.000000,0.000000,0.000000}%
\pgfsetstrokecolor{currentstroke}%
\pgfsetdash{}{0pt}%
\pgfsys@defobject{currentmarker}{\pgfqpoint{0.000000in}{-0.027778in}}{\pgfqpoint{0.000000in}{0.000000in}}{%
\pgfpathmoveto{\pgfqpoint{0.000000in}{0.000000in}}%
\pgfpathlineto{\pgfqpoint{0.000000in}{-0.027778in}}%
\pgfusepath{stroke,fill}%
}%
\begin{pgfscope}%
\pgfsys@transformshift{3.699849in}{0.552778in}%
\pgfsys@useobject{currentmarker}{}%
\end{pgfscope}%
\end{pgfscope}%
\begin{pgfscope}%
\pgfsetbuttcap%
\pgfsetroundjoin%
\definecolor{currentfill}{rgb}{0.000000,0.000000,0.000000}%
\pgfsetfillcolor{currentfill}%
\pgfsetlinewidth{0.602250pt}%
\definecolor{currentstroke}{rgb}{0.000000,0.000000,0.000000}%
\pgfsetstrokecolor{currentstroke}%
\pgfsetdash{}{0pt}%
\pgfsys@defobject{currentmarker}{\pgfqpoint{0.000000in}{0.000000in}}{\pgfqpoint{0.000000in}{0.027778in}}{%
\pgfpathmoveto{\pgfqpoint{0.000000in}{0.000000in}}%
\pgfpathlineto{\pgfqpoint{0.000000in}{0.027778in}}%
\pgfusepath{stroke,fill}%
}%
\begin{pgfscope}%
\pgfsys@transformshift{3.699849in}{3.801389in}%
\pgfsys@useobject{currentmarker}{}%
\end{pgfscope}%
\end{pgfscope}%
\begin{pgfscope}%
\pgfpathrectangle{\pgfqpoint{0.781944in}{0.552778in}}{\pgfqpoint{3.890972in}{3.248611in}}%
\pgfusepath{clip}%
\pgfsetrectcap%
\pgfsetroundjoin%
\pgfsetlinewidth{0.803000pt}%
\definecolor{currentstroke}{rgb}{0.690196,0.690196,0.690196}%
\pgfsetstrokecolor{currentstroke}%
\pgfsetstrokeopacity{0.300000}%
\pgfsetdash{}{0pt}%
\pgfpathmoveto{\pgfqpoint{3.738772in}{0.552778in}}%
\pgfpathlineto{\pgfqpoint{3.738772in}{3.801389in}}%
\pgfusepath{stroke}%
\end{pgfscope}%
\begin{pgfscope}%
\pgfsetbuttcap%
\pgfsetroundjoin%
\definecolor{currentfill}{rgb}{0.000000,0.000000,0.000000}%
\pgfsetfillcolor{currentfill}%
\pgfsetlinewidth{0.602250pt}%
\definecolor{currentstroke}{rgb}{0.000000,0.000000,0.000000}%
\pgfsetstrokecolor{currentstroke}%
\pgfsetdash{}{0pt}%
\pgfsys@defobject{currentmarker}{\pgfqpoint{0.000000in}{-0.027778in}}{\pgfqpoint{0.000000in}{0.000000in}}{%
\pgfpathmoveto{\pgfqpoint{0.000000in}{0.000000in}}%
\pgfpathlineto{\pgfqpoint{0.000000in}{-0.027778in}}%
\pgfusepath{stroke,fill}%
}%
\begin{pgfscope}%
\pgfsys@transformshift{3.738772in}{0.552778in}%
\pgfsys@useobject{currentmarker}{}%
\end{pgfscope}%
\end{pgfscope}%
\begin{pgfscope}%
\pgfsetbuttcap%
\pgfsetroundjoin%
\definecolor{currentfill}{rgb}{0.000000,0.000000,0.000000}%
\pgfsetfillcolor{currentfill}%
\pgfsetlinewidth{0.602250pt}%
\definecolor{currentstroke}{rgb}{0.000000,0.000000,0.000000}%
\pgfsetstrokecolor{currentstroke}%
\pgfsetdash{}{0pt}%
\pgfsys@defobject{currentmarker}{\pgfqpoint{0.000000in}{0.000000in}}{\pgfqpoint{0.000000in}{0.027778in}}{%
\pgfpathmoveto{\pgfqpoint{0.000000in}{0.000000in}}%
\pgfpathlineto{\pgfqpoint{0.000000in}{0.027778in}}%
\pgfusepath{stroke,fill}%
}%
\begin{pgfscope}%
\pgfsys@transformshift{3.738772in}{3.801389in}%
\pgfsys@useobject{currentmarker}{}%
\end{pgfscope}%
\end{pgfscope}%
\begin{pgfscope}%
\pgfpathrectangle{\pgfqpoint{0.781944in}{0.552778in}}{\pgfqpoint{3.890972in}{3.248611in}}%
\pgfusepath{clip}%
\pgfsetrectcap%
\pgfsetroundjoin%
\pgfsetlinewidth{0.803000pt}%
\definecolor{currentstroke}{rgb}{0.690196,0.690196,0.690196}%
\pgfsetstrokecolor{currentstroke}%
\pgfsetstrokeopacity{0.300000}%
\pgfsetdash{}{0pt}%
\pgfpathmoveto{\pgfqpoint{3.777695in}{0.552778in}}%
\pgfpathlineto{\pgfqpoint{3.777695in}{3.801389in}}%
\pgfusepath{stroke}%
\end{pgfscope}%
\begin{pgfscope}%
\pgfsetbuttcap%
\pgfsetroundjoin%
\definecolor{currentfill}{rgb}{0.000000,0.000000,0.000000}%
\pgfsetfillcolor{currentfill}%
\pgfsetlinewidth{0.602250pt}%
\definecolor{currentstroke}{rgb}{0.000000,0.000000,0.000000}%
\pgfsetstrokecolor{currentstroke}%
\pgfsetdash{}{0pt}%
\pgfsys@defobject{currentmarker}{\pgfqpoint{0.000000in}{-0.027778in}}{\pgfqpoint{0.000000in}{0.000000in}}{%
\pgfpathmoveto{\pgfqpoint{0.000000in}{0.000000in}}%
\pgfpathlineto{\pgfqpoint{0.000000in}{-0.027778in}}%
\pgfusepath{stroke,fill}%
}%
\begin{pgfscope}%
\pgfsys@transformshift{3.777695in}{0.552778in}%
\pgfsys@useobject{currentmarker}{}%
\end{pgfscope}%
\end{pgfscope}%
\begin{pgfscope}%
\pgfsetbuttcap%
\pgfsetroundjoin%
\definecolor{currentfill}{rgb}{0.000000,0.000000,0.000000}%
\pgfsetfillcolor{currentfill}%
\pgfsetlinewidth{0.602250pt}%
\definecolor{currentstroke}{rgb}{0.000000,0.000000,0.000000}%
\pgfsetstrokecolor{currentstroke}%
\pgfsetdash{}{0pt}%
\pgfsys@defobject{currentmarker}{\pgfqpoint{0.000000in}{0.000000in}}{\pgfqpoint{0.000000in}{0.027778in}}{%
\pgfpathmoveto{\pgfqpoint{0.000000in}{0.000000in}}%
\pgfpathlineto{\pgfqpoint{0.000000in}{0.027778in}}%
\pgfusepath{stroke,fill}%
}%
\begin{pgfscope}%
\pgfsys@transformshift{3.777695in}{3.801389in}%
\pgfsys@useobject{currentmarker}{}%
\end{pgfscope}%
\end{pgfscope}%
\begin{pgfscope}%
\pgfpathrectangle{\pgfqpoint{0.781944in}{0.552778in}}{\pgfqpoint{3.890972in}{3.248611in}}%
\pgfusepath{clip}%
\pgfsetrectcap%
\pgfsetroundjoin%
\pgfsetlinewidth{0.803000pt}%
\definecolor{currentstroke}{rgb}{0.690196,0.690196,0.690196}%
\pgfsetstrokecolor{currentstroke}%
\pgfsetstrokeopacity{0.300000}%
\pgfsetdash{}{0pt}%
\pgfpathmoveto{\pgfqpoint{3.816617in}{0.552778in}}%
\pgfpathlineto{\pgfqpoint{3.816617in}{3.801389in}}%
\pgfusepath{stroke}%
\end{pgfscope}%
\begin{pgfscope}%
\pgfsetbuttcap%
\pgfsetroundjoin%
\definecolor{currentfill}{rgb}{0.000000,0.000000,0.000000}%
\pgfsetfillcolor{currentfill}%
\pgfsetlinewidth{0.602250pt}%
\definecolor{currentstroke}{rgb}{0.000000,0.000000,0.000000}%
\pgfsetstrokecolor{currentstroke}%
\pgfsetdash{}{0pt}%
\pgfsys@defobject{currentmarker}{\pgfqpoint{0.000000in}{-0.027778in}}{\pgfqpoint{0.000000in}{0.000000in}}{%
\pgfpathmoveto{\pgfqpoint{0.000000in}{0.000000in}}%
\pgfpathlineto{\pgfqpoint{0.000000in}{-0.027778in}}%
\pgfusepath{stroke,fill}%
}%
\begin{pgfscope}%
\pgfsys@transformshift{3.816617in}{0.552778in}%
\pgfsys@useobject{currentmarker}{}%
\end{pgfscope}%
\end{pgfscope}%
\begin{pgfscope}%
\pgfsetbuttcap%
\pgfsetroundjoin%
\definecolor{currentfill}{rgb}{0.000000,0.000000,0.000000}%
\pgfsetfillcolor{currentfill}%
\pgfsetlinewidth{0.602250pt}%
\definecolor{currentstroke}{rgb}{0.000000,0.000000,0.000000}%
\pgfsetstrokecolor{currentstroke}%
\pgfsetdash{}{0pt}%
\pgfsys@defobject{currentmarker}{\pgfqpoint{0.000000in}{0.000000in}}{\pgfqpoint{0.000000in}{0.027778in}}{%
\pgfpathmoveto{\pgfqpoint{0.000000in}{0.000000in}}%
\pgfpathlineto{\pgfqpoint{0.000000in}{0.027778in}}%
\pgfusepath{stroke,fill}%
}%
\begin{pgfscope}%
\pgfsys@transformshift{3.816617in}{3.801389in}%
\pgfsys@useobject{currentmarker}{}%
\end{pgfscope}%
\end{pgfscope}%
\begin{pgfscope}%
\pgfpathrectangle{\pgfqpoint{0.781944in}{0.552778in}}{\pgfqpoint{3.890972in}{3.248611in}}%
\pgfusepath{clip}%
\pgfsetrectcap%
\pgfsetroundjoin%
\pgfsetlinewidth{0.803000pt}%
\definecolor{currentstroke}{rgb}{0.690196,0.690196,0.690196}%
\pgfsetstrokecolor{currentstroke}%
\pgfsetstrokeopacity{0.300000}%
\pgfsetdash{}{0pt}%
\pgfpathmoveto{\pgfqpoint{3.855540in}{0.552778in}}%
\pgfpathlineto{\pgfqpoint{3.855540in}{3.801389in}}%
\pgfusepath{stroke}%
\end{pgfscope}%
\begin{pgfscope}%
\pgfsetbuttcap%
\pgfsetroundjoin%
\definecolor{currentfill}{rgb}{0.000000,0.000000,0.000000}%
\pgfsetfillcolor{currentfill}%
\pgfsetlinewidth{0.602250pt}%
\definecolor{currentstroke}{rgb}{0.000000,0.000000,0.000000}%
\pgfsetstrokecolor{currentstroke}%
\pgfsetdash{}{0pt}%
\pgfsys@defobject{currentmarker}{\pgfqpoint{0.000000in}{-0.027778in}}{\pgfqpoint{0.000000in}{0.000000in}}{%
\pgfpathmoveto{\pgfqpoint{0.000000in}{0.000000in}}%
\pgfpathlineto{\pgfqpoint{0.000000in}{-0.027778in}}%
\pgfusepath{stroke,fill}%
}%
\begin{pgfscope}%
\pgfsys@transformshift{3.855540in}{0.552778in}%
\pgfsys@useobject{currentmarker}{}%
\end{pgfscope}%
\end{pgfscope}%
\begin{pgfscope}%
\pgfsetbuttcap%
\pgfsetroundjoin%
\definecolor{currentfill}{rgb}{0.000000,0.000000,0.000000}%
\pgfsetfillcolor{currentfill}%
\pgfsetlinewidth{0.602250pt}%
\definecolor{currentstroke}{rgb}{0.000000,0.000000,0.000000}%
\pgfsetstrokecolor{currentstroke}%
\pgfsetdash{}{0pt}%
\pgfsys@defobject{currentmarker}{\pgfqpoint{0.000000in}{0.000000in}}{\pgfqpoint{0.000000in}{0.027778in}}{%
\pgfpathmoveto{\pgfqpoint{0.000000in}{0.000000in}}%
\pgfpathlineto{\pgfqpoint{0.000000in}{0.027778in}}%
\pgfusepath{stroke,fill}%
}%
\begin{pgfscope}%
\pgfsys@transformshift{3.855540in}{3.801389in}%
\pgfsys@useobject{currentmarker}{}%
\end{pgfscope}%
\end{pgfscope}%
\begin{pgfscope}%
\pgfpathrectangle{\pgfqpoint{0.781944in}{0.552778in}}{\pgfqpoint{3.890972in}{3.248611in}}%
\pgfusepath{clip}%
\pgfsetrectcap%
\pgfsetroundjoin%
\pgfsetlinewidth{0.803000pt}%
\definecolor{currentstroke}{rgb}{0.690196,0.690196,0.690196}%
\pgfsetstrokecolor{currentstroke}%
\pgfsetstrokeopacity{0.300000}%
\pgfsetdash{}{0pt}%
\pgfpathmoveto{\pgfqpoint{3.933385in}{0.552778in}}%
\pgfpathlineto{\pgfqpoint{3.933385in}{3.801389in}}%
\pgfusepath{stroke}%
\end{pgfscope}%
\begin{pgfscope}%
\pgfsetbuttcap%
\pgfsetroundjoin%
\definecolor{currentfill}{rgb}{0.000000,0.000000,0.000000}%
\pgfsetfillcolor{currentfill}%
\pgfsetlinewidth{0.602250pt}%
\definecolor{currentstroke}{rgb}{0.000000,0.000000,0.000000}%
\pgfsetstrokecolor{currentstroke}%
\pgfsetdash{}{0pt}%
\pgfsys@defobject{currentmarker}{\pgfqpoint{0.000000in}{-0.027778in}}{\pgfqpoint{0.000000in}{0.000000in}}{%
\pgfpathmoveto{\pgfqpoint{0.000000in}{0.000000in}}%
\pgfpathlineto{\pgfqpoint{0.000000in}{-0.027778in}}%
\pgfusepath{stroke,fill}%
}%
\begin{pgfscope}%
\pgfsys@transformshift{3.933385in}{0.552778in}%
\pgfsys@useobject{currentmarker}{}%
\end{pgfscope}%
\end{pgfscope}%
\begin{pgfscope}%
\pgfsetbuttcap%
\pgfsetroundjoin%
\definecolor{currentfill}{rgb}{0.000000,0.000000,0.000000}%
\pgfsetfillcolor{currentfill}%
\pgfsetlinewidth{0.602250pt}%
\definecolor{currentstroke}{rgb}{0.000000,0.000000,0.000000}%
\pgfsetstrokecolor{currentstroke}%
\pgfsetdash{}{0pt}%
\pgfsys@defobject{currentmarker}{\pgfqpoint{0.000000in}{0.000000in}}{\pgfqpoint{0.000000in}{0.027778in}}{%
\pgfpathmoveto{\pgfqpoint{0.000000in}{0.000000in}}%
\pgfpathlineto{\pgfqpoint{0.000000in}{0.027778in}}%
\pgfusepath{stroke,fill}%
}%
\begin{pgfscope}%
\pgfsys@transformshift{3.933385in}{3.801389in}%
\pgfsys@useobject{currentmarker}{}%
\end{pgfscope}%
\end{pgfscope}%
\begin{pgfscope}%
\pgfpathrectangle{\pgfqpoint{0.781944in}{0.552778in}}{\pgfqpoint{3.890972in}{3.248611in}}%
\pgfusepath{clip}%
\pgfsetrectcap%
\pgfsetroundjoin%
\pgfsetlinewidth{0.803000pt}%
\definecolor{currentstroke}{rgb}{0.690196,0.690196,0.690196}%
\pgfsetstrokecolor{currentstroke}%
\pgfsetstrokeopacity{0.300000}%
\pgfsetdash{}{0pt}%
\pgfpathmoveto{\pgfqpoint{3.972308in}{0.552778in}}%
\pgfpathlineto{\pgfqpoint{3.972308in}{3.801389in}}%
\pgfusepath{stroke}%
\end{pgfscope}%
\begin{pgfscope}%
\pgfsetbuttcap%
\pgfsetroundjoin%
\definecolor{currentfill}{rgb}{0.000000,0.000000,0.000000}%
\pgfsetfillcolor{currentfill}%
\pgfsetlinewidth{0.602250pt}%
\definecolor{currentstroke}{rgb}{0.000000,0.000000,0.000000}%
\pgfsetstrokecolor{currentstroke}%
\pgfsetdash{}{0pt}%
\pgfsys@defobject{currentmarker}{\pgfqpoint{0.000000in}{-0.027778in}}{\pgfqpoint{0.000000in}{0.000000in}}{%
\pgfpathmoveto{\pgfqpoint{0.000000in}{0.000000in}}%
\pgfpathlineto{\pgfqpoint{0.000000in}{-0.027778in}}%
\pgfusepath{stroke,fill}%
}%
\begin{pgfscope}%
\pgfsys@transformshift{3.972308in}{0.552778in}%
\pgfsys@useobject{currentmarker}{}%
\end{pgfscope}%
\end{pgfscope}%
\begin{pgfscope}%
\pgfsetbuttcap%
\pgfsetroundjoin%
\definecolor{currentfill}{rgb}{0.000000,0.000000,0.000000}%
\pgfsetfillcolor{currentfill}%
\pgfsetlinewidth{0.602250pt}%
\definecolor{currentstroke}{rgb}{0.000000,0.000000,0.000000}%
\pgfsetstrokecolor{currentstroke}%
\pgfsetdash{}{0pt}%
\pgfsys@defobject{currentmarker}{\pgfqpoint{0.000000in}{0.000000in}}{\pgfqpoint{0.000000in}{0.027778in}}{%
\pgfpathmoveto{\pgfqpoint{0.000000in}{0.000000in}}%
\pgfpathlineto{\pgfqpoint{0.000000in}{0.027778in}}%
\pgfusepath{stroke,fill}%
}%
\begin{pgfscope}%
\pgfsys@transformshift{3.972308in}{3.801389in}%
\pgfsys@useobject{currentmarker}{}%
\end{pgfscope}%
\end{pgfscope}%
\begin{pgfscope}%
\pgfpathrectangle{\pgfqpoint{0.781944in}{0.552778in}}{\pgfqpoint{3.890972in}{3.248611in}}%
\pgfusepath{clip}%
\pgfsetrectcap%
\pgfsetroundjoin%
\pgfsetlinewidth{0.803000pt}%
\definecolor{currentstroke}{rgb}{0.690196,0.690196,0.690196}%
\pgfsetstrokecolor{currentstroke}%
\pgfsetstrokeopacity{0.300000}%
\pgfsetdash{}{0pt}%
\pgfpathmoveto{\pgfqpoint{4.011231in}{0.552778in}}%
\pgfpathlineto{\pgfqpoint{4.011231in}{3.801389in}}%
\pgfusepath{stroke}%
\end{pgfscope}%
\begin{pgfscope}%
\pgfsetbuttcap%
\pgfsetroundjoin%
\definecolor{currentfill}{rgb}{0.000000,0.000000,0.000000}%
\pgfsetfillcolor{currentfill}%
\pgfsetlinewidth{0.602250pt}%
\definecolor{currentstroke}{rgb}{0.000000,0.000000,0.000000}%
\pgfsetstrokecolor{currentstroke}%
\pgfsetdash{}{0pt}%
\pgfsys@defobject{currentmarker}{\pgfqpoint{0.000000in}{-0.027778in}}{\pgfqpoint{0.000000in}{0.000000in}}{%
\pgfpathmoveto{\pgfqpoint{0.000000in}{0.000000in}}%
\pgfpathlineto{\pgfqpoint{0.000000in}{-0.027778in}}%
\pgfusepath{stroke,fill}%
}%
\begin{pgfscope}%
\pgfsys@transformshift{4.011231in}{0.552778in}%
\pgfsys@useobject{currentmarker}{}%
\end{pgfscope}%
\end{pgfscope}%
\begin{pgfscope}%
\pgfsetbuttcap%
\pgfsetroundjoin%
\definecolor{currentfill}{rgb}{0.000000,0.000000,0.000000}%
\pgfsetfillcolor{currentfill}%
\pgfsetlinewidth{0.602250pt}%
\definecolor{currentstroke}{rgb}{0.000000,0.000000,0.000000}%
\pgfsetstrokecolor{currentstroke}%
\pgfsetdash{}{0pt}%
\pgfsys@defobject{currentmarker}{\pgfqpoint{0.000000in}{0.000000in}}{\pgfqpoint{0.000000in}{0.027778in}}{%
\pgfpathmoveto{\pgfqpoint{0.000000in}{0.000000in}}%
\pgfpathlineto{\pgfqpoint{0.000000in}{0.027778in}}%
\pgfusepath{stroke,fill}%
}%
\begin{pgfscope}%
\pgfsys@transformshift{4.011231in}{3.801389in}%
\pgfsys@useobject{currentmarker}{}%
\end{pgfscope}%
\end{pgfscope}%
\begin{pgfscope}%
\pgfpathrectangle{\pgfqpoint{0.781944in}{0.552778in}}{\pgfqpoint{3.890972in}{3.248611in}}%
\pgfusepath{clip}%
\pgfsetrectcap%
\pgfsetroundjoin%
\pgfsetlinewidth{0.803000pt}%
\definecolor{currentstroke}{rgb}{0.690196,0.690196,0.690196}%
\pgfsetstrokecolor{currentstroke}%
\pgfsetstrokeopacity{0.300000}%
\pgfsetdash{}{0pt}%
\pgfpathmoveto{\pgfqpoint{4.050154in}{0.552778in}}%
\pgfpathlineto{\pgfqpoint{4.050154in}{3.801389in}}%
\pgfusepath{stroke}%
\end{pgfscope}%
\begin{pgfscope}%
\pgfsetbuttcap%
\pgfsetroundjoin%
\definecolor{currentfill}{rgb}{0.000000,0.000000,0.000000}%
\pgfsetfillcolor{currentfill}%
\pgfsetlinewidth{0.602250pt}%
\definecolor{currentstroke}{rgb}{0.000000,0.000000,0.000000}%
\pgfsetstrokecolor{currentstroke}%
\pgfsetdash{}{0pt}%
\pgfsys@defobject{currentmarker}{\pgfqpoint{0.000000in}{-0.027778in}}{\pgfqpoint{0.000000in}{0.000000in}}{%
\pgfpathmoveto{\pgfqpoint{0.000000in}{0.000000in}}%
\pgfpathlineto{\pgfqpoint{0.000000in}{-0.027778in}}%
\pgfusepath{stroke,fill}%
}%
\begin{pgfscope}%
\pgfsys@transformshift{4.050154in}{0.552778in}%
\pgfsys@useobject{currentmarker}{}%
\end{pgfscope}%
\end{pgfscope}%
\begin{pgfscope}%
\pgfsetbuttcap%
\pgfsetroundjoin%
\definecolor{currentfill}{rgb}{0.000000,0.000000,0.000000}%
\pgfsetfillcolor{currentfill}%
\pgfsetlinewidth{0.602250pt}%
\definecolor{currentstroke}{rgb}{0.000000,0.000000,0.000000}%
\pgfsetstrokecolor{currentstroke}%
\pgfsetdash{}{0pt}%
\pgfsys@defobject{currentmarker}{\pgfqpoint{0.000000in}{0.000000in}}{\pgfqpoint{0.000000in}{0.027778in}}{%
\pgfpathmoveto{\pgfqpoint{0.000000in}{0.000000in}}%
\pgfpathlineto{\pgfqpoint{0.000000in}{0.027778in}}%
\pgfusepath{stroke,fill}%
}%
\begin{pgfscope}%
\pgfsys@transformshift{4.050154in}{3.801389in}%
\pgfsys@useobject{currentmarker}{}%
\end{pgfscope}%
\end{pgfscope}%
\begin{pgfscope}%
\pgfpathrectangle{\pgfqpoint{0.781944in}{0.552778in}}{\pgfqpoint{3.890972in}{3.248611in}}%
\pgfusepath{clip}%
\pgfsetrectcap%
\pgfsetroundjoin%
\pgfsetlinewidth{0.803000pt}%
\definecolor{currentstroke}{rgb}{0.690196,0.690196,0.690196}%
\pgfsetstrokecolor{currentstroke}%
\pgfsetstrokeopacity{0.300000}%
\pgfsetdash{}{0pt}%
\pgfpathmoveto{\pgfqpoint{4.089076in}{0.552778in}}%
\pgfpathlineto{\pgfqpoint{4.089076in}{3.801389in}}%
\pgfusepath{stroke}%
\end{pgfscope}%
\begin{pgfscope}%
\pgfsetbuttcap%
\pgfsetroundjoin%
\definecolor{currentfill}{rgb}{0.000000,0.000000,0.000000}%
\pgfsetfillcolor{currentfill}%
\pgfsetlinewidth{0.602250pt}%
\definecolor{currentstroke}{rgb}{0.000000,0.000000,0.000000}%
\pgfsetstrokecolor{currentstroke}%
\pgfsetdash{}{0pt}%
\pgfsys@defobject{currentmarker}{\pgfqpoint{0.000000in}{-0.027778in}}{\pgfqpoint{0.000000in}{0.000000in}}{%
\pgfpathmoveto{\pgfqpoint{0.000000in}{0.000000in}}%
\pgfpathlineto{\pgfqpoint{0.000000in}{-0.027778in}}%
\pgfusepath{stroke,fill}%
}%
\begin{pgfscope}%
\pgfsys@transformshift{4.089076in}{0.552778in}%
\pgfsys@useobject{currentmarker}{}%
\end{pgfscope}%
\end{pgfscope}%
\begin{pgfscope}%
\pgfsetbuttcap%
\pgfsetroundjoin%
\definecolor{currentfill}{rgb}{0.000000,0.000000,0.000000}%
\pgfsetfillcolor{currentfill}%
\pgfsetlinewidth{0.602250pt}%
\definecolor{currentstroke}{rgb}{0.000000,0.000000,0.000000}%
\pgfsetstrokecolor{currentstroke}%
\pgfsetdash{}{0pt}%
\pgfsys@defobject{currentmarker}{\pgfqpoint{0.000000in}{0.000000in}}{\pgfqpoint{0.000000in}{0.027778in}}{%
\pgfpathmoveto{\pgfqpoint{0.000000in}{0.000000in}}%
\pgfpathlineto{\pgfqpoint{0.000000in}{0.027778in}}%
\pgfusepath{stroke,fill}%
}%
\begin{pgfscope}%
\pgfsys@transformshift{4.089076in}{3.801389in}%
\pgfsys@useobject{currentmarker}{}%
\end{pgfscope}%
\end{pgfscope}%
\begin{pgfscope}%
\pgfpathrectangle{\pgfqpoint{0.781944in}{0.552778in}}{\pgfqpoint{3.890972in}{3.248611in}}%
\pgfusepath{clip}%
\pgfsetrectcap%
\pgfsetroundjoin%
\pgfsetlinewidth{0.803000pt}%
\definecolor{currentstroke}{rgb}{0.690196,0.690196,0.690196}%
\pgfsetstrokecolor{currentstroke}%
\pgfsetstrokeopacity{0.300000}%
\pgfsetdash{}{0pt}%
\pgfpathmoveto{\pgfqpoint{4.127999in}{0.552778in}}%
\pgfpathlineto{\pgfqpoint{4.127999in}{3.801389in}}%
\pgfusepath{stroke}%
\end{pgfscope}%
\begin{pgfscope}%
\pgfsetbuttcap%
\pgfsetroundjoin%
\definecolor{currentfill}{rgb}{0.000000,0.000000,0.000000}%
\pgfsetfillcolor{currentfill}%
\pgfsetlinewidth{0.602250pt}%
\definecolor{currentstroke}{rgb}{0.000000,0.000000,0.000000}%
\pgfsetstrokecolor{currentstroke}%
\pgfsetdash{}{0pt}%
\pgfsys@defobject{currentmarker}{\pgfqpoint{0.000000in}{-0.027778in}}{\pgfqpoint{0.000000in}{0.000000in}}{%
\pgfpathmoveto{\pgfqpoint{0.000000in}{0.000000in}}%
\pgfpathlineto{\pgfqpoint{0.000000in}{-0.027778in}}%
\pgfusepath{stroke,fill}%
}%
\begin{pgfscope}%
\pgfsys@transformshift{4.127999in}{0.552778in}%
\pgfsys@useobject{currentmarker}{}%
\end{pgfscope}%
\end{pgfscope}%
\begin{pgfscope}%
\pgfsetbuttcap%
\pgfsetroundjoin%
\definecolor{currentfill}{rgb}{0.000000,0.000000,0.000000}%
\pgfsetfillcolor{currentfill}%
\pgfsetlinewidth{0.602250pt}%
\definecolor{currentstroke}{rgb}{0.000000,0.000000,0.000000}%
\pgfsetstrokecolor{currentstroke}%
\pgfsetdash{}{0pt}%
\pgfsys@defobject{currentmarker}{\pgfqpoint{0.000000in}{0.000000in}}{\pgfqpoint{0.000000in}{0.027778in}}{%
\pgfpathmoveto{\pgfqpoint{0.000000in}{0.000000in}}%
\pgfpathlineto{\pgfqpoint{0.000000in}{0.027778in}}%
\pgfusepath{stroke,fill}%
}%
\begin{pgfscope}%
\pgfsys@transformshift{4.127999in}{3.801389in}%
\pgfsys@useobject{currentmarker}{}%
\end{pgfscope}%
\end{pgfscope}%
\begin{pgfscope}%
\pgfpathrectangle{\pgfqpoint{0.781944in}{0.552778in}}{\pgfqpoint{3.890972in}{3.248611in}}%
\pgfusepath{clip}%
\pgfsetrectcap%
\pgfsetroundjoin%
\pgfsetlinewidth{0.803000pt}%
\definecolor{currentstroke}{rgb}{0.690196,0.690196,0.690196}%
\pgfsetstrokecolor{currentstroke}%
\pgfsetstrokeopacity{0.300000}%
\pgfsetdash{}{0pt}%
\pgfpathmoveto{\pgfqpoint{4.166922in}{0.552778in}}%
\pgfpathlineto{\pgfqpoint{4.166922in}{3.801389in}}%
\pgfusepath{stroke}%
\end{pgfscope}%
\begin{pgfscope}%
\pgfsetbuttcap%
\pgfsetroundjoin%
\definecolor{currentfill}{rgb}{0.000000,0.000000,0.000000}%
\pgfsetfillcolor{currentfill}%
\pgfsetlinewidth{0.602250pt}%
\definecolor{currentstroke}{rgb}{0.000000,0.000000,0.000000}%
\pgfsetstrokecolor{currentstroke}%
\pgfsetdash{}{0pt}%
\pgfsys@defobject{currentmarker}{\pgfqpoint{0.000000in}{-0.027778in}}{\pgfqpoint{0.000000in}{0.000000in}}{%
\pgfpathmoveto{\pgfqpoint{0.000000in}{0.000000in}}%
\pgfpathlineto{\pgfqpoint{0.000000in}{-0.027778in}}%
\pgfusepath{stroke,fill}%
}%
\begin{pgfscope}%
\pgfsys@transformshift{4.166922in}{0.552778in}%
\pgfsys@useobject{currentmarker}{}%
\end{pgfscope}%
\end{pgfscope}%
\begin{pgfscope}%
\pgfsetbuttcap%
\pgfsetroundjoin%
\definecolor{currentfill}{rgb}{0.000000,0.000000,0.000000}%
\pgfsetfillcolor{currentfill}%
\pgfsetlinewidth{0.602250pt}%
\definecolor{currentstroke}{rgb}{0.000000,0.000000,0.000000}%
\pgfsetstrokecolor{currentstroke}%
\pgfsetdash{}{0pt}%
\pgfsys@defobject{currentmarker}{\pgfqpoint{0.000000in}{0.000000in}}{\pgfqpoint{0.000000in}{0.027778in}}{%
\pgfpathmoveto{\pgfqpoint{0.000000in}{0.000000in}}%
\pgfpathlineto{\pgfqpoint{0.000000in}{0.027778in}}%
\pgfusepath{stroke,fill}%
}%
\begin{pgfscope}%
\pgfsys@transformshift{4.166922in}{3.801389in}%
\pgfsys@useobject{currentmarker}{}%
\end{pgfscope}%
\end{pgfscope}%
\begin{pgfscope}%
\pgfpathrectangle{\pgfqpoint{0.781944in}{0.552778in}}{\pgfqpoint{3.890972in}{3.248611in}}%
\pgfusepath{clip}%
\pgfsetrectcap%
\pgfsetroundjoin%
\pgfsetlinewidth{0.803000pt}%
\definecolor{currentstroke}{rgb}{0.690196,0.690196,0.690196}%
\pgfsetstrokecolor{currentstroke}%
\pgfsetstrokeopacity{0.300000}%
\pgfsetdash{}{0pt}%
\pgfpathmoveto{\pgfqpoint{4.205844in}{0.552778in}}%
\pgfpathlineto{\pgfqpoint{4.205844in}{3.801389in}}%
\pgfusepath{stroke}%
\end{pgfscope}%
\begin{pgfscope}%
\pgfsetbuttcap%
\pgfsetroundjoin%
\definecolor{currentfill}{rgb}{0.000000,0.000000,0.000000}%
\pgfsetfillcolor{currentfill}%
\pgfsetlinewidth{0.602250pt}%
\definecolor{currentstroke}{rgb}{0.000000,0.000000,0.000000}%
\pgfsetstrokecolor{currentstroke}%
\pgfsetdash{}{0pt}%
\pgfsys@defobject{currentmarker}{\pgfqpoint{0.000000in}{-0.027778in}}{\pgfqpoint{0.000000in}{0.000000in}}{%
\pgfpathmoveto{\pgfqpoint{0.000000in}{0.000000in}}%
\pgfpathlineto{\pgfqpoint{0.000000in}{-0.027778in}}%
\pgfusepath{stroke,fill}%
}%
\begin{pgfscope}%
\pgfsys@transformshift{4.205844in}{0.552778in}%
\pgfsys@useobject{currentmarker}{}%
\end{pgfscope}%
\end{pgfscope}%
\begin{pgfscope}%
\pgfsetbuttcap%
\pgfsetroundjoin%
\definecolor{currentfill}{rgb}{0.000000,0.000000,0.000000}%
\pgfsetfillcolor{currentfill}%
\pgfsetlinewidth{0.602250pt}%
\definecolor{currentstroke}{rgb}{0.000000,0.000000,0.000000}%
\pgfsetstrokecolor{currentstroke}%
\pgfsetdash{}{0pt}%
\pgfsys@defobject{currentmarker}{\pgfqpoint{0.000000in}{0.000000in}}{\pgfqpoint{0.000000in}{0.027778in}}{%
\pgfpathmoveto{\pgfqpoint{0.000000in}{0.000000in}}%
\pgfpathlineto{\pgfqpoint{0.000000in}{0.027778in}}%
\pgfusepath{stroke,fill}%
}%
\begin{pgfscope}%
\pgfsys@transformshift{4.205844in}{3.801389in}%
\pgfsys@useobject{currentmarker}{}%
\end{pgfscope}%
\end{pgfscope}%
\begin{pgfscope}%
\pgfpathrectangle{\pgfqpoint{0.781944in}{0.552778in}}{\pgfqpoint{3.890972in}{3.248611in}}%
\pgfusepath{clip}%
\pgfsetrectcap%
\pgfsetroundjoin%
\pgfsetlinewidth{0.803000pt}%
\definecolor{currentstroke}{rgb}{0.690196,0.690196,0.690196}%
\pgfsetstrokecolor{currentstroke}%
\pgfsetstrokeopacity{0.300000}%
\pgfsetdash{}{0pt}%
\pgfpathmoveto{\pgfqpoint{4.244767in}{0.552778in}}%
\pgfpathlineto{\pgfqpoint{4.244767in}{3.801389in}}%
\pgfusepath{stroke}%
\end{pgfscope}%
\begin{pgfscope}%
\pgfsetbuttcap%
\pgfsetroundjoin%
\definecolor{currentfill}{rgb}{0.000000,0.000000,0.000000}%
\pgfsetfillcolor{currentfill}%
\pgfsetlinewidth{0.602250pt}%
\definecolor{currentstroke}{rgb}{0.000000,0.000000,0.000000}%
\pgfsetstrokecolor{currentstroke}%
\pgfsetdash{}{0pt}%
\pgfsys@defobject{currentmarker}{\pgfqpoint{0.000000in}{-0.027778in}}{\pgfqpoint{0.000000in}{0.000000in}}{%
\pgfpathmoveto{\pgfqpoint{0.000000in}{0.000000in}}%
\pgfpathlineto{\pgfqpoint{0.000000in}{-0.027778in}}%
\pgfusepath{stroke,fill}%
}%
\begin{pgfscope}%
\pgfsys@transformshift{4.244767in}{0.552778in}%
\pgfsys@useobject{currentmarker}{}%
\end{pgfscope}%
\end{pgfscope}%
\begin{pgfscope}%
\pgfsetbuttcap%
\pgfsetroundjoin%
\definecolor{currentfill}{rgb}{0.000000,0.000000,0.000000}%
\pgfsetfillcolor{currentfill}%
\pgfsetlinewidth{0.602250pt}%
\definecolor{currentstroke}{rgb}{0.000000,0.000000,0.000000}%
\pgfsetstrokecolor{currentstroke}%
\pgfsetdash{}{0pt}%
\pgfsys@defobject{currentmarker}{\pgfqpoint{0.000000in}{0.000000in}}{\pgfqpoint{0.000000in}{0.027778in}}{%
\pgfpathmoveto{\pgfqpoint{0.000000in}{0.000000in}}%
\pgfpathlineto{\pgfqpoint{0.000000in}{0.027778in}}%
\pgfusepath{stroke,fill}%
}%
\begin{pgfscope}%
\pgfsys@transformshift{4.244767in}{3.801389in}%
\pgfsys@useobject{currentmarker}{}%
\end{pgfscope}%
\end{pgfscope}%
\begin{pgfscope}%
\pgfpathrectangle{\pgfqpoint{0.781944in}{0.552778in}}{\pgfqpoint{3.890972in}{3.248611in}}%
\pgfusepath{clip}%
\pgfsetrectcap%
\pgfsetroundjoin%
\pgfsetlinewidth{0.803000pt}%
\definecolor{currentstroke}{rgb}{0.690196,0.690196,0.690196}%
\pgfsetstrokecolor{currentstroke}%
\pgfsetstrokeopacity{0.300000}%
\pgfsetdash{}{0pt}%
\pgfpathmoveto{\pgfqpoint{4.322612in}{0.552778in}}%
\pgfpathlineto{\pgfqpoint{4.322612in}{3.801389in}}%
\pgfusepath{stroke}%
\end{pgfscope}%
\begin{pgfscope}%
\pgfsetbuttcap%
\pgfsetroundjoin%
\definecolor{currentfill}{rgb}{0.000000,0.000000,0.000000}%
\pgfsetfillcolor{currentfill}%
\pgfsetlinewidth{0.602250pt}%
\definecolor{currentstroke}{rgb}{0.000000,0.000000,0.000000}%
\pgfsetstrokecolor{currentstroke}%
\pgfsetdash{}{0pt}%
\pgfsys@defobject{currentmarker}{\pgfqpoint{0.000000in}{-0.027778in}}{\pgfqpoint{0.000000in}{0.000000in}}{%
\pgfpathmoveto{\pgfqpoint{0.000000in}{0.000000in}}%
\pgfpathlineto{\pgfqpoint{0.000000in}{-0.027778in}}%
\pgfusepath{stroke,fill}%
}%
\begin{pgfscope}%
\pgfsys@transformshift{4.322612in}{0.552778in}%
\pgfsys@useobject{currentmarker}{}%
\end{pgfscope}%
\end{pgfscope}%
\begin{pgfscope}%
\pgfsetbuttcap%
\pgfsetroundjoin%
\definecolor{currentfill}{rgb}{0.000000,0.000000,0.000000}%
\pgfsetfillcolor{currentfill}%
\pgfsetlinewidth{0.602250pt}%
\definecolor{currentstroke}{rgb}{0.000000,0.000000,0.000000}%
\pgfsetstrokecolor{currentstroke}%
\pgfsetdash{}{0pt}%
\pgfsys@defobject{currentmarker}{\pgfqpoint{0.000000in}{0.000000in}}{\pgfqpoint{0.000000in}{0.027778in}}{%
\pgfpathmoveto{\pgfqpoint{0.000000in}{0.000000in}}%
\pgfpathlineto{\pgfqpoint{0.000000in}{0.027778in}}%
\pgfusepath{stroke,fill}%
}%
\begin{pgfscope}%
\pgfsys@transformshift{4.322612in}{3.801389in}%
\pgfsys@useobject{currentmarker}{}%
\end{pgfscope}%
\end{pgfscope}%
\begin{pgfscope}%
\pgfpathrectangle{\pgfqpoint{0.781944in}{0.552778in}}{\pgfqpoint{3.890972in}{3.248611in}}%
\pgfusepath{clip}%
\pgfsetrectcap%
\pgfsetroundjoin%
\pgfsetlinewidth{0.803000pt}%
\definecolor{currentstroke}{rgb}{0.690196,0.690196,0.690196}%
\pgfsetstrokecolor{currentstroke}%
\pgfsetstrokeopacity{0.300000}%
\pgfsetdash{}{0pt}%
\pgfpathmoveto{\pgfqpoint{4.361535in}{0.552778in}}%
\pgfpathlineto{\pgfqpoint{4.361535in}{3.801389in}}%
\pgfusepath{stroke}%
\end{pgfscope}%
\begin{pgfscope}%
\pgfsetbuttcap%
\pgfsetroundjoin%
\definecolor{currentfill}{rgb}{0.000000,0.000000,0.000000}%
\pgfsetfillcolor{currentfill}%
\pgfsetlinewidth{0.602250pt}%
\definecolor{currentstroke}{rgb}{0.000000,0.000000,0.000000}%
\pgfsetstrokecolor{currentstroke}%
\pgfsetdash{}{0pt}%
\pgfsys@defobject{currentmarker}{\pgfqpoint{0.000000in}{-0.027778in}}{\pgfqpoint{0.000000in}{0.000000in}}{%
\pgfpathmoveto{\pgfqpoint{0.000000in}{0.000000in}}%
\pgfpathlineto{\pgfqpoint{0.000000in}{-0.027778in}}%
\pgfusepath{stroke,fill}%
}%
\begin{pgfscope}%
\pgfsys@transformshift{4.361535in}{0.552778in}%
\pgfsys@useobject{currentmarker}{}%
\end{pgfscope}%
\end{pgfscope}%
\begin{pgfscope}%
\pgfsetbuttcap%
\pgfsetroundjoin%
\definecolor{currentfill}{rgb}{0.000000,0.000000,0.000000}%
\pgfsetfillcolor{currentfill}%
\pgfsetlinewidth{0.602250pt}%
\definecolor{currentstroke}{rgb}{0.000000,0.000000,0.000000}%
\pgfsetstrokecolor{currentstroke}%
\pgfsetdash{}{0pt}%
\pgfsys@defobject{currentmarker}{\pgfqpoint{0.000000in}{0.000000in}}{\pgfqpoint{0.000000in}{0.027778in}}{%
\pgfpathmoveto{\pgfqpoint{0.000000in}{0.000000in}}%
\pgfpathlineto{\pgfqpoint{0.000000in}{0.027778in}}%
\pgfusepath{stroke,fill}%
}%
\begin{pgfscope}%
\pgfsys@transformshift{4.361535in}{3.801389in}%
\pgfsys@useobject{currentmarker}{}%
\end{pgfscope}%
\end{pgfscope}%
\begin{pgfscope}%
\pgfpathrectangle{\pgfqpoint{0.781944in}{0.552778in}}{\pgfqpoint{3.890972in}{3.248611in}}%
\pgfusepath{clip}%
\pgfsetrectcap%
\pgfsetroundjoin%
\pgfsetlinewidth{0.803000pt}%
\definecolor{currentstroke}{rgb}{0.690196,0.690196,0.690196}%
\pgfsetstrokecolor{currentstroke}%
\pgfsetstrokeopacity{0.300000}%
\pgfsetdash{}{0pt}%
\pgfpathmoveto{\pgfqpoint{4.400458in}{0.552778in}}%
\pgfpathlineto{\pgfqpoint{4.400458in}{3.801389in}}%
\pgfusepath{stroke}%
\end{pgfscope}%
\begin{pgfscope}%
\pgfsetbuttcap%
\pgfsetroundjoin%
\definecolor{currentfill}{rgb}{0.000000,0.000000,0.000000}%
\pgfsetfillcolor{currentfill}%
\pgfsetlinewidth{0.602250pt}%
\definecolor{currentstroke}{rgb}{0.000000,0.000000,0.000000}%
\pgfsetstrokecolor{currentstroke}%
\pgfsetdash{}{0pt}%
\pgfsys@defobject{currentmarker}{\pgfqpoint{0.000000in}{-0.027778in}}{\pgfqpoint{0.000000in}{0.000000in}}{%
\pgfpathmoveto{\pgfqpoint{0.000000in}{0.000000in}}%
\pgfpathlineto{\pgfqpoint{0.000000in}{-0.027778in}}%
\pgfusepath{stroke,fill}%
}%
\begin{pgfscope}%
\pgfsys@transformshift{4.400458in}{0.552778in}%
\pgfsys@useobject{currentmarker}{}%
\end{pgfscope}%
\end{pgfscope}%
\begin{pgfscope}%
\pgfsetbuttcap%
\pgfsetroundjoin%
\definecolor{currentfill}{rgb}{0.000000,0.000000,0.000000}%
\pgfsetfillcolor{currentfill}%
\pgfsetlinewidth{0.602250pt}%
\definecolor{currentstroke}{rgb}{0.000000,0.000000,0.000000}%
\pgfsetstrokecolor{currentstroke}%
\pgfsetdash{}{0pt}%
\pgfsys@defobject{currentmarker}{\pgfqpoint{0.000000in}{0.000000in}}{\pgfqpoint{0.000000in}{0.027778in}}{%
\pgfpathmoveto{\pgfqpoint{0.000000in}{0.000000in}}%
\pgfpathlineto{\pgfqpoint{0.000000in}{0.027778in}}%
\pgfusepath{stroke,fill}%
}%
\begin{pgfscope}%
\pgfsys@transformshift{4.400458in}{3.801389in}%
\pgfsys@useobject{currentmarker}{}%
\end{pgfscope}%
\end{pgfscope}%
\begin{pgfscope}%
\pgfpathrectangle{\pgfqpoint{0.781944in}{0.552778in}}{\pgfqpoint{3.890972in}{3.248611in}}%
\pgfusepath{clip}%
\pgfsetrectcap%
\pgfsetroundjoin%
\pgfsetlinewidth{0.803000pt}%
\definecolor{currentstroke}{rgb}{0.690196,0.690196,0.690196}%
\pgfsetstrokecolor{currentstroke}%
\pgfsetstrokeopacity{0.300000}%
\pgfsetdash{}{0pt}%
\pgfpathmoveto{\pgfqpoint{4.439380in}{0.552778in}}%
\pgfpathlineto{\pgfqpoint{4.439380in}{3.801389in}}%
\pgfusepath{stroke}%
\end{pgfscope}%
\begin{pgfscope}%
\pgfsetbuttcap%
\pgfsetroundjoin%
\definecolor{currentfill}{rgb}{0.000000,0.000000,0.000000}%
\pgfsetfillcolor{currentfill}%
\pgfsetlinewidth{0.602250pt}%
\definecolor{currentstroke}{rgb}{0.000000,0.000000,0.000000}%
\pgfsetstrokecolor{currentstroke}%
\pgfsetdash{}{0pt}%
\pgfsys@defobject{currentmarker}{\pgfqpoint{0.000000in}{-0.027778in}}{\pgfqpoint{0.000000in}{0.000000in}}{%
\pgfpathmoveto{\pgfqpoint{0.000000in}{0.000000in}}%
\pgfpathlineto{\pgfqpoint{0.000000in}{-0.027778in}}%
\pgfusepath{stroke,fill}%
}%
\begin{pgfscope}%
\pgfsys@transformshift{4.439380in}{0.552778in}%
\pgfsys@useobject{currentmarker}{}%
\end{pgfscope}%
\end{pgfscope}%
\begin{pgfscope}%
\pgfsetbuttcap%
\pgfsetroundjoin%
\definecolor{currentfill}{rgb}{0.000000,0.000000,0.000000}%
\pgfsetfillcolor{currentfill}%
\pgfsetlinewidth{0.602250pt}%
\definecolor{currentstroke}{rgb}{0.000000,0.000000,0.000000}%
\pgfsetstrokecolor{currentstroke}%
\pgfsetdash{}{0pt}%
\pgfsys@defobject{currentmarker}{\pgfqpoint{0.000000in}{0.000000in}}{\pgfqpoint{0.000000in}{0.027778in}}{%
\pgfpathmoveto{\pgfqpoint{0.000000in}{0.000000in}}%
\pgfpathlineto{\pgfqpoint{0.000000in}{0.027778in}}%
\pgfusepath{stroke,fill}%
}%
\begin{pgfscope}%
\pgfsys@transformshift{4.439380in}{3.801389in}%
\pgfsys@useobject{currentmarker}{}%
\end{pgfscope}%
\end{pgfscope}%
\begin{pgfscope}%
\pgfpathrectangle{\pgfqpoint{0.781944in}{0.552778in}}{\pgfqpoint{3.890972in}{3.248611in}}%
\pgfusepath{clip}%
\pgfsetrectcap%
\pgfsetroundjoin%
\pgfsetlinewidth{0.803000pt}%
\definecolor{currentstroke}{rgb}{0.690196,0.690196,0.690196}%
\pgfsetstrokecolor{currentstroke}%
\pgfsetstrokeopacity{0.300000}%
\pgfsetdash{}{0pt}%
\pgfpathmoveto{\pgfqpoint{4.478303in}{0.552778in}}%
\pgfpathlineto{\pgfqpoint{4.478303in}{3.801389in}}%
\pgfusepath{stroke}%
\end{pgfscope}%
\begin{pgfscope}%
\pgfsetbuttcap%
\pgfsetroundjoin%
\definecolor{currentfill}{rgb}{0.000000,0.000000,0.000000}%
\pgfsetfillcolor{currentfill}%
\pgfsetlinewidth{0.602250pt}%
\definecolor{currentstroke}{rgb}{0.000000,0.000000,0.000000}%
\pgfsetstrokecolor{currentstroke}%
\pgfsetdash{}{0pt}%
\pgfsys@defobject{currentmarker}{\pgfqpoint{0.000000in}{-0.027778in}}{\pgfqpoint{0.000000in}{0.000000in}}{%
\pgfpathmoveto{\pgfqpoint{0.000000in}{0.000000in}}%
\pgfpathlineto{\pgfqpoint{0.000000in}{-0.027778in}}%
\pgfusepath{stroke,fill}%
}%
\begin{pgfscope}%
\pgfsys@transformshift{4.478303in}{0.552778in}%
\pgfsys@useobject{currentmarker}{}%
\end{pgfscope}%
\end{pgfscope}%
\begin{pgfscope}%
\pgfsetbuttcap%
\pgfsetroundjoin%
\definecolor{currentfill}{rgb}{0.000000,0.000000,0.000000}%
\pgfsetfillcolor{currentfill}%
\pgfsetlinewidth{0.602250pt}%
\definecolor{currentstroke}{rgb}{0.000000,0.000000,0.000000}%
\pgfsetstrokecolor{currentstroke}%
\pgfsetdash{}{0pt}%
\pgfsys@defobject{currentmarker}{\pgfqpoint{0.000000in}{0.000000in}}{\pgfqpoint{0.000000in}{0.027778in}}{%
\pgfpathmoveto{\pgfqpoint{0.000000in}{0.000000in}}%
\pgfpathlineto{\pgfqpoint{0.000000in}{0.027778in}}%
\pgfusepath{stroke,fill}%
}%
\begin{pgfscope}%
\pgfsys@transformshift{4.478303in}{3.801389in}%
\pgfsys@useobject{currentmarker}{}%
\end{pgfscope}%
\end{pgfscope}%
\begin{pgfscope}%
\pgfpathrectangle{\pgfqpoint{0.781944in}{0.552778in}}{\pgfqpoint{3.890972in}{3.248611in}}%
\pgfusepath{clip}%
\pgfsetrectcap%
\pgfsetroundjoin%
\pgfsetlinewidth{0.803000pt}%
\definecolor{currentstroke}{rgb}{0.690196,0.690196,0.690196}%
\pgfsetstrokecolor{currentstroke}%
\pgfsetstrokeopacity{0.300000}%
\pgfsetdash{}{0pt}%
\pgfpathmoveto{\pgfqpoint{4.517226in}{0.552778in}}%
\pgfpathlineto{\pgfqpoint{4.517226in}{3.801389in}}%
\pgfusepath{stroke}%
\end{pgfscope}%
\begin{pgfscope}%
\pgfsetbuttcap%
\pgfsetroundjoin%
\definecolor{currentfill}{rgb}{0.000000,0.000000,0.000000}%
\pgfsetfillcolor{currentfill}%
\pgfsetlinewidth{0.602250pt}%
\definecolor{currentstroke}{rgb}{0.000000,0.000000,0.000000}%
\pgfsetstrokecolor{currentstroke}%
\pgfsetdash{}{0pt}%
\pgfsys@defobject{currentmarker}{\pgfqpoint{0.000000in}{-0.027778in}}{\pgfqpoint{0.000000in}{0.000000in}}{%
\pgfpathmoveto{\pgfqpoint{0.000000in}{0.000000in}}%
\pgfpathlineto{\pgfqpoint{0.000000in}{-0.027778in}}%
\pgfusepath{stroke,fill}%
}%
\begin{pgfscope}%
\pgfsys@transformshift{4.517226in}{0.552778in}%
\pgfsys@useobject{currentmarker}{}%
\end{pgfscope}%
\end{pgfscope}%
\begin{pgfscope}%
\pgfsetbuttcap%
\pgfsetroundjoin%
\definecolor{currentfill}{rgb}{0.000000,0.000000,0.000000}%
\pgfsetfillcolor{currentfill}%
\pgfsetlinewidth{0.602250pt}%
\definecolor{currentstroke}{rgb}{0.000000,0.000000,0.000000}%
\pgfsetstrokecolor{currentstroke}%
\pgfsetdash{}{0pt}%
\pgfsys@defobject{currentmarker}{\pgfqpoint{0.000000in}{0.000000in}}{\pgfqpoint{0.000000in}{0.027778in}}{%
\pgfpathmoveto{\pgfqpoint{0.000000in}{0.000000in}}%
\pgfpathlineto{\pgfqpoint{0.000000in}{0.027778in}}%
\pgfusepath{stroke,fill}%
}%
\begin{pgfscope}%
\pgfsys@transformshift{4.517226in}{3.801389in}%
\pgfsys@useobject{currentmarker}{}%
\end{pgfscope}%
\end{pgfscope}%
\begin{pgfscope}%
\pgfpathrectangle{\pgfqpoint{0.781944in}{0.552778in}}{\pgfqpoint{3.890972in}{3.248611in}}%
\pgfusepath{clip}%
\pgfsetrectcap%
\pgfsetroundjoin%
\pgfsetlinewidth{0.803000pt}%
\definecolor{currentstroke}{rgb}{0.690196,0.690196,0.690196}%
\pgfsetstrokecolor{currentstroke}%
\pgfsetstrokeopacity{0.300000}%
\pgfsetdash{}{0pt}%
\pgfpathmoveto{\pgfqpoint{4.556149in}{0.552778in}}%
\pgfpathlineto{\pgfqpoint{4.556149in}{3.801389in}}%
\pgfusepath{stroke}%
\end{pgfscope}%
\begin{pgfscope}%
\pgfsetbuttcap%
\pgfsetroundjoin%
\definecolor{currentfill}{rgb}{0.000000,0.000000,0.000000}%
\pgfsetfillcolor{currentfill}%
\pgfsetlinewidth{0.602250pt}%
\definecolor{currentstroke}{rgb}{0.000000,0.000000,0.000000}%
\pgfsetstrokecolor{currentstroke}%
\pgfsetdash{}{0pt}%
\pgfsys@defobject{currentmarker}{\pgfqpoint{0.000000in}{-0.027778in}}{\pgfqpoint{0.000000in}{0.000000in}}{%
\pgfpathmoveto{\pgfqpoint{0.000000in}{0.000000in}}%
\pgfpathlineto{\pgfqpoint{0.000000in}{-0.027778in}}%
\pgfusepath{stroke,fill}%
}%
\begin{pgfscope}%
\pgfsys@transformshift{4.556149in}{0.552778in}%
\pgfsys@useobject{currentmarker}{}%
\end{pgfscope}%
\end{pgfscope}%
\begin{pgfscope}%
\pgfsetbuttcap%
\pgfsetroundjoin%
\definecolor{currentfill}{rgb}{0.000000,0.000000,0.000000}%
\pgfsetfillcolor{currentfill}%
\pgfsetlinewidth{0.602250pt}%
\definecolor{currentstroke}{rgb}{0.000000,0.000000,0.000000}%
\pgfsetstrokecolor{currentstroke}%
\pgfsetdash{}{0pt}%
\pgfsys@defobject{currentmarker}{\pgfqpoint{0.000000in}{0.000000in}}{\pgfqpoint{0.000000in}{0.027778in}}{%
\pgfpathmoveto{\pgfqpoint{0.000000in}{0.000000in}}%
\pgfpathlineto{\pgfqpoint{0.000000in}{0.027778in}}%
\pgfusepath{stroke,fill}%
}%
\begin{pgfscope}%
\pgfsys@transformshift{4.556149in}{3.801389in}%
\pgfsys@useobject{currentmarker}{}%
\end{pgfscope}%
\end{pgfscope}%
\begin{pgfscope}%
\pgfpathrectangle{\pgfqpoint{0.781944in}{0.552778in}}{\pgfqpoint{3.890972in}{3.248611in}}%
\pgfusepath{clip}%
\pgfsetrectcap%
\pgfsetroundjoin%
\pgfsetlinewidth{0.803000pt}%
\definecolor{currentstroke}{rgb}{0.690196,0.690196,0.690196}%
\pgfsetstrokecolor{currentstroke}%
\pgfsetstrokeopacity{0.300000}%
\pgfsetdash{}{0pt}%
\pgfpathmoveto{\pgfqpoint{4.595071in}{0.552778in}}%
\pgfpathlineto{\pgfqpoint{4.595071in}{3.801389in}}%
\pgfusepath{stroke}%
\end{pgfscope}%
\begin{pgfscope}%
\pgfsetbuttcap%
\pgfsetroundjoin%
\definecolor{currentfill}{rgb}{0.000000,0.000000,0.000000}%
\pgfsetfillcolor{currentfill}%
\pgfsetlinewidth{0.602250pt}%
\definecolor{currentstroke}{rgb}{0.000000,0.000000,0.000000}%
\pgfsetstrokecolor{currentstroke}%
\pgfsetdash{}{0pt}%
\pgfsys@defobject{currentmarker}{\pgfqpoint{0.000000in}{-0.027778in}}{\pgfqpoint{0.000000in}{0.000000in}}{%
\pgfpathmoveto{\pgfqpoint{0.000000in}{0.000000in}}%
\pgfpathlineto{\pgfqpoint{0.000000in}{-0.027778in}}%
\pgfusepath{stroke,fill}%
}%
\begin{pgfscope}%
\pgfsys@transformshift{4.595071in}{0.552778in}%
\pgfsys@useobject{currentmarker}{}%
\end{pgfscope}%
\end{pgfscope}%
\begin{pgfscope}%
\pgfsetbuttcap%
\pgfsetroundjoin%
\definecolor{currentfill}{rgb}{0.000000,0.000000,0.000000}%
\pgfsetfillcolor{currentfill}%
\pgfsetlinewidth{0.602250pt}%
\definecolor{currentstroke}{rgb}{0.000000,0.000000,0.000000}%
\pgfsetstrokecolor{currentstroke}%
\pgfsetdash{}{0pt}%
\pgfsys@defobject{currentmarker}{\pgfqpoint{0.000000in}{0.000000in}}{\pgfqpoint{0.000000in}{0.027778in}}{%
\pgfpathmoveto{\pgfqpoint{0.000000in}{0.000000in}}%
\pgfpathlineto{\pgfqpoint{0.000000in}{0.027778in}}%
\pgfusepath{stroke,fill}%
}%
\begin{pgfscope}%
\pgfsys@transformshift{4.595071in}{3.801389in}%
\pgfsys@useobject{currentmarker}{}%
\end{pgfscope}%
\end{pgfscope}%
\begin{pgfscope}%
\pgfpathrectangle{\pgfqpoint{0.781944in}{0.552778in}}{\pgfqpoint{3.890972in}{3.248611in}}%
\pgfusepath{clip}%
\pgfsetrectcap%
\pgfsetroundjoin%
\pgfsetlinewidth{0.803000pt}%
\definecolor{currentstroke}{rgb}{0.690196,0.690196,0.690196}%
\pgfsetstrokecolor{currentstroke}%
\pgfsetstrokeopacity{0.300000}%
\pgfsetdash{}{0pt}%
\pgfpathmoveto{\pgfqpoint{4.633994in}{0.552778in}}%
\pgfpathlineto{\pgfqpoint{4.633994in}{3.801389in}}%
\pgfusepath{stroke}%
\end{pgfscope}%
\begin{pgfscope}%
\pgfsetbuttcap%
\pgfsetroundjoin%
\definecolor{currentfill}{rgb}{0.000000,0.000000,0.000000}%
\pgfsetfillcolor{currentfill}%
\pgfsetlinewidth{0.602250pt}%
\definecolor{currentstroke}{rgb}{0.000000,0.000000,0.000000}%
\pgfsetstrokecolor{currentstroke}%
\pgfsetdash{}{0pt}%
\pgfsys@defobject{currentmarker}{\pgfqpoint{0.000000in}{-0.027778in}}{\pgfqpoint{0.000000in}{0.000000in}}{%
\pgfpathmoveto{\pgfqpoint{0.000000in}{0.000000in}}%
\pgfpathlineto{\pgfqpoint{0.000000in}{-0.027778in}}%
\pgfusepath{stroke,fill}%
}%
\begin{pgfscope}%
\pgfsys@transformshift{4.633994in}{0.552778in}%
\pgfsys@useobject{currentmarker}{}%
\end{pgfscope}%
\end{pgfscope}%
\begin{pgfscope}%
\pgfsetbuttcap%
\pgfsetroundjoin%
\definecolor{currentfill}{rgb}{0.000000,0.000000,0.000000}%
\pgfsetfillcolor{currentfill}%
\pgfsetlinewidth{0.602250pt}%
\definecolor{currentstroke}{rgb}{0.000000,0.000000,0.000000}%
\pgfsetstrokecolor{currentstroke}%
\pgfsetdash{}{0pt}%
\pgfsys@defobject{currentmarker}{\pgfqpoint{0.000000in}{0.000000in}}{\pgfqpoint{0.000000in}{0.027778in}}{%
\pgfpathmoveto{\pgfqpoint{0.000000in}{0.000000in}}%
\pgfpathlineto{\pgfqpoint{0.000000in}{0.027778in}}%
\pgfusepath{stroke,fill}%
}%
\begin{pgfscope}%
\pgfsys@transformshift{4.633994in}{3.801389in}%
\pgfsys@useobject{currentmarker}{}%
\end{pgfscope}%
\end{pgfscope}%
\begin{pgfscope}%
\definecolor{textcolor}{rgb}{0.000000,0.000000,0.000000}%
\pgfsetstrokecolor{textcolor}%
\pgfsetfillcolor{textcolor}%
\pgftext[x=2.727431in,y=0.276667in,,top]{\color{textcolor}\rmfamily\fontsize{10.000000}{12.000000}\selectfont Kanal}%
\end{pgfscope}%
\begin{pgfscope}%
\pgfpathrectangle{\pgfqpoint{0.781944in}{0.552778in}}{\pgfqpoint{3.890972in}{3.248611in}}%
\pgfusepath{clip}%
\pgfsetrectcap%
\pgfsetroundjoin%
\pgfsetlinewidth{0.803000pt}%
\definecolor{currentstroke}{rgb}{0.690196,0.690196,0.690196}%
\pgfsetstrokecolor{currentstroke}%
\pgfsetstrokeopacity{0.800000}%
\pgfsetdash{}{0pt}%
\pgfpathmoveto{\pgfqpoint{0.781944in}{0.552778in}}%
\pgfpathlineto{\pgfqpoint{4.672917in}{0.552778in}}%
\pgfusepath{stroke}%
\end{pgfscope}%
\begin{pgfscope}%
\pgfsetbuttcap%
\pgfsetroundjoin%
\definecolor{currentfill}{rgb}{0.000000,0.000000,0.000000}%
\pgfsetfillcolor{currentfill}%
\pgfsetlinewidth{0.803000pt}%
\definecolor{currentstroke}{rgb}{0.000000,0.000000,0.000000}%
\pgfsetstrokecolor{currentstroke}%
\pgfsetdash{}{0pt}%
\pgfsys@defobject{currentmarker}{\pgfqpoint{-0.048611in}{0.000000in}}{\pgfqpoint{0.000000in}{0.000000in}}{%
\pgfpathmoveto{\pgfqpoint{0.000000in}{0.000000in}}%
\pgfpathlineto{\pgfqpoint{-0.048611in}{0.000000in}}%
\pgfusepath{stroke,fill}%
}%
\begin{pgfscope}%
\pgfsys@transformshift{0.781944in}{0.552778in}%
\pgfsys@useobject{currentmarker}{}%
\end{pgfscope}%
\end{pgfscope}%
\begin{pgfscope}%
\pgfsetbuttcap%
\pgfsetroundjoin%
\definecolor{currentfill}{rgb}{0.000000,0.000000,0.000000}%
\pgfsetfillcolor{currentfill}%
\pgfsetlinewidth{0.803000pt}%
\definecolor{currentstroke}{rgb}{0.000000,0.000000,0.000000}%
\pgfsetstrokecolor{currentstroke}%
\pgfsetdash{}{0pt}%
\pgfsys@defobject{currentmarker}{\pgfqpoint{0.000000in}{0.000000in}}{\pgfqpoint{0.048611in}{0.000000in}}{%
\pgfpathmoveto{\pgfqpoint{0.000000in}{0.000000in}}%
\pgfpathlineto{\pgfqpoint{0.048611in}{0.000000in}}%
\pgfusepath{stroke,fill}%
}%
\begin{pgfscope}%
\pgfsys@transformshift{4.672917in}{0.552778in}%
\pgfsys@useobject{currentmarker}{}%
\end{pgfscope}%
\end{pgfscope}%
\begin{pgfscope}%
\definecolor{textcolor}{rgb}{0.000000,0.000000,0.000000}%
\pgfsetstrokecolor{textcolor}%
\pgfsetfillcolor{textcolor}%
\pgftext[x=0.615278in,y=0.504583in,left,base]{\color{textcolor}\rmfamily\fontsize{10.000000}{12.000000}\selectfont 0}%
\end{pgfscope}%
\begin{pgfscope}%
\pgfpathrectangle{\pgfqpoint{0.781944in}{0.552778in}}{\pgfqpoint{3.890972in}{3.248611in}}%
\pgfusepath{clip}%
\pgfsetrectcap%
\pgfsetroundjoin%
\pgfsetlinewidth{0.803000pt}%
\definecolor{currentstroke}{rgb}{0.690196,0.690196,0.690196}%
\pgfsetstrokecolor{currentstroke}%
\pgfsetstrokeopacity{0.800000}%
\pgfsetdash{}{0pt}%
\pgfpathmoveto{\pgfqpoint{0.781944in}{0.983985in}}%
\pgfpathlineto{\pgfqpoint{4.672917in}{0.983985in}}%
\pgfusepath{stroke}%
\end{pgfscope}%
\begin{pgfscope}%
\pgfsetbuttcap%
\pgfsetroundjoin%
\definecolor{currentfill}{rgb}{0.000000,0.000000,0.000000}%
\pgfsetfillcolor{currentfill}%
\pgfsetlinewidth{0.803000pt}%
\definecolor{currentstroke}{rgb}{0.000000,0.000000,0.000000}%
\pgfsetstrokecolor{currentstroke}%
\pgfsetdash{}{0pt}%
\pgfsys@defobject{currentmarker}{\pgfqpoint{-0.048611in}{0.000000in}}{\pgfqpoint{0.000000in}{0.000000in}}{%
\pgfpathmoveto{\pgfqpoint{0.000000in}{0.000000in}}%
\pgfpathlineto{\pgfqpoint{-0.048611in}{0.000000in}}%
\pgfusepath{stroke,fill}%
}%
\begin{pgfscope}%
\pgfsys@transformshift{0.781944in}{0.983985in}%
\pgfsys@useobject{currentmarker}{}%
\end{pgfscope}%
\end{pgfscope}%
\begin{pgfscope}%
\pgfsetbuttcap%
\pgfsetroundjoin%
\definecolor{currentfill}{rgb}{0.000000,0.000000,0.000000}%
\pgfsetfillcolor{currentfill}%
\pgfsetlinewidth{0.803000pt}%
\definecolor{currentstroke}{rgb}{0.000000,0.000000,0.000000}%
\pgfsetstrokecolor{currentstroke}%
\pgfsetdash{}{0pt}%
\pgfsys@defobject{currentmarker}{\pgfqpoint{0.000000in}{0.000000in}}{\pgfqpoint{0.048611in}{0.000000in}}{%
\pgfpathmoveto{\pgfqpoint{0.000000in}{0.000000in}}%
\pgfpathlineto{\pgfqpoint{0.048611in}{0.000000in}}%
\pgfusepath{stroke,fill}%
}%
\begin{pgfscope}%
\pgfsys@transformshift{4.672917in}{0.983985in}%
\pgfsys@useobject{currentmarker}{}%
\end{pgfscope}%
\end{pgfscope}%
\begin{pgfscope}%
\definecolor{textcolor}{rgb}{0.000000,0.000000,0.000000}%
\pgfsetstrokecolor{textcolor}%
\pgfsetfillcolor{textcolor}%
\pgftext[x=0.476389in,y=0.935791in,left,base]{\color{textcolor}\rmfamily\fontsize{10.000000}{12.000000}\selectfont 400}%
\end{pgfscope}%
\begin{pgfscope}%
\pgfpathrectangle{\pgfqpoint{0.781944in}{0.552778in}}{\pgfqpoint{3.890972in}{3.248611in}}%
\pgfusepath{clip}%
\pgfsetrectcap%
\pgfsetroundjoin%
\pgfsetlinewidth{0.803000pt}%
\definecolor{currentstroke}{rgb}{0.690196,0.690196,0.690196}%
\pgfsetstrokecolor{currentstroke}%
\pgfsetstrokeopacity{0.800000}%
\pgfsetdash{}{0pt}%
\pgfpathmoveto{\pgfqpoint{0.781944in}{1.415193in}}%
\pgfpathlineto{\pgfqpoint{4.672917in}{1.415193in}}%
\pgfusepath{stroke}%
\end{pgfscope}%
\begin{pgfscope}%
\pgfsetbuttcap%
\pgfsetroundjoin%
\definecolor{currentfill}{rgb}{0.000000,0.000000,0.000000}%
\pgfsetfillcolor{currentfill}%
\pgfsetlinewidth{0.803000pt}%
\definecolor{currentstroke}{rgb}{0.000000,0.000000,0.000000}%
\pgfsetstrokecolor{currentstroke}%
\pgfsetdash{}{0pt}%
\pgfsys@defobject{currentmarker}{\pgfqpoint{-0.048611in}{0.000000in}}{\pgfqpoint{0.000000in}{0.000000in}}{%
\pgfpathmoveto{\pgfqpoint{0.000000in}{0.000000in}}%
\pgfpathlineto{\pgfqpoint{-0.048611in}{0.000000in}}%
\pgfusepath{stroke,fill}%
}%
\begin{pgfscope}%
\pgfsys@transformshift{0.781944in}{1.415193in}%
\pgfsys@useobject{currentmarker}{}%
\end{pgfscope}%
\end{pgfscope}%
\begin{pgfscope}%
\pgfsetbuttcap%
\pgfsetroundjoin%
\definecolor{currentfill}{rgb}{0.000000,0.000000,0.000000}%
\pgfsetfillcolor{currentfill}%
\pgfsetlinewidth{0.803000pt}%
\definecolor{currentstroke}{rgb}{0.000000,0.000000,0.000000}%
\pgfsetstrokecolor{currentstroke}%
\pgfsetdash{}{0pt}%
\pgfsys@defobject{currentmarker}{\pgfqpoint{0.000000in}{0.000000in}}{\pgfqpoint{0.048611in}{0.000000in}}{%
\pgfpathmoveto{\pgfqpoint{0.000000in}{0.000000in}}%
\pgfpathlineto{\pgfqpoint{0.048611in}{0.000000in}}%
\pgfusepath{stroke,fill}%
}%
\begin{pgfscope}%
\pgfsys@transformshift{4.672917in}{1.415193in}%
\pgfsys@useobject{currentmarker}{}%
\end{pgfscope}%
\end{pgfscope}%
\begin{pgfscope}%
\definecolor{textcolor}{rgb}{0.000000,0.000000,0.000000}%
\pgfsetstrokecolor{textcolor}%
\pgfsetfillcolor{textcolor}%
\pgftext[x=0.476389in,y=1.366999in,left,base]{\color{textcolor}\rmfamily\fontsize{10.000000}{12.000000}\selectfont 800}%
\end{pgfscope}%
\begin{pgfscope}%
\pgfpathrectangle{\pgfqpoint{0.781944in}{0.552778in}}{\pgfqpoint{3.890972in}{3.248611in}}%
\pgfusepath{clip}%
\pgfsetrectcap%
\pgfsetroundjoin%
\pgfsetlinewidth{0.803000pt}%
\definecolor{currentstroke}{rgb}{0.690196,0.690196,0.690196}%
\pgfsetstrokecolor{currentstroke}%
\pgfsetstrokeopacity{0.800000}%
\pgfsetdash{}{0pt}%
\pgfpathmoveto{\pgfqpoint{0.781944in}{1.846401in}}%
\pgfpathlineto{\pgfqpoint{4.672917in}{1.846401in}}%
\pgfusepath{stroke}%
\end{pgfscope}%
\begin{pgfscope}%
\pgfsetbuttcap%
\pgfsetroundjoin%
\definecolor{currentfill}{rgb}{0.000000,0.000000,0.000000}%
\pgfsetfillcolor{currentfill}%
\pgfsetlinewidth{0.803000pt}%
\definecolor{currentstroke}{rgb}{0.000000,0.000000,0.000000}%
\pgfsetstrokecolor{currentstroke}%
\pgfsetdash{}{0pt}%
\pgfsys@defobject{currentmarker}{\pgfqpoint{-0.048611in}{0.000000in}}{\pgfqpoint{0.000000in}{0.000000in}}{%
\pgfpathmoveto{\pgfqpoint{0.000000in}{0.000000in}}%
\pgfpathlineto{\pgfqpoint{-0.048611in}{0.000000in}}%
\pgfusepath{stroke,fill}%
}%
\begin{pgfscope}%
\pgfsys@transformshift{0.781944in}{1.846401in}%
\pgfsys@useobject{currentmarker}{}%
\end{pgfscope}%
\end{pgfscope}%
\begin{pgfscope}%
\pgfsetbuttcap%
\pgfsetroundjoin%
\definecolor{currentfill}{rgb}{0.000000,0.000000,0.000000}%
\pgfsetfillcolor{currentfill}%
\pgfsetlinewidth{0.803000pt}%
\definecolor{currentstroke}{rgb}{0.000000,0.000000,0.000000}%
\pgfsetstrokecolor{currentstroke}%
\pgfsetdash{}{0pt}%
\pgfsys@defobject{currentmarker}{\pgfqpoint{0.000000in}{0.000000in}}{\pgfqpoint{0.048611in}{0.000000in}}{%
\pgfpathmoveto{\pgfqpoint{0.000000in}{0.000000in}}%
\pgfpathlineto{\pgfqpoint{0.048611in}{0.000000in}}%
\pgfusepath{stroke,fill}%
}%
\begin{pgfscope}%
\pgfsys@transformshift{4.672917in}{1.846401in}%
\pgfsys@useobject{currentmarker}{}%
\end{pgfscope}%
\end{pgfscope}%
\begin{pgfscope}%
\definecolor{textcolor}{rgb}{0.000000,0.000000,0.000000}%
\pgfsetstrokecolor{textcolor}%
\pgfsetfillcolor{textcolor}%
\pgftext[x=0.406944in,y=1.798206in,left,base]{\color{textcolor}\rmfamily\fontsize{10.000000}{12.000000}\selectfont 1200}%
\end{pgfscope}%
\begin{pgfscope}%
\pgfpathrectangle{\pgfqpoint{0.781944in}{0.552778in}}{\pgfqpoint{3.890972in}{3.248611in}}%
\pgfusepath{clip}%
\pgfsetrectcap%
\pgfsetroundjoin%
\pgfsetlinewidth{0.803000pt}%
\definecolor{currentstroke}{rgb}{0.690196,0.690196,0.690196}%
\pgfsetstrokecolor{currentstroke}%
\pgfsetstrokeopacity{0.800000}%
\pgfsetdash{}{0pt}%
\pgfpathmoveto{\pgfqpoint{0.781944in}{2.277609in}}%
\pgfpathlineto{\pgfqpoint{4.672917in}{2.277609in}}%
\pgfusepath{stroke}%
\end{pgfscope}%
\begin{pgfscope}%
\pgfsetbuttcap%
\pgfsetroundjoin%
\definecolor{currentfill}{rgb}{0.000000,0.000000,0.000000}%
\pgfsetfillcolor{currentfill}%
\pgfsetlinewidth{0.803000pt}%
\definecolor{currentstroke}{rgb}{0.000000,0.000000,0.000000}%
\pgfsetstrokecolor{currentstroke}%
\pgfsetdash{}{0pt}%
\pgfsys@defobject{currentmarker}{\pgfqpoint{-0.048611in}{0.000000in}}{\pgfqpoint{0.000000in}{0.000000in}}{%
\pgfpathmoveto{\pgfqpoint{0.000000in}{0.000000in}}%
\pgfpathlineto{\pgfqpoint{-0.048611in}{0.000000in}}%
\pgfusepath{stroke,fill}%
}%
\begin{pgfscope}%
\pgfsys@transformshift{0.781944in}{2.277609in}%
\pgfsys@useobject{currentmarker}{}%
\end{pgfscope}%
\end{pgfscope}%
\begin{pgfscope}%
\pgfsetbuttcap%
\pgfsetroundjoin%
\definecolor{currentfill}{rgb}{0.000000,0.000000,0.000000}%
\pgfsetfillcolor{currentfill}%
\pgfsetlinewidth{0.803000pt}%
\definecolor{currentstroke}{rgb}{0.000000,0.000000,0.000000}%
\pgfsetstrokecolor{currentstroke}%
\pgfsetdash{}{0pt}%
\pgfsys@defobject{currentmarker}{\pgfqpoint{0.000000in}{0.000000in}}{\pgfqpoint{0.048611in}{0.000000in}}{%
\pgfpathmoveto{\pgfqpoint{0.000000in}{0.000000in}}%
\pgfpathlineto{\pgfqpoint{0.048611in}{0.000000in}}%
\pgfusepath{stroke,fill}%
}%
\begin{pgfscope}%
\pgfsys@transformshift{4.672917in}{2.277609in}%
\pgfsys@useobject{currentmarker}{}%
\end{pgfscope}%
\end{pgfscope}%
\begin{pgfscope}%
\definecolor{textcolor}{rgb}{0.000000,0.000000,0.000000}%
\pgfsetstrokecolor{textcolor}%
\pgfsetfillcolor{textcolor}%
\pgftext[x=0.406944in,y=2.229414in,left,base]{\color{textcolor}\rmfamily\fontsize{10.000000}{12.000000}\selectfont 1600}%
\end{pgfscope}%
\begin{pgfscope}%
\pgfpathrectangle{\pgfqpoint{0.781944in}{0.552778in}}{\pgfqpoint{3.890972in}{3.248611in}}%
\pgfusepath{clip}%
\pgfsetrectcap%
\pgfsetroundjoin%
\pgfsetlinewidth{0.803000pt}%
\definecolor{currentstroke}{rgb}{0.690196,0.690196,0.690196}%
\pgfsetstrokecolor{currentstroke}%
\pgfsetstrokeopacity{0.800000}%
\pgfsetdash{}{0pt}%
\pgfpathmoveto{\pgfqpoint{0.781944in}{2.708816in}}%
\pgfpathlineto{\pgfqpoint{4.672917in}{2.708816in}}%
\pgfusepath{stroke}%
\end{pgfscope}%
\begin{pgfscope}%
\pgfsetbuttcap%
\pgfsetroundjoin%
\definecolor{currentfill}{rgb}{0.000000,0.000000,0.000000}%
\pgfsetfillcolor{currentfill}%
\pgfsetlinewidth{0.803000pt}%
\definecolor{currentstroke}{rgb}{0.000000,0.000000,0.000000}%
\pgfsetstrokecolor{currentstroke}%
\pgfsetdash{}{0pt}%
\pgfsys@defobject{currentmarker}{\pgfqpoint{-0.048611in}{0.000000in}}{\pgfqpoint{0.000000in}{0.000000in}}{%
\pgfpathmoveto{\pgfqpoint{0.000000in}{0.000000in}}%
\pgfpathlineto{\pgfqpoint{-0.048611in}{0.000000in}}%
\pgfusepath{stroke,fill}%
}%
\begin{pgfscope}%
\pgfsys@transformshift{0.781944in}{2.708816in}%
\pgfsys@useobject{currentmarker}{}%
\end{pgfscope}%
\end{pgfscope}%
\begin{pgfscope}%
\pgfsetbuttcap%
\pgfsetroundjoin%
\definecolor{currentfill}{rgb}{0.000000,0.000000,0.000000}%
\pgfsetfillcolor{currentfill}%
\pgfsetlinewidth{0.803000pt}%
\definecolor{currentstroke}{rgb}{0.000000,0.000000,0.000000}%
\pgfsetstrokecolor{currentstroke}%
\pgfsetdash{}{0pt}%
\pgfsys@defobject{currentmarker}{\pgfqpoint{0.000000in}{0.000000in}}{\pgfqpoint{0.048611in}{0.000000in}}{%
\pgfpathmoveto{\pgfqpoint{0.000000in}{0.000000in}}%
\pgfpathlineto{\pgfqpoint{0.048611in}{0.000000in}}%
\pgfusepath{stroke,fill}%
}%
\begin{pgfscope}%
\pgfsys@transformshift{4.672917in}{2.708816in}%
\pgfsys@useobject{currentmarker}{}%
\end{pgfscope}%
\end{pgfscope}%
\begin{pgfscope}%
\definecolor{textcolor}{rgb}{0.000000,0.000000,0.000000}%
\pgfsetstrokecolor{textcolor}%
\pgfsetfillcolor{textcolor}%
\pgftext[x=0.406944in,y=2.660622in,left,base]{\color{textcolor}\rmfamily\fontsize{10.000000}{12.000000}\selectfont 2000}%
\end{pgfscope}%
\begin{pgfscope}%
\pgfpathrectangle{\pgfqpoint{0.781944in}{0.552778in}}{\pgfqpoint{3.890972in}{3.248611in}}%
\pgfusepath{clip}%
\pgfsetrectcap%
\pgfsetroundjoin%
\pgfsetlinewidth{0.803000pt}%
\definecolor{currentstroke}{rgb}{0.690196,0.690196,0.690196}%
\pgfsetstrokecolor{currentstroke}%
\pgfsetstrokeopacity{0.800000}%
\pgfsetdash{}{0pt}%
\pgfpathmoveto{\pgfqpoint{0.781944in}{3.140024in}}%
\pgfpathlineto{\pgfqpoint{4.672917in}{3.140024in}}%
\pgfusepath{stroke}%
\end{pgfscope}%
\begin{pgfscope}%
\pgfsetbuttcap%
\pgfsetroundjoin%
\definecolor{currentfill}{rgb}{0.000000,0.000000,0.000000}%
\pgfsetfillcolor{currentfill}%
\pgfsetlinewidth{0.803000pt}%
\definecolor{currentstroke}{rgb}{0.000000,0.000000,0.000000}%
\pgfsetstrokecolor{currentstroke}%
\pgfsetdash{}{0pt}%
\pgfsys@defobject{currentmarker}{\pgfqpoint{-0.048611in}{0.000000in}}{\pgfqpoint{0.000000in}{0.000000in}}{%
\pgfpathmoveto{\pgfqpoint{0.000000in}{0.000000in}}%
\pgfpathlineto{\pgfqpoint{-0.048611in}{0.000000in}}%
\pgfusepath{stroke,fill}%
}%
\begin{pgfscope}%
\pgfsys@transformshift{0.781944in}{3.140024in}%
\pgfsys@useobject{currentmarker}{}%
\end{pgfscope}%
\end{pgfscope}%
\begin{pgfscope}%
\pgfsetbuttcap%
\pgfsetroundjoin%
\definecolor{currentfill}{rgb}{0.000000,0.000000,0.000000}%
\pgfsetfillcolor{currentfill}%
\pgfsetlinewidth{0.803000pt}%
\definecolor{currentstroke}{rgb}{0.000000,0.000000,0.000000}%
\pgfsetstrokecolor{currentstroke}%
\pgfsetdash{}{0pt}%
\pgfsys@defobject{currentmarker}{\pgfqpoint{0.000000in}{0.000000in}}{\pgfqpoint{0.048611in}{0.000000in}}{%
\pgfpathmoveto{\pgfqpoint{0.000000in}{0.000000in}}%
\pgfpathlineto{\pgfqpoint{0.048611in}{0.000000in}}%
\pgfusepath{stroke,fill}%
}%
\begin{pgfscope}%
\pgfsys@transformshift{4.672917in}{3.140024in}%
\pgfsys@useobject{currentmarker}{}%
\end{pgfscope}%
\end{pgfscope}%
\begin{pgfscope}%
\definecolor{textcolor}{rgb}{0.000000,0.000000,0.000000}%
\pgfsetstrokecolor{textcolor}%
\pgfsetfillcolor{textcolor}%
\pgftext[x=0.406944in,y=3.091830in,left,base]{\color{textcolor}\rmfamily\fontsize{10.000000}{12.000000}\selectfont 2400}%
\end{pgfscope}%
\begin{pgfscope}%
\pgfpathrectangle{\pgfqpoint{0.781944in}{0.552778in}}{\pgfqpoint{3.890972in}{3.248611in}}%
\pgfusepath{clip}%
\pgfsetrectcap%
\pgfsetroundjoin%
\pgfsetlinewidth{0.803000pt}%
\definecolor{currentstroke}{rgb}{0.690196,0.690196,0.690196}%
\pgfsetstrokecolor{currentstroke}%
\pgfsetstrokeopacity{0.800000}%
\pgfsetdash{}{0pt}%
\pgfpathmoveto{\pgfqpoint{0.781944in}{3.571232in}}%
\pgfpathlineto{\pgfqpoint{4.672917in}{3.571232in}}%
\pgfusepath{stroke}%
\end{pgfscope}%
\begin{pgfscope}%
\pgfsetbuttcap%
\pgfsetroundjoin%
\definecolor{currentfill}{rgb}{0.000000,0.000000,0.000000}%
\pgfsetfillcolor{currentfill}%
\pgfsetlinewidth{0.803000pt}%
\definecolor{currentstroke}{rgb}{0.000000,0.000000,0.000000}%
\pgfsetstrokecolor{currentstroke}%
\pgfsetdash{}{0pt}%
\pgfsys@defobject{currentmarker}{\pgfqpoint{-0.048611in}{0.000000in}}{\pgfqpoint{0.000000in}{0.000000in}}{%
\pgfpathmoveto{\pgfqpoint{0.000000in}{0.000000in}}%
\pgfpathlineto{\pgfqpoint{-0.048611in}{0.000000in}}%
\pgfusepath{stroke,fill}%
}%
\begin{pgfscope}%
\pgfsys@transformshift{0.781944in}{3.571232in}%
\pgfsys@useobject{currentmarker}{}%
\end{pgfscope}%
\end{pgfscope}%
\begin{pgfscope}%
\pgfsetbuttcap%
\pgfsetroundjoin%
\definecolor{currentfill}{rgb}{0.000000,0.000000,0.000000}%
\pgfsetfillcolor{currentfill}%
\pgfsetlinewidth{0.803000pt}%
\definecolor{currentstroke}{rgb}{0.000000,0.000000,0.000000}%
\pgfsetstrokecolor{currentstroke}%
\pgfsetdash{}{0pt}%
\pgfsys@defobject{currentmarker}{\pgfqpoint{0.000000in}{0.000000in}}{\pgfqpoint{0.048611in}{0.000000in}}{%
\pgfpathmoveto{\pgfqpoint{0.000000in}{0.000000in}}%
\pgfpathlineto{\pgfqpoint{0.048611in}{0.000000in}}%
\pgfusepath{stroke,fill}%
}%
\begin{pgfscope}%
\pgfsys@transformshift{4.672917in}{3.571232in}%
\pgfsys@useobject{currentmarker}{}%
\end{pgfscope}%
\end{pgfscope}%
\begin{pgfscope}%
\definecolor{textcolor}{rgb}{0.000000,0.000000,0.000000}%
\pgfsetstrokecolor{textcolor}%
\pgfsetfillcolor{textcolor}%
\pgftext[x=0.406944in,y=3.523037in,left,base]{\color{textcolor}\rmfamily\fontsize{10.000000}{12.000000}\selectfont 2800}%
\end{pgfscope}%
\begin{pgfscope}%
\pgfpathrectangle{\pgfqpoint{0.781944in}{0.552778in}}{\pgfqpoint{3.890972in}{3.248611in}}%
\pgfusepath{clip}%
\pgfsetrectcap%
\pgfsetroundjoin%
\pgfsetlinewidth{0.803000pt}%
\definecolor{currentstroke}{rgb}{0.690196,0.690196,0.690196}%
\pgfsetstrokecolor{currentstroke}%
\pgfsetstrokeopacity{0.300000}%
\pgfsetdash{}{0pt}%
\pgfpathmoveto{\pgfqpoint{0.781944in}{0.595899in}}%
\pgfpathlineto{\pgfqpoint{4.672917in}{0.595899in}}%
\pgfusepath{stroke}%
\end{pgfscope}%
\begin{pgfscope}%
\pgfsetbuttcap%
\pgfsetroundjoin%
\definecolor{currentfill}{rgb}{0.000000,0.000000,0.000000}%
\pgfsetfillcolor{currentfill}%
\pgfsetlinewidth{0.602250pt}%
\definecolor{currentstroke}{rgb}{0.000000,0.000000,0.000000}%
\pgfsetstrokecolor{currentstroke}%
\pgfsetdash{}{0pt}%
\pgfsys@defobject{currentmarker}{\pgfqpoint{-0.027778in}{0.000000in}}{\pgfqpoint{0.000000in}{0.000000in}}{%
\pgfpathmoveto{\pgfqpoint{0.000000in}{0.000000in}}%
\pgfpathlineto{\pgfqpoint{-0.027778in}{0.000000in}}%
\pgfusepath{stroke,fill}%
}%
\begin{pgfscope}%
\pgfsys@transformshift{0.781944in}{0.595899in}%
\pgfsys@useobject{currentmarker}{}%
\end{pgfscope}%
\end{pgfscope}%
\begin{pgfscope}%
\pgfsetbuttcap%
\pgfsetroundjoin%
\definecolor{currentfill}{rgb}{0.000000,0.000000,0.000000}%
\pgfsetfillcolor{currentfill}%
\pgfsetlinewidth{0.602250pt}%
\definecolor{currentstroke}{rgb}{0.000000,0.000000,0.000000}%
\pgfsetstrokecolor{currentstroke}%
\pgfsetdash{}{0pt}%
\pgfsys@defobject{currentmarker}{\pgfqpoint{0.000000in}{0.000000in}}{\pgfqpoint{0.027778in}{0.000000in}}{%
\pgfpathmoveto{\pgfqpoint{0.000000in}{0.000000in}}%
\pgfpathlineto{\pgfqpoint{0.027778in}{0.000000in}}%
\pgfusepath{stroke,fill}%
}%
\begin{pgfscope}%
\pgfsys@transformshift{4.672917in}{0.595899in}%
\pgfsys@useobject{currentmarker}{}%
\end{pgfscope}%
\end{pgfscope}%
\begin{pgfscope}%
\pgfpathrectangle{\pgfqpoint{0.781944in}{0.552778in}}{\pgfqpoint{3.890972in}{3.248611in}}%
\pgfusepath{clip}%
\pgfsetrectcap%
\pgfsetroundjoin%
\pgfsetlinewidth{0.803000pt}%
\definecolor{currentstroke}{rgb}{0.690196,0.690196,0.690196}%
\pgfsetstrokecolor{currentstroke}%
\pgfsetstrokeopacity{0.300000}%
\pgfsetdash{}{0pt}%
\pgfpathmoveto{\pgfqpoint{0.781944in}{0.639019in}}%
\pgfpathlineto{\pgfqpoint{4.672917in}{0.639019in}}%
\pgfusepath{stroke}%
\end{pgfscope}%
\begin{pgfscope}%
\pgfsetbuttcap%
\pgfsetroundjoin%
\definecolor{currentfill}{rgb}{0.000000,0.000000,0.000000}%
\pgfsetfillcolor{currentfill}%
\pgfsetlinewidth{0.602250pt}%
\definecolor{currentstroke}{rgb}{0.000000,0.000000,0.000000}%
\pgfsetstrokecolor{currentstroke}%
\pgfsetdash{}{0pt}%
\pgfsys@defobject{currentmarker}{\pgfqpoint{-0.027778in}{0.000000in}}{\pgfqpoint{0.000000in}{0.000000in}}{%
\pgfpathmoveto{\pgfqpoint{0.000000in}{0.000000in}}%
\pgfpathlineto{\pgfqpoint{-0.027778in}{0.000000in}}%
\pgfusepath{stroke,fill}%
}%
\begin{pgfscope}%
\pgfsys@transformshift{0.781944in}{0.639019in}%
\pgfsys@useobject{currentmarker}{}%
\end{pgfscope}%
\end{pgfscope}%
\begin{pgfscope}%
\pgfsetbuttcap%
\pgfsetroundjoin%
\definecolor{currentfill}{rgb}{0.000000,0.000000,0.000000}%
\pgfsetfillcolor{currentfill}%
\pgfsetlinewidth{0.602250pt}%
\definecolor{currentstroke}{rgb}{0.000000,0.000000,0.000000}%
\pgfsetstrokecolor{currentstroke}%
\pgfsetdash{}{0pt}%
\pgfsys@defobject{currentmarker}{\pgfqpoint{0.000000in}{0.000000in}}{\pgfqpoint{0.027778in}{0.000000in}}{%
\pgfpathmoveto{\pgfqpoint{0.000000in}{0.000000in}}%
\pgfpathlineto{\pgfqpoint{0.027778in}{0.000000in}}%
\pgfusepath{stroke,fill}%
}%
\begin{pgfscope}%
\pgfsys@transformshift{4.672917in}{0.639019in}%
\pgfsys@useobject{currentmarker}{}%
\end{pgfscope}%
\end{pgfscope}%
\begin{pgfscope}%
\pgfpathrectangle{\pgfqpoint{0.781944in}{0.552778in}}{\pgfqpoint{3.890972in}{3.248611in}}%
\pgfusepath{clip}%
\pgfsetrectcap%
\pgfsetroundjoin%
\pgfsetlinewidth{0.803000pt}%
\definecolor{currentstroke}{rgb}{0.690196,0.690196,0.690196}%
\pgfsetstrokecolor{currentstroke}%
\pgfsetstrokeopacity{0.300000}%
\pgfsetdash{}{0pt}%
\pgfpathmoveto{\pgfqpoint{0.781944in}{0.682140in}}%
\pgfpathlineto{\pgfqpoint{4.672917in}{0.682140in}}%
\pgfusepath{stroke}%
\end{pgfscope}%
\begin{pgfscope}%
\pgfsetbuttcap%
\pgfsetroundjoin%
\definecolor{currentfill}{rgb}{0.000000,0.000000,0.000000}%
\pgfsetfillcolor{currentfill}%
\pgfsetlinewidth{0.602250pt}%
\definecolor{currentstroke}{rgb}{0.000000,0.000000,0.000000}%
\pgfsetstrokecolor{currentstroke}%
\pgfsetdash{}{0pt}%
\pgfsys@defobject{currentmarker}{\pgfqpoint{-0.027778in}{0.000000in}}{\pgfqpoint{0.000000in}{0.000000in}}{%
\pgfpathmoveto{\pgfqpoint{0.000000in}{0.000000in}}%
\pgfpathlineto{\pgfqpoint{-0.027778in}{0.000000in}}%
\pgfusepath{stroke,fill}%
}%
\begin{pgfscope}%
\pgfsys@transformshift{0.781944in}{0.682140in}%
\pgfsys@useobject{currentmarker}{}%
\end{pgfscope}%
\end{pgfscope}%
\begin{pgfscope}%
\pgfsetbuttcap%
\pgfsetroundjoin%
\definecolor{currentfill}{rgb}{0.000000,0.000000,0.000000}%
\pgfsetfillcolor{currentfill}%
\pgfsetlinewidth{0.602250pt}%
\definecolor{currentstroke}{rgb}{0.000000,0.000000,0.000000}%
\pgfsetstrokecolor{currentstroke}%
\pgfsetdash{}{0pt}%
\pgfsys@defobject{currentmarker}{\pgfqpoint{0.000000in}{0.000000in}}{\pgfqpoint{0.027778in}{0.000000in}}{%
\pgfpathmoveto{\pgfqpoint{0.000000in}{0.000000in}}%
\pgfpathlineto{\pgfqpoint{0.027778in}{0.000000in}}%
\pgfusepath{stroke,fill}%
}%
\begin{pgfscope}%
\pgfsys@transformshift{4.672917in}{0.682140in}%
\pgfsys@useobject{currentmarker}{}%
\end{pgfscope}%
\end{pgfscope}%
\begin{pgfscope}%
\pgfpathrectangle{\pgfqpoint{0.781944in}{0.552778in}}{\pgfqpoint{3.890972in}{3.248611in}}%
\pgfusepath{clip}%
\pgfsetrectcap%
\pgfsetroundjoin%
\pgfsetlinewidth{0.803000pt}%
\definecolor{currentstroke}{rgb}{0.690196,0.690196,0.690196}%
\pgfsetstrokecolor{currentstroke}%
\pgfsetstrokeopacity{0.300000}%
\pgfsetdash{}{0pt}%
\pgfpathmoveto{\pgfqpoint{0.781944in}{0.725261in}}%
\pgfpathlineto{\pgfqpoint{4.672917in}{0.725261in}}%
\pgfusepath{stroke}%
\end{pgfscope}%
\begin{pgfscope}%
\pgfsetbuttcap%
\pgfsetroundjoin%
\definecolor{currentfill}{rgb}{0.000000,0.000000,0.000000}%
\pgfsetfillcolor{currentfill}%
\pgfsetlinewidth{0.602250pt}%
\definecolor{currentstroke}{rgb}{0.000000,0.000000,0.000000}%
\pgfsetstrokecolor{currentstroke}%
\pgfsetdash{}{0pt}%
\pgfsys@defobject{currentmarker}{\pgfqpoint{-0.027778in}{0.000000in}}{\pgfqpoint{0.000000in}{0.000000in}}{%
\pgfpathmoveto{\pgfqpoint{0.000000in}{0.000000in}}%
\pgfpathlineto{\pgfqpoint{-0.027778in}{0.000000in}}%
\pgfusepath{stroke,fill}%
}%
\begin{pgfscope}%
\pgfsys@transformshift{0.781944in}{0.725261in}%
\pgfsys@useobject{currentmarker}{}%
\end{pgfscope}%
\end{pgfscope}%
\begin{pgfscope}%
\pgfsetbuttcap%
\pgfsetroundjoin%
\definecolor{currentfill}{rgb}{0.000000,0.000000,0.000000}%
\pgfsetfillcolor{currentfill}%
\pgfsetlinewidth{0.602250pt}%
\definecolor{currentstroke}{rgb}{0.000000,0.000000,0.000000}%
\pgfsetstrokecolor{currentstroke}%
\pgfsetdash{}{0pt}%
\pgfsys@defobject{currentmarker}{\pgfqpoint{0.000000in}{0.000000in}}{\pgfqpoint{0.027778in}{0.000000in}}{%
\pgfpathmoveto{\pgfqpoint{0.000000in}{0.000000in}}%
\pgfpathlineto{\pgfqpoint{0.027778in}{0.000000in}}%
\pgfusepath{stroke,fill}%
}%
\begin{pgfscope}%
\pgfsys@transformshift{4.672917in}{0.725261in}%
\pgfsys@useobject{currentmarker}{}%
\end{pgfscope}%
\end{pgfscope}%
\begin{pgfscope}%
\pgfpathrectangle{\pgfqpoint{0.781944in}{0.552778in}}{\pgfqpoint{3.890972in}{3.248611in}}%
\pgfusepath{clip}%
\pgfsetrectcap%
\pgfsetroundjoin%
\pgfsetlinewidth{0.803000pt}%
\definecolor{currentstroke}{rgb}{0.690196,0.690196,0.690196}%
\pgfsetstrokecolor{currentstroke}%
\pgfsetstrokeopacity{0.300000}%
\pgfsetdash{}{0pt}%
\pgfpathmoveto{\pgfqpoint{0.781944in}{0.768382in}}%
\pgfpathlineto{\pgfqpoint{4.672917in}{0.768382in}}%
\pgfusepath{stroke}%
\end{pgfscope}%
\begin{pgfscope}%
\pgfsetbuttcap%
\pgfsetroundjoin%
\definecolor{currentfill}{rgb}{0.000000,0.000000,0.000000}%
\pgfsetfillcolor{currentfill}%
\pgfsetlinewidth{0.602250pt}%
\definecolor{currentstroke}{rgb}{0.000000,0.000000,0.000000}%
\pgfsetstrokecolor{currentstroke}%
\pgfsetdash{}{0pt}%
\pgfsys@defobject{currentmarker}{\pgfqpoint{-0.027778in}{0.000000in}}{\pgfqpoint{0.000000in}{0.000000in}}{%
\pgfpathmoveto{\pgfqpoint{0.000000in}{0.000000in}}%
\pgfpathlineto{\pgfqpoint{-0.027778in}{0.000000in}}%
\pgfusepath{stroke,fill}%
}%
\begin{pgfscope}%
\pgfsys@transformshift{0.781944in}{0.768382in}%
\pgfsys@useobject{currentmarker}{}%
\end{pgfscope}%
\end{pgfscope}%
\begin{pgfscope}%
\pgfsetbuttcap%
\pgfsetroundjoin%
\definecolor{currentfill}{rgb}{0.000000,0.000000,0.000000}%
\pgfsetfillcolor{currentfill}%
\pgfsetlinewidth{0.602250pt}%
\definecolor{currentstroke}{rgb}{0.000000,0.000000,0.000000}%
\pgfsetstrokecolor{currentstroke}%
\pgfsetdash{}{0pt}%
\pgfsys@defobject{currentmarker}{\pgfqpoint{0.000000in}{0.000000in}}{\pgfqpoint{0.027778in}{0.000000in}}{%
\pgfpathmoveto{\pgfqpoint{0.000000in}{0.000000in}}%
\pgfpathlineto{\pgfqpoint{0.027778in}{0.000000in}}%
\pgfusepath{stroke,fill}%
}%
\begin{pgfscope}%
\pgfsys@transformshift{4.672917in}{0.768382in}%
\pgfsys@useobject{currentmarker}{}%
\end{pgfscope}%
\end{pgfscope}%
\begin{pgfscope}%
\pgfpathrectangle{\pgfqpoint{0.781944in}{0.552778in}}{\pgfqpoint{3.890972in}{3.248611in}}%
\pgfusepath{clip}%
\pgfsetrectcap%
\pgfsetroundjoin%
\pgfsetlinewidth{0.803000pt}%
\definecolor{currentstroke}{rgb}{0.690196,0.690196,0.690196}%
\pgfsetstrokecolor{currentstroke}%
\pgfsetstrokeopacity{0.300000}%
\pgfsetdash{}{0pt}%
\pgfpathmoveto{\pgfqpoint{0.781944in}{0.811502in}}%
\pgfpathlineto{\pgfqpoint{4.672917in}{0.811502in}}%
\pgfusepath{stroke}%
\end{pgfscope}%
\begin{pgfscope}%
\pgfsetbuttcap%
\pgfsetroundjoin%
\definecolor{currentfill}{rgb}{0.000000,0.000000,0.000000}%
\pgfsetfillcolor{currentfill}%
\pgfsetlinewidth{0.602250pt}%
\definecolor{currentstroke}{rgb}{0.000000,0.000000,0.000000}%
\pgfsetstrokecolor{currentstroke}%
\pgfsetdash{}{0pt}%
\pgfsys@defobject{currentmarker}{\pgfqpoint{-0.027778in}{0.000000in}}{\pgfqpoint{0.000000in}{0.000000in}}{%
\pgfpathmoveto{\pgfqpoint{0.000000in}{0.000000in}}%
\pgfpathlineto{\pgfqpoint{-0.027778in}{0.000000in}}%
\pgfusepath{stroke,fill}%
}%
\begin{pgfscope}%
\pgfsys@transformshift{0.781944in}{0.811502in}%
\pgfsys@useobject{currentmarker}{}%
\end{pgfscope}%
\end{pgfscope}%
\begin{pgfscope}%
\pgfsetbuttcap%
\pgfsetroundjoin%
\definecolor{currentfill}{rgb}{0.000000,0.000000,0.000000}%
\pgfsetfillcolor{currentfill}%
\pgfsetlinewidth{0.602250pt}%
\definecolor{currentstroke}{rgb}{0.000000,0.000000,0.000000}%
\pgfsetstrokecolor{currentstroke}%
\pgfsetdash{}{0pt}%
\pgfsys@defobject{currentmarker}{\pgfqpoint{0.000000in}{0.000000in}}{\pgfqpoint{0.027778in}{0.000000in}}{%
\pgfpathmoveto{\pgfqpoint{0.000000in}{0.000000in}}%
\pgfpathlineto{\pgfqpoint{0.027778in}{0.000000in}}%
\pgfusepath{stroke,fill}%
}%
\begin{pgfscope}%
\pgfsys@transformshift{4.672917in}{0.811502in}%
\pgfsys@useobject{currentmarker}{}%
\end{pgfscope}%
\end{pgfscope}%
\begin{pgfscope}%
\pgfpathrectangle{\pgfqpoint{0.781944in}{0.552778in}}{\pgfqpoint{3.890972in}{3.248611in}}%
\pgfusepath{clip}%
\pgfsetrectcap%
\pgfsetroundjoin%
\pgfsetlinewidth{0.803000pt}%
\definecolor{currentstroke}{rgb}{0.690196,0.690196,0.690196}%
\pgfsetstrokecolor{currentstroke}%
\pgfsetstrokeopacity{0.300000}%
\pgfsetdash{}{0pt}%
\pgfpathmoveto{\pgfqpoint{0.781944in}{0.854623in}}%
\pgfpathlineto{\pgfqpoint{4.672917in}{0.854623in}}%
\pgfusepath{stroke}%
\end{pgfscope}%
\begin{pgfscope}%
\pgfsetbuttcap%
\pgfsetroundjoin%
\definecolor{currentfill}{rgb}{0.000000,0.000000,0.000000}%
\pgfsetfillcolor{currentfill}%
\pgfsetlinewidth{0.602250pt}%
\definecolor{currentstroke}{rgb}{0.000000,0.000000,0.000000}%
\pgfsetstrokecolor{currentstroke}%
\pgfsetdash{}{0pt}%
\pgfsys@defobject{currentmarker}{\pgfqpoint{-0.027778in}{0.000000in}}{\pgfqpoint{0.000000in}{0.000000in}}{%
\pgfpathmoveto{\pgfqpoint{0.000000in}{0.000000in}}%
\pgfpathlineto{\pgfqpoint{-0.027778in}{0.000000in}}%
\pgfusepath{stroke,fill}%
}%
\begin{pgfscope}%
\pgfsys@transformshift{0.781944in}{0.854623in}%
\pgfsys@useobject{currentmarker}{}%
\end{pgfscope}%
\end{pgfscope}%
\begin{pgfscope}%
\pgfsetbuttcap%
\pgfsetroundjoin%
\definecolor{currentfill}{rgb}{0.000000,0.000000,0.000000}%
\pgfsetfillcolor{currentfill}%
\pgfsetlinewidth{0.602250pt}%
\definecolor{currentstroke}{rgb}{0.000000,0.000000,0.000000}%
\pgfsetstrokecolor{currentstroke}%
\pgfsetdash{}{0pt}%
\pgfsys@defobject{currentmarker}{\pgfqpoint{0.000000in}{0.000000in}}{\pgfqpoint{0.027778in}{0.000000in}}{%
\pgfpathmoveto{\pgfqpoint{0.000000in}{0.000000in}}%
\pgfpathlineto{\pgfqpoint{0.027778in}{0.000000in}}%
\pgfusepath{stroke,fill}%
}%
\begin{pgfscope}%
\pgfsys@transformshift{4.672917in}{0.854623in}%
\pgfsys@useobject{currentmarker}{}%
\end{pgfscope}%
\end{pgfscope}%
\begin{pgfscope}%
\pgfpathrectangle{\pgfqpoint{0.781944in}{0.552778in}}{\pgfqpoint{3.890972in}{3.248611in}}%
\pgfusepath{clip}%
\pgfsetrectcap%
\pgfsetroundjoin%
\pgfsetlinewidth{0.803000pt}%
\definecolor{currentstroke}{rgb}{0.690196,0.690196,0.690196}%
\pgfsetstrokecolor{currentstroke}%
\pgfsetstrokeopacity{0.300000}%
\pgfsetdash{}{0pt}%
\pgfpathmoveto{\pgfqpoint{0.781944in}{0.897744in}}%
\pgfpathlineto{\pgfqpoint{4.672917in}{0.897744in}}%
\pgfusepath{stroke}%
\end{pgfscope}%
\begin{pgfscope}%
\pgfsetbuttcap%
\pgfsetroundjoin%
\definecolor{currentfill}{rgb}{0.000000,0.000000,0.000000}%
\pgfsetfillcolor{currentfill}%
\pgfsetlinewidth{0.602250pt}%
\definecolor{currentstroke}{rgb}{0.000000,0.000000,0.000000}%
\pgfsetstrokecolor{currentstroke}%
\pgfsetdash{}{0pt}%
\pgfsys@defobject{currentmarker}{\pgfqpoint{-0.027778in}{0.000000in}}{\pgfqpoint{0.000000in}{0.000000in}}{%
\pgfpathmoveto{\pgfqpoint{0.000000in}{0.000000in}}%
\pgfpathlineto{\pgfqpoint{-0.027778in}{0.000000in}}%
\pgfusepath{stroke,fill}%
}%
\begin{pgfscope}%
\pgfsys@transformshift{0.781944in}{0.897744in}%
\pgfsys@useobject{currentmarker}{}%
\end{pgfscope}%
\end{pgfscope}%
\begin{pgfscope}%
\pgfsetbuttcap%
\pgfsetroundjoin%
\definecolor{currentfill}{rgb}{0.000000,0.000000,0.000000}%
\pgfsetfillcolor{currentfill}%
\pgfsetlinewidth{0.602250pt}%
\definecolor{currentstroke}{rgb}{0.000000,0.000000,0.000000}%
\pgfsetstrokecolor{currentstroke}%
\pgfsetdash{}{0pt}%
\pgfsys@defobject{currentmarker}{\pgfqpoint{0.000000in}{0.000000in}}{\pgfqpoint{0.027778in}{0.000000in}}{%
\pgfpathmoveto{\pgfqpoint{0.000000in}{0.000000in}}%
\pgfpathlineto{\pgfqpoint{0.027778in}{0.000000in}}%
\pgfusepath{stroke,fill}%
}%
\begin{pgfscope}%
\pgfsys@transformshift{4.672917in}{0.897744in}%
\pgfsys@useobject{currentmarker}{}%
\end{pgfscope}%
\end{pgfscope}%
\begin{pgfscope}%
\pgfpathrectangle{\pgfqpoint{0.781944in}{0.552778in}}{\pgfqpoint{3.890972in}{3.248611in}}%
\pgfusepath{clip}%
\pgfsetrectcap%
\pgfsetroundjoin%
\pgfsetlinewidth{0.803000pt}%
\definecolor{currentstroke}{rgb}{0.690196,0.690196,0.690196}%
\pgfsetstrokecolor{currentstroke}%
\pgfsetstrokeopacity{0.300000}%
\pgfsetdash{}{0pt}%
\pgfpathmoveto{\pgfqpoint{0.781944in}{0.940865in}}%
\pgfpathlineto{\pgfqpoint{4.672917in}{0.940865in}}%
\pgfusepath{stroke}%
\end{pgfscope}%
\begin{pgfscope}%
\pgfsetbuttcap%
\pgfsetroundjoin%
\definecolor{currentfill}{rgb}{0.000000,0.000000,0.000000}%
\pgfsetfillcolor{currentfill}%
\pgfsetlinewidth{0.602250pt}%
\definecolor{currentstroke}{rgb}{0.000000,0.000000,0.000000}%
\pgfsetstrokecolor{currentstroke}%
\pgfsetdash{}{0pt}%
\pgfsys@defobject{currentmarker}{\pgfqpoint{-0.027778in}{0.000000in}}{\pgfqpoint{0.000000in}{0.000000in}}{%
\pgfpathmoveto{\pgfqpoint{0.000000in}{0.000000in}}%
\pgfpathlineto{\pgfqpoint{-0.027778in}{0.000000in}}%
\pgfusepath{stroke,fill}%
}%
\begin{pgfscope}%
\pgfsys@transformshift{0.781944in}{0.940865in}%
\pgfsys@useobject{currentmarker}{}%
\end{pgfscope}%
\end{pgfscope}%
\begin{pgfscope}%
\pgfsetbuttcap%
\pgfsetroundjoin%
\definecolor{currentfill}{rgb}{0.000000,0.000000,0.000000}%
\pgfsetfillcolor{currentfill}%
\pgfsetlinewidth{0.602250pt}%
\definecolor{currentstroke}{rgb}{0.000000,0.000000,0.000000}%
\pgfsetstrokecolor{currentstroke}%
\pgfsetdash{}{0pt}%
\pgfsys@defobject{currentmarker}{\pgfqpoint{0.000000in}{0.000000in}}{\pgfqpoint{0.027778in}{0.000000in}}{%
\pgfpathmoveto{\pgfqpoint{0.000000in}{0.000000in}}%
\pgfpathlineto{\pgfqpoint{0.027778in}{0.000000in}}%
\pgfusepath{stroke,fill}%
}%
\begin{pgfscope}%
\pgfsys@transformshift{4.672917in}{0.940865in}%
\pgfsys@useobject{currentmarker}{}%
\end{pgfscope}%
\end{pgfscope}%
\begin{pgfscope}%
\pgfpathrectangle{\pgfqpoint{0.781944in}{0.552778in}}{\pgfqpoint{3.890972in}{3.248611in}}%
\pgfusepath{clip}%
\pgfsetrectcap%
\pgfsetroundjoin%
\pgfsetlinewidth{0.803000pt}%
\definecolor{currentstroke}{rgb}{0.690196,0.690196,0.690196}%
\pgfsetstrokecolor{currentstroke}%
\pgfsetstrokeopacity{0.300000}%
\pgfsetdash{}{0pt}%
\pgfpathmoveto{\pgfqpoint{0.781944in}{1.027106in}}%
\pgfpathlineto{\pgfqpoint{4.672917in}{1.027106in}}%
\pgfusepath{stroke}%
\end{pgfscope}%
\begin{pgfscope}%
\pgfsetbuttcap%
\pgfsetroundjoin%
\definecolor{currentfill}{rgb}{0.000000,0.000000,0.000000}%
\pgfsetfillcolor{currentfill}%
\pgfsetlinewidth{0.602250pt}%
\definecolor{currentstroke}{rgb}{0.000000,0.000000,0.000000}%
\pgfsetstrokecolor{currentstroke}%
\pgfsetdash{}{0pt}%
\pgfsys@defobject{currentmarker}{\pgfqpoint{-0.027778in}{0.000000in}}{\pgfqpoint{0.000000in}{0.000000in}}{%
\pgfpathmoveto{\pgfqpoint{0.000000in}{0.000000in}}%
\pgfpathlineto{\pgfqpoint{-0.027778in}{0.000000in}}%
\pgfusepath{stroke,fill}%
}%
\begin{pgfscope}%
\pgfsys@transformshift{0.781944in}{1.027106in}%
\pgfsys@useobject{currentmarker}{}%
\end{pgfscope}%
\end{pgfscope}%
\begin{pgfscope}%
\pgfsetbuttcap%
\pgfsetroundjoin%
\definecolor{currentfill}{rgb}{0.000000,0.000000,0.000000}%
\pgfsetfillcolor{currentfill}%
\pgfsetlinewidth{0.602250pt}%
\definecolor{currentstroke}{rgb}{0.000000,0.000000,0.000000}%
\pgfsetstrokecolor{currentstroke}%
\pgfsetdash{}{0pt}%
\pgfsys@defobject{currentmarker}{\pgfqpoint{0.000000in}{0.000000in}}{\pgfqpoint{0.027778in}{0.000000in}}{%
\pgfpathmoveto{\pgfqpoint{0.000000in}{0.000000in}}%
\pgfpathlineto{\pgfqpoint{0.027778in}{0.000000in}}%
\pgfusepath{stroke,fill}%
}%
\begin{pgfscope}%
\pgfsys@transformshift{4.672917in}{1.027106in}%
\pgfsys@useobject{currentmarker}{}%
\end{pgfscope}%
\end{pgfscope}%
\begin{pgfscope}%
\pgfpathrectangle{\pgfqpoint{0.781944in}{0.552778in}}{\pgfqpoint{3.890972in}{3.248611in}}%
\pgfusepath{clip}%
\pgfsetrectcap%
\pgfsetroundjoin%
\pgfsetlinewidth{0.803000pt}%
\definecolor{currentstroke}{rgb}{0.690196,0.690196,0.690196}%
\pgfsetstrokecolor{currentstroke}%
\pgfsetstrokeopacity{0.300000}%
\pgfsetdash{}{0pt}%
\pgfpathmoveto{\pgfqpoint{0.781944in}{1.070227in}}%
\pgfpathlineto{\pgfqpoint{4.672917in}{1.070227in}}%
\pgfusepath{stroke}%
\end{pgfscope}%
\begin{pgfscope}%
\pgfsetbuttcap%
\pgfsetroundjoin%
\definecolor{currentfill}{rgb}{0.000000,0.000000,0.000000}%
\pgfsetfillcolor{currentfill}%
\pgfsetlinewidth{0.602250pt}%
\definecolor{currentstroke}{rgb}{0.000000,0.000000,0.000000}%
\pgfsetstrokecolor{currentstroke}%
\pgfsetdash{}{0pt}%
\pgfsys@defobject{currentmarker}{\pgfqpoint{-0.027778in}{0.000000in}}{\pgfqpoint{0.000000in}{0.000000in}}{%
\pgfpathmoveto{\pgfqpoint{0.000000in}{0.000000in}}%
\pgfpathlineto{\pgfqpoint{-0.027778in}{0.000000in}}%
\pgfusepath{stroke,fill}%
}%
\begin{pgfscope}%
\pgfsys@transformshift{0.781944in}{1.070227in}%
\pgfsys@useobject{currentmarker}{}%
\end{pgfscope}%
\end{pgfscope}%
\begin{pgfscope}%
\pgfsetbuttcap%
\pgfsetroundjoin%
\definecolor{currentfill}{rgb}{0.000000,0.000000,0.000000}%
\pgfsetfillcolor{currentfill}%
\pgfsetlinewidth{0.602250pt}%
\definecolor{currentstroke}{rgb}{0.000000,0.000000,0.000000}%
\pgfsetstrokecolor{currentstroke}%
\pgfsetdash{}{0pt}%
\pgfsys@defobject{currentmarker}{\pgfqpoint{0.000000in}{0.000000in}}{\pgfqpoint{0.027778in}{0.000000in}}{%
\pgfpathmoveto{\pgfqpoint{0.000000in}{0.000000in}}%
\pgfpathlineto{\pgfqpoint{0.027778in}{0.000000in}}%
\pgfusepath{stroke,fill}%
}%
\begin{pgfscope}%
\pgfsys@transformshift{4.672917in}{1.070227in}%
\pgfsys@useobject{currentmarker}{}%
\end{pgfscope}%
\end{pgfscope}%
\begin{pgfscope}%
\pgfpathrectangle{\pgfqpoint{0.781944in}{0.552778in}}{\pgfqpoint{3.890972in}{3.248611in}}%
\pgfusepath{clip}%
\pgfsetrectcap%
\pgfsetroundjoin%
\pgfsetlinewidth{0.803000pt}%
\definecolor{currentstroke}{rgb}{0.690196,0.690196,0.690196}%
\pgfsetstrokecolor{currentstroke}%
\pgfsetstrokeopacity{0.300000}%
\pgfsetdash{}{0pt}%
\pgfpathmoveto{\pgfqpoint{0.781944in}{1.113348in}}%
\pgfpathlineto{\pgfqpoint{4.672917in}{1.113348in}}%
\pgfusepath{stroke}%
\end{pgfscope}%
\begin{pgfscope}%
\pgfsetbuttcap%
\pgfsetroundjoin%
\definecolor{currentfill}{rgb}{0.000000,0.000000,0.000000}%
\pgfsetfillcolor{currentfill}%
\pgfsetlinewidth{0.602250pt}%
\definecolor{currentstroke}{rgb}{0.000000,0.000000,0.000000}%
\pgfsetstrokecolor{currentstroke}%
\pgfsetdash{}{0pt}%
\pgfsys@defobject{currentmarker}{\pgfqpoint{-0.027778in}{0.000000in}}{\pgfqpoint{0.000000in}{0.000000in}}{%
\pgfpathmoveto{\pgfqpoint{0.000000in}{0.000000in}}%
\pgfpathlineto{\pgfqpoint{-0.027778in}{0.000000in}}%
\pgfusepath{stroke,fill}%
}%
\begin{pgfscope}%
\pgfsys@transformshift{0.781944in}{1.113348in}%
\pgfsys@useobject{currentmarker}{}%
\end{pgfscope}%
\end{pgfscope}%
\begin{pgfscope}%
\pgfsetbuttcap%
\pgfsetroundjoin%
\definecolor{currentfill}{rgb}{0.000000,0.000000,0.000000}%
\pgfsetfillcolor{currentfill}%
\pgfsetlinewidth{0.602250pt}%
\definecolor{currentstroke}{rgb}{0.000000,0.000000,0.000000}%
\pgfsetstrokecolor{currentstroke}%
\pgfsetdash{}{0pt}%
\pgfsys@defobject{currentmarker}{\pgfqpoint{0.000000in}{0.000000in}}{\pgfqpoint{0.027778in}{0.000000in}}{%
\pgfpathmoveto{\pgfqpoint{0.000000in}{0.000000in}}%
\pgfpathlineto{\pgfqpoint{0.027778in}{0.000000in}}%
\pgfusepath{stroke,fill}%
}%
\begin{pgfscope}%
\pgfsys@transformshift{4.672917in}{1.113348in}%
\pgfsys@useobject{currentmarker}{}%
\end{pgfscope}%
\end{pgfscope}%
\begin{pgfscope}%
\pgfpathrectangle{\pgfqpoint{0.781944in}{0.552778in}}{\pgfqpoint{3.890972in}{3.248611in}}%
\pgfusepath{clip}%
\pgfsetrectcap%
\pgfsetroundjoin%
\pgfsetlinewidth{0.803000pt}%
\definecolor{currentstroke}{rgb}{0.690196,0.690196,0.690196}%
\pgfsetstrokecolor{currentstroke}%
\pgfsetstrokeopacity{0.300000}%
\pgfsetdash{}{0pt}%
\pgfpathmoveto{\pgfqpoint{0.781944in}{1.156469in}}%
\pgfpathlineto{\pgfqpoint{4.672917in}{1.156469in}}%
\pgfusepath{stroke}%
\end{pgfscope}%
\begin{pgfscope}%
\pgfsetbuttcap%
\pgfsetroundjoin%
\definecolor{currentfill}{rgb}{0.000000,0.000000,0.000000}%
\pgfsetfillcolor{currentfill}%
\pgfsetlinewidth{0.602250pt}%
\definecolor{currentstroke}{rgb}{0.000000,0.000000,0.000000}%
\pgfsetstrokecolor{currentstroke}%
\pgfsetdash{}{0pt}%
\pgfsys@defobject{currentmarker}{\pgfqpoint{-0.027778in}{0.000000in}}{\pgfqpoint{0.000000in}{0.000000in}}{%
\pgfpathmoveto{\pgfqpoint{0.000000in}{0.000000in}}%
\pgfpathlineto{\pgfqpoint{-0.027778in}{0.000000in}}%
\pgfusepath{stroke,fill}%
}%
\begin{pgfscope}%
\pgfsys@transformshift{0.781944in}{1.156469in}%
\pgfsys@useobject{currentmarker}{}%
\end{pgfscope}%
\end{pgfscope}%
\begin{pgfscope}%
\pgfsetbuttcap%
\pgfsetroundjoin%
\definecolor{currentfill}{rgb}{0.000000,0.000000,0.000000}%
\pgfsetfillcolor{currentfill}%
\pgfsetlinewidth{0.602250pt}%
\definecolor{currentstroke}{rgb}{0.000000,0.000000,0.000000}%
\pgfsetstrokecolor{currentstroke}%
\pgfsetdash{}{0pt}%
\pgfsys@defobject{currentmarker}{\pgfqpoint{0.000000in}{0.000000in}}{\pgfqpoint{0.027778in}{0.000000in}}{%
\pgfpathmoveto{\pgfqpoint{0.000000in}{0.000000in}}%
\pgfpathlineto{\pgfqpoint{0.027778in}{0.000000in}}%
\pgfusepath{stroke,fill}%
}%
\begin{pgfscope}%
\pgfsys@transformshift{4.672917in}{1.156469in}%
\pgfsys@useobject{currentmarker}{}%
\end{pgfscope}%
\end{pgfscope}%
\begin{pgfscope}%
\pgfpathrectangle{\pgfqpoint{0.781944in}{0.552778in}}{\pgfqpoint{3.890972in}{3.248611in}}%
\pgfusepath{clip}%
\pgfsetrectcap%
\pgfsetroundjoin%
\pgfsetlinewidth{0.803000pt}%
\definecolor{currentstroke}{rgb}{0.690196,0.690196,0.690196}%
\pgfsetstrokecolor{currentstroke}%
\pgfsetstrokeopacity{0.300000}%
\pgfsetdash{}{0pt}%
\pgfpathmoveto{\pgfqpoint{0.781944in}{1.199589in}}%
\pgfpathlineto{\pgfqpoint{4.672917in}{1.199589in}}%
\pgfusepath{stroke}%
\end{pgfscope}%
\begin{pgfscope}%
\pgfsetbuttcap%
\pgfsetroundjoin%
\definecolor{currentfill}{rgb}{0.000000,0.000000,0.000000}%
\pgfsetfillcolor{currentfill}%
\pgfsetlinewidth{0.602250pt}%
\definecolor{currentstroke}{rgb}{0.000000,0.000000,0.000000}%
\pgfsetstrokecolor{currentstroke}%
\pgfsetdash{}{0pt}%
\pgfsys@defobject{currentmarker}{\pgfqpoint{-0.027778in}{0.000000in}}{\pgfqpoint{0.000000in}{0.000000in}}{%
\pgfpathmoveto{\pgfqpoint{0.000000in}{0.000000in}}%
\pgfpathlineto{\pgfqpoint{-0.027778in}{0.000000in}}%
\pgfusepath{stroke,fill}%
}%
\begin{pgfscope}%
\pgfsys@transformshift{0.781944in}{1.199589in}%
\pgfsys@useobject{currentmarker}{}%
\end{pgfscope}%
\end{pgfscope}%
\begin{pgfscope}%
\pgfsetbuttcap%
\pgfsetroundjoin%
\definecolor{currentfill}{rgb}{0.000000,0.000000,0.000000}%
\pgfsetfillcolor{currentfill}%
\pgfsetlinewidth{0.602250pt}%
\definecolor{currentstroke}{rgb}{0.000000,0.000000,0.000000}%
\pgfsetstrokecolor{currentstroke}%
\pgfsetdash{}{0pt}%
\pgfsys@defobject{currentmarker}{\pgfqpoint{0.000000in}{0.000000in}}{\pgfqpoint{0.027778in}{0.000000in}}{%
\pgfpathmoveto{\pgfqpoint{0.000000in}{0.000000in}}%
\pgfpathlineto{\pgfqpoint{0.027778in}{0.000000in}}%
\pgfusepath{stroke,fill}%
}%
\begin{pgfscope}%
\pgfsys@transformshift{4.672917in}{1.199589in}%
\pgfsys@useobject{currentmarker}{}%
\end{pgfscope}%
\end{pgfscope}%
\begin{pgfscope}%
\pgfpathrectangle{\pgfqpoint{0.781944in}{0.552778in}}{\pgfqpoint{3.890972in}{3.248611in}}%
\pgfusepath{clip}%
\pgfsetrectcap%
\pgfsetroundjoin%
\pgfsetlinewidth{0.803000pt}%
\definecolor{currentstroke}{rgb}{0.690196,0.690196,0.690196}%
\pgfsetstrokecolor{currentstroke}%
\pgfsetstrokeopacity{0.300000}%
\pgfsetdash{}{0pt}%
\pgfpathmoveto{\pgfqpoint{0.781944in}{1.242710in}}%
\pgfpathlineto{\pgfqpoint{4.672917in}{1.242710in}}%
\pgfusepath{stroke}%
\end{pgfscope}%
\begin{pgfscope}%
\pgfsetbuttcap%
\pgfsetroundjoin%
\definecolor{currentfill}{rgb}{0.000000,0.000000,0.000000}%
\pgfsetfillcolor{currentfill}%
\pgfsetlinewidth{0.602250pt}%
\definecolor{currentstroke}{rgb}{0.000000,0.000000,0.000000}%
\pgfsetstrokecolor{currentstroke}%
\pgfsetdash{}{0pt}%
\pgfsys@defobject{currentmarker}{\pgfqpoint{-0.027778in}{0.000000in}}{\pgfqpoint{0.000000in}{0.000000in}}{%
\pgfpathmoveto{\pgfqpoint{0.000000in}{0.000000in}}%
\pgfpathlineto{\pgfqpoint{-0.027778in}{0.000000in}}%
\pgfusepath{stroke,fill}%
}%
\begin{pgfscope}%
\pgfsys@transformshift{0.781944in}{1.242710in}%
\pgfsys@useobject{currentmarker}{}%
\end{pgfscope}%
\end{pgfscope}%
\begin{pgfscope}%
\pgfsetbuttcap%
\pgfsetroundjoin%
\definecolor{currentfill}{rgb}{0.000000,0.000000,0.000000}%
\pgfsetfillcolor{currentfill}%
\pgfsetlinewidth{0.602250pt}%
\definecolor{currentstroke}{rgb}{0.000000,0.000000,0.000000}%
\pgfsetstrokecolor{currentstroke}%
\pgfsetdash{}{0pt}%
\pgfsys@defobject{currentmarker}{\pgfqpoint{0.000000in}{0.000000in}}{\pgfqpoint{0.027778in}{0.000000in}}{%
\pgfpathmoveto{\pgfqpoint{0.000000in}{0.000000in}}%
\pgfpathlineto{\pgfqpoint{0.027778in}{0.000000in}}%
\pgfusepath{stroke,fill}%
}%
\begin{pgfscope}%
\pgfsys@transformshift{4.672917in}{1.242710in}%
\pgfsys@useobject{currentmarker}{}%
\end{pgfscope}%
\end{pgfscope}%
\begin{pgfscope}%
\pgfpathrectangle{\pgfqpoint{0.781944in}{0.552778in}}{\pgfqpoint{3.890972in}{3.248611in}}%
\pgfusepath{clip}%
\pgfsetrectcap%
\pgfsetroundjoin%
\pgfsetlinewidth{0.803000pt}%
\definecolor{currentstroke}{rgb}{0.690196,0.690196,0.690196}%
\pgfsetstrokecolor{currentstroke}%
\pgfsetstrokeopacity{0.300000}%
\pgfsetdash{}{0pt}%
\pgfpathmoveto{\pgfqpoint{0.781944in}{1.285831in}}%
\pgfpathlineto{\pgfqpoint{4.672917in}{1.285831in}}%
\pgfusepath{stroke}%
\end{pgfscope}%
\begin{pgfscope}%
\pgfsetbuttcap%
\pgfsetroundjoin%
\definecolor{currentfill}{rgb}{0.000000,0.000000,0.000000}%
\pgfsetfillcolor{currentfill}%
\pgfsetlinewidth{0.602250pt}%
\definecolor{currentstroke}{rgb}{0.000000,0.000000,0.000000}%
\pgfsetstrokecolor{currentstroke}%
\pgfsetdash{}{0pt}%
\pgfsys@defobject{currentmarker}{\pgfqpoint{-0.027778in}{0.000000in}}{\pgfqpoint{0.000000in}{0.000000in}}{%
\pgfpathmoveto{\pgfqpoint{0.000000in}{0.000000in}}%
\pgfpathlineto{\pgfqpoint{-0.027778in}{0.000000in}}%
\pgfusepath{stroke,fill}%
}%
\begin{pgfscope}%
\pgfsys@transformshift{0.781944in}{1.285831in}%
\pgfsys@useobject{currentmarker}{}%
\end{pgfscope}%
\end{pgfscope}%
\begin{pgfscope}%
\pgfsetbuttcap%
\pgfsetroundjoin%
\definecolor{currentfill}{rgb}{0.000000,0.000000,0.000000}%
\pgfsetfillcolor{currentfill}%
\pgfsetlinewidth{0.602250pt}%
\definecolor{currentstroke}{rgb}{0.000000,0.000000,0.000000}%
\pgfsetstrokecolor{currentstroke}%
\pgfsetdash{}{0pt}%
\pgfsys@defobject{currentmarker}{\pgfqpoint{0.000000in}{0.000000in}}{\pgfqpoint{0.027778in}{0.000000in}}{%
\pgfpathmoveto{\pgfqpoint{0.000000in}{0.000000in}}%
\pgfpathlineto{\pgfqpoint{0.027778in}{0.000000in}}%
\pgfusepath{stroke,fill}%
}%
\begin{pgfscope}%
\pgfsys@transformshift{4.672917in}{1.285831in}%
\pgfsys@useobject{currentmarker}{}%
\end{pgfscope}%
\end{pgfscope}%
\begin{pgfscope}%
\pgfpathrectangle{\pgfqpoint{0.781944in}{0.552778in}}{\pgfqpoint{3.890972in}{3.248611in}}%
\pgfusepath{clip}%
\pgfsetrectcap%
\pgfsetroundjoin%
\pgfsetlinewidth{0.803000pt}%
\definecolor{currentstroke}{rgb}{0.690196,0.690196,0.690196}%
\pgfsetstrokecolor{currentstroke}%
\pgfsetstrokeopacity{0.300000}%
\pgfsetdash{}{0pt}%
\pgfpathmoveto{\pgfqpoint{0.781944in}{1.328952in}}%
\pgfpathlineto{\pgfqpoint{4.672917in}{1.328952in}}%
\pgfusepath{stroke}%
\end{pgfscope}%
\begin{pgfscope}%
\pgfsetbuttcap%
\pgfsetroundjoin%
\definecolor{currentfill}{rgb}{0.000000,0.000000,0.000000}%
\pgfsetfillcolor{currentfill}%
\pgfsetlinewidth{0.602250pt}%
\definecolor{currentstroke}{rgb}{0.000000,0.000000,0.000000}%
\pgfsetstrokecolor{currentstroke}%
\pgfsetdash{}{0pt}%
\pgfsys@defobject{currentmarker}{\pgfqpoint{-0.027778in}{0.000000in}}{\pgfqpoint{0.000000in}{0.000000in}}{%
\pgfpathmoveto{\pgfqpoint{0.000000in}{0.000000in}}%
\pgfpathlineto{\pgfqpoint{-0.027778in}{0.000000in}}%
\pgfusepath{stroke,fill}%
}%
\begin{pgfscope}%
\pgfsys@transformshift{0.781944in}{1.328952in}%
\pgfsys@useobject{currentmarker}{}%
\end{pgfscope}%
\end{pgfscope}%
\begin{pgfscope}%
\pgfsetbuttcap%
\pgfsetroundjoin%
\definecolor{currentfill}{rgb}{0.000000,0.000000,0.000000}%
\pgfsetfillcolor{currentfill}%
\pgfsetlinewidth{0.602250pt}%
\definecolor{currentstroke}{rgb}{0.000000,0.000000,0.000000}%
\pgfsetstrokecolor{currentstroke}%
\pgfsetdash{}{0pt}%
\pgfsys@defobject{currentmarker}{\pgfqpoint{0.000000in}{0.000000in}}{\pgfqpoint{0.027778in}{0.000000in}}{%
\pgfpathmoveto{\pgfqpoint{0.000000in}{0.000000in}}%
\pgfpathlineto{\pgfqpoint{0.027778in}{0.000000in}}%
\pgfusepath{stroke,fill}%
}%
\begin{pgfscope}%
\pgfsys@transformshift{4.672917in}{1.328952in}%
\pgfsys@useobject{currentmarker}{}%
\end{pgfscope}%
\end{pgfscope}%
\begin{pgfscope}%
\pgfpathrectangle{\pgfqpoint{0.781944in}{0.552778in}}{\pgfqpoint{3.890972in}{3.248611in}}%
\pgfusepath{clip}%
\pgfsetrectcap%
\pgfsetroundjoin%
\pgfsetlinewidth{0.803000pt}%
\definecolor{currentstroke}{rgb}{0.690196,0.690196,0.690196}%
\pgfsetstrokecolor{currentstroke}%
\pgfsetstrokeopacity{0.300000}%
\pgfsetdash{}{0pt}%
\pgfpathmoveto{\pgfqpoint{0.781944in}{1.372072in}}%
\pgfpathlineto{\pgfqpoint{4.672917in}{1.372072in}}%
\pgfusepath{stroke}%
\end{pgfscope}%
\begin{pgfscope}%
\pgfsetbuttcap%
\pgfsetroundjoin%
\definecolor{currentfill}{rgb}{0.000000,0.000000,0.000000}%
\pgfsetfillcolor{currentfill}%
\pgfsetlinewidth{0.602250pt}%
\definecolor{currentstroke}{rgb}{0.000000,0.000000,0.000000}%
\pgfsetstrokecolor{currentstroke}%
\pgfsetdash{}{0pt}%
\pgfsys@defobject{currentmarker}{\pgfqpoint{-0.027778in}{0.000000in}}{\pgfqpoint{0.000000in}{0.000000in}}{%
\pgfpathmoveto{\pgfqpoint{0.000000in}{0.000000in}}%
\pgfpathlineto{\pgfqpoint{-0.027778in}{0.000000in}}%
\pgfusepath{stroke,fill}%
}%
\begin{pgfscope}%
\pgfsys@transformshift{0.781944in}{1.372072in}%
\pgfsys@useobject{currentmarker}{}%
\end{pgfscope}%
\end{pgfscope}%
\begin{pgfscope}%
\pgfsetbuttcap%
\pgfsetroundjoin%
\definecolor{currentfill}{rgb}{0.000000,0.000000,0.000000}%
\pgfsetfillcolor{currentfill}%
\pgfsetlinewidth{0.602250pt}%
\definecolor{currentstroke}{rgb}{0.000000,0.000000,0.000000}%
\pgfsetstrokecolor{currentstroke}%
\pgfsetdash{}{0pt}%
\pgfsys@defobject{currentmarker}{\pgfqpoint{0.000000in}{0.000000in}}{\pgfqpoint{0.027778in}{0.000000in}}{%
\pgfpathmoveto{\pgfqpoint{0.000000in}{0.000000in}}%
\pgfpathlineto{\pgfqpoint{0.027778in}{0.000000in}}%
\pgfusepath{stroke,fill}%
}%
\begin{pgfscope}%
\pgfsys@transformshift{4.672917in}{1.372072in}%
\pgfsys@useobject{currentmarker}{}%
\end{pgfscope}%
\end{pgfscope}%
\begin{pgfscope}%
\pgfpathrectangle{\pgfqpoint{0.781944in}{0.552778in}}{\pgfqpoint{3.890972in}{3.248611in}}%
\pgfusepath{clip}%
\pgfsetrectcap%
\pgfsetroundjoin%
\pgfsetlinewidth{0.803000pt}%
\definecolor{currentstroke}{rgb}{0.690196,0.690196,0.690196}%
\pgfsetstrokecolor{currentstroke}%
\pgfsetstrokeopacity{0.300000}%
\pgfsetdash{}{0pt}%
\pgfpathmoveto{\pgfqpoint{0.781944in}{1.458314in}}%
\pgfpathlineto{\pgfqpoint{4.672917in}{1.458314in}}%
\pgfusepath{stroke}%
\end{pgfscope}%
\begin{pgfscope}%
\pgfsetbuttcap%
\pgfsetroundjoin%
\definecolor{currentfill}{rgb}{0.000000,0.000000,0.000000}%
\pgfsetfillcolor{currentfill}%
\pgfsetlinewidth{0.602250pt}%
\definecolor{currentstroke}{rgb}{0.000000,0.000000,0.000000}%
\pgfsetstrokecolor{currentstroke}%
\pgfsetdash{}{0pt}%
\pgfsys@defobject{currentmarker}{\pgfqpoint{-0.027778in}{0.000000in}}{\pgfqpoint{0.000000in}{0.000000in}}{%
\pgfpathmoveto{\pgfqpoint{0.000000in}{0.000000in}}%
\pgfpathlineto{\pgfqpoint{-0.027778in}{0.000000in}}%
\pgfusepath{stroke,fill}%
}%
\begin{pgfscope}%
\pgfsys@transformshift{0.781944in}{1.458314in}%
\pgfsys@useobject{currentmarker}{}%
\end{pgfscope}%
\end{pgfscope}%
\begin{pgfscope}%
\pgfsetbuttcap%
\pgfsetroundjoin%
\definecolor{currentfill}{rgb}{0.000000,0.000000,0.000000}%
\pgfsetfillcolor{currentfill}%
\pgfsetlinewidth{0.602250pt}%
\definecolor{currentstroke}{rgb}{0.000000,0.000000,0.000000}%
\pgfsetstrokecolor{currentstroke}%
\pgfsetdash{}{0pt}%
\pgfsys@defobject{currentmarker}{\pgfqpoint{0.000000in}{0.000000in}}{\pgfqpoint{0.027778in}{0.000000in}}{%
\pgfpathmoveto{\pgfqpoint{0.000000in}{0.000000in}}%
\pgfpathlineto{\pgfqpoint{0.027778in}{0.000000in}}%
\pgfusepath{stroke,fill}%
}%
\begin{pgfscope}%
\pgfsys@transformshift{4.672917in}{1.458314in}%
\pgfsys@useobject{currentmarker}{}%
\end{pgfscope}%
\end{pgfscope}%
\begin{pgfscope}%
\pgfpathrectangle{\pgfqpoint{0.781944in}{0.552778in}}{\pgfqpoint{3.890972in}{3.248611in}}%
\pgfusepath{clip}%
\pgfsetrectcap%
\pgfsetroundjoin%
\pgfsetlinewidth{0.803000pt}%
\definecolor{currentstroke}{rgb}{0.690196,0.690196,0.690196}%
\pgfsetstrokecolor{currentstroke}%
\pgfsetstrokeopacity{0.300000}%
\pgfsetdash{}{0pt}%
\pgfpathmoveto{\pgfqpoint{0.781944in}{1.501435in}}%
\pgfpathlineto{\pgfqpoint{4.672917in}{1.501435in}}%
\pgfusepath{stroke}%
\end{pgfscope}%
\begin{pgfscope}%
\pgfsetbuttcap%
\pgfsetroundjoin%
\definecolor{currentfill}{rgb}{0.000000,0.000000,0.000000}%
\pgfsetfillcolor{currentfill}%
\pgfsetlinewidth{0.602250pt}%
\definecolor{currentstroke}{rgb}{0.000000,0.000000,0.000000}%
\pgfsetstrokecolor{currentstroke}%
\pgfsetdash{}{0pt}%
\pgfsys@defobject{currentmarker}{\pgfqpoint{-0.027778in}{0.000000in}}{\pgfqpoint{0.000000in}{0.000000in}}{%
\pgfpathmoveto{\pgfqpoint{0.000000in}{0.000000in}}%
\pgfpathlineto{\pgfqpoint{-0.027778in}{0.000000in}}%
\pgfusepath{stroke,fill}%
}%
\begin{pgfscope}%
\pgfsys@transformshift{0.781944in}{1.501435in}%
\pgfsys@useobject{currentmarker}{}%
\end{pgfscope}%
\end{pgfscope}%
\begin{pgfscope}%
\pgfsetbuttcap%
\pgfsetroundjoin%
\definecolor{currentfill}{rgb}{0.000000,0.000000,0.000000}%
\pgfsetfillcolor{currentfill}%
\pgfsetlinewidth{0.602250pt}%
\definecolor{currentstroke}{rgb}{0.000000,0.000000,0.000000}%
\pgfsetstrokecolor{currentstroke}%
\pgfsetdash{}{0pt}%
\pgfsys@defobject{currentmarker}{\pgfqpoint{0.000000in}{0.000000in}}{\pgfqpoint{0.027778in}{0.000000in}}{%
\pgfpathmoveto{\pgfqpoint{0.000000in}{0.000000in}}%
\pgfpathlineto{\pgfqpoint{0.027778in}{0.000000in}}%
\pgfusepath{stroke,fill}%
}%
\begin{pgfscope}%
\pgfsys@transformshift{4.672917in}{1.501435in}%
\pgfsys@useobject{currentmarker}{}%
\end{pgfscope}%
\end{pgfscope}%
\begin{pgfscope}%
\pgfpathrectangle{\pgfqpoint{0.781944in}{0.552778in}}{\pgfqpoint{3.890972in}{3.248611in}}%
\pgfusepath{clip}%
\pgfsetrectcap%
\pgfsetroundjoin%
\pgfsetlinewidth{0.803000pt}%
\definecolor{currentstroke}{rgb}{0.690196,0.690196,0.690196}%
\pgfsetstrokecolor{currentstroke}%
\pgfsetstrokeopacity{0.300000}%
\pgfsetdash{}{0pt}%
\pgfpathmoveto{\pgfqpoint{0.781944in}{1.544556in}}%
\pgfpathlineto{\pgfqpoint{4.672917in}{1.544556in}}%
\pgfusepath{stroke}%
\end{pgfscope}%
\begin{pgfscope}%
\pgfsetbuttcap%
\pgfsetroundjoin%
\definecolor{currentfill}{rgb}{0.000000,0.000000,0.000000}%
\pgfsetfillcolor{currentfill}%
\pgfsetlinewidth{0.602250pt}%
\definecolor{currentstroke}{rgb}{0.000000,0.000000,0.000000}%
\pgfsetstrokecolor{currentstroke}%
\pgfsetdash{}{0pt}%
\pgfsys@defobject{currentmarker}{\pgfqpoint{-0.027778in}{0.000000in}}{\pgfqpoint{0.000000in}{0.000000in}}{%
\pgfpathmoveto{\pgfqpoint{0.000000in}{0.000000in}}%
\pgfpathlineto{\pgfqpoint{-0.027778in}{0.000000in}}%
\pgfusepath{stroke,fill}%
}%
\begin{pgfscope}%
\pgfsys@transformshift{0.781944in}{1.544556in}%
\pgfsys@useobject{currentmarker}{}%
\end{pgfscope}%
\end{pgfscope}%
\begin{pgfscope}%
\pgfsetbuttcap%
\pgfsetroundjoin%
\definecolor{currentfill}{rgb}{0.000000,0.000000,0.000000}%
\pgfsetfillcolor{currentfill}%
\pgfsetlinewidth{0.602250pt}%
\definecolor{currentstroke}{rgb}{0.000000,0.000000,0.000000}%
\pgfsetstrokecolor{currentstroke}%
\pgfsetdash{}{0pt}%
\pgfsys@defobject{currentmarker}{\pgfqpoint{0.000000in}{0.000000in}}{\pgfqpoint{0.027778in}{0.000000in}}{%
\pgfpathmoveto{\pgfqpoint{0.000000in}{0.000000in}}%
\pgfpathlineto{\pgfqpoint{0.027778in}{0.000000in}}%
\pgfusepath{stroke,fill}%
}%
\begin{pgfscope}%
\pgfsys@transformshift{4.672917in}{1.544556in}%
\pgfsys@useobject{currentmarker}{}%
\end{pgfscope}%
\end{pgfscope}%
\begin{pgfscope}%
\pgfpathrectangle{\pgfqpoint{0.781944in}{0.552778in}}{\pgfqpoint{3.890972in}{3.248611in}}%
\pgfusepath{clip}%
\pgfsetrectcap%
\pgfsetroundjoin%
\pgfsetlinewidth{0.803000pt}%
\definecolor{currentstroke}{rgb}{0.690196,0.690196,0.690196}%
\pgfsetstrokecolor{currentstroke}%
\pgfsetstrokeopacity{0.300000}%
\pgfsetdash{}{0pt}%
\pgfpathmoveto{\pgfqpoint{0.781944in}{1.587676in}}%
\pgfpathlineto{\pgfqpoint{4.672917in}{1.587676in}}%
\pgfusepath{stroke}%
\end{pgfscope}%
\begin{pgfscope}%
\pgfsetbuttcap%
\pgfsetroundjoin%
\definecolor{currentfill}{rgb}{0.000000,0.000000,0.000000}%
\pgfsetfillcolor{currentfill}%
\pgfsetlinewidth{0.602250pt}%
\definecolor{currentstroke}{rgb}{0.000000,0.000000,0.000000}%
\pgfsetstrokecolor{currentstroke}%
\pgfsetdash{}{0pt}%
\pgfsys@defobject{currentmarker}{\pgfqpoint{-0.027778in}{0.000000in}}{\pgfqpoint{0.000000in}{0.000000in}}{%
\pgfpathmoveto{\pgfqpoint{0.000000in}{0.000000in}}%
\pgfpathlineto{\pgfqpoint{-0.027778in}{0.000000in}}%
\pgfusepath{stroke,fill}%
}%
\begin{pgfscope}%
\pgfsys@transformshift{0.781944in}{1.587676in}%
\pgfsys@useobject{currentmarker}{}%
\end{pgfscope}%
\end{pgfscope}%
\begin{pgfscope}%
\pgfsetbuttcap%
\pgfsetroundjoin%
\definecolor{currentfill}{rgb}{0.000000,0.000000,0.000000}%
\pgfsetfillcolor{currentfill}%
\pgfsetlinewidth{0.602250pt}%
\definecolor{currentstroke}{rgb}{0.000000,0.000000,0.000000}%
\pgfsetstrokecolor{currentstroke}%
\pgfsetdash{}{0pt}%
\pgfsys@defobject{currentmarker}{\pgfqpoint{0.000000in}{0.000000in}}{\pgfqpoint{0.027778in}{0.000000in}}{%
\pgfpathmoveto{\pgfqpoint{0.000000in}{0.000000in}}%
\pgfpathlineto{\pgfqpoint{0.027778in}{0.000000in}}%
\pgfusepath{stroke,fill}%
}%
\begin{pgfscope}%
\pgfsys@transformshift{4.672917in}{1.587676in}%
\pgfsys@useobject{currentmarker}{}%
\end{pgfscope}%
\end{pgfscope}%
\begin{pgfscope}%
\pgfpathrectangle{\pgfqpoint{0.781944in}{0.552778in}}{\pgfqpoint{3.890972in}{3.248611in}}%
\pgfusepath{clip}%
\pgfsetrectcap%
\pgfsetroundjoin%
\pgfsetlinewidth{0.803000pt}%
\definecolor{currentstroke}{rgb}{0.690196,0.690196,0.690196}%
\pgfsetstrokecolor{currentstroke}%
\pgfsetstrokeopacity{0.300000}%
\pgfsetdash{}{0pt}%
\pgfpathmoveto{\pgfqpoint{0.781944in}{1.630797in}}%
\pgfpathlineto{\pgfqpoint{4.672917in}{1.630797in}}%
\pgfusepath{stroke}%
\end{pgfscope}%
\begin{pgfscope}%
\pgfsetbuttcap%
\pgfsetroundjoin%
\definecolor{currentfill}{rgb}{0.000000,0.000000,0.000000}%
\pgfsetfillcolor{currentfill}%
\pgfsetlinewidth{0.602250pt}%
\definecolor{currentstroke}{rgb}{0.000000,0.000000,0.000000}%
\pgfsetstrokecolor{currentstroke}%
\pgfsetdash{}{0pt}%
\pgfsys@defobject{currentmarker}{\pgfqpoint{-0.027778in}{0.000000in}}{\pgfqpoint{0.000000in}{0.000000in}}{%
\pgfpathmoveto{\pgfqpoint{0.000000in}{0.000000in}}%
\pgfpathlineto{\pgfqpoint{-0.027778in}{0.000000in}}%
\pgfusepath{stroke,fill}%
}%
\begin{pgfscope}%
\pgfsys@transformshift{0.781944in}{1.630797in}%
\pgfsys@useobject{currentmarker}{}%
\end{pgfscope}%
\end{pgfscope}%
\begin{pgfscope}%
\pgfsetbuttcap%
\pgfsetroundjoin%
\definecolor{currentfill}{rgb}{0.000000,0.000000,0.000000}%
\pgfsetfillcolor{currentfill}%
\pgfsetlinewidth{0.602250pt}%
\definecolor{currentstroke}{rgb}{0.000000,0.000000,0.000000}%
\pgfsetstrokecolor{currentstroke}%
\pgfsetdash{}{0pt}%
\pgfsys@defobject{currentmarker}{\pgfqpoint{0.000000in}{0.000000in}}{\pgfqpoint{0.027778in}{0.000000in}}{%
\pgfpathmoveto{\pgfqpoint{0.000000in}{0.000000in}}%
\pgfpathlineto{\pgfqpoint{0.027778in}{0.000000in}}%
\pgfusepath{stroke,fill}%
}%
\begin{pgfscope}%
\pgfsys@transformshift{4.672917in}{1.630797in}%
\pgfsys@useobject{currentmarker}{}%
\end{pgfscope}%
\end{pgfscope}%
\begin{pgfscope}%
\pgfpathrectangle{\pgfqpoint{0.781944in}{0.552778in}}{\pgfqpoint{3.890972in}{3.248611in}}%
\pgfusepath{clip}%
\pgfsetrectcap%
\pgfsetroundjoin%
\pgfsetlinewidth{0.803000pt}%
\definecolor{currentstroke}{rgb}{0.690196,0.690196,0.690196}%
\pgfsetstrokecolor{currentstroke}%
\pgfsetstrokeopacity{0.300000}%
\pgfsetdash{}{0pt}%
\pgfpathmoveto{\pgfqpoint{0.781944in}{1.673918in}}%
\pgfpathlineto{\pgfqpoint{4.672917in}{1.673918in}}%
\pgfusepath{stroke}%
\end{pgfscope}%
\begin{pgfscope}%
\pgfsetbuttcap%
\pgfsetroundjoin%
\definecolor{currentfill}{rgb}{0.000000,0.000000,0.000000}%
\pgfsetfillcolor{currentfill}%
\pgfsetlinewidth{0.602250pt}%
\definecolor{currentstroke}{rgb}{0.000000,0.000000,0.000000}%
\pgfsetstrokecolor{currentstroke}%
\pgfsetdash{}{0pt}%
\pgfsys@defobject{currentmarker}{\pgfqpoint{-0.027778in}{0.000000in}}{\pgfqpoint{0.000000in}{0.000000in}}{%
\pgfpathmoveto{\pgfqpoint{0.000000in}{0.000000in}}%
\pgfpathlineto{\pgfqpoint{-0.027778in}{0.000000in}}%
\pgfusepath{stroke,fill}%
}%
\begin{pgfscope}%
\pgfsys@transformshift{0.781944in}{1.673918in}%
\pgfsys@useobject{currentmarker}{}%
\end{pgfscope}%
\end{pgfscope}%
\begin{pgfscope}%
\pgfsetbuttcap%
\pgfsetroundjoin%
\definecolor{currentfill}{rgb}{0.000000,0.000000,0.000000}%
\pgfsetfillcolor{currentfill}%
\pgfsetlinewidth{0.602250pt}%
\definecolor{currentstroke}{rgb}{0.000000,0.000000,0.000000}%
\pgfsetstrokecolor{currentstroke}%
\pgfsetdash{}{0pt}%
\pgfsys@defobject{currentmarker}{\pgfqpoint{0.000000in}{0.000000in}}{\pgfqpoint{0.027778in}{0.000000in}}{%
\pgfpathmoveto{\pgfqpoint{0.000000in}{0.000000in}}%
\pgfpathlineto{\pgfqpoint{0.027778in}{0.000000in}}%
\pgfusepath{stroke,fill}%
}%
\begin{pgfscope}%
\pgfsys@transformshift{4.672917in}{1.673918in}%
\pgfsys@useobject{currentmarker}{}%
\end{pgfscope}%
\end{pgfscope}%
\begin{pgfscope}%
\pgfpathrectangle{\pgfqpoint{0.781944in}{0.552778in}}{\pgfqpoint{3.890972in}{3.248611in}}%
\pgfusepath{clip}%
\pgfsetrectcap%
\pgfsetroundjoin%
\pgfsetlinewidth{0.803000pt}%
\definecolor{currentstroke}{rgb}{0.690196,0.690196,0.690196}%
\pgfsetstrokecolor{currentstroke}%
\pgfsetstrokeopacity{0.300000}%
\pgfsetdash{}{0pt}%
\pgfpathmoveto{\pgfqpoint{0.781944in}{1.717039in}}%
\pgfpathlineto{\pgfqpoint{4.672917in}{1.717039in}}%
\pgfusepath{stroke}%
\end{pgfscope}%
\begin{pgfscope}%
\pgfsetbuttcap%
\pgfsetroundjoin%
\definecolor{currentfill}{rgb}{0.000000,0.000000,0.000000}%
\pgfsetfillcolor{currentfill}%
\pgfsetlinewidth{0.602250pt}%
\definecolor{currentstroke}{rgb}{0.000000,0.000000,0.000000}%
\pgfsetstrokecolor{currentstroke}%
\pgfsetdash{}{0pt}%
\pgfsys@defobject{currentmarker}{\pgfqpoint{-0.027778in}{0.000000in}}{\pgfqpoint{0.000000in}{0.000000in}}{%
\pgfpathmoveto{\pgfqpoint{0.000000in}{0.000000in}}%
\pgfpathlineto{\pgfqpoint{-0.027778in}{0.000000in}}%
\pgfusepath{stroke,fill}%
}%
\begin{pgfscope}%
\pgfsys@transformshift{0.781944in}{1.717039in}%
\pgfsys@useobject{currentmarker}{}%
\end{pgfscope}%
\end{pgfscope}%
\begin{pgfscope}%
\pgfsetbuttcap%
\pgfsetroundjoin%
\definecolor{currentfill}{rgb}{0.000000,0.000000,0.000000}%
\pgfsetfillcolor{currentfill}%
\pgfsetlinewidth{0.602250pt}%
\definecolor{currentstroke}{rgb}{0.000000,0.000000,0.000000}%
\pgfsetstrokecolor{currentstroke}%
\pgfsetdash{}{0pt}%
\pgfsys@defobject{currentmarker}{\pgfqpoint{0.000000in}{0.000000in}}{\pgfqpoint{0.027778in}{0.000000in}}{%
\pgfpathmoveto{\pgfqpoint{0.000000in}{0.000000in}}%
\pgfpathlineto{\pgfqpoint{0.027778in}{0.000000in}}%
\pgfusepath{stroke,fill}%
}%
\begin{pgfscope}%
\pgfsys@transformshift{4.672917in}{1.717039in}%
\pgfsys@useobject{currentmarker}{}%
\end{pgfscope}%
\end{pgfscope}%
\begin{pgfscope}%
\pgfpathrectangle{\pgfqpoint{0.781944in}{0.552778in}}{\pgfqpoint{3.890972in}{3.248611in}}%
\pgfusepath{clip}%
\pgfsetrectcap%
\pgfsetroundjoin%
\pgfsetlinewidth{0.803000pt}%
\definecolor{currentstroke}{rgb}{0.690196,0.690196,0.690196}%
\pgfsetstrokecolor{currentstroke}%
\pgfsetstrokeopacity{0.300000}%
\pgfsetdash{}{0pt}%
\pgfpathmoveto{\pgfqpoint{0.781944in}{1.760159in}}%
\pgfpathlineto{\pgfqpoint{4.672917in}{1.760159in}}%
\pgfusepath{stroke}%
\end{pgfscope}%
\begin{pgfscope}%
\pgfsetbuttcap%
\pgfsetroundjoin%
\definecolor{currentfill}{rgb}{0.000000,0.000000,0.000000}%
\pgfsetfillcolor{currentfill}%
\pgfsetlinewidth{0.602250pt}%
\definecolor{currentstroke}{rgb}{0.000000,0.000000,0.000000}%
\pgfsetstrokecolor{currentstroke}%
\pgfsetdash{}{0pt}%
\pgfsys@defobject{currentmarker}{\pgfqpoint{-0.027778in}{0.000000in}}{\pgfqpoint{0.000000in}{0.000000in}}{%
\pgfpathmoveto{\pgfqpoint{0.000000in}{0.000000in}}%
\pgfpathlineto{\pgfqpoint{-0.027778in}{0.000000in}}%
\pgfusepath{stroke,fill}%
}%
\begin{pgfscope}%
\pgfsys@transformshift{0.781944in}{1.760159in}%
\pgfsys@useobject{currentmarker}{}%
\end{pgfscope}%
\end{pgfscope}%
\begin{pgfscope}%
\pgfsetbuttcap%
\pgfsetroundjoin%
\definecolor{currentfill}{rgb}{0.000000,0.000000,0.000000}%
\pgfsetfillcolor{currentfill}%
\pgfsetlinewidth{0.602250pt}%
\definecolor{currentstroke}{rgb}{0.000000,0.000000,0.000000}%
\pgfsetstrokecolor{currentstroke}%
\pgfsetdash{}{0pt}%
\pgfsys@defobject{currentmarker}{\pgfqpoint{0.000000in}{0.000000in}}{\pgfqpoint{0.027778in}{0.000000in}}{%
\pgfpathmoveto{\pgfqpoint{0.000000in}{0.000000in}}%
\pgfpathlineto{\pgfqpoint{0.027778in}{0.000000in}}%
\pgfusepath{stroke,fill}%
}%
\begin{pgfscope}%
\pgfsys@transformshift{4.672917in}{1.760159in}%
\pgfsys@useobject{currentmarker}{}%
\end{pgfscope}%
\end{pgfscope}%
\begin{pgfscope}%
\pgfpathrectangle{\pgfqpoint{0.781944in}{0.552778in}}{\pgfqpoint{3.890972in}{3.248611in}}%
\pgfusepath{clip}%
\pgfsetrectcap%
\pgfsetroundjoin%
\pgfsetlinewidth{0.803000pt}%
\definecolor{currentstroke}{rgb}{0.690196,0.690196,0.690196}%
\pgfsetstrokecolor{currentstroke}%
\pgfsetstrokeopacity{0.300000}%
\pgfsetdash{}{0pt}%
\pgfpathmoveto{\pgfqpoint{0.781944in}{1.803280in}}%
\pgfpathlineto{\pgfqpoint{4.672917in}{1.803280in}}%
\pgfusepath{stroke}%
\end{pgfscope}%
\begin{pgfscope}%
\pgfsetbuttcap%
\pgfsetroundjoin%
\definecolor{currentfill}{rgb}{0.000000,0.000000,0.000000}%
\pgfsetfillcolor{currentfill}%
\pgfsetlinewidth{0.602250pt}%
\definecolor{currentstroke}{rgb}{0.000000,0.000000,0.000000}%
\pgfsetstrokecolor{currentstroke}%
\pgfsetdash{}{0pt}%
\pgfsys@defobject{currentmarker}{\pgfqpoint{-0.027778in}{0.000000in}}{\pgfqpoint{0.000000in}{0.000000in}}{%
\pgfpathmoveto{\pgfqpoint{0.000000in}{0.000000in}}%
\pgfpathlineto{\pgfqpoint{-0.027778in}{0.000000in}}%
\pgfusepath{stroke,fill}%
}%
\begin{pgfscope}%
\pgfsys@transformshift{0.781944in}{1.803280in}%
\pgfsys@useobject{currentmarker}{}%
\end{pgfscope}%
\end{pgfscope}%
\begin{pgfscope}%
\pgfsetbuttcap%
\pgfsetroundjoin%
\definecolor{currentfill}{rgb}{0.000000,0.000000,0.000000}%
\pgfsetfillcolor{currentfill}%
\pgfsetlinewidth{0.602250pt}%
\definecolor{currentstroke}{rgb}{0.000000,0.000000,0.000000}%
\pgfsetstrokecolor{currentstroke}%
\pgfsetdash{}{0pt}%
\pgfsys@defobject{currentmarker}{\pgfqpoint{0.000000in}{0.000000in}}{\pgfqpoint{0.027778in}{0.000000in}}{%
\pgfpathmoveto{\pgfqpoint{0.000000in}{0.000000in}}%
\pgfpathlineto{\pgfqpoint{0.027778in}{0.000000in}}%
\pgfusepath{stroke,fill}%
}%
\begin{pgfscope}%
\pgfsys@transformshift{4.672917in}{1.803280in}%
\pgfsys@useobject{currentmarker}{}%
\end{pgfscope}%
\end{pgfscope}%
\begin{pgfscope}%
\pgfpathrectangle{\pgfqpoint{0.781944in}{0.552778in}}{\pgfqpoint{3.890972in}{3.248611in}}%
\pgfusepath{clip}%
\pgfsetrectcap%
\pgfsetroundjoin%
\pgfsetlinewidth{0.803000pt}%
\definecolor{currentstroke}{rgb}{0.690196,0.690196,0.690196}%
\pgfsetstrokecolor{currentstroke}%
\pgfsetstrokeopacity{0.300000}%
\pgfsetdash{}{0pt}%
\pgfpathmoveto{\pgfqpoint{0.781944in}{1.889522in}}%
\pgfpathlineto{\pgfqpoint{4.672917in}{1.889522in}}%
\pgfusepath{stroke}%
\end{pgfscope}%
\begin{pgfscope}%
\pgfsetbuttcap%
\pgfsetroundjoin%
\definecolor{currentfill}{rgb}{0.000000,0.000000,0.000000}%
\pgfsetfillcolor{currentfill}%
\pgfsetlinewidth{0.602250pt}%
\definecolor{currentstroke}{rgb}{0.000000,0.000000,0.000000}%
\pgfsetstrokecolor{currentstroke}%
\pgfsetdash{}{0pt}%
\pgfsys@defobject{currentmarker}{\pgfqpoint{-0.027778in}{0.000000in}}{\pgfqpoint{0.000000in}{0.000000in}}{%
\pgfpathmoveto{\pgfqpoint{0.000000in}{0.000000in}}%
\pgfpathlineto{\pgfqpoint{-0.027778in}{0.000000in}}%
\pgfusepath{stroke,fill}%
}%
\begin{pgfscope}%
\pgfsys@transformshift{0.781944in}{1.889522in}%
\pgfsys@useobject{currentmarker}{}%
\end{pgfscope}%
\end{pgfscope}%
\begin{pgfscope}%
\pgfsetbuttcap%
\pgfsetroundjoin%
\definecolor{currentfill}{rgb}{0.000000,0.000000,0.000000}%
\pgfsetfillcolor{currentfill}%
\pgfsetlinewidth{0.602250pt}%
\definecolor{currentstroke}{rgb}{0.000000,0.000000,0.000000}%
\pgfsetstrokecolor{currentstroke}%
\pgfsetdash{}{0pt}%
\pgfsys@defobject{currentmarker}{\pgfqpoint{0.000000in}{0.000000in}}{\pgfqpoint{0.027778in}{0.000000in}}{%
\pgfpathmoveto{\pgfqpoint{0.000000in}{0.000000in}}%
\pgfpathlineto{\pgfqpoint{0.027778in}{0.000000in}}%
\pgfusepath{stroke,fill}%
}%
\begin{pgfscope}%
\pgfsys@transformshift{4.672917in}{1.889522in}%
\pgfsys@useobject{currentmarker}{}%
\end{pgfscope}%
\end{pgfscope}%
\begin{pgfscope}%
\pgfpathrectangle{\pgfqpoint{0.781944in}{0.552778in}}{\pgfqpoint{3.890972in}{3.248611in}}%
\pgfusepath{clip}%
\pgfsetrectcap%
\pgfsetroundjoin%
\pgfsetlinewidth{0.803000pt}%
\definecolor{currentstroke}{rgb}{0.690196,0.690196,0.690196}%
\pgfsetstrokecolor{currentstroke}%
\pgfsetstrokeopacity{0.300000}%
\pgfsetdash{}{0pt}%
\pgfpathmoveto{\pgfqpoint{0.781944in}{1.932642in}}%
\pgfpathlineto{\pgfqpoint{4.672917in}{1.932642in}}%
\pgfusepath{stroke}%
\end{pgfscope}%
\begin{pgfscope}%
\pgfsetbuttcap%
\pgfsetroundjoin%
\definecolor{currentfill}{rgb}{0.000000,0.000000,0.000000}%
\pgfsetfillcolor{currentfill}%
\pgfsetlinewidth{0.602250pt}%
\definecolor{currentstroke}{rgb}{0.000000,0.000000,0.000000}%
\pgfsetstrokecolor{currentstroke}%
\pgfsetdash{}{0pt}%
\pgfsys@defobject{currentmarker}{\pgfqpoint{-0.027778in}{0.000000in}}{\pgfqpoint{0.000000in}{0.000000in}}{%
\pgfpathmoveto{\pgfqpoint{0.000000in}{0.000000in}}%
\pgfpathlineto{\pgfqpoint{-0.027778in}{0.000000in}}%
\pgfusepath{stroke,fill}%
}%
\begin{pgfscope}%
\pgfsys@transformshift{0.781944in}{1.932642in}%
\pgfsys@useobject{currentmarker}{}%
\end{pgfscope}%
\end{pgfscope}%
\begin{pgfscope}%
\pgfsetbuttcap%
\pgfsetroundjoin%
\definecolor{currentfill}{rgb}{0.000000,0.000000,0.000000}%
\pgfsetfillcolor{currentfill}%
\pgfsetlinewidth{0.602250pt}%
\definecolor{currentstroke}{rgb}{0.000000,0.000000,0.000000}%
\pgfsetstrokecolor{currentstroke}%
\pgfsetdash{}{0pt}%
\pgfsys@defobject{currentmarker}{\pgfqpoint{0.000000in}{0.000000in}}{\pgfqpoint{0.027778in}{0.000000in}}{%
\pgfpathmoveto{\pgfqpoint{0.000000in}{0.000000in}}%
\pgfpathlineto{\pgfqpoint{0.027778in}{0.000000in}}%
\pgfusepath{stroke,fill}%
}%
\begin{pgfscope}%
\pgfsys@transformshift{4.672917in}{1.932642in}%
\pgfsys@useobject{currentmarker}{}%
\end{pgfscope}%
\end{pgfscope}%
\begin{pgfscope}%
\pgfpathrectangle{\pgfqpoint{0.781944in}{0.552778in}}{\pgfqpoint{3.890972in}{3.248611in}}%
\pgfusepath{clip}%
\pgfsetrectcap%
\pgfsetroundjoin%
\pgfsetlinewidth{0.803000pt}%
\definecolor{currentstroke}{rgb}{0.690196,0.690196,0.690196}%
\pgfsetstrokecolor{currentstroke}%
\pgfsetstrokeopacity{0.300000}%
\pgfsetdash{}{0pt}%
\pgfpathmoveto{\pgfqpoint{0.781944in}{1.975763in}}%
\pgfpathlineto{\pgfqpoint{4.672917in}{1.975763in}}%
\pgfusepath{stroke}%
\end{pgfscope}%
\begin{pgfscope}%
\pgfsetbuttcap%
\pgfsetroundjoin%
\definecolor{currentfill}{rgb}{0.000000,0.000000,0.000000}%
\pgfsetfillcolor{currentfill}%
\pgfsetlinewidth{0.602250pt}%
\definecolor{currentstroke}{rgb}{0.000000,0.000000,0.000000}%
\pgfsetstrokecolor{currentstroke}%
\pgfsetdash{}{0pt}%
\pgfsys@defobject{currentmarker}{\pgfqpoint{-0.027778in}{0.000000in}}{\pgfqpoint{0.000000in}{0.000000in}}{%
\pgfpathmoveto{\pgfqpoint{0.000000in}{0.000000in}}%
\pgfpathlineto{\pgfqpoint{-0.027778in}{0.000000in}}%
\pgfusepath{stroke,fill}%
}%
\begin{pgfscope}%
\pgfsys@transformshift{0.781944in}{1.975763in}%
\pgfsys@useobject{currentmarker}{}%
\end{pgfscope}%
\end{pgfscope}%
\begin{pgfscope}%
\pgfsetbuttcap%
\pgfsetroundjoin%
\definecolor{currentfill}{rgb}{0.000000,0.000000,0.000000}%
\pgfsetfillcolor{currentfill}%
\pgfsetlinewidth{0.602250pt}%
\definecolor{currentstroke}{rgb}{0.000000,0.000000,0.000000}%
\pgfsetstrokecolor{currentstroke}%
\pgfsetdash{}{0pt}%
\pgfsys@defobject{currentmarker}{\pgfqpoint{0.000000in}{0.000000in}}{\pgfqpoint{0.027778in}{0.000000in}}{%
\pgfpathmoveto{\pgfqpoint{0.000000in}{0.000000in}}%
\pgfpathlineto{\pgfqpoint{0.027778in}{0.000000in}}%
\pgfusepath{stroke,fill}%
}%
\begin{pgfscope}%
\pgfsys@transformshift{4.672917in}{1.975763in}%
\pgfsys@useobject{currentmarker}{}%
\end{pgfscope}%
\end{pgfscope}%
\begin{pgfscope}%
\pgfpathrectangle{\pgfqpoint{0.781944in}{0.552778in}}{\pgfqpoint{3.890972in}{3.248611in}}%
\pgfusepath{clip}%
\pgfsetrectcap%
\pgfsetroundjoin%
\pgfsetlinewidth{0.803000pt}%
\definecolor{currentstroke}{rgb}{0.690196,0.690196,0.690196}%
\pgfsetstrokecolor{currentstroke}%
\pgfsetstrokeopacity{0.300000}%
\pgfsetdash{}{0pt}%
\pgfpathmoveto{\pgfqpoint{0.781944in}{2.018884in}}%
\pgfpathlineto{\pgfqpoint{4.672917in}{2.018884in}}%
\pgfusepath{stroke}%
\end{pgfscope}%
\begin{pgfscope}%
\pgfsetbuttcap%
\pgfsetroundjoin%
\definecolor{currentfill}{rgb}{0.000000,0.000000,0.000000}%
\pgfsetfillcolor{currentfill}%
\pgfsetlinewidth{0.602250pt}%
\definecolor{currentstroke}{rgb}{0.000000,0.000000,0.000000}%
\pgfsetstrokecolor{currentstroke}%
\pgfsetdash{}{0pt}%
\pgfsys@defobject{currentmarker}{\pgfqpoint{-0.027778in}{0.000000in}}{\pgfqpoint{0.000000in}{0.000000in}}{%
\pgfpathmoveto{\pgfqpoint{0.000000in}{0.000000in}}%
\pgfpathlineto{\pgfqpoint{-0.027778in}{0.000000in}}%
\pgfusepath{stroke,fill}%
}%
\begin{pgfscope}%
\pgfsys@transformshift{0.781944in}{2.018884in}%
\pgfsys@useobject{currentmarker}{}%
\end{pgfscope}%
\end{pgfscope}%
\begin{pgfscope}%
\pgfsetbuttcap%
\pgfsetroundjoin%
\definecolor{currentfill}{rgb}{0.000000,0.000000,0.000000}%
\pgfsetfillcolor{currentfill}%
\pgfsetlinewidth{0.602250pt}%
\definecolor{currentstroke}{rgb}{0.000000,0.000000,0.000000}%
\pgfsetstrokecolor{currentstroke}%
\pgfsetdash{}{0pt}%
\pgfsys@defobject{currentmarker}{\pgfqpoint{0.000000in}{0.000000in}}{\pgfqpoint{0.027778in}{0.000000in}}{%
\pgfpathmoveto{\pgfqpoint{0.000000in}{0.000000in}}%
\pgfpathlineto{\pgfqpoint{0.027778in}{0.000000in}}%
\pgfusepath{stroke,fill}%
}%
\begin{pgfscope}%
\pgfsys@transformshift{4.672917in}{2.018884in}%
\pgfsys@useobject{currentmarker}{}%
\end{pgfscope}%
\end{pgfscope}%
\begin{pgfscope}%
\pgfpathrectangle{\pgfqpoint{0.781944in}{0.552778in}}{\pgfqpoint{3.890972in}{3.248611in}}%
\pgfusepath{clip}%
\pgfsetrectcap%
\pgfsetroundjoin%
\pgfsetlinewidth{0.803000pt}%
\definecolor{currentstroke}{rgb}{0.690196,0.690196,0.690196}%
\pgfsetstrokecolor{currentstroke}%
\pgfsetstrokeopacity{0.300000}%
\pgfsetdash{}{0pt}%
\pgfpathmoveto{\pgfqpoint{0.781944in}{2.062005in}}%
\pgfpathlineto{\pgfqpoint{4.672917in}{2.062005in}}%
\pgfusepath{stroke}%
\end{pgfscope}%
\begin{pgfscope}%
\pgfsetbuttcap%
\pgfsetroundjoin%
\definecolor{currentfill}{rgb}{0.000000,0.000000,0.000000}%
\pgfsetfillcolor{currentfill}%
\pgfsetlinewidth{0.602250pt}%
\definecolor{currentstroke}{rgb}{0.000000,0.000000,0.000000}%
\pgfsetstrokecolor{currentstroke}%
\pgfsetdash{}{0pt}%
\pgfsys@defobject{currentmarker}{\pgfqpoint{-0.027778in}{0.000000in}}{\pgfqpoint{0.000000in}{0.000000in}}{%
\pgfpathmoveto{\pgfqpoint{0.000000in}{0.000000in}}%
\pgfpathlineto{\pgfqpoint{-0.027778in}{0.000000in}}%
\pgfusepath{stroke,fill}%
}%
\begin{pgfscope}%
\pgfsys@transformshift{0.781944in}{2.062005in}%
\pgfsys@useobject{currentmarker}{}%
\end{pgfscope}%
\end{pgfscope}%
\begin{pgfscope}%
\pgfsetbuttcap%
\pgfsetroundjoin%
\definecolor{currentfill}{rgb}{0.000000,0.000000,0.000000}%
\pgfsetfillcolor{currentfill}%
\pgfsetlinewidth{0.602250pt}%
\definecolor{currentstroke}{rgb}{0.000000,0.000000,0.000000}%
\pgfsetstrokecolor{currentstroke}%
\pgfsetdash{}{0pt}%
\pgfsys@defobject{currentmarker}{\pgfqpoint{0.000000in}{0.000000in}}{\pgfqpoint{0.027778in}{0.000000in}}{%
\pgfpathmoveto{\pgfqpoint{0.000000in}{0.000000in}}%
\pgfpathlineto{\pgfqpoint{0.027778in}{0.000000in}}%
\pgfusepath{stroke,fill}%
}%
\begin{pgfscope}%
\pgfsys@transformshift{4.672917in}{2.062005in}%
\pgfsys@useobject{currentmarker}{}%
\end{pgfscope}%
\end{pgfscope}%
\begin{pgfscope}%
\pgfpathrectangle{\pgfqpoint{0.781944in}{0.552778in}}{\pgfqpoint{3.890972in}{3.248611in}}%
\pgfusepath{clip}%
\pgfsetrectcap%
\pgfsetroundjoin%
\pgfsetlinewidth{0.803000pt}%
\definecolor{currentstroke}{rgb}{0.690196,0.690196,0.690196}%
\pgfsetstrokecolor{currentstroke}%
\pgfsetstrokeopacity{0.300000}%
\pgfsetdash{}{0pt}%
\pgfpathmoveto{\pgfqpoint{0.781944in}{2.105126in}}%
\pgfpathlineto{\pgfqpoint{4.672917in}{2.105126in}}%
\pgfusepath{stroke}%
\end{pgfscope}%
\begin{pgfscope}%
\pgfsetbuttcap%
\pgfsetroundjoin%
\definecolor{currentfill}{rgb}{0.000000,0.000000,0.000000}%
\pgfsetfillcolor{currentfill}%
\pgfsetlinewidth{0.602250pt}%
\definecolor{currentstroke}{rgb}{0.000000,0.000000,0.000000}%
\pgfsetstrokecolor{currentstroke}%
\pgfsetdash{}{0pt}%
\pgfsys@defobject{currentmarker}{\pgfqpoint{-0.027778in}{0.000000in}}{\pgfqpoint{0.000000in}{0.000000in}}{%
\pgfpathmoveto{\pgfqpoint{0.000000in}{0.000000in}}%
\pgfpathlineto{\pgfqpoint{-0.027778in}{0.000000in}}%
\pgfusepath{stroke,fill}%
}%
\begin{pgfscope}%
\pgfsys@transformshift{0.781944in}{2.105126in}%
\pgfsys@useobject{currentmarker}{}%
\end{pgfscope}%
\end{pgfscope}%
\begin{pgfscope}%
\pgfsetbuttcap%
\pgfsetroundjoin%
\definecolor{currentfill}{rgb}{0.000000,0.000000,0.000000}%
\pgfsetfillcolor{currentfill}%
\pgfsetlinewidth{0.602250pt}%
\definecolor{currentstroke}{rgb}{0.000000,0.000000,0.000000}%
\pgfsetstrokecolor{currentstroke}%
\pgfsetdash{}{0pt}%
\pgfsys@defobject{currentmarker}{\pgfqpoint{0.000000in}{0.000000in}}{\pgfqpoint{0.027778in}{0.000000in}}{%
\pgfpathmoveto{\pgfqpoint{0.000000in}{0.000000in}}%
\pgfpathlineto{\pgfqpoint{0.027778in}{0.000000in}}%
\pgfusepath{stroke,fill}%
}%
\begin{pgfscope}%
\pgfsys@transformshift{4.672917in}{2.105126in}%
\pgfsys@useobject{currentmarker}{}%
\end{pgfscope}%
\end{pgfscope}%
\begin{pgfscope}%
\pgfpathrectangle{\pgfqpoint{0.781944in}{0.552778in}}{\pgfqpoint{3.890972in}{3.248611in}}%
\pgfusepath{clip}%
\pgfsetrectcap%
\pgfsetroundjoin%
\pgfsetlinewidth{0.803000pt}%
\definecolor{currentstroke}{rgb}{0.690196,0.690196,0.690196}%
\pgfsetstrokecolor{currentstroke}%
\pgfsetstrokeopacity{0.300000}%
\pgfsetdash{}{0pt}%
\pgfpathmoveto{\pgfqpoint{0.781944in}{2.148246in}}%
\pgfpathlineto{\pgfqpoint{4.672917in}{2.148246in}}%
\pgfusepath{stroke}%
\end{pgfscope}%
\begin{pgfscope}%
\pgfsetbuttcap%
\pgfsetroundjoin%
\definecolor{currentfill}{rgb}{0.000000,0.000000,0.000000}%
\pgfsetfillcolor{currentfill}%
\pgfsetlinewidth{0.602250pt}%
\definecolor{currentstroke}{rgb}{0.000000,0.000000,0.000000}%
\pgfsetstrokecolor{currentstroke}%
\pgfsetdash{}{0pt}%
\pgfsys@defobject{currentmarker}{\pgfqpoint{-0.027778in}{0.000000in}}{\pgfqpoint{0.000000in}{0.000000in}}{%
\pgfpathmoveto{\pgfqpoint{0.000000in}{0.000000in}}%
\pgfpathlineto{\pgfqpoint{-0.027778in}{0.000000in}}%
\pgfusepath{stroke,fill}%
}%
\begin{pgfscope}%
\pgfsys@transformshift{0.781944in}{2.148246in}%
\pgfsys@useobject{currentmarker}{}%
\end{pgfscope}%
\end{pgfscope}%
\begin{pgfscope}%
\pgfsetbuttcap%
\pgfsetroundjoin%
\definecolor{currentfill}{rgb}{0.000000,0.000000,0.000000}%
\pgfsetfillcolor{currentfill}%
\pgfsetlinewidth{0.602250pt}%
\definecolor{currentstroke}{rgb}{0.000000,0.000000,0.000000}%
\pgfsetstrokecolor{currentstroke}%
\pgfsetdash{}{0pt}%
\pgfsys@defobject{currentmarker}{\pgfqpoint{0.000000in}{0.000000in}}{\pgfqpoint{0.027778in}{0.000000in}}{%
\pgfpathmoveto{\pgfqpoint{0.000000in}{0.000000in}}%
\pgfpathlineto{\pgfqpoint{0.027778in}{0.000000in}}%
\pgfusepath{stroke,fill}%
}%
\begin{pgfscope}%
\pgfsys@transformshift{4.672917in}{2.148246in}%
\pgfsys@useobject{currentmarker}{}%
\end{pgfscope}%
\end{pgfscope}%
\begin{pgfscope}%
\pgfpathrectangle{\pgfqpoint{0.781944in}{0.552778in}}{\pgfqpoint{3.890972in}{3.248611in}}%
\pgfusepath{clip}%
\pgfsetrectcap%
\pgfsetroundjoin%
\pgfsetlinewidth{0.803000pt}%
\definecolor{currentstroke}{rgb}{0.690196,0.690196,0.690196}%
\pgfsetstrokecolor{currentstroke}%
\pgfsetstrokeopacity{0.300000}%
\pgfsetdash{}{0pt}%
\pgfpathmoveto{\pgfqpoint{0.781944in}{2.191367in}}%
\pgfpathlineto{\pgfqpoint{4.672917in}{2.191367in}}%
\pgfusepath{stroke}%
\end{pgfscope}%
\begin{pgfscope}%
\pgfsetbuttcap%
\pgfsetroundjoin%
\definecolor{currentfill}{rgb}{0.000000,0.000000,0.000000}%
\pgfsetfillcolor{currentfill}%
\pgfsetlinewidth{0.602250pt}%
\definecolor{currentstroke}{rgb}{0.000000,0.000000,0.000000}%
\pgfsetstrokecolor{currentstroke}%
\pgfsetdash{}{0pt}%
\pgfsys@defobject{currentmarker}{\pgfqpoint{-0.027778in}{0.000000in}}{\pgfqpoint{0.000000in}{0.000000in}}{%
\pgfpathmoveto{\pgfqpoint{0.000000in}{0.000000in}}%
\pgfpathlineto{\pgfqpoint{-0.027778in}{0.000000in}}%
\pgfusepath{stroke,fill}%
}%
\begin{pgfscope}%
\pgfsys@transformshift{0.781944in}{2.191367in}%
\pgfsys@useobject{currentmarker}{}%
\end{pgfscope}%
\end{pgfscope}%
\begin{pgfscope}%
\pgfsetbuttcap%
\pgfsetroundjoin%
\definecolor{currentfill}{rgb}{0.000000,0.000000,0.000000}%
\pgfsetfillcolor{currentfill}%
\pgfsetlinewidth{0.602250pt}%
\definecolor{currentstroke}{rgb}{0.000000,0.000000,0.000000}%
\pgfsetstrokecolor{currentstroke}%
\pgfsetdash{}{0pt}%
\pgfsys@defobject{currentmarker}{\pgfqpoint{0.000000in}{0.000000in}}{\pgfqpoint{0.027778in}{0.000000in}}{%
\pgfpathmoveto{\pgfqpoint{0.000000in}{0.000000in}}%
\pgfpathlineto{\pgfqpoint{0.027778in}{0.000000in}}%
\pgfusepath{stroke,fill}%
}%
\begin{pgfscope}%
\pgfsys@transformshift{4.672917in}{2.191367in}%
\pgfsys@useobject{currentmarker}{}%
\end{pgfscope}%
\end{pgfscope}%
\begin{pgfscope}%
\pgfpathrectangle{\pgfqpoint{0.781944in}{0.552778in}}{\pgfqpoint{3.890972in}{3.248611in}}%
\pgfusepath{clip}%
\pgfsetrectcap%
\pgfsetroundjoin%
\pgfsetlinewidth{0.803000pt}%
\definecolor{currentstroke}{rgb}{0.690196,0.690196,0.690196}%
\pgfsetstrokecolor{currentstroke}%
\pgfsetstrokeopacity{0.300000}%
\pgfsetdash{}{0pt}%
\pgfpathmoveto{\pgfqpoint{0.781944in}{2.234488in}}%
\pgfpathlineto{\pgfqpoint{4.672917in}{2.234488in}}%
\pgfusepath{stroke}%
\end{pgfscope}%
\begin{pgfscope}%
\pgfsetbuttcap%
\pgfsetroundjoin%
\definecolor{currentfill}{rgb}{0.000000,0.000000,0.000000}%
\pgfsetfillcolor{currentfill}%
\pgfsetlinewidth{0.602250pt}%
\definecolor{currentstroke}{rgb}{0.000000,0.000000,0.000000}%
\pgfsetstrokecolor{currentstroke}%
\pgfsetdash{}{0pt}%
\pgfsys@defobject{currentmarker}{\pgfqpoint{-0.027778in}{0.000000in}}{\pgfqpoint{0.000000in}{0.000000in}}{%
\pgfpathmoveto{\pgfqpoint{0.000000in}{0.000000in}}%
\pgfpathlineto{\pgfqpoint{-0.027778in}{0.000000in}}%
\pgfusepath{stroke,fill}%
}%
\begin{pgfscope}%
\pgfsys@transformshift{0.781944in}{2.234488in}%
\pgfsys@useobject{currentmarker}{}%
\end{pgfscope}%
\end{pgfscope}%
\begin{pgfscope}%
\pgfsetbuttcap%
\pgfsetroundjoin%
\definecolor{currentfill}{rgb}{0.000000,0.000000,0.000000}%
\pgfsetfillcolor{currentfill}%
\pgfsetlinewidth{0.602250pt}%
\definecolor{currentstroke}{rgb}{0.000000,0.000000,0.000000}%
\pgfsetstrokecolor{currentstroke}%
\pgfsetdash{}{0pt}%
\pgfsys@defobject{currentmarker}{\pgfqpoint{0.000000in}{0.000000in}}{\pgfqpoint{0.027778in}{0.000000in}}{%
\pgfpathmoveto{\pgfqpoint{0.000000in}{0.000000in}}%
\pgfpathlineto{\pgfqpoint{0.027778in}{0.000000in}}%
\pgfusepath{stroke,fill}%
}%
\begin{pgfscope}%
\pgfsys@transformshift{4.672917in}{2.234488in}%
\pgfsys@useobject{currentmarker}{}%
\end{pgfscope}%
\end{pgfscope}%
\begin{pgfscope}%
\pgfpathrectangle{\pgfqpoint{0.781944in}{0.552778in}}{\pgfqpoint{3.890972in}{3.248611in}}%
\pgfusepath{clip}%
\pgfsetrectcap%
\pgfsetroundjoin%
\pgfsetlinewidth{0.803000pt}%
\definecolor{currentstroke}{rgb}{0.690196,0.690196,0.690196}%
\pgfsetstrokecolor{currentstroke}%
\pgfsetstrokeopacity{0.300000}%
\pgfsetdash{}{0pt}%
\pgfpathmoveto{\pgfqpoint{0.781944in}{2.320729in}}%
\pgfpathlineto{\pgfqpoint{4.672917in}{2.320729in}}%
\pgfusepath{stroke}%
\end{pgfscope}%
\begin{pgfscope}%
\pgfsetbuttcap%
\pgfsetroundjoin%
\definecolor{currentfill}{rgb}{0.000000,0.000000,0.000000}%
\pgfsetfillcolor{currentfill}%
\pgfsetlinewidth{0.602250pt}%
\definecolor{currentstroke}{rgb}{0.000000,0.000000,0.000000}%
\pgfsetstrokecolor{currentstroke}%
\pgfsetdash{}{0pt}%
\pgfsys@defobject{currentmarker}{\pgfqpoint{-0.027778in}{0.000000in}}{\pgfqpoint{0.000000in}{0.000000in}}{%
\pgfpathmoveto{\pgfqpoint{0.000000in}{0.000000in}}%
\pgfpathlineto{\pgfqpoint{-0.027778in}{0.000000in}}%
\pgfusepath{stroke,fill}%
}%
\begin{pgfscope}%
\pgfsys@transformshift{0.781944in}{2.320729in}%
\pgfsys@useobject{currentmarker}{}%
\end{pgfscope}%
\end{pgfscope}%
\begin{pgfscope}%
\pgfsetbuttcap%
\pgfsetroundjoin%
\definecolor{currentfill}{rgb}{0.000000,0.000000,0.000000}%
\pgfsetfillcolor{currentfill}%
\pgfsetlinewidth{0.602250pt}%
\definecolor{currentstroke}{rgb}{0.000000,0.000000,0.000000}%
\pgfsetstrokecolor{currentstroke}%
\pgfsetdash{}{0pt}%
\pgfsys@defobject{currentmarker}{\pgfqpoint{0.000000in}{0.000000in}}{\pgfqpoint{0.027778in}{0.000000in}}{%
\pgfpathmoveto{\pgfqpoint{0.000000in}{0.000000in}}%
\pgfpathlineto{\pgfqpoint{0.027778in}{0.000000in}}%
\pgfusepath{stroke,fill}%
}%
\begin{pgfscope}%
\pgfsys@transformshift{4.672917in}{2.320729in}%
\pgfsys@useobject{currentmarker}{}%
\end{pgfscope}%
\end{pgfscope}%
\begin{pgfscope}%
\pgfpathrectangle{\pgfqpoint{0.781944in}{0.552778in}}{\pgfqpoint{3.890972in}{3.248611in}}%
\pgfusepath{clip}%
\pgfsetrectcap%
\pgfsetroundjoin%
\pgfsetlinewidth{0.803000pt}%
\definecolor{currentstroke}{rgb}{0.690196,0.690196,0.690196}%
\pgfsetstrokecolor{currentstroke}%
\pgfsetstrokeopacity{0.300000}%
\pgfsetdash{}{0pt}%
\pgfpathmoveto{\pgfqpoint{0.781944in}{2.363850in}}%
\pgfpathlineto{\pgfqpoint{4.672917in}{2.363850in}}%
\pgfusepath{stroke}%
\end{pgfscope}%
\begin{pgfscope}%
\pgfsetbuttcap%
\pgfsetroundjoin%
\definecolor{currentfill}{rgb}{0.000000,0.000000,0.000000}%
\pgfsetfillcolor{currentfill}%
\pgfsetlinewidth{0.602250pt}%
\definecolor{currentstroke}{rgb}{0.000000,0.000000,0.000000}%
\pgfsetstrokecolor{currentstroke}%
\pgfsetdash{}{0pt}%
\pgfsys@defobject{currentmarker}{\pgfqpoint{-0.027778in}{0.000000in}}{\pgfqpoint{0.000000in}{0.000000in}}{%
\pgfpathmoveto{\pgfqpoint{0.000000in}{0.000000in}}%
\pgfpathlineto{\pgfqpoint{-0.027778in}{0.000000in}}%
\pgfusepath{stroke,fill}%
}%
\begin{pgfscope}%
\pgfsys@transformshift{0.781944in}{2.363850in}%
\pgfsys@useobject{currentmarker}{}%
\end{pgfscope}%
\end{pgfscope}%
\begin{pgfscope}%
\pgfsetbuttcap%
\pgfsetroundjoin%
\definecolor{currentfill}{rgb}{0.000000,0.000000,0.000000}%
\pgfsetfillcolor{currentfill}%
\pgfsetlinewidth{0.602250pt}%
\definecolor{currentstroke}{rgb}{0.000000,0.000000,0.000000}%
\pgfsetstrokecolor{currentstroke}%
\pgfsetdash{}{0pt}%
\pgfsys@defobject{currentmarker}{\pgfqpoint{0.000000in}{0.000000in}}{\pgfqpoint{0.027778in}{0.000000in}}{%
\pgfpathmoveto{\pgfqpoint{0.000000in}{0.000000in}}%
\pgfpathlineto{\pgfqpoint{0.027778in}{0.000000in}}%
\pgfusepath{stroke,fill}%
}%
\begin{pgfscope}%
\pgfsys@transformshift{4.672917in}{2.363850in}%
\pgfsys@useobject{currentmarker}{}%
\end{pgfscope}%
\end{pgfscope}%
\begin{pgfscope}%
\pgfpathrectangle{\pgfqpoint{0.781944in}{0.552778in}}{\pgfqpoint{3.890972in}{3.248611in}}%
\pgfusepath{clip}%
\pgfsetrectcap%
\pgfsetroundjoin%
\pgfsetlinewidth{0.803000pt}%
\definecolor{currentstroke}{rgb}{0.690196,0.690196,0.690196}%
\pgfsetstrokecolor{currentstroke}%
\pgfsetstrokeopacity{0.300000}%
\pgfsetdash{}{0pt}%
\pgfpathmoveto{\pgfqpoint{0.781944in}{2.406971in}}%
\pgfpathlineto{\pgfqpoint{4.672917in}{2.406971in}}%
\pgfusepath{stroke}%
\end{pgfscope}%
\begin{pgfscope}%
\pgfsetbuttcap%
\pgfsetroundjoin%
\definecolor{currentfill}{rgb}{0.000000,0.000000,0.000000}%
\pgfsetfillcolor{currentfill}%
\pgfsetlinewidth{0.602250pt}%
\definecolor{currentstroke}{rgb}{0.000000,0.000000,0.000000}%
\pgfsetstrokecolor{currentstroke}%
\pgfsetdash{}{0pt}%
\pgfsys@defobject{currentmarker}{\pgfqpoint{-0.027778in}{0.000000in}}{\pgfqpoint{0.000000in}{0.000000in}}{%
\pgfpathmoveto{\pgfqpoint{0.000000in}{0.000000in}}%
\pgfpathlineto{\pgfqpoint{-0.027778in}{0.000000in}}%
\pgfusepath{stroke,fill}%
}%
\begin{pgfscope}%
\pgfsys@transformshift{0.781944in}{2.406971in}%
\pgfsys@useobject{currentmarker}{}%
\end{pgfscope}%
\end{pgfscope}%
\begin{pgfscope}%
\pgfsetbuttcap%
\pgfsetroundjoin%
\definecolor{currentfill}{rgb}{0.000000,0.000000,0.000000}%
\pgfsetfillcolor{currentfill}%
\pgfsetlinewidth{0.602250pt}%
\definecolor{currentstroke}{rgb}{0.000000,0.000000,0.000000}%
\pgfsetstrokecolor{currentstroke}%
\pgfsetdash{}{0pt}%
\pgfsys@defobject{currentmarker}{\pgfqpoint{0.000000in}{0.000000in}}{\pgfqpoint{0.027778in}{0.000000in}}{%
\pgfpathmoveto{\pgfqpoint{0.000000in}{0.000000in}}%
\pgfpathlineto{\pgfqpoint{0.027778in}{0.000000in}}%
\pgfusepath{stroke,fill}%
}%
\begin{pgfscope}%
\pgfsys@transformshift{4.672917in}{2.406971in}%
\pgfsys@useobject{currentmarker}{}%
\end{pgfscope}%
\end{pgfscope}%
\begin{pgfscope}%
\pgfpathrectangle{\pgfqpoint{0.781944in}{0.552778in}}{\pgfqpoint{3.890972in}{3.248611in}}%
\pgfusepath{clip}%
\pgfsetrectcap%
\pgfsetroundjoin%
\pgfsetlinewidth{0.803000pt}%
\definecolor{currentstroke}{rgb}{0.690196,0.690196,0.690196}%
\pgfsetstrokecolor{currentstroke}%
\pgfsetstrokeopacity{0.300000}%
\pgfsetdash{}{0pt}%
\pgfpathmoveto{\pgfqpoint{0.781944in}{2.450092in}}%
\pgfpathlineto{\pgfqpoint{4.672917in}{2.450092in}}%
\pgfusepath{stroke}%
\end{pgfscope}%
\begin{pgfscope}%
\pgfsetbuttcap%
\pgfsetroundjoin%
\definecolor{currentfill}{rgb}{0.000000,0.000000,0.000000}%
\pgfsetfillcolor{currentfill}%
\pgfsetlinewidth{0.602250pt}%
\definecolor{currentstroke}{rgb}{0.000000,0.000000,0.000000}%
\pgfsetstrokecolor{currentstroke}%
\pgfsetdash{}{0pt}%
\pgfsys@defobject{currentmarker}{\pgfqpoint{-0.027778in}{0.000000in}}{\pgfqpoint{0.000000in}{0.000000in}}{%
\pgfpathmoveto{\pgfqpoint{0.000000in}{0.000000in}}%
\pgfpathlineto{\pgfqpoint{-0.027778in}{0.000000in}}%
\pgfusepath{stroke,fill}%
}%
\begin{pgfscope}%
\pgfsys@transformshift{0.781944in}{2.450092in}%
\pgfsys@useobject{currentmarker}{}%
\end{pgfscope}%
\end{pgfscope}%
\begin{pgfscope}%
\pgfsetbuttcap%
\pgfsetroundjoin%
\definecolor{currentfill}{rgb}{0.000000,0.000000,0.000000}%
\pgfsetfillcolor{currentfill}%
\pgfsetlinewidth{0.602250pt}%
\definecolor{currentstroke}{rgb}{0.000000,0.000000,0.000000}%
\pgfsetstrokecolor{currentstroke}%
\pgfsetdash{}{0pt}%
\pgfsys@defobject{currentmarker}{\pgfqpoint{0.000000in}{0.000000in}}{\pgfqpoint{0.027778in}{0.000000in}}{%
\pgfpathmoveto{\pgfqpoint{0.000000in}{0.000000in}}%
\pgfpathlineto{\pgfqpoint{0.027778in}{0.000000in}}%
\pgfusepath{stroke,fill}%
}%
\begin{pgfscope}%
\pgfsys@transformshift{4.672917in}{2.450092in}%
\pgfsys@useobject{currentmarker}{}%
\end{pgfscope}%
\end{pgfscope}%
\begin{pgfscope}%
\pgfpathrectangle{\pgfqpoint{0.781944in}{0.552778in}}{\pgfqpoint{3.890972in}{3.248611in}}%
\pgfusepath{clip}%
\pgfsetrectcap%
\pgfsetroundjoin%
\pgfsetlinewidth{0.803000pt}%
\definecolor{currentstroke}{rgb}{0.690196,0.690196,0.690196}%
\pgfsetstrokecolor{currentstroke}%
\pgfsetstrokeopacity{0.300000}%
\pgfsetdash{}{0pt}%
\pgfpathmoveto{\pgfqpoint{0.781944in}{2.493212in}}%
\pgfpathlineto{\pgfqpoint{4.672917in}{2.493212in}}%
\pgfusepath{stroke}%
\end{pgfscope}%
\begin{pgfscope}%
\pgfsetbuttcap%
\pgfsetroundjoin%
\definecolor{currentfill}{rgb}{0.000000,0.000000,0.000000}%
\pgfsetfillcolor{currentfill}%
\pgfsetlinewidth{0.602250pt}%
\definecolor{currentstroke}{rgb}{0.000000,0.000000,0.000000}%
\pgfsetstrokecolor{currentstroke}%
\pgfsetdash{}{0pt}%
\pgfsys@defobject{currentmarker}{\pgfqpoint{-0.027778in}{0.000000in}}{\pgfqpoint{0.000000in}{0.000000in}}{%
\pgfpathmoveto{\pgfqpoint{0.000000in}{0.000000in}}%
\pgfpathlineto{\pgfqpoint{-0.027778in}{0.000000in}}%
\pgfusepath{stroke,fill}%
}%
\begin{pgfscope}%
\pgfsys@transformshift{0.781944in}{2.493212in}%
\pgfsys@useobject{currentmarker}{}%
\end{pgfscope}%
\end{pgfscope}%
\begin{pgfscope}%
\pgfsetbuttcap%
\pgfsetroundjoin%
\definecolor{currentfill}{rgb}{0.000000,0.000000,0.000000}%
\pgfsetfillcolor{currentfill}%
\pgfsetlinewidth{0.602250pt}%
\definecolor{currentstroke}{rgb}{0.000000,0.000000,0.000000}%
\pgfsetstrokecolor{currentstroke}%
\pgfsetdash{}{0pt}%
\pgfsys@defobject{currentmarker}{\pgfqpoint{0.000000in}{0.000000in}}{\pgfqpoint{0.027778in}{0.000000in}}{%
\pgfpathmoveto{\pgfqpoint{0.000000in}{0.000000in}}%
\pgfpathlineto{\pgfqpoint{0.027778in}{0.000000in}}%
\pgfusepath{stroke,fill}%
}%
\begin{pgfscope}%
\pgfsys@transformshift{4.672917in}{2.493212in}%
\pgfsys@useobject{currentmarker}{}%
\end{pgfscope}%
\end{pgfscope}%
\begin{pgfscope}%
\pgfpathrectangle{\pgfqpoint{0.781944in}{0.552778in}}{\pgfqpoint{3.890972in}{3.248611in}}%
\pgfusepath{clip}%
\pgfsetrectcap%
\pgfsetroundjoin%
\pgfsetlinewidth{0.803000pt}%
\definecolor{currentstroke}{rgb}{0.690196,0.690196,0.690196}%
\pgfsetstrokecolor{currentstroke}%
\pgfsetstrokeopacity{0.300000}%
\pgfsetdash{}{0pt}%
\pgfpathmoveto{\pgfqpoint{0.781944in}{2.536333in}}%
\pgfpathlineto{\pgfqpoint{4.672917in}{2.536333in}}%
\pgfusepath{stroke}%
\end{pgfscope}%
\begin{pgfscope}%
\pgfsetbuttcap%
\pgfsetroundjoin%
\definecolor{currentfill}{rgb}{0.000000,0.000000,0.000000}%
\pgfsetfillcolor{currentfill}%
\pgfsetlinewidth{0.602250pt}%
\definecolor{currentstroke}{rgb}{0.000000,0.000000,0.000000}%
\pgfsetstrokecolor{currentstroke}%
\pgfsetdash{}{0pt}%
\pgfsys@defobject{currentmarker}{\pgfqpoint{-0.027778in}{0.000000in}}{\pgfqpoint{0.000000in}{0.000000in}}{%
\pgfpathmoveto{\pgfqpoint{0.000000in}{0.000000in}}%
\pgfpathlineto{\pgfqpoint{-0.027778in}{0.000000in}}%
\pgfusepath{stroke,fill}%
}%
\begin{pgfscope}%
\pgfsys@transformshift{0.781944in}{2.536333in}%
\pgfsys@useobject{currentmarker}{}%
\end{pgfscope}%
\end{pgfscope}%
\begin{pgfscope}%
\pgfsetbuttcap%
\pgfsetroundjoin%
\definecolor{currentfill}{rgb}{0.000000,0.000000,0.000000}%
\pgfsetfillcolor{currentfill}%
\pgfsetlinewidth{0.602250pt}%
\definecolor{currentstroke}{rgb}{0.000000,0.000000,0.000000}%
\pgfsetstrokecolor{currentstroke}%
\pgfsetdash{}{0pt}%
\pgfsys@defobject{currentmarker}{\pgfqpoint{0.000000in}{0.000000in}}{\pgfqpoint{0.027778in}{0.000000in}}{%
\pgfpathmoveto{\pgfqpoint{0.000000in}{0.000000in}}%
\pgfpathlineto{\pgfqpoint{0.027778in}{0.000000in}}%
\pgfusepath{stroke,fill}%
}%
\begin{pgfscope}%
\pgfsys@transformshift{4.672917in}{2.536333in}%
\pgfsys@useobject{currentmarker}{}%
\end{pgfscope}%
\end{pgfscope}%
\begin{pgfscope}%
\pgfpathrectangle{\pgfqpoint{0.781944in}{0.552778in}}{\pgfqpoint{3.890972in}{3.248611in}}%
\pgfusepath{clip}%
\pgfsetrectcap%
\pgfsetroundjoin%
\pgfsetlinewidth{0.803000pt}%
\definecolor{currentstroke}{rgb}{0.690196,0.690196,0.690196}%
\pgfsetstrokecolor{currentstroke}%
\pgfsetstrokeopacity{0.300000}%
\pgfsetdash{}{0pt}%
\pgfpathmoveto{\pgfqpoint{0.781944in}{2.579454in}}%
\pgfpathlineto{\pgfqpoint{4.672917in}{2.579454in}}%
\pgfusepath{stroke}%
\end{pgfscope}%
\begin{pgfscope}%
\pgfsetbuttcap%
\pgfsetroundjoin%
\definecolor{currentfill}{rgb}{0.000000,0.000000,0.000000}%
\pgfsetfillcolor{currentfill}%
\pgfsetlinewidth{0.602250pt}%
\definecolor{currentstroke}{rgb}{0.000000,0.000000,0.000000}%
\pgfsetstrokecolor{currentstroke}%
\pgfsetdash{}{0pt}%
\pgfsys@defobject{currentmarker}{\pgfqpoint{-0.027778in}{0.000000in}}{\pgfqpoint{0.000000in}{0.000000in}}{%
\pgfpathmoveto{\pgfqpoint{0.000000in}{0.000000in}}%
\pgfpathlineto{\pgfqpoint{-0.027778in}{0.000000in}}%
\pgfusepath{stroke,fill}%
}%
\begin{pgfscope}%
\pgfsys@transformshift{0.781944in}{2.579454in}%
\pgfsys@useobject{currentmarker}{}%
\end{pgfscope}%
\end{pgfscope}%
\begin{pgfscope}%
\pgfsetbuttcap%
\pgfsetroundjoin%
\definecolor{currentfill}{rgb}{0.000000,0.000000,0.000000}%
\pgfsetfillcolor{currentfill}%
\pgfsetlinewidth{0.602250pt}%
\definecolor{currentstroke}{rgb}{0.000000,0.000000,0.000000}%
\pgfsetstrokecolor{currentstroke}%
\pgfsetdash{}{0pt}%
\pgfsys@defobject{currentmarker}{\pgfqpoint{0.000000in}{0.000000in}}{\pgfqpoint{0.027778in}{0.000000in}}{%
\pgfpathmoveto{\pgfqpoint{0.000000in}{0.000000in}}%
\pgfpathlineto{\pgfqpoint{0.027778in}{0.000000in}}%
\pgfusepath{stroke,fill}%
}%
\begin{pgfscope}%
\pgfsys@transformshift{4.672917in}{2.579454in}%
\pgfsys@useobject{currentmarker}{}%
\end{pgfscope}%
\end{pgfscope}%
\begin{pgfscope}%
\pgfpathrectangle{\pgfqpoint{0.781944in}{0.552778in}}{\pgfqpoint{3.890972in}{3.248611in}}%
\pgfusepath{clip}%
\pgfsetrectcap%
\pgfsetroundjoin%
\pgfsetlinewidth{0.803000pt}%
\definecolor{currentstroke}{rgb}{0.690196,0.690196,0.690196}%
\pgfsetstrokecolor{currentstroke}%
\pgfsetstrokeopacity{0.300000}%
\pgfsetdash{}{0pt}%
\pgfpathmoveto{\pgfqpoint{0.781944in}{2.622575in}}%
\pgfpathlineto{\pgfqpoint{4.672917in}{2.622575in}}%
\pgfusepath{stroke}%
\end{pgfscope}%
\begin{pgfscope}%
\pgfsetbuttcap%
\pgfsetroundjoin%
\definecolor{currentfill}{rgb}{0.000000,0.000000,0.000000}%
\pgfsetfillcolor{currentfill}%
\pgfsetlinewidth{0.602250pt}%
\definecolor{currentstroke}{rgb}{0.000000,0.000000,0.000000}%
\pgfsetstrokecolor{currentstroke}%
\pgfsetdash{}{0pt}%
\pgfsys@defobject{currentmarker}{\pgfqpoint{-0.027778in}{0.000000in}}{\pgfqpoint{0.000000in}{0.000000in}}{%
\pgfpathmoveto{\pgfqpoint{0.000000in}{0.000000in}}%
\pgfpathlineto{\pgfqpoint{-0.027778in}{0.000000in}}%
\pgfusepath{stroke,fill}%
}%
\begin{pgfscope}%
\pgfsys@transformshift{0.781944in}{2.622575in}%
\pgfsys@useobject{currentmarker}{}%
\end{pgfscope}%
\end{pgfscope}%
\begin{pgfscope}%
\pgfsetbuttcap%
\pgfsetroundjoin%
\definecolor{currentfill}{rgb}{0.000000,0.000000,0.000000}%
\pgfsetfillcolor{currentfill}%
\pgfsetlinewidth{0.602250pt}%
\definecolor{currentstroke}{rgb}{0.000000,0.000000,0.000000}%
\pgfsetstrokecolor{currentstroke}%
\pgfsetdash{}{0pt}%
\pgfsys@defobject{currentmarker}{\pgfqpoint{0.000000in}{0.000000in}}{\pgfqpoint{0.027778in}{0.000000in}}{%
\pgfpathmoveto{\pgfqpoint{0.000000in}{0.000000in}}%
\pgfpathlineto{\pgfqpoint{0.027778in}{0.000000in}}%
\pgfusepath{stroke,fill}%
}%
\begin{pgfscope}%
\pgfsys@transformshift{4.672917in}{2.622575in}%
\pgfsys@useobject{currentmarker}{}%
\end{pgfscope}%
\end{pgfscope}%
\begin{pgfscope}%
\pgfpathrectangle{\pgfqpoint{0.781944in}{0.552778in}}{\pgfqpoint{3.890972in}{3.248611in}}%
\pgfusepath{clip}%
\pgfsetrectcap%
\pgfsetroundjoin%
\pgfsetlinewidth{0.803000pt}%
\definecolor{currentstroke}{rgb}{0.690196,0.690196,0.690196}%
\pgfsetstrokecolor{currentstroke}%
\pgfsetstrokeopacity{0.300000}%
\pgfsetdash{}{0pt}%
\pgfpathmoveto{\pgfqpoint{0.781944in}{2.665696in}}%
\pgfpathlineto{\pgfqpoint{4.672917in}{2.665696in}}%
\pgfusepath{stroke}%
\end{pgfscope}%
\begin{pgfscope}%
\pgfsetbuttcap%
\pgfsetroundjoin%
\definecolor{currentfill}{rgb}{0.000000,0.000000,0.000000}%
\pgfsetfillcolor{currentfill}%
\pgfsetlinewidth{0.602250pt}%
\definecolor{currentstroke}{rgb}{0.000000,0.000000,0.000000}%
\pgfsetstrokecolor{currentstroke}%
\pgfsetdash{}{0pt}%
\pgfsys@defobject{currentmarker}{\pgfqpoint{-0.027778in}{0.000000in}}{\pgfqpoint{0.000000in}{0.000000in}}{%
\pgfpathmoveto{\pgfqpoint{0.000000in}{0.000000in}}%
\pgfpathlineto{\pgfqpoint{-0.027778in}{0.000000in}}%
\pgfusepath{stroke,fill}%
}%
\begin{pgfscope}%
\pgfsys@transformshift{0.781944in}{2.665696in}%
\pgfsys@useobject{currentmarker}{}%
\end{pgfscope}%
\end{pgfscope}%
\begin{pgfscope}%
\pgfsetbuttcap%
\pgfsetroundjoin%
\definecolor{currentfill}{rgb}{0.000000,0.000000,0.000000}%
\pgfsetfillcolor{currentfill}%
\pgfsetlinewidth{0.602250pt}%
\definecolor{currentstroke}{rgb}{0.000000,0.000000,0.000000}%
\pgfsetstrokecolor{currentstroke}%
\pgfsetdash{}{0pt}%
\pgfsys@defobject{currentmarker}{\pgfqpoint{0.000000in}{0.000000in}}{\pgfqpoint{0.027778in}{0.000000in}}{%
\pgfpathmoveto{\pgfqpoint{0.000000in}{0.000000in}}%
\pgfpathlineto{\pgfqpoint{0.027778in}{0.000000in}}%
\pgfusepath{stroke,fill}%
}%
\begin{pgfscope}%
\pgfsys@transformshift{4.672917in}{2.665696in}%
\pgfsys@useobject{currentmarker}{}%
\end{pgfscope}%
\end{pgfscope}%
\begin{pgfscope}%
\pgfpathrectangle{\pgfqpoint{0.781944in}{0.552778in}}{\pgfqpoint{3.890972in}{3.248611in}}%
\pgfusepath{clip}%
\pgfsetrectcap%
\pgfsetroundjoin%
\pgfsetlinewidth{0.803000pt}%
\definecolor{currentstroke}{rgb}{0.690196,0.690196,0.690196}%
\pgfsetstrokecolor{currentstroke}%
\pgfsetstrokeopacity{0.300000}%
\pgfsetdash{}{0pt}%
\pgfpathmoveto{\pgfqpoint{0.781944in}{2.751937in}}%
\pgfpathlineto{\pgfqpoint{4.672917in}{2.751937in}}%
\pgfusepath{stroke}%
\end{pgfscope}%
\begin{pgfscope}%
\pgfsetbuttcap%
\pgfsetroundjoin%
\definecolor{currentfill}{rgb}{0.000000,0.000000,0.000000}%
\pgfsetfillcolor{currentfill}%
\pgfsetlinewidth{0.602250pt}%
\definecolor{currentstroke}{rgb}{0.000000,0.000000,0.000000}%
\pgfsetstrokecolor{currentstroke}%
\pgfsetdash{}{0pt}%
\pgfsys@defobject{currentmarker}{\pgfqpoint{-0.027778in}{0.000000in}}{\pgfqpoint{0.000000in}{0.000000in}}{%
\pgfpathmoveto{\pgfqpoint{0.000000in}{0.000000in}}%
\pgfpathlineto{\pgfqpoint{-0.027778in}{0.000000in}}%
\pgfusepath{stroke,fill}%
}%
\begin{pgfscope}%
\pgfsys@transformshift{0.781944in}{2.751937in}%
\pgfsys@useobject{currentmarker}{}%
\end{pgfscope}%
\end{pgfscope}%
\begin{pgfscope}%
\pgfsetbuttcap%
\pgfsetroundjoin%
\definecolor{currentfill}{rgb}{0.000000,0.000000,0.000000}%
\pgfsetfillcolor{currentfill}%
\pgfsetlinewidth{0.602250pt}%
\definecolor{currentstroke}{rgb}{0.000000,0.000000,0.000000}%
\pgfsetstrokecolor{currentstroke}%
\pgfsetdash{}{0pt}%
\pgfsys@defobject{currentmarker}{\pgfqpoint{0.000000in}{0.000000in}}{\pgfqpoint{0.027778in}{0.000000in}}{%
\pgfpathmoveto{\pgfqpoint{0.000000in}{0.000000in}}%
\pgfpathlineto{\pgfqpoint{0.027778in}{0.000000in}}%
\pgfusepath{stroke,fill}%
}%
\begin{pgfscope}%
\pgfsys@transformshift{4.672917in}{2.751937in}%
\pgfsys@useobject{currentmarker}{}%
\end{pgfscope}%
\end{pgfscope}%
\begin{pgfscope}%
\pgfpathrectangle{\pgfqpoint{0.781944in}{0.552778in}}{\pgfqpoint{3.890972in}{3.248611in}}%
\pgfusepath{clip}%
\pgfsetrectcap%
\pgfsetroundjoin%
\pgfsetlinewidth{0.803000pt}%
\definecolor{currentstroke}{rgb}{0.690196,0.690196,0.690196}%
\pgfsetstrokecolor{currentstroke}%
\pgfsetstrokeopacity{0.300000}%
\pgfsetdash{}{0pt}%
\pgfpathmoveto{\pgfqpoint{0.781944in}{2.795058in}}%
\pgfpathlineto{\pgfqpoint{4.672917in}{2.795058in}}%
\pgfusepath{stroke}%
\end{pgfscope}%
\begin{pgfscope}%
\pgfsetbuttcap%
\pgfsetroundjoin%
\definecolor{currentfill}{rgb}{0.000000,0.000000,0.000000}%
\pgfsetfillcolor{currentfill}%
\pgfsetlinewidth{0.602250pt}%
\definecolor{currentstroke}{rgb}{0.000000,0.000000,0.000000}%
\pgfsetstrokecolor{currentstroke}%
\pgfsetdash{}{0pt}%
\pgfsys@defobject{currentmarker}{\pgfqpoint{-0.027778in}{0.000000in}}{\pgfqpoint{0.000000in}{0.000000in}}{%
\pgfpathmoveto{\pgfqpoint{0.000000in}{0.000000in}}%
\pgfpathlineto{\pgfqpoint{-0.027778in}{0.000000in}}%
\pgfusepath{stroke,fill}%
}%
\begin{pgfscope}%
\pgfsys@transformshift{0.781944in}{2.795058in}%
\pgfsys@useobject{currentmarker}{}%
\end{pgfscope}%
\end{pgfscope}%
\begin{pgfscope}%
\pgfsetbuttcap%
\pgfsetroundjoin%
\definecolor{currentfill}{rgb}{0.000000,0.000000,0.000000}%
\pgfsetfillcolor{currentfill}%
\pgfsetlinewidth{0.602250pt}%
\definecolor{currentstroke}{rgb}{0.000000,0.000000,0.000000}%
\pgfsetstrokecolor{currentstroke}%
\pgfsetdash{}{0pt}%
\pgfsys@defobject{currentmarker}{\pgfqpoint{0.000000in}{0.000000in}}{\pgfqpoint{0.027778in}{0.000000in}}{%
\pgfpathmoveto{\pgfqpoint{0.000000in}{0.000000in}}%
\pgfpathlineto{\pgfqpoint{0.027778in}{0.000000in}}%
\pgfusepath{stroke,fill}%
}%
\begin{pgfscope}%
\pgfsys@transformshift{4.672917in}{2.795058in}%
\pgfsys@useobject{currentmarker}{}%
\end{pgfscope}%
\end{pgfscope}%
\begin{pgfscope}%
\pgfpathrectangle{\pgfqpoint{0.781944in}{0.552778in}}{\pgfqpoint{3.890972in}{3.248611in}}%
\pgfusepath{clip}%
\pgfsetrectcap%
\pgfsetroundjoin%
\pgfsetlinewidth{0.803000pt}%
\definecolor{currentstroke}{rgb}{0.690196,0.690196,0.690196}%
\pgfsetstrokecolor{currentstroke}%
\pgfsetstrokeopacity{0.300000}%
\pgfsetdash{}{0pt}%
\pgfpathmoveto{\pgfqpoint{0.781944in}{2.838179in}}%
\pgfpathlineto{\pgfqpoint{4.672917in}{2.838179in}}%
\pgfusepath{stroke}%
\end{pgfscope}%
\begin{pgfscope}%
\pgfsetbuttcap%
\pgfsetroundjoin%
\definecolor{currentfill}{rgb}{0.000000,0.000000,0.000000}%
\pgfsetfillcolor{currentfill}%
\pgfsetlinewidth{0.602250pt}%
\definecolor{currentstroke}{rgb}{0.000000,0.000000,0.000000}%
\pgfsetstrokecolor{currentstroke}%
\pgfsetdash{}{0pt}%
\pgfsys@defobject{currentmarker}{\pgfqpoint{-0.027778in}{0.000000in}}{\pgfqpoint{0.000000in}{0.000000in}}{%
\pgfpathmoveto{\pgfqpoint{0.000000in}{0.000000in}}%
\pgfpathlineto{\pgfqpoint{-0.027778in}{0.000000in}}%
\pgfusepath{stroke,fill}%
}%
\begin{pgfscope}%
\pgfsys@transformshift{0.781944in}{2.838179in}%
\pgfsys@useobject{currentmarker}{}%
\end{pgfscope}%
\end{pgfscope}%
\begin{pgfscope}%
\pgfsetbuttcap%
\pgfsetroundjoin%
\definecolor{currentfill}{rgb}{0.000000,0.000000,0.000000}%
\pgfsetfillcolor{currentfill}%
\pgfsetlinewidth{0.602250pt}%
\definecolor{currentstroke}{rgb}{0.000000,0.000000,0.000000}%
\pgfsetstrokecolor{currentstroke}%
\pgfsetdash{}{0pt}%
\pgfsys@defobject{currentmarker}{\pgfqpoint{0.000000in}{0.000000in}}{\pgfqpoint{0.027778in}{0.000000in}}{%
\pgfpathmoveto{\pgfqpoint{0.000000in}{0.000000in}}%
\pgfpathlineto{\pgfqpoint{0.027778in}{0.000000in}}%
\pgfusepath{stroke,fill}%
}%
\begin{pgfscope}%
\pgfsys@transformshift{4.672917in}{2.838179in}%
\pgfsys@useobject{currentmarker}{}%
\end{pgfscope}%
\end{pgfscope}%
\begin{pgfscope}%
\pgfpathrectangle{\pgfqpoint{0.781944in}{0.552778in}}{\pgfqpoint{3.890972in}{3.248611in}}%
\pgfusepath{clip}%
\pgfsetrectcap%
\pgfsetroundjoin%
\pgfsetlinewidth{0.803000pt}%
\definecolor{currentstroke}{rgb}{0.690196,0.690196,0.690196}%
\pgfsetstrokecolor{currentstroke}%
\pgfsetstrokeopacity{0.300000}%
\pgfsetdash{}{0pt}%
\pgfpathmoveto{\pgfqpoint{0.781944in}{2.881299in}}%
\pgfpathlineto{\pgfqpoint{4.672917in}{2.881299in}}%
\pgfusepath{stroke}%
\end{pgfscope}%
\begin{pgfscope}%
\pgfsetbuttcap%
\pgfsetroundjoin%
\definecolor{currentfill}{rgb}{0.000000,0.000000,0.000000}%
\pgfsetfillcolor{currentfill}%
\pgfsetlinewidth{0.602250pt}%
\definecolor{currentstroke}{rgb}{0.000000,0.000000,0.000000}%
\pgfsetstrokecolor{currentstroke}%
\pgfsetdash{}{0pt}%
\pgfsys@defobject{currentmarker}{\pgfqpoint{-0.027778in}{0.000000in}}{\pgfqpoint{0.000000in}{0.000000in}}{%
\pgfpathmoveto{\pgfqpoint{0.000000in}{0.000000in}}%
\pgfpathlineto{\pgfqpoint{-0.027778in}{0.000000in}}%
\pgfusepath{stroke,fill}%
}%
\begin{pgfscope}%
\pgfsys@transformshift{0.781944in}{2.881299in}%
\pgfsys@useobject{currentmarker}{}%
\end{pgfscope}%
\end{pgfscope}%
\begin{pgfscope}%
\pgfsetbuttcap%
\pgfsetroundjoin%
\definecolor{currentfill}{rgb}{0.000000,0.000000,0.000000}%
\pgfsetfillcolor{currentfill}%
\pgfsetlinewidth{0.602250pt}%
\definecolor{currentstroke}{rgb}{0.000000,0.000000,0.000000}%
\pgfsetstrokecolor{currentstroke}%
\pgfsetdash{}{0pt}%
\pgfsys@defobject{currentmarker}{\pgfqpoint{0.000000in}{0.000000in}}{\pgfqpoint{0.027778in}{0.000000in}}{%
\pgfpathmoveto{\pgfqpoint{0.000000in}{0.000000in}}%
\pgfpathlineto{\pgfqpoint{0.027778in}{0.000000in}}%
\pgfusepath{stroke,fill}%
}%
\begin{pgfscope}%
\pgfsys@transformshift{4.672917in}{2.881299in}%
\pgfsys@useobject{currentmarker}{}%
\end{pgfscope}%
\end{pgfscope}%
\begin{pgfscope}%
\pgfpathrectangle{\pgfqpoint{0.781944in}{0.552778in}}{\pgfqpoint{3.890972in}{3.248611in}}%
\pgfusepath{clip}%
\pgfsetrectcap%
\pgfsetroundjoin%
\pgfsetlinewidth{0.803000pt}%
\definecolor{currentstroke}{rgb}{0.690196,0.690196,0.690196}%
\pgfsetstrokecolor{currentstroke}%
\pgfsetstrokeopacity{0.300000}%
\pgfsetdash{}{0pt}%
\pgfpathmoveto{\pgfqpoint{0.781944in}{2.924420in}}%
\pgfpathlineto{\pgfqpoint{4.672917in}{2.924420in}}%
\pgfusepath{stroke}%
\end{pgfscope}%
\begin{pgfscope}%
\pgfsetbuttcap%
\pgfsetroundjoin%
\definecolor{currentfill}{rgb}{0.000000,0.000000,0.000000}%
\pgfsetfillcolor{currentfill}%
\pgfsetlinewidth{0.602250pt}%
\definecolor{currentstroke}{rgb}{0.000000,0.000000,0.000000}%
\pgfsetstrokecolor{currentstroke}%
\pgfsetdash{}{0pt}%
\pgfsys@defobject{currentmarker}{\pgfqpoint{-0.027778in}{0.000000in}}{\pgfqpoint{0.000000in}{0.000000in}}{%
\pgfpathmoveto{\pgfqpoint{0.000000in}{0.000000in}}%
\pgfpathlineto{\pgfqpoint{-0.027778in}{0.000000in}}%
\pgfusepath{stroke,fill}%
}%
\begin{pgfscope}%
\pgfsys@transformshift{0.781944in}{2.924420in}%
\pgfsys@useobject{currentmarker}{}%
\end{pgfscope}%
\end{pgfscope}%
\begin{pgfscope}%
\pgfsetbuttcap%
\pgfsetroundjoin%
\definecolor{currentfill}{rgb}{0.000000,0.000000,0.000000}%
\pgfsetfillcolor{currentfill}%
\pgfsetlinewidth{0.602250pt}%
\definecolor{currentstroke}{rgb}{0.000000,0.000000,0.000000}%
\pgfsetstrokecolor{currentstroke}%
\pgfsetdash{}{0pt}%
\pgfsys@defobject{currentmarker}{\pgfqpoint{0.000000in}{0.000000in}}{\pgfqpoint{0.027778in}{0.000000in}}{%
\pgfpathmoveto{\pgfqpoint{0.000000in}{0.000000in}}%
\pgfpathlineto{\pgfqpoint{0.027778in}{0.000000in}}%
\pgfusepath{stroke,fill}%
}%
\begin{pgfscope}%
\pgfsys@transformshift{4.672917in}{2.924420in}%
\pgfsys@useobject{currentmarker}{}%
\end{pgfscope}%
\end{pgfscope}%
\begin{pgfscope}%
\pgfpathrectangle{\pgfqpoint{0.781944in}{0.552778in}}{\pgfqpoint{3.890972in}{3.248611in}}%
\pgfusepath{clip}%
\pgfsetrectcap%
\pgfsetroundjoin%
\pgfsetlinewidth{0.803000pt}%
\definecolor{currentstroke}{rgb}{0.690196,0.690196,0.690196}%
\pgfsetstrokecolor{currentstroke}%
\pgfsetstrokeopacity{0.300000}%
\pgfsetdash{}{0pt}%
\pgfpathmoveto{\pgfqpoint{0.781944in}{2.967541in}}%
\pgfpathlineto{\pgfqpoint{4.672917in}{2.967541in}}%
\pgfusepath{stroke}%
\end{pgfscope}%
\begin{pgfscope}%
\pgfsetbuttcap%
\pgfsetroundjoin%
\definecolor{currentfill}{rgb}{0.000000,0.000000,0.000000}%
\pgfsetfillcolor{currentfill}%
\pgfsetlinewidth{0.602250pt}%
\definecolor{currentstroke}{rgb}{0.000000,0.000000,0.000000}%
\pgfsetstrokecolor{currentstroke}%
\pgfsetdash{}{0pt}%
\pgfsys@defobject{currentmarker}{\pgfqpoint{-0.027778in}{0.000000in}}{\pgfqpoint{0.000000in}{0.000000in}}{%
\pgfpathmoveto{\pgfqpoint{0.000000in}{0.000000in}}%
\pgfpathlineto{\pgfqpoint{-0.027778in}{0.000000in}}%
\pgfusepath{stroke,fill}%
}%
\begin{pgfscope}%
\pgfsys@transformshift{0.781944in}{2.967541in}%
\pgfsys@useobject{currentmarker}{}%
\end{pgfscope}%
\end{pgfscope}%
\begin{pgfscope}%
\pgfsetbuttcap%
\pgfsetroundjoin%
\definecolor{currentfill}{rgb}{0.000000,0.000000,0.000000}%
\pgfsetfillcolor{currentfill}%
\pgfsetlinewidth{0.602250pt}%
\definecolor{currentstroke}{rgb}{0.000000,0.000000,0.000000}%
\pgfsetstrokecolor{currentstroke}%
\pgfsetdash{}{0pt}%
\pgfsys@defobject{currentmarker}{\pgfqpoint{0.000000in}{0.000000in}}{\pgfqpoint{0.027778in}{0.000000in}}{%
\pgfpathmoveto{\pgfqpoint{0.000000in}{0.000000in}}%
\pgfpathlineto{\pgfqpoint{0.027778in}{0.000000in}}%
\pgfusepath{stroke,fill}%
}%
\begin{pgfscope}%
\pgfsys@transformshift{4.672917in}{2.967541in}%
\pgfsys@useobject{currentmarker}{}%
\end{pgfscope}%
\end{pgfscope}%
\begin{pgfscope}%
\pgfpathrectangle{\pgfqpoint{0.781944in}{0.552778in}}{\pgfqpoint{3.890972in}{3.248611in}}%
\pgfusepath{clip}%
\pgfsetrectcap%
\pgfsetroundjoin%
\pgfsetlinewidth{0.803000pt}%
\definecolor{currentstroke}{rgb}{0.690196,0.690196,0.690196}%
\pgfsetstrokecolor{currentstroke}%
\pgfsetstrokeopacity{0.300000}%
\pgfsetdash{}{0pt}%
\pgfpathmoveto{\pgfqpoint{0.781944in}{3.010662in}}%
\pgfpathlineto{\pgfqpoint{4.672917in}{3.010662in}}%
\pgfusepath{stroke}%
\end{pgfscope}%
\begin{pgfscope}%
\pgfsetbuttcap%
\pgfsetroundjoin%
\definecolor{currentfill}{rgb}{0.000000,0.000000,0.000000}%
\pgfsetfillcolor{currentfill}%
\pgfsetlinewidth{0.602250pt}%
\definecolor{currentstroke}{rgb}{0.000000,0.000000,0.000000}%
\pgfsetstrokecolor{currentstroke}%
\pgfsetdash{}{0pt}%
\pgfsys@defobject{currentmarker}{\pgfqpoint{-0.027778in}{0.000000in}}{\pgfqpoint{0.000000in}{0.000000in}}{%
\pgfpathmoveto{\pgfqpoint{0.000000in}{0.000000in}}%
\pgfpathlineto{\pgfqpoint{-0.027778in}{0.000000in}}%
\pgfusepath{stroke,fill}%
}%
\begin{pgfscope}%
\pgfsys@transformshift{0.781944in}{3.010662in}%
\pgfsys@useobject{currentmarker}{}%
\end{pgfscope}%
\end{pgfscope}%
\begin{pgfscope}%
\pgfsetbuttcap%
\pgfsetroundjoin%
\definecolor{currentfill}{rgb}{0.000000,0.000000,0.000000}%
\pgfsetfillcolor{currentfill}%
\pgfsetlinewidth{0.602250pt}%
\definecolor{currentstroke}{rgb}{0.000000,0.000000,0.000000}%
\pgfsetstrokecolor{currentstroke}%
\pgfsetdash{}{0pt}%
\pgfsys@defobject{currentmarker}{\pgfqpoint{0.000000in}{0.000000in}}{\pgfqpoint{0.027778in}{0.000000in}}{%
\pgfpathmoveto{\pgfqpoint{0.000000in}{0.000000in}}%
\pgfpathlineto{\pgfqpoint{0.027778in}{0.000000in}}%
\pgfusepath{stroke,fill}%
}%
\begin{pgfscope}%
\pgfsys@transformshift{4.672917in}{3.010662in}%
\pgfsys@useobject{currentmarker}{}%
\end{pgfscope}%
\end{pgfscope}%
\begin{pgfscope}%
\pgfpathrectangle{\pgfqpoint{0.781944in}{0.552778in}}{\pgfqpoint{3.890972in}{3.248611in}}%
\pgfusepath{clip}%
\pgfsetrectcap%
\pgfsetroundjoin%
\pgfsetlinewidth{0.803000pt}%
\definecolor{currentstroke}{rgb}{0.690196,0.690196,0.690196}%
\pgfsetstrokecolor{currentstroke}%
\pgfsetstrokeopacity{0.300000}%
\pgfsetdash{}{0pt}%
\pgfpathmoveto{\pgfqpoint{0.781944in}{3.053783in}}%
\pgfpathlineto{\pgfqpoint{4.672917in}{3.053783in}}%
\pgfusepath{stroke}%
\end{pgfscope}%
\begin{pgfscope}%
\pgfsetbuttcap%
\pgfsetroundjoin%
\definecolor{currentfill}{rgb}{0.000000,0.000000,0.000000}%
\pgfsetfillcolor{currentfill}%
\pgfsetlinewidth{0.602250pt}%
\definecolor{currentstroke}{rgb}{0.000000,0.000000,0.000000}%
\pgfsetstrokecolor{currentstroke}%
\pgfsetdash{}{0pt}%
\pgfsys@defobject{currentmarker}{\pgfqpoint{-0.027778in}{0.000000in}}{\pgfqpoint{0.000000in}{0.000000in}}{%
\pgfpathmoveto{\pgfqpoint{0.000000in}{0.000000in}}%
\pgfpathlineto{\pgfqpoint{-0.027778in}{0.000000in}}%
\pgfusepath{stroke,fill}%
}%
\begin{pgfscope}%
\pgfsys@transformshift{0.781944in}{3.053783in}%
\pgfsys@useobject{currentmarker}{}%
\end{pgfscope}%
\end{pgfscope}%
\begin{pgfscope}%
\pgfsetbuttcap%
\pgfsetroundjoin%
\definecolor{currentfill}{rgb}{0.000000,0.000000,0.000000}%
\pgfsetfillcolor{currentfill}%
\pgfsetlinewidth{0.602250pt}%
\definecolor{currentstroke}{rgb}{0.000000,0.000000,0.000000}%
\pgfsetstrokecolor{currentstroke}%
\pgfsetdash{}{0pt}%
\pgfsys@defobject{currentmarker}{\pgfqpoint{0.000000in}{0.000000in}}{\pgfqpoint{0.027778in}{0.000000in}}{%
\pgfpathmoveto{\pgfqpoint{0.000000in}{0.000000in}}%
\pgfpathlineto{\pgfqpoint{0.027778in}{0.000000in}}%
\pgfusepath{stroke,fill}%
}%
\begin{pgfscope}%
\pgfsys@transformshift{4.672917in}{3.053783in}%
\pgfsys@useobject{currentmarker}{}%
\end{pgfscope}%
\end{pgfscope}%
\begin{pgfscope}%
\pgfpathrectangle{\pgfqpoint{0.781944in}{0.552778in}}{\pgfqpoint{3.890972in}{3.248611in}}%
\pgfusepath{clip}%
\pgfsetrectcap%
\pgfsetroundjoin%
\pgfsetlinewidth{0.803000pt}%
\definecolor{currentstroke}{rgb}{0.690196,0.690196,0.690196}%
\pgfsetstrokecolor{currentstroke}%
\pgfsetstrokeopacity{0.300000}%
\pgfsetdash{}{0pt}%
\pgfpathmoveto{\pgfqpoint{0.781944in}{3.096903in}}%
\pgfpathlineto{\pgfqpoint{4.672917in}{3.096903in}}%
\pgfusepath{stroke}%
\end{pgfscope}%
\begin{pgfscope}%
\pgfsetbuttcap%
\pgfsetroundjoin%
\definecolor{currentfill}{rgb}{0.000000,0.000000,0.000000}%
\pgfsetfillcolor{currentfill}%
\pgfsetlinewidth{0.602250pt}%
\definecolor{currentstroke}{rgb}{0.000000,0.000000,0.000000}%
\pgfsetstrokecolor{currentstroke}%
\pgfsetdash{}{0pt}%
\pgfsys@defobject{currentmarker}{\pgfqpoint{-0.027778in}{0.000000in}}{\pgfqpoint{0.000000in}{0.000000in}}{%
\pgfpathmoveto{\pgfqpoint{0.000000in}{0.000000in}}%
\pgfpathlineto{\pgfqpoint{-0.027778in}{0.000000in}}%
\pgfusepath{stroke,fill}%
}%
\begin{pgfscope}%
\pgfsys@transformshift{0.781944in}{3.096903in}%
\pgfsys@useobject{currentmarker}{}%
\end{pgfscope}%
\end{pgfscope}%
\begin{pgfscope}%
\pgfsetbuttcap%
\pgfsetroundjoin%
\definecolor{currentfill}{rgb}{0.000000,0.000000,0.000000}%
\pgfsetfillcolor{currentfill}%
\pgfsetlinewidth{0.602250pt}%
\definecolor{currentstroke}{rgb}{0.000000,0.000000,0.000000}%
\pgfsetstrokecolor{currentstroke}%
\pgfsetdash{}{0pt}%
\pgfsys@defobject{currentmarker}{\pgfqpoint{0.000000in}{0.000000in}}{\pgfqpoint{0.027778in}{0.000000in}}{%
\pgfpathmoveto{\pgfqpoint{0.000000in}{0.000000in}}%
\pgfpathlineto{\pgfqpoint{0.027778in}{0.000000in}}%
\pgfusepath{stroke,fill}%
}%
\begin{pgfscope}%
\pgfsys@transformshift{4.672917in}{3.096903in}%
\pgfsys@useobject{currentmarker}{}%
\end{pgfscope}%
\end{pgfscope}%
\begin{pgfscope}%
\pgfpathrectangle{\pgfqpoint{0.781944in}{0.552778in}}{\pgfqpoint{3.890972in}{3.248611in}}%
\pgfusepath{clip}%
\pgfsetrectcap%
\pgfsetroundjoin%
\pgfsetlinewidth{0.803000pt}%
\definecolor{currentstroke}{rgb}{0.690196,0.690196,0.690196}%
\pgfsetstrokecolor{currentstroke}%
\pgfsetstrokeopacity{0.300000}%
\pgfsetdash{}{0pt}%
\pgfpathmoveto{\pgfqpoint{0.781944in}{3.183145in}}%
\pgfpathlineto{\pgfqpoint{4.672917in}{3.183145in}}%
\pgfusepath{stroke}%
\end{pgfscope}%
\begin{pgfscope}%
\pgfsetbuttcap%
\pgfsetroundjoin%
\definecolor{currentfill}{rgb}{0.000000,0.000000,0.000000}%
\pgfsetfillcolor{currentfill}%
\pgfsetlinewidth{0.602250pt}%
\definecolor{currentstroke}{rgb}{0.000000,0.000000,0.000000}%
\pgfsetstrokecolor{currentstroke}%
\pgfsetdash{}{0pt}%
\pgfsys@defobject{currentmarker}{\pgfqpoint{-0.027778in}{0.000000in}}{\pgfqpoint{0.000000in}{0.000000in}}{%
\pgfpathmoveto{\pgfqpoint{0.000000in}{0.000000in}}%
\pgfpathlineto{\pgfqpoint{-0.027778in}{0.000000in}}%
\pgfusepath{stroke,fill}%
}%
\begin{pgfscope}%
\pgfsys@transformshift{0.781944in}{3.183145in}%
\pgfsys@useobject{currentmarker}{}%
\end{pgfscope}%
\end{pgfscope}%
\begin{pgfscope}%
\pgfsetbuttcap%
\pgfsetroundjoin%
\definecolor{currentfill}{rgb}{0.000000,0.000000,0.000000}%
\pgfsetfillcolor{currentfill}%
\pgfsetlinewidth{0.602250pt}%
\definecolor{currentstroke}{rgb}{0.000000,0.000000,0.000000}%
\pgfsetstrokecolor{currentstroke}%
\pgfsetdash{}{0pt}%
\pgfsys@defobject{currentmarker}{\pgfqpoint{0.000000in}{0.000000in}}{\pgfqpoint{0.027778in}{0.000000in}}{%
\pgfpathmoveto{\pgfqpoint{0.000000in}{0.000000in}}%
\pgfpathlineto{\pgfqpoint{0.027778in}{0.000000in}}%
\pgfusepath{stroke,fill}%
}%
\begin{pgfscope}%
\pgfsys@transformshift{4.672917in}{3.183145in}%
\pgfsys@useobject{currentmarker}{}%
\end{pgfscope}%
\end{pgfscope}%
\begin{pgfscope}%
\pgfpathrectangle{\pgfqpoint{0.781944in}{0.552778in}}{\pgfqpoint{3.890972in}{3.248611in}}%
\pgfusepath{clip}%
\pgfsetrectcap%
\pgfsetroundjoin%
\pgfsetlinewidth{0.803000pt}%
\definecolor{currentstroke}{rgb}{0.690196,0.690196,0.690196}%
\pgfsetstrokecolor{currentstroke}%
\pgfsetstrokeopacity{0.300000}%
\pgfsetdash{}{0pt}%
\pgfpathmoveto{\pgfqpoint{0.781944in}{3.226266in}}%
\pgfpathlineto{\pgfqpoint{4.672917in}{3.226266in}}%
\pgfusepath{stroke}%
\end{pgfscope}%
\begin{pgfscope}%
\pgfsetbuttcap%
\pgfsetroundjoin%
\definecolor{currentfill}{rgb}{0.000000,0.000000,0.000000}%
\pgfsetfillcolor{currentfill}%
\pgfsetlinewidth{0.602250pt}%
\definecolor{currentstroke}{rgb}{0.000000,0.000000,0.000000}%
\pgfsetstrokecolor{currentstroke}%
\pgfsetdash{}{0pt}%
\pgfsys@defobject{currentmarker}{\pgfqpoint{-0.027778in}{0.000000in}}{\pgfqpoint{0.000000in}{0.000000in}}{%
\pgfpathmoveto{\pgfqpoint{0.000000in}{0.000000in}}%
\pgfpathlineto{\pgfqpoint{-0.027778in}{0.000000in}}%
\pgfusepath{stroke,fill}%
}%
\begin{pgfscope}%
\pgfsys@transformshift{0.781944in}{3.226266in}%
\pgfsys@useobject{currentmarker}{}%
\end{pgfscope}%
\end{pgfscope}%
\begin{pgfscope}%
\pgfsetbuttcap%
\pgfsetroundjoin%
\definecolor{currentfill}{rgb}{0.000000,0.000000,0.000000}%
\pgfsetfillcolor{currentfill}%
\pgfsetlinewidth{0.602250pt}%
\definecolor{currentstroke}{rgb}{0.000000,0.000000,0.000000}%
\pgfsetstrokecolor{currentstroke}%
\pgfsetdash{}{0pt}%
\pgfsys@defobject{currentmarker}{\pgfqpoint{0.000000in}{0.000000in}}{\pgfqpoint{0.027778in}{0.000000in}}{%
\pgfpathmoveto{\pgfqpoint{0.000000in}{0.000000in}}%
\pgfpathlineto{\pgfqpoint{0.027778in}{0.000000in}}%
\pgfusepath{stroke,fill}%
}%
\begin{pgfscope}%
\pgfsys@transformshift{4.672917in}{3.226266in}%
\pgfsys@useobject{currentmarker}{}%
\end{pgfscope}%
\end{pgfscope}%
\begin{pgfscope}%
\pgfpathrectangle{\pgfqpoint{0.781944in}{0.552778in}}{\pgfqpoint{3.890972in}{3.248611in}}%
\pgfusepath{clip}%
\pgfsetrectcap%
\pgfsetroundjoin%
\pgfsetlinewidth{0.803000pt}%
\definecolor{currentstroke}{rgb}{0.690196,0.690196,0.690196}%
\pgfsetstrokecolor{currentstroke}%
\pgfsetstrokeopacity{0.300000}%
\pgfsetdash{}{0pt}%
\pgfpathmoveto{\pgfqpoint{0.781944in}{3.269386in}}%
\pgfpathlineto{\pgfqpoint{4.672917in}{3.269386in}}%
\pgfusepath{stroke}%
\end{pgfscope}%
\begin{pgfscope}%
\pgfsetbuttcap%
\pgfsetroundjoin%
\definecolor{currentfill}{rgb}{0.000000,0.000000,0.000000}%
\pgfsetfillcolor{currentfill}%
\pgfsetlinewidth{0.602250pt}%
\definecolor{currentstroke}{rgb}{0.000000,0.000000,0.000000}%
\pgfsetstrokecolor{currentstroke}%
\pgfsetdash{}{0pt}%
\pgfsys@defobject{currentmarker}{\pgfqpoint{-0.027778in}{0.000000in}}{\pgfqpoint{0.000000in}{0.000000in}}{%
\pgfpathmoveto{\pgfqpoint{0.000000in}{0.000000in}}%
\pgfpathlineto{\pgfqpoint{-0.027778in}{0.000000in}}%
\pgfusepath{stroke,fill}%
}%
\begin{pgfscope}%
\pgfsys@transformshift{0.781944in}{3.269386in}%
\pgfsys@useobject{currentmarker}{}%
\end{pgfscope}%
\end{pgfscope}%
\begin{pgfscope}%
\pgfsetbuttcap%
\pgfsetroundjoin%
\definecolor{currentfill}{rgb}{0.000000,0.000000,0.000000}%
\pgfsetfillcolor{currentfill}%
\pgfsetlinewidth{0.602250pt}%
\definecolor{currentstroke}{rgb}{0.000000,0.000000,0.000000}%
\pgfsetstrokecolor{currentstroke}%
\pgfsetdash{}{0pt}%
\pgfsys@defobject{currentmarker}{\pgfqpoint{0.000000in}{0.000000in}}{\pgfqpoint{0.027778in}{0.000000in}}{%
\pgfpathmoveto{\pgfqpoint{0.000000in}{0.000000in}}%
\pgfpathlineto{\pgfqpoint{0.027778in}{0.000000in}}%
\pgfusepath{stroke,fill}%
}%
\begin{pgfscope}%
\pgfsys@transformshift{4.672917in}{3.269386in}%
\pgfsys@useobject{currentmarker}{}%
\end{pgfscope}%
\end{pgfscope}%
\begin{pgfscope}%
\pgfpathrectangle{\pgfqpoint{0.781944in}{0.552778in}}{\pgfqpoint{3.890972in}{3.248611in}}%
\pgfusepath{clip}%
\pgfsetrectcap%
\pgfsetroundjoin%
\pgfsetlinewidth{0.803000pt}%
\definecolor{currentstroke}{rgb}{0.690196,0.690196,0.690196}%
\pgfsetstrokecolor{currentstroke}%
\pgfsetstrokeopacity{0.300000}%
\pgfsetdash{}{0pt}%
\pgfpathmoveto{\pgfqpoint{0.781944in}{3.312507in}}%
\pgfpathlineto{\pgfqpoint{4.672917in}{3.312507in}}%
\pgfusepath{stroke}%
\end{pgfscope}%
\begin{pgfscope}%
\pgfsetbuttcap%
\pgfsetroundjoin%
\definecolor{currentfill}{rgb}{0.000000,0.000000,0.000000}%
\pgfsetfillcolor{currentfill}%
\pgfsetlinewidth{0.602250pt}%
\definecolor{currentstroke}{rgb}{0.000000,0.000000,0.000000}%
\pgfsetstrokecolor{currentstroke}%
\pgfsetdash{}{0pt}%
\pgfsys@defobject{currentmarker}{\pgfqpoint{-0.027778in}{0.000000in}}{\pgfqpoint{0.000000in}{0.000000in}}{%
\pgfpathmoveto{\pgfqpoint{0.000000in}{0.000000in}}%
\pgfpathlineto{\pgfqpoint{-0.027778in}{0.000000in}}%
\pgfusepath{stroke,fill}%
}%
\begin{pgfscope}%
\pgfsys@transformshift{0.781944in}{3.312507in}%
\pgfsys@useobject{currentmarker}{}%
\end{pgfscope}%
\end{pgfscope}%
\begin{pgfscope}%
\pgfsetbuttcap%
\pgfsetroundjoin%
\definecolor{currentfill}{rgb}{0.000000,0.000000,0.000000}%
\pgfsetfillcolor{currentfill}%
\pgfsetlinewidth{0.602250pt}%
\definecolor{currentstroke}{rgb}{0.000000,0.000000,0.000000}%
\pgfsetstrokecolor{currentstroke}%
\pgfsetdash{}{0pt}%
\pgfsys@defobject{currentmarker}{\pgfqpoint{0.000000in}{0.000000in}}{\pgfqpoint{0.027778in}{0.000000in}}{%
\pgfpathmoveto{\pgfqpoint{0.000000in}{0.000000in}}%
\pgfpathlineto{\pgfqpoint{0.027778in}{0.000000in}}%
\pgfusepath{stroke,fill}%
}%
\begin{pgfscope}%
\pgfsys@transformshift{4.672917in}{3.312507in}%
\pgfsys@useobject{currentmarker}{}%
\end{pgfscope}%
\end{pgfscope}%
\begin{pgfscope}%
\pgfpathrectangle{\pgfqpoint{0.781944in}{0.552778in}}{\pgfqpoint{3.890972in}{3.248611in}}%
\pgfusepath{clip}%
\pgfsetrectcap%
\pgfsetroundjoin%
\pgfsetlinewidth{0.803000pt}%
\definecolor{currentstroke}{rgb}{0.690196,0.690196,0.690196}%
\pgfsetstrokecolor{currentstroke}%
\pgfsetstrokeopacity{0.300000}%
\pgfsetdash{}{0pt}%
\pgfpathmoveto{\pgfqpoint{0.781944in}{3.355628in}}%
\pgfpathlineto{\pgfqpoint{4.672917in}{3.355628in}}%
\pgfusepath{stroke}%
\end{pgfscope}%
\begin{pgfscope}%
\pgfsetbuttcap%
\pgfsetroundjoin%
\definecolor{currentfill}{rgb}{0.000000,0.000000,0.000000}%
\pgfsetfillcolor{currentfill}%
\pgfsetlinewidth{0.602250pt}%
\definecolor{currentstroke}{rgb}{0.000000,0.000000,0.000000}%
\pgfsetstrokecolor{currentstroke}%
\pgfsetdash{}{0pt}%
\pgfsys@defobject{currentmarker}{\pgfqpoint{-0.027778in}{0.000000in}}{\pgfqpoint{0.000000in}{0.000000in}}{%
\pgfpathmoveto{\pgfqpoint{0.000000in}{0.000000in}}%
\pgfpathlineto{\pgfqpoint{-0.027778in}{0.000000in}}%
\pgfusepath{stroke,fill}%
}%
\begin{pgfscope}%
\pgfsys@transformshift{0.781944in}{3.355628in}%
\pgfsys@useobject{currentmarker}{}%
\end{pgfscope}%
\end{pgfscope}%
\begin{pgfscope}%
\pgfsetbuttcap%
\pgfsetroundjoin%
\definecolor{currentfill}{rgb}{0.000000,0.000000,0.000000}%
\pgfsetfillcolor{currentfill}%
\pgfsetlinewidth{0.602250pt}%
\definecolor{currentstroke}{rgb}{0.000000,0.000000,0.000000}%
\pgfsetstrokecolor{currentstroke}%
\pgfsetdash{}{0pt}%
\pgfsys@defobject{currentmarker}{\pgfqpoint{0.000000in}{0.000000in}}{\pgfqpoint{0.027778in}{0.000000in}}{%
\pgfpathmoveto{\pgfqpoint{0.000000in}{0.000000in}}%
\pgfpathlineto{\pgfqpoint{0.027778in}{0.000000in}}%
\pgfusepath{stroke,fill}%
}%
\begin{pgfscope}%
\pgfsys@transformshift{4.672917in}{3.355628in}%
\pgfsys@useobject{currentmarker}{}%
\end{pgfscope}%
\end{pgfscope}%
\begin{pgfscope}%
\pgfpathrectangle{\pgfqpoint{0.781944in}{0.552778in}}{\pgfqpoint{3.890972in}{3.248611in}}%
\pgfusepath{clip}%
\pgfsetrectcap%
\pgfsetroundjoin%
\pgfsetlinewidth{0.803000pt}%
\definecolor{currentstroke}{rgb}{0.690196,0.690196,0.690196}%
\pgfsetstrokecolor{currentstroke}%
\pgfsetstrokeopacity{0.300000}%
\pgfsetdash{}{0pt}%
\pgfpathmoveto{\pgfqpoint{0.781944in}{3.398749in}}%
\pgfpathlineto{\pgfqpoint{4.672917in}{3.398749in}}%
\pgfusepath{stroke}%
\end{pgfscope}%
\begin{pgfscope}%
\pgfsetbuttcap%
\pgfsetroundjoin%
\definecolor{currentfill}{rgb}{0.000000,0.000000,0.000000}%
\pgfsetfillcolor{currentfill}%
\pgfsetlinewidth{0.602250pt}%
\definecolor{currentstroke}{rgb}{0.000000,0.000000,0.000000}%
\pgfsetstrokecolor{currentstroke}%
\pgfsetdash{}{0pt}%
\pgfsys@defobject{currentmarker}{\pgfqpoint{-0.027778in}{0.000000in}}{\pgfqpoint{0.000000in}{0.000000in}}{%
\pgfpathmoveto{\pgfqpoint{0.000000in}{0.000000in}}%
\pgfpathlineto{\pgfqpoint{-0.027778in}{0.000000in}}%
\pgfusepath{stroke,fill}%
}%
\begin{pgfscope}%
\pgfsys@transformshift{0.781944in}{3.398749in}%
\pgfsys@useobject{currentmarker}{}%
\end{pgfscope}%
\end{pgfscope}%
\begin{pgfscope}%
\pgfsetbuttcap%
\pgfsetroundjoin%
\definecolor{currentfill}{rgb}{0.000000,0.000000,0.000000}%
\pgfsetfillcolor{currentfill}%
\pgfsetlinewidth{0.602250pt}%
\definecolor{currentstroke}{rgb}{0.000000,0.000000,0.000000}%
\pgfsetstrokecolor{currentstroke}%
\pgfsetdash{}{0pt}%
\pgfsys@defobject{currentmarker}{\pgfqpoint{0.000000in}{0.000000in}}{\pgfqpoint{0.027778in}{0.000000in}}{%
\pgfpathmoveto{\pgfqpoint{0.000000in}{0.000000in}}%
\pgfpathlineto{\pgfqpoint{0.027778in}{0.000000in}}%
\pgfusepath{stroke,fill}%
}%
\begin{pgfscope}%
\pgfsys@transformshift{4.672917in}{3.398749in}%
\pgfsys@useobject{currentmarker}{}%
\end{pgfscope}%
\end{pgfscope}%
\begin{pgfscope}%
\pgfpathrectangle{\pgfqpoint{0.781944in}{0.552778in}}{\pgfqpoint{3.890972in}{3.248611in}}%
\pgfusepath{clip}%
\pgfsetrectcap%
\pgfsetroundjoin%
\pgfsetlinewidth{0.803000pt}%
\definecolor{currentstroke}{rgb}{0.690196,0.690196,0.690196}%
\pgfsetstrokecolor{currentstroke}%
\pgfsetstrokeopacity{0.300000}%
\pgfsetdash{}{0pt}%
\pgfpathmoveto{\pgfqpoint{0.781944in}{3.441869in}}%
\pgfpathlineto{\pgfqpoint{4.672917in}{3.441869in}}%
\pgfusepath{stroke}%
\end{pgfscope}%
\begin{pgfscope}%
\pgfsetbuttcap%
\pgfsetroundjoin%
\definecolor{currentfill}{rgb}{0.000000,0.000000,0.000000}%
\pgfsetfillcolor{currentfill}%
\pgfsetlinewidth{0.602250pt}%
\definecolor{currentstroke}{rgb}{0.000000,0.000000,0.000000}%
\pgfsetstrokecolor{currentstroke}%
\pgfsetdash{}{0pt}%
\pgfsys@defobject{currentmarker}{\pgfqpoint{-0.027778in}{0.000000in}}{\pgfqpoint{0.000000in}{0.000000in}}{%
\pgfpathmoveto{\pgfqpoint{0.000000in}{0.000000in}}%
\pgfpathlineto{\pgfqpoint{-0.027778in}{0.000000in}}%
\pgfusepath{stroke,fill}%
}%
\begin{pgfscope}%
\pgfsys@transformshift{0.781944in}{3.441869in}%
\pgfsys@useobject{currentmarker}{}%
\end{pgfscope}%
\end{pgfscope}%
\begin{pgfscope}%
\pgfsetbuttcap%
\pgfsetroundjoin%
\definecolor{currentfill}{rgb}{0.000000,0.000000,0.000000}%
\pgfsetfillcolor{currentfill}%
\pgfsetlinewidth{0.602250pt}%
\definecolor{currentstroke}{rgb}{0.000000,0.000000,0.000000}%
\pgfsetstrokecolor{currentstroke}%
\pgfsetdash{}{0pt}%
\pgfsys@defobject{currentmarker}{\pgfqpoint{0.000000in}{0.000000in}}{\pgfqpoint{0.027778in}{0.000000in}}{%
\pgfpathmoveto{\pgfqpoint{0.000000in}{0.000000in}}%
\pgfpathlineto{\pgfqpoint{0.027778in}{0.000000in}}%
\pgfusepath{stroke,fill}%
}%
\begin{pgfscope}%
\pgfsys@transformshift{4.672917in}{3.441869in}%
\pgfsys@useobject{currentmarker}{}%
\end{pgfscope}%
\end{pgfscope}%
\begin{pgfscope}%
\pgfpathrectangle{\pgfqpoint{0.781944in}{0.552778in}}{\pgfqpoint{3.890972in}{3.248611in}}%
\pgfusepath{clip}%
\pgfsetrectcap%
\pgfsetroundjoin%
\pgfsetlinewidth{0.803000pt}%
\definecolor{currentstroke}{rgb}{0.690196,0.690196,0.690196}%
\pgfsetstrokecolor{currentstroke}%
\pgfsetstrokeopacity{0.300000}%
\pgfsetdash{}{0pt}%
\pgfpathmoveto{\pgfqpoint{0.781944in}{3.484990in}}%
\pgfpathlineto{\pgfqpoint{4.672917in}{3.484990in}}%
\pgfusepath{stroke}%
\end{pgfscope}%
\begin{pgfscope}%
\pgfsetbuttcap%
\pgfsetroundjoin%
\definecolor{currentfill}{rgb}{0.000000,0.000000,0.000000}%
\pgfsetfillcolor{currentfill}%
\pgfsetlinewidth{0.602250pt}%
\definecolor{currentstroke}{rgb}{0.000000,0.000000,0.000000}%
\pgfsetstrokecolor{currentstroke}%
\pgfsetdash{}{0pt}%
\pgfsys@defobject{currentmarker}{\pgfqpoint{-0.027778in}{0.000000in}}{\pgfqpoint{0.000000in}{0.000000in}}{%
\pgfpathmoveto{\pgfqpoint{0.000000in}{0.000000in}}%
\pgfpathlineto{\pgfqpoint{-0.027778in}{0.000000in}}%
\pgfusepath{stroke,fill}%
}%
\begin{pgfscope}%
\pgfsys@transformshift{0.781944in}{3.484990in}%
\pgfsys@useobject{currentmarker}{}%
\end{pgfscope}%
\end{pgfscope}%
\begin{pgfscope}%
\pgfsetbuttcap%
\pgfsetroundjoin%
\definecolor{currentfill}{rgb}{0.000000,0.000000,0.000000}%
\pgfsetfillcolor{currentfill}%
\pgfsetlinewidth{0.602250pt}%
\definecolor{currentstroke}{rgb}{0.000000,0.000000,0.000000}%
\pgfsetstrokecolor{currentstroke}%
\pgfsetdash{}{0pt}%
\pgfsys@defobject{currentmarker}{\pgfqpoint{0.000000in}{0.000000in}}{\pgfqpoint{0.027778in}{0.000000in}}{%
\pgfpathmoveto{\pgfqpoint{0.000000in}{0.000000in}}%
\pgfpathlineto{\pgfqpoint{0.027778in}{0.000000in}}%
\pgfusepath{stroke,fill}%
}%
\begin{pgfscope}%
\pgfsys@transformshift{4.672917in}{3.484990in}%
\pgfsys@useobject{currentmarker}{}%
\end{pgfscope}%
\end{pgfscope}%
\begin{pgfscope}%
\pgfpathrectangle{\pgfqpoint{0.781944in}{0.552778in}}{\pgfqpoint{3.890972in}{3.248611in}}%
\pgfusepath{clip}%
\pgfsetrectcap%
\pgfsetroundjoin%
\pgfsetlinewidth{0.803000pt}%
\definecolor{currentstroke}{rgb}{0.690196,0.690196,0.690196}%
\pgfsetstrokecolor{currentstroke}%
\pgfsetstrokeopacity{0.300000}%
\pgfsetdash{}{0pt}%
\pgfpathmoveto{\pgfqpoint{0.781944in}{3.528111in}}%
\pgfpathlineto{\pgfqpoint{4.672917in}{3.528111in}}%
\pgfusepath{stroke}%
\end{pgfscope}%
\begin{pgfscope}%
\pgfsetbuttcap%
\pgfsetroundjoin%
\definecolor{currentfill}{rgb}{0.000000,0.000000,0.000000}%
\pgfsetfillcolor{currentfill}%
\pgfsetlinewidth{0.602250pt}%
\definecolor{currentstroke}{rgb}{0.000000,0.000000,0.000000}%
\pgfsetstrokecolor{currentstroke}%
\pgfsetdash{}{0pt}%
\pgfsys@defobject{currentmarker}{\pgfqpoint{-0.027778in}{0.000000in}}{\pgfqpoint{0.000000in}{0.000000in}}{%
\pgfpathmoveto{\pgfqpoint{0.000000in}{0.000000in}}%
\pgfpathlineto{\pgfqpoint{-0.027778in}{0.000000in}}%
\pgfusepath{stroke,fill}%
}%
\begin{pgfscope}%
\pgfsys@transformshift{0.781944in}{3.528111in}%
\pgfsys@useobject{currentmarker}{}%
\end{pgfscope}%
\end{pgfscope}%
\begin{pgfscope}%
\pgfsetbuttcap%
\pgfsetroundjoin%
\definecolor{currentfill}{rgb}{0.000000,0.000000,0.000000}%
\pgfsetfillcolor{currentfill}%
\pgfsetlinewidth{0.602250pt}%
\definecolor{currentstroke}{rgb}{0.000000,0.000000,0.000000}%
\pgfsetstrokecolor{currentstroke}%
\pgfsetdash{}{0pt}%
\pgfsys@defobject{currentmarker}{\pgfqpoint{0.000000in}{0.000000in}}{\pgfqpoint{0.027778in}{0.000000in}}{%
\pgfpathmoveto{\pgfqpoint{0.000000in}{0.000000in}}%
\pgfpathlineto{\pgfqpoint{0.027778in}{0.000000in}}%
\pgfusepath{stroke,fill}%
}%
\begin{pgfscope}%
\pgfsys@transformshift{4.672917in}{3.528111in}%
\pgfsys@useobject{currentmarker}{}%
\end{pgfscope}%
\end{pgfscope}%
\begin{pgfscope}%
\pgfpathrectangle{\pgfqpoint{0.781944in}{0.552778in}}{\pgfqpoint{3.890972in}{3.248611in}}%
\pgfusepath{clip}%
\pgfsetrectcap%
\pgfsetroundjoin%
\pgfsetlinewidth{0.803000pt}%
\definecolor{currentstroke}{rgb}{0.690196,0.690196,0.690196}%
\pgfsetstrokecolor{currentstroke}%
\pgfsetstrokeopacity{0.300000}%
\pgfsetdash{}{0pt}%
\pgfpathmoveto{\pgfqpoint{0.781944in}{3.614353in}}%
\pgfpathlineto{\pgfqpoint{4.672917in}{3.614353in}}%
\pgfusepath{stroke}%
\end{pgfscope}%
\begin{pgfscope}%
\pgfsetbuttcap%
\pgfsetroundjoin%
\definecolor{currentfill}{rgb}{0.000000,0.000000,0.000000}%
\pgfsetfillcolor{currentfill}%
\pgfsetlinewidth{0.602250pt}%
\definecolor{currentstroke}{rgb}{0.000000,0.000000,0.000000}%
\pgfsetstrokecolor{currentstroke}%
\pgfsetdash{}{0pt}%
\pgfsys@defobject{currentmarker}{\pgfqpoint{-0.027778in}{0.000000in}}{\pgfqpoint{0.000000in}{0.000000in}}{%
\pgfpathmoveto{\pgfqpoint{0.000000in}{0.000000in}}%
\pgfpathlineto{\pgfqpoint{-0.027778in}{0.000000in}}%
\pgfusepath{stroke,fill}%
}%
\begin{pgfscope}%
\pgfsys@transformshift{0.781944in}{3.614353in}%
\pgfsys@useobject{currentmarker}{}%
\end{pgfscope}%
\end{pgfscope}%
\begin{pgfscope}%
\pgfsetbuttcap%
\pgfsetroundjoin%
\definecolor{currentfill}{rgb}{0.000000,0.000000,0.000000}%
\pgfsetfillcolor{currentfill}%
\pgfsetlinewidth{0.602250pt}%
\definecolor{currentstroke}{rgb}{0.000000,0.000000,0.000000}%
\pgfsetstrokecolor{currentstroke}%
\pgfsetdash{}{0pt}%
\pgfsys@defobject{currentmarker}{\pgfqpoint{0.000000in}{0.000000in}}{\pgfqpoint{0.027778in}{0.000000in}}{%
\pgfpathmoveto{\pgfqpoint{0.000000in}{0.000000in}}%
\pgfpathlineto{\pgfqpoint{0.027778in}{0.000000in}}%
\pgfusepath{stroke,fill}%
}%
\begin{pgfscope}%
\pgfsys@transformshift{4.672917in}{3.614353in}%
\pgfsys@useobject{currentmarker}{}%
\end{pgfscope}%
\end{pgfscope}%
\begin{pgfscope}%
\pgfpathrectangle{\pgfqpoint{0.781944in}{0.552778in}}{\pgfqpoint{3.890972in}{3.248611in}}%
\pgfusepath{clip}%
\pgfsetrectcap%
\pgfsetroundjoin%
\pgfsetlinewidth{0.803000pt}%
\definecolor{currentstroke}{rgb}{0.690196,0.690196,0.690196}%
\pgfsetstrokecolor{currentstroke}%
\pgfsetstrokeopacity{0.300000}%
\pgfsetdash{}{0pt}%
\pgfpathmoveto{\pgfqpoint{0.781944in}{3.657473in}}%
\pgfpathlineto{\pgfqpoint{4.672917in}{3.657473in}}%
\pgfusepath{stroke}%
\end{pgfscope}%
\begin{pgfscope}%
\pgfsetbuttcap%
\pgfsetroundjoin%
\definecolor{currentfill}{rgb}{0.000000,0.000000,0.000000}%
\pgfsetfillcolor{currentfill}%
\pgfsetlinewidth{0.602250pt}%
\definecolor{currentstroke}{rgb}{0.000000,0.000000,0.000000}%
\pgfsetstrokecolor{currentstroke}%
\pgfsetdash{}{0pt}%
\pgfsys@defobject{currentmarker}{\pgfqpoint{-0.027778in}{0.000000in}}{\pgfqpoint{0.000000in}{0.000000in}}{%
\pgfpathmoveto{\pgfqpoint{0.000000in}{0.000000in}}%
\pgfpathlineto{\pgfqpoint{-0.027778in}{0.000000in}}%
\pgfusepath{stroke,fill}%
}%
\begin{pgfscope}%
\pgfsys@transformshift{0.781944in}{3.657473in}%
\pgfsys@useobject{currentmarker}{}%
\end{pgfscope}%
\end{pgfscope}%
\begin{pgfscope}%
\pgfsetbuttcap%
\pgfsetroundjoin%
\definecolor{currentfill}{rgb}{0.000000,0.000000,0.000000}%
\pgfsetfillcolor{currentfill}%
\pgfsetlinewidth{0.602250pt}%
\definecolor{currentstroke}{rgb}{0.000000,0.000000,0.000000}%
\pgfsetstrokecolor{currentstroke}%
\pgfsetdash{}{0pt}%
\pgfsys@defobject{currentmarker}{\pgfqpoint{0.000000in}{0.000000in}}{\pgfqpoint{0.027778in}{0.000000in}}{%
\pgfpathmoveto{\pgfqpoint{0.000000in}{0.000000in}}%
\pgfpathlineto{\pgfqpoint{0.027778in}{0.000000in}}%
\pgfusepath{stroke,fill}%
}%
\begin{pgfscope}%
\pgfsys@transformshift{4.672917in}{3.657473in}%
\pgfsys@useobject{currentmarker}{}%
\end{pgfscope}%
\end{pgfscope}%
\begin{pgfscope}%
\pgfpathrectangle{\pgfqpoint{0.781944in}{0.552778in}}{\pgfqpoint{3.890972in}{3.248611in}}%
\pgfusepath{clip}%
\pgfsetrectcap%
\pgfsetroundjoin%
\pgfsetlinewidth{0.803000pt}%
\definecolor{currentstroke}{rgb}{0.690196,0.690196,0.690196}%
\pgfsetstrokecolor{currentstroke}%
\pgfsetstrokeopacity{0.300000}%
\pgfsetdash{}{0pt}%
\pgfpathmoveto{\pgfqpoint{0.781944in}{3.700594in}}%
\pgfpathlineto{\pgfqpoint{4.672917in}{3.700594in}}%
\pgfusepath{stroke}%
\end{pgfscope}%
\begin{pgfscope}%
\pgfsetbuttcap%
\pgfsetroundjoin%
\definecolor{currentfill}{rgb}{0.000000,0.000000,0.000000}%
\pgfsetfillcolor{currentfill}%
\pgfsetlinewidth{0.602250pt}%
\definecolor{currentstroke}{rgb}{0.000000,0.000000,0.000000}%
\pgfsetstrokecolor{currentstroke}%
\pgfsetdash{}{0pt}%
\pgfsys@defobject{currentmarker}{\pgfqpoint{-0.027778in}{0.000000in}}{\pgfqpoint{0.000000in}{0.000000in}}{%
\pgfpathmoveto{\pgfqpoint{0.000000in}{0.000000in}}%
\pgfpathlineto{\pgfqpoint{-0.027778in}{0.000000in}}%
\pgfusepath{stroke,fill}%
}%
\begin{pgfscope}%
\pgfsys@transformshift{0.781944in}{3.700594in}%
\pgfsys@useobject{currentmarker}{}%
\end{pgfscope}%
\end{pgfscope}%
\begin{pgfscope}%
\pgfsetbuttcap%
\pgfsetroundjoin%
\definecolor{currentfill}{rgb}{0.000000,0.000000,0.000000}%
\pgfsetfillcolor{currentfill}%
\pgfsetlinewidth{0.602250pt}%
\definecolor{currentstroke}{rgb}{0.000000,0.000000,0.000000}%
\pgfsetstrokecolor{currentstroke}%
\pgfsetdash{}{0pt}%
\pgfsys@defobject{currentmarker}{\pgfqpoint{0.000000in}{0.000000in}}{\pgfqpoint{0.027778in}{0.000000in}}{%
\pgfpathmoveto{\pgfqpoint{0.000000in}{0.000000in}}%
\pgfpathlineto{\pgfqpoint{0.027778in}{0.000000in}}%
\pgfusepath{stroke,fill}%
}%
\begin{pgfscope}%
\pgfsys@transformshift{4.672917in}{3.700594in}%
\pgfsys@useobject{currentmarker}{}%
\end{pgfscope}%
\end{pgfscope}%
\begin{pgfscope}%
\pgfpathrectangle{\pgfqpoint{0.781944in}{0.552778in}}{\pgfqpoint{3.890972in}{3.248611in}}%
\pgfusepath{clip}%
\pgfsetrectcap%
\pgfsetroundjoin%
\pgfsetlinewidth{0.803000pt}%
\definecolor{currentstroke}{rgb}{0.690196,0.690196,0.690196}%
\pgfsetstrokecolor{currentstroke}%
\pgfsetstrokeopacity{0.300000}%
\pgfsetdash{}{0pt}%
\pgfpathmoveto{\pgfqpoint{0.781944in}{3.743715in}}%
\pgfpathlineto{\pgfqpoint{4.672917in}{3.743715in}}%
\pgfusepath{stroke}%
\end{pgfscope}%
\begin{pgfscope}%
\pgfsetbuttcap%
\pgfsetroundjoin%
\definecolor{currentfill}{rgb}{0.000000,0.000000,0.000000}%
\pgfsetfillcolor{currentfill}%
\pgfsetlinewidth{0.602250pt}%
\definecolor{currentstroke}{rgb}{0.000000,0.000000,0.000000}%
\pgfsetstrokecolor{currentstroke}%
\pgfsetdash{}{0pt}%
\pgfsys@defobject{currentmarker}{\pgfqpoint{-0.027778in}{0.000000in}}{\pgfqpoint{0.000000in}{0.000000in}}{%
\pgfpathmoveto{\pgfqpoint{0.000000in}{0.000000in}}%
\pgfpathlineto{\pgfqpoint{-0.027778in}{0.000000in}}%
\pgfusepath{stroke,fill}%
}%
\begin{pgfscope}%
\pgfsys@transformshift{0.781944in}{3.743715in}%
\pgfsys@useobject{currentmarker}{}%
\end{pgfscope}%
\end{pgfscope}%
\begin{pgfscope}%
\pgfsetbuttcap%
\pgfsetroundjoin%
\definecolor{currentfill}{rgb}{0.000000,0.000000,0.000000}%
\pgfsetfillcolor{currentfill}%
\pgfsetlinewidth{0.602250pt}%
\definecolor{currentstroke}{rgb}{0.000000,0.000000,0.000000}%
\pgfsetstrokecolor{currentstroke}%
\pgfsetdash{}{0pt}%
\pgfsys@defobject{currentmarker}{\pgfqpoint{0.000000in}{0.000000in}}{\pgfqpoint{0.027778in}{0.000000in}}{%
\pgfpathmoveto{\pgfqpoint{0.000000in}{0.000000in}}%
\pgfpathlineto{\pgfqpoint{0.027778in}{0.000000in}}%
\pgfusepath{stroke,fill}%
}%
\begin{pgfscope}%
\pgfsys@transformshift{4.672917in}{3.743715in}%
\pgfsys@useobject{currentmarker}{}%
\end{pgfscope}%
\end{pgfscope}%
\begin{pgfscope}%
\pgfpathrectangle{\pgfqpoint{0.781944in}{0.552778in}}{\pgfqpoint{3.890972in}{3.248611in}}%
\pgfusepath{clip}%
\pgfsetrectcap%
\pgfsetroundjoin%
\pgfsetlinewidth{0.803000pt}%
\definecolor{currentstroke}{rgb}{0.690196,0.690196,0.690196}%
\pgfsetstrokecolor{currentstroke}%
\pgfsetstrokeopacity{0.300000}%
\pgfsetdash{}{0pt}%
\pgfpathmoveto{\pgfqpoint{0.781944in}{3.786836in}}%
\pgfpathlineto{\pgfqpoint{4.672917in}{3.786836in}}%
\pgfusepath{stroke}%
\end{pgfscope}%
\begin{pgfscope}%
\pgfsetbuttcap%
\pgfsetroundjoin%
\definecolor{currentfill}{rgb}{0.000000,0.000000,0.000000}%
\pgfsetfillcolor{currentfill}%
\pgfsetlinewidth{0.602250pt}%
\definecolor{currentstroke}{rgb}{0.000000,0.000000,0.000000}%
\pgfsetstrokecolor{currentstroke}%
\pgfsetdash{}{0pt}%
\pgfsys@defobject{currentmarker}{\pgfqpoint{-0.027778in}{0.000000in}}{\pgfqpoint{0.000000in}{0.000000in}}{%
\pgfpathmoveto{\pgfqpoint{0.000000in}{0.000000in}}%
\pgfpathlineto{\pgfqpoint{-0.027778in}{0.000000in}}%
\pgfusepath{stroke,fill}%
}%
\begin{pgfscope}%
\pgfsys@transformshift{0.781944in}{3.786836in}%
\pgfsys@useobject{currentmarker}{}%
\end{pgfscope}%
\end{pgfscope}%
\begin{pgfscope}%
\pgfsetbuttcap%
\pgfsetroundjoin%
\definecolor{currentfill}{rgb}{0.000000,0.000000,0.000000}%
\pgfsetfillcolor{currentfill}%
\pgfsetlinewidth{0.602250pt}%
\definecolor{currentstroke}{rgb}{0.000000,0.000000,0.000000}%
\pgfsetstrokecolor{currentstroke}%
\pgfsetdash{}{0pt}%
\pgfsys@defobject{currentmarker}{\pgfqpoint{0.000000in}{0.000000in}}{\pgfqpoint{0.027778in}{0.000000in}}{%
\pgfpathmoveto{\pgfqpoint{0.000000in}{0.000000in}}%
\pgfpathlineto{\pgfqpoint{0.027778in}{0.000000in}}%
\pgfusepath{stroke,fill}%
}%
\begin{pgfscope}%
\pgfsys@transformshift{4.672917in}{3.786836in}%
\pgfsys@useobject{currentmarker}{}%
\end{pgfscope}%
\end{pgfscope}%
\begin{pgfscope}%
\definecolor{textcolor}{rgb}{0.000000,0.000000,0.000000}%
\pgfsetstrokecolor{textcolor}%
\pgfsetfillcolor{textcolor}%
\pgftext[x=0.351389in,y=2.177083in,,bottom,rotate=90.000000]{\color{textcolor}\rmfamily\fontsize{10.000000}{12.000000}\selectfont Ereignisszahl}%
\end{pgfscope}%
\begin{pgfscope}%
\pgfpathrectangle{\pgfqpoint{0.781944in}{0.552778in}}{\pgfqpoint{3.890972in}{3.248611in}}%
\pgfusepath{clip}%
\pgfsetrectcap%
\pgfsetroundjoin%
\pgfsetlinewidth{1.505625pt}%
\definecolor{currentstroke}{rgb}{0.121569,0.466667,0.705882}%
\pgfsetstrokecolor{currentstroke}%
\pgfsetdash{}{0pt}%
\pgfpathmoveto{\pgfqpoint{0.781944in}{0.553856in}}%
\pgfpathlineto{\pgfqpoint{0.781944in}{0.560324in}}%
\pgfpathlineto{\pgfqpoint{0.783242in}{0.552778in}}%
\pgfpathlineto{\pgfqpoint{0.993424in}{0.552778in}}%
\pgfpathlineto{\pgfqpoint{0.994722in}{0.571104in}}%
\pgfpathlineto{\pgfqpoint{0.996019in}{0.571104in}}%
\pgfpathlineto{\pgfqpoint{0.996019in}{0.559246in}}%
\pgfpathlineto{\pgfqpoint{0.997317in}{0.566792in}}%
\pgfpathlineto{\pgfqpoint{0.999912in}{0.565714in}}%
\pgfpathlineto{\pgfqpoint{0.999912in}{0.564636in}}%
\pgfpathlineto{\pgfqpoint{1.001209in}{0.567870in}}%
\pgfpathlineto{\pgfqpoint{1.002506in}{0.567870in}}%
\pgfpathlineto{\pgfqpoint{1.002506in}{0.565714in}}%
\pgfpathlineto{\pgfqpoint{1.003804in}{0.575416in}}%
\pgfpathlineto{\pgfqpoint{1.005101in}{0.575416in}}%
\pgfpathlineto{\pgfqpoint{1.005101in}{0.561402in}}%
\pgfpathlineto{\pgfqpoint{1.006399in}{0.564636in}}%
\pgfpathlineto{\pgfqpoint{1.008994in}{0.564636in}}%
\pgfpathlineto{\pgfqpoint{1.008994in}{0.561402in}}%
\pgfpathlineto{\pgfqpoint{1.010291in}{0.566792in}}%
\pgfpathlineto{\pgfqpoint{1.011588in}{0.566792in}}%
\pgfpathlineto{\pgfqpoint{1.011588in}{0.563558in}}%
\pgfpathlineto{\pgfqpoint{1.012886in}{0.566792in}}%
\pgfpathlineto{\pgfqpoint{1.014183in}{0.566792in}}%
\pgfpathlineto{\pgfqpoint{1.015481in}{0.556012in}}%
\pgfpathlineto{\pgfqpoint{1.016778in}{0.556012in}}%
\pgfpathlineto{\pgfqpoint{1.016778in}{0.566792in}}%
\pgfpathlineto{\pgfqpoint{1.018075in}{0.558168in}}%
\pgfpathlineto{\pgfqpoint{1.020670in}{0.559246in}}%
\pgfpathlineto{\pgfqpoint{1.020670in}{0.565714in}}%
\pgfpathlineto{\pgfqpoint{1.021968in}{0.562480in}}%
\pgfpathlineto{\pgfqpoint{1.023265in}{0.562480in}}%
\pgfpathlineto{\pgfqpoint{1.024563in}{0.567870in}}%
\pgfpathlineto{\pgfqpoint{1.027157in}{0.568948in}}%
\pgfpathlineto{\pgfqpoint{1.027157in}{0.560324in}}%
\pgfpathlineto{\pgfqpoint{1.028455in}{0.566792in}}%
\pgfpathlineto{\pgfqpoint{1.029752in}{0.566792in}}%
\pgfpathlineto{\pgfqpoint{1.029752in}{0.562480in}}%
\pgfpathlineto{\pgfqpoint{1.031050in}{0.563558in}}%
\pgfpathlineto{\pgfqpoint{1.032347in}{0.563558in}}%
\pgfpathlineto{\pgfqpoint{1.032347in}{0.560324in}}%
\pgfpathlineto{\pgfqpoint{1.033645in}{0.560324in}}%
\pgfpathlineto{\pgfqpoint{1.036239in}{0.560324in}}%
\pgfpathlineto{\pgfqpoint{1.037537in}{0.571104in}}%
\pgfpathlineto{\pgfqpoint{1.038834in}{0.571104in}}%
\pgfpathlineto{\pgfqpoint{1.038834in}{0.560324in}}%
\pgfpathlineto{\pgfqpoint{1.040132in}{0.562480in}}%
\pgfpathlineto{\pgfqpoint{1.041429in}{0.562480in}}%
\pgfpathlineto{\pgfqpoint{1.041429in}{0.573260in}}%
\pgfpathlineto{\pgfqpoint{1.042727in}{0.571104in}}%
\pgfpathlineto{\pgfqpoint{1.044024in}{0.571104in}}%
\pgfpathlineto{\pgfqpoint{1.044024in}{0.561402in}}%
\pgfpathlineto{\pgfqpoint{1.045321in}{0.566792in}}%
\pgfpathlineto{\pgfqpoint{1.049214in}{0.566792in}}%
\pgfpathlineto{\pgfqpoint{1.049214in}{0.568948in}}%
\pgfpathlineto{\pgfqpoint{1.050511in}{0.564636in}}%
\pgfpathlineto{\pgfqpoint{1.051808in}{0.564636in}}%
\pgfpathlineto{\pgfqpoint{1.051808in}{0.560324in}}%
\pgfpathlineto{\pgfqpoint{1.053106in}{0.564636in}}%
\pgfpathlineto{\pgfqpoint{1.055701in}{0.565714in}}%
\pgfpathlineto{\pgfqpoint{1.056998in}{0.559246in}}%
\pgfpathlineto{\pgfqpoint{1.058296in}{0.559246in}}%
\pgfpathlineto{\pgfqpoint{1.059593in}{0.566792in}}%
\pgfpathlineto{\pgfqpoint{1.060890in}{0.566792in}}%
\pgfpathlineto{\pgfqpoint{1.060890in}{0.560324in}}%
\pgfpathlineto{\pgfqpoint{1.062188in}{0.571104in}}%
\pgfpathlineto{\pgfqpoint{1.063485in}{0.571104in}}%
\pgfpathlineto{\pgfqpoint{1.063485in}{0.560324in}}%
\pgfpathlineto{\pgfqpoint{1.064783in}{0.565714in}}%
\pgfpathlineto{\pgfqpoint{1.067378in}{0.564636in}}%
\pgfpathlineto{\pgfqpoint{1.067378in}{0.560324in}}%
\pgfpathlineto{\pgfqpoint{1.068675in}{0.567870in}}%
\pgfpathlineto{\pgfqpoint{1.069972in}{0.567870in}}%
\pgfpathlineto{\pgfqpoint{1.069972in}{0.571104in}}%
\pgfpathlineto{\pgfqpoint{1.071270in}{0.564636in}}%
\pgfpathlineto{\pgfqpoint{1.072567in}{0.564636in}}%
\pgfpathlineto{\pgfqpoint{1.072567in}{0.567870in}}%
\pgfpathlineto{\pgfqpoint{1.073865in}{0.566792in}}%
\pgfpathlineto{\pgfqpoint{1.075162in}{0.566792in}}%
\pgfpathlineto{\pgfqpoint{1.075162in}{0.571104in}}%
\pgfpathlineto{\pgfqpoint{1.076460in}{0.562480in}}%
\pgfpathlineto{\pgfqpoint{1.079054in}{0.563558in}}%
\pgfpathlineto{\pgfqpoint{1.080352in}{0.572182in}}%
\pgfpathlineto{\pgfqpoint{1.081649in}{0.572182in}}%
\pgfpathlineto{\pgfqpoint{1.081649in}{0.559246in}}%
\pgfpathlineto{\pgfqpoint{1.082947in}{0.563558in}}%
\pgfpathlineto{\pgfqpoint{1.084244in}{0.563558in}}%
\pgfpathlineto{\pgfqpoint{1.084244in}{0.568948in}}%
\pgfpathlineto{\pgfqpoint{1.085541in}{0.562480in}}%
\pgfpathlineto{\pgfqpoint{1.086839in}{0.562480in}}%
\pgfpathlineto{\pgfqpoint{1.086839in}{0.567870in}}%
\pgfpathlineto{\pgfqpoint{1.088136in}{0.567870in}}%
\pgfpathlineto{\pgfqpoint{1.089434in}{0.567870in}}%
\pgfpathlineto{\pgfqpoint{1.090731in}{0.557090in}}%
\pgfpathlineto{\pgfqpoint{1.092029in}{0.557090in}}%
\pgfpathlineto{\pgfqpoint{1.092029in}{0.567870in}}%
\pgfpathlineto{\pgfqpoint{1.093326in}{0.564636in}}%
\pgfpathlineto{\pgfqpoint{1.094623in}{0.564636in}}%
\pgfpathlineto{\pgfqpoint{1.094623in}{0.561402in}}%
\pgfpathlineto{\pgfqpoint{1.095921in}{0.571104in}}%
\pgfpathlineto{\pgfqpoint{1.097218in}{0.571104in}}%
\pgfpathlineto{\pgfqpoint{1.097218in}{0.563558in}}%
\pgfpathlineto{\pgfqpoint{1.098516in}{0.564636in}}%
\pgfpathlineto{\pgfqpoint{1.099813in}{0.564636in}}%
\pgfpathlineto{\pgfqpoint{1.099813in}{0.568948in}}%
\pgfpathlineto{\pgfqpoint{1.101111in}{0.564636in}}%
\pgfpathlineto{\pgfqpoint{1.102408in}{0.564636in}}%
\pgfpathlineto{\pgfqpoint{1.102408in}{0.560324in}}%
\pgfpathlineto{\pgfqpoint{1.103705in}{0.563558in}}%
\pgfpathlineto{\pgfqpoint{1.105003in}{0.563558in}}%
\pgfpathlineto{\pgfqpoint{1.106300in}{0.566792in}}%
\pgfpathlineto{\pgfqpoint{1.107598in}{0.566792in}}%
\pgfpathlineto{\pgfqpoint{1.107598in}{0.563558in}}%
\pgfpathlineto{\pgfqpoint{1.108895in}{0.568948in}}%
\pgfpathlineto{\pgfqpoint{1.110193in}{0.568948in}}%
\pgfpathlineto{\pgfqpoint{1.110193in}{0.563558in}}%
\pgfpathlineto{\pgfqpoint{1.111490in}{0.565714in}}%
\pgfpathlineto{\pgfqpoint{1.112787in}{0.565714in}}%
\pgfpathlineto{\pgfqpoint{1.112787in}{0.570026in}}%
\pgfpathlineto{\pgfqpoint{1.114085in}{0.563558in}}%
\pgfpathlineto{\pgfqpoint{1.116680in}{0.562480in}}%
\pgfpathlineto{\pgfqpoint{1.117977in}{0.570026in}}%
\pgfpathlineto{\pgfqpoint{1.120572in}{0.568948in}}%
\pgfpathlineto{\pgfqpoint{1.120572in}{0.565714in}}%
\pgfpathlineto{\pgfqpoint{1.121869in}{0.570026in}}%
\pgfpathlineto{\pgfqpoint{1.125762in}{0.570026in}}%
\pgfpathlineto{\pgfqpoint{1.127059in}{0.563558in}}%
\pgfpathlineto{\pgfqpoint{1.128356in}{0.563558in}}%
\pgfpathlineto{\pgfqpoint{1.128356in}{0.560324in}}%
\pgfpathlineto{\pgfqpoint{1.129654in}{0.564636in}}%
\pgfpathlineto{\pgfqpoint{1.130951in}{0.564636in}}%
\pgfpathlineto{\pgfqpoint{1.130951in}{0.560324in}}%
\pgfpathlineto{\pgfqpoint{1.132249in}{0.571104in}}%
\pgfpathlineto{\pgfqpoint{1.133546in}{0.571104in}}%
\pgfpathlineto{\pgfqpoint{1.133546in}{0.567870in}}%
\pgfpathlineto{\pgfqpoint{1.134844in}{0.568948in}}%
\pgfpathlineto{\pgfqpoint{1.136141in}{0.568948in}}%
\pgfpathlineto{\pgfqpoint{1.136141in}{0.565714in}}%
\pgfpathlineto{\pgfqpoint{1.137438in}{0.573260in}}%
\pgfpathlineto{\pgfqpoint{1.138736in}{0.573260in}}%
\pgfpathlineto{\pgfqpoint{1.138736in}{0.562480in}}%
\pgfpathlineto{\pgfqpoint{1.140033in}{0.571104in}}%
\pgfpathlineto{\pgfqpoint{1.141331in}{0.571104in}}%
\pgfpathlineto{\pgfqpoint{1.142628in}{0.558168in}}%
\pgfpathlineto{\pgfqpoint{1.143926in}{0.558168in}}%
\pgfpathlineto{\pgfqpoint{1.145223in}{0.567870in}}%
\pgfpathlineto{\pgfqpoint{1.146520in}{0.567870in}}%
\pgfpathlineto{\pgfqpoint{1.147818in}{0.560324in}}%
\pgfpathlineto{\pgfqpoint{1.149115in}{0.560324in}}%
\pgfpathlineto{\pgfqpoint{1.149115in}{0.573260in}}%
\pgfpathlineto{\pgfqpoint{1.150413in}{0.567870in}}%
\pgfpathlineto{\pgfqpoint{1.151710in}{0.567870in}}%
\pgfpathlineto{\pgfqpoint{1.151710in}{0.560324in}}%
\pgfpathlineto{\pgfqpoint{1.153007in}{0.566792in}}%
\pgfpathlineto{\pgfqpoint{1.154305in}{0.566792in}}%
\pgfpathlineto{\pgfqpoint{1.154305in}{0.562480in}}%
\pgfpathlineto{\pgfqpoint{1.155602in}{0.571104in}}%
\pgfpathlineto{\pgfqpoint{1.156900in}{0.571104in}}%
\pgfpathlineto{\pgfqpoint{1.158197in}{0.564636in}}%
\pgfpathlineto{\pgfqpoint{1.159495in}{0.564636in}}%
\pgfpathlineto{\pgfqpoint{1.159495in}{0.568948in}}%
\pgfpathlineto{\pgfqpoint{1.160792in}{0.567870in}}%
\pgfpathlineto{\pgfqpoint{1.162089in}{0.567870in}}%
\pgfpathlineto{\pgfqpoint{1.163387in}{0.562480in}}%
\pgfpathlineto{\pgfqpoint{1.164684in}{0.562480in}}%
\pgfpathlineto{\pgfqpoint{1.164684in}{0.568948in}}%
\pgfpathlineto{\pgfqpoint{1.165982in}{0.563558in}}%
\pgfpathlineto{\pgfqpoint{1.167279in}{0.563558in}}%
\pgfpathlineto{\pgfqpoint{1.167279in}{0.577572in}}%
\pgfpathlineto{\pgfqpoint{1.168577in}{0.558168in}}%
\pgfpathlineto{\pgfqpoint{1.169874in}{0.558168in}}%
\pgfpathlineto{\pgfqpoint{1.169874in}{0.571104in}}%
\pgfpathlineto{\pgfqpoint{1.171171in}{0.571104in}}%
\pgfpathlineto{\pgfqpoint{1.175064in}{0.570026in}}%
\pgfpathlineto{\pgfqpoint{1.175064in}{0.566792in}}%
\pgfpathlineto{\pgfqpoint{1.176361in}{0.572182in}}%
\pgfpathlineto{\pgfqpoint{1.177659in}{0.572182in}}%
\pgfpathlineto{\pgfqpoint{1.177659in}{0.563558in}}%
\pgfpathlineto{\pgfqpoint{1.178956in}{0.568948in}}%
\pgfpathlineto{\pgfqpoint{1.180253in}{0.568948in}}%
\pgfpathlineto{\pgfqpoint{1.180253in}{0.563558in}}%
\pgfpathlineto{\pgfqpoint{1.181551in}{0.567870in}}%
\pgfpathlineto{\pgfqpoint{1.182848in}{0.567870in}}%
\pgfpathlineto{\pgfqpoint{1.182848in}{0.563558in}}%
\pgfpathlineto{\pgfqpoint{1.184146in}{0.567870in}}%
\pgfpathlineto{\pgfqpoint{1.186740in}{0.568948in}}%
\pgfpathlineto{\pgfqpoint{1.186740in}{0.563558in}}%
\pgfpathlineto{\pgfqpoint{1.188038in}{0.567870in}}%
\pgfpathlineto{\pgfqpoint{1.189335in}{0.567870in}}%
\pgfpathlineto{\pgfqpoint{1.190633in}{0.562480in}}%
\pgfpathlineto{\pgfqpoint{1.193228in}{0.563558in}}%
\pgfpathlineto{\pgfqpoint{1.193228in}{0.560324in}}%
\pgfpathlineto{\pgfqpoint{1.194525in}{0.565714in}}%
\pgfpathlineto{\pgfqpoint{1.195822in}{0.565714in}}%
\pgfpathlineto{\pgfqpoint{1.195822in}{0.561402in}}%
\pgfpathlineto{\pgfqpoint{1.197120in}{0.565714in}}%
\pgfpathlineto{\pgfqpoint{1.198417in}{0.565714in}}%
\pgfpathlineto{\pgfqpoint{1.198417in}{0.563558in}}%
\pgfpathlineto{\pgfqpoint{1.199715in}{0.570026in}}%
\pgfpathlineto{\pgfqpoint{1.201012in}{0.570026in}}%
\pgfpathlineto{\pgfqpoint{1.201012in}{0.562480in}}%
\pgfpathlineto{\pgfqpoint{1.202310in}{0.564636in}}%
\pgfpathlineto{\pgfqpoint{1.204904in}{0.565714in}}%
\pgfpathlineto{\pgfqpoint{1.204904in}{0.570026in}}%
\pgfpathlineto{\pgfqpoint{1.206202in}{0.564636in}}%
\pgfpathlineto{\pgfqpoint{1.208797in}{0.565714in}}%
\pgfpathlineto{\pgfqpoint{1.210094in}{0.568948in}}%
\pgfpathlineto{\pgfqpoint{1.211392in}{0.568948in}}%
\pgfpathlineto{\pgfqpoint{1.211392in}{0.573260in}}%
\pgfpathlineto{\pgfqpoint{1.212689in}{0.573260in}}%
\pgfpathlineto{\pgfqpoint{1.213986in}{0.573260in}}%
\pgfpathlineto{\pgfqpoint{1.213986in}{0.568948in}}%
\pgfpathlineto{\pgfqpoint{1.215284in}{0.568948in}}%
\pgfpathlineto{\pgfqpoint{1.216581in}{0.568948in}}%
\pgfpathlineto{\pgfqpoint{1.216581in}{0.571104in}}%
\pgfpathlineto{\pgfqpoint{1.217879in}{0.565714in}}%
\pgfpathlineto{\pgfqpoint{1.219176in}{0.565714in}}%
\pgfpathlineto{\pgfqpoint{1.219176in}{0.576494in}}%
\pgfpathlineto{\pgfqpoint{1.220473in}{0.564636in}}%
\pgfpathlineto{\pgfqpoint{1.221771in}{0.564636in}}%
\pgfpathlineto{\pgfqpoint{1.223068in}{0.568948in}}%
\pgfpathlineto{\pgfqpoint{1.224366in}{0.568948in}}%
\pgfpathlineto{\pgfqpoint{1.224366in}{0.560324in}}%
\pgfpathlineto{\pgfqpoint{1.225663in}{0.572182in}}%
\pgfpathlineto{\pgfqpoint{1.226961in}{0.572182in}}%
\pgfpathlineto{\pgfqpoint{1.226961in}{0.565714in}}%
\pgfpathlineto{\pgfqpoint{1.228258in}{0.570026in}}%
\pgfpathlineto{\pgfqpoint{1.229555in}{0.570026in}}%
\pgfpathlineto{\pgfqpoint{1.229555in}{0.575416in}}%
\pgfpathlineto{\pgfqpoint{1.230853in}{0.567870in}}%
\pgfpathlineto{\pgfqpoint{1.234745in}{0.566792in}}%
\pgfpathlineto{\pgfqpoint{1.234745in}{0.573260in}}%
\pgfpathlineto{\pgfqpoint{1.236043in}{0.563558in}}%
\pgfpathlineto{\pgfqpoint{1.237340in}{0.563558in}}%
\pgfpathlineto{\pgfqpoint{1.237340in}{0.575416in}}%
\pgfpathlineto{\pgfqpoint{1.238637in}{0.560324in}}%
\pgfpathlineto{\pgfqpoint{1.239935in}{0.560324in}}%
\pgfpathlineto{\pgfqpoint{1.239935in}{0.566792in}}%
\pgfpathlineto{\pgfqpoint{1.241232in}{0.560324in}}%
\pgfpathlineto{\pgfqpoint{1.242530in}{0.560324in}}%
\pgfpathlineto{\pgfqpoint{1.243827in}{0.571104in}}%
\pgfpathlineto{\pgfqpoint{1.245125in}{0.571104in}}%
\pgfpathlineto{\pgfqpoint{1.245125in}{0.573260in}}%
\pgfpathlineto{\pgfqpoint{1.246422in}{0.567870in}}%
\pgfpathlineto{\pgfqpoint{1.247719in}{0.567870in}}%
\pgfpathlineto{\pgfqpoint{1.247719in}{0.563558in}}%
\pgfpathlineto{\pgfqpoint{1.249017in}{0.565714in}}%
\pgfpathlineto{\pgfqpoint{1.251612in}{0.564636in}}%
\pgfpathlineto{\pgfqpoint{1.251612in}{0.579728in}}%
\pgfpathlineto{\pgfqpoint{1.252909in}{0.567870in}}%
\pgfpathlineto{\pgfqpoint{1.254206in}{0.567870in}}%
\pgfpathlineto{\pgfqpoint{1.254206in}{0.571104in}}%
\pgfpathlineto{\pgfqpoint{1.255504in}{0.562480in}}%
\pgfpathlineto{\pgfqpoint{1.256801in}{0.562480in}}%
\pgfpathlineto{\pgfqpoint{1.256801in}{0.577572in}}%
\pgfpathlineto{\pgfqpoint{1.258099in}{0.564636in}}%
\pgfpathlineto{\pgfqpoint{1.260694in}{0.563558in}}%
\pgfpathlineto{\pgfqpoint{1.260694in}{0.573260in}}%
\pgfpathlineto{\pgfqpoint{1.261991in}{0.558168in}}%
\pgfpathlineto{\pgfqpoint{1.263288in}{0.558168in}}%
\pgfpathlineto{\pgfqpoint{1.263288in}{0.564636in}}%
\pgfpathlineto{\pgfqpoint{1.264586in}{0.563558in}}%
\pgfpathlineto{\pgfqpoint{1.265883in}{0.563558in}}%
\pgfpathlineto{\pgfqpoint{1.265883in}{0.570026in}}%
\pgfpathlineto{\pgfqpoint{1.267181in}{0.562480in}}%
\pgfpathlineto{\pgfqpoint{1.268478in}{0.562480in}}%
\pgfpathlineto{\pgfqpoint{1.268478in}{0.574338in}}%
\pgfpathlineto{\pgfqpoint{1.269776in}{0.573260in}}%
\pgfpathlineto{\pgfqpoint{1.272370in}{0.574338in}}%
\pgfpathlineto{\pgfqpoint{1.272370in}{0.575416in}}%
\pgfpathlineto{\pgfqpoint{1.273668in}{0.568948in}}%
\pgfpathlineto{\pgfqpoint{1.274965in}{0.568948in}}%
\pgfpathlineto{\pgfqpoint{1.274965in}{0.571104in}}%
\pgfpathlineto{\pgfqpoint{1.276263in}{0.571104in}}%
\pgfpathlineto{\pgfqpoint{1.277560in}{0.571104in}}%
\pgfpathlineto{\pgfqpoint{1.277560in}{0.567870in}}%
\pgfpathlineto{\pgfqpoint{1.278858in}{0.568948in}}%
\pgfpathlineto{\pgfqpoint{1.280155in}{0.568948in}}%
\pgfpathlineto{\pgfqpoint{1.280155in}{0.576494in}}%
\pgfpathlineto{\pgfqpoint{1.281452in}{0.564636in}}%
\pgfpathlineto{\pgfqpoint{1.282750in}{0.564636in}}%
\pgfpathlineto{\pgfqpoint{1.284047in}{0.575416in}}%
\pgfpathlineto{\pgfqpoint{1.285345in}{0.575416in}}%
\pgfpathlineto{\pgfqpoint{1.285345in}{0.568948in}}%
\pgfpathlineto{\pgfqpoint{1.286642in}{0.575416in}}%
\pgfpathlineto{\pgfqpoint{1.287939in}{0.575416in}}%
\pgfpathlineto{\pgfqpoint{1.287939in}{0.570026in}}%
\pgfpathlineto{\pgfqpoint{1.289237in}{0.574338in}}%
\pgfpathlineto{\pgfqpoint{1.290534in}{0.574338in}}%
\pgfpathlineto{\pgfqpoint{1.290534in}{0.568948in}}%
\pgfpathlineto{\pgfqpoint{1.291832in}{0.575416in}}%
\pgfpathlineto{\pgfqpoint{1.294427in}{0.576494in}}%
\pgfpathlineto{\pgfqpoint{1.294427in}{0.572182in}}%
\pgfpathlineto{\pgfqpoint{1.295724in}{0.573260in}}%
\pgfpathlineto{\pgfqpoint{1.297021in}{0.573260in}}%
\pgfpathlineto{\pgfqpoint{1.297021in}{0.565714in}}%
\pgfpathlineto{\pgfqpoint{1.298319in}{0.572182in}}%
\pgfpathlineto{\pgfqpoint{1.299616in}{0.572182in}}%
\pgfpathlineto{\pgfqpoint{1.299616in}{0.574338in}}%
\pgfpathlineto{\pgfqpoint{1.300914in}{0.567870in}}%
\pgfpathlineto{\pgfqpoint{1.303509in}{0.567870in}}%
\pgfpathlineto{\pgfqpoint{1.303509in}{0.570026in}}%
\pgfpathlineto{\pgfqpoint{1.304806in}{0.570026in}}%
\pgfpathlineto{\pgfqpoint{1.312591in}{0.570026in}}%
\pgfpathlineto{\pgfqpoint{1.312591in}{0.574338in}}%
\pgfpathlineto{\pgfqpoint{1.313888in}{0.570026in}}%
\pgfpathlineto{\pgfqpoint{1.315185in}{0.570026in}}%
\pgfpathlineto{\pgfqpoint{1.315185in}{0.578650in}}%
\pgfpathlineto{\pgfqpoint{1.316483in}{0.567870in}}%
\pgfpathlineto{\pgfqpoint{1.317780in}{0.567870in}}%
\pgfpathlineto{\pgfqpoint{1.317780in}{0.587274in}}%
\pgfpathlineto{\pgfqpoint{1.319078in}{0.566792in}}%
\pgfpathlineto{\pgfqpoint{1.320375in}{0.566792in}}%
\pgfpathlineto{\pgfqpoint{1.321673in}{0.577572in}}%
\pgfpathlineto{\pgfqpoint{1.322970in}{0.577572in}}%
\pgfpathlineto{\pgfqpoint{1.322970in}{0.565714in}}%
\pgfpathlineto{\pgfqpoint{1.324267in}{0.578650in}}%
\pgfpathlineto{\pgfqpoint{1.325565in}{0.578650in}}%
\pgfpathlineto{\pgfqpoint{1.326862in}{0.570026in}}%
\pgfpathlineto{\pgfqpoint{1.330754in}{0.571104in}}%
\pgfpathlineto{\pgfqpoint{1.330754in}{0.574338in}}%
\pgfpathlineto{\pgfqpoint{1.332052in}{0.566792in}}%
\pgfpathlineto{\pgfqpoint{1.333349in}{0.566792in}}%
\pgfpathlineto{\pgfqpoint{1.334647in}{0.575416in}}%
\pgfpathlineto{\pgfqpoint{1.335944in}{0.575416in}}%
\pgfpathlineto{\pgfqpoint{1.337242in}{0.564636in}}%
\pgfpathlineto{\pgfqpoint{1.338539in}{0.564636in}}%
\pgfpathlineto{\pgfqpoint{1.339836in}{0.575416in}}%
\pgfpathlineto{\pgfqpoint{1.341134in}{0.575416in}}%
\pgfpathlineto{\pgfqpoint{1.341134in}{0.584040in}}%
\pgfpathlineto{\pgfqpoint{1.342431in}{0.572182in}}%
\pgfpathlineto{\pgfqpoint{1.343729in}{0.572182in}}%
\pgfpathlineto{\pgfqpoint{1.343729in}{0.579728in}}%
\pgfpathlineto{\pgfqpoint{1.345026in}{0.575416in}}%
\pgfpathlineto{\pgfqpoint{1.346324in}{0.575416in}}%
\pgfpathlineto{\pgfqpoint{1.346324in}{0.564636in}}%
\pgfpathlineto{\pgfqpoint{1.347621in}{0.580806in}}%
\pgfpathlineto{\pgfqpoint{1.348918in}{0.580806in}}%
\pgfpathlineto{\pgfqpoint{1.348918in}{0.572182in}}%
\pgfpathlineto{\pgfqpoint{1.350216in}{0.584040in}}%
\pgfpathlineto{\pgfqpoint{1.351513in}{0.584040in}}%
\pgfpathlineto{\pgfqpoint{1.352811in}{0.573260in}}%
\pgfpathlineto{\pgfqpoint{1.354108in}{0.573260in}}%
\pgfpathlineto{\pgfqpoint{1.354108in}{0.568948in}}%
\pgfpathlineto{\pgfqpoint{1.355406in}{0.578650in}}%
\pgfpathlineto{\pgfqpoint{1.356703in}{0.578650in}}%
\pgfpathlineto{\pgfqpoint{1.356703in}{0.573260in}}%
\pgfpathlineto{\pgfqpoint{1.358000in}{0.575416in}}%
\pgfpathlineto{\pgfqpoint{1.359298in}{0.575416in}}%
\pgfpathlineto{\pgfqpoint{1.359298in}{0.572182in}}%
\pgfpathlineto{\pgfqpoint{1.360595in}{0.573260in}}%
\pgfpathlineto{\pgfqpoint{1.361893in}{0.573260in}}%
\pgfpathlineto{\pgfqpoint{1.363190in}{0.588352in}}%
\pgfpathlineto{\pgfqpoint{1.364487in}{0.588352in}}%
\pgfpathlineto{\pgfqpoint{1.365785in}{0.572182in}}%
\pgfpathlineto{\pgfqpoint{1.367082in}{0.572182in}}%
\pgfpathlineto{\pgfqpoint{1.368380in}{0.579728in}}%
\pgfpathlineto{\pgfqpoint{1.369677in}{0.579728in}}%
\pgfpathlineto{\pgfqpoint{1.369677in}{0.572182in}}%
\pgfpathlineto{\pgfqpoint{1.370975in}{0.574338in}}%
\pgfpathlineto{\pgfqpoint{1.373569in}{0.573260in}}%
\pgfpathlineto{\pgfqpoint{1.373569in}{0.566792in}}%
\pgfpathlineto{\pgfqpoint{1.374867in}{0.568948in}}%
\pgfpathlineto{\pgfqpoint{1.376164in}{0.568948in}}%
\pgfpathlineto{\pgfqpoint{1.376164in}{0.577572in}}%
\pgfpathlineto{\pgfqpoint{1.377462in}{0.568948in}}%
\pgfpathlineto{\pgfqpoint{1.381354in}{0.570026in}}%
\pgfpathlineto{\pgfqpoint{1.381354in}{0.575416in}}%
\pgfpathlineto{\pgfqpoint{1.382651in}{0.574338in}}%
\pgfpathlineto{\pgfqpoint{1.383949in}{0.574338in}}%
\pgfpathlineto{\pgfqpoint{1.383949in}{0.572182in}}%
\pgfpathlineto{\pgfqpoint{1.385246in}{0.573260in}}%
\pgfpathlineto{\pgfqpoint{1.386544in}{0.573260in}}%
\pgfpathlineto{\pgfqpoint{1.387841in}{0.581884in}}%
\pgfpathlineto{\pgfqpoint{1.389139in}{0.581884in}}%
\pgfpathlineto{\pgfqpoint{1.389139in}{0.561402in}}%
\pgfpathlineto{\pgfqpoint{1.390436in}{0.573260in}}%
\pgfpathlineto{\pgfqpoint{1.391733in}{0.573260in}}%
\pgfpathlineto{\pgfqpoint{1.391733in}{0.580806in}}%
\pgfpathlineto{\pgfqpoint{1.393031in}{0.576494in}}%
\pgfpathlineto{\pgfqpoint{1.394328in}{0.576494in}}%
\pgfpathlineto{\pgfqpoint{1.394328in}{0.581884in}}%
\pgfpathlineto{\pgfqpoint{1.395626in}{0.575416in}}%
\pgfpathlineto{\pgfqpoint{1.396923in}{0.575416in}}%
\pgfpathlineto{\pgfqpoint{1.398220in}{0.566792in}}%
\pgfpathlineto{\pgfqpoint{1.399518in}{0.566792in}}%
\pgfpathlineto{\pgfqpoint{1.399518in}{0.578650in}}%
\pgfpathlineto{\pgfqpoint{1.400815in}{0.572182in}}%
\pgfpathlineto{\pgfqpoint{1.402113in}{0.572182in}}%
\pgfpathlineto{\pgfqpoint{1.402113in}{0.587274in}}%
\pgfpathlineto{\pgfqpoint{1.403410in}{0.579728in}}%
\pgfpathlineto{\pgfqpoint{1.404708in}{0.579728in}}%
\pgfpathlineto{\pgfqpoint{1.404708in}{0.573260in}}%
\pgfpathlineto{\pgfqpoint{1.406005in}{0.586196in}}%
\pgfpathlineto{\pgfqpoint{1.407302in}{0.586196in}}%
\pgfpathlineto{\pgfqpoint{1.408600in}{0.573260in}}%
\pgfpathlineto{\pgfqpoint{1.412492in}{0.574338in}}%
\pgfpathlineto{\pgfqpoint{1.413790in}{0.584040in}}%
\pgfpathlineto{\pgfqpoint{1.415087in}{0.584040in}}%
\pgfpathlineto{\pgfqpoint{1.416384in}{0.575416in}}%
\pgfpathlineto{\pgfqpoint{1.418979in}{0.575416in}}%
\pgfpathlineto{\pgfqpoint{1.418979in}{0.581884in}}%
\pgfpathlineto{\pgfqpoint{1.420277in}{0.579728in}}%
\pgfpathlineto{\pgfqpoint{1.422872in}{0.579728in}}%
\pgfpathlineto{\pgfqpoint{1.422872in}{0.582962in}}%
\pgfpathlineto{\pgfqpoint{1.424169in}{0.578650in}}%
\pgfpathlineto{\pgfqpoint{1.425466in}{0.578650in}}%
\pgfpathlineto{\pgfqpoint{1.425466in}{0.572182in}}%
\pgfpathlineto{\pgfqpoint{1.426764in}{0.590508in}}%
\pgfpathlineto{\pgfqpoint{1.428061in}{0.590508in}}%
\pgfpathlineto{\pgfqpoint{1.429359in}{0.579728in}}%
\pgfpathlineto{\pgfqpoint{1.430656in}{0.579728in}}%
\pgfpathlineto{\pgfqpoint{1.430656in}{0.575416in}}%
\pgfpathlineto{\pgfqpoint{1.431953in}{0.584040in}}%
\pgfpathlineto{\pgfqpoint{1.433251in}{0.584040in}}%
\pgfpathlineto{\pgfqpoint{1.433251in}{0.572182in}}%
\pgfpathlineto{\pgfqpoint{1.434548in}{0.581884in}}%
\pgfpathlineto{\pgfqpoint{1.435846in}{0.581884in}}%
\pgfpathlineto{\pgfqpoint{1.435846in}{0.578650in}}%
\pgfpathlineto{\pgfqpoint{1.437143in}{0.588352in}}%
\pgfpathlineto{\pgfqpoint{1.438441in}{0.588352in}}%
\pgfpathlineto{\pgfqpoint{1.438441in}{0.574338in}}%
\pgfpathlineto{\pgfqpoint{1.439738in}{0.586196in}}%
\pgfpathlineto{\pgfqpoint{1.441035in}{0.586196in}}%
\pgfpathlineto{\pgfqpoint{1.442333in}{0.565714in}}%
\pgfpathlineto{\pgfqpoint{1.443630in}{0.565714in}}%
\pgfpathlineto{\pgfqpoint{1.443630in}{0.591586in}}%
\pgfpathlineto{\pgfqpoint{1.444928in}{0.578650in}}%
\pgfpathlineto{\pgfqpoint{1.447523in}{0.578650in}}%
\pgfpathlineto{\pgfqpoint{1.448820in}{0.584040in}}%
\pgfpathlineto{\pgfqpoint{1.450117in}{0.584040in}}%
\pgfpathlineto{\pgfqpoint{1.450117in}{0.579728in}}%
\pgfpathlineto{\pgfqpoint{1.451415in}{0.580806in}}%
\pgfpathlineto{\pgfqpoint{1.452712in}{0.580806in}}%
\pgfpathlineto{\pgfqpoint{1.452712in}{0.584040in}}%
\pgfpathlineto{\pgfqpoint{1.454010in}{0.575416in}}%
\pgfpathlineto{\pgfqpoint{1.455307in}{0.575416in}}%
\pgfpathlineto{\pgfqpoint{1.455307in}{0.588352in}}%
\pgfpathlineto{\pgfqpoint{1.456605in}{0.580806in}}%
\pgfpathlineto{\pgfqpoint{1.457902in}{0.580806in}}%
\pgfpathlineto{\pgfqpoint{1.459199in}{0.588352in}}%
\pgfpathlineto{\pgfqpoint{1.460497in}{0.588352in}}%
\pgfpathlineto{\pgfqpoint{1.460497in}{0.573260in}}%
\pgfpathlineto{\pgfqpoint{1.461794in}{0.580806in}}%
\pgfpathlineto{\pgfqpoint{1.463092in}{0.580806in}}%
\pgfpathlineto{\pgfqpoint{1.463092in}{0.587274in}}%
\pgfpathlineto{\pgfqpoint{1.464389in}{0.586196in}}%
\pgfpathlineto{\pgfqpoint{1.465686in}{0.586196in}}%
\pgfpathlineto{\pgfqpoint{1.465686in}{0.584040in}}%
\pgfpathlineto{\pgfqpoint{1.466984in}{0.594821in}}%
\pgfpathlineto{\pgfqpoint{1.468281in}{0.594821in}}%
\pgfpathlineto{\pgfqpoint{1.469579in}{0.578650in}}%
\pgfpathlineto{\pgfqpoint{1.470876in}{0.578650in}}%
\pgfpathlineto{\pgfqpoint{1.470876in}{0.575416in}}%
\pgfpathlineto{\pgfqpoint{1.472174in}{0.585118in}}%
\pgfpathlineto{\pgfqpoint{1.474768in}{0.585118in}}%
\pgfpathlineto{\pgfqpoint{1.474768in}{0.579728in}}%
\pgfpathlineto{\pgfqpoint{1.476066in}{0.589430in}}%
\pgfpathlineto{\pgfqpoint{1.477363in}{0.589430in}}%
\pgfpathlineto{\pgfqpoint{1.477363in}{0.567870in}}%
\pgfpathlineto{\pgfqpoint{1.478661in}{0.586196in}}%
\pgfpathlineto{\pgfqpoint{1.479958in}{0.586196in}}%
\pgfpathlineto{\pgfqpoint{1.479958in}{0.579728in}}%
\pgfpathlineto{\pgfqpoint{1.481256in}{0.580806in}}%
\pgfpathlineto{\pgfqpoint{1.483850in}{0.579728in}}%
\pgfpathlineto{\pgfqpoint{1.483850in}{0.584040in}}%
\pgfpathlineto{\pgfqpoint{1.485148in}{0.578650in}}%
\pgfpathlineto{\pgfqpoint{1.486445in}{0.578650in}}%
\pgfpathlineto{\pgfqpoint{1.486445in}{0.575416in}}%
\pgfpathlineto{\pgfqpoint{1.487743in}{0.587274in}}%
\pgfpathlineto{\pgfqpoint{1.489040in}{0.587274in}}%
\pgfpathlineto{\pgfqpoint{1.489040in}{0.589430in}}%
\pgfpathlineto{\pgfqpoint{1.490338in}{0.582962in}}%
\pgfpathlineto{\pgfqpoint{1.491635in}{0.582962in}}%
\pgfpathlineto{\pgfqpoint{1.492932in}{0.578650in}}%
\pgfpathlineto{\pgfqpoint{1.494230in}{0.578650in}}%
\pgfpathlineto{\pgfqpoint{1.494230in}{0.571104in}}%
\pgfpathlineto{\pgfqpoint{1.495527in}{0.587274in}}%
\pgfpathlineto{\pgfqpoint{1.496825in}{0.587274in}}%
\pgfpathlineto{\pgfqpoint{1.496825in}{0.590508in}}%
\pgfpathlineto{\pgfqpoint{1.498122in}{0.584040in}}%
\pgfpathlineto{\pgfqpoint{1.499419in}{0.584040in}}%
\pgfpathlineto{\pgfqpoint{1.500717in}{0.578650in}}%
\pgfpathlineto{\pgfqpoint{1.502014in}{0.578650in}}%
\pgfpathlineto{\pgfqpoint{1.502014in}{0.591586in}}%
\pgfpathlineto{\pgfqpoint{1.503312in}{0.575416in}}%
\pgfpathlineto{\pgfqpoint{1.504609in}{0.575416in}}%
\pgfpathlineto{\pgfqpoint{1.504609in}{0.590508in}}%
\pgfpathlineto{\pgfqpoint{1.505907in}{0.578650in}}%
\pgfpathlineto{\pgfqpoint{1.507204in}{0.578650in}}%
\pgfpathlineto{\pgfqpoint{1.507204in}{0.584040in}}%
\pgfpathlineto{\pgfqpoint{1.508501in}{0.582962in}}%
\pgfpathlineto{\pgfqpoint{1.509799in}{0.582962in}}%
\pgfpathlineto{\pgfqpoint{1.509799in}{0.585118in}}%
\pgfpathlineto{\pgfqpoint{1.511096in}{0.574338in}}%
\pgfpathlineto{\pgfqpoint{1.512394in}{0.574338in}}%
\pgfpathlineto{\pgfqpoint{1.513691in}{0.593743in}}%
\pgfpathlineto{\pgfqpoint{1.514989in}{0.593743in}}%
\pgfpathlineto{\pgfqpoint{1.516286in}{0.582962in}}%
\pgfpathlineto{\pgfqpoint{1.517583in}{0.582962in}}%
\pgfpathlineto{\pgfqpoint{1.517583in}{0.580806in}}%
\pgfpathlineto{\pgfqpoint{1.518881in}{0.598055in}}%
\pgfpathlineto{\pgfqpoint{1.520178in}{0.598055in}}%
\pgfpathlineto{\pgfqpoint{1.521476in}{0.572182in}}%
\pgfpathlineto{\pgfqpoint{1.522773in}{0.572182in}}%
\pgfpathlineto{\pgfqpoint{1.524071in}{0.584040in}}%
\pgfpathlineto{\pgfqpoint{1.527963in}{0.582962in}}%
\pgfpathlineto{\pgfqpoint{1.527963in}{0.587274in}}%
\pgfpathlineto{\pgfqpoint{1.529260in}{0.584040in}}%
\pgfpathlineto{\pgfqpoint{1.530558in}{0.584040in}}%
\pgfpathlineto{\pgfqpoint{1.530558in}{0.577572in}}%
\pgfpathlineto{\pgfqpoint{1.531855in}{0.580806in}}%
\pgfpathlineto{\pgfqpoint{1.533152in}{0.580806in}}%
\pgfpathlineto{\pgfqpoint{1.533152in}{0.586196in}}%
\pgfpathlineto{\pgfqpoint{1.534450in}{0.585118in}}%
\pgfpathlineto{\pgfqpoint{1.535747in}{0.585118in}}%
\pgfpathlineto{\pgfqpoint{1.537045in}{0.600211in}}%
\pgfpathlineto{\pgfqpoint{1.538342in}{0.600211in}}%
\pgfpathlineto{\pgfqpoint{1.538342in}{0.584040in}}%
\pgfpathlineto{\pgfqpoint{1.539640in}{0.584040in}}%
\pgfpathlineto{\pgfqpoint{1.540937in}{0.584040in}}%
\pgfpathlineto{\pgfqpoint{1.540937in}{0.577572in}}%
\pgfpathlineto{\pgfqpoint{1.542234in}{0.595899in}}%
\pgfpathlineto{\pgfqpoint{1.543532in}{0.595899in}}%
\pgfpathlineto{\pgfqpoint{1.543532in}{0.585118in}}%
\pgfpathlineto{\pgfqpoint{1.544829in}{0.593743in}}%
\pgfpathlineto{\pgfqpoint{1.548722in}{0.594821in}}%
\pgfpathlineto{\pgfqpoint{1.550019in}{0.585118in}}%
\pgfpathlineto{\pgfqpoint{1.551316in}{0.585118in}}%
\pgfpathlineto{\pgfqpoint{1.551316in}{0.591586in}}%
\pgfpathlineto{\pgfqpoint{1.552614in}{0.577572in}}%
\pgfpathlineto{\pgfqpoint{1.553911in}{0.577572in}}%
\pgfpathlineto{\pgfqpoint{1.553911in}{0.586196in}}%
\pgfpathlineto{\pgfqpoint{1.555209in}{0.568948in}}%
\pgfpathlineto{\pgfqpoint{1.556506in}{0.568948in}}%
\pgfpathlineto{\pgfqpoint{1.556506in}{0.602367in}}%
\pgfpathlineto{\pgfqpoint{1.557804in}{0.585118in}}%
\pgfpathlineto{\pgfqpoint{1.559101in}{0.585118in}}%
\pgfpathlineto{\pgfqpoint{1.559101in}{0.601289in}}%
\pgfpathlineto{\pgfqpoint{1.560398in}{0.592664in}}%
\pgfpathlineto{\pgfqpoint{1.561696in}{0.592664in}}%
\pgfpathlineto{\pgfqpoint{1.561696in}{0.588352in}}%
\pgfpathlineto{\pgfqpoint{1.562993in}{0.590508in}}%
\pgfpathlineto{\pgfqpoint{1.564291in}{0.590508in}}%
\pgfpathlineto{\pgfqpoint{1.564291in}{0.579728in}}%
\pgfpathlineto{\pgfqpoint{1.565588in}{0.584040in}}%
\pgfpathlineto{\pgfqpoint{1.566885in}{0.584040in}}%
\pgfpathlineto{\pgfqpoint{1.566885in}{0.586196in}}%
\pgfpathlineto{\pgfqpoint{1.568183in}{0.584040in}}%
\pgfpathlineto{\pgfqpoint{1.569480in}{0.584040in}}%
\pgfpathlineto{\pgfqpoint{1.570778in}{0.596977in}}%
\pgfpathlineto{\pgfqpoint{1.572075in}{0.596977in}}%
\pgfpathlineto{\pgfqpoint{1.572075in}{0.586196in}}%
\pgfpathlineto{\pgfqpoint{1.573373in}{0.589430in}}%
\pgfpathlineto{\pgfqpoint{1.574670in}{0.589430in}}%
\pgfpathlineto{\pgfqpoint{1.574670in}{0.591586in}}%
\pgfpathlineto{\pgfqpoint{1.575967in}{0.576494in}}%
\pgfpathlineto{\pgfqpoint{1.577265in}{0.576494in}}%
\pgfpathlineto{\pgfqpoint{1.577265in}{0.585118in}}%
\pgfpathlineto{\pgfqpoint{1.578562in}{0.585118in}}%
\pgfpathlineto{\pgfqpoint{1.579860in}{0.585118in}}%
\pgfpathlineto{\pgfqpoint{1.579860in}{0.596977in}}%
\pgfpathlineto{\pgfqpoint{1.581157in}{0.580806in}}%
\pgfpathlineto{\pgfqpoint{1.582455in}{0.580806in}}%
\pgfpathlineto{\pgfqpoint{1.582455in}{0.594821in}}%
\pgfpathlineto{\pgfqpoint{1.583752in}{0.580806in}}%
\pgfpathlineto{\pgfqpoint{1.585049in}{0.580806in}}%
\pgfpathlineto{\pgfqpoint{1.585049in}{0.596977in}}%
\pgfpathlineto{\pgfqpoint{1.586347in}{0.592664in}}%
\pgfpathlineto{\pgfqpoint{1.587644in}{0.592664in}}%
\pgfpathlineto{\pgfqpoint{1.587644in}{0.586196in}}%
\pgfpathlineto{\pgfqpoint{1.588942in}{0.596977in}}%
\pgfpathlineto{\pgfqpoint{1.590239in}{0.596977in}}%
\pgfpathlineto{\pgfqpoint{1.590239in}{0.581884in}}%
\pgfpathlineto{\pgfqpoint{1.591537in}{0.584040in}}%
\pgfpathlineto{\pgfqpoint{1.594131in}{0.584040in}}%
\pgfpathlineto{\pgfqpoint{1.594131in}{0.591586in}}%
\pgfpathlineto{\pgfqpoint{1.595429in}{0.590508in}}%
\pgfpathlineto{\pgfqpoint{1.596726in}{0.590508in}}%
\pgfpathlineto{\pgfqpoint{1.596726in}{0.594821in}}%
\pgfpathlineto{\pgfqpoint{1.598024in}{0.589430in}}%
\pgfpathlineto{\pgfqpoint{1.599321in}{0.589430in}}%
\pgfpathlineto{\pgfqpoint{1.599321in}{0.599133in}}%
\pgfpathlineto{\pgfqpoint{1.600618in}{0.598055in}}%
\pgfpathlineto{\pgfqpoint{1.601916in}{0.598055in}}%
\pgfpathlineto{\pgfqpoint{1.603213in}{0.590508in}}%
\pgfpathlineto{\pgfqpoint{1.604511in}{0.590508in}}%
\pgfpathlineto{\pgfqpoint{1.604511in}{0.587274in}}%
\pgfpathlineto{\pgfqpoint{1.605808in}{0.599133in}}%
\pgfpathlineto{\pgfqpoint{1.607106in}{0.599133in}}%
\pgfpathlineto{\pgfqpoint{1.607106in}{0.584040in}}%
\pgfpathlineto{\pgfqpoint{1.608403in}{0.598055in}}%
\pgfpathlineto{\pgfqpoint{1.609700in}{0.598055in}}%
\pgfpathlineto{\pgfqpoint{1.609700in}{0.586196in}}%
\pgfpathlineto{\pgfqpoint{1.610998in}{0.592664in}}%
\pgfpathlineto{\pgfqpoint{1.612295in}{0.592664in}}%
\pgfpathlineto{\pgfqpoint{1.613593in}{0.604523in}}%
\pgfpathlineto{\pgfqpoint{1.614890in}{0.604523in}}%
\pgfpathlineto{\pgfqpoint{1.616188in}{0.587274in}}%
\pgfpathlineto{\pgfqpoint{1.617485in}{0.587274in}}%
\pgfpathlineto{\pgfqpoint{1.617485in}{0.590508in}}%
\pgfpathlineto{\pgfqpoint{1.618782in}{0.590508in}}%
\pgfpathlineto{\pgfqpoint{1.620080in}{0.590508in}}%
\pgfpathlineto{\pgfqpoint{1.620080in}{0.580806in}}%
\pgfpathlineto{\pgfqpoint{1.621377in}{0.595899in}}%
\pgfpathlineto{\pgfqpoint{1.623972in}{0.594821in}}%
\pgfpathlineto{\pgfqpoint{1.623972in}{0.584040in}}%
\pgfpathlineto{\pgfqpoint{1.625270in}{0.587274in}}%
\pgfpathlineto{\pgfqpoint{1.626567in}{0.587274in}}%
\pgfpathlineto{\pgfqpoint{1.626567in}{0.613147in}}%
\pgfpathlineto{\pgfqpoint{1.627864in}{0.590508in}}%
\pgfpathlineto{\pgfqpoint{1.629162in}{0.590508in}}%
\pgfpathlineto{\pgfqpoint{1.629162in}{0.602367in}}%
\pgfpathlineto{\pgfqpoint{1.630459in}{0.577572in}}%
\pgfpathlineto{\pgfqpoint{1.631757in}{0.577572in}}%
\pgfpathlineto{\pgfqpoint{1.631757in}{0.604523in}}%
\pgfpathlineto{\pgfqpoint{1.633054in}{0.594821in}}%
\pgfpathlineto{\pgfqpoint{1.635649in}{0.595899in}}%
\pgfpathlineto{\pgfqpoint{1.635649in}{0.596977in}}%
\pgfpathlineto{\pgfqpoint{1.636946in}{0.594821in}}%
\pgfpathlineto{\pgfqpoint{1.638244in}{0.594821in}}%
\pgfpathlineto{\pgfqpoint{1.639541in}{0.581884in}}%
\pgfpathlineto{\pgfqpoint{1.640839in}{0.581884in}}%
\pgfpathlineto{\pgfqpoint{1.642136in}{0.601289in}}%
\pgfpathlineto{\pgfqpoint{1.643433in}{0.601289in}}%
\pgfpathlineto{\pgfqpoint{1.644731in}{0.584040in}}%
\pgfpathlineto{\pgfqpoint{1.646028in}{0.584040in}}%
\pgfpathlineto{\pgfqpoint{1.646028in}{0.599133in}}%
\pgfpathlineto{\pgfqpoint{1.647326in}{0.591586in}}%
\pgfpathlineto{\pgfqpoint{1.648623in}{0.591586in}}%
\pgfpathlineto{\pgfqpoint{1.649921in}{0.596977in}}%
\pgfpathlineto{\pgfqpoint{1.651218in}{0.596977in}}%
\pgfpathlineto{\pgfqpoint{1.651218in}{0.584040in}}%
\pgfpathlineto{\pgfqpoint{1.652515in}{0.600211in}}%
\pgfpathlineto{\pgfqpoint{1.653813in}{0.600211in}}%
\pgfpathlineto{\pgfqpoint{1.653813in}{0.593743in}}%
\pgfpathlineto{\pgfqpoint{1.655110in}{0.600211in}}%
\pgfpathlineto{\pgfqpoint{1.656408in}{0.600211in}}%
\pgfpathlineto{\pgfqpoint{1.656408in}{0.586196in}}%
\pgfpathlineto{\pgfqpoint{1.657705in}{0.595899in}}%
\pgfpathlineto{\pgfqpoint{1.659003in}{0.595899in}}%
\pgfpathlineto{\pgfqpoint{1.659003in}{0.598055in}}%
\pgfpathlineto{\pgfqpoint{1.660300in}{0.598055in}}%
\pgfpathlineto{\pgfqpoint{1.661597in}{0.598055in}}%
\pgfpathlineto{\pgfqpoint{1.661597in}{0.613147in}}%
\pgfpathlineto{\pgfqpoint{1.662895in}{0.592664in}}%
\pgfpathlineto{\pgfqpoint{1.664192in}{0.592664in}}%
\pgfpathlineto{\pgfqpoint{1.664192in}{0.594821in}}%
\pgfpathlineto{\pgfqpoint{1.665490in}{0.592664in}}%
\pgfpathlineto{\pgfqpoint{1.666787in}{0.592664in}}%
\pgfpathlineto{\pgfqpoint{1.666787in}{0.610991in}}%
\pgfpathlineto{\pgfqpoint{1.668085in}{0.590508in}}%
\pgfpathlineto{\pgfqpoint{1.669382in}{0.590508in}}%
\pgfpathlineto{\pgfqpoint{1.669382in}{0.586196in}}%
\pgfpathlineto{\pgfqpoint{1.670679in}{0.612069in}}%
\pgfpathlineto{\pgfqpoint{1.671977in}{0.612069in}}%
\pgfpathlineto{\pgfqpoint{1.671977in}{0.596977in}}%
\pgfpathlineto{\pgfqpoint{1.673274in}{0.600211in}}%
\pgfpathlineto{\pgfqpoint{1.675869in}{0.599133in}}%
\pgfpathlineto{\pgfqpoint{1.675869in}{0.594821in}}%
\pgfpathlineto{\pgfqpoint{1.677166in}{0.599133in}}%
\pgfpathlineto{\pgfqpoint{1.678464in}{0.599133in}}%
\pgfpathlineto{\pgfqpoint{1.679761in}{0.594821in}}%
\pgfpathlineto{\pgfqpoint{1.681059in}{0.594821in}}%
\pgfpathlineto{\pgfqpoint{1.681059in}{0.607757in}}%
\pgfpathlineto{\pgfqpoint{1.682356in}{0.601289in}}%
\pgfpathlineto{\pgfqpoint{1.683654in}{0.601289in}}%
\pgfpathlineto{\pgfqpoint{1.683654in}{0.590508in}}%
\pgfpathlineto{\pgfqpoint{1.684951in}{0.608835in}}%
\pgfpathlineto{\pgfqpoint{1.686248in}{0.608835in}}%
\pgfpathlineto{\pgfqpoint{1.686248in}{0.596977in}}%
\pgfpathlineto{\pgfqpoint{1.687546in}{0.610991in}}%
\pgfpathlineto{\pgfqpoint{1.688843in}{0.610991in}}%
\pgfpathlineto{\pgfqpoint{1.688843in}{0.594821in}}%
\pgfpathlineto{\pgfqpoint{1.690141in}{0.608835in}}%
\pgfpathlineto{\pgfqpoint{1.691438in}{0.608835in}}%
\pgfpathlineto{\pgfqpoint{1.691438in}{0.591586in}}%
\pgfpathlineto{\pgfqpoint{1.692736in}{0.608835in}}%
\pgfpathlineto{\pgfqpoint{1.694033in}{0.608835in}}%
\pgfpathlineto{\pgfqpoint{1.694033in}{0.603445in}}%
\pgfpathlineto{\pgfqpoint{1.695330in}{0.607757in}}%
\pgfpathlineto{\pgfqpoint{1.696628in}{0.607757in}}%
\pgfpathlineto{\pgfqpoint{1.697925in}{0.588352in}}%
\pgfpathlineto{\pgfqpoint{1.699223in}{0.588352in}}%
\pgfpathlineto{\pgfqpoint{1.699223in}{0.604523in}}%
\pgfpathlineto{\pgfqpoint{1.700520in}{0.592664in}}%
\pgfpathlineto{\pgfqpoint{1.701818in}{0.592664in}}%
\pgfpathlineto{\pgfqpoint{1.701818in}{0.594821in}}%
\pgfpathlineto{\pgfqpoint{1.703115in}{0.592664in}}%
\pgfpathlineto{\pgfqpoint{1.704412in}{0.592664in}}%
\pgfpathlineto{\pgfqpoint{1.704412in}{0.608835in}}%
\pgfpathlineto{\pgfqpoint{1.705710in}{0.594821in}}%
\pgfpathlineto{\pgfqpoint{1.708305in}{0.595899in}}%
\pgfpathlineto{\pgfqpoint{1.708305in}{0.600211in}}%
\pgfpathlineto{\pgfqpoint{1.709602in}{0.587274in}}%
\pgfpathlineto{\pgfqpoint{1.710899in}{0.587274in}}%
\pgfpathlineto{\pgfqpoint{1.710899in}{0.607757in}}%
\pgfpathlineto{\pgfqpoint{1.712197in}{0.591586in}}%
\pgfpathlineto{\pgfqpoint{1.713494in}{0.591586in}}%
\pgfpathlineto{\pgfqpoint{1.713494in}{0.615303in}}%
\pgfpathlineto{\pgfqpoint{1.714792in}{0.613147in}}%
\pgfpathlineto{\pgfqpoint{1.716089in}{0.613147in}}%
\pgfpathlineto{\pgfqpoint{1.716089in}{0.619615in}}%
\pgfpathlineto{\pgfqpoint{1.717387in}{0.595899in}}%
\pgfpathlineto{\pgfqpoint{1.718684in}{0.595899in}}%
\pgfpathlineto{\pgfqpoint{1.718684in}{0.603445in}}%
\pgfpathlineto{\pgfqpoint{1.719981in}{0.595899in}}%
\pgfpathlineto{\pgfqpoint{1.721279in}{0.595899in}}%
\pgfpathlineto{\pgfqpoint{1.722576in}{0.612069in}}%
\pgfpathlineto{\pgfqpoint{1.723874in}{0.612069in}}%
\pgfpathlineto{\pgfqpoint{1.723874in}{0.593743in}}%
\pgfpathlineto{\pgfqpoint{1.725171in}{0.610991in}}%
\pgfpathlineto{\pgfqpoint{1.726469in}{0.610991in}}%
\pgfpathlineto{\pgfqpoint{1.726469in}{0.598055in}}%
\pgfpathlineto{\pgfqpoint{1.727766in}{0.604523in}}%
\pgfpathlineto{\pgfqpoint{1.730361in}{0.605601in}}%
\pgfpathlineto{\pgfqpoint{1.731658in}{0.612069in}}%
\pgfpathlineto{\pgfqpoint{1.732956in}{0.612069in}}%
\pgfpathlineto{\pgfqpoint{1.732956in}{0.607757in}}%
\pgfpathlineto{\pgfqpoint{1.734253in}{0.623927in}}%
\pgfpathlineto{\pgfqpoint{1.735551in}{0.623927in}}%
\pgfpathlineto{\pgfqpoint{1.736848in}{0.596977in}}%
\pgfpathlineto{\pgfqpoint{1.738145in}{0.596977in}}%
\pgfpathlineto{\pgfqpoint{1.739443in}{0.621771in}}%
\pgfpathlineto{\pgfqpoint{1.740740in}{0.621771in}}%
\pgfpathlineto{\pgfqpoint{1.742038in}{0.602367in}}%
\pgfpathlineto{\pgfqpoint{1.743335in}{0.602367in}}%
\pgfpathlineto{\pgfqpoint{1.743335in}{0.613147in}}%
\pgfpathlineto{\pgfqpoint{1.744632in}{0.599133in}}%
\pgfpathlineto{\pgfqpoint{1.745930in}{0.599133in}}%
\pgfpathlineto{\pgfqpoint{1.745930in}{0.610991in}}%
\pgfpathlineto{\pgfqpoint{1.747227in}{0.603445in}}%
\pgfpathlineto{\pgfqpoint{1.748525in}{0.603445in}}%
\pgfpathlineto{\pgfqpoint{1.749822in}{0.618537in}}%
\pgfpathlineto{\pgfqpoint{1.752417in}{0.617459in}}%
\pgfpathlineto{\pgfqpoint{1.753714in}{0.605601in}}%
\pgfpathlineto{\pgfqpoint{1.755012in}{0.605601in}}%
\pgfpathlineto{\pgfqpoint{1.755012in}{0.613147in}}%
\pgfpathlineto{\pgfqpoint{1.756309in}{0.608835in}}%
\pgfpathlineto{\pgfqpoint{1.757607in}{0.608835in}}%
\pgfpathlineto{\pgfqpoint{1.757607in}{0.627161in}}%
\pgfpathlineto{\pgfqpoint{1.758904in}{0.606679in}}%
\pgfpathlineto{\pgfqpoint{1.760202in}{0.606679in}}%
\pgfpathlineto{\pgfqpoint{1.760202in}{0.620693in}}%
\pgfpathlineto{\pgfqpoint{1.761499in}{0.608835in}}%
\pgfpathlineto{\pgfqpoint{1.762796in}{0.608835in}}%
\pgfpathlineto{\pgfqpoint{1.762796in}{0.623927in}}%
\pgfpathlineto{\pgfqpoint{1.764094in}{0.615303in}}%
\pgfpathlineto{\pgfqpoint{1.765391in}{0.615303in}}%
\pgfpathlineto{\pgfqpoint{1.765391in}{0.620693in}}%
\pgfpathlineto{\pgfqpoint{1.766689in}{0.616381in}}%
\pgfpathlineto{\pgfqpoint{1.767986in}{0.616381in}}%
\pgfpathlineto{\pgfqpoint{1.767986in}{0.600211in}}%
\pgfpathlineto{\pgfqpoint{1.769284in}{0.622849in}}%
\pgfpathlineto{\pgfqpoint{1.771878in}{0.622849in}}%
\pgfpathlineto{\pgfqpoint{1.773176in}{0.600211in}}%
\pgfpathlineto{\pgfqpoint{1.774473in}{0.600211in}}%
\pgfpathlineto{\pgfqpoint{1.774473in}{0.609913in}}%
\pgfpathlineto{\pgfqpoint{1.775771in}{0.601289in}}%
\pgfpathlineto{\pgfqpoint{1.777068in}{0.601289in}}%
\pgfpathlineto{\pgfqpoint{1.777068in}{0.637941in}}%
\pgfpathlineto{\pgfqpoint{1.778365in}{0.607757in}}%
\pgfpathlineto{\pgfqpoint{1.779663in}{0.607757in}}%
\pgfpathlineto{\pgfqpoint{1.779663in}{0.634707in}}%
\pgfpathlineto{\pgfqpoint{1.780960in}{0.628239in}}%
\pgfpathlineto{\pgfqpoint{1.782258in}{0.628239in}}%
\pgfpathlineto{\pgfqpoint{1.782258in}{0.623927in}}%
\pgfpathlineto{\pgfqpoint{1.783555in}{0.628239in}}%
\pgfpathlineto{\pgfqpoint{1.784853in}{0.628239in}}%
\pgfpathlineto{\pgfqpoint{1.784853in}{0.614225in}}%
\pgfpathlineto{\pgfqpoint{1.786150in}{0.635785in}}%
\pgfpathlineto{\pgfqpoint{1.787447in}{0.635785in}}%
\pgfpathlineto{\pgfqpoint{1.787447in}{0.608835in}}%
\pgfpathlineto{\pgfqpoint{1.788745in}{0.625005in}}%
\pgfpathlineto{\pgfqpoint{1.790042in}{0.625005in}}%
\pgfpathlineto{\pgfqpoint{1.791340in}{0.643331in}}%
\pgfpathlineto{\pgfqpoint{1.792637in}{0.643331in}}%
\pgfpathlineto{\pgfqpoint{1.793935in}{0.614225in}}%
\pgfpathlineto{\pgfqpoint{1.795232in}{0.614225in}}%
\pgfpathlineto{\pgfqpoint{1.796529in}{0.640097in}}%
\pgfpathlineto{\pgfqpoint{1.797827in}{0.640097in}}%
\pgfpathlineto{\pgfqpoint{1.797827in}{0.620693in}}%
\pgfpathlineto{\pgfqpoint{1.799124in}{0.626083in}}%
\pgfpathlineto{\pgfqpoint{1.800422in}{0.626083in}}%
\pgfpathlineto{\pgfqpoint{1.800422in}{0.634707in}}%
\pgfpathlineto{\pgfqpoint{1.801719in}{0.629317in}}%
\pgfpathlineto{\pgfqpoint{1.803017in}{0.629317in}}%
\pgfpathlineto{\pgfqpoint{1.804314in}{0.650878in}}%
\pgfpathlineto{\pgfqpoint{1.805611in}{0.650878in}}%
\pgfpathlineto{\pgfqpoint{1.805611in}{0.615303in}}%
\pgfpathlineto{\pgfqpoint{1.806909in}{0.645487in}}%
\pgfpathlineto{\pgfqpoint{1.808206in}{0.645487in}}%
\pgfpathlineto{\pgfqpoint{1.808206in}{0.635785in}}%
\pgfpathlineto{\pgfqpoint{1.809504in}{0.657346in}}%
\pgfpathlineto{\pgfqpoint{1.810801in}{0.657346in}}%
\pgfpathlineto{\pgfqpoint{1.810801in}{0.630395in}}%
\pgfpathlineto{\pgfqpoint{1.812098in}{0.643331in}}%
\pgfpathlineto{\pgfqpoint{1.813396in}{0.643331in}}%
\pgfpathlineto{\pgfqpoint{1.813396in}{0.629317in}}%
\pgfpathlineto{\pgfqpoint{1.814693in}{0.659502in}}%
\pgfpathlineto{\pgfqpoint{1.815991in}{0.659502in}}%
\pgfpathlineto{\pgfqpoint{1.817288in}{0.641175in}}%
\pgfpathlineto{\pgfqpoint{1.818586in}{0.641175in}}%
\pgfpathlineto{\pgfqpoint{1.818586in}{0.644409in}}%
\pgfpathlineto{\pgfqpoint{1.819883in}{0.639019in}}%
\pgfpathlineto{\pgfqpoint{1.821180in}{0.639019in}}%
\pgfpathlineto{\pgfqpoint{1.821180in}{0.659502in}}%
\pgfpathlineto{\pgfqpoint{1.822478in}{0.642253in}}%
\pgfpathlineto{\pgfqpoint{1.825073in}{0.642253in}}%
\pgfpathlineto{\pgfqpoint{1.826370in}{0.656268in}}%
\pgfpathlineto{\pgfqpoint{1.827668in}{0.656268in}}%
\pgfpathlineto{\pgfqpoint{1.827668in}{0.675672in}}%
\pgfpathlineto{\pgfqpoint{1.828965in}{0.632551in}}%
\pgfpathlineto{\pgfqpoint{1.830262in}{0.632551in}}%
\pgfpathlineto{\pgfqpoint{1.831560in}{0.645487in}}%
\pgfpathlineto{\pgfqpoint{1.832857in}{0.645487in}}%
\pgfpathlineto{\pgfqpoint{1.832857in}{0.676750in}}%
\pgfpathlineto{\pgfqpoint{1.834155in}{0.645487in}}%
\pgfpathlineto{\pgfqpoint{1.835452in}{0.645487in}}%
\pgfpathlineto{\pgfqpoint{1.835452in}{0.659502in}}%
\pgfpathlineto{\pgfqpoint{1.836750in}{0.650878in}}%
\pgfpathlineto{\pgfqpoint{1.838047in}{0.650878in}}%
\pgfpathlineto{\pgfqpoint{1.838047in}{0.671360in}}%
\pgfpathlineto{\pgfqpoint{1.839344in}{0.650878in}}%
\pgfpathlineto{\pgfqpoint{1.840642in}{0.650878in}}%
\pgfpathlineto{\pgfqpoint{1.840642in}{0.658424in}}%
\pgfpathlineto{\pgfqpoint{1.841939in}{0.647643in}}%
\pgfpathlineto{\pgfqpoint{1.843237in}{0.647643in}}%
\pgfpathlineto{\pgfqpoint{1.843237in}{0.659502in}}%
\pgfpathlineto{\pgfqpoint{1.844534in}{0.653034in}}%
\pgfpathlineto{\pgfqpoint{1.845831in}{0.653034in}}%
\pgfpathlineto{\pgfqpoint{1.847129in}{0.679984in}}%
\pgfpathlineto{\pgfqpoint{1.848426in}{0.679984in}}%
\pgfpathlineto{\pgfqpoint{1.848426in}{0.646565in}}%
\pgfpathlineto{\pgfqpoint{1.849724in}{0.695076in}}%
\pgfpathlineto{\pgfqpoint{1.851021in}{0.695076in}}%
\pgfpathlineto{\pgfqpoint{1.851021in}{0.671360in}}%
\pgfpathlineto{\pgfqpoint{1.852319in}{0.686452in}}%
\pgfpathlineto{\pgfqpoint{1.853616in}{0.686452in}}%
\pgfpathlineto{\pgfqpoint{1.854913in}{0.657346in}}%
\pgfpathlineto{\pgfqpoint{1.856211in}{0.657346in}}%
\pgfpathlineto{\pgfqpoint{1.856211in}{0.724183in}}%
\pgfpathlineto{\pgfqpoint{1.857508in}{0.670282in}}%
\pgfpathlineto{\pgfqpoint{1.858806in}{0.670282in}}%
\pgfpathlineto{\pgfqpoint{1.858806in}{0.692920in}}%
\pgfpathlineto{\pgfqpoint{1.860103in}{0.690764in}}%
\pgfpathlineto{\pgfqpoint{1.861401in}{0.690764in}}%
\pgfpathlineto{\pgfqpoint{1.861401in}{0.701544in}}%
\pgfpathlineto{\pgfqpoint{1.862698in}{0.683218in}}%
\pgfpathlineto{\pgfqpoint{1.863995in}{0.683218in}}%
\pgfpathlineto{\pgfqpoint{1.863995in}{0.703700in}}%
\pgfpathlineto{\pgfqpoint{1.865293in}{0.672438in}}%
\pgfpathlineto{\pgfqpoint{1.866590in}{0.672438in}}%
\pgfpathlineto{\pgfqpoint{1.866590in}{0.691842in}}%
\pgfpathlineto{\pgfqpoint{1.867888in}{0.689686in}}%
\pgfpathlineto{\pgfqpoint{1.869185in}{0.689686in}}%
\pgfpathlineto{\pgfqpoint{1.869185in}{0.687530in}}%
\pgfpathlineto{\pgfqpoint{1.870483in}{0.734963in}}%
\pgfpathlineto{\pgfqpoint{1.871780in}{0.734963in}}%
\pgfpathlineto{\pgfqpoint{1.871780in}{0.677828in}}%
\pgfpathlineto{\pgfqpoint{1.873077in}{0.695076in}}%
\pgfpathlineto{\pgfqpoint{1.875672in}{0.693998in}}%
\pgfpathlineto{\pgfqpoint{1.875672in}{0.724183in}}%
\pgfpathlineto{\pgfqpoint{1.876970in}{0.692920in}}%
\pgfpathlineto{\pgfqpoint{1.878267in}{0.692920in}}%
\pgfpathlineto{\pgfqpoint{1.878267in}{0.682140in}}%
\pgfpathlineto{\pgfqpoint{1.879564in}{0.734963in}}%
\pgfpathlineto{\pgfqpoint{1.880862in}{0.734963in}}%
\pgfpathlineto{\pgfqpoint{1.880862in}{0.683218in}}%
\pgfpathlineto{\pgfqpoint{1.882159in}{0.729573in}}%
\pgfpathlineto{\pgfqpoint{1.883457in}{0.729573in}}%
\pgfpathlineto{\pgfqpoint{1.883457in}{0.715559in}}%
\pgfpathlineto{\pgfqpoint{1.884754in}{0.724183in}}%
\pgfpathlineto{\pgfqpoint{1.886052in}{0.724183in}}%
\pgfpathlineto{\pgfqpoint{1.886052in}{0.698310in}}%
\pgfpathlineto{\pgfqpoint{1.887349in}{0.726339in}}%
\pgfpathlineto{\pgfqpoint{1.889944in}{0.726339in}}%
\pgfpathlineto{\pgfqpoint{1.889944in}{0.713403in}}%
\pgfpathlineto{\pgfqpoint{1.891241in}{0.742509in}}%
\pgfpathlineto{\pgfqpoint{1.892539in}{0.742509in}}%
\pgfpathlineto{\pgfqpoint{1.893836in}{0.777006in}}%
\pgfpathlineto{\pgfqpoint{1.895134in}{0.777006in}}%
\pgfpathlineto{\pgfqpoint{1.895134in}{0.734963in}}%
\pgfpathlineto{\pgfqpoint{1.896431in}{0.759757in}}%
\pgfpathlineto{\pgfqpoint{1.897728in}{0.759757in}}%
\pgfpathlineto{\pgfqpoint{1.897728in}{0.730651in}}%
\pgfpathlineto{\pgfqpoint{1.899026in}{0.752211in}}%
\pgfpathlineto{\pgfqpoint{1.900323in}{0.752211in}}%
\pgfpathlineto{\pgfqpoint{1.900323in}{0.786708in}}%
\pgfpathlineto{\pgfqpoint{1.901621in}{0.732807in}}%
\pgfpathlineto{\pgfqpoint{1.902918in}{0.732807in}}%
\pgfpathlineto{\pgfqpoint{1.902918in}{0.760835in}}%
\pgfpathlineto{\pgfqpoint{1.904216in}{0.731729in}}%
\pgfpathlineto{\pgfqpoint{1.905513in}{0.731729in}}%
\pgfpathlineto{\pgfqpoint{1.905513in}{0.762992in}}%
\pgfpathlineto{\pgfqpoint{1.906810in}{0.745743in}}%
\pgfpathlineto{\pgfqpoint{1.908108in}{0.745743in}}%
\pgfpathlineto{\pgfqpoint{1.908108in}{0.800722in}}%
\pgfpathlineto{\pgfqpoint{1.909405in}{0.744665in}}%
\pgfpathlineto{\pgfqpoint{1.910703in}{0.744665in}}%
\pgfpathlineto{\pgfqpoint{1.910703in}{0.785630in}}%
\pgfpathlineto{\pgfqpoint{1.912000in}{0.785630in}}%
\pgfpathlineto{\pgfqpoint{1.913297in}{0.785630in}}%
\pgfpathlineto{\pgfqpoint{1.913297in}{0.775928in}}%
\pgfpathlineto{\pgfqpoint{1.914595in}{0.796410in}}%
\pgfpathlineto{\pgfqpoint{1.915892in}{0.796410in}}%
\pgfpathlineto{\pgfqpoint{1.915892in}{0.772694in}}%
\pgfpathlineto{\pgfqpoint{1.917190in}{0.793176in}}%
\pgfpathlineto{\pgfqpoint{1.918487in}{0.793176in}}%
\pgfpathlineto{\pgfqpoint{1.918487in}{0.770538in}}%
\pgfpathlineto{\pgfqpoint{1.919785in}{0.788864in}}%
\pgfpathlineto{\pgfqpoint{1.921082in}{0.788864in}}%
\pgfpathlineto{\pgfqpoint{1.921082in}{0.760835in}}%
\pgfpathlineto{\pgfqpoint{1.922379in}{0.825517in}}%
\pgfpathlineto{\pgfqpoint{1.923677in}{0.825517in}}%
\pgfpathlineto{\pgfqpoint{1.923677in}{0.771616in}}%
\pgfpathlineto{\pgfqpoint{1.924974in}{0.830907in}}%
\pgfpathlineto{\pgfqpoint{1.926272in}{0.830907in}}%
\pgfpathlineto{\pgfqpoint{1.927569in}{0.773772in}}%
\pgfpathlineto{\pgfqpoint{1.928867in}{0.773772in}}%
\pgfpathlineto{\pgfqpoint{1.928867in}{0.872950in}}%
\pgfpathlineto{\pgfqpoint{1.930164in}{0.783474in}}%
\pgfpathlineto{\pgfqpoint{1.931461in}{0.783474in}}%
\pgfpathlineto{\pgfqpoint{1.931461in}{0.803956in}}%
\pgfpathlineto{\pgfqpoint{1.932759in}{0.786708in}}%
\pgfpathlineto{\pgfqpoint{1.934056in}{0.786708in}}%
\pgfpathlineto{\pgfqpoint{1.935354in}{0.822283in}}%
\pgfpathlineto{\pgfqpoint{1.936651in}{0.822283in}}%
\pgfpathlineto{\pgfqpoint{1.937949in}{0.871871in}}%
\pgfpathlineto{\pgfqpoint{1.939246in}{0.871871in}}%
\pgfpathlineto{\pgfqpoint{1.939246in}{0.836297in}}%
\pgfpathlineto{\pgfqpoint{1.940543in}{0.880496in}}%
\pgfpathlineto{\pgfqpoint{1.941841in}{0.880496in}}%
\pgfpathlineto{\pgfqpoint{1.941841in}{0.825517in}}%
\pgfpathlineto{\pgfqpoint{1.943138in}{0.857857in}}%
\pgfpathlineto{\pgfqpoint{1.944436in}{0.857857in}}%
\pgfpathlineto{\pgfqpoint{1.944436in}{0.824439in}}%
\pgfpathlineto{\pgfqpoint{1.945733in}{0.885886in}}%
\pgfpathlineto{\pgfqpoint{1.947030in}{0.885886in}}%
\pgfpathlineto{\pgfqpoint{1.947030in}{0.839531in}}%
\pgfpathlineto{\pgfqpoint{1.948328in}{0.882652in}}%
\pgfpathlineto{\pgfqpoint{1.949625in}{0.882652in}}%
\pgfpathlineto{\pgfqpoint{1.949625in}{0.910680in}}%
\pgfpathlineto{\pgfqpoint{1.950923in}{0.835219in}}%
\pgfpathlineto{\pgfqpoint{1.952220in}{0.835219in}}%
\pgfpathlineto{\pgfqpoint{1.953518in}{0.878340in}}%
\pgfpathlineto{\pgfqpoint{1.954815in}{0.878340in}}%
\pgfpathlineto{\pgfqpoint{1.954815in}{0.892354in}}%
\pgfpathlineto{\pgfqpoint{1.956112in}{0.877262in}}%
\pgfpathlineto{\pgfqpoint{1.957410in}{0.877262in}}%
\pgfpathlineto{\pgfqpoint{1.957410in}{0.948411in}}%
\pgfpathlineto{\pgfqpoint{1.958707in}{0.845999in}}%
\pgfpathlineto{\pgfqpoint{1.960005in}{0.845999in}}%
\pgfpathlineto{\pgfqpoint{1.960005in}{0.923616in}}%
\pgfpathlineto{\pgfqpoint{1.961302in}{0.923616in}}%
\pgfpathlineto{\pgfqpoint{1.962600in}{0.923616in}}%
\pgfpathlineto{\pgfqpoint{1.962600in}{0.845999in}}%
\pgfpathlineto{\pgfqpoint{1.963897in}{0.929007in}}%
\pgfpathlineto{\pgfqpoint{1.965194in}{0.929007in}}%
\pgfpathlineto{\pgfqpoint{1.965194in}{0.891276in}}%
\pgfpathlineto{\pgfqpoint{1.966492in}{0.925772in}}%
\pgfpathlineto{\pgfqpoint{1.967789in}{0.925772in}}%
\pgfpathlineto{\pgfqpoint{1.967789in}{0.875106in}}%
\pgfpathlineto{\pgfqpoint{1.969087in}{0.966737in}}%
\pgfpathlineto{\pgfqpoint{1.970384in}{0.966737in}}%
\pgfpathlineto{\pgfqpoint{1.970384in}{0.911758in}}%
\pgfpathlineto{\pgfqpoint{1.971682in}{0.957035in}}%
\pgfpathlineto{\pgfqpoint{1.972979in}{0.957035in}}%
\pgfpathlineto{\pgfqpoint{1.972979in}{0.960269in}}%
\pgfpathlineto{\pgfqpoint{1.974276in}{0.909602in}}%
\pgfpathlineto{\pgfqpoint{1.975574in}{0.909602in}}%
\pgfpathlineto{\pgfqpoint{1.975574in}{0.936553in}}%
\pgfpathlineto{\pgfqpoint{1.976871in}{0.918226in}}%
\pgfpathlineto{\pgfqpoint{1.978169in}{0.918226in}}%
\pgfpathlineto{\pgfqpoint{1.978169in}{1.000156in}}%
\pgfpathlineto{\pgfqpoint{1.979466in}{0.921460in}}%
\pgfpathlineto{\pgfqpoint{1.980763in}{0.921460in}}%
\pgfpathlineto{\pgfqpoint{1.980763in}{0.996922in}}%
\pgfpathlineto{\pgfqpoint{1.982061in}{0.898822in}}%
\pgfpathlineto{\pgfqpoint{1.983358in}{0.898822in}}%
\pgfpathlineto{\pgfqpoint{1.983358in}{1.027106in}}%
\pgfpathlineto{\pgfqpoint{1.984656in}{1.004468in}}%
\pgfpathlineto{\pgfqpoint{1.985953in}{1.004468in}}%
\pgfpathlineto{\pgfqpoint{1.985953in}{0.935475in}}%
\pgfpathlineto{\pgfqpoint{1.987251in}{1.010936in}}%
\pgfpathlineto{\pgfqpoint{1.988548in}{1.010936in}}%
\pgfpathlineto{\pgfqpoint{1.988548in}{0.957035in}}%
\pgfpathlineto{\pgfqpoint{1.989845in}{1.048667in}}%
\pgfpathlineto{\pgfqpoint{1.991143in}{1.048667in}}%
\pgfpathlineto{\pgfqpoint{1.991143in}{0.978595in}}%
\pgfpathlineto{\pgfqpoint{1.992440in}{1.029262in}}%
\pgfpathlineto{\pgfqpoint{1.993738in}{1.029262in}}%
\pgfpathlineto{\pgfqpoint{1.993738in}{0.962425in}}%
\pgfpathlineto{\pgfqpoint{1.995035in}{1.041121in}}%
\pgfpathlineto{\pgfqpoint{1.996333in}{1.041121in}}%
\pgfpathlineto{\pgfqpoint{1.996333in}{0.982907in}}%
\pgfpathlineto{\pgfqpoint{1.997630in}{1.088553in}}%
\pgfpathlineto{\pgfqpoint{1.998927in}{1.088553in}}%
\pgfpathlineto{\pgfqpoint{2.000225in}{1.009858in}}%
\pgfpathlineto{\pgfqpoint{2.001522in}{1.009858in}}%
\pgfpathlineto{\pgfqpoint{2.001522in}{1.116582in}}%
\pgfpathlineto{\pgfqpoint{2.002820in}{0.990454in}}%
\pgfpathlineto{\pgfqpoint{2.004117in}{0.990454in}}%
\pgfpathlineto{\pgfqpoint{2.004117in}{1.057291in}}%
\pgfpathlineto{\pgfqpoint{2.005415in}{1.022794in}}%
\pgfpathlineto{\pgfqpoint{2.006712in}{1.022794in}}%
\pgfpathlineto{\pgfqpoint{2.006712in}{1.078851in}}%
\pgfpathlineto{\pgfqpoint{2.008009in}{1.016326in}}%
\pgfpathlineto{\pgfqpoint{2.009307in}{1.016326in}}%
\pgfpathlineto{\pgfqpoint{2.010604in}{1.093943in}}%
\pgfpathlineto{\pgfqpoint{2.011902in}{1.093943in}}%
\pgfpathlineto{\pgfqpoint{2.011902in}{1.017404in}}%
\pgfpathlineto{\pgfqpoint{2.013199in}{1.129518in}}%
\pgfpathlineto{\pgfqpoint{2.014496in}{1.129518in}}%
\pgfpathlineto{\pgfqpoint{2.014496in}{1.097178in}}%
\pgfpathlineto{\pgfqpoint{2.015794in}{1.169405in}}%
\pgfpathlineto{\pgfqpoint{2.017091in}{1.169405in}}%
\pgfpathlineto{\pgfqpoint{2.017091in}{1.031418in}}%
\pgfpathlineto{\pgfqpoint{2.018389in}{1.172639in}}%
\pgfpathlineto{\pgfqpoint{2.019686in}{1.172639in}}%
\pgfpathlineto{\pgfqpoint{2.019686in}{1.065915in}}%
\pgfpathlineto{\pgfqpoint{2.020984in}{1.141376in}}%
\pgfpathlineto{\pgfqpoint{2.022281in}{1.141376in}}%
\pgfpathlineto{\pgfqpoint{2.022281in}{1.150000in}}%
\pgfpathlineto{\pgfqpoint{2.023578in}{1.130596in}}%
\pgfpathlineto{\pgfqpoint{2.024876in}{1.130596in}}%
\pgfpathlineto{\pgfqpoint{2.024876in}{1.228696in}}%
\pgfpathlineto{\pgfqpoint{2.026173in}{1.093943in}}%
\pgfpathlineto{\pgfqpoint{2.027471in}{1.093943in}}%
\pgfpathlineto{\pgfqpoint{2.027471in}{1.214682in}}%
\pgfpathlineto{\pgfqpoint{2.028768in}{1.109036in}}%
\pgfpathlineto{\pgfqpoint{2.030066in}{1.109036in}}%
\pgfpathlineto{\pgfqpoint{2.030066in}{1.220072in}}%
\pgfpathlineto{\pgfqpoint{2.031363in}{1.110114in}}%
\pgfpathlineto{\pgfqpoint{2.032660in}{1.110114in}}%
\pgfpathlineto{\pgfqpoint{2.033958in}{1.332186in}}%
\pgfpathlineto{\pgfqpoint{2.035255in}{1.332186in}}%
\pgfpathlineto{\pgfqpoint{2.036553in}{1.155391in}}%
\pgfpathlineto{\pgfqpoint{2.036553in}{1.266427in}}%
\pgfpathlineto{\pgfqpoint{2.037850in}{1.266427in}}%
\pgfpathlineto{\pgfqpoint{2.037850in}{1.193121in}}%
\pgfpathlineto{\pgfqpoint{2.039148in}{1.368838in}}%
\pgfpathlineto{\pgfqpoint{2.040445in}{1.368838in}}%
\pgfpathlineto{\pgfqpoint{2.040445in}{1.215760in}}%
\pgfpathlineto{\pgfqpoint{2.041742in}{1.286909in}}%
\pgfpathlineto{\pgfqpoint{2.043040in}{1.286909in}}%
\pgfpathlineto{\pgfqpoint{2.043040in}{1.229774in}}%
\pgfpathlineto{\pgfqpoint{2.044337in}{1.374228in}}%
\pgfpathlineto{\pgfqpoint{2.045635in}{1.374228in}}%
\pgfpathlineto{\pgfqpoint{2.046932in}{1.244866in}}%
\pgfpathlineto{\pgfqpoint{2.048230in}{1.244866in}}%
\pgfpathlineto{\pgfqpoint{2.048230in}{1.466938in}}%
\pgfpathlineto{\pgfqpoint{2.049527in}{1.340810in}}%
\pgfpathlineto{\pgfqpoint{2.050824in}{1.340810in}}%
\pgfpathlineto{\pgfqpoint{2.050824in}{1.441066in}}%
\pgfpathlineto{\pgfqpoint{2.052122in}{1.304157in}}%
\pgfpathlineto{\pgfqpoint{2.053419in}{1.304157in}}%
\pgfpathlineto{\pgfqpoint{2.053419in}{1.528385in}}%
\pgfpathlineto{\pgfqpoint{2.054717in}{1.319249in}}%
\pgfpathlineto{\pgfqpoint{2.056014in}{1.319249in}}%
\pgfpathlineto{\pgfqpoint{2.057311in}{1.576896in}}%
\pgfpathlineto{\pgfqpoint{2.058609in}{1.576896in}}%
\pgfpathlineto{\pgfqpoint{2.058609in}{1.350512in}}%
\pgfpathlineto{\pgfqpoint{2.059906in}{1.485264in}}%
\pgfpathlineto{\pgfqpoint{2.061204in}{1.485264in}}%
\pgfpathlineto{\pgfqpoint{2.061204in}{1.416271in}}%
\pgfpathlineto{\pgfqpoint{2.062501in}{1.664216in}}%
\pgfpathlineto{\pgfqpoint{2.063799in}{1.664216in}}%
\pgfpathlineto{\pgfqpoint{2.063799in}{1.493889in}}%
\pgfpathlineto{\pgfqpoint{2.065096in}{1.594144in}}%
\pgfpathlineto{\pgfqpoint{2.066393in}{1.594144in}}%
\pgfpathlineto{\pgfqpoint{2.066393in}{1.513293in}}%
\pgfpathlineto{\pgfqpoint{2.067691in}{1.750457in}}%
\pgfpathlineto{\pgfqpoint{2.068988in}{1.750457in}}%
\pgfpathlineto{\pgfqpoint{2.070286in}{1.551024in}}%
\pgfpathlineto{\pgfqpoint{2.071583in}{1.551024in}}%
\pgfpathlineto{\pgfqpoint{2.071583in}{1.798968in}}%
\pgfpathlineto{\pgfqpoint{2.072881in}{1.662060in}}%
\pgfpathlineto{\pgfqpoint{2.074178in}{1.662060in}}%
\pgfpathlineto{\pgfqpoint{2.074178in}{1.714883in}}%
\pgfpathlineto{\pgfqpoint{2.075475in}{1.612471in}}%
\pgfpathlineto{\pgfqpoint{2.076773in}{1.612471in}}%
\pgfpathlineto{\pgfqpoint{2.076773in}{1.932642in}}%
\pgfpathlineto{\pgfqpoint{2.078070in}{1.704102in}}%
\pgfpathlineto{\pgfqpoint{2.079368in}{1.704102in}}%
\pgfpathlineto{\pgfqpoint{2.079368in}{1.817294in}}%
\pgfpathlineto{\pgfqpoint{2.080665in}{1.747223in}}%
\pgfpathlineto{\pgfqpoint{2.081963in}{1.747223in}}%
\pgfpathlineto{\pgfqpoint{2.081963in}{1.959593in}}%
\pgfpathlineto{\pgfqpoint{2.083260in}{1.939111in}}%
\pgfpathlineto{\pgfqpoint{2.084557in}{1.939111in}}%
\pgfpathlineto{\pgfqpoint{2.084557in}{1.751535in}}%
\pgfpathlineto{\pgfqpoint{2.085855in}{2.096501in}}%
\pgfpathlineto{\pgfqpoint{2.087152in}{2.096501in}}%
\pgfpathlineto{\pgfqpoint{2.087152in}{1.880898in}}%
\pgfpathlineto{\pgfqpoint{2.088450in}{2.122374in}}%
\pgfpathlineto{\pgfqpoint{2.089747in}{2.122374in}}%
\pgfpathlineto{\pgfqpoint{2.089747in}{1.900302in}}%
\pgfpathlineto{\pgfqpoint{2.091044in}{2.153636in}}%
\pgfpathlineto{\pgfqpoint{2.092342in}{2.153636in}}%
\pgfpathlineto{\pgfqpoint{2.092342in}{1.939111in}}%
\pgfpathlineto{\pgfqpoint{2.093639in}{2.129920in}}%
\pgfpathlineto{\pgfqpoint{2.094937in}{2.129920in}}%
\pgfpathlineto{\pgfqpoint{2.094937in}{2.273297in}}%
\pgfpathlineto{\pgfqpoint{2.096234in}{2.078175in}}%
\pgfpathlineto{\pgfqpoint{2.097532in}{2.078175in}}%
\pgfpathlineto{\pgfqpoint{2.097532in}{2.363850in}}%
\pgfpathlineto{\pgfqpoint{2.098829in}{2.033976in}}%
\pgfpathlineto{\pgfqpoint{2.100126in}{2.033976in}}%
\pgfpathlineto{\pgfqpoint{2.100126in}{2.444702in}}%
\pgfpathlineto{\pgfqpoint{2.101424in}{2.222630in}}%
\pgfpathlineto{\pgfqpoint{2.102721in}{2.222630in}}%
\pgfpathlineto{\pgfqpoint{2.102721in}{2.492134in}}%
\pgfpathlineto{\pgfqpoint{2.104019in}{2.322885in}}%
\pgfpathlineto{\pgfqpoint{2.105316in}{2.322885in}}%
\pgfpathlineto{\pgfqpoint{2.105316in}{2.599936in}}%
\pgfpathlineto{\pgfqpoint{2.106614in}{2.388645in}}%
\pgfpathlineto{\pgfqpoint{2.107911in}{2.388645in}}%
\pgfpathlineto{\pgfqpoint{2.107911in}{2.367084in}}%
\pgfpathlineto{\pgfqpoint{2.109208in}{2.715284in}}%
\pgfpathlineto{\pgfqpoint{2.110506in}{2.715284in}}%
\pgfpathlineto{\pgfqpoint{2.110506in}{2.395113in}}%
\pgfpathlineto{\pgfqpoint{2.111803in}{2.717440in}}%
\pgfpathlineto{\pgfqpoint{2.113101in}{2.717440in}}%
\pgfpathlineto{\pgfqpoint{2.113101in}{2.529865in}}%
\pgfpathlineto{\pgfqpoint{2.114398in}{2.943825in}}%
\pgfpathlineto{\pgfqpoint{2.115696in}{2.943825in}}%
\pgfpathlineto{\pgfqpoint{2.115696in}{2.472730in}}%
\pgfpathlineto{\pgfqpoint{2.116993in}{2.806916in}}%
\pgfpathlineto{\pgfqpoint{2.118290in}{2.806916in}}%
\pgfpathlineto{\pgfqpoint{2.118290in}{3.022520in}}%
\pgfpathlineto{\pgfqpoint{2.119588in}{2.757327in}}%
\pgfpathlineto{\pgfqpoint{2.120885in}{2.757327in}}%
\pgfpathlineto{\pgfqpoint{2.120885in}{2.914718in}}%
\pgfpathlineto{\pgfqpoint{2.122183in}{2.622575in}}%
\pgfpathlineto{\pgfqpoint{2.123480in}{2.622575in}}%
\pgfpathlineto{\pgfqpoint{2.123480in}{3.255372in}}%
\pgfpathlineto{\pgfqpoint{2.124777in}{2.770263in}}%
\pgfpathlineto{\pgfqpoint{2.126075in}{2.770263in}}%
\pgfpathlineto{\pgfqpoint{2.126075in}{3.076421in}}%
\pgfpathlineto{\pgfqpoint{2.127372in}{2.756249in}}%
\pgfpathlineto{\pgfqpoint{2.128670in}{2.756249in}}%
\pgfpathlineto{\pgfqpoint{2.128670in}{3.274776in}}%
\pgfpathlineto{\pgfqpoint{2.129967in}{3.174521in}}%
\pgfpathlineto{\pgfqpoint{2.131265in}{3.174521in}}%
\pgfpathlineto{\pgfqpoint{2.131265in}{3.017130in}}%
\pgfpathlineto{\pgfqpoint{2.132562in}{3.538891in}}%
\pgfpathlineto{\pgfqpoint{2.133859in}{3.538891in}}%
\pgfpathlineto{\pgfqpoint{2.133859in}{3.017130in}}%
\pgfpathlineto{\pgfqpoint{2.135157in}{3.337302in}}%
\pgfpathlineto{\pgfqpoint{2.136454in}{3.337302in}}%
\pgfpathlineto{\pgfqpoint{2.136454in}{3.028988in}}%
\pgfpathlineto{\pgfqpoint{2.137752in}{3.458040in}}%
\pgfpathlineto{\pgfqpoint{2.139049in}{3.458040in}}%
\pgfpathlineto{\pgfqpoint{2.139049in}{2.996647in}}%
\pgfpathlineto{\pgfqpoint{2.140347in}{3.368564in}}%
\pgfpathlineto{\pgfqpoint{2.141644in}{3.368564in}}%
\pgfpathlineto{\pgfqpoint{2.141644in}{3.484990in}}%
\pgfpathlineto{\pgfqpoint{2.142941in}{3.124932in}}%
\pgfpathlineto{\pgfqpoint{2.144239in}{3.124932in}}%
\pgfpathlineto{\pgfqpoint{2.144239in}{3.418153in}}%
\pgfpathlineto{\pgfqpoint{2.145536in}{3.043002in}}%
\pgfpathlineto{\pgfqpoint{2.146834in}{3.043002in}}%
\pgfpathlineto{\pgfqpoint{2.146834in}{3.362096in}}%
\pgfpathlineto{\pgfqpoint{2.148131in}{3.136790in}}%
\pgfpathlineto{\pgfqpoint{2.149429in}{3.136790in}}%
\pgfpathlineto{\pgfqpoint{2.149429in}{3.473132in}}%
\pgfpathlineto{\pgfqpoint{2.150726in}{3.198237in}}%
\pgfpathlineto{\pgfqpoint{2.152023in}{3.198237in}}%
\pgfpathlineto{\pgfqpoint{2.153321in}{3.584168in}}%
\pgfpathlineto{\pgfqpoint{2.153321in}{3.134634in}}%
\pgfpathlineto{\pgfqpoint{2.154618in}{3.134634in}}%
\pgfpathlineto{\pgfqpoint{2.155916in}{3.521643in}}%
\pgfpathlineto{\pgfqpoint{2.157213in}{3.521643in}}%
\pgfpathlineto{\pgfqpoint{2.157213in}{2.996647in}}%
\pgfpathlineto{\pgfqpoint{2.158510in}{3.588480in}}%
\pgfpathlineto{\pgfqpoint{2.159808in}{3.588480in}}%
\pgfpathlineto{\pgfqpoint{2.159808in}{3.009584in}}%
\pgfpathlineto{\pgfqpoint{2.161105in}{3.646693in}}%
\pgfpathlineto{\pgfqpoint{2.162403in}{3.646693in}}%
\pgfpathlineto{\pgfqpoint{2.162403in}{2.940590in}}%
\pgfpathlineto{\pgfqpoint{2.163700in}{3.448338in}}%
\pgfpathlineto{\pgfqpoint{2.164998in}{3.448338in}}%
\pgfpathlineto{\pgfqpoint{2.164998in}{2.981555in}}%
\pgfpathlineto{\pgfqpoint{2.166295in}{3.513019in}}%
\pgfpathlineto{\pgfqpoint{2.167592in}{3.513019in}}%
\pgfpathlineto{\pgfqpoint{2.168890in}{2.898548in}}%
\pgfpathlineto{\pgfqpoint{2.170187in}{2.898548in}}%
\pgfpathlineto{\pgfqpoint{2.170187in}{3.413841in}}%
\pgfpathlineto{\pgfqpoint{2.171485in}{2.957839in}}%
\pgfpathlineto{\pgfqpoint{2.172782in}{2.957839in}}%
\pgfpathlineto{\pgfqpoint{2.172782in}{3.103371in}}%
\pgfpathlineto{\pgfqpoint{2.174080in}{2.948137in}}%
\pgfpathlineto{\pgfqpoint{2.175377in}{2.948137in}}%
\pgfpathlineto{\pgfqpoint{2.175377in}{3.233812in}}%
\pgfpathlineto{\pgfqpoint{2.176674in}{2.822008in}}%
\pgfpathlineto{\pgfqpoint{2.177972in}{2.822008in}}%
\pgfpathlineto{\pgfqpoint{2.179269in}{3.450494in}}%
\pgfpathlineto{\pgfqpoint{2.180567in}{3.450494in}}%
\pgfpathlineto{\pgfqpoint{2.180567in}{2.799370in}}%
\pgfpathlineto{\pgfqpoint{2.181864in}{3.092591in}}%
\pgfpathlineto{\pgfqpoint{2.183162in}{3.092591in}}%
\pgfpathlineto{\pgfqpoint{2.183162in}{2.751937in}}%
\pgfpathlineto{\pgfqpoint{2.184459in}{3.117386in}}%
\pgfpathlineto{\pgfqpoint{2.185756in}{3.117386in}}%
\pgfpathlineto{\pgfqpoint{2.185756in}{2.660305in}}%
\pgfpathlineto{\pgfqpoint{2.187054in}{3.014974in}}%
\pgfpathlineto{\pgfqpoint{2.188351in}{3.014974in}}%
\pgfpathlineto{\pgfqpoint{2.188351in}{2.658149in}}%
\pgfpathlineto{\pgfqpoint{2.189649in}{3.121698in}}%
\pgfpathlineto{\pgfqpoint{2.190946in}{3.121698in}}%
\pgfpathlineto{\pgfqpoint{2.192243in}{2.510461in}}%
\pgfpathlineto{\pgfqpoint{2.193541in}{2.510461in}}%
\pgfpathlineto{\pgfqpoint{2.193541in}{2.879143in}}%
\pgfpathlineto{\pgfqpoint{2.194838in}{2.468418in}}%
\pgfpathlineto{\pgfqpoint{2.196136in}{2.468418in}}%
\pgfpathlineto{\pgfqpoint{2.196136in}{2.688334in}}%
\pgfpathlineto{\pgfqpoint{2.197433in}{2.479198in}}%
\pgfpathlineto{\pgfqpoint{2.198731in}{2.479198in}}%
\pgfpathlineto{\pgfqpoint{2.198731in}{2.710972in}}%
\pgfpathlineto{\pgfqpoint{2.200028in}{2.416673in}}%
\pgfpathlineto{\pgfqpoint{2.201325in}{2.416673in}}%
\pgfpathlineto{\pgfqpoint{2.202623in}{2.644135in}}%
\pgfpathlineto{\pgfqpoint{2.203920in}{2.644135in}}%
\pgfpathlineto{\pgfqpoint{2.203920in}{2.178431in}}%
\pgfpathlineto{\pgfqpoint{2.205218in}{2.386489in}}%
\pgfpathlineto{\pgfqpoint{2.206515in}{2.386489in}}%
\pgfpathlineto{\pgfqpoint{2.206515in}{2.099735in}}%
\pgfpathlineto{\pgfqpoint{2.207813in}{2.451170in}}%
\pgfpathlineto{\pgfqpoint{2.209110in}{2.451170in}}%
\pgfpathlineto{\pgfqpoint{2.209110in}{2.050147in}}%
\pgfpathlineto{\pgfqpoint{2.210407in}{2.197835in}}%
\pgfpathlineto{\pgfqpoint{2.211705in}{2.197835in}}%
\pgfpathlineto{\pgfqpoint{2.211705in}{2.063083in}}%
\pgfpathlineto{\pgfqpoint{2.213002in}{2.237722in}}%
\pgfpathlineto{\pgfqpoint{2.214300in}{2.237722in}}%
\pgfpathlineto{\pgfqpoint{2.215597in}{1.874429in}}%
\pgfpathlineto{\pgfqpoint{2.216895in}{1.874429in}}%
\pgfpathlineto{\pgfqpoint{2.216895in}{2.125608in}}%
\pgfpathlineto{\pgfqpoint{2.218192in}{1.808670in}}%
\pgfpathlineto{\pgfqpoint{2.219489in}{1.808670in}}%
\pgfpathlineto{\pgfqpoint{2.219489in}{1.967139in}}%
\pgfpathlineto{\pgfqpoint{2.220787in}{1.772018in}}%
\pgfpathlineto{\pgfqpoint{2.222084in}{1.772018in}}%
\pgfpathlineto{\pgfqpoint{2.222084in}{1.955281in}}%
\pgfpathlineto{\pgfqpoint{2.223382in}{1.623251in}}%
\pgfpathlineto{\pgfqpoint{2.224679in}{1.623251in}}%
\pgfpathlineto{\pgfqpoint{2.224679in}{1.849635in}}%
\pgfpathlineto{\pgfqpoint{2.225976in}{1.640499in}}%
\pgfpathlineto{\pgfqpoint{2.227274in}{1.640499in}}%
\pgfpathlineto{\pgfqpoint{2.227274in}{1.818372in}}%
\pgfpathlineto{\pgfqpoint{2.228571in}{1.701946in}}%
\pgfpathlineto{\pgfqpoint{2.229869in}{1.701946in}}%
\pgfpathlineto{\pgfqpoint{2.229869in}{1.627563in}}%
\pgfpathlineto{\pgfqpoint{2.231166in}{1.760159in}}%
\pgfpathlineto{\pgfqpoint{2.232464in}{1.760159in}}%
\pgfpathlineto{\pgfqpoint{2.232464in}{1.547790in}}%
\pgfpathlineto{\pgfqpoint{2.233761in}{1.615705in}}%
\pgfpathlineto{\pgfqpoint{2.235058in}{1.615705in}}%
\pgfpathlineto{\pgfqpoint{2.235058in}{1.470172in}}%
\pgfpathlineto{\pgfqpoint{2.236356in}{1.628641in}}%
\pgfpathlineto{\pgfqpoint{2.237653in}{1.628641in}}%
\pgfpathlineto{\pgfqpoint{2.237653in}{1.417349in}}%
\pgfpathlineto{\pgfqpoint{2.238951in}{1.513293in}}%
\pgfpathlineto{\pgfqpoint{2.240248in}{1.513293in}}%
\pgfpathlineto{\pgfqpoint{2.240248in}{1.568272in}}%
\pgfpathlineto{\pgfqpoint{2.241546in}{1.414115in}}%
\pgfpathlineto{\pgfqpoint{2.242843in}{1.414115in}}%
\pgfpathlineto{\pgfqpoint{2.242843in}{1.471250in}}%
\pgfpathlineto{\pgfqpoint{2.244140in}{1.345122in}}%
\pgfpathlineto{\pgfqpoint{2.245438in}{1.345122in}}%
\pgfpathlineto{\pgfqpoint{2.245438in}{1.411959in}}%
\pgfpathlineto{\pgfqpoint{2.246735in}{1.290143in}}%
\pgfpathlineto{\pgfqpoint{2.248033in}{1.290143in}}%
\pgfpathlineto{\pgfqpoint{2.248033in}{1.399023in}}%
\pgfpathlineto{\pgfqpoint{2.249330in}{1.256724in}}%
\pgfpathlineto{\pgfqpoint{2.250628in}{1.256724in}}%
\pgfpathlineto{\pgfqpoint{2.250628in}{1.374228in}}%
\pgfpathlineto{\pgfqpoint{2.251925in}{1.354824in}}%
\pgfpathlineto{\pgfqpoint{2.253222in}{1.354824in}}%
\pgfpathlineto{\pgfqpoint{2.253222in}{1.210370in}}%
\pgfpathlineto{\pgfqpoint{2.254520in}{1.326796in}}%
\pgfpathlineto{\pgfqpoint{2.255817in}{1.326796in}}%
\pgfpathlineto{\pgfqpoint{2.255817in}{1.186653in}}%
\pgfpathlineto{\pgfqpoint{2.257115in}{1.281519in}}%
\pgfpathlineto{\pgfqpoint{2.258412in}{1.281519in}}%
\pgfpathlineto{\pgfqpoint{2.258412in}{1.192043in}}%
\pgfpathlineto{\pgfqpoint{2.259709in}{1.255646in}}%
\pgfpathlineto{\pgfqpoint{2.261007in}{1.255646in}}%
\pgfpathlineto{\pgfqpoint{2.261007in}{1.150000in}}%
\pgfpathlineto{\pgfqpoint{2.262304in}{1.188809in}}%
\pgfpathlineto{\pgfqpoint{2.263602in}{1.188809in}}%
\pgfpathlineto{\pgfqpoint{2.263602in}{1.221150in}}%
\pgfpathlineto{\pgfqpoint{2.264899in}{1.103646in}}%
\pgfpathlineto{\pgfqpoint{2.266197in}{1.103646in}}%
\pgfpathlineto{\pgfqpoint{2.267494in}{1.200667in}}%
\pgfpathlineto{\pgfqpoint{2.267494in}{1.089631in}}%
\pgfpathlineto{\pgfqpoint{2.268791in}{1.089631in}}%
\pgfpathlineto{\pgfqpoint{2.268791in}{1.167249in}}%
\pgfpathlineto{\pgfqpoint{2.270089in}{1.052979in}}%
\pgfpathlineto{\pgfqpoint{2.271386in}{1.052979in}}%
\pgfpathlineto{\pgfqpoint{2.271386in}{1.135986in}}%
\pgfpathlineto{\pgfqpoint{2.272684in}{1.032496in}}%
\pgfpathlineto{\pgfqpoint{2.273981in}{1.032496in}}%
\pgfpathlineto{\pgfqpoint{2.273981in}{1.119816in}}%
\pgfpathlineto{\pgfqpoint{2.275279in}{1.062681in}}%
\pgfpathlineto{\pgfqpoint{2.276576in}{1.062681in}}%
\pgfpathlineto{\pgfqpoint{2.276576in}{1.040042in}}%
\pgfpathlineto{\pgfqpoint{2.277873in}{1.065915in}}%
\pgfpathlineto{\pgfqpoint{2.279171in}{1.065915in}}%
\pgfpathlineto{\pgfqpoint{2.279171in}{1.040042in}}%
\pgfpathlineto{\pgfqpoint{2.280468in}{1.042199in}}%
\pgfpathlineto{\pgfqpoint{2.281766in}{1.042199in}}%
\pgfpathlineto{\pgfqpoint{2.281766in}{1.012014in}}%
\pgfpathlineto{\pgfqpoint{2.283063in}{1.036808in}}%
\pgfpathlineto{\pgfqpoint{2.284361in}{1.036808in}}%
\pgfpathlineto{\pgfqpoint{2.284361in}{0.953801in}}%
\pgfpathlineto{\pgfqpoint{2.285658in}{0.964581in}}%
\pgfpathlineto{\pgfqpoint{2.286955in}{0.964581in}}%
\pgfpathlineto{\pgfqpoint{2.286955in}{1.036808in}}%
\pgfpathlineto{\pgfqpoint{2.288253in}{0.954879in}}%
\pgfpathlineto{\pgfqpoint{2.289550in}{0.954879in}}%
\pgfpathlineto{\pgfqpoint{2.289550in}{1.026028in}}%
\pgfpathlineto{\pgfqpoint{2.290848in}{0.959191in}}%
\pgfpathlineto{\pgfqpoint{2.292145in}{0.959191in}}%
\pgfpathlineto{\pgfqpoint{2.292145in}{0.990454in}}%
\pgfpathlineto{\pgfqpoint{2.293442in}{0.907446in}}%
\pgfpathlineto{\pgfqpoint{2.294740in}{0.907446in}}%
\pgfpathlineto{\pgfqpoint{2.294740in}{0.945177in}}%
\pgfpathlineto{\pgfqpoint{2.296037in}{0.897744in}}%
\pgfpathlineto{\pgfqpoint{2.297335in}{0.897744in}}%
\pgfpathlineto{\pgfqpoint{2.297335in}{0.989376in}}%
\pgfpathlineto{\pgfqpoint{2.298632in}{0.916070in}}%
\pgfpathlineto{\pgfqpoint{2.299930in}{0.916070in}}%
\pgfpathlineto{\pgfqpoint{2.299930in}{0.896666in}}%
\pgfpathlineto{\pgfqpoint{2.301227in}{0.948411in}}%
\pgfpathlineto{\pgfqpoint{2.302524in}{0.948411in}}%
\pgfpathlineto{\pgfqpoint{2.302524in}{0.908524in}}%
\pgfpathlineto{\pgfqpoint{2.303822in}{0.908524in}}%
\pgfpathlineto{\pgfqpoint{2.305119in}{0.908524in}}%
\pgfpathlineto{\pgfqpoint{2.305119in}{0.888042in}}%
\pgfpathlineto{\pgfqpoint{2.306417in}{0.932241in}}%
\pgfpathlineto{\pgfqpoint{2.307714in}{0.932241in}}%
\pgfpathlineto{\pgfqpoint{2.307714in}{0.865403in}}%
\pgfpathlineto{\pgfqpoint{2.309012in}{0.900978in}}%
\pgfpathlineto{\pgfqpoint{2.310309in}{0.900978in}}%
\pgfpathlineto{\pgfqpoint{2.310309in}{0.855701in}}%
\pgfpathlineto{\pgfqpoint{2.311606in}{0.894510in}}%
\pgfpathlineto{\pgfqpoint{2.312904in}{0.894510in}}%
\pgfpathlineto{\pgfqpoint{2.314201in}{0.852467in}}%
\pgfpathlineto{\pgfqpoint{2.315499in}{0.852467in}}%
\pgfpathlineto{\pgfqpoint{2.315499in}{0.865403in}}%
\pgfpathlineto{\pgfqpoint{2.316796in}{0.831985in}}%
\pgfpathlineto{\pgfqpoint{2.318094in}{0.831985in}}%
\pgfpathlineto{\pgfqpoint{2.318094in}{0.850311in}}%
\pgfpathlineto{\pgfqpoint{2.319391in}{0.829829in}}%
\pgfpathlineto{\pgfqpoint{2.320688in}{0.829829in}}%
\pgfpathlineto{\pgfqpoint{2.320688in}{0.841687in}}%
\pgfpathlineto{\pgfqpoint{2.321986in}{0.815814in}}%
\pgfpathlineto{\pgfqpoint{2.323283in}{0.815814in}}%
\pgfpathlineto{\pgfqpoint{2.324581in}{0.875106in}}%
\pgfpathlineto{\pgfqpoint{2.325878in}{0.875106in}}%
\pgfpathlineto{\pgfqpoint{2.325878in}{0.798566in}}%
\pgfpathlineto{\pgfqpoint{2.327175in}{0.810424in}}%
\pgfpathlineto{\pgfqpoint{2.328473in}{0.810424in}}%
\pgfpathlineto{\pgfqpoint{2.329770in}{0.822283in}}%
\pgfpathlineto{\pgfqpoint{2.331068in}{0.822283in}}%
\pgfpathlineto{\pgfqpoint{2.331068in}{0.801800in}}%
\pgfpathlineto{\pgfqpoint{2.332365in}{0.806112in}}%
\pgfpathlineto{\pgfqpoint{2.333663in}{0.806112in}}%
\pgfpathlineto{\pgfqpoint{2.333663in}{0.791020in}}%
\pgfpathlineto{\pgfqpoint{2.334960in}{0.838453in}}%
\pgfpathlineto{\pgfqpoint{2.336257in}{0.838453in}}%
\pgfpathlineto{\pgfqpoint{2.337555in}{0.752211in}}%
\pgfpathlineto{\pgfqpoint{2.338852in}{0.752211in}}%
\pgfpathlineto{\pgfqpoint{2.338852in}{0.794254in}}%
\pgfpathlineto{\pgfqpoint{2.340150in}{0.788864in}}%
\pgfpathlineto{\pgfqpoint{2.341447in}{0.788864in}}%
\pgfpathlineto{\pgfqpoint{2.342745in}{0.745743in}}%
\pgfpathlineto{\pgfqpoint{2.344042in}{0.745743in}}%
\pgfpathlineto{\pgfqpoint{2.344042in}{0.772694in}}%
\pgfpathlineto{\pgfqpoint{2.345339in}{0.740353in}}%
\pgfpathlineto{\pgfqpoint{2.346637in}{0.740353in}}%
\pgfpathlineto{\pgfqpoint{2.347934in}{0.812580in}}%
\pgfpathlineto{\pgfqpoint{2.349232in}{0.812580in}}%
\pgfpathlineto{\pgfqpoint{2.349232in}{0.742509in}}%
\pgfpathlineto{\pgfqpoint{2.350529in}{0.774850in}}%
\pgfpathlineto{\pgfqpoint{2.351827in}{0.774850in}}%
\pgfpathlineto{\pgfqpoint{2.351827in}{0.737119in}}%
\pgfpathlineto{\pgfqpoint{2.353124in}{0.771616in}}%
\pgfpathlineto{\pgfqpoint{2.354421in}{0.771616in}}%
\pgfpathlineto{\pgfqpoint{2.355719in}{0.727417in}}%
\pgfpathlineto{\pgfqpoint{2.355719in}{0.773772in}}%
\pgfpathlineto{\pgfqpoint{2.357016in}{0.773772in}}%
\pgfpathlineto{\pgfqpoint{2.357016in}{0.728495in}}%
\pgfpathlineto{\pgfqpoint{2.358314in}{0.768382in}}%
\pgfpathlineto{\pgfqpoint{2.359611in}{0.768382in}}%
\pgfpathlineto{\pgfqpoint{2.360908in}{0.715559in}}%
\pgfpathlineto{\pgfqpoint{2.362206in}{0.715559in}}%
\pgfpathlineto{\pgfqpoint{2.362206in}{0.733885in}}%
\pgfpathlineto{\pgfqpoint{2.363503in}{0.732807in}}%
\pgfpathlineto{\pgfqpoint{2.366098in}{0.732807in}}%
\pgfpathlineto{\pgfqpoint{2.366098in}{0.730651in}}%
\pgfpathlineto{\pgfqpoint{2.367396in}{0.740353in}}%
\pgfpathlineto{\pgfqpoint{2.368693in}{0.740353in}}%
\pgfpathlineto{\pgfqpoint{2.368693in}{0.716637in}}%
\pgfpathlineto{\pgfqpoint{2.369990in}{0.722027in}}%
\pgfpathlineto{\pgfqpoint{2.371288in}{0.722027in}}%
\pgfpathlineto{\pgfqpoint{2.371288in}{0.689686in}}%
\pgfpathlineto{\pgfqpoint{2.372585in}{0.720949in}}%
\pgfpathlineto{\pgfqpoint{2.373883in}{0.720949in}}%
\pgfpathlineto{\pgfqpoint{2.375180in}{0.681062in}}%
\pgfpathlineto{\pgfqpoint{2.376478in}{0.681062in}}%
\pgfpathlineto{\pgfqpoint{2.376478in}{0.734963in}}%
\pgfpathlineto{\pgfqpoint{2.377775in}{0.687530in}}%
\pgfpathlineto{\pgfqpoint{2.379072in}{0.687530in}}%
\pgfpathlineto{\pgfqpoint{2.379072in}{0.701544in}}%
\pgfpathlineto{\pgfqpoint{2.380370in}{0.692920in}}%
\pgfpathlineto{\pgfqpoint{2.381667in}{0.692920in}}%
\pgfpathlineto{\pgfqpoint{2.381667in}{0.708013in}}%
\pgfpathlineto{\pgfqpoint{2.382965in}{0.674594in}}%
\pgfpathlineto{\pgfqpoint{2.384262in}{0.674594in}}%
\pgfpathlineto{\pgfqpoint{2.384262in}{0.711247in}}%
\pgfpathlineto{\pgfqpoint{2.385560in}{0.711247in}}%
\pgfpathlineto{\pgfqpoint{2.386857in}{0.711247in}}%
\pgfpathlineto{\pgfqpoint{2.386857in}{0.672438in}}%
\pgfpathlineto{\pgfqpoint{2.388154in}{0.712325in}}%
\pgfpathlineto{\pgfqpoint{2.389452in}{0.712325in}}%
\pgfpathlineto{\pgfqpoint{2.389452in}{0.677828in}}%
\pgfpathlineto{\pgfqpoint{2.390749in}{0.717715in}}%
\pgfpathlineto{\pgfqpoint{2.392047in}{0.717715in}}%
\pgfpathlineto{\pgfqpoint{2.392047in}{0.665970in}}%
\pgfpathlineto{\pgfqpoint{2.393344in}{0.728495in}}%
\pgfpathlineto{\pgfqpoint{2.394642in}{0.728495in}}%
\pgfpathlineto{\pgfqpoint{2.394642in}{0.678906in}}%
\pgfpathlineto{\pgfqpoint{2.395939in}{0.717715in}}%
\pgfpathlineto{\pgfqpoint{2.397236in}{0.717715in}}%
\pgfpathlineto{\pgfqpoint{2.398534in}{0.674594in}}%
\pgfpathlineto{\pgfqpoint{2.399831in}{0.674594in}}%
\pgfpathlineto{\pgfqpoint{2.399831in}{0.672438in}}%
\pgfpathlineto{\pgfqpoint{2.401129in}{0.685374in}}%
\pgfpathlineto{\pgfqpoint{2.402426in}{0.685374in}}%
\pgfpathlineto{\pgfqpoint{2.403723in}{0.651956in}}%
\pgfpathlineto{\pgfqpoint{2.405021in}{0.651956in}}%
\pgfpathlineto{\pgfqpoint{2.405021in}{0.698310in}}%
\pgfpathlineto{\pgfqpoint{2.406318in}{0.657346in}}%
\pgfpathlineto{\pgfqpoint{2.407616in}{0.657346in}}%
\pgfpathlineto{\pgfqpoint{2.407616in}{0.687530in}}%
\pgfpathlineto{\pgfqpoint{2.408913in}{0.686452in}}%
\pgfpathlineto{\pgfqpoint{2.410211in}{0.686452in}}%
\pgfpathlineto{\pgfqpoint{2.410211in}{0.644409in}}%
\pgfpathlineto{\pgfqpoint{2.411508in}{0.679984in}}%
\pgfpathlineto{\pgfqpoint{2.412805in}{0.679984in}}%
\pgfpathlineto{\pgfqpoint{2.412805in}{0.658424in}}%
\pgfpathlineto{\pgfqpoint{2.414103in}{0.678906in}}%
\pgfpathlineto{\pgfqpoint{2.415400in}{0.678906in}}%
\pgfpathlineto{\pgfqpoint{2.415400in}{0.665970in}}%
\pgfpathlineto{\pgfqpoint{2.416698in}{0.677828in}}%
\pgfpathlineto{\pgfqpoint{2.417995in}{0.677828in}}%
\pgfpathlineto{\pgfqpoint{2.417995in}{0.661658in}}%
\pgfpathlineto{\pgfqpoint{2.419293in}{0.700466in}}%
\pgfpathlineto{\pgfqpoint{2.420590in}{0.700466in}}%
\pgfpathlineto{\pgfqpoint{2.420590in}{0.667048in}}%
\pgfpathlineto{\pgfqpoint{2.421887in}{0.673516in}}%
\pgfpathlineto{\pgfqpoint{2.423185in}{0.673516in}}%
\pgfpathlineto{\pgfqpoint{2.423185in}{0.681062in}}%
\pgfpathlineto{\pgfqpoint{2.424482in}{0.654112in}}%
\pgfpathlineto{\pgfqpoint{2.425780in}{0.654112in}}%
\pgfpathlineto{\pgfqpoint{2.427077in}{0.643331in}}%
\pgfpathlineto{\pgfqpoint{2.428375in}{0.643331in}}%
\pgfpathlineto{\pgfqpoint{2.428375in}{0.698310in}}%
\pgfpathlineto{\pgfqpoint{2.429672in}{0.646565in}}%
\pgfpathlineto{\pgfqpoint{2.430969in}{0.646565in}}%
\pgfpathlineto{\pgfqpoint{2.432267in}{0.685374in}}%
\pgfpathlineto{\pgfqpoint{2.433564in}{0.685374in}}%
\pgfpathlineto{\pgfqpoint{2.433564in}{0.659502in}}%
\pgfpathlineto{\pgfqpoint{2.434862in}{0.661658in}}%
\pgfpathlineto{\pgfqpoint{2.436159in}{0.661658in}}%
\pgfpathlineto{\pgfqpoint{2.436159in}{0.658424in}}%
\pgfpathlineto{\pgfqpoint{2.437456in}{0.658424in}}%
\pgfpathlineto{\pgfqpoint{2.438754in}{0.658424in}}%
\pgfpathlineto{\pgfqpoint{2.438754in}{0.656268in}}%
\pgfpathlineto{\pgfqpoint{2.440051in}{0.669204in}}%
\pgfpathlineto{\pgfqpoint{2.441349in}{0.669204in}}%
\pgfpathlineto{\pgfqpoint{2.441349in}{0.634707in}}%
\pgfpathlineto{\pgfqpoint{2.442646in}{0.660580in}}%
\pgfpathlineto{\pgfqpoint{2.443944in}{0.660580in}}%
\pgfpathlineto{\pgfqpoint{2.443944in}{0.627161in}}%
\pgfpathlineto{\pgfqpoint{2.445241in}{0.651956in}}%
\pgfpathlineto{\pgfqpoint{2.446538in}{0.651956in}}%
\pgfpathlineto{\pgfqpoint{2.447836in}{0.639019in}}%
\pgfpathlineto{\pgfqpoint{2.449133in}{0.639019in}}%
\pgfpathlineto{\pgfqpoint{2.449133in}{0.641175in}}%
\pgfpathlineto{\pgfqpoint{2.450431in}{0.622849in}}%
\pgfpathlineto{\pgfqpoint{2.451728in}{0.622849in}}%
\pgfpathlineto{\pgfqpoint{2.453026in}{0.634707in}}%
\pgfpathlineto{\pgfqpoint{2.454323in}{0.634707in}}%
\pgfpathlineto{\pgfqpoint{2.454323in}{0.644409in}}%
\pgfpathlineto{\pgfqpoint{2.455620in}{0.622849in}}%
\pgfpathlineto{\pgfqpoint{2.456918in}{0.622849in}}%
\pgfpathlineto{\pgfqpoint{2.458215in}{0.650878in}}%
\pgfpathlineto{\pgfqpoint{2.460810in}{0.649800in}}%
\pgfpathlineto{\pgfqpoint{2.460810in}{0.634707in}}%
\pgfpathlineto{\pgfqpoint{2.462108in}{0.634707in}}%
\pgfpathlineto{\pgfqpoint{2.463405in}{0.634707in}}%
\pgfpathlineto{\pgfqpoint{2.463405in}{0.642253in}}%
\pgfpathlineto{\pgfqpoint{2.464702in}{0.633629in}}%
\pgfpathlineto{\pgfqpoint{2.466000in}{0.633629in}}%
\pgfpathlineto{\pgfqpoint{2.466000in}{0.643331in}}%
\pgfpathlineto{\pgfqpoint{2.467297in}{0.627161in}}%
\pgfpathlineto{\pgfqpoint{2.468595in}{0.627161in}}%
\pgfpathlineto{\pgfqpoint{2.468595in}{0.647643in}}%
\pgfpathlineto{\pgfqpoint{2.469892in}{0.643331in}}%
\pgfpathlineto{\pgfqpoint{2.471189in}{0.643331in}}%
\pgfpathlineto{\pgfqpoint{2.472487in}{0.632551in}}%
\pgfpathlineto{\pgfqpoint{2.473784in}{0.632551in}}%
\pgfpathlineto{\pgfqpoint{2.473784in}{0.644409in}}%
\pgfpathlineto{\pgfqpoint{2.475082in}{0.637941in}}%
\pgfpathlineto{\pgfqpoint{2.476379in}{0.637941in}}%
\pgfpathlineto{\pgfqpoint{2.476379in}{0.626083in}}%
\pgfpathlineto{\pgfqpoint{2.477677in}{0.637941in}}%
\pgfpathlineto{\pgfqpoint{2.478974in}{0.637941in}}%
\pgfpathlineto{\pgfqpoint{2.478974in}{0.617459in}}%
\pgfpathlineto{\pgfqpoint{2.480271in}{0.643331in}}%
\pgfpathlineto{\pgfqpoint{2.481569in}{0.643331in}}%
\pgfpathlineto{\pgfqpoint{2.481569in}{0.658424in}}%
\pgfpathlineto{\pgfqpoint{2.482866in}{0.619615in}}%
\pgfpathlineto{\pgfqpoint{2.484164in}{0.619615in}}%
\pgfpathlineto{\pgfqpoint{2.484164in}{0.647643in}}%
\pgfpathlineto{\pgfqpoint{2.485461in}{0.620693in}}%
\pgfpathlineto{\pgfqpoint{2.486759in}{0.620693in}}%
\pgfpathlineto{\pgfqpoint{2.486759in}{0.627161in}}%
\pgfpathlineto{\pgfqpoint{2.488056in}{0.605601in}}%
\pgfpathlineto{\pgfqpoint{2.489353in}{0.605601in}}%
\pgfpathlineto{\pgfqpoint{2.489353in}{0.633629in}}%
\pgfpathlineto{\pgfqpoint{2.490651in}{0.608835in}}%
\pgfpathlineto{\pgfqpoint{2.491948in}{0.608835in}}%
\pgfpathlineto{\pgfqpoint{2.491948in}{0.628239in}}%
\pgfpathlineto{\pgfqpoint{2.493246in}{0.616381in}}%
\pgfpathlineto{\pgfqpoint{2.494543in}{0.616381in}}%
\pgfpathlineto{\pgfqpoint{2.495841in}{0.635785in}}%
\pgfpathlineto{\pgfqpoint{2.497138in}{0.635785in}}%
\pgfpathlineto{\pgfqpoint{2.497138in}{0.616381in}}%
\pgfpathlineto{\pgfqpoint{2.498435in}{0.634707in}}%
\pgfpathlineto{\pgfqpoint{2.499733in}{0.634707in}}%
\pgfpathlineto{\pgfqpoint{2.499733in}{0.615303in}}%
\pgfpathlineto{\pgfqpoint{2.501030in}{0.636863in}}%
\pgfpathlineto{\pgfqpoint{2.502328in}{0.636863in}}%
\pgfpathlineto{\pgfqpoint{2.503625in}{0.618537in}}%
\pgfpathlineto{\pgfqpoint{2.504922in}{0.618537in}}%
\pgfpathlineto{\pgfqpoint{2.506220in}{0.610991in}}%
\pgfpathlineto{\pgfqpoint{2.507517in}{0.610991in}}%
\pgfpathlineto{\pgfqpoint{2.507517in}{0.619615in}}%
\pgfpathlineto{\pgfqpoint{2.508815in}{0.609913in}}%
\pgfpathlineto{\pgfqpoint{2.512707in}{0.609913in}}%
\pgfpathlineto{\pgfqpoint{2.514004in}{0.628239in}}%
\pgfpathlineto{\pgfqpoint{2.515302in}{0.628239in}}%
\pgfpathlineto{\pgfqpoint{2.516599in}{0.599133in}}%
\pgfpathlineto{\pgfqpoint{2.517897in}{0.599133in}}%
\pgfpathlineto{\pgfqpoint{2.519194in}{0.616381in}}%
\pgfpathlineto{\pgfqpoint{2.520492in}{0.616381in}}%
\pgfpathlineto{\pgfqpoint{2.520492in}{0.598055in}}%
\pgfpathlineto{\pgfqpoint{2.521789in}{0.600211in}}%
\pgfpathlineto{\pgfqpoint{2.523086in}{0.600211in}}%
\pgfpathlineto{\pgfqpoint{2.524384in}{0.613147in}}%
\pgfpathlineto{\pgfqpoint{2.525681in}{0.613147in}}%
\pgfpathlineto{\pgfqpoint{2.526979in}{0.591586in}}%
\pgfpathlineto{\pgfqpoint{2.528276in}{0.591586in}}%
\pgfpathlineto{\pgfqpoint{2.528276in}{0.586196in}}%
\pgfpathlineto{\pgfqpoint{2.529574in}{0.614225in}}%
\pgfpathlineto{\pgfqpoint{2.530871in}{0.614225in}}%
\pgfpathlineto{\pgfqpoint{2.532168in}{0.600211in}}%
\pgfpathlineto{\pgfqpoint{2.533466in}{0.600211in}}%
\pgfpathlineto{\pgfqpoint{2.533466in}{0.608835in}}%
\pgfpathlineto{\pgfqpoint{2.534763in}{0.599133in}}%
\pgfpathlineto{\pgfqpoint{2.536061in}{0.599133in}}%
\pgfpathlineto{\pgfqpoint{2.536061in}{0.615303in}}%
\pgfpathlineto{\pgfqpoint{2.537358in}{0.594821in}}%
\pgfpathlineto{\pgfqpoint{2.538655in}{0.594821in}}%
\pgfpathlineto{\pgfqpoint{2.538655in}{0.613147in}}%
\pgfpathlineto{\pgfqpoint{2.539953in}{0.594821in}}%
\pgfpathlineto{\pgfqpoint{2.541250in}{0.594821in}}%
\pgfpathlineto{\pgfqpoint{2.541250in}{0.607757in}}%
\pgfpathlineto{\pgfqpoint{2.542548in}{0.603445in}}%
\pgfpathlineto{\pgfqpoint{2.543845in}{0.603445in}}%
\pgfpathlineto{\pgfqpoint{2.543845in}{0.595899in}}%
\pgfpathlineto{\pgfqpoint{2.545143in}{0.605601in}}%
\pgfpathlineto{\pgfqpoint{2.546440in}{0.605601in}}%
\pgfpathlineto{\pgfqpoint{2.546440in}{0.596977in}}%
\pgfpathlineto{\pgfqpoint{2.547737in}{0.607757in}}%
\pgfpathlineto{\pgfqpoint{2.549035in}{0.607757in}}%
\pgfpathlineto{\pgfqpoint{2.549035in}{0.601289in}}%
\pgfpathlineto{\pgfqpoint{2.550332in}{0.609913in}}%
\pgfpathlineto{\pgfqpoint{2.551630in}{0.609913in}}%
\pgfpathlineto{\pgfqpoint{2.551630in}{0.598055in}}%
\pgfpathlineto{\pgfqpoint{2.552927in}{0.599133in}}%
\pgfpathlineto{\pgfqpoint{2.554225in}{0.599133in}}%
\pgfpathlineto{\pgfqpoint{2.554225in}{0.613147in}}%
\pgfpathlineto{\pgfqpoint{2.555522in}{0.601289in}}%
\pgfpathlineto{\pgfqpoint{2.556819in}{0.601289in}}%
\pgfpathlineto{\pgfqpoint{2.556819in}{0.595899in}}%
\pgfpathlineto{\pgfqpoint{2.558117in}{0.602367in}}%
\pgfpathlineto{\pgfqpoint{2.559414in}{0.602367in}}%
\pgfpathlineto{\pgfqpoint{2.559414in}{0.607757in}}%
\pgfpathlineto{\pgfqpoint{2.560712in}{0.604523in}}%
\pgfpathlineto{\pgfqpoint{2.562009in}{0.604523in}}%
\pgfpathlineto{\pgfqpoint{2.562009in}{0.599133in}}%
\pgfpathlineto{\pgfqpoint{2.563307in}{0.600211in}}%
\pgfpathlineto{\pgfqpoint{2.564604in}{0.600211in}}%
\pgfpathlineto{\pgfqpoint{2.565901in}{0.608835in}}%
\pgfpathlineto{\pgfqpoint{2.567199in}{0.608835in}}%
\pgfpathlineto{\pgfqpoint{2.567199in}{0.602367in}}%
\pgfpathlineto{\pgfqpoint{2.568496in}{0.612069in}}%
\pgfpathlineto{\pgfqpoint{2.569794in}{0.612069in}}%
\pgfpathlineto{\pgfqpoint{2.569794in}{0.596977in}}%
\pgfpathlineto{\pgfqpoint{2.571091in}{0.605601in}}%
\pgfpathlineto{\pgfqpoint{2.572388in}{0.605601in}}%
\pgfpathlineto{\pgfqpoint{2.572388in}{0.580806in}}%
\pgfpathlineto{\pgfqpoint{2.573686in}{0.608835in}}%
\pgfpathlineto{\pgfqpoint{2.574983in}{0.608835in}}%
\pgfpathlineto{\pgfqpoint{2.576281in}{0.594821in}}%
\pgfpathlineto{\pgfqpoint{2.577578in}{0.594821in}}%
\pgfpathlineto{\pgfqpoint{2.577578in}{0.600211in}}%
\pgfpathlineto{\pgfqpoint{2.578876in}{0.592664in}}%
\pgfpathlineto{\pgfqpoint{2.580173in}{0.592664in}}%
\pgfpathlineto{\pgfqpoint{2.580173in}{0.595899in}}%
\pgfpathlineto{\pgfqpoint{2.581470in}{0.581884in}}%
\pgfpathlineto{\pgfqpoint{2.582768in}{0.581884in}}%
\pgfpathlineto{\pgfqpoint{2.584065in}{0.600211in}}%
\pgfpathlineto{\pgfqpoint{2.585363in}{0.600211in}}%
\pgfpathlineto{\pgfqpoint{2.585363in}{0.595899in}}%
\pgfpathlineto{\pgfqpoint{2.586660in}{0.598055in}}%
\pgfpathlineto{\pgfqpoint{2.587958in}{0.598055in}}%
\pgfpathlineto{\pgfqpoint{2.589255in}{0.586196in}}%
\pgfpathlineto{\pgfqpoint{2.590552in}{0.586196in}}%
\pgfpathlineto{\pgfqpoint{2.591850in}{0.621771in}}%
\pgfpathlineto{\pgfqpoint{2.593147in}{0.621771in}}%
\pgfpathlineto{\pgfqpoint{2.593147in}{0.584040in}}%
\pgfpathlineto{\pgfqpoint{2.594445in}{0.599133in}}%
\pgfpathlineto{\pgfqpoint{2.595742in}{0.599133in}}%
\pgfpathlineto{\pgfqpoint{2.597040in}{0.586196in}}%
\pgfpathlineto{\pgfqpoint{2.598337in}{0.586196in}}%
\pgfpathlineto{\pgfqpoint{2.598337in}{0.578650in}}%
\pgfpathlineto{\pgfqpoint{2.599634in}{0.591586in}}%
\pgfpathlineto{\pgfqpoint{2.600932in}{0.591586in}}%
\pgfpathlineto{\pgfqpoint{2.602229in}{0.585118in}}%
\pgfpathlineto{\pgfqpoint{2.603527in}{0.585118in}}%
\pgfpathlineto{\pgfqpoint{2.603527in}{0.587274in}}%
\pgfpathlineto{\pgfqpoint{2.604824in}{0.586196in}}%
\pgfpathlineto{\pgfqpoint{2.606121in}{0.586196in}}%
\pgfpathlineto{\pgfqpoint{2.607419in}{0.606679in}}%
\pgfpathlineto{\pgfqpoint{2.608716in}{0.606679in}}%
\pgfpathlineto{\pgfqpoint{2.610014in}{0.578650in}}%
\pgfpathlineto{\pgfqpoint{2.611311in}{0.578650in}}%
\pgfpathlineto{\pgfqpoint{2.611311in}{0.601289in}}%
\pgfpathlineto{\pgfqpoint{2.612609in}{0.578650in}}%
\pgfpathlineto{\pgfqpoint{2.613906in}{0.578650in}}%
\pgfpathlineto{\pgfqpoint{2.613906in}{0.589430in}}%
\pgfpathlineto{\pgfqpoint{2.615203in}{0.585118in}}%
\pgfpathlineto{\pgfqpoint{2.619096in}{0.584040in}}%
\pgfpathlineto{\pgfqpoint{2.619096in}{0.574338in}}%
\pgfpathlineto{\pgfqpoint{2.620393in}{0.595899in}}%
\pgfpathlineto{\pgfqpoint{2.621691in}{0.595899in}}%
\pgfpathlineto{\pgfqpoint{2.621691in}{0.586196in}}%
\pgfpathlineto{\pgfqpoint{2.622988in}{0.601289in}}%
\pgfpathlineto{\pgfqpoint{2.624285in}{0.601289in}}%
\pgfpathlineto{\pgfqpoint{2.624285in}{0.579728in}}%
\pgfpathlineto{\pgfqpoint{2.625583in}{0.582962in}}%
\pgfpathlineto{\pgfqpoint{2.626880in}{0.582962in}}%
\pgfpathlineto{\pgfqpoint{2.626880in}{0.613147in}}%
\pgfpathlineto{\pgfqpoint{2.628178in}{0.585118in}}%
\pgfpathlineto{\pgfqpoint{2.629475in}{0.585118in}}%
\pgfpathlineto{\pgfqpoint{2.629475in}{0.600211in}}%
\pgfpathlineto{\pgfqpoint{2.630773in}{0.592664in}}%
\pgfpathlineto{\pgfqpoint{2.632070in}{0.592664in}}%
\pgfpathlineto{\pgfqpoint{2.632070in}{0.585118in}}%
\pgfpathlineto{\pgfqpoint{2.633367in}{0.599133in}}%
\pgfpathlineto{\pgfqpoint{2.634665in}{0.599133in}}%
\pgfpathlineto{\pgfqpoint{2.635962in}{0.578650in}}%
\pgfpathlineto{\pgfqpoint{2.637260in}{0.578650in}}%
\pgfpathlineto{\pgfqpoint{2.637260in}{0.596977in}}%
\pgfpathlineto{\pgfqpoint{2.638557in}{0.594821in}}%
\pgfpathlineto{\pgfqpoint{2.639854in}{0.594821in}}%
\pgfpathlineto{\pgfqpoint{2.641152in}{0.584040in}}%
\pgfpathlineto{\pgfqpoint{2.642449in}{0.584040in}}%
\pgfpathlineto{\pgfqpoint{2.642449in}{0.581884in}}%
\pgfpathlineto{\pgfqpoint{2.643747in}{0.592664in}}%
\pgfpathlineto{\pgfqpoint{2.647639in}{0.592664in}}%
\pgfpathlineto{\pgfqpoint{2.647639in}{0.581884in}}%
\pgfpathlineto{\pgfqpoint{2.648936in}{0.585118in}}%
\pgfpathlineto{\pgfqpoint{2.650234in}{0.585118in}}%
\pgfpathlineto{\pgfqpoint{2.650234in}{0.590508in}}%
\pgfpathlineto{\pgfqpoint{2.651531in}{0.587274in}}%
\pgfpathlineto{\pgfqpoint{2.652829in}{0.587274in}}%
\pgfpathlineto{\pgfqpoint{2.652829in}{0.582962in}}%
\pgfpathlineto{\pgfqpoint{2.654126in}{0.587274in}}%
\pgfpathlineto{\pgfqpoint{2.655424in}{0.587274in}}%
\pgfpathlineto{\pgfqpoint{2.656721in}{0.574338in}}%
\pgfpathlineto{\pgfqpoint{2.658018in}{0.574338in}}%
\pgfpathlineto{\pgfqpoint{2.658018in}{0.587274in}}%
\pgfpathlineto{\pgfqpoint{2.659316in}{0.587274in}}%
\pgfpathlineto{\pgfqpoint{2.660613in}{0.587274in}}%
\pgfpathlineto{\pgfqpoint{2.660613in}{0.598055in}}%
\pgfpathlineto{\pgfqpoint{2.661911in}{0.578650in}}%
\pgfpathlineto{\pgfqpoint{2.663208in}{0.578650in}}%
\pgfpathlineto{\pgfqpoint{2.663208in}{0.594821in}}%
\pgfpathlineto{\pgfqpoint{2.664506in}{0.589430in}}%
\pgfpathlineto{\pgfqpoint{2.665803in}{0.589430in}}%
\pgfpathlineto{\pgfqpoint{2.665803in}{0.577572in}}%
\pgfpathlineto{\pgfqpoint{2.667100in}{0.596977in}}%
\pgfpathlineto{\pgfqpoint{2.668398in}{0.596977in}}%
\pgfpathlineto{\pgfqpoint{2.668398in}{0.575416in}}%
\pgfpathlineto{\pgfqpoint{2.669695in}{0.585118in}}%
\pgfpathlineto{\pgfqpoint{2.672290in}{0.585118in}}%
\pgfpathlineto{\pgfqpoint{2.672290in}{0.592664in}}%
\pgfpathlineto{\pgfqpoint{2.673587in}{0.582962in}}%
\pgfpathlineto{\pgfqpoint{2.674885in}{0.582962in}}%
\pgfpathlineto{\pgfqpoint{2.676182in}{0.594821in}}%
\pgfpathlineto{\pgfqpoint{2.677480in}{0.594821in}}%
\pgfpathlineto{\pgfqpoint{2.677480in}{0.590508in}}%
\pgfpathlineto{\pgfqpoint{2.678777in}{0.593743in}}%
\pgfpathlineto{\pgfqpoint{2.680075in}{0.593743in}}%
\pgfpathlineto{\pgfqpoint{2.680075in}{0.580806in}}%
\pgfpathlineto{\pgfqpoint{2.681372in}{0.589430in}}%
\pgfpathlineto{\pgfqpoint{2.682669in}{0.589430in}}%
\pgfpathlineto{\pgfqpoint{2.682669in}{0.586196in}}%
\pgfpathlineto{\pgfqpoint{2.683967in}{0.591586in}}%
\pgfpathlineto{\pgfqpoint{2.685264in}{0.591586in}}%
\pgfpathlineto{\pgfqpoint{2.685264in}{0.576494in}}%
\pgfpathlineto{\pgfqpoint{2.686562in}{0.595899in}}%
\pgfpathlineto{\pgfqpoint{2.687859in}{0.595899in}}%
\pgfpathlineto{\pgfqpoint{2.689157in}{0.582962in}}%
\pgfpathlineto{\pgfqpoint{2.690454in}{0.582962in}}%
\pgfpathlineto{\pgfqpoint{2.690454in}{0.599133in}}%
\pgfpathlineto{\pgfqpoint{2.691751in}{0.578650in}}%
\pgfpathlineto{\pgfqpoint{2.693049in}{0.578650in}}%
\pgfpathlineto{\pgfqpoint{2.693049in}{0.589430in}}%
\pgfpathlineto{\pgfqpoint{2.694346in}{0.574338in}}%
\pgfpathlineto{\pgfqpoint{2.695644in}{0.574338in}}%
\pgfpathlineto{\pgfqpoint{2.696941in}{0.588352in}}%
\pgfpathlineto{\pgfqpoint{2.698239in}{0.588352in}}%
\pgfpathlineto{\pgfqpoint{2.698239in}{0.584040in}}%
\pgfpathlineto{\pgfqpoint{2.699536in}{0.584040in}}%
\pgfpathlineto{\pgfqpoint{2.700833in}{0.584040in}}%
\pgfpathlineto{\pgfqpoint{2.702131in}{0.575416in}}%
\pgfpathlineto{\pgfqpoint{2.703428in}{0.575416in}}%
\pgfpathlineto{\pgfqpoint{2.704726in}{0.580806in}}%
\pgfpathlineto{\pgfqpoint{2.706023in}{0.580806in}}%
\pgfpathlineto{\pgfqpoint{2.707320in}{0.590508in}}%
\pgfpathlineto{\pgfqpoint{2.708618in}{0.590508in}}%
\pgfpathlineto{\pgfqpoint{2.708618in}{0.580806in}}%
\pgfpathlineto{\pgfqpoint{2.709915in}{0.589430in}}%
\pgfpathlineto{\pgfqpoint{2.711213in}{0.589430in}}%
\pgfpathlineto{\pgfqpoint{2.711213in}{0.592664in}}%
\pgfpathlineto{\pgfqpoint{2.712510in}{0.576494in}}%
\pgfpathlineto{\pgfqpoint{2.713808in}{0.576494in}}%
\pgfpathlineto{\pgfqpoint{2.715105in}{0.591586in}}%
\pgfpathlineto{\pgfqpoint{2.716402in}{0.591586in}}%
\pgfpathlineto{\pgfqpoint{2.717700in}{0.581884in}}%
\pgfpathlineto{\pgfqpoint{2.718997in}{0.581884in}}%
\pgfpathlineto{\pgfqpoint{2.718997in}{0.588352in}}%
\pgfpathlineto{\pgfqpoint{2.720295in}{0.582962in}}%
\pgfpathlineto{\pgfqpoint{2.721592in}{0.582962in}}%
\pgfpathlineto{\pgfqpoint{2.721592in}{0.592664in}}%
\pgfpathlineto{\pgfqpoint{2.722890in}{0.588352in}}%
\pgfpathlineto{\pgfqpoint{2.724187in}{0.588352in}}%
\pgfpathlineto{\pgfqpoint{2.724187in}{0.578650in}}%
\pgfpathlineto{\pgfqpoint{2.725484in}{0.582962in}}%
\pgfpathlineto{\pgfqpoint{2.726782in}{0.582962in}}%
\pgfpathlineto{\pgfqpoint{2.728079in}{0.588352in}}%
\pgfpathlineto{\pgfqpoint{2.729377in}{0.588352in}}%
\pgfpathlineto{\pgfqpoint{2.729377in}{0.579728in}}%
\pgfpathlineto{\pgfqpoint{2.730674in}{0.581884in}}%
\pgfpathlineto{\pgfqpoint{2.731972in}{0.581884in}}%
\pgfpathlineto{\pgfqpoint{2.731972in}{0.570026in}}%
\pgfpathlineto{\pgfqpoint{2.733269in}{0.592664in}}%
\pgfpathlineto{\pgfqpoint{2.734566in}{0.592664in}}%
\pgfpathlineto{\pgfqpoint{2.734566in}{0.576494in}}%
\pgfpathlineto{\pgfqpoint{2.735864in}{0.581884in}}%
\pgfpathlineto{\pgfqpoint{2.737161in}{0.581884in}}%
\pgfpathlineto{\pgfqpoint{2.737161in}{0.586196in}}%
\pgfpathlineto{\pgfqpoint{2.738459in}{0.581884in}}%
\pgfpathlineto{\pgfqpoint{2.739756in}{0.581884in}}%
\pgfpathlineto{\pgfqpoint{2.741053in}{0.573260in}}%
\pgfpathlineto{\pgfqpoint{2.742351in}{0.573260in}}%
\pgfpathlineto{\pgfqpoint{2.742351in}{0.596977in}}%
\pgfpathlineto{\pgfqpoint{2.743648in}{0.579728in}}%
\pgfpathlineto{\pgfqpoint{2.747541in}{0.578650in}}%
\pgfpathlineto{\pgfqpoint{2.747541in}{0.591586in}}%
\pgfpathlineto{\pgfqpoint{2.748838in}{0.587274in}}%
\pgfpathlineto{\pgfqpoint{2.750135in}{0.587274in}}%
\pgfpathlineto{\pgfqpoint{2.750135in}{0.580806in}}%
\pgfpathlineto{\pgfqpoint{2.751433in}{0.588352in}}%
\pgfpathlineto{\pgfqpoint{2.754028in}{0.588352in}}%
\pgfpathlineto{\pgfqpoint{2.755325in}{0.572182in}}%
\pgfpathlineto{\pgfqpoint{2.756623in}{0.572182in}}%
\pgfpathlineto{\pgfqpoint{2.756623in}{0.582962in}}%
\pgfpathlineto{\pgfqpoint{2.757920in}{0.579728in}}%
\pgfpathlineto{\pgfqpoint{2.759217in}{0.579728in}}%
\pgfpathlineto{\pgfqpoint{2.759217in}{0.592664in}}%
\pgfpathlineto{\pgfqpoint{2.760515in}{0.580806in}}%
\pgfpathlineto{\pgfqpoint{2.763110in}{0.580806in}}%
\pgfpathlineto{\pgfqpoint{2.763110in}{0.587274in}}%
\pgfpathlineto{\pgfqpoint{2.764407in}{0.577572in}}%
\pgfpathlineto{\pgfqpoint{2.765705in}{0.577572in}}%
\pgfpathlineto{\pgfqpoint{2.765705in}{0.596977in}}%
\pgfpathlineto{\pgfqpoint{2.767002in}{0.587274in}}%
\pgfpathlineto{\pgfqpoint{2.768299in}{0.587274in}}%
\pgfpathlineto{\pgfqpoint{2.768299in}{0.596977in}}%
\pgfpathlineto{\pgfqpoint{2.769597in}{0.570026in}}%
\pgfpathlineto{\pgfqpoint{2.770894in}{0.570026in}}%
\pgfpathlineto{\pgfqpoint{2.772192in}{0.594821in}}%
\pgfpathlineto{\pgfqpoint{2.773489in}{0.594821in}}%
\pgfpathlineto{\pgfqpoint{2.773489in}{0.582962in}}%
\pgfpathlineto{\pgfqpoint{2.774787in}{0.589430in}}%
\pgfpathlineto{\pgfqpoint{2.776084in}{0.589430in}}%
\pgfpathlineto{\pgfqpoint{2.776084in}{0.582962in}}%
\pgfpathlineto{\pgfqpoint{2.777381in}{0.586196in}}%
\pgfpathlineto{\pgfqpoint{2.778679in}{0.586196in}}%
\pgfpathlineto{\pgfqpoint{2.778679in}{0.575416in}}%
\pgfpathlineto{\pgfqpoint{2.779976in}{0.579728in}}%
\pgfpathlineto{\pgfqpoint{2.781274in}{0.579728in}}%
\pgfpathlineto{\pgfqpoint{2.782571in}{0.588352in}}%
\pgfpathlineto{\pgfqpoint{2.783868in}{0.588352in}}%
\pgfpathlineto{\pgfqpoint{2.783868in}{0.585118in}}%
\pgfpathlineto{\pgfqpoint{2.785166in}{0.587274in}}%
\pgfpathlineto{\pgfqpoint{2.786463in}{0.587274in}}%
\pgfpathlineto{\pgfqpoint{2.787761in}{0.576494in}}%
\pgfpathlineto{\pgfqpoint{2.789058in}{0.576494in}}%
\pgfpathlineto{\pgfqpoint{2.789058in}{0.586196in}}%
\pgfpathlineto{\pgfqpoint{2.790356in}{0.578650in}}%
\pgfpathlineto{\pgfqpoint{2.792950in}{0.577572in}}%
\pgfpathlineto{\pgfqpoint{2.794248in}{0.593743in}}%
\pgfpathlineto{\pgfqpoint{2.795545in}{0.593743in}}%
\pgfpathlineto{\pgfqpoint{2.796843in}{0.582962in}}%
\pgfpathlineto{\pgfqpoint{2.798140in}{0.582962in}}%
\pgfpathlineto{\pgfqpoint{2.798140in}{0.576494in}}%
\pgfpathlineto{\pgfqpoint{2.799438in}{0.579728in}}%
\pgfpathlineto{\pgfqpoint{2.800735in}{0.579728in}}%
\pgfpathlineto{\pgfqpoint{2.802032in}{0.591586in}}%
\pgfpathlineto{\pgfqpoint{2.803330in}{0.591586in}}%
\pgfpathlineto{\pgfqpoint{2.803330in}{0.588352in}}%
\pgfpathlineto{\pgfqpoint{2.804627in}{0.588352in}}%
\pgfpathlineto{\pgfqpoint{2.805925in}{0.588352in}}%
\pgfpathlineto{\pgfqpoint{2.807222in}{0.585118in}}%
\pgfpathlineto{\pgfqpoint{2.808520in}{0.585118in}}%
\pgfpathlineto{\pgfqpoint{2.808520in}{0.568948in}}%
\pgfpathlineto{\pgfqpoint{2.809817in}{0.580806in}}%
\pgfpathlineto{\pgfqpoint{2.811114in}{0.580806in}}%
\pgfpathlineto{\pgfqpoint{2.811114in}{0.567870in}}%
\pgfpathlineto{\pgfqpoint{2.812412in}{0.591586in}}%
\pgfpathlineto{\pgfqpoint{2.813709in}{0.591586in}}%
\pgfpathlineto{\pgfqpoint{2.813709in}{0.574338in}}%
\pgfpathlineto{\pgfqpoint{2.815007in}{0.581884in}}%
\pgfpathlineto{\pgfqpoint{2.816304in}{0.581884in}}%
\pgfpathlineto{\pgfqpoint{2.816304in}{0.579728in}}%
\pgfpathlineto{\pgfqpoint{2.817601in}{0.584040in}}%
\pgfpathlineto{\pgfqpoint{2.818899in}{0.584040in}}%
\pgfpathlineto{\pgfqpoint{2.818899in}{0.577572in}}%
\pgfpathlineto{\pgfqpoint{2.820196in}{0.588352in}}%
\pgfpathlineto{\pgfqpoint{2.821494in}{0.588352in}}%
\pgfpathlineto{\pgfqpoint{2.822791in}{0.576494in}}%
\pgfpathlineto{\pgfqpoint{2.824089in}{0.576494in}}%
\pgfpathlineto{\pgfqpoint{2.824089in}{0.586196in}}%
\pgfpathlineto{\pgfqpoint{2.825386in}{0.577572in}}%
\pgfpathlineto{\pgfqpoint{2.826683in}{0.577572in}}%
\pgfpathlineto{\pgfqpoint{2.826683in}{0.581884in}}%
\pgfpathlineto{\pgfqpoint{2.827981in}{0.576494in}}%
\pgfpathlineto{\pgfqpoint{2.830576in}{0.575416in}}%
\pgfpathlineto{\pgfqpoint{2.830576in}{0.578650in}}%
\pgfpathlineto{\pgfqpoint{2.831873in}{0.578650in}}%
\pgfpathlineto{\pgfqpoint{2.833171in}{0.578650in}}%
\pgfpathlineto{\pgfqpoint{2.833171in}{0.586196in}}%
\pgfpathlineto{\pgfqpoint{2.834468in}{0.585118in}}%
\pgfpathlineto{\pgfqpoint{2.835765in}{0.585118in}}%
\pgfpathlineto{\pgfqpoint{2.835765in}{0.592664in}}%
\pgfpathlineto{\pgfqpoint{2.837063in}{0.573260in}}%
\pgfpathlineto{\pgfqpoint{2.838360in}{0.573260in}}%
\pgfpathlineto{\pgfqpoint{2.839658in}{0.586196in}}%
\pgfpathlineto{\pgfqpoint{2.840955in}{0.586196in}}%
\pgfpathlineto{\pgfqpoint{2.842253in}{0.574338in}}%
\pgfpathlineto{\pgfqpoint{2.844847in}{0.573260in}}%
\pgfpathlineto{\pgfqpoint{2.844847in}{0.582962in}}%
\pgfpathlineto{\pgfqpoint{2.846145in}{0.578650in}}%
\pgfpathlineto{\pgfqpoint{2.848740in}{0.577572in}}%
\pgfpathlineto{\pgfqpoint{2.848740in}{0.574338in}}%
\pgfpathlineto{\pgfqpoint{2.850037in}{0.586196in}}%
\pgfpathlineto{\pgfqpoint{2.851334in}{0.586196in}}%
\pgfpathlineto{\pgfqpoint{2.851334in}{0.570026in}}%
\pgfpathlineto{\pgfqpoint{2.852632in}{0.582962in}}%
\pgfpathlineto{\pgfqpoint{2.853929in}{0.582962in}}%
\pgfpathlineto{\pgfqpoint{2.853929in}{0.574338in}}%
\pgfpathlineto{\pgfqpoint{2.855227in}{0.581884in}}%
\pgfpathlineto{\pgfqpoint{2.856524in}{0.581884in}}%
\pgfpathlineto{\pgfqpoint{2.856524in}{0.571104in}}%
\pgfpathlineto{\pgfqpoint{2.857822in}{0.574338in}}%
\pgfpathlineto{\pgfqpoint{2.859119in}{0.574338in}}%
\pgfpathlineto{\pgfqpoint{2.859119in}{0.582962in}}%
\pgfpathlineto{\pgfqpoint{2.860416in}{0.580806in}}%
\pgfpathlineto{\pgfqpoint{2.864309in}{0.581884in}}%
\pgfpathlineto{\pgfqpoint{2.865606in}{0.567870in}}%
\pgfpathlineto{\pgfqpoint{2.866904in}{0.567870in}}%
\pgfpathlineto{\pgfqpoint{2.866904in}{0.585118in}}%
\pgfpathlineto{\pgfqpoint{2.868201in}{0.582962in}}%
\pgfpathlineto{\pgfqpoint{2.869498in}{0.582962in}}%
\pgfpathlineto{\pgfqpoint{2.869498in}{0.580806in}}%
\pgfpathlineto{\pgfqpoint{2.870796in}{0.580806in}}%
\pgfpathlineto{\pgfqpoint{2.872093in}{0.580806in}}%
\pgfpathlineto{\pgfqpoint{2.872093in}{0.576494in}}%
\pgfpathlineto{\pgfqpoint{2.873391in}{0.576494in}}%
\pgfpathlineto{\pgfqpoint{2.874688in}{0.576494in}}%
\pgfpathlineto{\pgfqpoint{2.874688in}{0.581884in}}%
\pgfpathlineto{\pgfqpoint{2.875986in}{0.573260in}}%
\pgfpathlineto{\pgfqpoint{2.877283in}{0.573260in}}%
\pgfpathlineto{\pgfqpoint{2.878580in}{0.584040in}}%
\pgfpathlineto{\pgfqpoint{2.879878in}{0.584040in}}%
\pgfpathlineto{\pgfqpoint{2.881175in}{0.575416in}}%
\pgfpathlineto{\pgfqpoint{2.882473in}{0.575416in}}%
\pgfpathlineto{\pgfqpoint{2.882473in}{0.584040in}}%
\pgfpathlineto{\pgfqpoint{2.883770in}{0.568948in}}%
\pgfpathlineto{\pgfqpoint{2.885067in}{0.568948in}}%
\pgfpathlineto{\pgfqpoint{2.885067in}{0.589430in}}%
\pgfpathlineto{\pgfqpoint{2.886365in}{0.575416in}}%
\pgfpathlineto{\pgfqpoint{2.887662in}{0.575416in}}%
\pgfpathlineto{\pgfqpoint{2.887662in}{0.581884in}}%
\pgfpathlineto{\pgfqpoint{2.888960in}{0.580806in}}%
\pgfpathlineto{\pgfqpoint{2.890257in}{0.580806in}}%
\pgfpathlineto{\pgfqpoint{2.891555in}{0.572182in}}%
\pgfpathlineto{\pgfqpoint{2.892852in}{0.572182in}}%
\pgfpathlineto{\pgfqpoint{2.892852in}{0.581884in}}%
\pgfpathlineto{\pgfqpoint{2.894149in}{0.576494in}}%
\pgfpathlineto{\pgfqpoint{2.895447in}{0.576494in}}%
\pgfpathlineto{\pgfqpoint{2.895447in}{0.567870in}}%
\pgfpathlineto{\pgfqpoint{2.896744in}{0.577572in}}%
\pgfpathlineto{\pgfqpoint{2.900637in}{0.577572in}}%
\pgfpathlineto{\pgfqpoint{2.900637in}{0.579728in}}%
\pgfpathlineto{\pgfqpoint{2.901934in}{0.573260in}}%
\pgfpathlineto{\pgfqpoint{2.903231in}{0.573260in}}%
\pgfpathlineto{\pgfqpoint{2.903231in}{0.570026in}}%
\pgfpathlineto{\pgfqpoint{2.904529in}{0.571104in}}%
\pgfpathlineto{\pgfqpoint{2.905826in}{0.571104in}}%
\pgfpathlineto{\pgfqpoint{2.905826in}{0.565714in}}%
\pgfpathlineto{\pgfqpoint{2.907124in}{0.575416in}}%
\pgfpathlineto{\pgfqpoint{2.909719in}{0.574338in}}%
\pgfpathlineto{\pgfqpoint{2.909719in}{0.581884in}}%
\pgfpathlineto{\pgfqpoint{2.911016in}{0.574338in}}%
\pgfpathlineto{\pgfqpoint{2.912313in}{0.574338in}}%
\pgfpathlineto{\pgfqpoint{2.912313in}{0.584040in}}%
\pgfpathlineto{\pgfqpoint{2.913611in}{0.576494in}}%
\pgfpathlineto{\pgfqpoint{2.914908in}{0.576494in}}%
\pgfpathlineto{\pgfqpoint{2.914908in}{0.568948in}}%
\pgfpathlineto{\pgfqpoint{2.916206in}{0.570026in}}%
\pgfpathlineto{\pgfqpoint{2.917503in}{0.570026in}}%
\pgfpathlineto{\pgfqpoint{2.917503in}{0.576494in}}%
\pgfpathlineto{\pgfqpoint{2.918800in}{0.574338in}}%
\pgfpathlineto{\pgfqpoint{2.920098in}{0.574338in}}%
\pgfpathlineto{\pgfqpoint{2.921395in}{0.566792in}}%
\pgfpathlineto{\pgfqpoint{2.922693in}{0.566792in}}%
\pgfpathlineto{\pgfqpoint{2.922693in}{0.576494in}}%
\pgfpathlineto{\pgfqpoint{2.923990in}{0.570026in}}%
\pgfpathlineto{\pgfqpoint{2.925288in}{0.570026in}}%
\pgfpathlineto{\pgfqpoint{2.926585in}{0.582962in}}%
\pgfpathlineto{\pgfqpoint{2.929180in}{0.584040in}}%
\pgfpathlineto{\pgfqpoint{2.930477in}{0.565714in}}%
\pgfpathlineto{\pgfqpoint{2.931775in}{0.565714in}}%
\pgfpathlineto{\pgfqpoint{2.931775in}{0.576494in}}%
\pgfpathlineto{\pgfqpoint{2.933072in}{0.568948in}}%
\pgfpathlineto{\pgfqpoint{2.934370in}{0.568948in}}%
\pgfpathlineto{\pgfqpoint{2.934370in}{0.577572in}}%
\pgfpathlineto{\pgfqpoint{2.935667in}{0.573260in}}%
\pgfpathlineto{\pgfqpoint{2.936964in}{0.573260in}}%
\pgfpathlineto{\pgfqpoint{2.938262in}{0.565714in}}%
\pgfpathlineto{\pgfqpoint{2.939559in}{0.565714in}}%
\pgfpathlineto{\pgfqpoint{2.940857in}{0.574338in}}%
\pgfpathlineto{\pgfqpoint{2.944749in}{0.574338in}}%
\pgfpathlineto{\pgfqpoint{2.944749in}{0.572182in}}%
\pgfpathlineto{\pgfqpoint{2.946046in}{0.577572in}}%
\pgfpathlineto{\pgfqpoint{2.947344in}{0.577572in}}%
\pgfpathlineto{\pgfqpoint{2.947344in}{0.584040in}}%
\pgfpathlineto{\pgfqpoint{2.948641in}{0.576494in}}%
\pgfpathlineto{\pgfqpoint{2.949939in}{0.576494in}}%
\pgfpathlineto{\pgfqpoint{2.949939in}{0.572182in}}%
\pgfpathlineto{\pgfqpoint{2.951236in}{0.576494in}}%
\pgfpathlineto{\pgfqpoint{2.952533in}{0.576494in}}%
\pgfpathlineto{\pgfqpoint{2.952533in}{0.573260in}}%
\pgfpathlineto{\pgfqpoint{2.953831in}{0.578650in}}%
\pgfpathlineto{\pgfqpoint{2.955128in}{0.578650in}}%
\pgfpathlineto{\pgfqpoint{2.956426in}{0.568948in}}%
\pgfpathlineto{\pgfqpoint{2.957723in}{0.568948in}}%
\pgfpathlineto{\pgfqpoint{2.957723in}{0.565714in}}%
\pgfpathlineto{\pgfqpoint{2.959021in}{0.568948in}}%
\pgfpathlineto{\pgfqpoint{2.960318in}{0.568948in}}%
\pgfpathlineto{\pgfqpoint{2.960318in}{0.571104in}}%
\pgfpathlineto{\pgfqpoint{2.961615in}{0.571104in}}%
\pgfpathlineto{\pgfqpoint{2.962913in}{0.571104in}}%
\pgfpathlineto{\pgfqpoint{2.962913in}{0.565714in}}%
\pgfpathlineto{\pgfqpoint{2.964210in}{0.571104in}}%
\pgfpathlineto{\pgfqpoint{2.966805in}{0.572182in}}%
\pgfpathlineto{\pgfqpoint{2.966805in}{0.574338in}}%
\pgfpathlineto{\pgfqpoint{2.968103in}{0.567870in}}%
\pgfpathlineto{\pgfqpoint{2.970697in}{0.567870in}}%
\pgfpathlineto{\pgfqpoint{2.970697in}{0.571104in}}%
\pgfpathlineto{\pgfqpoint{2.971995in}{0.566792in}}%
\pgfpathlineto{\pgfqpoint{2.973292in}{0.566792in}}%
\pgfpathlineto{\pgfqpoint{2.974590in}{0.573260in}}%
\pgfpathlineto{\pgfqpoint{2.975887in}{0.573260in}}%
\pgfpathlineto{\pgfqpoint{2.975887in}{0.567870in}}%
\pgfpathlineto{\pgfqpoint{2.977185in}{0.567870in}}%
\pgfpathlineto{\pgfqpoint{2.978482in}{0.567870in}}%
\pgfpathlineto{\pgfqpoint{2.978482in}{0.576494in}}%
\pgfpathlineto{\pgfqpoint{2.979779in}{0.568948in}}%
\pgfpathlineto{\pgfqpoint{2.981077in}{0.568948in}}%
\pgfpathlineto{\pgfqpoint{2.981077in}{0.576494in}}%
\pgfpathlineto{\pgfqpoint{2.982374in}{0.571104in}}%
\pgfpathlineto{\pgfqpoint{2.983672in}{0.571104in}}%
\pgfpathlineto{\pgfqpoint{2.983672in}{0.575416in}}%
\pgfpathlineto{\pgfqpoint{2.984969in}{0.565714in}}%
\pgfpathlineto{\pgfqpoint{2.986266in}{0.565714in}}%
\pgfpathlineto{\pgfqpoint{2.986266in}{0.575416in}}%
\pgfpathlineto{\pgfqpoint{2.987564in}{0.573260in}}%
\pgfpathlineto{\pgfqpoint{2.988861in}{0.573260in}}%
\pgfpathlineto{\pgfqpoint{2.988861in}{0.571104in}}%
\pgfpathlineto{\pgfqpoint{2.990159in}{0.574338in}}%
\pgfpathlineto{\pgfqpoint{2.991456in}{0.574338in}}%
\pgfpathlineto{\pgfqpoint{2.991456in}{0.562480in}}%
\pgfpathlineto{\pgfqpoint{2.992754in}{0.580806in}}%
\pgfpathlineto{\pgfqpoint{2.994051in}{0.580806in}}%
\pgfpathlineto{\pgfqpoint{2.994051in}{0.566792in}}%
\pgfpathlineto{\pgfqpoint{2.995348in}{0.584040in}}%
\pgfpathlineto{\pgfqpoint{2.996646in}{0.584040in}}%
\pgfpathlineto{\pgfqpoint{2.997943in}{0.571104in}}%
\pgfpathlineto{\pgfqpoint{2.999241in}{0.571104in}}%
\pgfpathlineto{\pgfqpoint{3.000538in}{0.576494in}}%
\pgfpathlineto{\pgfqpoint{3.001836in}{0.576494in}}%
\pgfpathlineto{\pgfqpoint{3.001836in}{0.566792in}}%
\pgfpathlineto{\pgfqpoint{3.003133in}{0.566792in}}%
\pgfpathlineto{\pgfqpoint{3.004430in}{0.566792in}}%
\pgfpathlineto{\pgfqpoint{3.004430in}{0.579728in}}%
\pgfpathlineto{\pgfqpoint{3.005728in}{0.575416in}}%
\pgfpathlineto{\pgfqpoint{3.007025in}{0.575416in}}%
\pgfpathlineto{\pgfqpoint{3.008323in}{0.560324in}}%
\pgfpathlineto{\pgfqpoint{3.009620in}{0.560324in}}%
\pgfpathlineto{\pgfqpoint{3.009620in}{0.566792in}}%
\pgfpathlineto{\pgfqpoint{3.010918in}{0.566792in}}%
\pgfpathlineto{\pgfqpoint{3.012215in}{0.566792in}}%
\pgfpathlineto{\pgfqpoint{3.012215in}{0.574338in}}%
\pgfpathlineto{\pgfqpoint{3.013512in}{0.570026in}}%
\pgfpathlineto{\pgfqpoint{3.014810in}{0.570026in}}%
\pgfpathlineto{\pgfqpoint{3.014810in}{0.566792in}}%
\pgfpathlineto{\pgfqpoint{3.016107in}{0.568948in}}%
\pgfpathlineto{\pgfqpoint{3.017405in}{0.568948in}}%
\pgfpathlineto{\pgfqpoint{3.017405in}{0.561402in}}%
\pgfpathlineto{\pgfqpoint{3.018702in}{0.575416in}}%
\pgfpathlineto{\pgfqpoint{3.019999in}{0.575416in}}%
\pgfpathlineto{\pgfqpoint{3.021297in}{0.565714in}}%
\pgfpathlineto{\pgfqpoint{3.023892in}{0.564636in}}%
\pgfpathlineto{\pgfqpoint{3.023892in}{0.567870in}}%
\pgfpathlineto{\pgfqpoint{3.025189in}{0.567870in}}%
\pgfpathlineto{\pgfqpoint{3.026487in}{0.567870in}}%
\pgfpathlineto{\pgfqpoint{3.026487in}{0.563558in}}%
\pgfpathlineto{\pgfqpoint{3.027784in}{0.571104in}}%
\pgfpathlineto{\pgfqpoint{3.029081in}{0.571104in}}%
\pgfpathlineto{\pgfqpoint{3.030379in}{0.566792in}}%
\pgfpathlineto{\pgfqpoint{3.032974in}{0.567870in}}%
\pgfpathlineto{\pgfqpoint{3.032974in}{0.561402in}}%
\pgfpathlineto{\pgfqpoint{3.034271in}{0.568948in}}%
\pgfpathlineto{\pgfqpoint{3.036866in}{0.567870in}}%
\pgfpathlineto{\pgfqpoint{3.036866in}{0.564636in}}%
\pgfpathlineto{\pgfqpoint{3.038163in}{0.572182in}}%
\pgfpathlineto{\pgfqpoint{3.039461in}{0.572182in}}%
\pgfpathlineto{\pgfqpoint{3.040758in}{0.562480in}}%
\pgfpathlineto{\pgfqpoint{3.042056in}{0.562480in}}%
\pgfpathlineto{\pgfqpoint{3.043353in}{0.573260in}}%
\pgfpathlineto{\pgfqpoint{3.045948in}{0.572182in}}%
\pgfpathlineto{\pgfqpoint{3.045948in}{0.565714in}}%
\pgfpathlineto{\pgfqpoint{3.047245in}{0.567870in}}%
\pgfpathlineto{\pgfqpoint{3.048543in}{0.567870in}}%
\pgfpathlineto{\pgfqpoint{3.048543in}{0.560324in}}%
\pgfpathlineto{\pgfqpoint{3.049840in}{0.570026in}}%
\pgfpathlineto{\pgfqpoint{3.051138in}{0.570026in}}%
\pgfpathlineto{\pgfqpoint{3.051138in}{0.575416in}}%
\pgfpathlineto{\pgfqpoint{3.052435in}{0.566792in}}%
\pgfpathlineto{\pgfqpoint{3.053732in}{0.566792in}}%
\pgfpathlineto{\pgfqpoint{3.053732in}{0.570026in}}%
\pgfpathlineto{\pgfqpoint{3.055030in}{0.565714in}}%
\pgfpathlineto{\pgfqpoint{3.056327in}{0.565714in}}%
\pgfpathlineto{\pgfqpoint{3.056327in}{0.570026in}}%
\pgfpathlineto{\pgfqpoint{3.057625in}{0.567870in}}%
\pgfpathlineto{\pgfqpoint{3.058922in}{0.567870in}}%
\pgfpathlineto{\pgfqpoint{3.058922in}{0.570026in}}%
\pgfpathlineto{\pgfqpoint{3.060220in}{0.567870in}}%
\pgfpathlineto{\pgfqpoint{3.061517in}{0.567870in}}%
\pgfpathlineto{\pgfqpoint{3.061517in}{0.570026in}}%
\pgfpathlineto{\pgfqpoint{3.062814in}{0.565714in}}%
\pgfpathlineto{\pgfqpoint{3.064112in}{0.565714in}}%
\pgfpathlineto{\pgfqpoint{3.064112in}{0.567870in}}%
\pgfpathlineto{\pgfqpoint{3.065409in}{0.567870in}}%
\pgfpathlineto{\pgfqpoint{3.066707in}{0.567870in}}%
\pgfpathlineto{\pgfqpoint{3.066707in}{0.561402in}}%
\pgfpathlineto{\pgfqpoint{3.068004in}{0.570026in}}%
\pgfpathlineto{\pgfqpoint{3.070599in}{0.571104in}}%
\pgfpathlineto{\pgfqpoint{3.070599in}{0.560324in}}%
\pgfpathlineto{\pgfqpoint{3.071896in}{0.567870in}}%
\pgfpathlineto{\pgfqpoint{3.074491in}{0.567870in}}%
\pgfpathlineto{\pgfqpoint{3.074491in}{0.563558in}}%
\pgfpathlineto{\pgfqpoint{3.075789in}{0.563558in}}%
\pgfpathlineto{\pgfqpoint{3.078384in}{0.563558in}}%
\pgfpathlineto{\pgfqpoint{3.078384in}{0.568948in}}%
\pgfpathlineto{\pgfqpoint{3.079681in}{0.568948in}}%
\pgfpathlineto{\pgfqpoint{3.080978in}{0.568948in}}%
\pgfpathlineto{\pgfqpoint{3.080978in}{0.572182in}}%
\pgfpathlineto{\pgfqpoint{3.082276in}{0.558168in}}%
\pgfpathlineto{\pgfqpoint{3.083573in}{0.558168in}}%
\pgfpathlineto{\pgfqpoint{3.084871in}{0.573260in}}%
\pgfpathlineto{\pgfqpoint{3.086168in}{0.573260in}}%
\pgfpathlineto{\pgfqpoint{3.086168in}{0.563558in}}%
\pgfpathlineto{\pgfqpoint{3.087465in}{0.568948in}}%
\pgfpathlineto{\pgfqpoint{3.088763in}{0.568948in}}%
\pgfpathlineto{\pgfqpoint{3.088763in}{0.577572in}}%
\pgfpathlineto{\pgfqpoint{3.090060in}{0.575416in}}%
\pgfpathlineto{\pgfqpoint{3.091358in}{0.575416in}}%
\pgfpathlineto{\pgfqpoint{3.091358in}{0.568948in}}%
\pgfpathlineto{\pgfqpoint{3.092655in}{0.573260in}}%
\pgfpathlineto{\pgfqpoint{3.093953in}{0.573260in}}%
\pgfpathlineto{\pgfqpoint{3.095250in}{0.563558in}}%
\pgfpathlineto{\pgfqpoint{3.097845in}{0.563558in}}%
\pgfpathlineto{\pgfqpoint{3.097845in}{0.572182in}}%
\pgfpathlineto{\pgfqpoint{3.099142in}{0.561402in}}%
\pgfpathlineto{\pgfqpoint{3.100440in}{0.561402in}}%
\pgfpathlineto{\pgfqpoint{3.100440in}{0.563558in}}%
\pgfpathlineto{\pgfqpoint{3.101737in}{0.563558in}}%
\pgfpathlineto{\pgfqpoint{3.103035in}{0.563558in}}%
\pgfpathlineto{\pgfqpoint{3.103035in}{0.577572in}}%
\pgfpathlineto{\pgfqpoint{3.104332in}{0.568948in}}%
\pgfpathlineto{\pgfqpoint{3.106927in}{0.567870in}}%
\pgfpathlineto{\pgfqpoint{3.106927in}{0.562480in}}%
\pgfpathlineto{\pgfqpoint{3.108224in}{0.573260in}}%
\pgfpathlineto{\pgfqpoint{3.110819in}{0.572182in}}%
\pgfpathlineto{\pgfqpoint{3.112117in}{0.565714in}}%
\pgfpathlineto{\pgfqpoint{3.114711in}{0.565714in}}%
\pgfpathlineto{\pgfqpoint{3.114711in}{0.567870in}}%
\pgfpathlineto{\pgfqpoint{3.116009in}{0.565714in}}%
\pgfpathlineto{\pgfqpoint{3.117306in}{0.565714in}}%
\pgfpathlineto{\pgfqpoint{3.117306in}{0.562480in}}%
\pgfpathlineto{\pgfqpoint{3.118604in}{0.572182in}}%
\pgfpathlineto{\pgfqpoint{3.119901in}{0.572182in}}%
\pgfpathlineto{\pgfqpoint{3.121199in}{0.563558in}}%
\pgfpathlineto{\pgfqpoint{3.122496in}{0.563558in}}%
\pgfpathlineto{\pgfqpoint{3.122496in}{0.574338in}}%
\pgfpathlineto{\pgfqpoint{3.123793in}{0.568948in}}%
\pgfpathlineto{\pgfqpoint{3.126388in}{0.567870in}}%
\pgfpathlineto{\pgfqpoint{3.126388in}{0.573260in}}%
\pgfpathlineto{\pgfqpoint{3.127686in}{0.562480in}}%
\pgfpathlineto{\pgfqpoint{3.128983in}{0.562480in}}%
\pgfpathlineto{\pgfqpoint{3.128983in}{0.566792in}}%
\pgfpathlineto{\pgfqpoint{3.130280in}{0.561402in}}%
\pgfpathlineto{\pgfqpoint{3.131578in}{0.561402in}}%
\pgfpathlineto{\pgfqpoint{3.131578in}{0.568948in}}%
\pgfpathlineto{\pgfqpoint{3.132875in}{0.565714in}}%
\pgfpathlineto{\pgfqpoint{3.134173in}{0.565714in}}%
\pgfpathlineto{\pgfqpoint{3.134173in}{0.567870in}}%
\pgfpathlineto{\pgfqpoint{3.135470in}{0.560324in}}%
\pgfpathlineto{\pgfqpoint{3.136768in}{0.560324in}}%
\pgfpathlineto{\pgfqpoint{3.136768in}{0.567870in}}%
\pgfpathlineto{\pgfqpoint{3.138065in}{0.567870in}}%
\pgfpathlineto{\pgfqpoint{3.139362in}{0.567870in}}%
\pgfpathlineto{\pgfqpoint{3.139362in}{0.564636in}}%
\pgfpathlineto{\pgfqpoint{3.140660in}{0.565714in}}%
\pgfpathlineto{\pgfqpoint{3.145850in}{0.564636in}}%
\pgfpathlineto{\pgfqpoint{3.147147in}{0.571104in}}%
\pgfpathlineto{\pgfqpoint{3.148444in}{0.571104in}}%
\pgfpathlineto{\pgfqpoint{3.148444in}{0.563558in}}%
\pgfpathlineto{\pgfqpoint{3.149742in}{0.571104in}}%
\pgfpathlineto{\pgfqpoint{3.151039in}{0.571104in}}%
\pgfpathlineto{\pgfqpoint{3.151039in}{0.560324in}}%
\pgfpathlineto{\pgfqpoint{3.152337in}{0.566792in}}%
\pgfpathlineto{\pgfqpoint{3.156229in}{0.565714in}}%
\pgfpathlineto{\pgfqpoint{3.156229in}{0.563558in}}%
\pgfpathlineto{\pgfqpoint{3.157526in}{0.566792in}}%
\pgfpathlineto{\pgfqpoint{3.164013in}{0.565714in}}%
\pgfpathlineto{\pgfqpoint{3.164013in}{0.568948in}}%
\pgfpathlineto{\pgfqpoint{3.165311in}{0.564636in}}%
\pgfpathlineto{\pgfqpoint{3.166608in}{0.564636in}}%
\pgfpathlineto{\pgfqpoint{3.166608in}{0.568948in}}%
\pgfpathlineto{\pgfqpoint{3.167906in}{0.568948in}}%
\pgfpathlineto{\pgfqpoint{3.169203in}{0.568948in}}%
\pgfpathlineto{\pgfqpoint{3.169203in}{0.563558in}}%
\pgfpathlineto{\pgfqpoint{3.170501in}{0.568948in}}%
\pgfpathlineto{\pgfqpoint{3.173095in}{0.567870in}}%
\pgfpathlineto{\pgfqpoint{3.173095in}{0.559246in}}%
\pgfpathlineto{\pgfqpoint{3.174393in}{0.561402in}}%
\pgfpathlineto{\pgfqpoint{3.175690in}{0.561402in}}%
\pgfpathlineto{\pgfqpoint{3.175690in}{0.565714in}}%
\pgfpathlineto{\pgfqpoint{3.176988in}{0.562480in}}%
\pgfpathlineto{\pgfqpoint{3.180880in}{0.561402in}}%
\pgfpathlineto{\pgfqpoint{3.180880in}{0.570026in}}%
\pgfpathlineto{\pgfqpoint{3.182177in}{0.564636in}}%
\pgfpathlineto{\pgfqpoint{3.183475in}{0.564636in}}%
\pgfpathlineto{\pgfqpoint{3.183475in}{0.566792in}}%
\pgfpathlineto{\pgfqpoint{3.184772in}{0.558168in}}%
\pgfpathlineto{\pgfqpoint{3.186070in}{0.558168in}}%
\pgfpathlineto{\pgfqpoint{3.186070in}{0.564636in}}%
\pgfpathlineto{\pgfqpoint{3.187367in}{0.560324in}}%
\pgfpathlineto{\pgfqpoint{3.188665in}{0.560324in}}%
\pgfpathlineto{\pgfqpoint{3.188665in}{0.570026in}}%
\pgfpathlineto{\pgfqpoint{3.189962in}{0.563558in}}%
\pgfpathlineto{\pgfqpoint{3.191259in}{0.563558in}}%
\pgfpathlineto{\pgfqpoint{3.191259in}{0.566792in}}%
\pgfpathlineto{\pgfqpoint{3.192557in}{0.566792in}}%
\pgfpathlineto{\pgfqpoint{3.193854in}{0.566792in}}%
\pgfpathlineto{\pgfqpoint{3.193854in}{0.570026in}}%
\pgfpathlineto{\pgfqpoint{3.195152in}{0.563558in}}%
\pgfpathlineto{\pgfqpoint{3.196449in}{0.563558in}}%
\pgfpathlineto{\pgfqpoint{3.196449in}{0.574338in}}%
\pgfpathlineto{\pgfqpoint{3.197746in}{0.564636in}}%
\pgfpathlineto{\pgfqpoint{3.199044in}{0.564636in}}%
\pgfpathlineto{\pgfqpoint{3.199044in}{0.567870in}}%
\pgfpathlineto{\pgfqpoint{3.200341in}{0.566792in}}%
\pgfpathlineto{\pgfqpoint{3.202936in}{0.567870in}}%
\pgfpathlineto{\pgfqpoint{3.204234in}{0.561402in}}%
\pgfpathlineto{\pgfqpoint{3.205531in}{0.561402in}}%
\pgfpathlineto{\pgfqpoint{3.206828in}{0.577572in}}%
\pgfpathlineto{\pgfqpoint{3.208126in}{0.577572in}}%
\pgfpathlineto{\pgfqpoint{3.209423in}{0.563558in}}%
\pgfpathlineto{\pgfqpoint{3.210721in}{0.563558in}}%
\pgfpathlineto{\pgfqpoint{3.210721in}{0.570026in}}%
\pgfpathlineto{\pgfqpoint{3.212018in}{0.560324in}}%
\pgfpathlineto{\pgfqpoint{3.215910in}{0.560324in}}%
\pgfpathlineto{\pgfqpoint{3.215910in}{0.571104in}}%
\pgfpathlineto{\pgfqpoint{3.217208in}{0.559246in}}%
\pgfpathlineto{\pgfqpoint{3.218505in}{0.559246in}}%
\pgfpathlineto{\pgfqpoint{3.219803in}{0.568948in}}%
\pgfpathlineto{\pgfqpoint{3.221100in}{0.568948in}}%
\pgfpathlineto{\pgfqpoint{3.221100in}{0.561402in}}%
\pgfpathlineto{\pgfqpoint{3.222398in}{0.568948in}}%
\pgfpathlineto{\pgfqpoint{3.223695in}{0.568948in}}%
\pgfpathlineto{\pgfqpoint{3.224992in}{0.561402in}}%
\pgfpathlineto{\pgfqpoint{3.226290in}{0.561402in}}%
\pgfpathlineto{\pgfqpoint{3.227587in}{0.573260in}}%
\pgfpathlineto{\pgfqpoint{3.228885in}{0.573260in}}%
\pgfpathlineto{\pgfqpoint{3.228885in}{0.560324in}}%
\pgfpathlineto{\pgfqpoint{3.230182in}{0.562480in}}%
\pgfpathlineto{\pgfqpoint{3.231479in}{0.562480in}}%
\pgfpathlineto{\pgfqpoint{3.231479in}{0.565714in}}%
\pgfpathlineto{\pgfqpoint{3.232777in}{0.560324in}}%
\pgfpathlineto{\pgfqpoint{3.234074in}{0.560324in}}%
\pgfpathlineto{\pgfqpoint{3.234074in}{0.562480in}}%
\pgfpathlineto{\pgfqpoint{3.235372in}{0.561402in}}%
\pgfpathlineto{\pgfqpoint{3.236669in}{0.561402in}}%
\pgfpathlineto{\pgfqpoint{3.236669in}{0.559246in}}%
\pgfpathlineto{\pgfqpoint{3.237967in}{0.565714in}}%
\pgfpathlineto{\pgfqpoint{3.239264in}{0.565714in}}%
\pgfpathlineto{\pgfqpoint{3.239264in}{0.567870in}}%
\pgfpathlineto{\pgfqpoint{3.240561in}{0.563558in}}%
\pgfpathlineto{\pgfqpoint{3.241859in}{0.563558in}}%
\pgfpathlineto{\pgfqpoint{3.241859in}{0.572182in}}%
\pgfpathlineto{\pgfqpoint{3.243156in}{0.563558in}}%
\pgfpathlineto{\pgfqpoint{3.244454in}{0.563558in}}%
\pgfpathlineto{\pgfqpoint{3.244454in}{0.556012in}}%
\pgfpathlineto{\pgfqpoint{3.245751in}{0.564636in}}%
\pgfpathlineto{\pgfqpoint{3.247049in}{0.564636in}}%
\pgfpathlineto{\pgfqpoint{3.247049in}{0.562480in}}%
\pgfpathlineto{\pgfqpoint{3.248346in}{0.562480in}}%
\pgfpathlineto{\pgfqpoint{3.249643in}{0.562480in}}%
\pgfpathlineto{\pgfqpoint{3.249643in}{0.566792in}}%
\pgfpathlineto{\pgfqpoint{3.250941in}{0.560324in}}%
\pgfpathlineto{\pgfqpoint{3.252238in}{0.560324in}}%
\pgfpathlineto{\pgfqpoint{3.252238in}{0.566792in}}%
\pgfpathlineto{\pgfqpoint{3.253536in}{0.566792in}}%
\pgfpathlineto{\pgfqpoint{3.254833in}{0.566792in}}%
\pgfpathlineto{\pgfqpoint{3.254833in}{0.562480in}}%
\pgfpathlineto{\pgfqpoint{3.256131in}{0.565714in}}%
\pgfpathlineto{\pgfqpoint{3.257428in}{0.565714in}}%
\pgfpathlineto{\pgfqpoint{3.258725in}{0.561402in}}%
\pgfpathlineto{\pgfqpoint{3.261320in}{0.561402in}}%
\pgfpathlineto{\pgfqpoint{3.261320in}{0.567870in}}%
\pgfpathlineto{\pgfqpoint{3.262618in}{0.565714in}}%
\pgfpathlineto{\pgfqpoint{3.263915in}{0.565714in}}%
\pgfpathlineto{\pgfqpoint{3.263915in}{0.562480in}}%
\pgfpathlineto{\pgfqpoint{3.265212in}{0.562480in}}%
\pgfpathlineto{\pgfqpoint{3.266510in}{0.562480in}}%
\pgfpathlineto{\pgfqpoint{3.266510in}{0.559246in}}%
\pgfpathlineto{\pgfqpoint{3.267807in}{0.564636in}}%
\pgfpathlineto{\pgfqpoint{3.269105in}{0.564636in}}%
\pgfpathlineto{\pgfqpoint{3.269105in}{0.570026in}}%
\pgfpathlineto{\pgfqpoint{3.270402in}{0.563558in}}%
\pgfpathlineto{\pgfqpoint{3.271700in}{0.563558in}}%
\pgfpathlineto{\pgfqpoint{3.271700in}{0.559246in}}%
\pgfpathlineto{\pgfqpoint{3.272997in}{0.566792in}}%
\pgfpathlineto{\pgfqpoint{3.274294in}{0.566792in}}%
\pgfpathlineto{\pgfqpoint{3.274294in}{0.562480in}}%
\pgfpathlineto{\pgfqpoint{3.275592in}{0.563558in}}%
\pgfpathlineto{\pgfqpoint{3.276889in}{0.563558in}}%
\pgfpathlineto{\pgfqpoint{3.276889in}{0.565714in}}%
\pgfpathlineto{\pgfqpoint{3.278187in}{0.559246in}}%
\pgfpathlineto{\pgfqpoint{3.279484in}{0.559246in}}%
\pgfpathlineto{\pgfqpoint{3.280782in}{0.566792in}}%
\pgfpathlineto{\pgfqpoint{3.282079in}{0.566792in}}%
\pgfpathlineto{\pgfqpoint{3.282079in}{0.564636in}}%
\pgfpathlineto{\pgfqpoint{3.283376in}{0.565714in}}%
\pgfpathlineto{\pgfqpoint{3.284674in}{0.565714in}}%
\pgfpathlineto{\pgfqpoint{3.285971in}{0.558168in}}%
\pgfpathlineto{\pgfqpoint{3.288566in}{0.559246in}}%
\pgfpathlineto{\pgfqpoint{3.288566in}{0.565714in}}%
\pgfpathlineto{\pgfqpoint{3.289864in}{0.563558in}}%
\pgfpathlineto{\pgfqpoint{3.293756in}{0.563558in}}%
\pgfpathlineto{\pgfqpoint{3.293756in}{0.565714in}}%
\pgfpathlineto{\pgfqpoint{3.295053in}{0.562480in}}%
\pgfpathlineto{\pgfqpoint{3.296351in}{0.562480in}}%
\pgfpathlineto{\pgfqpoint{3.296351in}{0.570026in}}%
\pgfpathlineto{\pgfqpoint{3.297648in}{0.559246in}}%
\pgfpathlineto{\pgfqpoint{3.298945in}{0.559246in}}%
\pgfpathlineto{\pgfqpoint{3.300243in}{0.564636in}}%
\pgfpathlineto{\pgfqpoint{3.302838in}{0.564636in}}%
\pgfpathlineto{\pgfqpoint{3.302838in}{0.567870in}}%
\pgfpathlineto{\pgfqpoint{3.304135in}{0.558168in}}%
\pgfpathlineto{\pgfqpoint{3.305433in}{0.558168in}}%
\pgfpathlineto{\pgfqpoint{3.305433in}{0.567870in}}%
\pgfpathlineto{\pgfqpoint{3.306730in}{0.561402in}}%
\pgfpathlineto{\pgfqpoint{3.308027in}{0.561402in}}%
\pgfpathlineto{\pgfqpoint{3.308027in}{0.567870in}}%
\pgfpathlineto{\pgfqpoint{3.309325in}{0.564636in}}%
\pgfpathlineto{\pgfqpoint{3.313217in}{0.564636in}}%
\pgfpathlineto{\pgfqpoint{3.313217in}{0.557090in}}%
\pgfpathlineto{\pgfqpoint{3.314515in}{0.561402in}}%
\pgfpathlineto{\pgfqpoint{3.317109in}{0.562480in}}%
\pgfpathlineto{\pgfqpoint{3.317109in}{0.559246in}}%
\pgfpathlineto{\pgfqpoint{3.318407in}{0.567870in}}%
\pgfpathlineto{\pgfqpoint{3.319704in}{0.567870in}}%
\pgfpathlineto{\pgfqpoint{3.319704in}{0.561402in}}%
\pgfpathlineto{\pgfqpoint{3.321002in}{0.564636in}}%
\pgfpathlineto{\pgfqpoint{3.323597in}{0.565714in}}%
\pgfpathlineto{\pgfqpoint{3.323597in}{0.561402in}}%
\pgfpathlineto{\pgfqpoint{3.324894in}{0.561402in}}%
\pgfpathlineto{\pgfqpoint{3.326191in}{0.561402in}}%
\pgfpathlineto{\pgfqpoint{3.326191in}{0.565714in}}%
\pgfpathlineto{\pgfqpoint{3.327489in}{0.565714in}}%
\pgfpathlineto{\pgfqpoint{3.328786in}{0.565714in}}%
\pgfpathlineto{\pgfqpoint{3.328786in}{0.567870in}}%
\pgfpathlineto{\pgfqpoint{3.330084in}{0.564636in}}%
\pgfpathlineto{\pgfqpoint{3.331381in}{0.564636in}}%
\pgfpathlineto{\pgfqpoint{3.331381in}{0.558168in}}%
\pgfpathlineto{\pgfqpoint{3.332678in}{0.561402in}}%
\pgfpathlineto{\pgfqpoint{3.333976in}{0.561402in}}%
\pgfpathlineto{\pgfqpoint{3.333976in}{0.564636in}}%
\pgfpathlineto{\pgfqpoint{3.335273in}{0.558168in}}%
\pgfpathlineto{\pgfqpoint{3.336571in}{0.558168in}}%
\pgfpathlineto{\pgfqpoint{3.336571in}{0.565714in}}%
\pgfpathlineto{\pgfqpoint{3.337868in}{0.562480in}}%
\pgfpathlineto{\pgfqpoint{3.339166in}{0.562480in}}%
\pgfpathlineto{\pgfqpoint{3.339166in}{0.566792in}}%
\pgfpathlineto{\pgfqpoint{3.340463in}{0.563558in}}%
\pgfpathlineto{\pgfqpoint{3.344355in}{0.564636in}}%
\pgfpathlineto{\pgfqpoint{3.344355in}{0.561402in}}%
\pgfpathlineto{\pgfqpoint{3.345653in}{0.561402in}}%
\pgfpathlineto{\pgfqpoint{3.346950in}{0.561402in}}%
\pgfpathlineto{\pgfqpoint{3.346950in}{0.564636in}}%
\pgfpathlineto{\pgfqpoint{3.348248in}{0.562480in}}%
\pgfpathlineto{\pgfqpoint{3.349545in}{0.562480in}}%
\pgfpathlineto{\pgfqpoint{3.349545in}{0.565714in}}%
\pgfpathlineto{\pgfqpoint{3.350842in}{0.559246in}}%
\pgfpathlineto{\pgfqpoint{3.353437in}{0.560324in}}%
\pgfpathlineto{\pgfqpoint{3.354735in}{0.568948in}}%
\pgfpathlineto{\pgfqpoint{3.356032in}{0.568948in}}%
\pgfpathlineto{\pgfqpoint{3.357330in}{0.564636in}}%
\pgfpathlineto{\pgfqpoint{3.358627in}{0.564636in}}%
\pgfpathlineto{\pgfqpoint{3.358627in}{0.562480in}}%
\pgfpathlineto{\pgfqpoint{3.359924in}{0.563558in}}%
\pgfpathlineto{\pgfqpoint{3.361222in}{0.563558in}}%
\pgfpathlineto{\pgfqpoint{3.361222in}{0.570026in}}%
\pgfpathlineto{\pgfqpoint{3.362519in}{0.567870in}}%
\pgfpathlineto{\pgfqpoint{3.363817in}{0.567870in}}%
\pgfpathlineto{\pgfqpoint{3.363817in}{0.571104in}}%
\pgfpathlineto{\pgfqpoint{3.365114in}{0.567870in}}%
\pgfpathlineto{\pgfqpoint{3.366411in}{0.567870in}}%
\pgfpathlineto{\pgfqpoint{3.366411in}{0.558168in}}%
\pgfpathlineto{\pgfqpoint{3.367709in}{0.566792in}}%
\pgfpathlineto{\pgfqpoint{3.369006in}{0.566792in}}%
\pgfpathlineto{\pgfqpoint{3.370304in}{0.560324in}}%
\pgfpathlineto{\pgfqpoint{3.374196in}{0.561402in}}%
\pgfpathlineto{\pgfqpoint{3.374196in}{0.564636in}}%
\pgfpathlineto{\pgfqpoint{3.375493in}{0.559246in}}%
\pgfpathlineto{\pgfqpoint{3.376791in}{0.559246in}}%
\pgfpathlineto{\pgfqpoint{3.376791in}{0.564636in}}%
\pgfpathlineto{\pgfqpoint{3.378088in}{0.563558in}}%
\pgfpathlineto{\pgfqpoint{3.379386in}{0.563558in}}%
\pgfpathlineto{\pgfqpoint{3.379386in}{0.565714in}}%
\pgfpathlineto{\pgfqpoint{3.380683in}{0.561402in}}%
\pgfpathlineto{\pgfqpoint{3.381981in}{0.561402in}}%
\pgfpathlineto{\pgfqpoint{3.381981in}{0.565714in}}%
\pgfpathlineto{\pgfqpoint{3.383278in}{0.562480in}}%
\pgfpathlineto{\pgfqpoint{3.384575in}{0.562480in}}%
\pgfpathlineto{\pgfqpoint{3.384575in}{0.560324in}}%
\pgfpathlineto{\pgfqpoint{3.385873in}{0.560324in}}%
\pgfpathlineto{\pgfqpoint{3.387170in}{0.560324in}}%
\pgfpathlineto{\pgfqpoint{3.387170in}{0.567870in}}%
\pgfpathlineto{\pgfqpoint{3.388468in}{0.559246in}}%
\pgfpathlineto{\pgfqpoint{3.389765in}{0.559246in}}%
\pgfpathlineto{\pgfqpoint{3.389765in}{0.553856in}}%
\pgfpathlineto{\pgfqpoint{3.391063in}{0.559246in}}%
\pgfpathlineto{\pgfqpoint{3.392360in}{0.559246in}}%
\pgfpathlineto{\pgfqpoint{3.392360in}{0.566792in}}%
\pgfpathlineto{\pgfqpoint{3.393657in}{0.563558in}}%
\pgfpathlineto{\pgfqpoint{3.394955in}{0.563558in}}%
\pgfpathlineto{\pgfqpoint{3.394955in}{0.566792in}}%
\pgfpathlineto{\pgfqpoint{3.396252in}{0.557090in}}%
\pgfpathlineto{\pgfqpoint{3.397550in}{0.557090in}}%
\pgfpathlineto{\pgfqpoint{3.398847in}{0.566792in}}%
\pgfpathlineto{\pgfqpoint{3.400144in}{0.566792in}}%
\pgfpathlineto{\pgfqpoint{3.400144in}{0.564636in}}%
\pgfpathlineto{\pgfqpoint{3.401442in}{0.565714in}}%
\pgfpathlineto{\pgfqpoint{3.404037in}{0.566792in}}%
\pgfpathlineto{\pgfqpoint{3.404037in}{0.559246in}}%
\pgfpathlineto{\pgfqpoint{3.405334in}{0.562480in}}%
\pgfpathlineto{\pgfqpoint{3.406632in}{0.562480in}}%
\pgfpathlineto{\pgfqpoint{3.407929in}{0.559246in}}%
\pgfpathlineto{\pgfqpoint{3.409226in}{0.559246in}}%
\pgfpathlineto{\pgfqpoint{3.409226in}{0.563558in}}%
\pgfpathlineto{\pgfqpoint{3.410524in}{0.559246in}}%
\pgfpathlineto{\pgfqpoint{3.411821in}{0.559246in}}%
\pgfpathlineto{\pgfqpoint{3.411821in}{0.561402in}}%
\pgfpathlineto{\pgfqpoint{3.413119in}{0.558168in}}%
\pgfpathlineto{\pgfqpoint{3.414416in}{0.558168in}}%
\pgfpathlineto{\pgfqpoint{3.414416in}{0.567870in}}%
\pgfpathlineto{\pgfqpoint{3.415714in}{0.566792in}}%
\pgfpathlineto{\pgfqpoint{3.417011in}{0.566792in}}%
\pgfpathlineto{\pgfqpoint{3.417011in}{0.570026in}}%
\pgfpathlineto{\pgfqpoint{3.418308in}{0.560324in}}%
\pgfpathlineto{\pgfqpoint{3.419606in}{0.560324in}}%
\pgfpathlineto{\pgfqpoint{3.419606in}{0.568948in}}%
\pgfpathlineto{\pgfqpoint{3.420903in}{0.561402in}}%
\pgfpathlineto{\pgfqpoint{3.428688in}{0.562480in}}%
\pgfpathlineto{\pgfqpoint{3.428688in}{0.559246in}}%
\pgfpathlineto{\pgfqpoint{3.429985in}{0.563558in}}%
\pgfpathlineto{\pgfqpoint{3.432580in}{0.562480in}}%
\pgfpathlineto{\pgfqpoint{3.432580in}{0.560324in}}%
\pgfpathlineto{\pgfqpoint{3.433877in}{0.570026in}}%
\pgfpathlineto{\pgfqpoint{3.435175in}{0.570026in}}%
\pgfpathlineto{\pgfqpoint{3.435175in}{0.559246in}}%
\pgfpathlineto{\pgfqpoint{3.436472in}{0.563558in}}%
\pgfpathlineto{\pgfqpoint{3.437770in}{0.563558in}}%
\pgfpathlineto{\pgfqpoint{3.437770in}{0.559246in}}%
\pgfpathlineto{\pgfqpoint{3.439067in}{0.560324in}}%
\pgfpathlineto{\pgfqpoint{3.441662in}{0.559246in}}%
\pgfpathlineto{\pgfqpoint{3.441662in}{0.557090in}}%
\pgfpathlineto{\pgfqpoint{3.442959in}{0.560324in}}%
\pgfpathlineto{\pgfqpoint{3.444257in}{0.560324in}}%
\pgfpathlineto{\pgfqpoint{3.444257in}{0.563558in}}%
\pgfpathlineto{\pgfqpoint{3.445554in}{0.562480in}}%
\pgfpathlineto{\pgfqpoint{3.448149in}{0.563558in}}%
\pgfpathlineto{\pgfqpoint{3.448149in}{0.560324in}}%
\pgfpathlineto{\pgfqpoint{3.449447in}{0.562480in}}%
\pgfpathlineto{\pgfqpoint{3.450744in}{0.562480in}}%
\pgfpathlineto{\pgfqpoint{3.450744in}{0.558168in}}%
\pgfpathlineto{\pgfqpoint{3.452041in}{0.559246in}}%
\pgfpathlineto{\pgfqpoint{3.454636in}{0.560324in}}%
\pgfpathlineto{\pgfqpoint{3.454636in}{0.561402in}}%
\pgfpathlineto{\pgfqpoint{3.455934in}{0.561402in}}%
\pgfpathlineto{\pgfqpoint{3.458529in}{0.562480in}}%
\pgfpathlineto{\pgfqpoint{3.458529in}{0.567870in}}%
\pgfpathlineto{\pgfqpoint{3.459826in}{0.561402in}}%
\pgfpathlineto{\pgfqpoint{3.463718in}{0.560324in}}%
\pgfpathlineto{\pgfqpoint{3.463718in}{0.557090in}}%
\pgfpathlineto{\pgfqpoint{3.465016in}{0.560324in}}%
\pgfpathlineto{\pgfqpoint{3.466313in}{0.560324in}}%
\pgfpathlineto{\pgfqpoint{3.466313in}{0.564636in}}%
\pgfpathlineto{\pgfqpoint{3.467610in}{0.556012in}}%
\pgfpathlineto{\pgfqpoint{3.468908in}{0.556012in}}%
\pgfpathlineto{\pgfqpoint{3.470205in}{0.562480in}}%
\pgfpathlineto{\pgfqpoint{3.472800in}{0.561402in}}%
\pgfpathlineto{\pgfqpoint{3.472800in}{0.558168in}}%
\pgfpathlineto{\pgfqpoint{3.474098in}{0.559246in}}%
\pgfpathlineto{\pgfqpoint{3.476692in}{0.558168in}}%
\pgfpathlineto{\pgfqpoint{3.476692in}{0.561402in}}%
\pgfpathlineto{\pgfqpoint{3.477990in}{0.557090in}}%
\pgfpathlineto{\pgfqpoint{3.479287in}{0.557090in}}%
\pgfpathlineto{\pgfqpoint{3.479287in}{0.562480in}}%
\pgfpathlineto{\pgfqpoint{3.480585in}{0.562480in}}%
\pgfpathlineto{\pgfqpoint{3.483180in}{0.561402in}}%
\pgfpathlineto{\pgfqpoint{3.483180in}{0.559246in}}%
\pgfpathlineto{\pgfqpoint{3.484477in}{0.560324in}}%
\pgfpathlineto{\pgfqpoint{3.485774in}{0.560324in}}%
\pgfpathlineto{\pgfqpoint{3.487072in}{0.563558in}}%
\pgfpathlineto{\pgfqpoint{3.488369in}{0.563558in}}%
\pgfpathlineto{\pgfqpoint{3.488369in}{0.561402in}}%
\pgfpathlineto{\pgfqpoint{3.489667in}{0.562480in}}%
\pgfpathlineto{\pgfqpoint{3.492262in}{0.562480in}}%
\pgfpathlineto{\pgfqpoint{3.492262in}{0.564636in}}%
\pgfpathlineto{\pgfqpoint{3.493559in}{0.559246in}}%
\pgfpathlineto{\pgfqpoint{3.494856in}{0.559246in}}%
\pgfpathlineto{\pgfqpoint{3.496154in}{0.564636in}}%
\pgfpathlineto{\pgfqpoint{3.500046in}{0.564636in}}%
\pgfpathlineto{\pgfqpoint{3.500046in}{0.559246in}}%
\pgfpathlineto{\pgfqpoint{3.501344in}{0.565714in}}%
\pgfpathlineto{\pgfqpoint{3.502641in}{0.565714in}}%
\pgfpathlineto{\pgfqpoint{3.502641in}{0.559246in}}%
\pgfpathlineto{\pgfqpoint{3.503938in}{0.559246in}}%
\pgfpathlineto{\pgfqpoint{3.505236in}{0.559246in}}%
\pgfpathlineto{\pgfqpoint{3.505236in}{0.556012in}}%
\pgfpathlineto{\pgfqpoint{3.506533in}{0.561402in}}%
\pgfpathlineto{\pgfqpoint{3.509128in}{0.562480in}}%
\pgfpathlineto{\pgfqpoint{3.509128in}{0.559246in}}%
\pgfpathlineto{\pgfqpoint{3.510425in}{0.560324in}}%
\pgfpathlineto{\pgfqpoint{3.516913in}{0.559246in}}%
\pgfpathlineto{\pgfqpoint{3.516913in}{0.562480in}}%
\pgfpathlineto{\pgfqpoint{3.518210in}{0.554934in}}%
\pgfpathlineto{\pgfqpoint{3.519507in}{0.554934in}}%
\pgfpathlineto{\pgfqpoint{3.520805in}{0.562480in}}%
\pgfpathlineto{\pgfqpoint{3.522102in}{0.562480in}}%
\pgfpathlineto{\pgfqpoint{3.522102in}{0.553856in}}%
\pgfpathlineto{\pgfqpoint{3.523400in}{0.564636in}}%
\pgfpathlineto{\pgfqpoint{3.525995in}{0.564636in}}%
\pgfpathlineto{\pgfqpoint{3.525995in}{0.559246in}}%
\pgfpathlineto{\pgfqpoint{3.527292in}{0.563558in}}%
\pgfpathlineto{\pgfqpoint{3.528589in}{0.563558in}}%
\pgfpathlineto{\pgfqpoint{3.528589in}{0.561402in}}%
\pgfpathlineto{\pgfqpoint{3.529887in}{0.564636in}}%
\pgfpathlineto{\pgfqpoint{3.531184in}{0.564636in}}%
\pgfpathlineto{\pgfqpoint{3.531184in}{0.556012in}}%
\pgfpathlineto{\pgfqpoint{3.532482in}{0.565714in}}%
\pgfpathlineto{\pgfqpoint{3.535077in}{0.564636in}}%
\pgfpathlineto{\pgfqpoint{3.536374in}{0.560324in}}%
\pgfpathlineto{\pgfqpoint{3.541564in}{0.560324in}}%
\pgfpathlineto{\pgfqpoint{3.541564in}{0.562480in}}%
\pgfpathlineto{\pgfqpoint{3.542861in}{0.561402in}}%
\pgfpathlineto{\pgfqpoint{3.544158in}{0.561402in}}%
\pgfpathlineto{\pgfqpoint{3.544158in}{0.557090in}}%
\pgfpathlineto{\pgfqpoint{3.545456in}{0.558168in}}%
\pgfpathlineto{\pgfqpoint{3.546753in}{0.558168in}}%
\pgfpathlineto{\pgfqpoint{3.546753in}{0.562480in}}%
\pgfpathlineto{\pgfqpoint{3.548051in}{0.560324in}}%
\pgfpathlineto{\pgfqpoint{3.549348in}{0.560324in}}%
\pgfpathlineto{\pgfqpoint{3.549348in}{0.557090in}}%
\pgfpathlineto{\pgfqpoint{3.550646in}{0.560324in}}%
\pgfpathlineto{\pgfqpoint{3.551943in}{0.560324in}}%
\pgfpathlineto{\pgfqpoint{3.551943in}{0.562480in}}%
\pgfpathlineto{\pgfqpoint{3.553240in}{0.556012in}}%
\pgfpathlineto{\pgfqpoint{3.554538in}{0.556012in}}%
\pgfpathlineto{\pgfqpoint{3.554538in}{0.561402in}}%
\pgfpathlineto{\pgfqpoint{3.555835in}{0.561402in}}%
\pgfpathlineto{\pgfqpoint{3.557133in}{0.561402in}}%
\pgfpathlineto{\pgfqpoint{3.557133in}{0.559246in}}%
\pgfpathlineto{\pgfqpoint{3.558430in}{0.562480in}}%
\pgfpathlineto{\pgfqpoint{3.559728in}{0.562480in}}%
\pgfpathlineto{\pgfqpoint{3.559728in}{0.565714in}}%
\pgfpathlineto{\pgfqpoint{3.561025in}{0.559246in}}%
\pgfpathlineto{\pgfqpoint{3.562322in}{0.559246in}}%
\pgfpathlineto{\pgfqpoint{3.562322in}{0.564636in}}%
\pgfpathlineto{\pgfqpoint{3.563620in}{0.559246in}}%
\pgfpathlineto{\pgfqpoint{3.566215in}{0.558168in}}%
\pgfpathlineto{\pgfqpoint{3.566215in}{0.557090in}}%
\pgfpathlineto{\pgfqpoint{3.567512in}{0.563558in}}%
\pgfpathlineto{\pgfqpoint{3.568810in}{0.563558in}}%
\pgfpathlineto{\pgfqpoint{3.568810in}{0.554934in}}%
\pgfpathlineto{\pgfqpoint{3.570107in}{0.556012in}}%
\pgfpathlineto{\pgfqpoint{3.571404in}{0.556012in}}%
\pgfpathlineto{\pgfqpoint{3.571404in}{0.562480in}}%
\pgfpathlineto{\pgfqpoint{3.572702in}{0.561402in}}%
\pgfpathlineto{\pgfqpoint{3.573999in}{0.561402in}}%
\pgfpathlineto{\pgfqpoint{3.573999in}{0.559246in}}%
\pgfpathlineto{\pgfqpoint{3.575297in}{0.560324in}}%
\pgfpathlineto{\pgfqpoint{3.577891in}{0.560324in}}%
\pgfpathlineto{\pgfqpoint{3.577891in}{0.563558in}}%
\pgfpathlineto{\pgfqpoint{3.579189in}{0.559246in}}%
\pgfpathlineto{\pgfqpoint{3.580486in}{0.559246in}}%
\pgfpathlineto{\pgfqpoint{3.580486in}{0.557090in}}%
\pgfpathlineto{\pgfqpoint{3.581784in}{0.560324in}}%
\pgfpathlineto{\pgfqpoint{3.584379in}{0.560324in}}%
\pgfpathlineto{\pgfqpoint{3.585676in}{0.567870in}}%
\pgfpathlineto{\pgfqpoint{3.586973in}{0.567870in}}%
\pgfpathlineto{\pgfqpoint{3.586973in}{0.556012in}}%
\pgfpathlineto{\pgfqpoint{3.588271in}{0.560324in}}%
\pgfpathlineto{\pgfqpoint{3.589568in}{0.560324in}}%
\pgfpathlineto{\pgfqpoint{3.590866in}{0.556012in}}%
\pgfpathlineto{\pgfqpoint{3.592163in}{0.556012in}}%
\pgfpathlineto{\pgfqpoint{3.592163in}{0.558168in}}%
\pgfpathlineto{\pgfqpoint{3.593461in}{0.556012in}}%
\pgfpathlineto{\pgfqpoint{3.594758in}{0.556012in}}%
\pgfpathlineto{\pgfqpoint{3.594758in}{0.561402in}}%
\pgfpathlineto{\pgfqpoint{3.596055in}{0.560324in}}%
\pgfpathlineto{\pgfqpoint{3.601245in}{0.560324in}}%
\pgfpathlineto{\pgfqpoint{3.602543in}{0.563558in}}%
\pgfpathlineto{\pgfqpoint{3.603840in}{0.563558in}}%
\pgfpathlineto{\pgfqpoint{3.603840in}{0.559246in}}%
\pgfpathlineto{\pgfqpoint{3.605137in}{0.564636in}}%
\pgfpathlineto{\pgfqpoint{3.606435in}{0.564636in}}%
\pgfpathlineto{\pgfqpoint{3.606435in}{0.559246in}}%
\pgfpathlineto{\pgfqpoint{3.607732in}{0.561402in}}%
\pgfpathlineto{\pgfqpoint{3.609030in}{0.561402in}}%
\pgfpathlineto{\pgfqpoint{3.609030in}{0.556012in}}%
\pgfpathlineto{\pgfqpoint{3.610327in}{0.560324in}}%
\pgfpathlineto{\pgfqpoint{3.612922in}{0.559246in}}%
\pgfpathlineto{\pgfqpoint{3.612922in}{0.562480in}}%
\pgfpathlineto{\pgfqpoint{3.614219in}{0.560324in}}%
\pgfpathlineto{\pgfqpoint{3.615517in}{0.560324in}}%
\pgfpathlineto{\pgfqpoint{3.615517in}{0.557090in}}%
\pgfpathlineto{\pgfqpoint{3.616814in}{0.562480in}}%
\pgfpathlineto{\pgfqpoint{3.618112in}{0.562480in}}%
\pgfpathlineto{\pgfqpoint{3.618112in}{0.560324in}}%
\pgfpathlineto{\pgfqpoint{3.619409in}{0.562480in}}%
\pgfpathlineto{\pgfqpoint{3.620706in}{0.562480in}}%
\pgfpathlineto{\pgfqpoint{3.620706in}{0.560324in}}%
\pgfpathlineto{\pgfqpoint{3.622004in}{0.560324in}}%
\pgfpathlineto{\pgfqpoint{3.625896in}{0.559246in}}%
\pgfpathlineto{\pgfqpoint{3.627194in}{0.553856in}}%
\pgfpathlineto{\pgfqpoint{3.628491in}{0.553856in}}%
\pgfpathlineto{\pgfqpoint{3.628491in}{0.559246in}}%
\pgfpathlineto{\pgfqpoint{3.629788in}{0.557090in}}%
\pgfpathlineto{\pgfqpoint{3.631086in}{0.557090in}}%
\pgfpathlineto{\pgfqpoint{3.631086in}{0.559246in}}%
\pgfpathlineto{\pgfqpoint{3.632383in}{0.557090in}}%
\pgfpathlineto{\pgfqpoint{3.634978in}{0.556012in}}%
\pgfpathlineto{\pgfqpoint{3.634978in}{0.559246in}}%
\pgfpathlineto{\pgfqpoint{3.636276in}{0.559246in}}%
\pgfpathlineto{\pgfqpoint{3.638870in}{0.559246in}}%
\pgfpathlineto{\pgfqpoint{3.640168in}{0.553856in}}%
\pgfpathlineto{\pgfqpoint{3.641465in}{0.553856in}}%
\pgfpathlineto{\pgfqpoint{3.641465in}{0.559246in}}%
\pgfpathlineto{\pgfqpoint{3.642763in}{0.559246in}}%
\pgfpathlineto{\pgfqpoint{3.651845in}{0.559246in}}%
\pgfpathlineto{\pgfqpoint{3.653142in}{0.554934in}}%
\pgfpathlineto{\pgfqpoint{3.654439in}{0.554934in}}%
\pgfpathlineto{\pgfqpoint{3.655737in}{0.561402in}}%
\pgfpathlineto{\pgfqpoint{3.657034in}{0.561402in}}%
\pgfpathlineto{\pgfqpoint{3.658332in}{0.557090in}}%
\pgfpathlineto{\pgfqpoint{3.660927in}{0.558168in}}%
\pgfpathlineto{\pgfqpoint{3.662224in}{0.560324in}}%
\pgfpathlineto{\pgfqpoint{3.663521in}{0.560324in}}%
\pgfpathlineto{\pgfqpoint{3.663521in}{0.562480in}}%
\pgfpathlineto{\pgfqpoint{3.664819in}{0.556012in}}%
\pgfpathlineto{\pgfqpoint{3.666116in}{0.556012in}}%
\pgfpathlineto{\pgfqpoint{3.666116in}{0.560324in}}%
\pgfpathlineto{\pgfqpoint{3.667414in}{0.560324in}}%
\pgfpathlineto{\pgfqpoint{3.668711in}{0.560324in}}%
\pgfpathlineto{\pgfqpoint{3.670009in}{0.556012in}}%
\pgfpathlineto{\pgfqpoint{3.677793in}{0.556012in}}%
\pgfpathlineto{\pgfqpoint{3.677793in}{0.562480in}}%
\pgfpathlineto{\pgfqpoint{3.679090in}{0.556012in}}%
\pgfpathlineto{\pgfqpoint{3.680388in}{0.556012in}}%
\pgfpathlineto{\pgfqpoint{3.680388in}{0.561402in}}%
\pgfpathlineto{\pgfqpoint{3.681685in}{0.557090in}}%
\pgfpathlineto{\pgfqpoint{3.685578in}{0.556012in}}%
\pgfpathlineto{\pgfqpoint{3.685578in}{0.553856in}}%
\pgfpathlineto{\pgfqpoint{3.686875in}{0.560324in}}%
\pgfpathlineto{\pgfqpoint{3.688172in}{0.560324in}}%
\pgfpathlineto{\pgfqpoint{3.689470in}{0.557090in}}%
\pgfpathlineto{\pgfqpoint{3.690767in}{0.557090in}}%
\pgfpathlineto{\pgfqpoint{3.690767in}{0.561402in}}%
\pgfpathlineto{\pgfqpoint{3.692065in}{0.556012in}}%
\pgfpathlineto{\pgfqpoint{3.693362in}{0.556012in}}%
\pgfpathlineto{\pgfqpoint{3.693362in}{0.559246in}}%
\pgfpathlineto{\pgfqpoint{3.694660in}{0.558168in}}%
\pgfpathlineto{\pgfqpoint{3.695957in}{0.558168in}}%
\pgfpathlineto{\pgfqpoint{3.695957in}{0.556012in}}%
\pgfpathlineto{\pgfqpoint{3.697254in}{0.557090in}}%
\pgfpathlineto{\pgfqpoint{3.698552in}{0.557090in}}%
\pgfpathlineto{\pgfqpoint{3.698552in}{0.562480in}}%
\pgfpathlineto{\pgfqpoint{3.699849in}{0.562480in}}%
\pgfpathlineto{\pgfqpoint{3.701147in}{0.562480in}}%
\pgfpathlineto{\pgfqpoint{3.702444in}{0.559246in}}%
\pgfpathlineto{\pgfqpoint{3.703742in}{0.559246in}}%
\pgfpathlineto{\pgfqpoint{3.705039in}{0.565714in}}%
\pgfpathlineto{\pgfqpoint{3.706336in}{0.565714in}}%
\pgfpathlineto{\pgfqpoint{3.706336in}{0.556012in}}%
\pgfpathlineto{\pgfqpoint{3.707634in}{0.561402in}}%
\pgfpathlineto{\pgfqpoint{3.708931in}{0.561402in}}%
\pgfpathlineto{\pgfqpoint{3.708931in}{0.554934in}}%
\pgfpathlineto{\pgfqpoint{3.710229in}{0.559246in}}%
\pgfpathlineto{\pgfqpoint{3.711526in}{0.559246in}}%
\pgfpathlineto{\pgfqpoint{3.711526in}{0.554934in}}%
\pgfpathlineto{\pgfqpoint{3.712823in}{0.556012in}}%
\pgfpathlineto{\pgfqpoint{3.714121in}{0.556012in}}%
\pgfpathlineto{\pgfqpoint{3.714121in}{0.563558in}}%
\pgfpathlineto{\pgfqpoint{3.715418in}{0.559246in}}%
\pgfpathlineto{\pgfqpoint{3.716716in}{0.559246in}}%
\pgfpathlineto{\pgfqpoint{3.718013in}{0.554934in}}%
\pgfpathlineto{\pgfqpoint{3.720608in}{0.556012in}}%
\pgfpathlineto{\pgfqpoint{3.720608in}{0.558168in}}%
\pgfpathlineto{\pgfqpoint{3.721905in}{0.556012in}}%
\pgfpathlineto{\pgfqpoint{3.723203in}{0.556012in}}%
\pgfpathlineto{\pgfqpoint{3.724500in}{0.559246in}}%
\pgfpathlineto{\pgfqpoint{3.725798in}{0.559246in}}%
\pgfpathlineto{\pgfqpoint{3.727095in}{0.556012in}}%
\pgfpathlineto{\pgfqpoint{3.728393in}{0.556012in}}%
\pgfpathlineto{\pgfqpoint{3.729690in}{0.559246in}}%
\pgfpathlineto{\pgfqpoint{3.730987in}{0.559246in}}%
\pgfpathlineto{\pgfqpoint{3.730987in}{0.556012in}}%
\pgfpathlineto{\pgfqpoint{3.732285in}{0.556012in}}%
\pgfpathlineto{\pgfqpoint{3.733582in}{0.556012in}}%
\pgfpathlineto{\pgfqpoint{3.733582in}{0.558168in}}%
\pgfpathlineto{\pgfqpoint{3.734880in}{0.558168in}}%
\pgfpathlineto{\pgfqpoint{3.736177in}{0.558168in}}%
\pgfpathlineto{\pgfqpoint{3.736177in}{0.562480in}}%
\pgfpathlineto{\pgfqpoint{3.737475in}{0.560324in}}%
\pgfpathlineto{\pgfqpoint{3.738772in}{0.560324in}}%
\pgfpathlineto{\pgfqpoint{3.738772in}{0.557090in}}%
\pgfpathlineto{\pgfqpoint{3.740069in}{0.558168in}}%
\pgfpathlineto{\pgfqpoint{3.741367in}{0.558168in}}%
\pgfpathlineto{\pgfqpoint{3.741367in}{0.556012in}}%
\pgfpathlineto{\pgfqpoint{3.742664in}{0.557090in}}%
\pgfpathlineto{\pgfqpoint{3.743962in}{0.557090in}}%
\pgfpathlineto{\pgfqpoint{3.743962in}{0.561402in}}%
\pgfpathlineto{\pgfqpoint{3.745259in}{0.559246in}}%
\pgfpathlineto{\pgfqpoint{3.747854in}{0.558168in}}%
\pgfpathlineto{\pgfqpoint{3.747854in}{0.561402in}}%
\pgfpathlineto{\pgfqpoint{3.749151in}{0.559246in}}%
\pgfpathlineto{\pgfqpoint{3.751746in}{0.559246in}}%
\pgfpathlineto{\pgfqpoint{3.751746in}{0.557090in}}%
\pgfpathlineto{\pgfqpoint{3.753044in}{0.561402in}}%
\pgfpathlineto{\pgfqpoint{3.755638in}{0.562480in}}%
\pgfpathlineto{\pgfqpoint{3.755638in}{0.558168in}}%
\pgfpathlineto{\pgfqpoint{3.756936in}{0.558168in}}%
\pgfpathlineto{\pgfqpoint{3.758233in}{0.558168in}}%
\pgfpathlineto{\pgfqpoint{3.758233in}{0.553856in}}%
\pgfpathlineto{\pgfqpoint{3.759531in}{0.557090in}}%
\pgfpathlineto{\pgfqpoint{3.760828in}{0.557090in}}%
\pgfpathlineto{\pgfqpoint{3.760828in}{0.554934in}}%
\pgfpathlineto{\pgfqpoint{3.762126in}{0.559246in}}%
\pgfpathlineto{\pgfqpoint{3.763423in}{0.559246in}}%
\pgfpathlineto{\pgfqpoint{3.763423in}{0.557090in}}%
\pgfpathlineto{\pgfqpoint{3.764720in}{0.558168in}}%
\pgfpathlineto{\pgfqpoint{3.768613in}{0.557090in}}%
\pgfpathlineto{\pgfqpoint{3.768613in}{0.554934in}}%
\pgfpathlineto{\pgfqpoint{3.769910in}{0.556012in}}%
\pgfpathlineto{\pgfqpoint{3.771208in}{0.556012in}}%
\pgfpathlineto{\pgfqpoint{3.771208in}{0.560324in}}%
\pgfpathlineto{\pgfqpoint{3.772505in}{0.558168in}}%
\pgfpathlineto{\pgfqpoint{3.775100in}{0.557090in}}%
\pgfpathlineto{\pgfqpoint{3.776397in}{0.553856in}}%
\pgfpathlineto{\pgfqpoint{3.777695in}{0.553856in}}%
\pgfpathlineto{\pgfqpoint{3.777695in}{0.564636in}}%
\pgfpathlineto{\pgfqpoint{3.778992in}{0.552778in}}%
\pgfpathlineto{\pgfqpoint{3.780289in}{0.552778in}}%
\pgfpathlineto{\pgfqpoint{3.780289in}{0.559246in}}%
\pgfpathlineto{\pgfqpoint{3.781587in}{0.556012in}}%
\pgfpathlineto{\pgfqpoint{3.782884in}{0.556012in}}%
\pgfpathlineto{\pgfqpoint{3.782884in}{0.558168in}}%
\pgfpathlineto{\pgfqpoint{3.784182in}{0.558168in}}%
\pgfpathlineto{\pgfqpoint{3.788074in}{0.559246in}}%
\pgfpathlineto{\pgfqpoint{3.788074in}{0.561402in}}%
\pgfpathlineto{\pgfqpoint{3.789371in}{0.554934in}}%
\pgfpathlineto{\pgfqpoint{3.790669in}{0.554934in}}%
\pgfpathlineto{\pgfqpoint{3.790669in}{0.558168in}}%
\pgfpathlineto{\pgfqpoint{3.791966in}{0.557090in}}%
\pgfpathlineto{\pgfqpoint{3.795859in}{0.557090in}}%
\pgfpathlineto{\pgfqpoint{3.795859in}{0.560324in}}%
\pgfpathlineto{\pgfqpoint{3.797156in}{0.560324in}}%
\pgfpathlineto{\pgfqpoint{3.798453in}{0.560324in}}%
\pgfpathlineto{\pgfqpoint{3.799751in}{0.557090in}}%
\pgfpathlineto{\pgfqpoint{3.801048in}{0.557090in}}%
\pgfpathlineto{\pgfqpoint{3.801048in}{0.559246in}}%
\pgfpathlineto{\pgfqpoint{3.802346in}{0.557090in}}%
\pgfpathlineto{\pgfqpoint{3.803643in}{0.557090in}}%
\pgfpathlineto{\pgfqpoint{3.803643in}{0.560324in}}%
\pgfpathlineto{\pgfqpoint{3.804941in}{0.559246in}}%
\pgfpathlineto{\pgfqpoint{3.806238in}{0.559246in}}%
\pgfpathlineto{\pgfqpoint{3.806238in}{0.561402in}}%
\pgfpathlineto{\pgfqpoint{3.807535in}{0.556012in}}%
\pgfpathlineto{\pgfqpoint{3.810130in}{0.557090in}}%
\pgfpathlineto{\pgfqpoint{3.810130in}{0.561402in}}%
\pgfpathlineto{\pgfqpoint{3.811428in}{0.560324in}}%
\pgfpathlineto{\pgfqpoint{3.812725in}{0.560324in}}%
\pgfpathlineto{\pgfqpoint{3.812725in}{0.556012in}}%
\pgfpathlineto{\pgfqpoint{3.814022in}{0.560324in}}%
\pgfpathlineto{\pgfqpoint{3.816617in}{0.561402in}}%
\pgfpathlineto{\pgfqpoint{3.816617in}{0.556012in}}%
\pgfpathlineto{\pgfqpoint{3.817915in}{0.559246in}}%
\pgfpathlineto{\pgfqpoint{3.819212in}{0.559246in}}%
\pgfpathlineto{\pgfqpoint{3.819212in}{0.556012in}}%
\pgfpathlineto{\pgfqpoint{3.820510in}{0.557090in}}%
\pgfpathlineto{\pgfqpoint{3.821807in}{0.557090in}}%
\pgfpathlineto{\pgfqpoint{3.821807in}{0.554934in}}%
\pgfpathlineto{\pgfqpoint{3.823104in}{0.554934in}}%
\pgfpathlineto{\pgfqpoint{3.824402in}{0.554934in}}%
\pgfpathlineto{\pgfqpoint{3.824402in}{0.557090in}}%
\pgfpathlineto{\pgfqpoint{3.825699in}{0.556012in}}%
\pgfpathlineto{\pgfqpoint{3.829592in}{0.556012in}}%
\pgfpathlineto{\pgfqpoint{3.829592in}{0.558168in}}%
\pgfpathlineto{\pgfqpoint{3.830889in}{0.554934in}}%
\pgfpathlineto{\pgfqpoint{3.832186in}{0.554934in}}%
\pgfpathlineto{\pgfqpoint{3.832186in}{0.563558in}}%
\pgfpathlineto{\pgfqpoint{3.833484in}{0.558168in}}%
\pgfpathlineto{\pgfqpoint{3.834781in}{0.558168in}}%
\pgfpathlineto{\pgfqpoint{3.834781in}{0.553856in}}%
\pgfpathlineto{\pgfqpoint{3.836079in}{0.558168in}}%
\pgfpathlineto{\pgfqpoint{3.837376in}{0.558168in}}%
\pgfpathlineto{\pgfqpoint{3.837376in}{0.561402in}}%
\pgfpathlineto{\pgfqpoint{3.838674in}{0.559246in}}%
\pgfpathlineto{\pgfqpoint{3.839971in}{0.559246in}}%
\pgfpathlineto{\pgfqpoint{3.839971in}{0.556012in}}%
\pgfpathlineto{\pgfqpoint{3.841268in}{0.558168in}}%
\pgfpathlineto{\pgfqpoint{3.842566in}{0.558168in}}%
\pgfpathlineto{\pgfqpoint{3.842566in}{0.553856in}}%
\pgfpathlineto{\pgfqpoint{3.843863in}{0.561402in}}%
\pgfpathlineto{\pgfqpoint{3.845161in}{0.561402in}}%
\pgfpathlineto{\pgfqpoint{3.845161in}{0.558168in}}%
\pgfpathlineto{\pgfqpoint{3.846458in}{0.559246in}}%
\pgfpathlineto{\pgfqpoint{3.849053in}{0.558168in}}%
\pgfpathlineto{\pgfqpoint{3.849053in}{0.556012in}}%
\pgfpathlineto{\pgfqpoint{3.850350in}{0.557090in}}%
\pgfpathlineto{\pgfqpoint{3.852945in}{0.557090in}}%
\pgfpathlineto{\pgfqpoint{3.852945in}{0.561402in}}%
\pgfpathlineto{\pgfqpoint{3.854243in}{0.553856in}}%
\pgfpathlineto{\pgfqpoint{3.855540in}{0.553856in}}%
\pgfpathlineto{\pgfqpoint{3.855540in}{0.557090in}}%
\pgfpathlineto{\pgfqpoint{3.856837in}{0.557090in}}%
\pgfpathlineto{\pgfqpoint{3.858135in}{0.557090in}}%
\pgfpathlineto{\pgfqpoint{3.858135in}{0.561402in}}%
\pgfpathlineto{\pgfqpoint{3.859432in}{0.557090in}}%
\pgfpathlineto{\pgfqpoint{3.862027in}{0.557090in}}%
\pgfpathlineto{\pgfqpoint{3.863325in}{0.563558in}}%
\pgfpathlineto{\pgfqpoint{3.864622in}{0.563558in}}%
\pgfpathlineto{\pgfqpoint{3.864622in}{0.556012in}}%
\pgfpathlineto{\pgfqpoint{3.865919in}{0.560324in}}%
\pgfpathlineto{\pgfqpoint{3.867217in}{0.560324in}}%
\pgfpathlineto{\pgfqpoint{3.867217in}{0.556012in}}%
\pgfpathlineto{\pgfqpoint{3.868514in}{0.560324in}}%
\pgfpathlineto{\pgfqpoint{3.869812in}{0.560324in}}%
\pgfpathlineto{\pgfqpoint{3.869812in}{0.558168in}}%
\pgfpathlineto{\pgfqpoint{3.871109in}{0.558168in}}%
\pgfpathlineto{\pgfqpoint{3.873704in}{0.557090in}}%
\pgfpathlineto{\pgfqpoint{3.873704in}{0.556012in}}%
\pgfpathlineto{\pgfqpoint{3.875001in}{0.556012in}}%
\pgfpathlineto{\pgfqpoint{3.877596in}{0.554934in}}%
\pgfpathlineto{\pgfqpoint{3.877596in}{0.561402in}}%
\pgfpathlineto{\pgfqpoint{3.878894in}{0.554934in}}%
\pgfpathlineto{\pgfqpoint{3.880191in}{0.554934in}}%
\pgfpathlineto{\pgfqpoint{3.880191in}{0.560324in}}%
\pgfpathlineto{\pgfqpoint{3.881489in}{0.560324in}}%
\pgfpathlineto{\pgfqpoint{3.884083in}{0.559246in}}%
\pgfpathlineto{\pgfqpoint{3.884083in}{0.558168in}}%
\pgfpathlineto{\pgfqpoint{3.885381in}{0.559246in}}%
\pgfpathlineto{\pgfqpoint{3.886678in}{0.559246in}}%
\pgfpathlineto{\pgfqpoint{3.886678in}{0.554934in}}%
\pgfpathlineto{\pgfqpoint{3.887976in}{0.560324in}}%
\pgfpathlineto{\pgfqpoint{3.889273in}{0.560324in}}%
\pgfpathlineto{\pgfqpoint{3.889273in}{0.556012in}}%
\pgfpathlineto{\pgfqpoint{3.890570in}{0.557090in}}%
\pgfpathlineto{\pgfqpoint{3.891868in}{0.557090in}}%
\pgfpathlineto{\pgfqpoint{3.891868in}{0.552778in}}%
\pgfpathlineto{\pgfqpoint{3.893165in}{0.556012in}}%
\pgfpathlineto{\pgfqpoint{3.894463in}{0.556012in}}%
\pgfpathlineto{\pgfqpoint{3.895760in}{0.560324in}}%
\pgfpathlineto{\pgfqpoint{3.897058in}{0.560324in}}%
\pgfpathlineto{\pgfqpoint{3.897058in}{0.556012in}}%
\pgfpathlineto{\pgfqpoint{3.898355in}{0.556012in}}%
\pgfpathlineto{\pgfqpoint{3.900950in}{0.556012in}}%
\pgfpathlineto{\pgfqpoint{3.900950in}{0.558168in}}%
\pgfpathlineto{\pgfqpoint{3.902247in}{0.557090in}}%
\pgfpathlineto{\pgfqpoint{3.906140in}{0.558168in}}%
\pgfpathlineto{\pgfqpoint{3.906140in}{0.560324in}}%
\pgfpathlineto{\pgfqpoint{3.907437in}{0.558168in}}%
\pgfpathlineto{\pgfqpoint{3.908734in}{0.558168in}}%
\pgfpathlineto{\pgfqpoint{3.908734in}{0.553856in}}%
\pgfpathlineto{\pgfqpoint{3.910032in}{0.558168in}}%
\pgfpathlineto{\pgfqpoint{3.913924in}{0.557090in}}%
\pgfpathlineto{\pgfqpoint{3.913924in}{0.560324in}}%
\pgfpathlineto{\pgfqpoint{3.915222in}{0.560324in}}%
\pgfpathlineto{\pgfqpoint{3.917816in}{0.559246in}}%
\pgfpathlineto{\pgfqpoint{3.917816in}{0.554934in}}%
\pgfpathlineto{\pgfqpoint{3.919114in}{0.558168in}}%
\pgfpathlineto{\pgfqpoint{3.921709in}{0.557090in}}%
\pgfpathlineto{\pgfqpoint{3.921709in}{0.556012in}}%
\pgfpathlineto{\pgfqpoint{3.923006in}{0.557090in}}%
\pgfpathlineto{\pgfqpoint{3.924303in}{0.557090in}}%
\pgfpathlineto{\pgfqpoint{3.924303in}{0.552778in}}%
\pgfpathlineto{\pgfqpoint{3.925601in}{0.556012in}}%
\pgfpathlineto{\pgfqpoint{3.928196in}{0.557090in}}%
\pgfpathlineto{\pgfqpoint{3.928196in}{0.560324in}}%
\pgfpathlineto{\pgfqpoint{3.929493in}{0.556012in}}%
\pgfpathlineto{\pgfqpoint{3.932088in}{0.556012in}}%
\pgfpathlineto{\pgfqpoint{3.932088in}{0.553856in}}%
\pgfpathlineto{\pgfqpoint{3.933385in}{0.557090in}}%
\pgfpathlineto{\pgfqpoint{3.937278in}{0.556012in}}%
\pgfpathlineto{\pgfqpoint{3.938575in}{0.560324in}}%
\pgfpathlineto{\pgfqpoint{3.941170in}{0.560324in}}%
\pgfpathlineto{\pgfqpoint{3.941170in}{0.556012in}}%
\pgfpathlineto{\pgfqpoint{3.942467in}{0.558168in}}%
\pgfpathlineto{\pgfqpoint{3.943765in}{0.558168in}}%
\pgfpathlineto{\pgfqpoint{3.943765in}{0.556012in}}%
\pgfpathlineto{\pgfqpoint{3.945062in}{0.556012in}}%
\pgfpathlineto{\pgfqpoint{3.946360in}{0.556012in}}%
\pgfpathlineto{\pgfqpoint{3.946360in}{0.559246in}}%
\pgfpathlineto{\pgfqpoint{3.947657in}{0.557090in}}%
\pgfpathlineto{\pgfqpoint{3.948955in}{0.557090in}}%
\pgfpathlineto{\pgfqpoint{3.948955in}{0.554934in}}%
\pgfpathlineto{\pgfqpoint{3.950252in}{0.563558in}}%
\pgfpathlineto{\pgfqpoint{3.951549in}{0.563558in}}%
\pgfpathlineto{\pgfqpoint{3.951549in}{0.553856in}}%
\pgfpathlineto{\pgfqpoint{3.952847in}{0.553856in}}%
\pgfpathlineto{\pgfqpoint{3.954144in}{0.553856in}}%
\pgfpathlineto{\pgfqpoint{3.954144in}{0.560324in}}%
\pgfpathlineto{\pgfqpoint{3.955442in}{0.559246in}}%
\pgfpathlineto{\pgfqpoint{3.956739in}{0.559246in}}%
\pgfpathlineto{\pgfqpoint{3.956739in}{0.556012in}}%
\pgfpathlineto{\pgfqpoint{3.958036in}{0.562480in}}%
\pgfpathlineto{\pgfqpoint{3.959334in}{0.562480in}}%
\pgfpathlineto{\pgfqpoint{3.960631in}{0.557090in}}%
\pgfpathlineto{\pgfqpoint{3.963226in}{0.556012in}}%
\pgfpathlineto{\pgfqpoint{3.964524in}{0.560324in}}%
\pgfpathlineto{\pgfqpoint{3.965821in}{0.560324in}}%
\pgfpathlineto{\pgfqpoint{3.967118in}{0.553856in}}%
\pgfpathlineto{\pgfqpoint{3.969713in}{0.554934in}}%
\pgfpathlineto{\pgfqpoint{3.969713in}{0.561402in}}%
\pgfpathlineto{\pgfqpoint{3.971011in}{0.557090in}}%
\pgfpathlineto{\pgfqpoint{3.977498in}{0.556012in}}%
\pgfpathlineto{\pgfqpoint{3.977498in}{0.560324in}}%
\pgfpathlineto{\pgfqpoint{3.978795in}{0.554934in}}%
\pgfpathlineto{\pgfqpoint{3.981390in}{0.554934in}}%
\pgfpathlineto{\pgfqpoint{3.981390in}{0.560324in}}%
\pgfpathlineto{\pgfqpoint{3.982688in}{0.552778in}}%
\pgfpathlineto{\pgfqpoint{3.983985in}{0.552778in}}%
\pgfpathlineto{\pgfqpoint{3.983985in}{0.559246in}}%
\pgfpathlineto{\pgfqpoint{3.985282in}{0.554934in}}%
\pgfpathlineto{\pgfqpoint{3.986580in}{0.554934in}}%
\pgfpathlineto{\pgfqpoint{3.986580in}{0.560324in}}%
\pgfpathlineto{\pgfqpoint{3.987877in}{0.556012in}}%
\pgfpathlineto{\pgfqpoint{3.990472in}{0.556012in}}%
\pgfpathlineto{\pgfqpoint{3.990472in}{0.558168in}}%
\pgfpathlineto{\pgfqpoint{3.991769in}{0.557090in}}%
\pgfpathlineto{\pgfqpoint{3.996959in}{0.556012in}}%
\pgfpathlineto{\pgfqpoint{3.996959in}{0.554934in}}%
\pgfpathlineto{\pgfqpoint{3.998257in}{0.557090in}}%
\pgfpathlineto{\pgfqpoint{3.999554in}{0.557090in}}%
\pgfpathlineto{\pgfqpoint{3.999554in}{0.554934in}}%
\pgfpathlineto{\pgfqpoint{4.000851in}{0.556012in}}%
\pgfpathlineto{\pgfqpoint{4.003446in}{0.554934in}}%
\pgfpathlineto{\pgfqpoint{4.003446in}{0.559246in}}%
\pgfpathlineto{\pgfqpoint{4.004744in}{0.557090in}}%
\pgfpathlineto{\pgfqpoint{4.009933in}{0.557090in}}%
\pgfpathlineto{\pgfqpoint{4.009933in}{0.554934in}}%
\pgfpathlineto{\pgfqpoint{4.011231in}{0.558168in}}%
\pgfpathlineto{\pgfqpoint{4.013826in}{0.559246in}}%
\pgfpathlineto{\pgfqpoint{4.013826in}{0.554934in}}%
\pgfpathlineto{\pgfqpoint{4.015123in}{0.556012in}}%
\pgfpathlineto{\pgfqpoint{4.017718in}{0.557090in}}%
\pgfpathlineto{\pgfqpoint{4.017718in}{0.559246in}}%
\pgfpathlineto{\pgfqpoint{4.019015in}{0.556012in}}%
\pgfpathlineto{\pgfqpoint{4.022908in}{0.557090in}}%
\pgfpathlineto{\pgfqpoint{4.024205in}{0.560324in}}%
\pgfpathlineto{\pgfqpoint{4.025502in}{0.560324in}}%
\pgfpathlineto{\pgfqpoint{4.026800in}{0.554934in}}%
\pgfpathlineto{\pgfqpoint{4.028097in}{0.554934in}}%
\pgfpathlineto{\pgfqpoint{4.029395in}{0.559246in}}%
\pgfpathlineto{\pgfqpoint{4.034584in}{0.559246in}}%
\pgfpathlineto{\pgfqpoint{4.034584in}{0.553856in}}%
\pgfpathlineto{\pgfqpoint{4.035882in}{0.557090in}}%
\pgfpathlineto{\pgfqpoint{4.038477in}{0.557090in}}%
\pgfpathlineto{\pgfqpoint{4.039774in}{0.553856in}}%
\pgfpathlineto{\pgfqpoint{4.041072in}{0.553856in}}%
\pgfpathlineto{\pgfqpoint{4.041072in}{0.559246in}}%
\pgfpathlineto{\pgfqpoint{4.042369in}{0.554934in}}%
\pgfpathlineto{\pgfqpoint{4.044964in}{0.553856in}}%
\pgfpathlineto{\pgfqpoint{4.044964in}{0.559246in}}%
\pgfpathlineto{\pgfqpoint{4.046261in}{0.553856in}}%
\pgfpathlineto{\pgfqpoint{4.047559in}{0.553856in}}%
\pgfpathlineto{\pgfqpoint{4.047559in}{0.560324in}}%
\pgfpathlineto{\pgfqpoint{4.048856in}{0.556012in}}%
\pgfpathlineto{\pgfqpoint{4.050154in}{0.556012in}}%
\pgfpathlineto{\pgfqpoint{4.050154in}{0.563558in}}%
\pgfpathlineto{\pgfqpoint{4.051451in}{0.558168in}}%
\pgfpathlineto{\pgfqpoint{4.055343in}{0.559246in}}%
\pgfpathlineto{\pgfqpoint{4.055343in}{0.561402in}}%
\pgfpathlineto{\pgfqpoint{4.056641in}{0.558168in}}%
\pgfpathlineto{\pgfqpoint{4.057938in}{0.558168in}}%
\pgfpathlineto{\pgfqpoint{4.057938in}{0.553856in}}%
\pgfpathlineto{\pgfqpoint{4.059235in}{0.554934in}}%
\pgfpathlineto{\pgfqpoint{4.060533in}{0.554934in}}%
\pgfpathlineto{\pgfqpoint{4.060533in}{0.558168in}}%
\pgfpathlineto{\pgfqpoint{4.061830in}{0.553856in}}%
\pgfpathlineto{\pgfqpoint{4.064425in}{0.554934in}}%
\pgfpathlineto{\pgfqpoint{4.065723in}{0.557090in}}%
\pgfpathlineto{\pgfqpoint{4.067020in}{0.557090in}}%
\pgfpathlineto{\pgfqpoint{4.067020in}{0.554934in}}%
\pgfpathlineto{\pgfqpoint{4.068317in}{0.559246in}}%
\pgfpathlineto{\pgfqpoint{4.070912in}{0.559246in}}%
\pgfpathlineto{\pgfqpoint{4.072210in}{0.554934in}}%
\pgfpathlineto{\pgfqpoint{4.074805in}{0.554934in}}%
\pgfpathlineto{\pgfqpoint{4.074805in}{0.552778in}}%
\pgfpathlineto{\pgfqpoint{4.076102in}{0.557090in}}%
\pgfpathlineto{\pgfqpoint{4.077399in}{0.557090in}}%
\pgfpathlineto{\pgfqpoint{4.077399in}{0.552778in}}%
\pgfpathlineto{\pgfqpoint{4.078697in}{0.557090in}}%
\pgfpathlineto{\pgfqpoint{4.079994in}{0.557090in}}%
\pgfpathlineto{\pgfqpoint{4.079994in}{0.552778in}}%
\pgfpathlineto{\pgfqpoint{4.081292in}{0.554934in}}%
\pgfpathlineto{\pgfqpoint{4.083887in}{0.554934in}}%
\pgfpathlineto{\pgfqpoint{4.083887in}{0.558168in}}%
\pgfpathlineto{\pgfqpoint{4.085184in}{0.557090in}}%
\pgfpathlineto{\pgfqpoint{4.086481in}{0.557090in}}%
\pgfpathlineto{\pgfqpoint{4.086481in}{0.560324in}}%
\pgfpathlineto{\pgfqpoint{4.087779in}{0.558168in}}%
\pgfpathlineto{\pgfqpoint{4.090374in}{0.557090in}}%
\pgfpathlineto{\pgfqpoint{4.090374in}{0.556012in}}%
\pgfpathlineto{\pgfqpoint{4.091671in}{0.556012in}}%
\pgfpathlineto{\pgfqpoint{4.092968in}{0.556012in}}%
\pgfpathlineto{\pgfqpoint{4.094266in}{0.561402in}}%
\pgfpathlineto{\pgfqpoint{4.095563in}{0.561402in}}%
\pgfpathlineto{\pgfqpoint{4.095563in}{0.554934in}}%
\pgfpathlineto{\pgfqpoint{4.096861in}{0.558168in}}%
\pgfpathlineto{\pgfqpoint{4.098158in}{0.558168in}}%
\pgfpathlineto{\pgfqpoint{4.098158in}{0.553856in}}%
\pgfpathlineto{\pgfqpoint{4.099456in}{0.556012in}}%
\pgfpathlineto{\pgfqpoint{4.105943in}{0.556012in}}%
\pgfpathlineto{\pgfqpoint{4.105943in}{0.558168in}}%
\pgfpathlineto{\pgfqpoint{4.107240in}{0.553856in}}%
\pgfpathlineto{\pgfqpoint{4.108538in}{0.553856in}}%
\pgfpathlineto{\pgfqpoint{4.108538in}{0.558168in}}%
\pgfpathlineto{\pgfqpoint{4.109835in}{0.557090in}}%
\pgfpathlineto{\pgfqpoint{4.111132in}{0.557090in}}%
\pgfpathlineto{\pgfqpoint{4.111132in}{0.560324in}}%
\pgfpathlineto{\pgfqpoint{4.112430in}{0.554934in}}%
\pgfpathlineto{\pgfqpoint{4.115025in}{0.556012in}}%
\pgfpathlineto{\pgfqpoint{4.116322in}{0.560324in}}%
\pgfpathlineto{\pgfqpoint{4.117620in}{0.560324in}}%
\pgfpathlineto{\pgfqpoint{4.117620in}{0.557090in}}%
\pgfpathlineto{\pgfqpoint{4.118917in}{0.560324in}}%
\pgfpathlineto{\pgfqpoint{4.121512in}{0.559246in}}%
\pgfpathlineto{\pgfqpoint{4.121512in}{0.552778in}}%
\pgfpathlineto{\pgfqpoint{4.122809in}{0.557090in}}%
\pgfpathlineto{\pgfqpoint{4.125404in}{0.558168in}}%
\pgfpathlineto{\pgfqpoint{4.125404in}{0.559246in}}%
\pgfpathlineto{\pgfqpoint{4.126701in}{0.556012in}}%
\pgfpathlineto{\pgfqpoint{4.129296in}{0.554934in}}%
\pgfpathlineto{\pgfqpoint{4.129296in}{0.553856in}}%
\pgfpathlineto{\pgfqpoint{4.130594in}{0.556012in}}%
\pgfpathlineto{\pgfqpoint{4.134486in}{0.556012in}}%
\pgfpathlineto{\pgfqpoint{4.134486in}{0.560324in}}%
\pgfpathlineto{\pgfqpoint{4.135783in}{0.554934in}}%
\pgfpathlineto{\pgfqpoint{4.137081in}{0.554934in}}%
\pgfpathlineto{\pgfqpoint{4.137081in}{0.558168in}}%
\pgfpathlineto{\pgfqpoint{4.138378in}{0.556012in}}%
\pgfpathlineto{\pgfqpoint{4.140973in}{0.557090in}}%
\pgfpathlineto{\pgfqpoint{4.142271in}{0.552778in}}%
\pgfpathlineto{\pgfqpoint{4.143568in}{0.552778in}}%
\pgfpathlineto{\pgfqpoint{4.143568in}{0.554934in}}%
\pgfpathlineto{\pgfqpoint{4.144865in}{0.554934in}}%
\pgfpathlineto{\pgfqpoint{4.152650in}{0.556012in}}%
\pgfpathlineto{\pgfqpoint{4.152650in}{0.558168in}}%
\pgfpathlineto{\pgfqpoint{4.153947in}{0.553856in}}%
\pgfpathlineto{\pgfqpoint{4.155245in}{0.553856in}}%
\pgfpathlineto{\pgfqpoint{4.155245in}{0.556012in}}%
\pgfpathlineto{\pgfqpoint{4.156542in}{0.556012in}}%
\pgfpathlineto{\pgfqpoint{4.157840in}{0.556012in}}%
\pgfpathlineto{\pgfqpoint{4.157840in}{0.558168in}}%
\pgfpathlineto{\pgfqpoint{4.159137in}{0.554934in}}%
\pgfpathlineto{\pgfqpoint{4.160434in}{0.554934in}}%
\pgfpathlineto{\pgfqpoint{4.160434in}{0.559246in}}%
\pgfpathlineto{\pgfqpoint{4.161732in}{0.553856in}}%
\pgfpathlineto{\pgfqpoint{4.163029in}{0.553856in}}%
\pgfpathlineto{\pgfqpoint{4.163029in}{0.557090in}}%
\pgfpathlineto{\pgfqpoint{4.164327in}{0.554934in}}%
\pgfpathlineto{\pgfqpoint{4.165624in}{0.554934in}}%
\pgfpathlineto{\pgfqpoint{4.165624in}{0.559246in}}%
\pgfpathlineto{\pgfqpoint{4.166922in}{0.554934in}}%
\pgfpathlineto{\pgfqpoint{4.169516in}{0.556012in}}%
\pgfpathlineto{\pgfqpoint{4.169516in}{0.558168in}}%
\pgfpathlineto{\pgfqpoint{4.170814in}{0.556012in}}%
\pgfpathlineto{\pgfqpoint{4.172111in}{0.556012in}}%
\pgfpathlineto{\pgfqpoint{4.172111in}{0.553856in}}%
\pgfpathlineto{\pgfqpoint{4.173409in}{0.554934in}}%
\pgfpathlineto{\pgfqpoint{4.174706in}{0.554934in}}%
\pgfpathlineto{\pgfqpoint{4.176004in}{0.559246in}}%
\pgfpathlineto{\pgfqpoint{4.178598in}{0.558168in}}%
\pgfpathlineto{\pgfqpoint{4.178598in}{0.554934in}}%
\pgfpathlineto{\pgfqpoint{4.179896in}{0.557090in}}%
\pgfpathlineto{\pgfqpoint{4.181193in}{0.557090in}}%
\pgfpathlineto{\pgfqpoint{4.181193in}{0.554934in}}%
\pgfpathlineto{\pgfqpoint{4.182491in}{0.556012in}}%
\pgfpathlineto{\pgfqpoint{4.188978in}{0.556012in}}%
\pgfpathlineto{\pgfqpoint{4.188978in}{0.558168in}}%
\pgfpathlineto{\pgfqpoint{4.190275in}{0.558168in}}%
\pgfpathlineto{\pgfqpoint{4.191573in}{0.558168in}}%
\pgfpathlineto{\pgfqpoint{4.191573in}{0.554934in}}%
\pgfpathlineto{\pgfqpoint{4.192870in}{0.554934in}}%
\pgfpathlineto{\pgfqpoint{4.195465in}{0.554934in}}%
\pgfpathlineto{\pgfqpoint{4.195465in}{0.560324in}}%
\pgfpathlineto{\pgfqpoint{4.196762in}{0.552778in}}%
\pgfpathlineto{\pgfqpoint{4.198060in}{0.552778in}}%
\pgfpathlineto{\pgfqpoint{4.198060in}{0.557090in}}%
\pgfpathlineto{\pgfqpoint{4.199357in}{0.556012in}}%
\pgfpathlineto{\pgfqpoint{4.200655in}{0.556012in}}%
\pgfpathlineto{\pgfqpoint{4.200655in}{0.552778in}}%
\pgfpathlineto{\pgfqpoint{4.201952in}{0.558168in}}%
\pgfpathlineto{\pgfqpoint{4.203249in}{0.558168in}}%
\pgfpathlineto{\pgfqpoint{4.203249in}{0.556012in}}%
\pgfpathlineto{\pgfqpoint{4.204547in}{0.561402in}}%
\pgfpathlineto{\pgfqpoint{4.205844in}{0.561402in}}%
\pgfpathlineto{\pgfqpoint{4.207142in}{0.556012in}}%
\pgfpathlineto{\pgfqpoint{4.214926in}{0.554934in}}%
\pgfpathlineto{\pgfqpoint{4.214926in}{0.558168in}}%
\pgfpathlineto{\pgfqpoint{4.216224in}{0.552778in}}%
\pgfpathlineto{\pgfqpoint{4.217521in}{0.552778in}}%
\pgfpathlineto{\pgfqpoint{4.217521in}{0.560324in}}%
\pgfpathlineto{\pgfqpoint{4.218819in}{0.554934in}}%
\pgfpathlineto{\pgfqpoint{4.222711in}{0.556012in}}%
\pgfpathlineto{\pgfqpoint{4.222711in}{0.559246in}}%
\pgfpathlineto{\pgfqpoint{4.224008in}{0.557090in}}%
\pgfpathlineto{\pgfqpoint{4.225306in}{0.557090in}}%
\pgfpathlineto{\pgfqpoint{4.225306in}{0.554934in}}%
\pgfpathlineto{\pgfqpoint{4.226603in}{0.557090in}}%
\pgfpathlineto{\pgfqpoint{4.227901in}{0.557090in}}%
\pgfpathlineto{\pgfqpoint{4.227901in}{0.561402in}}%
\pgfpathlineto{\pgfqpoint{4.229198in}{0.559246in}}%
\pgfpathlineto{\pgfqpoint{4.230495in}{0.559246in}}%
\pgfpathlineto{\pgfqpoint{4.230495in}{0.553856in}}%
\pgfpathlineto{\pgfqpoint{4.231793in}{0.557090in}}%
\pgfpathlineto{\pgfqpoint{4.236982in}{0.557090in}}%
\pgfpathlineto{\pgfqpoint{4.236982in}{0.561402in}}%
\pgfpathlineto{\pgfqpoint{4.238280in}{0.557090in}}%
\pgfpathlineto{\pgfqpoint{4.239577in}{0.557090in}}%
\pgfpathlineto{\pgfqpoint{4.239577in}{0.553856in}}%
\pgfpathlineto{\pgfqpoint{4.240875in}{0.557090in}}%
\pgfpathlineto{\pgfqpoint{4.242172in}{0.557090in}}%
\pgfpathlineto{\pgfqpoint{4.242172in}{0.552778in}}%
\pgfpathlineto{\pgfqpoint{4.243470in}{0.554934in}}%
\pgfpathlineto{\pgfqpoint{4.251254in}{0.554934in}}%
\pgfpathlineto{\pgfqpoint{4.251254in}{0.558168in}}%
\pgfpathlineto{\pgfqpoint{4.252552in}{0.556012in}}%
\pgfpathlineto{\pgfqpoint{4.253849in}{0.556012in}}%
\pgfpathlineto{\pgfqpoint{4.253849in}{0.553856in}}%
\pgfpathlineto{\pgfqpoint{4.255146in}{0.557090in}}%
\pgfpathlineto{\pgfqpoint{4.259039in}{0.556012in}}%
\pgfpathlineto{\pgfqpoint{4.259039in}{0.553856in}}%
\pgfpathlineto{\pgfqpoint{4.260336in}{0.557090in}}%
\pgfpathlineto{\pgfqpoint{4.264228in}{0.556012in}}%
\pgfpathlineto{\pgfqpoint{4.264228in}{0.553856in}}%
\pgfpathlineto{\pgfqpoint{4.265526in}{0.558168in}}%
\pgfpathlineto{\pgfqpoint{4.266823in}{0.558168in}}%
\pgfpathlineto{\pgfqpoint{4.268121in}{0.552778in}}%
\pgfpathlineto{\pgfqpoint{4.269418in}{0.552778in}}%
\pgfpathlineto{\pgfqpoint{4.269418in}{0.557090in}}%
\pgfpathlineto{\pgfqpoint{4.270715in}{0.554934in}}%
\pgfpathlineto{\pgfqpoint{4.272013in}{0.554934in}}%
\pgfpathlineto{\pgfqpoint{4.272013in}{0.557090in}}%
\pgfpathlineto{\pgfqpoint{4.273310in}{0.554934in}}%
\pgfpathlineto{\pgfqpoint{4.274608in}{0.554934in}}%
\pgfpathlineto{\pgfqpoint{4.274608in}{0.557090in}}%
\pgfpathlineto{\pgfqpoint{4.275905in}{0.557090in}}%
\pgfpathlineto{\pgfqpoint{4.279797in}{0.556012in}}%
\pgfpathlineto{\pgfqpoint{4.279797in}{0.554934in}}%
\pgfpathlineto{\pgfqpoint{4.281095in}{0.556012in}}%
\pgfpathlineto{\pgfqpoint{4.282392in}{0.556012in}}%
\pgfpathlineto{\pgfqpoint{4.282392in}{0.558168in}}%
\pgfpathlineto{\pgfqpoint{4.283690in}{0.554934in}}%
\pgfpathlineto{\pgfqpoint{4.284987in}{0.554934in}}%
\pgfpathlineto{\pgfqpoint{4.284987in}{0.552778in}}%
\pgfpathlineto{\pgfqpoint{4.286285in}{0.554934in}}%
\pgfpathlineto{\pgfqpoint{4.288879in}{0.554934in}}%
\pgfpathlineto{\pgfqpoint{4.290177in}{0.558168in}}%
\pgfpathlineto{\pgfqpoint{4.291474in}{0.558168in}}%
\pgfpathlineto{\pgfqpoint{4.291474in}{0.556012in}}%
\pgfpathlineto{\pgfqpoint{4.292772in}{0.556012in}}%
\pgfpathlineto{\pgfqpoint{4.299259in}{0.556012in}}%
\pgfpathlineto{\pgfqpoint{4.299259in}{0.553856in}}%
\pgfpathlineto{\pgfqpoint{4.300556in}{0.557090in}}%
\pgfpathlineto{\pgfqpoint{4.301854in}{0.557090in}}%
\pgfpathlineto{\pgfqpoint{4.301854in}{0.560324in}}%
\pgfpathlineto{\pgfqpoint{4.303151in}{0.556012in}}%
\pgfpathlineto{\pgfqpoint{4.304448in}{0.556012in}}%
\pgfpathlineto{\pgfqpoint{4.304448in}{0.552778in}}%
\pgfpathlineto{\pgfqpoint{4.305746in}{0.553856in}}%
\pgfpathlineto{\pgfqpoint{4.309638in}{0.552778in}}%
\pgfpathlineto{\pgfqpoint{4.309638in}{0.556012in}}%
\pgfpathlineto{\pgfqpoint{4.310936in}{0.552778in}}%
\pgfpathlineto{\pgfqpoint{4.312233in}{0.552778in}}%
\pgfpathlineto{\pgfqpoint{4.312233in}{0.554934in}}%
\pgfpathlineto{\pgfqpoint{4.313530in}{0.552778in}}%
\pgfpathlineto{\pgfqpoint{4.316125in}{0.553856in}}%
\pgfpathlineto{\pgfqpoint{4.316125in}{0.556012in}}%
\pgfpathlineto{\pgfqpoint{4.317423in}{0.554934in}}%
\pgfpathlineto{\pgfqpoint{4.321315in}{0.553856in}}%
\pgfpathlineto{\pgfqpoint{4.321315in}{0.558168in}}%
\pgfpathlineto{\pgfqpoint{4.322612in}{0.554934in}}%
\pgfpathlineto{\pgfqpoint{4.329100in}{0.554934in}}%
\pgfpathlineto{\pgfqpoint{4.329100in}{0.552778in}}%
\pgfpathlineto{\pgfqpoint{4.330397in}{0.553856in}}%
\pgfpathlineto{\pgfqpoint{4.331694in}{0.553856in}}%
\pgfpathlineto{\pgfqpoint{4.331694in}{0.558168in}}%
\pgfpathlineto{\pgfqpoint{4.332992in}{0.554934in}}%
\pgfpathlineto{\pgfqpoint{4.336884in}{0.553856in}}%
\pgfpathlineto{\pgfqpoint{4.336884in}{0.552778in}}%
\pgfpathlineto{\pgfqpoint{4.338181in}{0.558168in}}%
\pgfpathlineto{\pgfqpoint{4.339479in}{0.558168in}}%
\pgfpathlineto{\pgfqpoint{4.339479in}{0.556012in}}%
\pgfpathlineto{\pgfqpoint{4.340776in}{0.558168in}}%
\pgfpathlineto{\pgfqpoint{4.343371in}{0.557090in}}%
\pgfpathlineto{\pgfqpoint{4.343371in}{0.554934in}}%
\pgfpathlineto{\pgfqpoint{4.344669in}{0.557090in}}%
\pgfpathlineto{\pgfqpoint{4.345966in}{0.557090in}}%
\pgfpathlineto{\pgfqpoint{4.345966in}{0.554934in}}%
\pgfpathlineto{\pgfqpoint{4.347263in}{0.556012in}}%
\pgfpathlineto{\pgfqpoint{4.349858in}{0.554934in}}%
\pgfpathlineto{\pgfqpoint{4.349858in}{0.553856in}}%
\pgfpathlineto{\pgfqpoint{4.351156in}{0.557090in}}%
\pgfpathlineto{\pgfqpoint{4.352453in}{0.557090in}}%
\pgfpathlineto{\pgfqpoint{4.352453in}{0.560324in}}%
\pgfpathlineto{\pgfqpoint{4.353751in}{0.559246in}}%
\pgfpathlineto{\pgfqpoint{4.357643in}{0.560324in}}%
\pgfpathlineto{\pgfqpoint{4.357643in}{0.557090in}}%
\pgfpathlineto{\pgfqpoint{4.358940in}{0.557090in}}%
\pgfpathlineto{\pgfqpoint{4.360238in}{0.557090in}}%
\pgfpathlineto{\pgfqpoint{4.360238in}{0.553856in}}%
\pgfpathlineto{\pgfqpoint{4.361535in}{0.557090in}}%
\pgfpathlineto{\pgfqpoint{4.362833in}{0.557090in}}%
\pgfpathlineto{\pgfqpoint{4.362833in}{0.553856in}}%
\pgfpathlineto{\pgfqpoint{4.364130in}{0.556012in}}%
\pgfpathlineto{\pgfqpoint{4.366725in}{0.557090in}}%
\pgfpathlineto{\pgfqpoint{4.366725in}{0.558168in}}%
\pgfpathlineto{\pgfqpoint{4.368022in}{0.552778in}}%
\pgfpathlineto{\pgfqpoint{4.369320in}{0.552778in}}%
\pgfpathlineto{\pgfqpoint{4.369320in}{0.559246in}}%
\pgfpathlineto{\pgfqpoint{4.370617in}{0.554934in}}%
\pgfpathlineto{\pgfqpoint{4.383591in}{0.554934in}}%
\pgfpathlineto{\pgfqpoint{4.383591in}{0.558168in}}%
\pgfpathlineto{\pgfqpoint{4.384889in}{0.553856in}}%
\pgfpathlineto{\pgfqpoint{4.387484in}{0.554934in}}%
\pgfpathlineto{\pgfqpoint{4.387484in}{0.557090in}}%
\pgfpathlineto{\pgfqpoint{4.388781in}{0.556012in}}%
\pgfpathlineto{\pgfqpoint{4.392673in}{0.557090in}}%
\pgfpathlineto{\pgfqpoint{4.392673in}{0.560324in}}%
\pgfpathlineto{\pgfqpoint{4.393971in}{0.556012in}}%
\pgfpathlineto{\pgfqpoint{4.395268in}{0.556012in}}%
\pgfpathlineto{\pgfqpoint{4.395268in}{0.553856in}}%
\pgfpathlineto{\pgfqpoint{4.396566in}{0.554934in}}%
\pgfpathlineto{\pgfqpoint{4.400458in}{0.554934in}}%
\pgfpathlineto{\pgfqpoint{4.400458in}{0.552778in}}%
\pgfpathlineto{\pgfqpoint{4.401755in}{0.558168in}}%
\pgfpathlineto{\pgfqpoint{4.403053in}{0.558168in}}%
\pgfpathlineto{\pgfqpoint{4.403053in}{0.560324in}}%
\pgfpathlineto{\pgfqpoint{4.404350in}{0.554934in}}%
\pgfpathlineto{\pgfqpoint{4.410837in}{0.556012in}}%
\pgfpathlineto{\pgfqpoint{4.410837in}{0.558168in}}%
\pgfpathlineto{\pgfqpoint{4.412135in}{0.553856in}}%
\pgfpathlineto{\pgfqpoint{4.414729in}{0.554934in}}%
\pgfpathlineto{\pgfqpoint{4.416027in}{0.558168in}}%
\pgfpathlineto{\pgfqpoint{4.417324in}{0.558168in}}%
\pgfpathlineto{\pgfqpoint{4.417324in}{0.560324in}}%
\pgfpathlineto{\pgfqpoint{4.418622in}{0.556012in}}%
\pgfpathlineto{\pgfqpoint{4.421217in}{0.557090in}}%
\pgfpathlineto{\pgfqpoint{4.421217in}{0.553856in}}%
\pgfpathlineto{\pgfqpoint{4.422514in}{0.558168in}}%
\pgfpathlineto{\pgfqpoint{4.423811in}{0.558168in}}%
\pgfpathlineto{\pgfqpoint{4.423811in}{0.554934in}}%
\pgfpathlineto{\pgfqpoint{4.425109in}{0.557090in}}%
\pgfpathlineto{\pgfqpoint{4.426406in}{0.557090in}}%
\pgfpathlineto{\pgfqpoint{4.426406in}{0.554934in}}%
\pgfpathlineto{\pgfqpoint{4.427704in}{0.557090in}}%
\pgfpathlineto{\pgfqpoint{4.434191in}{0.557090in}}%
\pgfpathlineto{\pgfqpoint{4.434191in}{0.554934in}}%
\pgfpathlineto{\pgfqpoint{4.435488in}{0.554934in}}%
\pgfpathlineto{\pgfqpoint{4.438083in}{0.554934in}}%
\pgfpathlineto{\pgfqpoint{4.438083in}{0.558168in}}%
\pgfpathlineto{\pgfqpoint{4.439380in}{0.556012in}}%
\pgfpathlineto{\pgfqpoint{4.440678in}{0.556012in}}%
\pgfpathlineto{\pgfqpoint{4.440678in}{0.558168in}}%
\pgfpathlineto{\pgfqpoint{4.441975in}{0.553856in}}%
\pgfpathlineto{\pgfqpoint{4.443273in}{0.553856in}}%
\pgfpathlineto{\pgfqpoint{4.443273in}{0.557090in}}%
\pgfpathlineto{\pgfqpoint{4.444570in}{0.557090in}}%
\pgfpathlineto{\pgfqpoint{4.445868in}{0.557090in}}%
\pgfpathlineto{\pgfqpoint{4.445868in}{0.554934in}}%
\pgfpathlineto{\pgfqpoint{4.447165in}{0.558168in}}%
\pgfpathlineto{\pgfqpoint{4.449760in}{0.559246in}}%
\pgfpathlineto{\pgfqpoint{4.449760in}{0.553856in}}%
\pgfpathlineto{\pgfqpoint{4.451057in}{0.556012in}}%
\pgfpathlineto{\pgfqpoint{4.452355in}{0.556012in}}%
\pgfpathlineto{\pgfqpoint{4.452355in}{0.558168in}}%
\pgfpathlineto{\pgfqpoint{4.453652in}{0.557090in}}%
\pgfpathlineto{\pgfqpoint{4.456247in}{0.556012in}}%
\pgfpathlineto{\pgfqpoint{4.456247in}{0.559246in}}%
\pgfpathlineto{\pgfqpoint{4.457544in}{0.557090in}}%
\pgfpathlineto{\pgfqpoint{4.460139in}{0.557090in}}%
\pgfpathlineto{\pgfqpoint{4.460139in}{0.553856in}}%
\pgfpathlineto{\pgfqpoint{4.461437in}{0.554934in}}%
\pgfpathlineto{\pgfqpoint{4.466626in}{0.553856in}}%
\pgfpathlineto{\pgfqpoint{4.466626in}{0.557090in}}%
\pgfpathlineto{\pgfqpoint{4.467924in}{0.554934in}}%
\pgfpathlineto{\pgfqpoint{4.474411in}{0.554934in}}%
\pgfpathlineto{\pgfqpoint{4.474411in}{0.557090in}}%
\pgfpathlineto{\pgfqpoint{4.475708in}{0.554934in}}%
\pgfpathlineto{\pgfqpoint{4.479601in}{0.553856in}}%
\pgfpathlineto{\pgfqpoint{4.479601in}{0.558168in}}%
\pgfpathlineto{\pgfqpoint{4.480898in}{0.554934in}}%
\pgfpathlineto{\pgfqpoint{4.484790in}{0.556012in}}%
\pgfpathlineto{\pgfqpoint{4.484790in}{0.557090in}}%
\pgfpathlineto{\pgfqpoint{4.486088in}{0.557090in}}%
\pgfpathlineto{\pgfqpoint{4.487385in}{0.557090in}}%
\pgfpathlineto{\pgfqpoint{4.487385in}{0.553856in}}%
\pgfpathlineto{\pgfqpoint{4.488683in}{0.556012in}}%
\pgfpathlineto{\pgfqpoint{4.489980in}{0.556012in}}%
\pgfpathlineto{\pgfqpoint{4.489980in}{0.560324in}}%
\pgfpathlineto{\pgfqpoint{4.491277in}{0.556012in}}%
\pgfpathlineto{\pgfqpoint{4.492575in}{0.556012in}}%
\pgfpathlineto{\pgfqpoint{4.493872in}{0.561402in}}%
\pgfpathlineto{\pgfqpoint{4.495170in}{0.561402in}}%
\pgfpathlineto{\pgfqpoint{4.496467in}{0.553856in}}%
\pgfpathlineto{\pgfqpoint{4.497765in}{0.553856in}}%
\pgfpathlineto{\pgfqpoint{4.497765in}{0.557090in}}%
\pgfpathlineto{\pgfqpoint{4.499062in}{0.554934in}}%
\pgfpathlineto{\pgfqpoint{4.500359in}{0.554934in}}%
\pgfpathlineto{\pgfqpoint{4.500359in}{0.559246in}}%
\pgfpathlineto{\pgfqpoint{4.501657in}{0.557090in}}%
\pgfpathlineto{\pgfqpoint{4.502954in}{0.557090in}}%
\pgfpathlineto{\pgfqpoint{4.502954in}{0.554934in}}%
\pgfpathlineto{\pgfqpoint{4.504252in}{0.556012in}}%
\pgfpathlineto{\pgfqpoint{4.505549in}{0.556012in}}%
\pgfpathlineto{\pgfqpoint{4.505549in}{0.553856in}}%
\pgfpathlineto{\pgfqpoint{4.506846in}{0.558168in}}%
\pgfpathlineto{\pgfqpoint{4.508144in}{0.558168in}}%
\pgfpathlineto{\pgfqpoint{4.508144in}{0.556012in}}%
\pgfpathlineto{\pgfqpoint{4.509441in}{0.557090in}}%
\pgfpathlineto{\pgfqpoint{4.512036in}{0.558168in}}%
\pgfpathlineto{\pgfqpoint{4.512036in}{0.559246in}}%
\pgfpathlineto{\pgfqpoint{4.513334in}{0.553856in}}%
\pgfpathlineto{\pgfqpoint{4.514631in}{0.553856in}}%
\pgfpathlineto{\pgfqpoint{4.514631in}{0.556012in}}%
\pgfpathlineto{\pgfqpoint{4.515928in}{0.554934in}}%
\pgfpathlineto{\pgfqpoint{4.518523in}{0.553856in}}%
\pgfpathlineto{\pgfqpoint{4.518523in}{0.558168in}}%
\pgfpathlineto{\pgfqpoint{4.519821in}{0.557090in}}%
\pgfpathlineto{\pgfqpoint{4.521118in}{0.557090in}}%
\pgfpathlineto{\pgfqpoint{4.521118in}{0.554934in}}%
\pgfpathlineto{\pgfqpoint{4.522416in}{0.556012in}}%
\pgfpathlineto{\pgfqpoint{4.525010in}{0.556012in}}%
\pgfpathlineto{\pgfqpoint{4.525010in}{0.553856in}}%
\pgfpathlineto{\pgfqpoint{4.526308in}{0.553856in}}%
\pgfpathlineto{\pgfqpoint{4.528903in}{0.554934in}}%
\pgfpathlineto{\pgfqpoint{4.530200in}{0.558168in}}%
\pgfpathlineto{\pgfqpoint{4.531498in}{0.558168in}}%
\pgfpathlineto{\pgfqpoint{4.531498in}{0.554934in}}%
\pgfpathlineto{\pgfqpoint{4.532795in}{0.556012in}}%
\pgfpathlineto{\pgfqpoint{4.535390in}{0.554934in}}%
\pgfpathlineto{\pgfqpoint{4.535390in}{0.553856in}}%
\pgfpathlineto{\pgfqpoint{4.536687in}{0.556012in}}%
\pgfpathlineto{\pgfqpoint{4.537985in}{0.556012in}}%
\pgfpathlineto{\pgfqpoint{4.537985in}{0.552778in}}%
\pgfpathlineto{\pgfqpoint{4.539282in}{0.553856in}}%
\pgfpathlineto{\pgfqpoint{4.540579in}{0.553856in}}%
\pgfpathlineto{\pgfqpoint{4.540579in}{0.557090in}}%
\pgfpathlineto{\pgfqpoint{4.541877in}{0.556012in}}%
\pgfpathlineto{\pgfqpoint{4.545769in}{0.557090in}}%
\pgfpathlineto{\pgfqpoint{4.545769in}{0.553856in}}%
\pgfpathlineto{\pgfqpoint{4.547067in}{0.553856in}}%
\pgfpathlineto{\pgfqpoint{4.550959in}{0.553856in}}%
\pgfpathlineto{\pgfqpoint{4.550959in}{0.558168in}}%
\pgfpathlineto{\pgfqpoint{4.552256in}{0.554934in}}%
\pgfpathlineto{\pgfqpoint{4.553554in}{0.554934in}}%
\pgfpathlineto{\pgfqpoint{4.553554in}{0.558168in}}%
\pgfpathlineto{\pgfqpoint{4.554851in}{0.556012in}}%
\pgfpathlineto{\pgfqpoint{4.558743in}{0.557090in}}%
\pgfpathlineto{\pgfqpoint{4.558743in}{0.561402in}}%
\pgfpathlineto{\pgfqpoint{4.560041in}{0.556012in}}%
\pgfpathlineto{\pgfqpoint{4.561338in}{0.556012in}}%
\pgfpathlineto{\pgfqpoint{4.561338in}{0.560324in}}%
\pgfpathlineto{\pgfqpoint{4.562636in}{0.556012in}}%
\pgfpathlineto{\pgfqpoint{4.565231in}{0.554934in}}%
\pgfpathlineto{\pgfqpoint{4.565231in}{0.553856in}}%
\pgfpathlineto{\pgfqpoint{4.566528in}{0.557090in}}%
\pgfpathlineto{\pgfqpoint{4.569123in}{0.558168in}}%
\pgfpathlineto{\pgfqpoint{4.569123in}{0.552778in}}%
\pgfpathlineto{\pgfqpoint{4.570420in}{0.552778in}}%
\pgfpathlineto{\pgfqpoint{4.571718in}{0.552778in}}%
\pgfpathlineto{\pgfqpoint{4.571718in}{0.559246in}}%
\pgfpathlineto{\pgfqpoint{4.573015in}{0.558168in}}%
\pgfpathlineto{\pgfqpoint{4.574313in}{0.558168in}}%
\pgfpathlineto{\pgfqpoint{4.574313in}{0.561402in}}%
\pgfpathlineto{\pgfqpoint{4.575610in}{0.556012in}}%
\pgfpathlineto{\pgfqpoint{4.578205in}{0.554934in}}%
\pgfpathlineto{\pgfqpoint{4.578205in}{0.558168in}}%
\pgfpathlineto{\pgfqpoint{4.579502in}{0.554934in}}%
\pgfpathlineto{\pgfqpoint{4.582097in}{0.554934in}}%
\pgfpathlineto{\pgfqpoint{4.582097in}{0.562480in}}%
\pgfpathlineto{\pgfqpoint{4.583394in}{0.558168in}}%
\pgfpathlineto{\pgfqpoint{4.584692in}{0.558168in}}%
\pgfpathlineto{\pgfqpoint{4.584692in}{0.552778in}}%
\pgfpathlineto{\pgfqpoint{4.585989in}{0.554934in}}%
\pgfpathlineto{\pgfqpoint{4.598964in}{0.556012in}}%
\pgfpathlineto{\pgfqpoint{4.598964in}{0.557090in}}%
\pgfpathlineto{\pgfqpoint{4.600261in}{0.556012in}}%
\pgfpathlineto{\pgfqpoint{4.601558in}{0.556012in}}%
\pgfpathlineto{\pgfqpoint{4.601558in}{0.553856in}}%
\pgfpathlineto{\pgfqpoint{4.602856in}{0.557090in}}%
\pgfpathlineto{\pgfqpoint{4.605451in}{0.556012in}}%
\pgfpathlineto{\pgfqpoint{4.605451in}{0.559246in}}%
\pgfpathlineto{\pgfqpoint{4.606748in}{0.553856in}}%
\pgfpathlineto{\pgfqpoint{4.608046in}{0.553856in}}%
\pgfpathlineto{\pgfqpoint{4.608046in}{0.560324in}}%
\pgfpathlineto{\pgfqpoint{4.609343in}{0.553856in}}%
\pgfpathlineto{\pgfqpoint{4.613235in}{0.554934in}}%
\pgfpathlineto{\pgfqpoint{4.613235in}{0.556012in}}%
\pgfpathlineto{\pgfqpoint{4.614533in}{0.554934in}}%
\pgfpathlineto{\pgfqpoint{4.619722in}{0.553856in}}%
\pgfpathlineto{\pgfqpoint{4.619722in}{0.557090in}}%
\pgfpathlineto{\pgfqpoint{4.621020in}{0.557090in}}%
\pgfpathlineto{\pgfqpoint{4.623615in}{0.558168in}}%
\pgfpathlineto{\pgfqpoint{4.624912in}{0.553856in}}%
\pgfpathlineto{\pgfqpoint{4.626209in}{0.553856in}}%
\pgfpathlineto{\pgfqpoint{4.626209in}{0.557090in}}%
\pgfpathlineto{\pgfqpoint{4.627507in}{0.554934in}}%
\pgfpathlineto{\pgfqpoint{4.630102in}{0.554934in}}%
\pgfpathlineto{\pgfqpoint{4.630102in}{0.558168in}}%
\pgfpathlineto{\pgfqpoint{4.631399in}{0.554934in}}%
\pgfpathlineto{\pgfqpoint{4.635291in}{0.556012in}}%
\pgfpathlineto{\pgfqpoint{4.635291in}{0.557090in}}%
\pgfpathlineto{\pgfqpoint{4.636589in}{0.556012in}}%
\pgfpathlineto{\pgfqpoint{4.637886in}{0.556012in}}%
\pgfpathlineto{\pgfqpoint{4.637886in}{0.553856in}}%
\pgfpathlineto{\pgfqpoint{4.639184in}{0.554934in}}%
\pgfpathlineto{\pgfqpoint{4.640481in}{0.554934in}}%
\pgfpathlineto{\pgfqpoint{4.640481in}{0.557090in}}%
\pgfpathlineto{\pgfqpoint{4.641779in}{0.553856in}}%
\pgfpathlineto{\pgfqpoint{4.644373in}{0.553856in}}%
\pgfpathlineto{\pgfqpoint{4.644373in}{0.558168in}}%
\pgfpathlineto{\pgfqpoint{4.645671in}{0.557090in}}%
\pgfpathlineto{\pgfqpoint{4.646968in}{0.557090in}}%
\pgfpathlineto{\pgfqpoint{4.648266in}{0.553856in}}%
\pgfpathlineto{\pgfqpoint{4.649563in}{0.553856in}}%
\pgfpathlineto{\pgfqpoint{4.649563in}{0.556012in}}%
\pgfpathlineto{\pgfqpoint{4.650860in}{0.556012in}}%
\pgfpathlineto{\pgfqpoint{4.652158in}{0.556012in}}%
\pgfpathlineto{\pgfqpoint{4.653455in}{0.552778in}}%
\pgfpathlineto{\pgfqpoint{4.654753in}{0.552778in}}%
\pgfpathlineto{\pgfqpoint{4.654753in}{0.556012in}}%
\pgfpathlineto{\pgfqpoint{4.656050in}{0.554934in}}%
\pgfpathlineto{\pgfqpoint{4.662537in}{0.556012in}}%
\pgfpathlineto{\pgfqpoint{4.662537in}{0.557090in}}%
\pgfpathlineto{\pgfqpoint{4.663835in}{0.557090in}}%
\pgfpathlineto{\pgfqpoint{4.665132in}{0.557090in}}%
\pgfpathlineto{\pgfqpoint{4.665132in}{0.552778in}}%
\pgfpathlineto{\pgfqpoint{4.666430in}{0.557090in}}%
\pgfpathlineto{\pgfqpoint{4.670322in}{0.556012in}}%
\pgfpathlineto{\pgfqpoint{4.671619in}{0.553856in}}%
\pgfpathlineto{\pgfqpoint{4.672917in}{0.553856in}}%
\pgfpathlineto{\pgfqpoint{4.672917in}{0.553856in}}%
\pgfusepath{stroke}%
\end{pgfscope}%
\begin{pgfscope}%
\pgfpathrectangle{\pgfqpoint{0.781944in}{0.552778in}}{\pgfqpoint{3.890972in}{3.248611in}}%
\pgfusepath{clip}%
\pgfsetrectcap%
\pgfsetroundjoin%
\pgfsetlinewidth{1.505625pt}%
\definecolor{currentstroke}{rgb}{1.000000,0.000000,0.000000}%
\pgfsetstrokecolor{currentstroke}%
\pgfsetdash{}{0pt}%
\pgfpathmoveto{\pgfqpoint{2.039148in}{0.552778in}}%
\pgfpathlineto{\pgfqpoint{2.039148in}{3.801389in}}%
\pgfusepath{stroke}%
\end{pgfscope}%
\begin{pgfscope}%
\pgfpathrectangle{\pgfqpoint{0.781944in}{0.552778in}}{\pgfqpoint{3.890972in}{3.248611in}}%
\pgfusepath{clip}%
\pgfsetrectcap%
\pgfsetroundjoin%
\pgfsetlinewidth{1.505625pt}%
\definecolor{currentstroke}{rgb}{1.000000,0.000000,0.000000}%
\pgfsetstrokecolor{currentstroke}%
\pgfsetdash{}{0pt}%
\pgfpathmoveto{\pgfqpoint{2.259709in}{0.552778in}}%
\pgfpathlineto{\pgfqpoint{2.259709in}{3.801389in}}%
\pgfusepath{stroke}%
\end{pgfscope}%
\begin{pgfscope}%
\pgfsetrectcap%
\pgfsetmiterjoin%
\pgfsetlinewidth{0.803000pt}%
\definecolor{currentstroke}{rgb}{0.000000,0.000000,0.000000}%
\pgfsetstrokecolor{currentstroke}%
\pgfsetdash{}{0pt}%
\pgfpathmoveto{\pgfqpoint{0.781944in}{0.552778in}}%
\pgfpathlineto{\pgfqpoint{0.781944in}{3.801389in}}%
\pgfusepath{stroke}%
\end{pgfscope}%
\begin{pgfscope}%
\pgfsetrectcap%
\pgfsetmiterjoin%
\pgfsetlinewidth{0.803000pt}%
\definecolor{currentstroke}{rgb}{0.000000,0.000000,0.000000}%
\pgfsetstrokecolor{currentstroke}%
\pgfsetdash{}{0pt}%
\pgfpathmoveto{\pgfqpoint{4.672917in}{0.552778in}}%
\pgfpathlineto{\pgfqpoint{4.672917in}{3.801389in}}%
\pgfusepath{stroke}%
\end{pgfscope}%
\begin{pgfscope}%
\pgfsetrectcap%
\pgfsetmiterjoin%
\pgfsetlinewidth{0.803000pt}%
\definecolor{currentstroke}{rgb}{0.000000,0.000000,0.000000}%
\pgfsetstrokecolor{currentstroke}%
\pgfsetdash{}{0pt}%
\pgfpathmoveto{\pgfqpoint{0.781944in}{0.552778in}}%
\pgfpathlineto{\pgfqpoint{4.672917in}{0.552778in}}%
\pgfusepath{stroke}%
\end{pgfscope}%
\begin{pgfscope}%
\pgfsetrectcap%
\pgfsetmiterjoin%
\pgfsetlinewidth{0.803000pt}%
\definecolor{currentstroke}{rgb}{0.000000,0.000000,0.000000}%
\pgfsetstrokecolor{currentstroke}%
\pgfsetdash{}{0pt}%
\pgfpathmoveto{\pgfqpoint{0.781944in}{3.801389in}}%
\pgfpathlineto{\pgfqpoint{4.672917in}{3.801389in}}%
\pgfusepath{stroke}%
\end{pgfscope}%
\begin{pgfscope}%
\pgfsetbuttcap%
\pgfsetmiterjoin%
\definecolor{currentfill}{rgb}{1.000000,1.000000,1.000000}%
\pgfsetfillcolor{currentfill}%
\pgfsetfillopacity{0.800000}%
\pgfsetlinewidth{1.003750pt}%
\definecolor{currentstroke}{rgb}{0.800000,0.800000,0.800000}%
\pgfsetstrokecolor{currentstroke}%
\pgfsetstrokeopacity{0.800000}%
\pgfsetdash{}{0pt}%
\pgfpathmoveto{\pgfqpoint{3.224167in}{3.301389in}}%
\pgfpathlineto{\pgfqpoint{4.575694in}{3.301389in}}%
\pgfpathquadraticcurveto{\pgfqpoint{4.603472in}{3.301389in}}{\pgfqpoint{4.603472in}{3.329167in}}%
\pgfpathlineto{\pgfqpoint{4.603472in}{3.704167in}}%
\pgfpathquadraticcurveto{\pgfqpoint{4.603472in}{3.731944in}}{\pgfqpoint{4.575694in}{3.731944in}}%
\pgfpathlineto{\pgfqpoint{3.224167in}{3.731944in}}%
\pgfpathquadraticcurveto{\pgfqpoint{3.196389in}{3.731944in}}{\pgfqpoint{3.196389in}{3.704167in}}%
\pgfpathlineto{\pgfqpoint{3.196389in}{3.329167in}}%
\pgfpathquadraticcurveto{\pgfqpoint{3.196389in}{3.301389in}}{\pgfqpoint{3.224167in}{3.301389in}}%
\pgfpathclose%
\pgfusepath{stroke,fill}%
\end{pgfscope}%
\begin{pgfscope}%
\pgfsetrectcap%
\pgfsetroundjoin%
\pgfsetlinewidth{1.505625pt}%
\definecolor{currentstroke}{rgb}{0.121569,0.466667,0.705882}%
\pgfsetstrokecolor{currentstroke}%
\pgfsetdash{}{0pt}%
\pgfpathmoveto{\pgfqpoint{3.251944in}{3.627778in}}%
\pgfpathlineto{\pgfqpoint{3.529722in}{3.627778in}}%
\pgfusepath{stroke}%
\end{pgfscope}%
\begin{pgfscope}%
\definecolor{textcolor}{rgb}{0.000000,0.000000,0.000000}%
\pgfsetstrokecolor{textcolor}%
\pgfsetfillcolor{textcolor}%
\pgftext[x=3.640833in,y=3.579167in,left,base]{\color{textcolor}\rmfamily\fontsize{10.000000}{12.000000}\selectfont Ereignisszahl}%
\end{pgfscope}%
\begin{pgfscope}%
\pgfsetrectcap%
\pgfsetroundjoin%
\pgfsetlinewidth{1.505625pt}%
\definecolor{currentstroke}{rgb}{1.000000,0.000000,0.000000}%
\pgfsetstrokecolor{currentstroke}%
\pgfsetdash{}{0pt}%
\pgfpathmoveto{\pgfqpoint{3.251944in}{3.432500in}}%
\pgfpathlineto{\pgfqpoint{3.529722in}{3.432500in}}%
\pgfusepath{stroke}%
\end{pgfscope}%
\begin{pgfscope}%
\definecolor{textcolor}{rgb}{0.000000,0.000000,0.000000}%
\pgfsetstrokecolor{textcolor}%
\pgfsetfillcolor{textcolor}%
\pgftext[x=3.640833in,y=3.383889in,left,base]{\color{textcolor}\rmfamily\fontsize{10.000000}{12.000000}\selectfont Schranken Zeit}%
\end{pgfscope}%
\end{pgfpicture}%
\makeatother%
\endgroup%

  \caption{Ereignisszahl \"uber Kanal zur Bestimmung der Zeitschranken.}
  \label{fig:calibration-time_range}
\end{figure}

\begin{align}
  \label{eq:timeint}
  R_T &= [970, 1140] \hat{=} [46.813\pm 0.027, 55.024\pm
        0.030]\,\si{\nano\second} = [t_1, t_2]
\end{align}

\subsubsection{Koinzidenzaufl\"osungszeit, Anteil zuf\"alliger
  Koinzidenze und Koinzidenznachweiseffektivit\"at}
\label{sec:koaufl}

Nimmt man die Differnz der beiden Endpunkte von~\eqref{eq:timeint} so
ergiebt sich die Koinzidenzaufl\"osungszeit \(\tau\) zu:
\begin{equation}
  \label{eq:koauf}
  \tau = t_2 - t_1 \pm \sqrt{\qty(\Delta t_1)^2 + \qty(\Delta t_2)^2}
  = \SI{8.21\pm .04}{\nano\second}
\end{equation}

Dieser Wert liegt in der Gr\"o\ss{}enordnung der Lichtlaufzeit
zweischen den Detektoren (ca. \SI{1}{\nano\second}).

Die Koinzidenzz\"ahlrate (ohne Filterung nach Energie) ergibt sich
durch Summierung der Ereignisszahlen (\(\Delta N = \sqrt{N}\),
Poisson) im Zeitinterval~\eqref{eq:timeint} und der Division
durch~\eqref{eq:caltime}.

\begin{equation}
  \label{eq:ctrate}
  \mathfrak{R} = \frac{N}{T} \pm \sqrt{\frac{N}{T^2} +
    \qty(\frac{N}{T^2}\cdot \Delta T)^2} = \SI{422.7\pm 1.5}{\second^{-1}}
\end{equation}

Die Aktivität der \ce{22^Na} Kalibrierungsprobe ergibt sich mit
\(t=\SI{212198400}{\second},\, t_{1/2} = \SI{2.6027\pm .0010}{\second}\) und \(A_0
= \SI{1.36}{\mega\becquerel}\) (1.10.2014, Abw. \SI{3}{\percent}).

\begin{equation}
  \label{eq:acttoday}
  A = A_0\cdot \qty(\frac{1}{2})^{t/t_{1/2}} = \SI{.227\pm.007}{\mega\becquerel}
\end{equation}

Die Z\"ahlate der zuf\"alligen Koinzidenzen ergibt sich aus
\todo{!!!!! GLEICHUNG} mit der Kantenl\"ange des Detektors
\(a=\SI{54}{\milli\meter}\) und dem Detektorabstand
\(D=\SI{386}{\milli\meter}\) sowie
\(\Omega_{min} = \frac{4a^2}{D^2} = 0.078,\, P_\beta = .90382\pm
.00021,\, \)

\begin{align}
  \label{eq:incidentalrate}
  \mathfrak{R}_Z &= \qty[4\tau\cdot A\cdot
  \qty(\frac{\Omega_{min}}{2\pi})]\cdot \mathfrak{R} \pm
  \sqrt{\qty(\frac{\Delta\tau}{\tau})^2 + \qty(\frac{\Delta A}{A})^2 +
    \qty(\frac{\Delta\Omega_{min}}{\Omega_{min}})^2 +
                   \qty(\frac{\Delta\mathfrak{R}}{\mathfrak{R}})^2} \\
                 &= \SI{0.0392\pm .0012}{\second^{-1}} \approx \SI{.01}{\percent}
\end{align}

Die zuf\"alligen Koinzidenzen spielen hier also nur eine
untergeordnete Rolle und k\"onnen in guter N\"aherung vernachl\"assigt werden.

Die Koinzidenznachweiseffektivit\"at bestimmt sich
\todo{equathsitoenshtioesnht} zu:

\begin{align}
  \epsilon & = \frac{\mathfrak{R}}{P_\beta\cdot A\cdot
  \frac{\Omega_\min}{2\pi}} \pm \sqrt{\qty(\frac{\Delta A}{A})^2 +
  \qty(\frac{\Delta\Omega_{min}}{\Omega_{min}})^2 +
  \qty(\frac{\Delta P_\beta}{P_\beta})^2 +
             \qty(\frac{\Delta\mathfrak{R}}{\mathfrak{R}})^2} \\
  & = \SI{16.6\pm 1.2}{\percent}
\end{align}

Damit ist die Effektivit\"at, wie im folgenden zu sehen, zwar
ausreichend f\"ur einigerma\ss{}en z\"ugigie Messungen, k\"onnte aber
optimiert werden. Der hier gefundene Wert stellt einen sch\"atzer
f\"ur die Effektivit\"at mit bestimmten Zeit-, aber unbestimmten
Energieintervall da.


\subsubsection{Einfluss der Quellposition}
\label{sec:quellpos}

Die Energie und Zeitspektren f\"ur verschiedene Quellpositionen sind
in~\ref{fig:calibration-comp} dargestellt.
\begin{figure}[H]\centering
  %% Creator: Matplotlib, PGF backend
%%
%% To include the figure in your LaTeX document, write
%%   \input{<filename>.pgf}
%%
%% Make sure the required packages are loaded in your preamble
%%   \usepackage{pgf}
%%
%% Figures using additional raster images can only be included by \input if
%% they are in the same directory as the main LaTeX file. For loading figures
%% from other directories you can use the `import` package
%%   \usepackage{import}
%% and then include the figures with
%%   \import{<path to file>}{<filename>.pgf}
%%
%% Matplotlib used the following preamble
%%   \usepackage{fontspec}
%%
\begingroup%
\makeatletter%
\begin{pgfpicture}%
\pgfpathrectangle{\pgfpointorigin}{\pgfqpoint{6.000000in}{5.000000in}}%
\pgfusepath{use as bounding box, clip}%
\begin{pgfscope}%
\pgfsetbuttcap%
\pgfsetmiterjoin%
\definecolor{currentfill}{rgb}{1.000000,1.000000,1.000000}%
\pgfsetfillcolor{currentfill}%
\pgfsetlinewidth{0.000000pt}%
\definecolor{currentstroke}{rgb}{1.000000,1.000000,1.000000}%
\pgfsetstrokecolor{currentstroke}%
\pgfsetdash{}{0pt}%
\pgfpathmoveto{\pgfqpoint{0.000000in}{0.000000in}}%
\pgfpathlineto{\pgfqpoint{6.000000in}{0.000000in}}%
\pgfpathlineto{\pgfqpoint{6.000000in}{5.000000in}}%
\pgfpathlineto{\pgfqpoint{0.000000in}{5.000000in}}%
\pgfpathclose%
\pgfusepath{fill}%
\end{pgfscope}%
\begin{pgfscope}%
\pgfsetbuttcap%
\pgfsetmiterjoin%
\definecolor{currentfill}{rgb}{1.000000,1.000000,1.000000}%
\pgfsetfillcolor{currentfill}%
\pgfsetlinewidth{0.000000pt}%
\definecolor{currentstroke}{rgb}{0.000000,0.000000,0.000000}%
\pgfsetstrokecolor{currentstroke}%
\pgfsetstrokeopacity{0.000000}%
\pgfsetdash{}{0pt}%
\pgfpathmoveto{\pgfqpoint{0.781944in}{2.977778in}}%
\pgfpathlineto{\pgfqpoint{5.801389in}{2.977778in}}%
\pgfpathlineto{\pgfqpoint{5.801389in}{4.627778in}}%
\pgfpathlineto{\pgfqpoint{0.781944in}{4.627778in}}%
\pgfpathclose%
\pgfusepath{fill}%
\end{pgfscope}%
\begin{pgfscope}%
\pgfpathrectangle{\pgfqpoint{0.781944in}{2.977778in}}{\pgfqpoint{5.019444in}{1.650000in}}%
\pgfusepath{clip}%
\pgfsetrectcap%
\pgfsetroundjoin%
\pgfsetlinewidth{0.803000pt}%
\definecolor{currentstroke}{rgb}{0.690196,0.690196,0.690196}%
\pgfsetstrokecolor{currentstroke}%
\pgfsetstrokeopacity{0.800000}%
\pgfsetdash{}{0pt}%
\pgfpathmoveto{\pgfqpoint{1.032917in}{2.977778in}}%
\pgfpathlineto{\pgfqpoint{1.032917in}{4.627778in}}%
\pgfusepath{stroke}%
\end{pgfscope}%
\begin{pgfscope}%
\pgfsetbuttcap%
\pgfsetroundjoin%
\definecolor{currentfill}{rgb}{0.000000,0.000000,0.000000}%
\pgfsetfillcolor{currentfill}%
\pgfsetlinewidth{0.803000pt}%
\definecolor{currentstroke}{rgb}{0.000000,0.000000,0.000000}%
\pgfsetstrokecolor{currentstroke}%
\pgfsetdash{}{0pt}%
\pgfsys@defobject{currentmarker}{\pgfqpoint{0.000000in}{-0.048611in}}{\pgfqpoint{0.000000in}{0.000000in}}{%
\pgfpathmoveto{\pgfqpoint{0.000000in}{0.000000in}}%
\pgfpathlineto{\pgfqpoint{0.000000in}{-0.048611in}}%
\pgfusepath{stroke,fill}%
}%
\begin{pgfscope}%
\pgfsys@transformshift{1.032917in}{2.977778in}%
\pgfsys@useobject{currentmarker}{}%
\end{pgfscope}%
\end{pgfscope}%
\begin{pgfscope}%
\pgfsetbuttcap%
\pgfsetroundjoin%
\definecolor{currentfill}{rgb}{0.000000,0.000000,0.000000}%
\pgfsetfillcolor{currentfill}%
\pgfsetlinewidth{0.803000pt}%
\definecolor{currentstroke}{rgb}{0.000000,0.000000,0.000000}%
\pgfsetstrokecolor{currentstroke}%
\pgfsetdash{}{0pt}%
\pgfsys@defobject{currentmarker}{\pgfqpoint{0.000000in}{0.000000in}}{\pgfqpoint{0.000000in}{0.048611in}}{%
\pgfpathmoveto{\pgfqpoint{0.000000in}{0.000000in}}%
\pgfpathlineto{\pgfqpoint{0.000000in}{0.048611in}}%
\pgfusepath{stroke,fill}%
}%
\begin{pgfscope}%
\pgfsys@transformshift{1.032917in}{4.627778in}%
\pgfsys@useobject{currentmarker}{}%
\end{pgfscope}%
\end{pgfscope}%
\begin{pgfscope}%
\definecolor{textcolor}{rgb}{0.000000,0.000000,0.000000}%
\pgfsetstrokecolor{textcolor}%
\pgfsetfillcolor{textcolor}%
\pgftext[x=1.032917in,y=2.880556in,,top]{\color{textcolor}\rmfamily\fontsize{10.000000}{12.000000}\selectfont 32}%
\end{pgfscope}%
\begin{pgfscope}%
\pgfpathrectangle{\pgfqpoint{0.781944in}{2.977778in}}{\pgfqpoint{5.019444in}{1.650000in}}%
\pgfusepath{clip}%
\pgfsetrectcap%
\pgfsetroundjoin%
\pgfsetlinewidth{0.803000pt}%
\definecolor{currentstroke}{rgb}{0.690196,0.690196,0.690196}%
\pgfsetstrokecolor{currentstroke}%
\pgfsetstrokeopacity{0.800000}%
\pgfsetdash{}{0pt}%
\pgfpathmoveto{\pgfqpoint{1.534861in}{2.977778in}}%
\pgfpathlineto{\pgfqpoint{1.534861in}{4.627778in}}%
\pgfusepath{stroke}%
\end{pgfscope}%
\begin{pgfscope}%
\pgfsetbuttcap%
\pgfsetroundjoin%
\definecolor{currentfill}{rgb}{0.000000,0.000000,0.000000}%
\pgfsetfillcolor{currentfill}%
\pgfsetlinewidth{0.803000pt}%
\definecolor{currentstroke}{rgb}{0.000000,0.000000,0.000000}%
\pgfsetstrokecolor{currentstroke}%
\pgfsetdash{}{0pt}%
\pgfsys@defobject{currentmarker}{\pgfqpoint{0.000000in}{-0.048611in}}{\pgfqpoint{0.000000in}{0.000000in}}{%
\pgfpathmoveto{\pgfqpoint{0.000000in}{0.000000in}}%
\pgfpathlineto{\pgfqpoint{0.000000in}{-0.048611in}}%
\pgfusepath{stroke,fill}%
}%
\begin{pgfscope}%
\pgfsys@transformshift{1.534861in}{2.977778in}%
\pgfsys@useobject{currentmarker}{}%
\end{pgfscope}%
\end{pgfscope}%
\begin{pgfscope}%
\pgfsetbuttcap%
\pgfsetroundjoin%
\definecolor{currentfill}{rgb}{0.000000,0.000000,0.000000}%
\pgfsetfillcolor{currentfill}%
\pgfsetlinewidth{0.803000pt}%
\definecolor{currentstroke}{rgb}{0.000000,0.000000,0.000000}%
\pgfsetstrokecolor{currentstroke}%
\pgfsetdash{}{0pt}%
\pgfsys@defobject{currentmarker}{\pgfqpoint{0.000000in}{0.000000in}}{\pgfqpoint{0.000000in}{0.048611in}}{%
\pgfpathmoveto{\pgfqpoint{0.000000in}{0.000000in}}%
\pgfpathlineto{\pgfqpoint{0.000000in}{0.048611in}}%
\pgfusepath{stroke,fill}%
}%
\begin{pgfscope}%
\pgfsys@transformshift{1.534861in}{4.627778in}%
\pgfsys@useobject{currentmarker}{}%
\end{pgfscope}%
\end{pgfscope}%
\begin{pgfscope}%
\definecolor{textcolor}{rgb}{0.000000,0.000000,0.000000}%
\pgfsetstrokecolor{textcolor}%
\pgfsetfillcolor{textcolor}%
\pgftext[x=1.534861in,y=2.880556in,,top]{\color{textcolor}\rmfamily\fontsize{10.000000}{12.000000}\selectfont 36}%
\end{pgfscope}%
\begin{pgfscope}%
\pgfpathrectangle{\pgfqpoint{0.781944in}{2.977778in}}{\pgfqpoint{5.019444in}{1.650000in}}%
\pgfusepath{clip}%
\pgfsetrectcap%
\pgfsetroundjoin%
\pgfsetlinewidth{0.803000pt}%
\definecolor{currentstroke}{rgb}{0.690196,0.690196,0.690196}%
\pgfsetstrokecolor{currentstroke}%
\pgfsetstrokeopacity{0.800000}%
\pgfsetdash{}{0pt}%
\pgfpathmoveto{\pgfqpoint{2.036806in}{2.977778in}}%
\pgfpathlineto{\pgfqpoint{2.036806in}{4.627778in}}%
\pgfusepath{stroke}%
\end{pgfscope}%
\begin{pgfscope}%
\pgfsetbuttcap%
\pgfsetroundjoin%
\definecolor{currentfill}{rgb}{0.000000,0.000000,0.000000}%
\pgfsetfillcolor{currentfill}%
\pgfsetlinewidth{0.803000pt}%
\definecolor{currentstroke}{rgb}{0.000000,0.000000,0.000000}%
\pgfsetstrokecolor{currentstroke}%
\pgfsetdash{}{0pt}%
\pgfsys@defobject{currentmarker}{\pgfqpoint{0.000000in}{-0.048611in}}{\pgfqpoint{0.000000in}{0.000000in}}{%
\pgfpathmoveto{\pgfqpoint{0.000000in}{0.000000in}}%
\pgfpathlineto{\pgfqpoint{0.000000in}{-0.048611in}}%
\pgfusepath{stroke,fill}%
}%
\begin{pgfscope}%
\pgfsys@transformshift{2.036806in}{2.977778in}%
\pgfsys@useobject{currentmarker}{}%
\end{pgfscope}%
\end{pgfscope}%
\begin{pgfscope}%
\pgfsetbuttcap%
\pgfsetroundjoin%
\definecolor{currentfill}{rgb}{0.000000,0.000000,0.000000}%
\pgfsetfillcolor{currentfill}%
\pgfsetlinewidth{0.803000pt}%
\definecolor{currentstroke}{rgb}{0.000000,0.000000,0.000000}%
\pgfsetstrokecolor{currentstroke}%
\pgfsetdash{}{0pt}%
\pgfsys@defobject{currentmarker}{\pgfqpoint{0.000000in}{0.000000in}}{\pgfqpoint{0.000000in}{0.048611in}}{%
\pgfpathmoveto{\pgfqpoint{0.000000in}{0.000000in}}%
\pgfpathlineto{\pgfqpoint{0.000000in}{0.048611in}}%
\pgfusepath{stroke,fill}%
}%
\begin{pgfscope}%
\pgfsys@transformshift{2.036806in}{4.627778in}%
\pgfsys@useobject{currentmarker}{}%
\end{pgfscope}%
\end{pgfscope}%
\begin{pgfscope}%
\definecolor{textcolor}{rgb}{0.000000,0.000000,0.000000}%
\pgfsetstrokecolor{textcolor}%
\pgfsetfillcolor{textcolor}%
\pgftext[x=2.036806in,y=2.880556in,,top]{\color{textcolor}\rmfamily\fontsize{10.000000}{12.000000}\selectfont 40}%
\end{pgfscope}%
\begin{pgfscope}%
\pgfpathrectangle{\pgfqpoint{0.781944in}{2.977778in}}{\pgfqpoint{5.019444in}{1.650000in}}%
\pgfusepath{clip}%
\pgfsetrectcap%
\pgfsetroundjoin%
\pgfsetlinewidth{0.803000pt}%
\definecolor{currentstroke}{rgb}{0.690196,0.690196,0.690196}%
\pgfsetstrokecolor{currentstroke}%
\pgfsetstrokeopacity{0.800000}%
\pgfsetdash{}{0pt}%
\pgfpathmoveto{\pgfqpoint{2.538750in}{2.977778in}}%
\pgfpathlineto{\pgfqpoint{2.538750in}{4.627778in}}%
\pgfusepath{stroke}%
\end{pgfscope}%
\begin{pgfscope}%
\pgfsetbuttcap%
\pgfsetroundjoin%
\definecolor{currentfill}{rgb}{0.000000,0.000000,0.000000}%
\pgfsetfillcolor{currentfill}%
\pgfsetlinewidth{0.803000pt}%
\definecolor{currentstroke}{rgb}{0.000000,0.000000,0.000000}%
\pgfsetstrokecolor{currentstroke}%
\pgfsetdash{}{0pt}%
\pgfsys@defobject{currentmarker}{\pgfqpoint{0.000000in}{-0.048611in}}{\pgfqpoint{0.000000in}{0.000000in}}{%
\pgfpathmoveto{\pgfqpoint{0.000000in}{0.000000in}}%
\pgfpathlineto{\pgfqpoint{0.000000in}{-0.048611in}}%
\pgfusepath{stroke,fill}%
}%
\begin{pgfscope}%
\pgfsys@transformshift{2.538750in}{2.977778in}%
\pgfsys@useobject{currentmarker}{}%
\end{pgfscope}%
\end{pgfscope}%
\begin{pgfscope}%
\pgfsetbuttcap%
\pgfsetroundjoin%
\definecolor{currentfill}{rgb}{0.000000,0.000000,0.000000}%
\pgfsetfillcolor{currentfill}%
\pgfsetlinewidth{0.803000pt}%
\definecolor{currentstroke}{rgb}{0.000000,0.000000,0.000000}%
\pgfsetstrokecolor{currentstroke}%
\pgfsetdash{}{0pt}%
\pgfsys@defobject{currentmarker}{\pgfqpoint{0.000000in}{0.000000in}}{\pgfqpoint{0.000000in}{0.048611in}}{%
\pgfpathmoveto{\pgfqpoint{0.000000in}{0.000000in}}%
\pgfpathlineto{\pgfqpoint{0.000000in}{0.048611in}}%
\pgfusepath{stroke,fill}%
}%
\begin{pgfscope}%
\pgfsys@transformshift{2.538750in}{4.627778in}%
\pgfsys@useobject{currentmarker}{}%
\end{pgfscope}%
\end{pgfscope}%
\begin{pgfscope}%
\definecolor{textcolor}{rgb}{0.000000,0.000000,0.000000}%
\pgfsetstrokecolor{textcolor}%
\pgfsetfillcolor{textcolor}%
\pgftext[x=2.538750in,y=2.880556in,,top]{\color{textcolor}\rmfamily\fontsize{10.000000}{12.000000}\selectfont 44}%
\end{pgfscope}%
\begin{pgfscope}%
\pgfpathrectangle{\pgfqpoint{0.781944in}{2.977778in}}{\pgfqpoint{5.019444in}{1.650000in}}%
\pgfusepath{clip}%
\pgfsetrectcap%
\pgfsetroundjoin%
\pgfsetlinewidth{0.803000pt}%
\definecolor{currentstroke}{rgb}{0.690196,0.690196,0.690196}%
\pgfsetstrokecolor{currentstroke}%
\pgfsetstrokeopacity{0.800000}%
\pgfsetdash{}{0pt}%
\pgfpathmoveto{\pgfqpoint{3.040694in}{2.977778in}}%
\pgfpathlineto{\pgfqpoint{3.040694in}{4.627778in}}%
\pgfusepath{stroke}%
\end{pgfscope}%
\begin{pgfscope}%
\pgfsetbuttcap%
\pgfsetroundjoin%
\definecolor{currentfill}{rgb}{0.000000,0.000000,0.000000}%
\pgfsetfillcolor{currentfill}%
\pgfsetlinewidth{0.803000pt}%
\definecolor{currentstroke}{rgb}{0.000000,0.000000,0.000000}%
\pgfsetstrokecolor{currentstroke}%
\pgfsetdash{}{0pt}%
\pgfsys@defobject{currentmarker}{\pgfqpoint{0.000000in}{-0.048611in}}{\pgfqpoint{0.000000in}{0.000000in}}{%
\pgfpathmoveto{\pgfqpoint{0.000000in}{0.000000in}}%
\pgfpathlineto{\pgfqpoint{0.000000in}{-0.048611in}}%
\pgfusepath{stroke,fill}%
}%
\begin{pgfscope}%
\pgfsys@transformshift{3.040694in}{2.977778in}%
\pgfsys@useobject{currentmarker}{}%
\end{pgfscope}%
\end{pgfscope}%
\begin{pgfscope}%
\pgfsetbuttcap%
\pgfsetroundjoin%
\definecolor{currentfill}{rgb}{0.000000,0.000000,0.000000}%
\pgfsetfillcolor{currentfill}%
\pgfsetlinewidth{0.803000pt}%
\definecolor{currentstroke}{rgb}{0.000000,0.000000,0.000000}%
\pgfsetstrokecolor{currentstroke}%
\pgfsetdash{}{0pt}%
\pgfsys@defobject{currentmarker}{\pgfqpoint{0.000000in}{0.000000in}}{\pgfqpoint{0.000000in}{0.048611in}}{%
\pgfpathmoveto{\pgfqpoint{0.000000in}{0.000000in}}%
\pgfpathlineto{\pgfqpoint{0.000000in}{0.048611in}}%
\pgfusepath{stroke,fill}%
}%
\begin{pgfscope}%
\pgfsys@transformshift{3.040694in}{4.627778in}%
\pgfsys@useobject{currentmarker}{}%
\end{pgfscope}%
\end{pgfscope}%
\begin{pgfscope}%
\definecolor{textcolor}{rgb}{0.000000,0.000000,0.000000}%
\pgfsetstrokecolor{textcolor}%
\pgfsetfillcolor{textcolor}%
\pgftext[x=3.040694in,y=2.880556in,,top]{\color{textcolor}\rmfamily\fontsize{10.000000}{12.000000}\selectfont 48}%
\end{pgfscope}%
\begin{pgfscope}%
\pgfpathrectangle{\pgfqpoint{0.781944in}{2.977778in}}{\pgfqpoint{5.019444in}{1.650000in}}%
\pgfusepath{clip}%
\pgfsetrectcap%
\pgfsetroundjoin%
\pgfsetlinewidth{0.803000pt}%
\definecolor{currentstroke}{rgb}{0.690196,0.690196,0.690196}%
\pgfsetstrokecolor{currentstroke}%
\pgfsetstrokeopacity{0.800000}%
\pgfsetdash{}{0pt}%
\pgfpathmoveto{\pgfqpoint{3.542639in}{2.977778in}}%
\pgfpathlineto{\pgfqpoint{3.542639in}{4.627778in}}%
\pgfusepath{stroke}%
\end{pgfscope}%
\begin{pgfscope}%
\pgfsetbuttcap%
\pgfsetroundjoin%
\definecolor{currentfill}{rgb}{0.000000,0.000000,0.000000}%
\pgfsetfillcolor{currentfill}%
\pgfsetlinewidth{0.803000pt}%
\definecolor{currentstroke}{rgb}{0.000000,0.000000,0.000000}%
\pgfsetstrokecolor{currentstroke}%
\pgfsetdash{}{0pt}%
\pgfsys@defobject{currentmarker}{\pgfqpoint{0.000000in}{-0.048611in}}{\pgfqpoint{0.000000in}{0.000000in}}{%
\pgfpathmoveto{\pgfqpoint{0.000000in}{0.000000in}}%
\pgfpathlineto{\pgfqpoint{0.000000in}{-0.048611in}}%
\pgfusepath{stroke,fill}%
}%
\begin{pgfscope}%
\pgfsys@transformshift{3.542639in}{2.977778in}%
\pgfsys@useobject{currentmarker}{}%
\end{pgfscope}%
\end{pgfscope}%
\begin{pgfscope}%
\pgfsetbuttcap%
\pgfsetroundjoin%
\definecolor{currentfill}{rgb}{0.000000,0.000000,0.000000}%
\pgfsetfillcolor{currentfill}%
\pgfsetlinewidth{0.803000pt}%
\definecolor{currentstroke}{rgb}{0.000000,0.000000,0.000000}%
\pgfsetstrokecolor{currentstroke}%
\pgfsetdash{}{0pt}%
\pgfsys@defobject{currentmarker}{\pgfqpoint{0.000000in}{0.000000in}}{\pgfqpoint{0.000000in}{0.048611in}}{%
\pgfpathmoveto{\pgfqpoint{0.000000in}{0.000000in}}%
\pgfpathlineto{\pgfqpoint{0.000000in}{0.048611in}}%
\pgfusepath{stroke,fill}%
}%
\begin{pgfscope}%
\pgfsys@transformshift{3.542639in}{4.627778in}%
\pgfsys@useobject{currentmarker}{}%
\end{pgfscope}%
\end{pgfscope}%
\begin{pgfscope}%
\definecolor{textcolor}{rgb}{0.000000,0.000000,0.000000}%
\pgfsetstrokecolor{textcolor}%
\pgfsetfillcolor{textcolor}%
\pgftext[x=3.542639in,y=2.880556in,,top]{\color{textcolor}\rmfamily\fontsize{10.000000}{12.000000}\selectfont 52}%
\end{pgfscope}%
\begin{pgfscope}%
\pgfpathrectangle{\pgfqpoint{0.781944in}{2.977778in}}{\pgfqpoint{5.019444in}{1.650000in}}%
\pgfusepath{clip}%
\pgfsetrectcap%
\pgfsetroundjoin%
\pgfsetlinewidth{0.803000pt}%
\definecolor{currentstroke}{rgb}{0.690196,0.690196,0.690196}%
\pgfsetstrokecolor{currentstroke}%
\pgfsetstrokeopacity{0.800000}%
\pgfsetdash{}{0pt}%
\pgfpathmoveto{\pgfqpoint{4.044583in}{2.977778in}}%
\pgfpathlineto{\pgfqpoint{4.044583in}{4.627778in}}%
\pgfusepath{stroke}%
\end{pgfscope}%
\begin{pgfscope}%
\pgfsetbuttcap%
\pgfsetroundjoin%
\definecolor{currentfill}{rgb}{0.000000,0.000000,0.000000}%
\pgfsetfillcolor{currentfill}%
\pgfsetlinewidth{0.803000pt}%
\definecolor{currentstroke}{rgb}{0.000000,0.000000,0.000000}%
\pgfsetstrokecolor{currentstroke}%
\pgfsetdash{}{0pt}%
\pgfsys@defobject{currentmarker}{\pgfqpoint{0.000000in}{-0.048611in}}{\pgfqpoint{0.000000in}{0.000000in}}{%
\pgfpathmoveto{\pgfqpoint{0.000000in}{0.000000in}}%
\pgfpathlineto{\pgfqpoint{0.000000in}{-0.048611in}}%
\pgfusepath{stroke,fill}%
}%
\begin{pgfscope}%
\pgfsys@transformshift{4.044583in}{2.977778in}%
\pgfsys@useobject{currentmarker}{}%
\end{pgfscope}%
\end{pgfscope}%
\begin{pgfscope}%
\pgfsetbuttcap%
\pgfsetroundjoin%
\definecolor{currentfill}{rgb}{0.000000,0.000000,0.000000}%
\pgfsetfillcolor{currentfill}%
\pgfsetlinewidth{0.803000pt}%
\definecolor{currentstroke}{rgb}{0.000000,0.000000,0.000000}%
\pgfsetstrokecolor{currentstroke}%
\pgfsetdash{}{0pt}%
\pgfsys@defobject{currentmarker}{\pgfqpoint{0.000000in}{0.000000in}}{\pgfqpoint{0.000000in}{0.048611in}}{%
\pgfpathmoveto{\pgfqpoint{0.000000in}{0.000000in}}%
\pgfpathlineto{\pgfqpoint{0.000000in}{0.048611in}}%
\pgfusepath{stroke,fill}%
}%
\begin{pgfscope}%
\pgfsys@transformshift{4.044583in}{4.627778in}%
\pgfsys@useobject{currentmarker}{}%
\end{pgfscope}%
\end{pgfscope}%
\begin{pgfscope}%
\definecolor{textcolor}{rgb}{0.000000,0.000000,0.000000}%
\pgfsetstrokecolor{textcolor}%
\pgfsetfillcolor{textcolor}%
\pgftext[x=4.044583in,y=2.880556in,,top]{\color{textcolor}\rmfamily\fontsize{10.000000}{12.000000}\selectfont 56}%
\end{pgfscope}%
\begin{pgfscope}%
\pgfpathrectangle{\pgfqpoint{0.781944in}{2.977778in}}{\pgfqpoint{5.019444in}{1.650000in}}%
\pgfusepath{clip}%
\pgfsetrectcap%
\pgfsetroundjoin%
\pgfsetlinewidth{0.803000pt}%
\definecolor{currentstroke}{rgb}{0.690196,0.690196,0.690196}%
\pgfsetstrokecolor{currentstroke}%
\pgfsetstrokeopacity{0.800000}%
\pgfsetdash{}{0pt}%
\pgfpathmoveto{\pgfqpoint{4.546528in}{2.977778in}}%
\pgfpathlineto{\pgfqpoint{4.546528in}{4.627778in}}%
\pgfusepath{stroke}%
\end{pgfscope}%
\begin{pgfscope}%
\pgfsetbuttcap%
\pgfsetroundjoin%
\definecolor{currentfill}{rgb}{0.000000,0.000000,0.000000}%
\pgfsetfillcolor{currentfill}%
\pgfsetlinewidth{0.803000pt}%
\definecolor{currentstroke}{rgb}{0.000000,0.000000,0.000000}%
\pgfsetstrokecolor{currentstroke}%
\pgfsetdash{}{0pt}%
\pgfsys@defobject{currentmarker}{\pgfqpoint{0.000000in}{-0.048611in}}{\pgfqpoint{0.000000in}{0.000000in}}{%
\pgfpathmoveto{\pgfqpoint{0.000000in}{0.000000in}}%
\pgfpathlineto{\pgfqpoint{0.000000in}{-0.048611in}}%
\pgfusepath{stroke,fill}%
}%
\begin{pgfscope}%
\pgfsys@transformshift{4.546528in}{2.977778in}%
\pgfsys@useobject{currentmarker}{}%
\end{pgfscope}%
\end{pgfscope}%
\begin{pgfscope}%
\pgfsetbuttcap%
\pgfsetroundjoin%
\definecolor{currentfill}{rgb}{0.000000,0.000000,0.000000}%
\pgfsetfillcolor{currentfill}%
\pgfsetlinewidth{0.803000pt}%
\definecolor{currentstroke}{rgb}{0.000000,0.000000,0.000000}%
\pgfsetstrokecolor{currentstroke}%
\pgfsetdash{}{0pt}%
\pgfsys@defobject{currentmarker}{\pgfqpoint{0.000000in}{0.000000in}}{\pgfqpoint{0.000000in}{0.048611in}}{%
\pgfpathmoveto{\pgfqpoint{0.000000in}{0.000000in}}%
\pgfpathlineto{\pgfqpoint{0.000000in}{0.048611in}}%
\pgfusepath{stroke,fill}%
}%
\begin{pgfscope}%
\pgfsys@transformshift{4.546528in}{4.627778in}%
\pgfsys@useobject{currentmarker}{}%
\end{pgfscope}%
\end{pgfscope}%
\begin{pgfscope}%
\definecolor{textcolor}{rgb}{0.000000,0.000000,0.000000}%
\pgfsetstrokecolor{textcolor}%
\pgfsetfillcolor{textcolor}%
\pgftext[x=4.546528in,y=2.880556in,,top]{\color{textcolor}\rmfamily\fontsize{10.000000}{12.000000}\selectfont 60}%
\end{pgfscope}%
\begin{pgfscope}%
\pgfpathrectangle{\pgfqpoint{0.781944in}{2.977778in}}{\pgfqpoint{5.019444in}{1.650000in}}%
\pgfusepath{clip}%
\pgfsetrectcap%
\pgfsetroundjoin%
\pgfsetlinewidth{0.803000pt}%
\definecolor{currentstroke}{rgb}{0.690196,0.690196,0.690196}%
\pgfsetstrokecolor{currentstroke}%
\pgfsetstrokeopacity{0.800000}%
\pgfsetdash{}{0pt}%
\pgfpathmoveto{\pgfqpoint{5.048472in}{2.977778in}}%
\pgfpathlineto{\pgfqpoint{5.048472in}{4.627778in}}%
\pgfusepath{stroke}%
\end{pgfscope}%
\begin{pgfscope}%
\pgfsetbuttcap%
\pgfsetroundjoin%
\definecolor{currentfill}{rgb}{0.000000,0.000000,0.000000}%
\pgfsetfillcolor{currentfill}%
\pgfsetlinewidth{0.803000pt}%
\definecolor{currentstroke}{rgb}{0.000000,0.000000,0.000000}%
\pgfsetstrokecolor{currentstroke}%
\pgfsetdash{}{0pt}%
\pgfsys@defobject{currentmarker}{\pgfqpoint{0.000000in}{-0.048611in}}{\pgfqpoint{0.000000in}{0.000000in}}{%
\pgfpathmoveto{\pgfqpoint{0.000000in}{0.000000in}}%
\pgfpathlineto{\pgfqpoint{0.000000in}{-0.048611in}}%
\pgfusepath{stroke,fill}%
}%
\begin{pgfscope}%
\pgfsys@transformshift{5.048472in}{2.977778in}%
\pgfsys@useobject{currentmarker}{}%
\end{pgfscope}%
\end{pgfscope}%
\begin{pgfscope}%
\pgfsetbuttcap%
\pgfsetroundjoin%
\definecolor{currentfill}{rgb}{0.000000,0.000000,0.000000}%
\pgfsetfillcolor{currentfill}%
\pgfsetlinewidth{0.803000pt}%
\definecolor{currentstroke}{rgb}{0.000000,0.000000,0.000000}%
\pgfsetstrokecolor{currentstroke}%
\pgfsetdash{}{0pt}%
\pgfsys@defobject{currentmarker}{\pgfqpoint{0.000000in}{0.000000in}}{\pgfqpoint{0.000000in}{0.048611in}}{%
\pgfpathmoveto{\pgfqpoint{0.000000in}{0.000000in}}%
\pgfpathlineto{\pgfqpoint{0.000000in}{0.048611in}}%
\pgfusepath{stroke,fill}%
}%
\begin{pgfscope}%
\pgfsys@transformshift{5.048472in}{4.627778in}%
\pgfsys@useobject{currentmarker}{}%
\end{pgfscope}%
\end{pgfscope}%
\begin{pgfscope}%
\definecolor{textcolor}{rgb}{0.000000,0.000000,0.000000}%
\pgfsetstrokecolor{textcolor}%
\pgfsetfillcolor{textcolor}%
\pgftext[x=5.048472in,y=2.880556in,,top]{\color{textcolor}\rmfamily\fontsize{10.000000}{12.000000}\selectfont 64}%
\end{pgfscope}%
\begin{pgfscope}%
\pgfpathrectangle{\pgfqpoint{0.781944in}{2.977778in}}{\pgfqpoint{5.019444in}{1.650000in}}%
\pgfusepath{clip}%
\pgfsetrectcap%
\pgfsetroundjoin%
\pgfsetlinewidth{0.803000pt}%
\definecolor{currentstroke}{rgb}{0.690196,0.690196,0.690196}%
\pgfsetstrokecolor{currentstroke}%
\pgfsetstrokeopacity{0.800000}%
\pgfsetdash{}{0pt}%
\pgfpathmoveto{\pgfqpoint{5.550417in}{2.977778in}}%
\pgfpathlineto{\pgfqpoint{5.550417in}{4.627778in}}%
\pgfusepath{stroke}%
\end{pgfscope}%
\begin{pgfscope}%
\pgfsetbuttcap%
\pgfsetroundjoin%
\definecolor{currentfill}{rgb}{0.000000,0.000000,0.000000}%
\pgfsetfillcolor{currentfill}%
\pgfsetlinewidth{0.803000pt}%
\definecolor{currentstroke}{rgb}{0.000000,0.000000,0.000000}%
\pgfsetstrokecolor{currentstroke}%
\pgfsetdash{}{0pt}%
\pgfsys@defobject{currentmarker}{\pgfqpoint{0.000000in}{-0.048611in}}{\pgfqpoint{0.000000in}{0.000000in}}{%
\pgfpathmoveto{\pgfqpoint{0.000000in}{0.000000in}}%
\pgfpathlineto{\pgfqpoint{0.000000in}{-0.048611in}}%
\pgfusepath{stroke,fill}%
}%
\begin{pgfscope}%
\pgfsys@transformshift{5.550417in}{2.977778in}%
\pgfsys@useobject{currentmarker}{}%
\end{pgfscope}%
\end{pgfscope}%
\begin{pgfscope}%
\pgfsetbuttcap%
\pgfsetroundjoin%
\definecolor{currentfill}{rgb}{0.000000,0.000000,0.000000}%
\pgfsetfillcolor{currentfill}%
\pgfsetlinewidth{0.803000pt}%
\definecolor{currentstroke}{rgb}{0.000000,0.000000,0.000000}%
\pgfsetstrokecolor{currentstroke}%
\pgfsetdash{}{0pt}%
\pgfsys@defobject{currentmarker}{\pgfqpoint{0.000000in}{0.000000in}}{\pgfqpoint{0.000000in}{0.048611in}}{%
\pgfpathmoveto{\pgfqpoint{0.000000in}{0.000000in}}%
\pgfpathlineto{\pgfqpoint{0.000000in}{0.048611in}}%
\pgfusepath{stroke,fill}%
}%
\begin{pgfscope}%
\pgfsys@transformshift{5.550417in}{4.627778in}%
\pgfsys@useobject{currentmarker}{}%
\end{pgfscope}%
\end{pgfscope}%
\begin{pgfscope}%
\definecolor{textcolor}{rgb}{0.000000,0.000000,0.000000}%
\pgfsetstrokecolor{textcolor}%
\pgfsetfillcolor{textcolor}%
\pgftext[x=5.550417in,y=2.880556in,,top]{\color{textcolor}\rmfamily\fontsize{10.000000}{12.000000}\selectfont 68}%
\end{pgfscope}%
\begin{pgfscope}%
\pgfpathrectangle{\pgfqpoint{0.781944in}{2.977778in}}{\pgfqpoint{5.019444in}{1.650000in}}%
\pgfusepath{clip}%
\pgfsetrectcap%
\pgfsetroundjoin%
\pgfsetlinewidth{0.803000pt}%
\definecolor{currentstroke}{rgb}{0.690196,0.690196,0.690196}%
\pgfsetstrokecolor{currentstroke}%
\pgfsetstrokeopacity{0.300000}%
\pgfsetdash{}{0pt}%
\pgfpathmoveto{\pgfqpoint{0.781944in}{2.977778in}}%
\pgfpathlineto{\pgfqpoint{0.781944in}{4.627778in}}%
\pgfusepath{stroke}%
\end{pgfscope}%
\begin{pgfscope}%
\pgfsetbuttcap%
\pgfsetroundjoin%
\definecolor{currentfill}{rgb}{0.000000,0.000000,0.000000}%
\pgfsetfillcolor{currentfill}%
\pgfsetlinewidth{0.602250pt}%
\definecolor{currentstroke}{rgb}{0.000000,0.000000,0.000000}%
\pgfsetstrokecolor{currentstroke}%
\pgfsetdash{}{0pt}%
\pgfsys@defobject{currentmarker}{\pgfqpoint{0.000000in}{-0.027778in}}{\pgfqpoint{0.000000in}{0.000000in}}{%
\pgfpathmoveto{\pgfqpoint{0.000000in}{0.000000in}}%
\pgfpathlineto{\pgfqpoint{0.000000in}{-0.027778in}}%
\pgfusepath{stroke,fill}%
}%
\begin{pgfscope}%
\pgfsys@transformshift{0.781944in}{2.977778in}%
\pgfsys@useobject{currentmarker}{}%
\end{pgfscope}%
\end{pgfscope}%
\begin{pgfscope}%
\pgfsetbuttcap%
\pgfsetroundjoin%
\definecolor{currentfill}{rgb}{0.000000,0.000000,0.000000}%
\pgfsetfillcolor{currentfill}%
\pgfsetlinewidth{0.602250pt}%
\definecolor{currentstroke}{rgb}{0.000000,0.000000,0.000000}%
\pgfsetstrokecolor{currentstroke}%
\pgfsetdash{}{0pt}%
\pgfsys@defobject{currentmarker}{\pgfqpoint{0.000000in}{0.000000in}}{\pgfqpoint{0.000000in}{0.027778in}}{%
\pgfpathmoveto{\pgfqpoint{0.000000in}{0.000000in}}%
\pgfpathlineto{\pgfqpoint{0.000000in}{0.027778in}}%
\pgfusepath{stroke,fill}%
}%
\begin{pgfscope}%
\pgfsys@transformshift{0.781944in}{4.627778in}%
\pgfsys@useobject{currentmarker}{}%
\end{pgfscope}%
\end{pgfscope}%
\begin{pgfscope}%
\pgfpathrectangle{\pgfqpoint{0.781944in}{2.977778in}}{\pgfqpoint{5.019444in}{1.650000in}}%
\pgfusepath{clip}%
\pgfsetrectcap%
\pgfsetroundjoin%
\pgfsetlinewidth{0.803000pt}%
\definecolor{currentstroke}{rgb}{0.690196,0.690196,0.690196}%
\pgfsetstrokecolor{currentstroke}%
\pgfsetstrokeopacity{0.300000}%
\pgfsetdash{}{0pt}%
\pgfpathmoveto{\pgfqpoint{0.832139in}{2.977778in}}%
\pgfpathlineto{\pgfqpoint{0.832139in}{4.627778in}}%
\pgfusepath{stroke}%
\end{pgfscope}%
\begin{pgfscope}%
\pgfsetbuttcap%
\pgfsetroundjoin%
\definecolor{currentfill}{rgb}{0.000000,0.000000,0.000000}%
\pgfsetfillcolor{currentfill}%
\pgfsetlinewidth{0.602250pt}%
\definecolor{currentstroke}{rgb}{0.000000,0.000000,0.000000}%
\pgfsetstrokecolor{currentstroke}%
\pgfsetdash{}{0pt}%
\pgfsys@defobject{currentmarker}{\pgfqpoint{0.000000in}{-0.027778in}}{\pgfqpoint{0.000000in}{0.000000in}}{%
\pgfpathmoveto{\pgfqpoint{0.000000in}{0.000000in}}%
\pgfpathlineto{\pgfqpoint{0.000000in}{-0.027778in}}%
\pgfusepath{stroke,fill}%
}%
\begin{pgfscope}%
\pgfsys@transformshift{0.832139in}{2.977778in}%
\pgfsys@useobject{currentmarker}{}%
\end{pgfscope}%
\end{pgfscope}%
\begin{pgfscope}%
\pgfsetbuttcap%
\pgfsetroundjoin%
\definecolor{currentfill}{rgb}{0.000000,0.000000,0.000000}%
\pgfsetfillcolor{currentfill}%
\pgfsetlinewidth{0.602250pt}%
\definecolor{currentstroke}{rgb}{0.000000,0.000000,0.000000}%
\pgfsetstrokecolor{currentstroke}%
\pgfsetdash{}{0pt}%
\pgfsys@defobject{currentmarker}{\pgfqpoint{0.000000in}{0.000000in}}{\pgfqpoint{0.000000in}{0.027778in}}{%
\pgfpathmoveto{\pgfqpoint{0.000000in}{0.000000in}}%
\pgfpathlineto{\pgfqpoint{0.000000in}{0.027778in}}%
\pgfusepath{stroke,fill}%
}%
\begin{pgfscope}%
\pgfsys@transformshift{0.832139in}{4.627778in}%
\pgfsys@useobject{currentmarker}{}%
\end{pgfscope}%
\end{pgfscope}%
\begin{pgfscope}%
\pgfpathrectangle{\pgfqpoint{0.781944in}{2.977778in}}{\pgfqpoint{5.019444in}{1.650000in}}%
\pgfusepath{clip}%
\pgfsetrectcap%
\pgfsetroundjoin%
\pgfsetlinewidth{0.803000pt}%
\definecolor{currentstroke}{rgb}{0.690196,0.690196,0.690196}%
\pgfsetstrokecolor{currentstroke}%
\pgfsetstrokeopacity{0.300000}%
\pgfsetdash{}{0pt}%
\pgfpathmoveto{\pgfqpoint{0.882333in}{2.977778in}}%
\pgfpathlineto{\pgfqpoint{0.882333in}{4.627778in}}%
\pgfusepath{stroke}%
\end{pgfscope}%
\begin{pgfscope}%
\pgfsetbuttcap%
\pgfsetroundjoin%
\definecolor{currentfill}{rgb}{0.000000,0.000000,0.000000}%
\pgfsetfillcolor{currentfill}%
\pgfsetlinewidth{0.602250pt}%
\definecolor{currentstroke}{rgb}{0.000000,0.000000,0.000000}%
\pgfsetstrokecolor{currentstroke}%
\pgfsetdash{}{0pt}%
\pgfsys@defobject{currentmarker}{\pgfqpoint{0.000000in}{-0.027778in}}{\pgfqpoint{0.000000in}{0.000000in}}{%
\pgfpathmoveto{\pgfqpoint{0.000000in}{0.000000in}}%
\pgfpathlineto{\pgfqpoint{0.000000in}{-0.027778in}}%
\pgfusepath{stroke,fill}%
}%
\begin{pgfscope}%
\pgfsys@transformshift{0.882333in}{2.977778in}%
\pgfsys@useobject{currentmarker}{}%
\end{pgfscope}%
\end{pgfscope}%
\begin{pgfscope}%
\pgfsetbuttcap%
\pgfsetroundjoin%
\definecolor{currentfill}{rgb}{0.000000,0.000000,0.000000}%
\pgfsetfillcolor{currentfill}%
\pgfsetlinewidth{0.602250pt}%
\definecolor{currentstroke}{rgb}{0.000000,0.000000,0.000000}%
\pgfsetstrokecolor{currentstroke}%
\pgfsetdash{}{0pt}%
\pgfsys@defobject{currentmarker}{\pgfqpoint{0.000000in}{0.000000in}}{\pgfqpoint{0.000000in}{0.027778in}}{%
\pgfpathmoveto{\pgfqpoint{0.000000in}{0.000000in}}%
\pgfpathlineto{\pgfqpoint{0.000000in}{0.027778in}}%
\pgfusepath{stroke,fill}%
}%
\begin{pgfscope}%
\pgfsys@transformshift{0.882333in}{4.627778in}%
\pgfsys@useobject{currentmarker}{}%
\end{pgfscope}%
\end{pgfscope}%
\begin{pgfscope}%
\pgfpathrectangle{\pgfqpoint{0.781944in}{2.977778in}}{\pgfqpoint{5.019444in}{1.650000in}}%
\pgfusepath{clip}%
\pgfsetrectcap%
\pgfsetroundjoin%
\pgfsetlinewidth{0.803000pt}%
\definecolor{currentstroke}{rgb}{0.690196,0.690196,0.690196}%
\pgfsetstrokecolor{currentstroke}%
\pgfsetstrokeopacity{0.300000}%
\pgfsetdash{}{0pt}%
\pgfpathmoveto{\pgfqpoint{0.932528in}{2.977778in}}%
\pgfpathlineto{\pgfqpoint{0.932528in}{4.627778in}}%
\pgfusepath{stroke}%
\end{pgfscope}%
\begin{pgfscope}%
\pgfsetbuttcap%
\pgfsetroundjoin%
\definecolor{currentfill}{rgb}{0.000000,0.000000,0.000000}%
\pgfsetfillcolor{currentfill}%
\pgfsetlinewidth{0.602250pt}%
\definecolor{currentstroke}{rgb}{0.000000,0.000000,0.000000}%
\pgfsetstrokecolor{currentstroke}%
\pgfsetdash{}{0pt}%
\pgfsys@defobject{currentmarker}{\pgfqpoint{0.000000in}{-0.027778in}}{\pgfqpoint{0.000000in}{0.000000in}}{%
\pgfpathmoveto{\pgfqpoint{0.000000in}{0.000000in}}%
\pgfpathlineto{\pgfqpoint{0.000000in}{-0.027778in}}%
\pgfusepath{stroke,fill}%
}%
\begin{pgfscope}%
\pgfsys@transformshift{0.932528in}{2.977778in}%
\pgfsys@useobject{currentmarker}{}%
\end{pgfscope}%
\end{pgfscope}%
\begin{pgfscope}%
\pgfsetbuttcap%
\pgfsetroundjoin%
\definecolor{currentfill}{rgb}{0.000000,0.000000,0.000000}%
\pgfsetfillcolor{currentfill}%
\pgfsetlinewidth{0.602250pt}%
\definecolor{currentstroke}{rgb}{0.000000,0.000000,0.000000}%
\pgfsetstrokecolor{currentstroke}%
\pgfsetdash{}{0pt}%
\pgfsys@defobject{currentmarker}{\pgfqpoint{0.000000in}{0.000000in}}{\pgfqpoint{0.000000in}{0.027778in}}{%
\pgfpathmoveto{\pgfqpoint{0.000000in}{0.000000in}}%
\pgfpathlineto{\pgfqpoint{0.000000in}{0.027778in}}%
\pgfusepath{stroke,fill}%
}%
\begin{pgfscope}%
\pgfsys@transformshift{0.932528in}{4.627778in}%
\pgfsys@useobject{currentmarker}{}%
\end{pgfscope}%
\end{pgfscope}%
\begin{pgfscope}%
\pgfpathrectangle{\pgfqpoint{0.781944in}{2.977778in}}{\pgfqpoint{5.019444in}{1.650000in}}%
\pgfusepath{clip}%
\pgfsetrectcap%
\pgfsetroundjoin%
\pgfsetlinewidth{0.803000pt}%
\definecolor{currentstroke}{rgb}{0.690196,0.690196,0.690196}%
\pgfsetstrokecolor{currentstroke}%
\pgfsetstrokeopacity{0.300000}%
\pgfsetdash{}{0pt}%
\pgfpathmoveto{\pgfqpoint{0.982722in}{2.977778in}}%
\pgfpathlineto{\pgfqpoint{0.982722in}{4.627778in}}%
\pgfusepath{stroke}%
\end{pgfscope}%
\begin{pgfscope}%
\pgfsetbuttcap%
\pgfsetroundjoin%
\definecolor{currentfill}{rgb}{0.000000,0.000000,0.000000}%
\pgfsetfillcolor{currentfill}%
\pgfsetlinewidth{0.602250pt}%
\definecolor{currentstroke}{rgb}{0.000000,0.000000,0.000000}%
\pgfsetstrokecolor{currentstroke}%
\pgfsetdash{}{0pt}%
\pgfsys@defobject{currentmarker}{\pgfqpoint{0.000000in}{-0.027778in}}{\pgfqpoint{0.000000in}{0.000000in}}{%
\pgfpathmoveto{\pgfqpoint{0.000000in}{0.000000in}}%
\pgfpathlineto{\pgfqpoint{0.000000in}{-0.027778in}}%
\pgfusepath{stroke,fill}%
}%
\begin{pgfscope}%
\pgfsys@transformshift{0.982722in}{2.977778in}%
\pgfsys@useobject{currentmarker}{}%
\end{pgfscope}%
\end{pgfscope}%
\begin{pgfscope}%
\pgfsetbuttcap%
\pgfsetroundjoin%
\definecolor{currentfill}{rgb}{0.000000,0.000000,0.000000}%
\pgfsetfillcolor{currentfill}%
\pgfsetlinewidth{0.602250pt}%
\definecolor{currentstroke}{rgb}{0.000000,0.000000,0.000000}%
\pgfsetstrokecolor{currentstroke}%
\pgfsetdash{}{0pt}%
\pgfsys@defobject{currentmarker}{\pgfqpoint{0.000000in}{0.000000in}}{\pgfqpoint{0.000000in}{0.027778in}}{%
\pgfpathmoveto{\pgfqpoint{0.000000in}{0.000000in}}%
\pgfpathlineto{\pgfqpoint{0.000000in}{0.027778in}}%
\pgfusepath{stroke,fill}%
}%
\begin{pgfscope}%
\pgfsys@transformshift{0.982722in}{4.627778in}%
\pgfsys@useobject{currentmarker}{}%
\end{pgfscope}%
\end{pgfscope}%
\begin{pgfscope}%
\pgfpathrectangle{\pgfqpoint{0.781944in}{2.977778in}}{\pgfqpoint{5.019444in}{1.650000in}}%
\pgfusepath{clip}%
\pgfsetrectcap%
\pgfsetroundjoin%
\pgfsetlinewidth{0.803000pt}%
\definecolor{currentstroke}{rgb}{0.690196,0.690196,0.690196}%
\pgfsetstrokecolor{currentstroke}%
\pgfsetstrokeopacity{0.300000}%
\pgfsetdash{}{0pt}%
\pgfpathmoveto{\pgfqpoint{1.083111in}{2.977778in}}%
\pgfpathlineto{\pgfqpoint{1.083111in}{4.627778in}}%
\pgfusepath{stroke}%
\end{pgfscope}%
\begin{pgfscope}%
\pgfsetbuttcap%
\pgfsetroundjoin%
\definecolor{currentfill}{rgb}{0.000000,0.000000,0.000000}%
\pgfsetfillcolor{currentfill}%
\pgfsetlinewidth{0.602250pt}%
\definecolor{currentstroke}{rgb}{0.000000,0.000000,0.000000}%
\pgfsetstrokecolor{currentstroke}%
\pgfsetdash{}{0pt}%
\pgfsys@defobject{currentmarker}{\pgfqpoint{0.000000in}{-0.027778in}}{\pgfqpoint{0.000000in}{0.000000in}}{%
\pgfpathmoveto{\pgfqpoint{0.000000in}{0.000000in}}%
\pgfpathlineto{\pgfqpoint{0.000000in}{-0.027778in}}%
\pgfusepath{stroke,fill}%
}%
\begin{pgfscope}%
\pgfsys@transformshift{1.083111in}{2.977778in}%
\pgfsys@useobject{currentmarker}{}%
\end{pgfscope}%
\end{pgfscope}%
\begin{pgfscope}%
\pgfsetbuttcap%
\pgfsetroundjoin%
\definecolor{currentfill}{rgb}{0.000000,0.000000,0.000000}%
\pgfsetfillcolor{currentfill}%
\pgfsetlinewidth{0.602250pt}%
\definecolor{currentstroke}{rgb}{0.000000,0.000000,0.000000}%
\pgfsetstrokecolor{currentstroke}%
\pgfsetdash{}{0pt}%
\pgfsys@defobject{currentmarker}{\pgfqpoint{0.000000in}{0.000000in}}{\pgfqpoint{0.000000in}{0.027778in}}{%
\pgfpathmoveto{\pgfqpoint{0.000000in}{0.000000in}}%
\pgfpathlineto{\pgfqpoint{0.000000in}{0.027778in}}%
\pgfusepath{stroke,fill}%
}%
\begin{pgfscope}%
\pgfsys@transformshift{1.083111in}{4.627778in}%
\pgfsys@useobject{currentmarker}{}%
\end{pgfscope}%
\end{pgfscope}%
\begin{pgfscope}%
\pgfpathrectangle{\pgfqpoint{0.781944in}{2.977778in}}{\pgfqpoint{5.019444in}{1.650000in}}%
\pgfusepath{clip}%
\pgfsetrectcap%
\pgfsetroundjoin%
\pgfsetlinewidth{0.803000pt}%
\definecolor{currentstroke}{rgb}{0.690196,0.690196,0.690196}%
\pgfsetstrokecolor{currentstroke}%
\pgfsetstrokeopacity{0.300000}%
\pgfsetdash{}{0pt}%
\pgfpathmoveto{\pgfqpoint{1.133306in}{2.977778in}}%
\pgfpathlineto{\pgfqpoint{1.133306in}{4.627778in}}%
\pgfusepath{stroke}%
\end{pgfscope}%
\begin{pgfscope}%
\pgfsetbuttcap%
\pgfsetroundjoin%
\definecolor{currentfill}{rgb}{0.000000,0.000000,0.000000}%
\pgfsetfillcolor{currentfill}%
\pgfsetlinewidth{0.602250pt}%
\definecolor{currentstroke}{rgb}{0.000000,0.000000,0.000000}%
\pgfsetstrokecolor{currentstroke}%
\pgfsetdash{}{0pt}%
\pgfsys@defobject{currentmarker}{\pgfqpoint{0.000000in}{-0.027778in}}{\pgfqpoint{0.000000in}{0.000000in}}{%
\pgfpathmoveto{\pgfqpoint{0.000000in}{0.000000in}}%
\pgfpathlineto{\pgfqpoint{0.000000in}{-0.027778in}}%
\pgfusepath{stroke,fill}%
}%
\begin{pgfscope}%
\pgfsys@transformshift{1.133306in}{2.977778in}%
\pgfsys@useobject{currentmarker}{}%
\end{pgfscope}%
\end{pgfscope}%
\begin{pgfscope}%
\pgfsetbuttcap%
\pgfsetroundjoin%
\definecolor{currentfill}{rgb}{0.000000,0.000000,0.000000}%
\pgfsetfillcolor{currentfill}%
\pgfsetlinewidth{0.602250pt}%
\definecolor{currentstroke}{rgb}{0.000000,0.000000,0.000000}%
\pgfsetstrokecolor{currentstroke}%
\pgfsetdash{}{0pt}%
\pgfsys@defobject{currentmarker}{\pgfqpoint{0.000000in}{0.000000in}}{\pgfqpoint{0.000000in}{0.027778in}}{%
\pgfpathmoveto{\pgfqpoint{0.000000in}{0.000000in}}%
\pgfpathlineto{\pgfqpoint{0.000000in}{0.027778in}}%
\pgfusepath{stroke,fill}%
}%
\begin{pgfscope}%
\pgfsys@transformshift{1.133306in}{4.627778in}%
\pgfsys@useobject{currentmarker}{}%
\end{pgfscope}%
\end{pgfscope}%
\begin{pgfscope}%
\pgfpathrectangle{\pgfqpoint{0.781944in}{2.977778in}}{\pgfqpoint{5.019444in}{1.650000in}}%
\pgfusepath{clip}%
\pgfsetrectcap%
\pgfsetroundjoin%
\pgfsetlinewidth{0.803000pt}%
\definecolor{currentstroke}{rgb}{0.690196,0.690196,0.690196}%
\pgfsetstrokecolor{currentstroke}%
\pgfsetstrokeopacity{0.300000}%
\pgfsetdash{}{0pt}%
\pgfpathmoveto{\pgfqpoint{1.183500in}{2.977778in}}%
\pgfpathlineto{\pgfqpoint{1.183500in}{4.627778in}}%
\pgfusepath{stroke}%
\end{pgfscope}%
\begin{pgfscope}%
\pgfsetbuttcap%
\pgfsetroundjoin%
\definecolor{currentfill}{rgb}{0.000000,0.000000,0.000000}%
\pgfsetfillcolor{currentfill}%
\pgfsetlinewidth{0.602250pt}%
\definecolor{currentstroke}{rgb}{0.000000,0.000000,0.000000}%
\pgfsetstrokecolor{currentstroke}%
\pgfsetdash{}{0pt}%
\pgfsys@defobject{currentmarker}{\pgfqpoint{0.000000in}{-0.027778in}}{\pgfqpoint{0.000000in}{0.000000in}}{%
\pgfpathmoveto{\pgfqpoint{0.000000in}{0.000000in}}%
\pgfpathlineto{\pgfqpoint{0.000000in}{-0.027778in}}%
\pgfusepath{stroke,fill}%
}%
\begin{pgfscope}%
\pgfsys@transformshift{1.183500in}{2.977778in}%
\pgfsys@useobject{currentmarker}{}%
\end{pgfscope}%
\end{pgfscope}%
\begin{pgfscope}%
\pgfsetbuttcap%
\pgfsetroundjoin%
\definecolor{currentfill}{rgb}{0.000000,0.000000,0.000000}%
\pgfsetfillcolor{currentfill}%
\pgfsetlinewidth{0.602250pt}%
\definecolor{currentstroke}{rgb}{0.000000,0.000000,0.000000}%
\pgfsetstrokecolor{currentstroke}%
\pgfsetdash{}{0pt}%
\pgfsys@defobject{currentmarker}{\pgfqpoint{0.000000in}{0.000000in}}{\pgfqpoint{0.000000in}{0.027778in}}{%
\pgfpathmoveto{\pgfqpoint{0.000000in}{0.000000in}}%
\pgfpathlineto{\pgfqpoint{0.000000in}{0.027778in}}%
\pgfusepath{stroke,fill}%
}%
\begin{pgfscope}%
\pgfsys@transformshift{1.183500in}{4.627778in}%
\pgfsys@useobject{currentmarker}{}%
\end{pgfscope}%
\end{pgfscope}%
\begin{pgfscope}%
\pgfpathrectangle{\pgfqpoint{0.781944in}{2.977778in}}{\pgfqpoint{5.019444in}{1.650000in}}%
\pgfusepath{clip}%
\pgfsetrectcap%
\pgfsetroundjoin%
\pgfsetlinewidth{0.803000pt}%
\definecolor{currentstroke}{rgb}{0.690196,0.690196,0.690196}%
\pgfsetstrokecolor{currentstroke}%
\pgfsetstrokeopacity{0.300000}%
\pgfsetdash{}{0pt}%
\pgfpathmoveto{\pgfqpoint{1.233694in}{2.977778in}}%
\pgfpathlineto{\pgfqpoint{1.233694in}{4.627778in}}%
\pgfusepath{stroke}%
\end{pgfscope}%
\begin{pgfscope}%
\pgfsetbuttcap%
\pgfsetroundjoin%
\definecolor{currentfill}{rgb}{0.000000,0.000000,0.000000}%
\pgfsetfillcolor{currentfill}%
\pgfsetlinewidth{0.602250pt}%
\definecolor{currentstroke}{rgb}{0.000000,0.000000,0.000000}%
\pgfsetstrokecolor{currentstroke}%
\pgfsetdash{}{0pt}%
\pgfsys@defobject{currentmarker}{\pgfqpoint{0.000000in}{-0.027778in}}{\pgfqpoint{0.000000in}{0.000000in}}{%
\pgfpathmoveto{\pgfqpoint{0.000000in}{0.000000in}}%
\pgfpathlineto{\pgfqpoint{0.000000in}{-0.027778in}}%
\pgfusepath{stroke,fill}%
}%
\begin{pgfscope}%
\pgfsys@transformshift{1.233694in}{2.977778in}%
\pgfsys@useobject{currentmarker}{}%
\end{pgfscope}%
\end{pgfscope}%
\begin{pgfscope}%
\pgfsetbuttcap%
\pgfsetroundjoin%
\definecolor{currentfill}{rgb}{0.000000,0.000000,0.000000}%
\pgfsetfillcolor{currentfill}%
\pgfsetlinewidth{0.602250pt}%
\definecolor{currentstroke}{rgb}{0.000000,0.000000,0.000000}%
\pgfsetstrokecolor{currentstroke}%
\pgfsetdash{}{0pt}%
\pgfsys@defobject{currentmarker}{\pgfqpoint{0.000000in}{0.000000in}}{\pgfqpoint{0.000000in}{0.027778in}}{%
\pgfpathmoveto{\pgfqpoint{0.000000in}{0.000000in}}%
\pgfpathlineto{\pgfqpoint{0.000000in}{0.027778in}}%
\pgfusepath{stroke,fill}%
}%
\begin{pgfscope}%
\pgfsys@transformshift{1.233694in}{4.627778in}%
\pgfsys@useobject{currentmarker}{}%
\end{pgfscope}%
\end{pgfscope}%
\begin{pgfscope}%
\pgfpathrectangle{\pgfqpoint{0.781944in}{2.977778in}}{\pgfqpoint{5.019444in}{1.650000in}}%
\pgfusepath{clip}%
\pgfsetrectcap%
\pgfsetroundjoin%
\pgfsetlinewidth{0.803000pt}%
\definecolor{currentstroke}{rgb}{0.690196,0.690196,0.690196}%
\pgfsetstrokecolor{currentstroke}%
\pgfsetstrokeopacity{0.300000}%
\pgfsetdash{}{0pt}%
\pgfpathmoveto{\pgfqpoint{1.283889in}{2.977778in}}%
\pgfpathlineto{\pgfqpoint{1.283889in}{4.627778in}}%
\pgfusepath{stroke}%
\end{pgfscope}%
\begin{pgfscope}%
\pgfsetbuttcap%
\pgfsetroundjoin%
\definecolor{currentfill}{rgb}{0.000000,0.000000,0.000000}%
\pgfsetfillcolor{currentfill}%
\pgfsetlinewidth{0.602250pt}%
\definecolor{currentstroke}{rgb}{0.000000,0.000000,0.000000}%
\pgfsetstrokecolor{currentstroke}%
\pgfsetdash{}{0pt}%
\pgfsys@defobject{currentmarker}{\pgfqpoint{0.000000in}{-0.027778in}}{\pgfqpoint{0.000000in}{0.000000in}}{%
\pgfpathmoveto{\pgfqpoint{0.000000in}{0.000000in}}%
\pgfpathlineto{\pgfqpoint{0.000000in}{-0.027778in}}%
\pgfusepath{stroke,fill}%
}%
\begin{pgfscope}%
\pgfsys@transformshift{1.283889in}{2.977778in}%
\pgfsys@useobject{currentmarker}{}%
\end{pgfscope}%
\end{pgfscope}%
\begin{pgfscope}%
\pgfsetbuttcap%
\pgfsetroundjoin%
\definecolor{currentfill}{rgb}{0.000000,0.000000,0.000000}%
\pgfsetfillcolor{currentfill}%
\pgfsetlinewidth{0.602250pt}%
\definecolor{currentstroke}{rgb}{0.000000,0.000000,0.000000}%
\pgfsetstrokecolor{currentstroke}%
\pgfsetdash{}{0pt}%
\pgfsys@defobject{currentmarker}{\pgfqpoint{0.000000in}{0.000000in}}{\pgfqpoint{0.000000in}{0.027778in}}{%
\pgfpathmoveto{\pgfqpoint{0.000000in}{0.000000in}}%
\pgfpathlineto{\pgfqpoint{0.000000in}{0.027778in}}%
\pgfusepath{stroke,fill}%
}%
\begin{pgfscope}%
\pgfsys@transformshift{1.283889in}{4.627778in}%
\pgfsys@useobject{currentmarker}{}%
\end{pgfscope}%
\end{pgfscope}%
\begin{pgfscope}%
\pgfpathrectangle{\pgfqpoint{0.781944in}{2.977778in}}{\pgfqpoint{5.019444in}{1.650000in}}%
\pgfusepath{clip}%
\pgfsetrectcap%
\pgfsetroundjoin%
\pgfsetlinewidth{0.803000pt}%
\definecolor{currentstroke}{rgb}{0.690196,0.690196,0.690196}%
\pgfsetstrokecolor{currentstroke}%
\pgfsetstrokeopacity{0.300000}%
\pgfsetdash{}{0pt}%
\pgfpathmoveto{\pgfqpoint{1.334083in}{2.977778in}}%
\pgfpathlineto{\pgfqpoint{1.334083in}{4.627778in}}%
\pgfusepath{stroke}%
\end{pgfscope}%
\begin{pgfscope}%
\pgfsetbuttcap%
\pgfsetroundjoin%
\definecolor{currentfill}{rgb}{0.000000,0.000000,0.000000}%
\pgfsetfillcolor{currentfill}%
\pgfsetlinewidth{0.602250pt}%
\definecolor{currentstroke}{rgb}{0.000000,0.000000,0.000000}%
\pgfsetstrokecolor{currentstroke}%
\pgfsetdash{}{0pt}%
\pgfsys@defobject{currentmarker}{\pgfqpoint{0.000000in}{-0.027778in}}{\pgfqpoint{0.000000in}{0.000000in}}{%
\pgfpathmoveto{\pgfqpoint{0.000000in}{0.000000in}}%
\pgfpathlineto{\pgfqpoint{0.000000in}{-0.027778in}}%
\pgfusepath{stroke,fill}%
}%
\begin{pgfscope}%
\pgfsys@transformshift{1.334083in}{2.977778in}%
\pgfsys@useobject{currentmarker}{}%
\end{pgfscope}%
\end{pgfscope}%
\begin{pgfscope}%
\pgfsetbuttcap%
\pgfsetroundjoin%
\definecolor{currentfill}{rgb}{0.000000,0.000000,0.000000}%
\pgfsetfillcolor{currentfill}%
\pgfsetlinewidth{0.602250pt}%
\definecolor{currentstroke}{rgb}{0.000000,0.000000,0.000000}%
\pgfsetstrokecolor{currentstroke}%
\pgfsetdash{}{0pt}%
\pgfsys@defobject{currentmarker}{\pgfqpoint{0.000000in}{0.000000in}}{\pgfqpoint{0.000000in}{0.027778in}}{%
\pgfpathmoveto{\pgfqpoint{0.000000in}{0.000000in}}%
\pgfpathlineto{\pgfqpoint{0.000000in}{0.027778in}}%
\pgfusepath{stroke,fill}%
}%
\begin{pgfscope}%
\pgfsys@transformshift{1.334083in}{4.627778in}%
\pgfsys@useobject{currentmarker}{}%
\end{pgfscope}%
\end{pgfscope}%
\begin{pgfscope}%
\pgfpathrectangle{\pgfqpoint{0.781944in}{2.977778in}}{\pgfqpoint{5.019444in}{1.650000in}}%
\pgfusepath{clip}%
\pgfsetrectcap%
\pgfsetroundjoin%
\pgfsetlinewidth{0.803000pt}%
\definecolor{currentstroke}{rgb}{0.690196,0.690196,0.690196}%
\pgfsetstrokecolor{currentstroke}%
\pgfsetstrokeopacity{0.300000}%
\pgfsetdash{}{0pt}%
\pgfpathmoveto{\pgfqpoint{1.384278in}{2.977778in}}%
\pgfpathlineto{\pgfqpoint{1.384278in}{4.627778in}}%
\pgfusepath{stroke}%
\end{pgfscope}%
\begin{pgfscope}%
\pgfsetbuttcap%
\pgfsetroundjoin%
\definecolor{currentfill}{rgb}{0.000000,0.000000,0.000000}%
\pgfsetfillcolor{currentfill}%
\pgfsetlinewidth{0.602250pt}%
\definecolor{currentstroke}{rgb}{0.000000,0.000000,0.000000}%
\pgfsetstrokecolor{currentstroke}%
\pgfsetdash{}{0pt}%
\pgfsys@defobject{currentmarker}{\pgfqpoint{0.000000in}{-0.027778in}}{\pgfqpoint{0.000000in}{0.000000in}}{%
\pgfpathmoveto{\pgfqpoint{0.000000in}{0.000000in}}%
\pgfpathlineto{\pgfqpoint{0.000000in}{-0.027778in}}%
\pgfusepath{stroke,fill}%
}%
\begin{pgfscope}%
\pgfsys@transformshift{1.384278in}{2.977778in}%
\pgfsys@useobject{currentmarker}{}%
\end{pgfscope}%
\end{pgfscope}%
\begin{pgfscope}%
\pgfsetbuttcap%
\pgfsetroundjoin%
\definecolor{currentfill}{rgb}{0.000000,0.000000,0.000000}%
\pgfsetfillcolor{currentfill}%
\pgfsetlinewidth{0.602250pt}%
\definecolor{currentstroke}{rgb}{0.000000,0.000000,0.000000}%
\pgfsetstrokecolor{currentstroke}%
\pgfsetdash{}{0pt}%
\pgfsys@defobject{currentmarker}{\pgfqpoint{0.000000in}{0.000000in}}{\pgfqpoint{0.000000in}{0.027778in}}{%
\pgfpathmoveto{\pgfqpoint{0.000000in}{0.000000in}}%
\pgfpathlineto{\pgfqpoint{0.000000in}{0.027778in}}%
\pgfusepath{stroke,fill}%
}%
\begin{pgfscope}%
\pgfsys@transformshift{1.384278in}{4.627778in}%
\pgfsys@useobject{currentmarker}{}%
\end{pgfscope}%
\end{pgfscope}%
\begin{pgfscope}%
\pgfpathrectangle{\pgfqpoint{0.781944in}{2.977778in}}{\pgfqpoint{5.019444in}{1.650000in}}%
\pgfusepath{clip}%
\pgfsetrectcap%
\pgfsetroundjoin%
\pgfsetlinewidth{0.803000pt}%
\definecolor{currentstroke}{rgb}{0.690196,0.690196,0.690196}%
\pgfsetstrokecolor{currentstroke}%
\pgfsetstrokeopacity{0.300000}%
\pgfsetdash{}{0pt}%
\pgfpathmoveto{\pgfqpoint{1.434472in}{2.977778in}}%
\pgfpathlineto{\pgfqpoint{1.434472in}{4.627778in}}%
\pgfusepath{stroke}%
\end{pgfscope}%
\begin{pgfscope}%
\pgfsetbuttcap%
\pgfsetroundjoin%
\definecolor{currentfill}{rgb}{0.000000,0.000000,0.000000}%
\pgfsetfillcolor{currentfill}%
\pgfsetlinewidth{0.602250pt}%
\definecolor{currentstroke}{rgb}{0.000000,0.000000,0.000000}%
\pgfsetstrokecolor{currentstroke}%
\pgfsetdash{}{0pt}%
\pgfsys@defobject{currentmarker}{\pgfqpoint{0.000000in}{-0.027778in}}{\pgfqpoint{0.000000in}{0.000000in}}{%
\pgfpathmoveto{\pgfqpoint{0.000000in}{0.000000in}}%
\pgfpathlineto{\pgfqpoint{0.000000in}{-0.027778in}}%
\pgfusepath{stroke,fill}%
}%
\begin{pgfscope}%
\pgfsys@transformshift{1.434472in}{2.977778in}%
\pgfsys@useobject{currentmarker}{}%
\end{pgfscope}%
\end{pgfscope}%
\begin{pgfscope}%
\pgfsetbuttcap%
\pgfsetroundjoin%
\definecolor{currentfill}{rgb}{0.000000,0.000000,0.000000}%
\pgfsetfillcolor{currentfill}%
\pgfsetlinewidth{0.602250pt}%
\definecolor{currentstroke}{rgb}{0.000000,0.000000,0.000000}%
\pgfsetstrokecolor{currentstroke}%
\pgfsetdash{}{0pt}%
\pgfsys@defobject{currentmarker}{\pgfqpoint{0.000000in}{0.000000in}}{\pgfqpoint{0.000000in}{0.027778in}}{%
\pgfpathmoveto{\pgfqpoint{0.000000in}{0.000000in}}%
\pgfpathlineto{\pgfqpoint{0.000000in}{0.027778in}}%
\pgfusepath{stroke,fill}%
}%
\begin{pgfscope}%
\pgfsys@transformshift{1.434472in}{4.627778in}%
\pgfsys@useobject{currentmarker}{}%
\end{pgfscope}%
\end{pgfscope}%
\begin{pgfscope}%
\pgfpathrectangle{\pgfqpoint{0.781944in}{2.977778in}}{\pgfqpoint{5.019444in}{1.650000in}}%
\pgfusepath{clip}%
\pgfsetrectcap%
\pgfsetroundjoin%
\pgfsetlinewidth{0.803000pt}%
\definecolor{currentstroke}{rgb}{0.690196,0.690196,0.690196}%
\pgfsetstrokecolor{currentstroke}%
\pgfsetstrokeopacity{0.300000}%
\pgfsetdash{}{0pt}%
\pgfpathmoveto{\pgfqpoint{1.484667in}{2.977778in}}%
\pgfpathlineto{\pgfqpoint{1.484667in}{4.627778in}}%
\pgfusepath{stroke}%
\end{pgfscope}%
\begin{pgfscope}%
\pgfsetbuttcap%
\pgfsetroundjoin%
\definecolor{currentfill}{rgb}{0.000000,0.000000,0.000000}%
\pgfsetfillcolor{currentfill}%
\pgfsetlinewidth{0.602250pt}%
\definecolor{currentstroke}{rgb}{0.000000,0.000000,0.000000}%
\pgfsetstrokecolor{currentstroke}%
\pgfsetdash{}{0pt}%
\pgfsys@defobject{currentmarker}{\pgfqpoint{0.000000in}{-0.027778in}}{\pgfqpoint{0.000000in}{0.000000in}}{%
\pgfpathmoveto{\pgfqpoint{0.000000in}{0.000000in}}%
\pgfpathlineto{\pgfqpoint{0.000000in}{-0.027778in}}%
\pgfusepath{stroke,fill}%
}%
\begin{pgfscope}%
\pgfsys@transformshift{1.484667in}{2.977778in}%
\pgfsys@useobject{currentmarker}{}%
\end{pgfscope}%
\end{pgfscope}%
\begin{pgfscope}%
\pgfsetbuttcap%
\pgfsetroundjoin%
\definecolor{currentfill}{rgb}{0.000000,0.000000,0.000000}%
\pgfsetfillcolor{currentfill}%
\pgfsetlinewidth{0.602250pt}%
\definecolor{currentstroke}{rgb}{0.000000,0.000000,0.000000}%
\pgfsetstrokecolor{currentstroke}%
\pgfsetdash{}{0pt}%
\pgfsys@defobject{currentmarker}{\pgfqpoint{0.000000in}{0.000000in}}{\pgfqpoint{0.000000in}{0.027778in}}{%
\pgfpathmoveto{\pgfqpoint{0.000000in}{0.000000in}}%
\pgfpathlineto{\pgfqpoint{0.000000in}{0.027778in}}%
\pgfusepath{stroke,fill}%
}%
\begin{pgfscope}%
\pgfsys@transformshift{1.484667in}{4.627778in}%
\pgfsys@useobject{currentmarker}{}%
\end{pgfscope}%
\end{pgfscope}%
\begin{pgfscope}%
\pgfpathrectangle{\pgfqpoint{0.781944in}{2.977778in}}{\pgfqpoint{5.019444in}{1.650000in}}%
\pgfusepath{clip}%
\pgfsetrectcap%
\pgfsetroundjoin%
\pgfsetlinewidth{0.803000pt}%
\definecolor{currentstroke}{rgb}{0.690196,0.690196,0.690196}%
\pgfsetstrokecolor{currentstroke}%
\pgfsetstrokeopacity{0.300000}%
\pgfsetdash{}{0pt}%
\pgfpathmoveto{\pgfqpoint{1.585056in}{2.977778in}}%
\pgfpathlineto{\pgfqpoint{1.585056in}{4.627778in}}%
\pgfusepath{stroke}%
\end{pgfscope}%
\begin{pgfscope}%
\pgfsetbuttcap%
\pgfsetroundjoin%
\definecolor{currentfill}{rgb}{0.000000,0.000000,0.000000}%
\pgfsetfillcolor{currentfill}%
\pgfsetlinewidth{0.602250pt}%
\definecolor{currentstroke}{rgb}{0.000000,0.000000,0.000000}%
\pgfsetstrokecolor{currentstroke}%
\pgfsetdash{}{0pt}%
\pgfsys@defobject{currentmarker}{\pgfqpoint{0.000000in}{-0.027778in}}{\pgfqpoint{0.000000in}{0.000000in}}{%
\pgfpathmoveto{\pgfqpoint{0.000000in}{0.000000in}}%
\pgfpathlineto{\pgfqpoint{0.000000in}{-0.027778in}}%
\pgfusepath{stroke,fill}%
}%
\begin{pgfscope}%
\pgfsys@transformshift{1.585056in}{2.977778in}%
\pgfsys@useobject{currentmarker}{}%
\end{pgfscope}%
\end{pgfscope}%
\begin{pgfscope}%
\pgfsetbuttcap%
\pgfsetroundjoin%
\definecolor{currentfill}{rgb}{0.000000,0.000000,0.000000}%
\pgfsetfillcolor{currentfill}%
\pgfsetlinewidth{0.602250pt}%
\definecolor{currentstroke}{rgb}{0.000000,0.000000,0.000000}%
\pgfsetstrokecolor{currentstroke}%
\pgfsetdash{}{0pt}%
\pgfsys@defobject{currentmarker}{\pgfqpoint{0.000000in}{0.000000in}}{\pgfqpoint{0.000000in}{0.027778in}}{%
\pgfpathmoveto{\pgfqpoint{0.000000in}{0.000000in}}%
\pgfpathlineto{\pgfqpoint{0.000000in}{0.027778in}}%
\pgfusepath{stroke,fill}%
}%
\begin{pgfscope}%
\pgfsys@transformshift{1.585056in}{4.627778in}%
\pgfsys@useobject{currentmarker}{}%
\end{pgfscope}%
\end{pgfscope}%
\begin{pgfscope}%
\pgfpathrectangle{\pgfqpoint{0.781944in}{2.977778in}}{\pgfqpoint{5.019444in}{1.650000in}}%
\pgfusepath{clip}%
\pgfsetrectcap%
\pgfsetroundjoin%
\pgfsetlinewidth{0.803000pt}%
\definecolor{currentstroke}{rgb}{0.690196,0.690196,0.690196}%
\pgfsetstrokecolor{currentstroke}%
\pgfsetstrokeopacity{0.300000}%
\pgfsetdash{}{0pt}%
\pgfpathmoveto{\pgfqpoint{1.635250in}{2.977778in}}%
\pgfpathlineto{\pgfqpoint{1.635250in}{4.627778in}}%
\pgfusepath{stroke}%
\end{pgfscope}%
\begin{pgfscope}%
\pgfsetbuttcap%
\pgfsetroundjoin%
\definecolor{currentfill}{rgb}{0.000000,0.000000,0.000000}%
\pgfsetfillcolor{currentfill}%
\pgfsetlinewidth{0.602250pt}%
\definecolor{currentstroke}{rgb}{0.000000,0.000000,0.000000}%
\pgfsetstrokecolor{currentstroke}%
\pgfsetdash{}{0pt}%
\pgfsys@defobject{currentmarker}{\pgfqpoint{0.000000in}{-0.027778in}}{\pgfqpoint{0.000000in}{0.000000in}}{%
\pgfpathmoveto{\pgfqpoint{0.000000in}{0.000000in}}%
\pgfpathlineto{\pgfqpoint{0.000000in}{-0.027778in}}%
\pgfusepath{stroke,fill}%
}%
\begin{pgfscope}%
\pgfsys@transformshift{1.635250in}{2.977778in}%
\pgfsys@useobject{currentmarker}{}%
\end{pgfscope}%
\end{pgfscope}%
\begin{pgfscope}%
\pgfsetbuttcap%
\pgfsetroundjoin%
\definecolor{currentfill}{rgb}{0.000000,0.000000,0.000000}%
\pgfsetfillcolor{currentfill}%
\pgfsetlinewidth{0.602250pt}%
\definecolor{currentstroke}{rgb}{0.000000,0.000000,0.000000}%
\pgfsetstrokecolor{currentstroke}%
\pgfsetdash{}{0pt}%
\pgfsys@defobject{currentmarker}{\pgfqpoint{0.000000in}{0.000000in}}{\pgfqpoint{0.000000in}{0.027778in}}{%
\pgfpathmoveto{\pgfqpoint{0.000000in}{0.000000in}}%
\pgfpathlineto{\pgfqpoint{0.000000in}{0.027778in}}%
\pgfusepath{stroke,fill}%
}%
\begin{pgfscope}%
\pgfsys@transformshift{1.635250in}{4.627778in}%
\pgfsys@useobject{currentmarker}{}%
\end{pgfscope}%
\end{pgfscope}%
\begin{pgfscope}%
\pgfpathrectangle{\pgfqpoint{0.781944in}{2.977778in}}{\pgfqpoint{5.019444in}{1.650000in}}%
\pgfusepath{clip}%
\pgfsetrectcap%
\pgfsetroundjoin%
\pgfsetlinewidth{0.803000pt}%
\definecolor{currentstroke}{rgb}{0.690196,0.690196,0.690196}%
\pgfsetstrokecolor{currentstroke}%
\pgfsetstrokeopacity{0.300000}%
\pgfsetdash{}{0pt}%
\pgfpathmoveto{\pgfqpoint{1.685444in}{2.977778in}}%
\pgfpathlineto{\pgfqpoint{1.685444in}{4.627778in}}%
\pgfusepath{stroke}%
\end{pgfscope}%
\begin{pgfscope}%
\pgfsetbuttcap%
\pgfsetroundjoin%
\definecolor{currentfill}{rgb}{0.000000,0.000000,0.000000}%
\pgfsetfillcolor{currentfill}%
\pgfsetlinewidth{0.602250pt}%
\definecolor{currentstroke}{rgb}{0.000000,0.000000,0.000000}%
\pgfsetstrokecolor{currentstroke}%
\pgfsetdash{}{0pt}%
\pgfsys@defobject{currentmarker}{\pgfqpoint{0.000000in}{-0.027778in}}{\pgfqpoint{0.000000in}{0.000000in}}{%
\pgfpathmoveto{\pgfqpoint{0.000000in}{0.000000in}}%
\pgfpathlineto{\pgfqpoint{0.000000in}{-0.027778in}}%
\pgfusepath{stroke,fill}%
}%
\begin{pgfscope}%
\pgfsys@transformshift{1.685444in}{2.977778in}%
\pgfsys@useobject{currentmarker}{}%
\end{pgfscope}%
\end{pgfscope}%
\begin{pgfscope}%
\pgfsetbuttcap%
\pgfsetroundjoin%
\definecolor{currentfill}{rgb}{0.000000,0.000000,0.000000}%
\pgfsetfillcolor{currentfill}%
\pgfsetlinewidth{0.602250pt}%
\definecolor{currentstroke}{rgb}{0.000000,0.000000,0.000000}%
\pgfsetstrokecolor{currentstroke}%
\pgfsetdash{}{0pt}%
\pgfsys@defobject{currentmarker}{\pgfqpoint{0.000000in}{0.000000in}}{\pgfqpoint{0.000000in}{0.027778in}}{%
\pgfpathmoveto{\pgfqpoint{0.000000in}{0.000000in}}%
\pgfpathlineto{\pgfqpoint{0.000000in}{0.027778in}}%
\pgfusepath{stroke,fill}%
}%
\begin{pgfscope}%
\pgfsys@transformshift{1.685444in}{4.627778in}%
\pgfsys@useobject{currentmarker}{}%
\end{pgfscope}%
\end{pgfscope}%
\begin{pgfscope}%
\pgfpathrectangle{\pgfqpoint{0.781944in}{2.977778in}}{\pgfqpoint{5.019444in}{1.650000in}}%
\pgfusepath{clip}%
\pgfsetrectcap%
\pgfsetroundjoin%
\pgfsetlinewidth{0.803000pt}%
\definecolor{currentstroke}{rgb}{0.690196,0.690196,0.690196}%
\pgfsetstrokecolor{currentstroke}%
\pgfsetstrokeopacity{0.300000}%
\pgfsetdash{}{0pt}%
\pgfpathmoveto{\pgfqpoint{1.735639in}{2.977778in}}%
\pgfpathlineto{\pgfqpoint{1.735639in}{4.627778in}}%
\pgfusepath{stroke}%
\end{pgfscope}%
\begin{pgfscope}%
\pgfsetbuttcap%
\pgfsetroundjoin%
\definecolor{currentfill}{rgb}{0.000000,0.000000,0.000000}%
\pgfsetfillcolor{currentfill}%
\pgfsetlinewidth{0.602250pt}%
\definecolor{currentstroke}{rgb}{0.000000,0.000000,0.000000}%
\pgfsetstrokecolor{currentstroke}%
\pgfsetdash{}{0pt}%
\pgfsys@defobject{currentmarker}{\pgfqpoint{0.000000in}{-0.027778in}}{\pgfqpoint{0.000000in}{0.000000in}}{%
\pgfpathmoveto{\pgfqpoint{0.000000in}{0.000000in}}%
\pgfpathlineto{\pgfqpoint{0.000000in}{-0.027778in}}%
\pgfusepath{stroke,fill}%
}%
\begin{pgfscope}%
\pgfsys@transformshift{1.735639in}{2.977778in}%
\pgfsys@useobject{currentmarker}{}%
\end{pgfscope}%
\end{pgfscope}%
\begin{pgfscope}%
\pgfsetbuttcap%
\pgfsetroundjoin%
\definecolor{currentfill}{rgb}{0.000000,0.000000,0.000000}%
\pgfsetfillcolor{currentfill}%
\pgfsetlinewidth{0.602250pt}%
\definecolor{currentstroke}{rgb}{0.000000,0.000000,0.000000}%
\pgfsetstrokecolor{currentstroke}%
\pgfsetdash{}{0pt}%
\pgfsys@defobject{currentmarker}{\pgfqpoint{0.000000in}{0.000000in}}{\pgfqpoint{0.000000in}{0.027778in}}{%
\pgfpathmoveto{\pgfqpoint{0.000000in}{0.000000in}}%
\pgfpathlineto{\pgfqpoint{0.000000in}{0.027778in}}%
\pgfusepath{stroke,fill}%
}%
\begin{pgfscope}%
\pgfsys@transformshift{1.735639in}{4.627778in}%
\pgfsys@useobject{currentmarker}{}%
\end{pgfscope}%
\end{pgfscope}%
\begin{pgfscope}%
\pgfpathrectangle{\pgfqpoint{0.781944in}{2.977778in}}{\pgfqpoint{5.019444in}{1.650000in}}%
\pgfusepath{clip}%
\pgfsetrectcap%
\pgfsetroundjoin%
\pgfsetlinewidth{0.803000pt}%
\definecolor{currentstroke}{rgb}{0.690196,0.690196,0.690196}%
\pgfsetstrokecolor{currentstroke}%
\pgfsetstrokeopacity{0.300000}%
\pgfsetdash{}{0pt}%
\pgfpathmoveto{\pgfqpoint{1.785833in}{2.977778in}}%
\pgfpathlineto{\pgfqpoint{1.785833in}{4.627778in}}%
\pgfusepath{stroke}%
\end{pgfscope}%
\begin{pgfscope}%
\pgfsetbuttcap%
\pgfsetroundjoin%
\definecolor{currentfill}{rgb}{0.000000,0.000000,0.000000}%
\pgfsetfillcolor{currentfill}%
\pgfsetlinewidth{0.602250pt}%
\definecolor{currentstroke}{rgb}{0.000000,0.000000,0.000000}%
\pgfsetstrokecolor{currentstroke}%
\pgfsetdash{}{0pt}%
\pgfsys@defobject{currentmarker}{\pgfqpoint{0.000000in}{-0.027778in}}{\pgfqpoint{0.000000in}{0.000000in}}{%
\pgfpathmoveto{\pgfqpoint{0.000000in}{0.000000in}}%
\pgfpathlineto{\pgfqpoint{0.000000in}{-0.027778in}}%
\pgfusepath{stroke,fill}%
}%
\begin{pgfscope}%
\pgfsys@transformshift{1.785833in}{2.977778in}%
\pgfsys@useobject{currentmarker}{}%
\end{pgfscope}%
\end{pgfscope}%
\begin{pgfscope}%
\pgfsetbuttcap%
\pgfsetroundjoin%
\definecolor{currentfill}{rgb}{0.000000,0.000000,0.000000}%
\pgfsetfillcolor{currentfill}%
\pgfsetlinewidth{0.602250pt}%
\definecolor{currentstroke}{rgb}{0.000000,0.000000,0.000000}%
\pgfsetstrokecolor{currentstroke}%
\pgfsetdash{}{0pt}%
\pgfsys@defobject{currentmarker}{\pgfqpoint{0.000000in}{0.000000in}}{\pgfqpoint{0.000000in}{0.027778in}}{%
\pgfpathmoveto{\pgfqpoint{0.000000in}{0.000000in}}%
\pgfpathlineto{\pgfqpoint{0.000000in}{0.027778in}}%
\pgfusepath{stroke,fill}%
}%
\begin{pgfscope}%
\pgfsys@transformshift{1.785833in}{4.627778in}%
\pgfsys@useobject{currentmarker}{}%
\end{pgfscope}%
\end{pgfscope}%
\begin{pgfscope}%
\pgfpathrectangle{\pgfqpoint{0.781944in}{2.977778in}}{\pgfqpoint{5.019444in}{1.650000in}}%
\pgfusepath{clip}%
\pgfsetrectcap%
\pgfsetroundjoin%
\pgfsetlinewidth{0.803000pt}%
\definecolor{currentstroke}{rgb}{0.690196,0.690196,0.690196}%
\pgfsetstrokecolor{currentstroke}%
\pgfsetstrokeopacity{0.300000}%
\pgfsetdash{}{0pt}%
\pgfpathmoveto{\pgfqpoint{1.836028in}{2.977778in}}%
\pgfpathlineto{\pgfqpoint{1.836028in}{4.627778in}}%
\pgfusepath{stroke}%
\end{pgfscope}%
\begin{pgfscope}%
\pgfsetbuttcap%
\pgfsetroundjoin%
\definecolor{currentfill}{rgb}{0.000000,0.000000,0.000000}%
\pgfsetfillcolor{currentfill}%
\pgfsetlinewidth{0.602250pt}%
\definecolor{currentstroke}{rgb}{0.000000,0.000000,0.000000}%
\pgfsetstrokecolor{currentstroke}%
\pgfsetdash{}{0pt}%
\pgfsys@defobject{currentmarker}{\pgfqpoint{0.000000in}{-0.027778in}}{\pgfqpoint{0.000000in}{0.000000in}}{%
\pgfpathmoveto{\pgfqpoint{0.000000in}{0.000000in}}%
\pgfpathlineto{\pgfqpoint{0.000000in}{-0.027778in}}%
\pgfusepath{stroke,fill}%
}%
\begin{pgfscope}%
\pgfsys@transformshift{1.836028in}{2.977778in}%
\pgfsys@useobject{currentmarker}{}%
\end{pgfscope}%
\end{pgfscope}%
\begin{pgfscope}%
\pgfsetbuttcap%
\pgfsetroundjoin%
\definecolor{currentfill}{rgb}{0.000000,0.000000,0.000000}%
\pgfsetfillcolor{currentfill}%
\pgfsetlinewidth{0.602250pt}%
\definecolor{currentstroke}{rgb}{0.000000,0.000000,0.000000}%
\pgfsetstrokecolor{currentstroke}%
\pgfsetdash{}{0pt}%
\pgfsys@defobject{currentmarker}{\pgfqpoint{0.000000in}{0.000000in}}{\pgfqpoint{0.000000in}{0.027778in}}{%
\pgfpathmoveto{\pgfqpoint{0.000000in}{0.000000in}}%
\pgfpathlineto{\pgfqpoint{0.000000in}{0.027778in}}%
\pgfusepath{stroke,fill}%
}%
\begin{pgfscope}%
\pgfsys@transformshift{1.836028in}{4.627778in}%
\pgfsys@useobject{currentmarker}{}%
\end{pgfscope}%
\end{pgfscope}%
\begin{pgfscope}%
\pgfpathrectangle{\pgfqpoint{0.781944in}{2.977778in}}{\pgfqpoint{5.019444in}{1.650000in}}%
\pgfusepath{clip}%
\pgfsetrectcap%
\pgfsetroundjoin%
\pgfsetlinewidth{0.803000pt}%
\definecolor{currentstroke}{rgb}{0.690196,0.690196,0.690196}%
\pgfsetstrokecolor{currentstroke}%
\pgfsetstrokeopacity{0.300000}%
\pgfsetdash{}{0pt}%
\pgfpathmoveto{\pgfqpoint{1.886222in}{2.977778in}}%
\pgfpathlineto{\pgfqpoint{1.886222in}{4.627778in}}%
\pgfusepath{stroke}%
\end{pgfscope}%
\begin{pgfscope}%
\pgfsetbuttcap%
\pgfsetroundjoin%
\definecolor{currentfill}{rgb}{0.000000,0.000000,0.000000}%
\pgfsetfillcolor{currentfill}%
\pgfsetlinewidth{0.602250pt}%
\definecolor{currentstroke}{rgb}{0.000000,0.000000,0.000000}%
\pgfsetstrokecolor{currentstroke}%
\pgfsetdash{}{0pt}%
\pgfsys@defobject{currentmarker}{\pgfqpoint{0.000000in}{-0.027778in}}{\pgfqpoint{0.000000in}{0.000000in}}{%
\pgfpathmoveto{\pgfqpoint{0.000000in}{0.000000in}}%
\pgfpathlineto{\pgfqpoint{0.000000in}{-0.027778in}}%
\pgfusepath{stroke,fill}%
}%
\begin{pgfscope}%
\pgfsys@transformshift{1.886222in}{2.977778in}%
\pgfsys@useobject{currentmarker}{}%
\end{pgfscope}%
\end{pgfscope}%
\begin{pgfscope}%
\pgfsetbuttcap%
\pgfsetroundjoin%
\definecolor{currentfill}{rgb}{0.000000,0.000000,0.000000}%
\pgfsetfillcolor{currentfill}%
\pgfsetlinewidth{0.602250pt}%
\definecolor{currentstroke}{rgb}{0.000000,0.000000,0.000000}%
\pgfsetstrokecolor{currentstroke}%
\pgfsetdash{}{0pt}%
\pgfsys@defobject{currentmarker}{\pgfqpoint{0.000000in}{0.000000in}}{\pgfqpoint{0.000000in}{0.027778in}}{%
\pgfpathmoveto{\pgfqpoint{0.000000in}{0.000000in}}%
\pgfpathlineto{\pgfqpoint{0.000000in}{0.027778in}}%
\pgfusepath{stroke,fill}%
}%
\begin{pgfscope}%
\pgfsys@transformshift{1.886222in}{4.627778in}%
\pgfsys@useobject{currentmarker}{}%
\end{pgfscope}%
\end{pgfscope}%
\begin{pgfscope}%
\pgfpathrectangle{\pgfqpoint{0.781944in}{2.977778in}}{\pgfqpoint{5.019444in}{1.650000in}}%
\pgfusepath{clip}%
\pgfsetrectcap%
\pgfsetroundjoin%
\pgfsetlinewidth{0.803000pt}%
\definecolor{currentstroke}{rgb}{0.690196,0.690196,0.690196}%
\pgfsetstrokecolor{currentstroke}%
\pgfsetstrokeopacity{0.300000}%
\pgfsetdash{}{0pt}%
\pgfpathmoveto{\pgfqpoint{1.936417in}{2.977778in}}%
\pgfpathlineto{\pgfqpoint{1.936417in}{4.627778in}}%
\pgfusepath{stroke}%
\end{pgfscope}%
\begin{pgfscope}%
\pgfsetbuttcap%
\pgfsetroundjoin%
\definecolor{currentfill}{rgb}{0.000000,0.000000,0.000000}%
\pgfsetfillcolor{currentfill}%
\pgfsetlinewidth{0.602250pt}%
\definecolor{currentstroke}{rgb}{0.000000,0.000000,0.000000}%
\pgfsetstrokecolor{currentstroke}%
\pgfsetdash{}{0pt}%
\pgfsys@defobject{currentmarker}{\pgfqpoint{0.000000in}{-0.027778in}}{\pgfqpoint{0.000000in}{0.000000in}}{%
\pgfpathmoveto{\pgfqpoint{0.000000in}{0.000000in}}%
\pgfpathlineto{\pgfqpoint{0.000000in}{-0.027778in}}%
\pgfusepath{stroke,fill}%
}%
\begin{pgfscope}%
\pgfsys@transformshift{1.936417in}{2.977778in}%
\pgfsys@useobject{currentmarker}{}%
\end{pgfscope}%
\end{pgfscope}%
\begin{pgfscope}%
\pgfsetbuttcap%
\pgfsetroundjoin%
\definecolor{currentfill}{rgb}{0.000000,0.000000,0.000000}%
\pgfsetfillcolor{currentfill}%
\pgfsetlinewidth{0.602250pt}%
\definecolor{currentstroke}{rgb}{0.000000,0.000000,0.000000}%
\pgfsetstrokecolor{currentstroke}%
\pgfsetdash{}{0pt}%
\pgfsys@defobject{currentmarker}{\pgfqpoint{0.000000in}{0.000000in}}{\pgfqpoint{0.000000in}{0.027778in}}{%
\pgfpathmoveto{\pgfqpoint{0.000000in}{0.000000in}}%
\pgfpathlineto{\pgfqpoint{0.000000in}{0.027778in}}%
\pgfusepath{stroke,fill}%
}%
\begin{pgfscope}%
\pgfsys@transformshift{1.936417in}{4.627778in}%
\pgfsys@useobject{currentmarker}{}%
\end{pgfscope}%
\end{pgfscope}%
\begin{pgfscope}%
\pgfpathrectangle{\pgfqpoint{0.781944in}{2.977778in}}{\pgfqpoint{5.019444in}{1.650000in}}%
\pgfusepath{clip}%
\pgfsetrectcap%
\pgfsetroundjoin%
\pgfsetlinewidth{0.803000pt}%
\definecolor{currentstroke}{rgb}{0.690196,0.690196,0.690196}%
\pgfsetstrokecolor{currentstroke}%
\pgfsetstrokeopacity{0.300000}%
\pgfsetdash{}{0pt}%
\pgfpathmoveto{\pgfqpoint{1.986611in}{2.977778in}}%
\pgfpathlineto{\pgfqpoint{1.986611in}{4.627778in}}%
\pgfusepath{stroke}%
\end{pgfscope}%
\begin{pgfscope}%
\pgfsetbuttcap%
\pgfsetroundjoin%
\definecolor{currentfill}{rgb}{0.000000,0.000000,0.000000}%
\pgfsetfillcolor{currentfill}%
\pgfsetlinewidth{0.602250pt}%
\definecolor{currentstroke}{rgb}{0.000000,0.000000,0.000000}%
\pgfsetstrokecolor{currentstroke}%
\pgfsetdash{}{0pt}%
\pgfsys@defobject{currentmarker}{\pgfqpoint{0.000000in}{-0.027778in}}{\pgfqpoint{0.000000in}{0.000000in}}{%
\pgfpathmoveto{\pgfqpoint{0.000000in}{0.000000in}}%
\pgfpathlineto{\pgfqpoint{0.000000in}{-0.027778in}}%
\pgfusepath{stroke,fill}%
}%
\begin{pgfscope}%
\pgfsys@transformshift{1.986611in}{2.977778in}%
\pgfsys@useobject{currentmarker}{}%
\end{pgfscope}%
\end{pgfscope}%
\begin{pgfscope}%
\pgfsetbuttcap%
\pgfsetroundjoin%
\definecolor{currentfill}{rgb}{0.000000,0.000000,0.000000}%
\pgfsetfillcolor{currentfill}%
\pgfsetlinewidth{0.602250pt}%
\definecolor{currentstroke}{rgb}{0.000000,0.000000,0.000000}%
\pgfsetstrokecolor{currentstroke}%
\pgfsetdash{}{0pt}%
\pgfsys@defobject{currentmarker}{\pgfqpoint{0.000000in}{0.000000in}}{\pgfqpoint{0.000000in}{0.027778in}}{%
\pgfpathmoveto{\pgfqpoint{0.000000in}{0.000000in}}%
\pgfpathlineto{\pgfqpoint{0.000000in}{0.027778in}}%
\pgfusepath{stroke,fill}%
}%
\begin{pgfscope}%
\pgfsys@transformshift{1.986611in}{4.627778in}%
\pgfsys@useobject{currentmarker}{}%
\end{pgfscope}%
\end{pgfscope}%
\begin{pgfscope}%
\pgfpathrectangle{\pgfqpoint{0.781944in}{2.977778in}}{\pgfqpoint{5.019444in}{1.650000in}}%
\pgfusepath{clip}%
\pgfsetrectcap%
\pgfsetroundjoin%
\pgfsetlinewidth{0.803000pt}%
\definecolor{currentstroke}{rgb}{0.690196,0.690196,0.690196}%
\pgfsetstrokecolor{currentstroke}%
\pgfsetstrokeopacity{0.300000}%
\pgfsetdash{}{0pt}%
\pgfpathmoveto{\pgfqpoint{2.087000in}{2.977778in}}%
\pgfpathlineto{\pgfqpoint{2.087000in}{4.627778in}}%
\pgfusepath{stroke}%
\end{pgfscope}%
\begin{pgfscope}%
\pgfsetbuttcap%
\pgfsetroundjoin%
\definecolor{currentfill}{rgb}{0.000000,0.000000,0.000000}%
\pgfsetfillcolor{currentfill}%
\pgfsetlinewidth{0.602250pt}%
\definecolor{currentstroke}{rgb}{0.000000,0.000000,0.000000}%
\pgfsetstrokecolor{currentstroke}%
\pgfsetdash{}{0pt}%
\pgfsys@defobject{currentmarker}{\pgfqpoint{0.000000in}{-0.027778in}}{\pgfqpoint{0.000000in}{0.000000in}}{%
\pgfpathmoveto{\pgfqpoint{0.000000in}{0.000000in}}%
\pgfpathlineto{\pgfqpoint{0.000000in}{-0.027778in}}%
\pgfusepath{stroke,fill}%
}%
\begin{pgfscope}%
\pgfsys@transformshift{2.087000in}{2.977778in}%
\pgfsys@useobject{currentmarker}{}%
\end{pgfscope}%
\end{pgfscope}%
\begin{pgfscope}%
\pgfsetbuttcap%
\pgfsetroundjoin%
\definecolor{currentfill}{rgb}{0.000000,0.000000,0.000000}%
\pgfsetfillcolor{currentfill}%
\pgfsetlinewidth{0.602250pt}%
\definecolor{currentstroke}{rgb}{0.000000,0.000000,0.000000}%
\pgfsetstrokecolor{currentstroke}%
\pgfsetdash{}{0pt}%
\pgfsys@defobject{currentmarker}{\pgfqpoint{0.000000in}{0.000000in}}{\pgfqpoint{0.000000in}{0.027778in}}{%
\pgfpathmoveto{\pgfqpoint{0.000000in}{0.000000in}}%
\pgfpathlineto{\pgfqpoint{0.000000in}{0.027778in}}%
\pgfusepath{stroke,fill}%
}%
\begin{pgfscope}%
\pgfsys@transformshift{2.087000in}{4.627778in}%
\pgfsys@useobject{currentmarker}{}%
\end{pgfscope}%
\end{pgfscope}%
\begin{pgfscope}%
\pgfpathrectangle{\pgfqpoint{0.781944in}{2.977778in}}{\pgfqpoint{5.019444in}{1.650000in}}%
\pgfusepath{clip}%
\pgfsetrectcap%
\pgfsetroundjoin%
\pgfsetlinewidth{0.803000pt}%
\definecolor{currentstroke}{rgb}{0.690196,0.690196,0.690196}%
\pgfsetstrokecolor{currentstroke}%
\pgfsetstrokeopacity{0.300000}%
\pgfsetdash{}{0pt}%
\pgfpathmoveto{\pgfqpoint{2.137194in}{2.977778in}}%
\pgfpathlineto{\pgfqpoint{2.137194in}{4.627778in}}%
\pgfusepath{stroke}%
\end{pgfscope}%
\begin{pgfscope}%
\pgfsetbuttcap%
\pgfsetroundjoin%
\definecolor{currentfill}{rgb}{0.000000,0.000000,0.000000}%
\pgfsetfillcolor{currentfill}%
\pgfsetlinewidth{0.602250pt}%
\definecolor{currentstroke}{rgb}{0.000000,0.000000,0.000000}%
\pgfsetstrokecolor{currentstroke}%
\pgfsetdash{}{0pt}%
\pgfsys@defobject{currentmarker}{\pgfqpoint{0.000000in}{-0.027778in}}{\pgfqpoint{0.000000in}{0.000000in}}{%
\pgfpathmoveto{\pgfqpoint{0.000000in}{0.000000in}}%
\pgfpathlineto{\pgfqpoint{0.000000in}{-0.027778in}}%
\pgfusepath{stroke,fill}%
}%
\begin{pgfscope}%
\pgfsys@transformshift{2.137194in}{2.977778in}%
\pgfsys@useobject{currentmarker}{}%
\end{pgfscope}%
\end{pgfscope}%
\begin{pgfscope}%
\pgfsetbuttcap%
\pgfsetroundjoin%
\definecolor{currentfill}{rgb}{0.000000,0.000000,0.000000}%
\pgfsetfillcolor{currentfill}%
\pgfsetlinewidth{0.602250pt}%
\definecolor{currentstroke}{rgb}{0.000000,0.000000,0.000000}%
\pgfsetstrokecolor{currentstroke}%
\pgfsetdash{}{0pt}%
\pgfsys@defobject{currentmarker}{\pgfqpoint{0.000000in}{0.000000in}}{\pgfqpoint{0.000000in}{0.027778in}}{%
\pgfpathmoveto{\pgfqpoint{0.000000in}{0.000000in}}%
\pgfpathlineto{\pgfqpoint{0.000000in}{0.027778in}}%
\pgfusepath{stroke,fill}%
}%
\begin{pgfscope}%
\pgfsys@transformshift{2.137194in}{4.627778in}%
\pgfsys@useobject{currentmarker}{}%
\end{pgfscope}%
\end{pgfscope}%
\begin{pgfscope}%
\pgfpathrectangle{\pgfqpoint{0.781944in}{2.977778in}}{\pgfqpoint{5.019444in}{1.650000in}}%
\pgfusepath{clip}%
\pgfsetrectcap%
\pgfsetroundjoin%
\pgfsetlinewidth{0.803000pt}%
\definecolor{currentstroke}{rgb}{0.690196,0.690196,0.690196}%
\pgfsetstrokecolor{currentstroke}%
\pgfsetstrokeopacity{0.300000}%
\pgfsetdash{}{0pt}%
\pgfpathmoveto{\pgfqpoint{2.187389in}{2.977778in}}%
\pgfpathlineto{\pgfqpoint{2.187389in}{4.627778in}}%
\pgfusepath{stroke}%
\end{pgfscope}%
\begin{pgfscope}%
\pgfsetbuttcap%
\pgfsetroundjoin%
\definecolor{currentfill}{rgb}{0.000000,0.000000,0.000000}%
\pgfsetfillcolor{currentfill}%
\pgfsetlinewidth{0.602250pt}%
\definecolor{currentstroke}{rgb}{0.000000,0.000000,0.000000}%
\pgfsetstrokecolor{currentstroke}%
\pgfsetdash{}{0pt}%
\pgfsys@defobject{currentmarker}{\pgfqpoint{0.000000in}{-0.027778in}}{\pgfqpoint{0.000000in}{0.000000in}}{%
\pgfpathmoveto{\pgfqpoint{0.000000in}{0.000000in}}%
\pgfpathlineto{\pgfqpoint{0.000000in}{-0.027778in}}%
\pgfusepath{stroke,fill}%
}%
\begin{pgfscope}%
\pgfsys@transformshift{2.187389in}{2.977778in}%
\pgfsys@useobject{currentmarker}{}%
\end{pgfscope}%
\end{pgfscope}%
\begin{pgfscope}%
\pgfsetbuttcap%
\pgfsetroundjoin%
\definecolor{currentfill}{rgb}{0.000000,0.000000,0.000000}%
\pgfsetfillcolor{currentfill}%
\pgfsetlinewidth{0.602250pt}%
\definecolor{currentstroke}{rgb}{0.000000,0.000000,0.000000}%
\pgfsetstrokecolor{currentstroke}%
\pgfsetdash{}{0pt}%
\pgfsys@defobject{currentmarker}{\pgfqpoint{0.000000in}{0.000000in}}{\pgfqpoint{0.000000in}{0.027778in}}{%
\pgfpathmoveto{\pgfqpoint{0.000000in}{0.000000in}}%
\pgfpathlineto{\pgfqpoint{0.000000in}{0.027778in}}%
\pgfusepath{stroke,fill}%
}%
\begin{pgfscope}%
\pgfsys@transformshift{2.187389in}{4.627778in}%
\pgfsys@useobject{currentmarker}{}%
\end{pgfscope}%
\end{pgfscope}%
\begin{pgfscope}%
\pgfpathrectangle{\pgfqpoint{0.781944in}{2.977778in}}{\pgfqpoint{5.019444in}{1.650000in}}%
\pgfusepath{clip}%
\pgfsetrectcap%
\pgfsetroundjoin%
\pgfsetlinewidth{0.803000pt}%
\definecolor{currentstroke}{rgb}{0.690196,0.690196,0.690196}%
\pgfsetstrokecolor{currentstroke}%
\pgfsetstrokeopacity{0.300000}%
\pgfsetdash{}{0pt}%
\pgfpathmoveto{\pgfqpoint{2.237583in}{2.977778in}}%
\pgfpathlineto{\pgfqpoint{2.237583in}{4.627778in}}%
\pgfusepath{stroke}%
\end{pgfscope}%
\begin{pgfscope}%
\pgfsetbuttcap%
\pgfsetroundjoin%
\definecolor{currentfill}{rgb}{0.000000,0.000000,0.000000}%
\pgfsetfillcolor{currentfill}%
\pgfsetlinewidth{0.602250pt}%
\definecolor{currentstroke}{rgb}{0.000000,0.000000,0.000000}%
\pgfsetstrokecolor{currentstroke}%
\pgfsetdash{}{0pt}%
\pgfsys@defobject{currentmarker}{\pgfqpoint{0.000000in}{-0.027778in}}{\pgfqpoint{0.000000in}{0.000000in}}{%
\pgfpathmoveto{\pgfqpoint{0.000000in}{0.000000in}}%
\pgfpathlineto{\pgfqpoint{0.000000in}{-0.027778in}}%
\pgfusepath{stroke,fill}%
}%
\begin{pgfscope}%
\pgfsys@transformshift{2.237583in}{2.977778in}%
\pgfsys@useobject{currentmarker}{}%
\end{pgfscope}%
\end{pgfscope}%
\begin{pgfscope}%
\pgfsetbuttcap%
\pgfsetroundjoin%
\definecolor{currentfill}{rgb}{0.000000,0.000000,0.000000}%
\pgfsetfillcolor{currentfill}%
\pgfsetlinewidth{0.602250pt}%
\definecolor{currentstroke}{rgb}{0.000000,0.000000,0.000000}%
\pgfsetstrokecolor{currentstroke}%
\pgfsetdash{}{0pt}%
\pgfsys@defobject{currentmarker}{\pgfqpoint{0.000000in}{0.000000in}}{\pgfqpoint{0.000000in}{0.027778in}}{%
\pgfpathmoveto{\pgfqpoint{0.000000in}{0.000000in}}%
\pgfpathlineto{\pgfqpoint{0.000000in}{0.027778in}}%
\pgfusepath{stroke,fill}%
}%
\begin{pgfscope}%
\pgfsys@transformshift{2.237583in}{4.627778in}%
\pgfsys@useobject{currentmarker}{}%
\end{pgfscope}%
\end{pgfscope}%
\begin{pgfscope}%
\pgfpathrectangle{\pgfqpoint{0.781944in}{2.977778in}}{\pgfqpoint{5.019444in}{1.650000in}}%
\pgfusepath{clip}%
\pgfsetrectcap%
\pgfsetroundjoin%
\pgfsetlinewidth{0.803000pt}%
\definecolor{currentstroke}{rgb}{0.690196,0.690196,0.690196}%
\pgfsetstrokecolor{currentstroke}%
\pgfsetstrokeopacity{0.300000}%
\pgfsetdash{}{0pt}%
\pgfpathmoveto{\pgfqpoint{2.287778in}{2.977778in}}%
\pgfpathlineto{\pgfqpoint{2.287778in}{4.627778in}}%
\pgfusepath{stroke}%
\end{pgfscope}%
\begin{pgfscope}%
\pgfsetbuttcap%
\pgfsetroundjoin%
\definecolor{currentfill}{rgb}{0.000000,0.000000,0.000000}%
\pgfsetfillcolor{currentfill}%
\pgfsetlinewidth{0.602250pt}%
\definecolor{currentstroke}{rgb}{0.000000,0.000000,0.000000}%
\pgfsetstrokecolor{currentstroke}%
\pgfsetdash{}{0pt}%
\pgfsys@defobject{currentmarker}{\pgfqpoint{0.000000in}{-0.027778in}}{\pgfqpoint{0.000000in}{0.000000in}}{%
\pgfpathmoveto{\pgfqpoint{0.000000in}{0.000000in}}%
\pgfpathlineto{\pgfqpoint{0.000000in}{-0.027778in}}%
\pgfusepath{stroke,fill}%
}%
\begin{pgfscope}%
\pgfsys@transformshift{2.287778in}{2.977778in}%
\pgfsys@useobject{currentmarker}{}%
\end{pgfscope}%
\end{pgfscope}%
\begin{pgfscope}%
\pgfsetbuttcap%
\pgfsetroundjoin%
\definecolor{currentfill}{rgb}{0.000000,0.000000,0.000000}%
\pgfsetfillcolor{currentfill}%
\pgfsetlinewidth{0.602250pt}%
\definecolor{currentstroke}{rgb}{0.000000,0.000000,0.000000}%
\pgfsetstrokecolor{currentstroke}%
\pgfsetdash{}{0pt}%
\pgfsys@defobject{currentmarker}{\pgfqpoint{0.000000in}{0.000000in}}{\pgfqpoint{0.000000in}{0.027778in}}{%
\pgfpathmoveto{\pgfqpoint{0.000000in}{0.000000in}}%
\pgfpathlineto{\pgfqpoint{0.000000in}{0.027778in}}%
\pgfusepath{stroke,fill}%
}%
\begin{pgfscope}%
\pgfsys@transformshift{2.287778in}{4.627778in}%
\pgfsys@useobject{currentmarker}{}%
\end{pgfscope}%
\end{pgfscope}%
\begin{pgfscope}%
\pgfpathrectangle{\pgfqpoint{0.781944in}{2.977778in}}{\pgfqpoint{5.019444in}{1.650000in}}%
\pgfusepath{clip}%
\pgfsetrectcap%
\pgfsetroundjoin%
\pgfsetlinewidth{0.803000pt}%
\definecolor{currentstroke}{rgb}{0.690196,0.690196,0.690196}%
\pgfsetstrokecolor{currentstroke}%
\pgfsetstrokeopacity{0.300000}%
\pgfsetdash{}{0pt}%
\pgfpathmoveto{\pgfqpoint{2.337972in}{2.977778in}}%
\pgfpathlineto{\pgfqpoint{2.337972in}{4.627778in}}%
\pgfusepath{stroke}%
\end{pgfscope}%
\begin{pgfscope}%
\pgfsetbuttcap%
\pgfsetroundjoin%
\definecolor{currentfill}{rgb}{0.000000,0.000000,0.000000}%
\pgfsetfillcolor{currentfill}%
\pgfsetlinewidth{0.602250pt}%
\definecolor{currentstroke}{rgb}{0.000000,0.000000,0.000000}%
\pgfsetstrokecolor{currentstroke}%
\pgfsetdash{}{0pt}%
\pgfsys@defobject{currentmarker}{\pgfqpoint{0.000000in}{-0.027778in}}{\pgfqpoint{0.000000in}{0.000000in}}{%
\pgfpathmoveto{\pgfqpoint{0.000000in}{0.000000in}}%
\pgfpathlineto{\pgfqpoint{0.000000in}{-0.027778in}}%
\pgfusepath{stroke,fill}%
}%
\begin{pgfscope}%
\pgfsys@transformshift{2.337972in}{2.977778in}%
\pgfsys@useobject{currentmarker}{}%
\end{pgfscope}%
\end{pgfscope}%
\begin{pgfscope}%
\pgfsetbuttcap%
\pgfsetroundjoin%
\definecolor{currentfill}{rgb}{0.000000,0.000000,0.000000}%
\pgfsetfillcolor{currentfill}%
\pgfsetlinewidth{0.602250pt}%
\definecolor{currentstroke}{rgb}{0.000000,0.000000,0.000000}%
\pgfsetstrokecolor{currentstroke}%
\pgfsetdash{}{0pt}%
\pgfsys@defobject{currentmarker}{\pgfqpoint{0.000000in}{0.000000in}}{\pgfqpoint{0.000000in}{0.027778in}}{%
\pgfpathmoveto{\pgfqpoint{0.000000in}{0.000000in}}%
\pgfpathlineto{\pgfqpoint{0.000000in}{0.027778in}}%
\pgfusepath{stroke,fill}%
}%
\begin{pgfscope}%
\pgfsys@transformshift{2.337972in}{4.627778in}%
\pgfsys@useobject{currentmarker}{}%
\end{pgfscope}%
\end{pgfscope}%
\begin{pgfscope}%
\pgfpathrectangle{\pgfqpoint{0.781944in}{2.977778in}}{\pgfqpoint{5.019444in}{1.650000in}}%
\pgfusepath{clip}%
\pgfsetrectcap%
\pgfsetroundjoin%
\pgfsetlinewidth{0.803000pt}%
\definecolor{currentstroke}{rgb}{0.690196,0.690196,0.690196}%
\pgfsetstrokecolor{currentstroke}%
\pgfsetstrokeopacity{0.300000}%
\pgfsetdash{}{0pt}%
\pgfpathmoveto{\pgfqpoint{2.388167in}{2.977778in}}%
\pgfpathlineto{\pgfqpoint{2.388167in}{4.627778in}}%
\pgfusepath{stroke}%
\end{pgfscope}%
\begin{pgfscope}%
\pgfsetbuttcap%
\pgfsetroundjoin%
\definecolor{currentfill}{rgb}{0.000000,0.000000,0.000000}%
\pgfsetfillcolor{currentfill}%
\pgfsetlinewidth{0.602250pt}%
\definecolor{currentstroke}{rgb}{0.000000,0.000000,0.000000}%
\pgfsetstrokecolor{currentstroke}%
\pgfsetdash{}{0pt}%
\pgfsys@defobject{currentmarker}{\pgfqpoint{0.000000in}{-0.027778in}}{\pgfqpoint{0.000000in}{0.000000in}}{%
\pgfpathmoveto{\pgfqpoint{0.000000in}{0.000000in}}%
\pgfpathlineto{\pgfqpoint{0.000000in}{-0.027778in}}%
\pgfusepath{stroke,fill}%
}%
\begin{pgfscope}%
\pgfsys@transformshift{2.388167in}{2.977778in}%
\pgfsys@useobject{currentmarker}{}%
\end{pgfscope}%
\end{pgfscope}%
\begin{pgfscope}%
\pgfsetbuttcap%
\pgfsetroundjoin%
\definecolor{currentfill}{rgb}{0.000000,0.000000,0.000000}%
\pgfsetfillcolor{currentfill}%
\pgfsetlinewidth{0.602250pt}%
\definecolor{currentstroke}{rgb}{0.000000,0.000000,0.000000}%
\pgfsetstrokecolor{currentstroke}%
\pgfsetdash{}{0pt}%
\pgfsys@defobject{currentmarker}{\pgfqpoint{0.000000in}{0.000000in}}{\pgfqpoint{0.000000in}{0.027778in}}{%
\pgfpathmoveto{\pgfqpoint{0.000000in}{0.000000in}}%
\pgfpathlineto{\pgfqpoint{0.000000in}{0.027778in}}%
\pgfusepath{stroke,fill}%
}%
\begin{pgfscope}%
\pgfsys@transformshift{2.388167in}{4.627778in}%
\pgfsys@useobject{currentmarker}{}%
\end{pgfscope}%
\end{pgfscope}%
\begin{pgfscope}%
\pgfpathrectangle{\pgfqpoint{0.781944in}{2.977778in}}{\pgfqpoint{5.019444in}{1.650000in}}%
\pgfusepath{clip}%
\pgfsetrectcap%
\pgfsetroundjoin%
\pgfsetlinewidth{0.803000pt}%
\definecolor{currentstroke}{rgb}{0.690196,0.690196,0.690196}%
\pgfsetstrokecolor{currentstroke}%
\pgfsetstrokeopacity{0.300000}%
\pgfsetdash{}{0pt}%
\pgfpathmoveto{\pgfqpoint{2.438361in}{2.977778in}}%
\pgfpathlineto{\pgfqpoint{2.438361in}{4.627778in}}%
\pgfusepath{stroke}%
\end{pgfscope}%
\begin{pgfscope}%
\pgfsetbuttcap%
\pgfsetroundjoin%
\definecolor{currentfill}{rgb}{0.000000,0.000000,0.000000}%
\pgfsetfillcolor{currentfill}%
\pgfsetlinewidth{0.602250pt}%
\definecolor{currentstroke}{rgb}{0.000000,0.000000,0.000000}%
\pgfsetstrokecolor{currentstroke}%
\pgfsetdash{}{0pt}%
\pgfsys@defobject{currentmarker}{\pgfqpoint{0.000000in}{-0.027778in}}{\pgfqpoint{0.000000in}{0.000000in}}{%
\pgfpathmoveto{\pgfqpoint{0.000000in}{0.000000in}}%
\pgfpathlineto{\pgfqpoint{0.000000in}{-0.027778in}}%
\pgfusepath{stroke,fill}%
}%
\begin{pgfscope}%
\pgfsys@transformshift{2.438361in}{2.977778in}%
\pgfsys@useobject{currentmarker}{}%
\end{pgfscope}%
\end{pgfscope}%
\begin{pgfscope}%
\pgfsetbuttcap%
\pgfsetroundjoin%
\definecolor{currentfill}{rgb}{0.000000,0.000000,0.000000}%
\pgfsetfillcolor{currentfill}%
\pgfsetlinewidth{0.602250pt}%
\definecolor{currentstroke}{rgb}{0.000000,0.000000,0.000000}%
\pgfsetstrokecolor{currentstroke}%
\pgfsetdash{}{0pt}%
\pgfsys@defobject{currentmarker}{\pgfqpoint{0.000000in}{0.000000in}}{\pgfqpoint{0.000000in}{0.027778in}}{%
\pgfpathmoveto{\pgfqpoint{0.000000in}{0.000000in}}%
\pgfpathlineto{\pgfqpoint{0.000000in}{0.027778in}}%
\pgfusepath{stroke,fill}%
}%
\begin{pgfscope}%
\pgfsys@transformshift{2.438361in}{4.627778in}%
\pgfsys@useobject{currentmarker}{}%
\end{pgfscope}%
\end{pgfscope}%
\begin{pgfscope}%
\pgfpathrectangle{\pgfqpoint{0.781944in}{2.977778in}}{\pgfqpoint{5.019444in}{1.650000in}}%
\pgfusepath{clip}%
\pgfsetrectcap%
\pgfsetroundjoin%
\pgfsetlinewidth{0.803000pt}%
\definecolor{currentstroke}{rgb}{0.690196,0.690196,0.690196}%
\pgfsetstrokecolor{currentstroke}%
\pgfsetstrokeopacity{0.300000}%
\pgfsetdash{}{0pt}%
\pgfpathmoveto{\pgfqpoint{2.488556in}{2.977778in}}%
\pgfpathlineto{\pgfqpoint{2.488556in}{4.627778in}}%
\pgfusepath{stroke}%
\end{pgfscope}%
\begin{pgfscope}%
\pgfsetbuttcap%
\pgfsetroundjoin%
\definecolor{currentfill}{rgb}{0.000000,0.000000,0.000000}%
\pgfsetfillcolor{currentfill}%
\pgfsetlinewidth{0.602250pt}%
\definecolor{currentstroke}{rgb}{0.000000,0.000000,0.000000}%
\pgfsetstrokecolor{currentstroke}%
\pgfsetdash{}{0pt}%
\pgfsys@defobject{currentmarker}{\pgfqpoint{0.000000in}{-0.027778in}}{\pgfqpoint{0.000000in}{0.000000in}}{%
\pgfpathmoveto{\pgfqpoint{0.000000in}{0.000000in}}%
\pgfpathlineto{\pgfqpoint{0.000000in}{-0.027778in}}%
\pgfusepath{stroke,fill}%
}%
\begin{pgfscope}%
\pgfsys@transformshift{2.488556in}{2.977778in}%
\pgfsys@useobject{currentmarker}{}%
\end{pgfscope}%
\end{pgfscope}%
\begin{pgfscope}%
\pgfsetbuttcap%
\pgfsetroundjoin%
\definecolor{currentfill}{rgb}{0.000000,0.000000,0.000000}%
\pgfsetfillcolor{currentfill}%
\pgfsetlinewidth{0.602250pt}%
\definecolor{currentstroke}{rgb}{0.000000,0.000000,0.000000}%
\pgfsetstrokecolor{currentstroke}%
\pgfsetdash{}{0pt}%
\pgfsys@defobject{currentmarker}{\pgfqpoint{0.000000in}{0.000000in}}{\pgfqpoint{0.000000in}{0.027778in}}{%
\pgfpathmoveto{\pgfqpoint{0.000000in}{0.000000in}}%
\pgfpathlineto{\pgfqpoint{0.000000in}{0.027778in}}%
\pgfusepath{stroke,fill}%
}%
\begin{pgfscope}%
\pgfsys@transformshift{2.488556in}{4.627778in}%
\pgfsys@useobject{currentmarker}{}%
\end{pgfscope}%
\end{pgfscope}%
\begin{pgfscope}%
\pgfpathrectangle{\pgfqpoint{0.781944in}{2.977778in}}{\pgfqpoint{5.019444in}{1.650000in}}%
\pgfusepath{clip}%
\pgfsetrectcap%
\pgfsetroundjoin%
\pgfsetlinewidth{0.803000pt}%
\definecolor{currentstroke}{rgb}{0.690196,0.690196,0.690196}%
\pgfsetstrokecolor{currentstroke}%
\pgfsetstrokeopacity{0.300000}%
\pgfsetdash{}{0pt}%
\pgfpathmoveto{\pgfqpoint{2.588944in}{2.977778in}}%
\pgfpathlineto{\pgfqpoint{2.588944in}{4.627778in}}%
\pgfusepath{stroke}%
\end{pgfscope}%
\begin{pgfscope}%
\pgfsetbuttcap%
\pgfsetroundjoin%
\definecolor{currentfill}{rgb}{0.000000,0.000000,0.000000}%
\pgfsetfillcolor{currentfill}%
\pgfsetlinewidth{0.602250pt}%
\definecolor{currentstroke}{rgb}{0.000000,0.000000,0.000000}%
\pgfsetstrokecolor{currentstroke}%
\pgfsetdash{}{0pt}%
\pgfsys@defobject{currentmarker}{\pgfqpoint{0.000000in}{-0.027778in}}{\pgfqpoint{0.000000in}{0.000000in}}{%
\pgfpathmoveto{\pgfqpoint{0.000000in}{0.000000in}}%
\pgfpathlineto{\pgfqpoint{0.000000in}{-0.027778in}}%
\pgfusepath{stroke,fill}%
}%
\begin{pgfscope}%
\pgfsys@transformshift{2.588944in}{2.977778in}%
\pgfsys@useobject{currentmarker}{}%
\end{pgfscope}%
\end{pgfscope}%
\begin{pgfscope}%
\pgfsetbuttcap%
\pgfsetroundjoin%
\definecolor{currentfill}{rgb}{0.000000,0.000000,0.000000}%
\pgfsetfillcolor{currentfill}%
\pgfsetlinewidth{0.602250pt}%
\definecolor{currentstroke}{rgb}{0.000000,0.000000,0.000000}%
\pgfsetstrokecolor{currentstroke}%
\pgfsetdash{}{0pt}%
\pgfsys@defobject{currentmarker}{\pgfqpoint{0.000000in}{0.000000in}}{\pgfqpoint{0.000000in}{0.027778in}}{%
\pgfpathmoveto{\pgfqpoint{0.000000in}{0.000000in}}%
\pgfpathlineto{\pgfqpoint{0.000000in}{0.027778in}}%
\pgfusepath{stroke,fill}%
}%
\begin{pgfscope}%
\pgfsys@transformshift{2.588944in}{4.627778in}%
\pgfsys@useobject{currentmarker}{}%
\end{pgfscope}%
\end{pgfscope}%
\begin{pgfscope}%
\pgfpathrectangle{\pgfqpoint{0.781944in}{2.977778in}}{\pgfqpoint{5.019444in}{1.650000in}}%
\pgfusepath{clip}%
\pgfsetrectcap%
\pgfsetroundjoin%
\pgfsetlinewidth{0.803000pt}%
\definecolor{currentstroke}{rgb}{0.690196,0.690196,0.690196}%
\pgfsetstrokecolor{currentstroke}%
\pgfsetstrokeopacity{0.300000}%
\pgfsetdash{}{0pt}%
\pgfpathmoveto{\pgfqpoint{2.639139in}{2.977778in}}%
\pgfpathlineto{\pgfqpoint{2.639139in}{4.627778in}}%
\pgfusepath{stroke}%
\end{pgfscope}%
\begin{pgfscope}%
\pgfsetbuttcap%
\pgfsetroundjoin%
\definecolor{currentfill}{rgb}{0.000000,0.000000,0.000000}%
\pgfsetfillcolor{currentfill}%
\pgfsetlinewidth{0.602250pt}%
\definecolor{currentstroke}{rgb}{0.000000,0.000000,0.000000}%
\pgfsetstrokecolor{currentstroke}%
\pgfsetdash{}{0pt}%
\pgfsys@defobject{currentmarker}{\pgfqpoint{0.000000in}{-0.027778in}}{\pgfqpoint{0.000000in}{0.000000in}}{%
\pgfpathmoveto{\pgfqpoint{0.000000in}{0.000000in}}%
\pgfpathlineto{\pgfqpoint{0.000000in}{-0.027778in}}%
\pgfusepath{stroke,fill}%
}%
\begin{pgfscope}%
\pgfsys@transformshift{2.639139in}{2.977778in}%
\pgfsys@useobject{currentmarker}{}%
\end{pgfscope}%
\end{pgfscope}%
\begin{pgfscope}%
\pgfsetbuttcap%
\pgfsetroundjoin%
\definecolor{currentfill}{rgb}{0.000000,0.000000,0.000000}%
\pgfsetfillcolor{currentfill}%
\pgfsetlinewidth{0.602250pt}%
\definecolor{currentstroke}{rgb}{0.000000,0.000000,0.000000}%
\pgfsetstrokecolor{currentstroke}%
\pgfsetdash{}{0pt}%
\pgfsys@defobject{currentmarker}{\pgfqpoint{0.000000in}{0.000000in}}{\pgfqpoint{0.000000in}{0.027778in}}{%
\pgfpathmoveto{\pgfqpoint{0.000000in}{0.000000in}}%
\pgfpathlineto{\pgfqpoint{0.000000in}{0.027778in}}%
\pgfusepath{stroke,fill}%
}%
\begin{pgfscope}%
\pgfsys@transformshift{2.639139in}{4.627778in}%
\pgfsys@useobject{currentmarker}{}%
\end{pgfscope}%
\end{pgfscope}%
\begin{pgfscope}%
\pgfpathrectangle{\pgfqpoint{0.781944in}{2.977778in}}{\pgfqpoint{5.019444in}{1.650000in}}%
\pgfusepath{clip}%
\pgfsetrectcap%
\pgfsetroundjoin%
\pgfsetlinewidth{0.803000pt}%
\definecolor{currentstroke}{rgb}{0.690196,0.690196,0.690196}%
\pgfsetstrokecolor{currentstroke}%
\pgfsetstrokeopacity{0.300000}%
\pgfsetdash{}{0pt}%
\pgfpathmoveto{\pgfqpoint{2.689333in}{2.977778in}}%
\pgfpathlineto{\pgfqpoint{2.689333in}{4.627778in}}%
\pgfusepath{stroke}%
\end{pgfscope}%
\begin{pgfscope}%
\pgfsetbuttcap%
\pgfsetroundjoin%
\definecolor{currentfill}{rgb}{0.000000,0.000000,0.000000}%
\pgfsetfillcolor{currentfill}%
\pgfsetlinewidth{0.602250pt}%
\definecolor{currentstroke}{rgb}{0.000000,0.000000,0.000000}%
\pgfsetstrokecolor{currentstroke}%
\pgfsetdash{}{0pt}%
\pgfsys@defobject{currentmarker}{\pgfqpoint{0.000000in}{-0.027778in}}{\pgfqpoint{0.000000in}{0.000000in}}{%
\pgfpathmoveto{\pgfqpoint{0.000000in}{0.000000in}}%
\pgfpathlineto{\pgfqpoint{0.000000in}{-0.027778in}}%
\pgfusepath{stroke,fill}%
}%
\begin{pgfscope}%
\pgfsys@transformshift{2.689333in}{2.977778in}%
\pgfsys@useobject{currentmarker}{}%
\end{pgfscope}%
\end{pgfscope}%
\begin{pgfscope}%
\pgfsetbuttcap%
\pgfsetroundjoin%
\definecolor{currentfill}{rgb}{0.000000,0.000000,0.000000}%
\pgfsetfillcolor{currentfill}%
\pgfsetlinewidth{0.602250pt}%
\definecolor{currentstroke}{rgb}{0.000000,0.000000,0.000000}%
\pgfsetstrokecolor{currentstroke}%
\pgfsetdash{}{0pt}%
\pgfsys@defobject{currentmarker}{\pgfqpoint{0.000000in}{0.000000in}}{\pgfqpoint{0.000000in}{0.027778in}}{%
\pgfpathmoveto{\pgfqpoint{0.000000in}{0.000000in}}%
\pgfpathlineto{\pgfqpoint{0.000000in}{0.027778in}}%
\pgfusepath{stroke,fill}%
}%
\begin{pgfscope}%
\pgfsys@transformshift{2.689333in}{4.627778in}%
\pgfsys@useobject{currentmarker}{}%
\end{pgfscope}%
\end{pgfscope}%
\begin{pgfscope}%
\pgfpathrectangle{\pgfqpoint{0.781944in}{2.977778in}}{\pgfqpoint{5.019444in}{1.650000in}}%
\pgfusepath{clip}%
\pgfsetrectcap%
\pgfsetroundjoin%
\pgfsetlinewidth{0.803000pt}%
\definecolor{currentstroke}{rgb}{0.690196,0.690196,0.690196}%
\pgfsetstrokecolor{currentstroke}%
\pgfsetstrokeopacity{0.300000}%
\pgfsetdash{}{0pt}%
\pgfpathmoveto{\pgfqpoint{2.739528in}{2.977778in}}%
\pgfpathlineto{\pgfqpoint{2.739528in}{4.627778in}}%
\pgfusepath{stroke}%
\end{pgfscope}%
\begin{pgfscope}%
\pgfsetbuttcap%
\pgfsetroundjoin%
\definecolor{currentfill}{rgb}{0.000000,0.000000,0.000000}%
\pgfsetfillcolor{currentfill}%
\pgfsetlinewidth{0.602250pt}%
\definecolor{currentstroke}{rgb}{0.000000,0.000000,0.000000}%
\pgfsetstrokecolor{currentstroke}%
\pgfsetdash{}{0pt}%
\pgfsys@defobject{currentmarker}{\pgfqpoint{0.000000in}{-0.027778in}}{\pgfqpoint{0.000000in}{0.000000in}}{%
\pgfpathmoveto{\pgfqpoint{0.000000in}{0.000000in}}%
\pgfpathlineto{\pgfqpoint{0.000000in}{-0.027778in}}%
\pgfusepath{stroke,fill}%
}%
\begin{pgfscope}%
\pgfsys@transformshift{2.739528in}{2.977778in}%
\pgfsys@useobject{currentmarker}{}%
\end{pgfscope}%
\end{pgfscope}%
\begin{pgfscope}%
\pgfsetbuttcap%
\pgfsetroundjoin%
\definecolor{currentfill}{rgb}{0.000000,0.000000,0.000000}%
\pgfsetfillcolor{currentfill}%
\pgfsetlinewidth{0.602250pt}%
\definecolor{currentstroke}{rgb}{0.000000,0.000000,0.000000}%
\pgfsetstrokecolor{currentstroke}%
\pgfsetdash{}{0pt}%
\pgfsys@defobject{currentmarker}{\pgfqpoint{0.000000in}{0.000000in}}{\pgfqpoint{0.000000in}{0.027778in}}{%
\pgfpathmoveto{\pgfqpoint{0.000000in}{0.000000in}}%
\pgfpathlineto{\pgfqpoint{0.000000in}{0.027778in}}%
\pgfusepath{stroke,fill}%
}%
\begin{pgfscope}%
\pgfsys@transformshift{2.739528in}{4.627778in}%
\pgfsys@useobject{currentmarker}{}%
\end{pgfscope}%
\end{pgfscope}%
\begin{pgfscope}%
\pgfpathrectangle{\pgfqpoint{0.781944in}{2.977778in}}{\pgfqpoint{5.019444in}{1.650000in}}%
\pgfusepath{clip}%
\pgfsetrectcap%
\pgfsetroundjoin%
\pgfsetlinewidth{0.803000pt}%
\definecolor{currentstroke}{rgb}{0.690196,0.690196,0.690196}%
\pgfsetstrokecolor{currentstroke}%
\pgfsetstrokeopacity{0.300000}%
\pgfsetdash{}{0pt}%
\pgfpathmoveto{\pgfqpoint{2.789722in}{2.977778in}}%
\pgfpathlineto{\pgfqpoint{2.789722in}{4.627778in}}%
\pgfusepath{stroke}%
\end{pgfscope}%
\begin{pgfscope}%
\pgfsetbuttcap%
\pgfsetroundjoin%
\definecolor{currentfill}{rgb}{0.000000,0.000000,0.000000}%
\pgfsetfillcolor{currentfill}%
\pgfsetlinewidth{0.602250pt}%
\definecolor{currentstroke}{rgb}{0.000000,0.000000,0.000000}%
\pgfsetstrokecolor{currentstroke}%
\pgfsetdash{}{0pt}%
\pgfsys@defobject{currentmarker}{\pgfqpoint{0.000000in}{-0.027778in}}{\pgfqpoint{0.000000in}{0.000000in}}{%
\pgfpathmoveto{\pgfqpoint{0.000000in}{0.000000in}}%
\pgfpathlineto{\pgfqpoint{0.000000in}{-0.027778in}}%
\pgfusepath{stroke,fill}%
}%
\begin{pgfscope}%
\pgfsys@transformshift{2.789722in}{2.977778in}%
\pgfsys@useobject{currentmarker}{}%
\end{pgfscope}%
\end{pgfscope}%
\begin{pgfscope}%
\pgfsetbuttcap%
\pgfsetroundjoin%
\definecolor{currentfill}{rgb}{0.000000,0.000000,0.000000}%
\pgfsetfillcolor{currentfill}%
\pgfsetlinewidth{0.602250pt}%
\definecolor{currentstroke}{rgb}{0.000000,0.000000,0.000000}%
\pgfsetstrokecolor{currentstroke}%
\pgfsetdash{}{0pt}%
\pgfsys@defobject{currentmarker}{\pgfqpoint{0.000000in}{0.000000in}}{\pgfqpoint{0.000000in}{0.027778in}}{%
\pgfpathmoveto{\pgfqpoint{0.000000in}{0.000000in}}%
\pgfpathlineto{\pgfqpoint{0.000000in}{0.027778in}}%
\pgfusepath{stroke,fill}%
}%
\begin{pgfscope}%
\pgfsys@transformshift{2.789722in}{4.627778in}%
\pgfsys@useobject{currentmarker}{}%
\end{pgfscope}%
\end{pgfscope}%
\begin{pgfscope}%
\pgfpathrectangle{\pgfqpoint{0.781944in}{2.977778in}}{\pgfqpoint{5.019444in}{1.650000in}}%
\pgfusepath{clip}%
\pgfsetrectcap%
\pgfsetroundjoin%
\pgfsetlinewidth{0.803000pt}%
\definecolor{currentstroke}{rgb}{0.690196,0.690196,0.690196}%
\pgfsetstrokecolor{currentstroke}%
\pgfsetstrokeopacity{0.300000}%
\pgfsetdash{}{0pt}%
\pgfpathmoveto{\pgfqpoint{2.839917in}{2.977778in}}%
\pgfpathlineto{\pgfqpoint{2.839917in}{4.627778in}}%
\pgfusepath{stroke}%
\end{pgfscope}%
\begin{pgfscope}%
\pgfsetbuttcap%
\pgfsetroundjoin%
\definecolor{currentfill}{rgb}{0.000000,0.000000,0.000000}%
\pgfsetfillcolor{currentfill}%
\pgfsetlinewidth{0.602250pt}%
\definecolor{currentstroke}{rgb}{0.000000,0.000000,0.000000}%
\pgfsetstrokecolor{currentstroke}%
\pgfsetdash{}{0pt}%
\pgfsys@defobject{currentmarker}{\pgfqpoint{0.000000in}{-0.027778in}}{\pgfqpoint{0.000000in}{0.000000in}}{%
\pgfpathmoveto{\pgfqpoint{0.000000in}{0.000000in}}%
\pgfpathlineto{\pgfqpoint{0.000000in}{-0.027778in}}%
\pgfusepath{stroke,fill}%
}%
\begin{pgfscope}%
\pgfsys@transformshift{2.839917in}{2.977778in}%
\pgfsys@useobject{currentmarker}{}%
\end{pgfscope}%
\end{pgfscope}%
\begin{pgfscope}%
\pgfsetbuttcap%
\pgfsetroundjoin%
\definecolor{currentfill}{rgb}{0.000000,0.000000,0.000000}%
\pgfsetfillcolor{currentfill}%
\pgfsetlinewidth{0.602250pt}%
\definecolor{currentstroke}{rgb}{0.000000,0.000000,0.000000}%
\pgfsetstrokecolor{currentstroke}%
\pgfsetdash{}{0pt}%
\pgfsys@defobject{currentmarker}{\pgfqpoint{0.000000in}{0.000000in}}{\pgfqpoint{0.000000in}{0.027778in}}{%
\pgfpathmoveto{\pgfqpoint{0.000000in}{0.000000in}}%
\pgfpathlineto{\pgfqpoint{0.000000in}{0.027778in}}%
\pgfusepath{stroke,fill}%
}%
\begin{pgfscope}%
\pgfsys@transformshift{2.839917in}{4.627778in}%
\pgfsys@useobject{currentmarker}{}%
\end{pgfscope}%
\end{pgfscope}%
\begin{pgfscope}%
\pgfpathrectangle{\pgfqpoint{0.781944in}{2.977778in}}{\pgfqpoint{5.019444in}{1.650000in}}%
\pgfusepath{clip}%
\pgfsetrectcap%
\pgfsetroundjoin%
\pgfsetlinewidth{0.803000pt}%
\definecolor{currentstroke}{rgb}{0.690196,0.690196,0.690196}%
\pgfsetstrokecolor{currentstroke}%
\pgfsetstrokeopacity{0.300000}%
\pgfsetdash{}{0pt}%
\pgfpathmoveto{\pgfqpoint{2.890111in}{2.977778in}}%
\pgfpathlineto{\pgfqpoint{2.890111in}{4.627778in}}%
\pgfusepath{stroke}%
\end{pgfscope}%
\begin{pgfscope}%
\pgfsetbuttcap%
\pgfsetroundjoin%
\definecolor{currentfill}{rgb}{0.000000,0.000000,0.000000}%
\pgfsetfillcolor{currentfill}%
\pgfsetlinewidth{0.602250pt}%
\definecolor{currentstroke}{rgb}{0.000000,0.000000,0.000000}%
\pgfsetstrokecolor{currentstroke}%
\pgfsetdash{}{0pt}%
\pgfsys@defobject{currentmarker}{\pgfqpoint{0.000000in}{-0.027778in}}{\pgfqpoint{0.000000in}{0.000000in}}{%
\pgfpathmoveto{\pgfqpoint{0.000000in}{0.000000in}}%
\pgfpathlineto{\pgfqpoint{0.000000in}{-0.027778in}}%
\pgfusepath{stroke,fill}%
}%
\begin{pgfscope}%
\pgfsys@transformshift{2.890111in}{2.977778in}%
\pgfsys@useobject{currentmarker}{}%
\end{pgfscope}%
\end{pgfscope}%
\begin{pgfscope}%
\pgfsetbuttcap%
\pgfsetroundjoin%
\definecolor{currentfill}{rgb}{0.000000,0.000000,0.000000}%
\pgfsetfillcolor{currentfill}%
\pgfsetlinewidth{0.602250pt}%
\definecolor{currentstroke}{rgb}{0.000000,0.000000,0.000000}%
\pgfsetstrokecolor{currentstroke}%
\pgfsetdash{}{0pt}%
\pgfsys@defobject{currentmarker}{\pgfqpoint{0.000000in}{0.000000in}}{\pgfqpoint{0.000000in}{0.027778in}}{%
\pgfpathmoveto{\pgfqpoint{0.000000in}{0.000000in}}%
\pgfpathlineto{\pgfqpoint{0.000000in}{0.027778in}}%
\pgfusepath{stroke,fill}%
}%
\begin{pgfscope}%
\pgfsys@transformshift{2.890111in}{4.627778in}%
\pgfsys@useobject{currentmarker}{}%
\end{pgfscope}%
\end{pgfscope}%
\begin{pgfscope}%
\pgfpathrectangle{\pgfqpoint{0.781944in}{2.977778in}}{\pgfqpoint{5.019444in}{1.650000in}}%
\pgfusepath{clip}%
\pgfsetrectcap%
\pgfsetroundjoin%
\pgfsetlinewidth{0.803000pt}%
\definecolor{currentstroke}{rgb}{0.690196,0.690196,0.690196}%
\pgfsetstrokecolor{currentstroke}%
\pgfsetstrokeopacity{0.300000}%
\pgfsetdash{}{0pt}%
\pgfpathmoveto{\pgfqpoint{2.940306in}{2.977778in}}%
\pgfpathlineto{\pgfqpoint{2.940306in}{4.627778in}}%
\pgfusepath{stroke}%
\end{pgfscope}%
\begin{pgfscope}%
\pgfsetbuttcap%
\pgfsetroundjoin%
\definecolor{currentfill}{rgb}{0.000000,0.000000,0.000000}%
\pgfsetfillcolor{currentfill}%
\pgfsetlinewidth{0.602250pt}%
\definecolor{currentstroke}{rgb}{0.000000,0.000000,0.000000}%
\pgfsetstrokecolor{currentstroke}%
\pgfsetdash{}{0pt}%
\pgfsys@defobject{currentmarker}{\pgfqpoint{0.000000in}{-0.027778in}}{\pgfqpoint{0.000000in}{0.000000in}}{%
\pgfpathmoveto{\pgfqpoint{0.000000in}{0.000000in}}%
\pgfpathlineto{\pgfqpoint{0.000000in}{-0.027778in}}%
\pgfusepath{stroke,fill}%
}%
\begin{pgfscope}%
\pgfsys@transformshift{2.940306in}{2.977778in}%
\pgfsys@useobject{currentmarker}{}%
\end{pgfscope}%
\end{pgfscope}%
\begin{pgfscope}%
\pgfsetbuttcap%
\pgfsetroundjoin%
\definecolor{currentfill}{rgb}{0.000000,0.000000,0.000000}%
\pgfsetfillcolor{currentfill}%
\pgfsetlinewidth{0.602250pt}%
\definecolor{currentstroke}{rgb}{0.000000,0.000000,0.000000}%
\pgfsetstrokecolor{currentstroke}%
\pgfsetdash{}{0pt}%
\pgfsys@defobject{currentmarker}{\pgfqpoint{0.000000in}{0.000000in}}{\pgfqpoint{0.000000in}{0.027778in}}{%
\pgfpathmoveto{\pgfqpoint{0.000000in}{0.000000in}}%
\pgfpathlineto{\pgfqpoint{0.000000in}{0.027778in}}%
\pgfusepath{stroke,fill}%
}%
\begin{pgfscope}%
\pgfsys@transformshift{2.940306in}{4.627778in}%
\pgfsys@useobject{currentmarker}{}%
\end{pgfscope}%
\end{pgfscope}%
\begin{pgfscope}%
\pgfpathrectangle{\pgfqpoint{0.781944in}{2.977778in}}{\pgfqpoint{5.019444in}{1.650000in}}%
\pgfusepath{clip}%
\pgfsetrectcap%
\pgfsetroundjoin%
\pgfsetlinewidth{0.803000pt}%
\definecolor{currentstroke}{rgb}{0.690196,0.690196,0.690196}%
\pgfsetstrokecolor{currentstroke}%
\pgfsetstrokeopacity{0.300000}%
\pgfsetdash{}{0pt}%
\pgfpathmoveto{\pgfqpoint{2.990500in}{2.977778in}}%
\pgfpathlineto{\pgfqpoint{2.990500in}{4.627778in}}%
\pgfusepath{stroke}%
\end{pgfscope}%
\begin{pgfscope}%
\pgfsetbuttcap%
\pgfsetroundjoin%
\definecolor{currentfill}{rgb}{0.000000,0.000000,0.000000}%
\pgfsetfillcolor{currentfill}%
\pgfsetlinewidth{0.602250pt}%
\definecolor{currentstroke}{rgb}{0.000000,0.000000,0.000000}%
\pgfsetstrokecolor{currentstroke}%
\pgfsetdash{}{0pt}%
\pgfsys@defobject{currentmarker}{\pgfqpoint{0.000000in}{-0.027778in}}{\pgfqpoint{0.000000in}{0.000000in}}{%
\pgfpathmoveto{\pgfqpoint{0.000000in}{0.000000in}}%
\pgfpathlineto{\pgfqpoint{0.000000in}{-0.027778in}}%
\pgfusepath{stroke,fill}%
}%
\begin{pgfscope}%
\pgfsys@transformshift{2.990500in}{2.977778in}%
\pgfsys@useobject{currentmarker}{}%
\end{pgfscope}%
\end{pgfscope}%
\begin{pgfscope}%
\pgfsetbuttcap%
\pgfsetroundjoin%
\definecolor{currentfill}{rgb}{0.000000,0.000000,0.000000}%
\pgfsetfillcolor{currentfill}%
\pgfsetlinewidth{0.602250pt}%
\definecolor{currentstroke}{rgb}{0.000000,0.000000,0.000000}%
\pgfsetstrokecolor{currentstroke}%
\pgfsetdash{}{0pt}%
\pgfsys@defobject{currentmarker}{\pgfqpoint{0.000000in}{0.000000in}}{\pgfqpoint{0.000000in}{0.027778in}}{%
\pgfpathmoveto{\pgfqpoint{0.000000in}{0.000000in}}%
\pgfpathlineto{\pgfqpoint{0.000000in}{0.027778in}}%
\pgfusepath{stroke,fill}%
}%
\begin{pgfscope}%
\pgfsys@transformshift{2.990500in}{4.627778in}%
\pgfsys@useobject{currentmarker}{}%
\end{pgfscope}%
\end{pgfscope}%
\begin{pgfscope}%
\pgfpathrectangle{\pgfqpoint{0.781944in}{2.977778in}}{\pgfqpoint{5.019444in}{1.650000in}}%
\pgfusepath{clip}%
\pgfsetrectcap%
\pgfsetroundjoin%
\pgfsetlinewidth{0.803000pt}%
\definecolor{currentstroke}{rgb}{0.690196,0.690196,0.690196}%
\pgfsetstrokecolor{currentstroke}%
\pgfsetstrokeopacity{0.300000}%
\pgfsetdash{}{0pt}%
\pgfpathmoveto{\pgfqpoint{3.090889in}{2.977778in}}%
\pgfpathlineto{\pgfqpoint{3.090889in}{4.627778in}}%
\pgfusepath{stroke}%
\end{pgfscope}%
\begin{pgfscope}%
\pgfsetbuttcap%
\pgfsetroundjoin%
\definecolor{currentfill}{rgb}{0.000000,0.000000,0.000000}%
\pgfsetfillcolor{currentfill}%
\pgfsetlinewidth{0.602250pt}%
\definecolor{currentstroke}{rgb}{0.000000,0.000000,0.000000}%
\pgfsetstrokecolor{currentstroke}%
\pgfsetdash{}{0pt}%
\pgfsys@defobject{currentmarker}{\pgfqpoint{0.000000in}{-0.027778in}}{\pgfqpoint{0.000000in}{0.000000in}}{%
\pgfpathmoveto{\pgfqpoint{0.000000in}{0.000000in}}%
\pgfpathlineto{\pgfqpoint{0.000000in}{-0.027778in}}%
\pgfusepath{stroke,fill}%
}%
\begin{pgfscope}%
\pgfsys@transformshift{3.090889in}{2.977778in}%
\pgfsys@useobject{currentmarker}{}%
\end{pgfscope}%
\end{pgfscope}%
\begin{pgfscope}%
\pgfsetbuttcap%
\pgfsetroundjoin%
\definecolor{currentfill}{rgb}{0.000000,0.000000,0.000000}%
\pgfsetfillcolor{currentfill}%
\pgfsetlinewidth{0.602250pt}%
\definecolor{currentstroke}{rgb}{0.000000,0.000000,0.000000}%
\pgfsetstrokecolor{currentstroke}%
\pgfsetdash{}{0pt}%
\pgfsys@defobject{currentmarker}{\pgfqpoint{0.000000in}{0.000000in}}{\pgfqpoint{0.000000in}{0.027778in}}{%
\pgfpathmoveto{\pgfqpoint{0.000000in}{0.000000in}}%
\pgfpathlineto{\pgfqpoint{0.000000in}{0.027778in}}%
\pgfusepath{stroke,fill}%
}%
\begin{pgfscope}%
\pgfsys@transformshift{3.090889in}{4.627778in}%
\pgfsys@useobject{currentmarker}{}%
\end{pgfscope}%
\end{pgfscope}%
\begin{pgfscope}%
\pgfpathrectangle{\pgfqpoint{0.781944in}{2.977778in}}{\pgfqpoint{5.019444in}{1.650000in}}%
\pgfusepath{clip}%
\pgfsetrectcap%
\pgfsetroundjoin%
\pgfsetlinewidth{0.803000pt}%
\definecolor{currentstroke}{rgb}{0.690196,0.690196,0.690196}%
\pgfsetstrokecolor{currentstroke}%
\pgfsetstrokeopacity{0.300000}%
\pgfsetdash{}{0pt}%
\pgfpathmoveto{\pgfqpoint{3.141083in}{2.977778in}}%
\pgfpathlineto{\pgfqpoint{3.141083in}{4.627778in}}%
\pgfusepath{stroke}%
\end{pgfscope}%
\begin{pgfscope}%
\pgfsetbuttcap%
\pgfsetroundjoin%
\definecolor{currentfill}{rgb}{0.000000,0.000000,0.000000}%
\pgfsetfillcolor{currentfill}%
\pgfsetlinewidth{0.602250pt}%
\definecolor{currentstroke}{rgb}{0.000000,0.000000,0.000000}%
\pgfsetstrokecolor{currentstroke}%
\pgfsetdash{}{0pt}%
\pgfsys@defobject{currentmarker}{\pgfqpoint{0.000000in}{-0.027778in}}{\pgfqpoint{0.000000in}{0.000000in}}{%
\pgfpathmoveto{\pgfqpoint{0.000000in}{0.000000in}}%
\pgfpathlineto{\pgfqpoint{0.000000in}{-0.027778in}}%
\pgfusepath{stroke,fill}%
}%
\begin{pgfscope}%
\pgfsys@transformshift{3.141083in}{2.977778in}%
\pgfsys@useobject{currentmarker}{}%
\end{pgfscope}%
\end{pgfscope}%
\begin{pgfscope}%
\pgfsetbuttcap%
\pgfsetroundjoin%
\definecolor{currentfill}{rgb}{0.000000,0.000000,0.000000}%
\pgfsetfillcolor{currentfill}%
\pgfsetlinewidth{0.602250pt}%
\definecolor{currentstroke}{rgb}{0.000000,0.000000,0.000000}%
\pgfsetstrokecolor{currentstroke}%
\pgfsetdash{}{0pt}%
\pgfsys@defobject{currentmarker}{\pgfqpoint{0.000000in}{0.000000in}}{\pgfqpoint{0.000000in}{0.027778in}}{%
\pgfpathmoveto{\pgfqpoint{0.000000in}{0.000000in}}%
\pgfpathlineto{\pgfqpoint{0.000000in}{0.027778in}}%
\pgfusepath{stroke,fill}%
}%
\begin{pgfscope}%
\pgfsys@transformshift{3.141083in}{4.627778in}%
\pgfsys@useobject{currentmarker}{}%
\end{pgfscope}%
\end{pgfscope}%
\begin{pgfscope}%
\pgfpathrectangle{\pgfqpoint{0.781944in}{2.977778in}}{\pgfqpoint{5.019444in}{1.650000in}}%
\pgfusepath{clip}%
\pgfsetrectcap%
\pgfsetroundjoin%
\pgfsetlinewidth{0.803000pt}%
\definecolor{currentstroke}{rgb}{0.690196,0.690196,0.690196}%
\pgfsetstrokecolor{currentstroke}%
\pgfsetstrokeopacity{0.300000}%
\pgfsetdash{}{0pt}%
\pgfpathmoveto{\pgfqpoint{3.191278in}{2.977778in}}%
\pgfpathlineto{\pgfqpoint{3.191278in}{4.627778in}}%
\pgfusepath{stroke}%
\end{pgfscope}%
\begin{pgfscope}%
\pgfsetbuttcap%
\pgfsetroundjoin%
\definecolor{currentfill}{rgb}{0.000000,0.000000,0.000000}%
\pgfsetfillcolor{currentfill}%
\pgfsetlinewidth{0.602250pt}%
\definecolor{currentstroke}{rgb}{0.000000,0.000000,0.000000}%
\pgfsetstrokecolor{currentstroke}%
\pgfsetdash{}{0pt}%
\pgfsys@defobject{currentmarker}{\pgfqpoint{0.000000in}{-0.027778in}}{\pgfqpoint{0.000000in}{0.000000in}}{%
\pgfpathmoveto{\pgfqpoint{0.000000in}{0.000000in}}%
\pgfpathlineto{\pgfqpoint{0.000000in}{-0.027778in}}%
\pgfusepath{stroke,fill}%
}%
\begin{pgfscope}%
\pgfsys@transformshift{3.191278in}{2.977778in}%
\pgfsys@useobject{currentmarker}{}%
\end{pgfscope}%
\end{pgfscope}%
\begin{pgfscope}%
\pgfsetbuttcap%
\pgfsetroundjoin%
\definecolor{currentfill}{rgb}{0.000000,0.000000,0.000000}%
\pgfsetfillcolor{currentfill}%
\pgfsetlinewidth{0.602250pt}%
\definecolor{currentstroke}{rgb}{0.000000,0.000000,0.000000}%
\pgfsetstrokecolor{currentstroke}%
\pgfsetdash{}{0pt}%
\pgfsys@defobject{currentmarker}{\pgfqpoint{0.000000in}{0.000000in}}{\pgfqpoint{0.000000in}{0.027778in}}{%
\pgfpathmoveto{\pgfqpoint{0.000000in}{0.000000in}}%
\pgfpathlineto{\pgfqpoint{0.000000in}{0.027778in}}%
\pgfusepath{stroke,fill}%
}%
\begin{pgfscope}%
\pgfsys@transformshift{3.191278in}{4.627778in}%
\pgfsys@useobject{currentmarker}{}%
\end{pgfscope}%
\end{pgfscope}%
\begin{pgfscope}%
\pgfpathrectangle{\pgfqpoint{0.781944in}{2.977778in}}{\pgfqpoint{5.019444in}{1.650000in}}%
\pgfusepath{clip}%
\pgfsetrectcap%
\pgfsetroundjoin%
\pgfsetlinewidth{0.803000pt}%
\definecolor{currentstroke}{rgb}{0.690196,0.690196,0.690196}%
\pgfsetstrokecolor{currentstroke}%
\pgfsetstrokeopacity{0.300000}%
\pgfsetdash{}{0pt}%
\pgfpathmoveto{\pgfqpoint{3.241472in}{2.977778in}}%
\pgfpathlineto{\pgfqpoint{3.241472in}{4.627778in}}%
\pgfusepath{stroke}%
\end{pgfscope}%
\begin{pgfscope}%
\pgfsetbuttcap%
\pgfsetroundjoin%
\definecolor{currentfill}{rgb}{0.000000,0.000000,0.000000}%
\pgfsetfillcolor{currentfill}%
\pgfsetlinewidth{0.602250pt}%
\definecolor{currentstroke}{rgb}{0.000000,0.000000,0.000000}%
\pgfsetstrokecolor{currentstroke}%
\pgfsetdash{}{0pt}%
\pgfsys@defobject{currentmarker}{\pgfqpoint{0.000000in}{-0.027778in}}{\pgfqpoint{0.000000in}{0.000000in}}{%
\pgfpathmoveto{\pgfqpoint{0.000000in}{0.000000in}}%
\pgfpathlineto{\pgfqpoint{0.000000in}{-0.027778in}}%
\pgfusepath{stroke,fill}%
}%
\begin{pgfscope}%
\pgfsys@transformshift{3.241472in}{2.977778in}%
\pgfsys@useobject{currentmarker}{}%
\end{pgfscope}%
\end{pgfscope}%
\begin{pgfscope}%
\pgfsetbuttcap%
\pgfsetroundjoin%
\definecolor{currentfill}{rgb}{0.000000,0.000000,0.000000}%
\pgfsetfillcolor{currentfill}%
\pgfsetlinewidth{0.602250pt}%
\definecolor{currentstroke}{rgb}{0.000000,0.000000,0.000000}%
\pgfsetstrokecolor{currentstroke}%
\pgfsetdash{}{0pt}%
\pgfsys@defobject{currentmarker}{\pgfqpoint{0.000000in}{0.000000in}}{\pgfqpoint{0.000000in}{0.027778in}}{%
\pgfpathmoveto{\pgfqpoint{0.000000in}{0.000000in}}%
\pgfpathlineto{\pgfqpoint{0.000000in}{0.027778in}}%
\pgfusepath{stroke,fill}%
}%
\begin{pgfscope}%
\pgfsys@transformshift{3.241472in}{4.627778in}%
\pgfsys@useobject{currentmarker}{}%
\end{pgfscope}%
\end{pgfscope}%
\begin{pgfscope}%
\pgfpathrectangle{\pgfqpoint{0.781944in}{2.977778in}}{\pgfqpoint{5.019444in}{1.650000in}}%
\pgfusepath{clip}%
\pgfsetrectcap%
\pgfsetroundjoin%
\pgfsetlinewidth{0.803000pt}%
\definecolor{currentstroke}{rgb}{0.690196,0.690196,0.690196}%
\pgfsetstrokecolor{currentstroke}%
\pgfsetstrokeopacity{0.300000}%
\pgfsetdash{}{0pt}%
\pgfpathmoveto{\pgfqpoint{3.291667in}{2.977778in}}%
\pgfpathlineto{\pgfqpoint{3.291667in}{4.627778in}}%
\pgfusepath{stroke}%
\end{pgfscope}%
\begin{pgfscope}%
\pgfsetbuttcap%
\pgfsetroundjoin%
\definecolor{currentfill}{rgb}{0.000000,0.000000,0.000000}%
\pgfsetfillcolor{currentfill}%
\pgfsetlinewidth{0.602250pt}%
\definecolor{currentstroke}{rgb}{0.000000,0.000000,0.000000}%
\pgfsetstrokecolor{currentstroke}%
\pgfsetdash{}{0pt}%
\pgfsys@defobject{currentmarker}{\pgfqpoint{0.000000in}{-0.027778in}}{\pgfqpoint{0.000000in}{0.000000in}}{%
\pgfpathmoveto{\pgfqpoint{0.000000in}{0.000000in}}%
\pgfpathlineto{\pgfqpoint{0.000000in}{-0.027778in}}%
\pgfusepath{stroke,fill}%
}%
\begin{pgfscope}%
\pgfsys@transformshift{3.291667in}{2.977778in}%
\pgfsys@useobject{currentmarker}{}%
\end{pgfscope}%
\end{pgfscope}%
\begin{pgfscope}%
\pgfsetbuttcap%
\pgfsetroundjoin%
\definecolor{currentfill}{rgb}{0.000000,0.000000,0.000000}%
\pgfsetfillcolor{currentfill}%
\pgfsetlinewidth{0.602250pt}%
\definecolor{currentstroke}{rgb}{0.000000,0.000000,0.000000}%
\pgfsetstrokecolor{currentstroke}%
\pgfsetdash{}{0pt}%
\pgfsys@defobject{currentmarker}{\pgfqpoint{0.000000in}{0.000000in}}{\pgfqpoint{0.000000in}{0.027778in}}{%
\pgfpathmoveto{\pgfqpoint{0.000000in}{0.000000in}}%
\pgfpathlineto{\pgfqpoint{0.000000in}{0.027778in}}%
\pgfusepath{stroke,fill}%
}%
\begin{pgfscope}%
\pgfsys@transformshift{3.291667in}{4.627778in}%
\pgfsys@useobject{currentmarker}{}%
\end{pgfscope}%
\end{pgfscope}%
\begin{pgfscope}%
\pgfpathrectangle{\pgfqpoint{0.781944in}{2.977778in}}{\pgfqpoint{5.019444in}{1.650000in}}%
\pgfusepath{clip}%
\pgfsetrectcap%
\pgfsetroundjoin%
\pgfsetlinewidth{0.803000pt}%
\definecolor{currentstroke}{rgb}{0.690196,0.690196,0.690196}%
\pgfsetstrokecolor{currentstroke}%
\pgfsetstrokeopacity{0.300000}%
\pgfsetdash{}{0pt}%
\pgfpathmoveto{\pgfqpoint{3.341861in}{2.977778in}}%
\pgfpathlineto{\pgfqpoint{3.341861in}{4.627778in}}%
\pgfusepath{stroke}%
\end{pgfscope}%
\begin{pgfscope}%
\pgfsetbuttcap%
\pgfsetroundjoin%
\definecolor{currentfill}{rgb}{0.000000,0.000000,0.000000}%
\pgfsetfillcolor{currentfill}%
\pgfsetlinewidth{0.602250pt}%
\definecolor{currentstroke}{rgb}{0.000000,0.000000,0.000000}%
\pgfsetstrokecolor{currentstroke}%
\pgfsetdash{}{0pt}%
\pgfsys@defobject{currentmarker}{\pgfqpoint{0.000000in}{-0.027778in}}{\pgfqpoint{0.000000in}{0.000000in}}{%
\pgfpathmoveto{\pgfqpoint{0.000000in}{0.000000in}}%
\pgfpathlineto{\pgfqpoint{0.000000in}{-0.027778in}}%
\pgfusepath{stroke,fill}%
}%
\begin{pgfscope}%
\pgfsys@transformshift{3.341861in}{2.977778in}%
\pgfsys@useobject{currentmarker}{}%
\end{pgfscope}%
\end{pgfscope}%
\begin{pgfscope}%
\pgfsetbuttcap%
\pgfsetroundjoin%
\definecolor{currentfill}{rgb}{0.000000,0.000000,0.000000}%
\pgfsetfillcolor{currentfill}%
\pgfsetlinewidth{0.602250pt}%
\definecolor{currentstroke}{rgb}{0.000000,0.000000,0.000000}%
\pgfsetstrokecolor{currentstroke}%
\pgfsetdash{}{0pt}%
\pgfsys@defobject{currentmarker}{\pgfqpoint{0.000000in}{0.000000in}}{\pgfqpoint{0.000000in}{0.027778in}}{%
\pgfpathmoveto{\pgfqpoint{0.000000in}{0.000000in}}%
\pgfpathlineto{\pgfqpoint{0.000000in}{0.027778in}}%
\pgfusepath{stroke,fill}%
}%
\begin{pgfscope}%
\pgfsys@transformshift{3.341861in}{4.627778in}%
\pgfsys@useobject{currentmarker}{}%
\end{pgfscope}%
\end{pgfscope}%
\begin{pgfscope}%
\pgfpathrectangle{\pgfqpoint{0.781944in}{2.977778in}}{\pgfqpoint{5.019444in}{1.650000in}}%
\pgfusepath{clip}%
\pgfsetrectcap%
\pgfsetroundjoin%
\pgfsetlinewidth{0.803000pt}%
\definecolor{currentstroke}{rgb}{0.690196,0.690196,0.690196}%
\pgfsetstrokecolor{currentstroke}%
\pgfsetstrokeopacity{0.300000}%
\pgfsetdash{}{0pt}%
\pgfpathmoveto{\pgfqpoint{3.392056in}{2.977778in}}%
\pgfpathlineto{\pgfqpoint{3.392056in}{4.627778in}}%
\pgfusepath{stroke}%
\end{pgfscope}%
\begin{pgfscope}%
\pgfsetbuttcap%
\pgfsetroundjoin%
\definecolor{currentfill}{rgb}{0.000000,0.000000,0.000000}%
\pgfsetfillcolor{currentfill}%
\pgfsetlinewidth{0.602250pt}%
\definecolor{currentstroke}{rgb}{0.000000,0.000000,0.000000}%
\pgfsetstrokecolor{currentstroke}%
\pgfsetdash{}{0pt}%
\pgfsys@defobject{currentmarker}{\pgfqpoint{0.000000in}{-0.027778in}}{\pgfqpoint{0.000000in}{0.000000in}}{%
\pgfpathmoveto{\pgfqpoint{0.000000in}{0.000000in}}%
\pgfpathlineto{\pgfqpoint{0.000000in}{-0.027778in}}%
\pgfusepath{stroke,fill}%
}%
\begin{pgfscope}%
\pgfsys@transformshift{3.392056in}{2.977778in}%
\pgfsys@useobject{currentmarker}{}%
\end{pgfscope}%
\end{pgfscope}%
\begin{pgfscope}%
\pgfsetbuttcap%
\pgfsetroundjoin%
\definecolor{currentfill}{rgb}{0.000000,0.000000,0.000000}%
\pgfsetfillcolor{currentfill}%
\pgfsetlinewidth{0.602250pt}%
\definecolor{currentstroke}{rgb}{0.000000,0.000000,0.000000}%
\pgfsetstrokecolor{currentstroke}%
\pgfsetdash{}{0pt}%
\pgfsys@defobject{currentmarker}{\pgfqpoint{0.000000in}{0.000000in}}{\pgfqpoint{0.000000in}{0.027778in}}{%
\pgfpathmoveto{\pgfqpoint{0.000000in}{0.000000in}}%
\pgfpathlineto{\pgfqpoint{0.000000in}{0.027778in}}%
\pgfusepath{stroke,fill}%
}%
\begin{pgfscope}%
\pgfsys@transformshift{3.392056in}{4.627778in}%
\pgfsys@useobject{currentmarker}{}%
\end{pgfscope}%
\end{pgfscope}%
\begin{pgfscope}%
\pgfpathrectangle{\pgfqpoint{0.781944in}{2.977778in}}{\pgfqpoint{5.019444in}{1.650000in}}%
\pgfusepath{clip}%
\pgfsetrectcap%
\pgfsetroundjoin%
\pgfsetlinewidth{0.803000pt}%
\definecolor{currentstroke}{rgb}{0.690196,0.690196,0.690196}%
\pgfsetstrokecolor{currentstroke}%
\pgfsetstrokeopacity{0.300000}%
\pgfsetdash{}{0pt}%
\pgfpathmoveto{\pgfqpoint{3.442250in}{2.977778in}}%
\pgfpathlineto{\pgfqpoint{3.442250in}{4.627778in}}%
\pgfusepath{stroke}%
\end{pgfscope}%
\begin{pgfscope}%
\pgfsetbuttcap%
\pgfsetroundjoin%
\definecolor{currentfill}{rgb}{0.000000,0.000000,0.000000}%
\pgfsetfillcolor{currentfill}%
\pgfsetlinewidth{0.602250pt}%
\definecolor{currentstroke}{rgb}{0.000000,0.000000,0.000000}%
\pgfsetstrokecolor{currentstroke}%
\pgfsetdash{}{0pt}%
\pgfsys@defobject{currentmarker}{\pgfqpoint{0.000000in}{-0.027778in}}{\pgfqpoint{0.000000in}{0.000000in}}{%
\pgfpathmoveto{\pgfqpoint{0.000000in}{0.000000in}}%
\pgfpathlineto{\pgfqpoint{0.000000in}{-0.027778in}}%
\pgfusepath{stroke,fill}%
}%
\begin{pgfscope}%
\pgfsys@transformshift{3.442250in}{2.977778in}%
\pgfsys@useobject{currentmarker}{}%
\end{pgfscope}%
\end{pgfscope}%
\begin{pgfscope}%
\pgfsetbuttcap%
\pgfsetroundjoin%
\definecolor{currentfill}{rgb}{0.000000,0.000000,0.000000}%
\pgfsetfillcolor{currentfill}%
\pgfsetlinewidth{0.602250pt}%
\definecolor{currentstroke}{rgb}{0.000000,0.000000,0.000000}%
\pgfsetstrokecolor{currentstroke}%
\pgfsetdash{}{0pt}%
\pgfsys@defobject{currentmarker}{\pgfqpoint{0.000000in}{0.000000in}}{\pgfqpoint{0.000000in}{0.027778in}}{%
\pgfpathmoveto{\pgfqpoint{0.000000in}{0.000000in}}%
\pgfpathlineto{\pgfqpoint{0.000000in}{0.027778in}}%
\pgfusepath{stroke,fill}%
}%
\begin{pgfscope}%
\pgfsys@transformshift{3.442250in}{4.627778in}%
\pgfsys@useobject{currentmarker}{}%
\end{pgfscope}%
\end{pgfscope}%
\begin{pgfscope}%
\pgfpathrectangle{\pgfqpoint{0.781944in}{2.977778in}}{\pgfqpoint{5.019444in}{1.650000in}}%
\pgfusepath{clip}%
\pgfsetrectcap%
\pgfsetroundjoin%
\pgfsetlinewidth{0.803000pt}%
\definecolor{currentstroke}{rgb}{0.690196,0.690196,0.690196}%
\pgfsetstrokecolor{currentstroke}%
\pgfsetstrokeopacity{0.300000}%
\pgfsetdash{}{0pt}%
\pgfpathmoveto{\pgfqpoint{3.492444in}{2.977778in}}%
\pgfpathlineto{\pgfqpoint{3.492444in}{4.627778in}}%
\pgfusepath{stroke}%
\end{pgfscope}%
\begin{pgfscope}%
\pgfsetbuttcap%
\pgfsetroundjoin%
\definecolor{currentfill}{rgb}{0.000000,0.000000,0.000000}%
\pgfsetfillcolor{currentfill}%
\pgfsetlinewidth{0.602250pt}%
\definecolor{currentstroke}{rgb}{0.000000,0.000000,0.000000}%
\pgfsetstrokecolor{currentstroke}%
\pgfsetdash{}{0pt}%
\pgfsys@defobject{currentmarker}{\pgfqpoint{0.000000in}{-0.027778in}}{\pgfqpoint{0.000000in}{0.000000in}}{%
\pgfpathmoveto{\pgfqpoint{0.000000in}{0.000000in}}%
\pgfpathlineto{\pgfqpoint{0.000000in}{-0.027778in}}%
\pgfusepath{stroke,fill}%
}%
\begin{pgfscope}%
\pgfsys@transformshift{3.492444in}{2.977778in}%
\pgfsys@useobject{currentmarker}{}%
\end{pgfscope}%
\end{pgfscope}%
\begin{pgfscope}%
\pgfsetbuttcap%
\pgfsetroundjoin%
\definecolor{currentfill}{rgb}{0.000000,0.000000,0.000000}%
\pgfsetfillcolor{currentfill}%
\pgfsetlinewidth{0.602250pt}%
\definecolor{currentstroke}{rgb}{0.000000,0.000000,0.000000}%
\pgfsetstrokecolor{currentstroke}%
\pgfsetdash{}{0pt}%
\pgfsys@defobject{currentmarker}{\pgfqpoint{0.000000in}{0.000000in}}{\pgfqpoint{0.000000in}{0.027778in}}{%
\pgfpathmoveto{\pgfqpoint{0.000000in}{0.000000in}}%
\pgfpathlineto{\pgfqpoint{0.000000in}{0.027778in}}%
\pgfusepath{stroke,fill}%
}%
\begin{pgfscope}%
\pgfsys@transformshift{3.492444in}{4.627778in}%
\pgfsys@useobject{currentmarker}{}%
\end{pgfscope}%
\end{pgfscope}%
\begin{pgfscope}%
\pgfpathrectangle{\pgfqpoint{0.781944in}{2.977778in}}{\pgfqpoint{5.019444in}{1.650000in}}%
\pgfusepath{clip}%
\pgfsetrectcap%
\pgfsetroundjoin%
\pgfsetlinewidth{0.803000pt}%
\definecolor{currentstroke}{rgb}{0.690196,0.690196,0.690196}%
\pgfsetstrokecolor{currentstroke}%
\pgfsetstrokeopacity{0.300000}%
\pgfsetdash{}{0pt}%
\pgfpathmoveto{\pgfqpoint{3.592833in}{2.977778in}}%
\pgfpathlineto{\pgfqpoint{3.592833in}{4.627778in}}%
\pgfusepath{stroke}%
\end{pgfscope}%
\begin{pgfscope}%
\pgfsetbuttcap%
\pgfsetroundjoin%
\definecolor{currentfill}{rgb}{0.000000,0.000000,0.000000}%
\pgfsetfillcolor{currentfill}%
\pgfsetlinewidth{0.602250pt}%
\definecolor{currentstroke}{rgb}{0.000000,0.000000,0.000000}%
\pgfsetstrokecolor{currentstroke}%
\pgfsetdash{}{0pt}%
\pgfsys@defobject{currentmarker}{\pgfqpoint{0.000000in}{-0.027778in}}{\pgfqpoint{0.000000in}{0.000000in}}{%
\pgfpathmoveto{\pgfqpoint{0.000000in}{0.000000in}}%
\pgfpathlineto{\pgfqpoint{0.000000in}{-0.027778in}}%
\pgfusepath{stroke,fill}%
}%
\begin{pgfscope}%
\pgfsys@transformshift{3.592833in}{2.977778in}%
\pgfsys@useobject{currentmarker}{}%
\end{pgfscope}%
\end{pgfscope}%
\begin{pgfscope}%
\pgfsetbuttcap%
\pgfsetroundjoin%
\definecolor{currentfill}{rgb}{0.000000,0.000000,0.000000}%
\pgfsetfillcolor{currentfill}%
\pgfsetlinewidth{0.602250pt}%
\definecolor{currentstroke}{rgb}{0.000000,0.000000,0.000000}%
\pgfsetstrokecolor{currentstroke}%
\pgfsetdash{}{0pt}%
\pgfsys@defobject{currentmarker}{\pgfqpoint{0.000000in}{0.000000in}}{\pgfqpoint{0.000000in}{0.027778in}}{%
\pgfpathmoveto{\pgfqpoint{0.000000in}{0.000000in}}%
\pgfpathlineto{\pgfqpoint{0.000000in}{0.027778in}}%
\pgfusepath{stroke,fill}%
}%
\begin{pgfscope}%
\pgfsys@transformshift{3.592833in}{4.627778in}%
\pgfsys@useobject{currentmarker}{}%
\end{pgfscope}%
\end{pgfscope}%
\begin{pgfscope}%
\pgfpathrectangle{\pgfqpoint{0.781944in}{2.977778in}}{\pgfqpoint{5.019444in}{1.650000in}}%
\pgfusepath{clip}%
\pgfsetrectcap%
\pgfsetroundjoin%
\pgfsetlinewidth{0.803000pt}%
\definecolor{currentstroke}{rgb}{0.690196,0.690196,0.690196}%
\pgfsetstrokecolor{currentstroke}%
\pgfsetstrokeopacity{0.300000}%
\pgfsetdash{}{0pt}%
\pgfpathmoveto{\pgfqpoint{3.643028in}{2.977778in}}%
\pgfpathlineto{\pgfqpoint{3.643028in}{4.627778in}}%
\pgfusepath{stroke}%
\end{pgfscope}%
\begin{pgfscope}%
\pgfsetbuttcap%
\pgfsetroundjoin%
\definecolor{currentfill}{rgb}{0.000000,0.000000,0.000000}%
\pgfsetfillcolor{currentfill}%
\pgfsetlinewidth{0.602250pt}%
\definecolor{currentstroke}{rgb}{0.000000,0.000000,0.000000}%
\pgfsetstrokecolor{currentstroke}%
\pgfsetdash{}{0pt}%
\pgfsys@defobject{currentmarker}{\pgfqpoint{0.000000in}{-0.027778in}}{\pgfqpoint{0.000000in}{0.000000in}}{%
\pgfpathmoveto{\pgfqpoint{0.000000in}{0.000000in}}%
\pgfpathlineto{\pgfqpoint{0.000000in}{-0.027778in}}%
\pgfusepath{stroke,fill}%
}%
\begin{pgfscope}%
\pgfsys@transformshift{3.643028in}{2.977778in}%
\pgfsys@useobject{currentmarker}{}%
\end{pgfscope}%
\end{pgfscope}%
\begin{pgfscope}%
\pgfsetbuttcap%
\pgfsetroundjoin%
\definecolor{currentfill}{rgb}{0.000000,0.000000,0.000000}%
\pgfsetfillcolor{currentfill}%
\pgfsetlinewidth{0.602250pt}%
\definecolor{currentstroke}{rgb}{0.000000,0.000000,0.000000}%
\pgfsetstrokecolor{currentstroke}%
\pgfsetdash{}{0pt}%
\pgfsys@defobject{currentmarker}{\pgfqpoint{0.000000in}{0.000000in}}{\pgfqpoint{0.000000in}{0.027778in}}{%
\pgfpathmoveto{\pgfqpoint{0.000000in}{0.000000in}}%
\pgfpathlineto{\pgfqpoint{0.000000in}{0.027778in}}%
\pgfusepath{stroke,fill}%
}%
\begin{pgfscope}%
\pgfsys@transformshift{3.643028in}{4.627778in}%
\pgfsys@useobject{currentmarker}{}%
\end{pgfscope}%
\end{pgfscope}%
\begin{pgfscope}%
\pgfpathrectangle{\pgfqpoint{0.781944in}{2.977778in}}{\pgfqpoint{5.019444in}{1.650000in}}%
\pgfusepath{clip}%
\pgfsetrectcap%
\pgfsetroundjoin%
\pgfsetlinewidth{0.803000pt}%
\definecolor{currentstroke}{rgb}{0.690196,0.690196,0.690196}%
\pgfsetstrokecolor{currentstroke}%
\pgfsetstrokeopacity{0.300000}%
\pgfsetdash{}{0pt}%
\pgfpathmoveto{\pgfqpoint{3.693222in}{2.977778in}}%
\pgfpathlineto{\pgfqpoint{3.693222in}{4.627778in}}%
\pgfusepath{stroke}%
\end{pgfscope}%
\begin{pgfscope}%
\pgfsetbuttcap%
\pgfsetroundjoin%
\definecolor{currentfill}{rgb}{0.000000,0.000000,0.000000}%
\pgfsetfillcolor{currentfill}%
\pgfsetlinewidth{0.602250pt}%
\definecolor{currentstroke}{rgb}{0.000000,0.000000,0.000000}%
\pgfsetstrokecolor{currentstroke}%
\pgfsetdash{}{0pt}%
\pgfsys@defobject{currentmarker}{\pgfqpoint{0.000000in}{-0.027778in}}{\pgfqpoint{0.000000in}{0.000000in}}{%
\pgfpathmoveto{\pgfqpoint{0.000000in}{0.000000in}}%
\pgfpathlineto{\pgfqpoint{0.000000in}{-0.027778in}}%
\pgfusepath{stroke,fill}%
}%
\begin{pgfscope}%
\pgfsys@transformshift{3.693222in}{2.977778in}%
\pgfsys@useobject{currentmarker}{}%
\end{pgfscope}%
\end{pgfscope}%
\begin{pgfscope}%
\pgfsetbuttcap%
\pgfsetroundjoin%
\definecolor{currentfill}{rgb}{0.000000,0.000000,0.000000}%
\pgfsetfillcolor{currentfill}%
\pgfsetlinewidth{0.602250pt}%
\definecolor{currentstroke}{rgb}{0.000000,0.000000,0.000000}%
\pgfsetstrokecolor{currentstroke}%
\pgfsetdash{}{0pt}%
\pgfsys@defobject{currentmarker}{\pgfqpoint{0.000000in}{0.000000in}}{\pgfqpoint{0.000000in}{0.027778in}}{%
\pgfpathmoveto{\pgfqpoint{0.000000in}{0.000000in}}%
\pgfpathlineto{\pgfqpoint{0.000000in}{0.027778in}}%
\pgfusepath{stroke,fill}%
}%
\begin{pgfscope}%
\pgfsys@transformshift{3.693222in}{4.627778in}%
\pgfsys@useobject{currentmarker}{}%
\end{pgfscope}%
\end{pgfscope}%
\begin{pgfscope}%
\pgfpathrectangle{\pgfqpoint{0.781944in}{2.977778in}}{\pgfqpoint{5.019444in}{1.650000in}}%
\pgfusepath{clip}%
\pgfsetrectcap%
\pgfsetroundjoin%
\pgfsetlinewidth{0.803000pt}%
\definecolor{currentstroke}{rgb}{0.690196,0.690196,0.690196}%
\pgfsetstrokecolor{currentstroke}%
\pgfsetstrokeopacity{0.300000}%
\pgfsetdash{}{0pt}%
\pgfpathmoveto{\pgfqpoint{3.743417in}{2.977778in}}%
\pgfpathlineto{\pgfqpoint{3.743417in}{4.627778in}}%
\pgfusepath{stroke}%
\end{pgfscope}%
\begin{pgfscope}%
\pgfsetbuttcap%
\pgfsetroundjoin%
\definecolor{currentfill}{rgb}{0.000000,0.000000,0.000000}%
\pgfsetfillcolor{currentfill}%
\pgfsetlinewidth{0.602250pt}%
\definecolor{currentstroke}{rgb}{0.000000,0.000000,0.000000}%
\pgfsetstrokecolor{currentstroke}%
\pgfsetdash{}{0pt}%
\pgfsys@defobject{currentmarker}{\pgfqpoint{0.000000in}{-0.027778in}}{\pgfqpoint{0.000000in}{0.000000in}}{%
\pgfpathmoveto{\pgfqpoint{0.000000in}{0.000000in}}%
\pgfpathlineto{\pgfqpoint{0.000000in}{-0.027778in}}%
\pgfusepath{stroke,fill}%
}%
\begin{pgfscope}%
\pgfsys@transformshift{3.743417in}{2.977778in}%
\pgfsys@useobject{currentmarker}{}%
\end{pgfscope}%
\end{pgfscope}%
\begin{pgfscope}%
\pgfsetbuttcap%
\pgfsetroundjoin%
\definecolor{currentfill}{rgb}{0.000000,0.000000,0.000000}%
\pgfsetfillcolor{currentfill}%
\pgfsetlinewidth{0.602250pt}%
\definecolor{currentstroke}{rgb}{0.000000,0.000000,0.000000}%
\pgfsetstrokecolor{currentstroke}%
\pgfsetdash{}{0pt}%
\pgfsys@defobject{currentmarker}{\pgfqpoint{0.000000in}{0.000000in}}{\pgfqpoint{0.000000in}{0.027778in}}{%
\pgfpathmoveto{\pgfqpoint{0.000000in}{0.000000in}}%
\pgfpathlineto{\pgfqpoint{0.000000in}{0.027778in}}%
\pgfusepath{stroke,fill}%
}%
\begin{pgfscope}%
\pgfsys@transformshift{3.743417in}{4.627778in}%
\pgfsys@useobject{currentmarker}{}%
\end{pgfscope}%
\end{pgfscope}%
\begin{pgfscope}%
\pgfpathrectangle{\pgfqpoint{0.781944in}{2.977778in}}{\pgfqpoint{5.019444in}{1.650000in}}%
\pgfusepath{clip}%
\pgfsetrectcap%
\pgfsetroundjoin%
\pgfsetlinewidth{0.803000pt}%
\definecolor{currentstroke}{rgb}{0.690196,0.690196,0.690196}%
\pgfsetstrokecolor{currentstroke}%
\pgfsetstrokeopacity{0.300000}%
\pgfsetdash{}{0pt}%
\pgfpathmoveto{\pgfqpoint{3.793611in}{2.977778in}}%
\pgfpathlineto{\pgfqpoint{3.793611in}{4.627778in}}%
\pgfusepath{stroke}%
\end{pgfscope}%
\begin{pgfscope}%
\pgfsetbuttcap%
\pgfsetroundjoin%
\definecolor{currentfill}{rgb}{0.000000,0.000000,0.000000}%
\pgfsetfillcolor{currentfill}%
\pgfsetlinewidth{0.602250pt}%
\definecolor{currentstroke}{rgb}{0.000000,0.000000,0.000000}%
\pgfsetstrokecolor{currentstroke}%
\pgfsetdash{}{0pt}%
\pgfsys@defobject{currentmarker}{\pgfqpoint{0.000000in}{-0.027778in}}{\pgfqpoint{0.000000in}{0.000000in}}{%
\pgfpathmoveto{\pgfqpoint{0.000000in}{0.000000in}}%
\pgfpathlineto{\pgfqpoint{0.000000in}{-0.027778in}}%
\pgfusepath{stroke,fill}%
}%
\begin{pgfscope}%
\pgfsys@transformshift{3.793611in}{2.977778in}%
\pgfsys@useobject{currentmarker}{}%
\end{pgfscope}%
\end{pgfscope}%
\begin{pgfscope}%
\pgfsetbuttcap%
\pgfsetroundjoin%
\definecolor{currentfill}{rgb}{0.000000,0.000000,0.000000}%
\pgfsetfillcolor{currentfill}%
\pgfsetlinewidth{0.602250pt}%
\definecolor{currentstroke}{rgb}{0.000000,0.000000,0.000000}%
\pgfsetstrokecolor{currentstroke}%
\pgfsetdash{}{0pt}%
\pgfsys@defobject{currentmarker}{\pgfqpoint{0.000000in}{0.000000in}}{\pgfqpoint{0.000000in}{0.027778in}}{%
\pgfpathmoveto{\pgfqpoint{0.000000in}{0.000000in}}%
\pgfpathlineto{\pgfqpoint{0.000000in}{0.027778in}}%
\pgfusepath{stroke,fill}%
}%
\begin{pgfscope}%
\pgfsys@transformshift{3.793611in}{4.627778in}%
\pgfsys@useobject{currentmarker}{}%
\end{pgfscope}%
\end{pgfscope}%
\begin{pgfscope}%
\pgfpathrectangle{\pgfqpoint{0.781944in}{2.977778in}}{\pgfqpoint{5.019444in}{1.650000in}}%
\pgfusepath{clip}%
\pgfsetrectcap%
\pgfsetroundjoin%
\pgfsetlinewidth{0.803000pt}%
\definecolor{currentstroke}{rgb}{0.690196,0.690196,0.690196}%
\pgfsetstrokecolor{currentstroke}%
\pgfsetstrokeopacity{0.300000}%
\pgfsetdash{}{0pt}%
\pgfpathmoveto{\pgfqpoint{3.843806in}{2.977778in}}%
\pgfpathlineto{\pgfqpoint{3.843806in}{4.627778in}}%
\pgfusepath{stroke}%
\end{pgfscope}%
\begin{pgfscope}%
\pgfsetbuttcap%
\pgfsetroundjoin%
\definecolor{currentfill}{rgb}{0.000000,0.000000,0.000000}%
\pgfsetfillcolor{currentfill}%
\pgfsetlinewidth{0.602250pt}%
\definecolor{currentstroke}{rgb}{0.000000,0.000000,0.000000}%
\pgfsetstrokecolor{currentstroke}%
\pgfsetdash{}{0pt}%
\pgfsys@defobject{currentmarker}{\pgfqpoint{0.000000in}{-0.027778in}}{\pgfqpoint{0.000000in}{0.000000in}}{%
\pgfpathmoveto{\pgfqpoint{0.000000in}{0.000000in}}%
\pgfpathlineto{\pgfqpoint{0.000000in}{-0.027778in}}%
\pgfusepath{stroke,fill}%
}%
\begin{pgfscope}%
\pgfsys@transformshift{3.843806in}{2.977778in}%
\pgfsys@useobject{currentmarker}{}%
\end{pgfscope}%
\end{pgfscope}%
\begin{pgfscope}%
\pgfsetbuttcap%
\pgfsetroundjoin%
\definecolor{currentfill}{rgb}{0.000000,0.000000,0.000000}%
\pgfsetfillcolor{currentfill}%
\pgfsetlinewidth{0.602250pt}%
\definecolor{currentstroke}{rgb}{0.000000,0.000000,0.000000}%
\pgfsetstrokecolor{currentstroke}%
\pgfsetdash{}{0pt}%
\pgfsys@defobject{currentmarker}{\pgfqpoint{0.000000in}{0.000000in}}{\pgfqpoint{0.000000in}{0.027778in}}{%
\pgfpathmoveto{\pgfqpoint{0.000000in}{0.000000in}}%
\pgfpathlineto{\pgfqpoint{0.000000in}{0.027778in}}%
\pgfusepath{stroke,fill}%
}%
\begin{pgfscope}%
\pgfsys@transformshift{3.843806in}{4.627778in}%
\pgfsys@useobject{currentmarker}{}%
\end{pgfscope}%
\end{pgfscope}%
\begin{pgfscope}%
\pgfpathrectangle{\pgfqpoint{0.781944in}{2.977778in}}{\pgfqpoint{5.019444in}{1.650000in}}%
\pgfusepath{clip}%
\pgfsetrectcap%
\pgfsetroundjoin%
\pgfsetlinewidth{0.803000pt}%
\definecolor{currentstroke}{rgb}{0.690196,0.690196,0.690196}%
\pgfsetstrokecolor{currentstroke}%
\pgfsetstrokeopacity{0.300000}%
\pgfsetdash{}{0pt}%
\pgfpathmoveto{\pgfqpoint{3.894000in}{2.977778in}}%
\pgfpathlineto{\pgfqpoint{3.894000in}{4.627778in}}%
\pgfusepath{stroke}%
\end{pgfscope}%
\begin{pgfscope}%
\pgfsetbuttcap%
\pgfsetroundjoin%
\definecolor{currentfill}{rgb}{0.000000,0.000000,0.000000}%
\pgfsetfillcolor{currentfill}%
\pgfsetlinewidth{0.602250pt}%
\definecolor{currentstroke}{rgb}{0.000000,0.000000,0.000000}%
\pgfsetstrokecolor{currentstroke}%
\pgfsetdash{}{0pt}%
\pgfsys@defobject{currentmarker}{\pgfqpoint{0.000000in}{-0.027778in}}{\pgfqpoint{0.000000in}{0.000000in}}{%
\pgfpathmoveto{\pgfqpoint{0.000000in}{0.000000in}}%
\pgfpathlineto{\pgfqpoint{0.000000in}{-0.027778in}}%
\pgfusepath{stroke,fill}%
}%
\begin{pgfscope}%
\pgfsys@transformshift{3.894000in}{2.977778in}%
\pgfsys@useobject{currentmarker}{}%
\end{pgfscope}%
\end{pgfscope}%
\begin{pgfscope}%
\pgfsetbuttcap%
\pgfsetroundjoin%
\definecolor{currentfill}{rgb}{0.000000,0.000000,0.000000}%
\pgfsetfillcolor{currentfill}%
\pgfsetlinewidth{0.602250pt}%
\definecolor{currentstroke}{rgb}{0.000000,0.000000,0.000000}%
\pgfsetstrokecolor{currentstroke}%
\pgfsetdash{}{0pt}%
\pgfsys@defobject{currentmarker}{\pgfqpoint{0.000000in}{0.000000in}}{\pgfqpoint{0.000000in}{0.027778in}}{%
\pgfpathmoveto{\pgfqpoint{0.000000in}{0.000000in}}%
\pgfpathlineto{\pgfqpoint{0.000000in}{0.027778in}}%
\pgfusepath{stroke,fill}%
}%
\begin{pgfscope}%
\pgfsys@transformshift{3.894000in}{4.627778in}%
\pgfsys@useobject{currentmarker}{}%
\end{pgfscope}%
\end{pgfscope}%
\begin{pgfscope}%
\pgfpathrectangle{\pgfqpoint{0.781944in}{2.977778in}}{\pgfqpoint{5.019444in}{1.650000in}}%
\pgfusepath{clip}%
\pgfsetrectcap%
\pgfsetroundjoin%
\pgfsetlinewidth{0.803000pt}%
\definecolor{currentstroke}{rgb}{0.690196,0.690196,0.690196}%
\pgfsetstrokecolor{currentstroke}%
\pgfsetstrokeopacity{0.300000}%
\pgfsetdash{}{0pt}%
\pgfpathmoveto{\pgfqpoint{3.944194in}{2.977778in}}%
\pgfpathlineto{\pgfqpoint{3.944194in}{4.627778in}}%
\pgfusepath{stroke}%
\end{pgfscope}%
\begin{pgfscope}%
\pgfsetbuttcap%
\pgfsetroundjoin%
\definecolor{currentfill}{rgb}{0.000000,0.000000,0.000000}%
\pgfsetfillcolor{currentfill}%
\pgfsetlinewidth{0.602250pt}%
\definecolor{currentstroke}{rgb}{0.000000,0.000000,0.000000}%
\pgfsetstrokecolor{currentstroke}%
\pgfsetdash{}{0pt}%
\pgfsys@defobject{currentmarker}{\pgfqpoint{0.000000in}{-0.027778in}}{\pgfqpoint{0.000000in}{0.000000in}}{%
\pgfpathmoveto{\pgfqpoint{0.000000in}{0.000000in}}%
\pgfpathlineto{\pgfqpoint{0.000000in}{-0.027778in}}%
\pgfusepath{stroke,fill}%
}%
\begin{pgfscope}%
\pgfsys@transformshift{3.944194in}{2.977778in}%
\pgfsys@useobject{currentmarker}{}%
\end{pgfscope}%
\end{pgfscope}%
\begin{pgfscope}%
\pgfsetbuttcap%
\pgfsetroundjoin%
\definecolor{currentfill}{rgb}{0.000000,0.000000,0.000000}%
\pgfsetfillcolor{currentfill}%
\pgfsetlinewidth{0.602250pt}%
\definecolor{currentstroke}{rgb}{0.000000,0.000000,0.000000}%
\pgfsetstrokecolor{currentstroke}%
\pgfsetdash{}{0pt}%
\pgfsys@defobject{currentmarker}{\pgfqpoint{0.000000in}{0.000000in}}{\pgfqpoint{0.000000in}{0.027778in}}{%
\pgfpathmoveto{\pgfqpoint{0.000000in}{0.000000in}}%
\pgfpathlineto{\pgfqpoint{0.000000in}{0.027778in}}%
\pgfusepath{stroke,fill}%
}%
\begin{pgfscope}%
\pgfsys@transformshift{3.944194in}{4.627778in}%
\pgfsys@useobject{currentmarker}{}%
\end{pgfscope}%
\end{pgfscope}%
\begin{pgfscope}%
\pgfpathrectangle{\pgfqpoint{0.781944in}{2.977778in}}{\pgfqpoint{5.019444in}{1.650000in}}%
\pgfusepath{clip}%
\pgfsetrectcap%
\pgfsetroundjoin%
\pgfsetlinewidth{0.803000pt}%
\definecolor{currentstroke}{rgb}{0.690196,0.690196,0.690196}%
\pgfsetstrokecolor{currentstroke}%
\pgfsetstrokeopacity{0.300000}%
\pgfsetdash{}{0pt}%
\pgfpathmoveto{\pgfqpoint{3.994389in}{2.977778in}}%
\pgfpathlineto{\pgfqpoint{3.994389in}{4.627778in}}%
\pgfusepath{stroke}%
\end{pgfscope}%
\begin{pgfscope}%
\pgfsetbuttcap%
\pgfsetroundjoin%
\definecolor{currentfill}{rgb}{0.000000,0.000000,0.000000}%
\pgfsetfillcolor{currentfill}%
\pgfsetlinewidth{0.602250pt}%
\definecolor{currentstroke}{rgb}{0.000000,0.000000,0.000000}%
\pgfsetstrokecolor{currentstroke}%
\pgfsetdash{}{0pt}%
\pgfsys@defobject{currentmarker}{\pgfqpoint{0.000000in}{-0.027778in}}{\pgfqpoint{0.000000in}{0.000000in}}{%
\pgfpathmoveto{\pgfqpoint{0.000000in}{0.000000in}}%
\pgfpathlineto{\pgfqpoint{0.000000in}{-0.027778in}}%
\pgfusepath{stroke,fill}%
}%
\begin{pgfscope}%
\pgfsys@transformshift{3.994389in}{2.977778in}%
\pgfsys@useobject{currentmarker}{}%
\end{pgfscope}%
\end{pgfscope}%
\begin{pgfscope}%
\pgfsetbuttcap%
\pgfsetroundjoin%
\definecolor{currentfill}{rgb}{0.000000,0.000000,0.000000}%
\pgfsetfillcolor{currentfill}%
\pgfsetlinewidth{0.602250pt}%
\definecolor{currentstroke}{rgb}{0.000000,0.000000,0.000000}%
\pgfsetstrokecolor{currentstroke}%
\pgfsetdash{}{0pt}%
\pgfsys@defobject{currentmarker}{\pgfqpoint{0.000000in}{0.000000in}}{\pgfqpoint{0.000000in}{0.027778in}}{%
\pgfpathmoveto{\pgfqpoint{0.000000in}{0.000000in}}%
\pgfpathlineto{\pgfqpoint{0.000000in}{0.027778in}}%
\pgfusepath{stroke,fill}%
}%
\begin{pgfscope}%
\pgfsys@transformshift{3.994389in}{4.627778in}%
\pgfsys@useobject{currentmarker}{}%
\end{pgfscope}%
\end{pgfscope}%
\begin{pgfscope}%
\pgfpathrectangle{\pgfqpoint{0.781944in}{2.977778in}}{\pgfqpoint{5.019444in}{1.650000in}}%
\pgfusepath{clip}%
\pgfsetrectcap%
\pgfsetroundjoin%
\pgfsetlinewidth{0.803000pt}%
\definecolor{currentstroke}{rgb}{0.690196,0.690196,0.690196}%
\pgfsetstrokecolor{currentstroke}%
\pgfsetstrokeopacity{0.300000}%
\pgfsetdash{}{0pt}%
\pgfpathmoveto{\pgfqpoint{4.094778in}{2.977778in}}%
\pgfpathlineto{\pgfqpoint{4.094778in}{4.627778in}}%
\pgfusepath{stroke}%
\end{pgfscope}%
\begin{pgfscope}%
\pgfsetbuttcap%
\pgfsetroundjoin%
\definecolor{currentfill}{rgb}{0.000000,0.000000,0.000000}%
\pgfsetfillcolor{currentfill}%
\pgfsetlinewidth{0.602250pt}%
\definecolor{currentstroke}{rgb}{0.000000,0.000000,0.000000}%
\pgfsetstrokecolor{currentstroke}%
\pgfsetdash{}{0pt}%
\pgfsys@defobject{currentmarker}{\pgfqpoint{0.000000in}{-0.027778in}}{\pgfqpoint{0.000000in}{0.000000in}}{%
\pgfpathmoveto{\pgfqpoint{0.000000in}{0.000000in}}%
\pgfpathlineto{\pgfqpoint{0.000000in}{-0.027778in}}%
\pgfusepath{stroke,fill}%
}%
\begin{pgfscope}%
\pgfsys@transformshift{4.094778in}{2.977778in}%
\pgfsys@useobject{currentmarker}{}%
\end{pgfscope}%
\end{pgfscope}%
\begin{pgfscope}%
\pgfsetbuttcap%
\pgfsetroundjoin%
\definecolor{currentfill}{rgb}{0.000000,0.000000,0.000000}%
\pgfsetfillcolor{currentfill}%
\pgfsetlinewidth{0.602250pt}%
\definecolor{currentstroke}{rgb}{0.000000,0.000000,0.000000}%
\pgfsetstrokecolor{currentstroke}%
\pgfsetdash{}{0pt}%
\pgfsys@defobject{currentmarker}{\pgfqpoint{0.000000in}{0.000000in}}{\pgfqpoint{0.000000in}{0.027778in}}{%
\pgfpathmoveto{\pgfqpoint{0.000000in}{0.000000in}}%
\pgfpathlineto{\pgfqpoint{0.000000in}{0.027778in}}%
\pgfusepath{stroke,fill}%
}%
\begin{pgfscope}%
\pgfsys@transformshift{4.094778in}{4.627778in}%
\pgfsys@useobject{currentmarker}{}%
\end{pgfscope}%
\end{pgfscope}%
\begin{pgfscope}%
\pgfpathrectangle{\pgfqpoint{0.781944in}{2.977778in}}{\pgfqpoint{5.019444in}{1.650000in}}%
\pgfusepath{clip}%
\pgfsetrectcap%
\pgfsetroundjoin%
\pgfsetlinewidth{0.803000pt}%
\definecolor{currentstroke}{rgb}{0.690196,0.690196,0.690196}%
\pgfsetstrokecolor{currentstroke}%
\pgfsetstrokeopacity{0.300000}%
\pgfsetdash{}{0pt}%
\pgfpathmoveto{\pgfqpoint{4.144972in}{2.977778in}}%
\pgfpathlineto{\pgfqpoint{4.144972in}{4.627778in}}%
\pgfusepath{stroke}%
\end{pgfscope}%
\begin{pgfscope}%
\pgfsetbuttcap%
\pgfsetroundjoin%
\definecolor{currentfill}{rgb}{0.000000,0.000000,0.000000}%
\pgfsetfillcolor{currentfill}%
\pgfsetlinewidth{0.602250pt}%
\definecolor{currentstroke}{rgb}{0.000000,0.000000,0.000000}%
\pgfsetstrokecolor{currentstroke}%
\pgfsetdash{}{0pt}%
\pgfsys@defobject{currentmarker}{\pgfqpoint{0.000000in}{-0.027778in}}{\pgfqpoint{0.000000in}{0.000000in}}{%
\pgfpathmoveto{\pgfqpoint{0.000000in}{0.000000in}}%
\pgfpathlineto{\pgfqpoint{0.000000in}{-0.027778in}}%
\pgfusepath{stroke,fill}%
}%
\begin{pgfscope}%
\pgfsys@transformshift{4.144972in}{2.977778in}%
\pgfsys@useobject{currentmarker}{}%
\end{pgfscope}%
\end{pgfscope}%
\begin{pgfscope}%
\pgfsetbuttcap%
\pgfsetroundjoin%
\definecolor{currentfill}{rgb}{0.000000,0.000000,0.000000}%
\pgfsetfillcolor{currentfill}%
\pgfsetlinewidth{0.602250pt}%
\definecolor{currentstroke}{rgb}{0.000000,0.000000,0.000000}%
\pgfsetstrokecolor{currentstroke}%
\pgfsetdash{}{0pt}%
\pgfsys@defobject{currentmarker}{\pgfqpoint{0.000000in}{0.000000in}}{\pgfqpoint{0.000000in}{0.027778in}}{%
\pgfpathmoveto{\pgfqpoint{0.000000in}{0.000000in}}%
\pgfpathlineto{\pgfqpoint{0.000000in}{0.027778in}}%
\pgfusepath{stroke,fill}%
}%
\begin{pgfscope}%
\pgfsys@transformshift{4.144972in}{4.627778in}%
\pgfsys@useobject{currentmarker}{}%
\end{pgfscope}%
\end{pgfscope}%
\begin{pgfscope}%
\pgfpathrectangle{\pgfqpoint{0.781944in}{2.977778in}}{\pgfqpoint{5.019444in}{1.650000in}}%
\pgfusepath{clip}%
\pgfsetrectcap%
\pgfsetroundjoin%
\pgfsetlinewidth{0.803000pt}%
\definecolor{currentstroke}{rgb}{0.690196,0.690196,0.690196}%
\pgfsetstrokecolor{currentstroke}%
\pgfsetstrokeopacity{0.300000}%
\pgfsetdash{}{0pt}%
\pgfpathmoveto{\pgfqpoint{4.195167in}{2.977778in}}%
\pgfpathlineto{\pgfqpoint{4.195167in}{4.627778in}}%
\pgfusepath{stroke}%
\end{pgfscope}%
\begin{pgfscope}%
\pgfsetbuttcap%
\pgfsetroundjoin%
\definecolor{currentfill}{rgb}{0.000000,0.000000,0.000000}%
\pgfsetfillcolor{currentfill}%
\pgfsetlinewidth{0.602250pt}%
\definecolor{currentstroke}{rgb}{0.000000,0.000000,0.000000}%
\pgfsetstrokecolor{currentstroke}%
\pgfsetdash{}{0pt}%
\pgfsys@defobject{currentmarker}{\pgfqpoint{0.000000in}{-0.027778in}}{\pgfqpoint{0.000000in}{0.000000in}}{%
\pgfpathmoveto{\pgfqpoint{0.000000in}{0.000000in}}%
\pgfpathlineto{\pgfqpoint{0.000000in}{-0.027778in}}%
\pgfusepath{stroke,fill}%
}%
\begin{pgfscope}%
\pgfsys@transformshift{4.195167in}{2.977778in}%
\pgfsys@useobject{currentmarker}{}%
\end{pgfscope}%
\end{pgfscope}%
\begin{pgfscope}%
\pgfsetbuttcap%
\pgfsetroundjoin%
\definecolor{currentfill}{rgb}{0.000000,0.000000,0.000000}%
\pgfsetfillcolor{currentfill}%
\pgfsetlinewidth{0.602250pt}%
\definecolor{currentstroke}{rgb}{0.000000,0.000000,0.000000}%
\pgfsetstrokecolor{currentstroke}%
\pgfsetdash{}{0pt}%
\pgfsys@defobject{currentmarker}{\pgfqpoint{0.000000in}{0.000000in}}{\pgfqpoint{0.000000in}{0.027778in}}{%
\pgfpathmoveto{\pgfqpoint{0.000000in}{0.000000in}}%
\pgfpathlineto{\pgfqpoint{0.000000in}{0.027778in}}%
\pgfusepath{stroke,fill}%
}%
\begin{pgfscope}%
\pgfsys@transformshift{4.195167in}{4.627778in}%
\pgfsys@useobject{currentmarker}{}%
\end{pgfscope}%
\end{pgfscope}%
\begin{pgfscope}%
\pgfpathrectangle{\pgfqpoint{0.781944in}{2.977778in}}{\pgfqpoint{5.019444in}{1.650000in}}%
\pgfusepath{clip}%
\pgfsetrectcap%
\pgfsetroundjoin%
\pgfsetlinewidth{0.803000pt}%
\definecolor{currentstroke}{rgb}{0.690196,0.690196,0.690196}%
\pgfsetstrokecolor{currentstroke}%
\pgfsetstrokeopacity{0.300000}%
\pgfsetdash{}{0pt}%
\pgfpathmoveto{\pgfqpoint{4.245361in}{2.977778in}}%
\pgfpathlineto{\pgfqpoint{4.245361in}{4.627778in}}%
\pgfusepath{stroke}%
\end{pgfscope}%
\begin{pgfscope}%
\pgfsetbuttcap%
\pgfsetroundjoin%
\definecolor{currentfill}{rgb}{0.000000,0.000000,0.000000}%
\pgfsetfillcolor{currentfill}%
\pgfsetlinewidth{0.602250pt}%
\definecolor{currentstroke}{rgb}{0.000000,0.000000,0.000000}%
\pgfsetstrokecolor{currentstroke}%
\pgfsetdash{}{0pt}%
\pgfsys@defobject{currentmarker}{\pgfqpoint{0.000000in}{-0.027778in}}{\pgfqpoint{0.000000in}{0.000000in}}{%
\pgfpathmoveto{\pgfqpoint{0.000000in}{0.000000in}}%
\pgfpathlineto{\pgfqpoint{0.000000in}{-0.027778in}}%
\pgfusepath{stroke,fill}%
}%
\begin{pgfscope}%
\pgfsys@transformshift{4.245361in}{2.977778in}%
\pgfsys@useobject{currentmarker}{}%
\end{pgfscope}%
\end{pgfscope}%
\begin{pgfscope}%
\pgfsetbuttcap%
\pgfsetroundjoin%
\definecolor{currentfill}{rgb}{0.000000,0.000000,0.000000}%
\pgfsetfillcolor{currentfill}%
\pgfsetlinewidth{0.602250pt}%
\definecolor{currentstroke}{rgb}{0.000000,0.000000,0.000000}%
\pgfsetstrokecolor{currentstroke}%
\pgfsetdash{}{0pt}%
\pgfsys@defobject{currentmarker}{\pgfqpoint{0.000000in}{0.000000in}}{\pgfqpoint{0.000000in}{0.027778in}}{%
\pgfpathmoveto{\pgfqpoint{0.000000in}{0.000000in}}%
\pgfpathlineto{\pgfqpoint{0.000000in}{0.027778in}}%
\pgfusepath{stroke,fill}%
}%
\begin{pgfscope}%
\pgfsys@transformshift{4.245361in}{4.627778in}%
\pgfsys@useobject{currentmarker}{}%
\end{pgfscope}%
\end{pgfscope}%
\begin{pgfscope}%
\pgfpathrectangle{\pgfqpoint{0.781944in}{2.977778in}}{\pgfqpoint{5.019444in}{1.650000in}}%
\pgfusepath{clip}%
\pgfsetrectcap%
\pgfsetroundjoin%
\pgfsetlinewidth{0.803000pt}%
\definecolor{currentstroke}{rgb}{0.690196,0.690196,0.690196}%
\pgfsetstrokecolor{currentstroke}%
\pgfsetstrokeopacity{0.300000}%
\pgfsetdash{}{0pt}%
\pgfpathmoveto{\pgfqpoint{4.295556in}{2.977778in}}%
\pgfpathlineto{\pgfqpoint{4.295556in}{4.627778in}}%
\pgfusepath{stroke}%
\end{pgfscope}%
\begin{pgfscope}%
\pgfsetbuttcap%
\pgfsetroundjoin%
\definecolor{currentfill}{rgb}{0.000000,0.000000,0.000000}%
\pgfsetfillcolor{currentfill}%
\pgfsetlinewidth{0.602250pt}%
\definecolor{currentstroke}{rgb}{0.000000,0.000000,0.000000}%
\pgfsetstrokecolor{currentstroke}%
\pgfsetdash{}{0pt}%
\pgfsys@defobject{currentmarker}{\pgfqpoint{0.000000in}{-0.027778in}}{\pgfqpoint{0.000000in}{0.000000in}}{%
\pgfpathmoveto{\pgfqpoint{0.000000in}{0.000000in}}%
\pgfpathlineto{\pgfqpoint{0.000000in}{-0.027778in}}%
\pgfusepath{stroke,fill}%
}%
\begin{pgfscope}%
\pgfsys@transformshift{4.295556in}{2.977778in}%
\pgfsys@useobject{currentmarker}{}%
\end{pgfscope}%
\end{pgfscope}%
\begin{pgfscope}%
\pgfsetbuttcap%
\pgfsetroundjoin%
\definecolor{currentfill}{rgb}{0.000000,0.000000,0.000000}%
\pgfsetfillcolor{currentfill}%
\pgfsetlinewidth{0.602250pt}%
\definecolor{currentstroke}{rgb}{0.000000,0.000000,0.000000}%
\pgfsetstrokecolor{currentstroke}%
\pgfsetdash{}{0pt}%
\pgfsys@defobject{currentmarker}{\pgfqpoint{0.000000in}{0.000000in}}{\pgfqpoint{0.000000in}{0.027778in}}{%
\pgfpathmoveto{\pgfqpoint{0.000000in}{0.000000in}}%
\pgfpathlineto{\pgfqpoint{0.000000in}{0.027778in}}%
\pgfusepath{stroke,fill}%
}%
\begin{pgfscope}%
\pgfsys@transformshift{4.295556in}{4.627778in}%
\pgfsys@useobject{currentmarker}{}%
\end{pgfscope}%
\end{pgfscope}%
\begin{pgfscope}%
\pgfpathrectangle{\pgfqpoint{0.781944in}{2.977778in}}{\pgfqpoint{5.019444in}{1.650000in}}%
\pgfusepath{clip}%
\pgfsetrectcap%
\pgfsetroundjoin%
\pgfsetlinewidth{0.803000pt}%
\definecolor{currentstroke}{rgb}{0.690196,0.690196,0.690196}%
\pgfsetstrokecolor{currentstroke}%
\pgfsetstrokeopacity{0.300000}%
\pgfsetdash{}{0pt}%
\pgfpathmoveto{\pgfqpoint{4.345750in}{2.977778in}}%
\pgfpathlineto{\pgfqpoint{4.345750in}{4.627778in}}%
\pgfusepath{stroke}%
\end{pgfscope}%
\begin{pgfscope}%
\pgfsetbuttcap%
\pgfsetroundjoin%
\definecolor{currentfill}{rgb}{0.000000,0.000000,0.000000}%
\pgfsetfillcolor{currentfill}%
\pgfsetlinewidth{0.602250pt}%
\definecolor{currentstroke}{rgb}{0.000000,0.000000,0.000000}%
\pgfsetstrokecolor{currentstroke}%
\pgfsetdash{}{0pt}%
\pgfsys@defobject{currentmarker}{\pgfqpoint{0.000000in}{-0.027778in}}{\pgfqpoint{0.000000in}{0.000000in}}{%
\pgfpathmoveto{\pgfqpoint{0.000000in}{0.000000in}}%
\pgfpathlineto{\pgfqpoint{0.000000in}{-0.027778in}}%
\pgfusepath{stroke,fill}%
}%
\begin{pgfscope}%
\pgfsys@transformshift{4.345750in}{2.977778in}%
\pgfsys@useobject{currentmarker}{}%
\end{pgfscope}%
\end{pgfscope}%
\begin{pgfscope}%
\pgfsetbuttcap%
\pgfsetroundjoin%
\definecolor{currentfill}{rgb}{0.000000,0.000000,0.000000}%
\pgfsetfillcolor{currentfill}%
\pgfsetlinewidth{0.602250pt}%
\definecolor{currentstroke}{rgb}{0.000000,0.000000,0.000000}%
\pgfsetstrokecolor{currentstroke}%
\pgfsetdash{}{0pt}%
\pgfsys@defobject{currentmarker}{\pgfqpoint{0.000000in}{0.000000in}}{\pgfqpoint{0.000000in}{0.027778in}}{%
\pgfpathmoveto{\pgfqpoint{0.000000in}{0.000000in}}%
\pgfpathlineto{\pgfqpoint{0.000000in}{0.027778in}}%
\pgfusepath{stroke,fill}%
}%
\begin{pgfscope}%
\pgfsys@transformshift{4.345750in}{4.627778in}%
\pgfsys@useobject{currentmarker}{}%
\end{pgfscope}%
\end{pgfscope}%
\begin{pgfscope}%
\pgfpathrectangle{\pgfqpoint{0.781944in}{2.977778in}}{\pgfqpoint{5.019444in}{1.650000in}}%
\pgfusepath{clip}%
\pgfsetrectcap%
\pgfsetroundjoin%
\pgfsetlinewidth{0.803000pt}%
\definecolor{currentstroke}{rgb}{0.690196,0.690196,0.690196}%
\pgfsetstrokecolor{currentstroke}%
\pgfsetstrokeopacity{0.300000}%
\pgfsetdash{}{0pt}%
\pgfpathmoveto{\pgfqpoint{4.395944in}{2.977778in}}%
\pgfpathlineto{\pgfqpoint{4.395944in}{4.627778in}}%
\pgfusepath{stroke}%
\end{pgfscope}%
\begin{pgfscope}%
\pgfsetbuttcap%
\pgfsetroundjoin%
\definecolor{currentfill}{rgb}{0.000000,0.000000,0.000000}%
\pgfsetfillcolor{currentfill}%
\pgfsetlinewidth{0.602250pt}%
\definecolor{currentstroke}{rgb}{0.000000,0.000000,0.000000}%
\pgfsetstrokecolor{currentstroke}%
\pgfsetdash{}{0pt}%
\pgfsys@defobject{currentmarker}{\pgfqpoint{0.000000in}{-0.027778in}}{\pgfqpoint{0.000000in}{0.000000in}}{%
\pgfpathmoveto{\pgfqpoint{0.000000in}{0.000000in}}%
\pgfpathlineto{\pgfqpoint{0.000000in}{-0.027778in}}%
\pgfusepath{stroke,fill}%
}%
\begin{pgfscope}%
\pgfsys@transformshift{4.395944in}{2.977778in}%
\pgfsys@useobject{currentmarker}{}%
\end{pgfscope}%
\end{pgfscope}%
\begin{pgfscope}%
\pgfsetbuttcap%
\pgfsetroundjoin%
\definecolor{currentfill}{rgb}{0.000000,0.000000,0.000000}%
\pgfsetfillcolor{currentfill}%
\pgfsetlinewidth{0.602250pt}%
\definecolor{currentstroke}{rgb}{0.000000,0.000000,0.000000}%
\pgfsetstrokecolor{currentstroke}%
\pgfsetdash{}{0pt}%
\pgfsys@defobject{currentmarker}{\pgfqpoint{0.000000in}{0.000000in}}{\pgfqpoint{0.000000in}{0.027778in}}{%
\pgfpathmoveto{\pgfqpoint{0.000000in}{0.000000in}}%
\pgfpathlineto{\pgfqpoint{0.000000in}{0.027778in}}%
\pgfusepath{stroke,fill}%
}%
\begin{pgfscope}%
\pgfsys@transformshift{4.395944in}{4.627778in}%
\pgfsys@useobject{currentmarker}{}%
\end{pgfscope}%
\end{pgfscope}%
\begin{pgfscope}%
\pgfpathrectangle{\pgfqpoint{0.781944in}{2.977778in}}{\pgfqpoint{5.019444in}{1.650000in}}%
\pgfusepath{clip}%
\pgfsetrectcap%
\pgfsetroundjoin%
\pgfsetlinewidth{0.803000pt}%
\definecolor{currentstroke}{rgb}{0.690196,0.690196,0.690196}%
\pgfsetstrokecolor{currentstroke}%
\pgfsetstrokeopacity{0.300000}%
\pgfsetdash{}{0pt}%
\pgfpathmoveto{\pgfqpoint{4.446139in}{2.977778in}}%
\pgfpathlineto{\pgfqpoint{4.446139in}{4.627778in}}%
\pgfusepath{stroke}%
\end{pgfscope}%
\begin{pgfscope}%
\pgfsetbuttcap%
\pgfsetroundjoin%
\definecolor{currentfill}{rgb}{0.000000,0.000000,0.000000}%
\pgfsetfillcolor{currentfill}%
\pgfsetlinewidth{0.602250pt}%
\definecolor{currentstroke}{rgb}{0.000000,0.000000,0.000000}%
\pgfsetstrokecolor{currentstroke}%
\pgfsetdash{}{0pt}%
\pgfsys@defobject{currentmarker}{\pgfqpoint{0.000000in}{-0.027778in}}{\pgfqpoint{0.000000in}{0.000000in}}{%
\pgfpathmoveto{\pgfqpoint{0.000000in}{0.000000in}}%
\pgfpathlineto{\pgfqpoint{0.000000in}{-0.027778in}}%
\pgfusepath{stroke,fill}%
}%
\begin{pgfscope}%
\pgfsys@transformshift{4.446139in}{2.977778in}%
\pgfsys@useobject{currentmarker}{}%
\end{pgfscope}%
\end{pgfscope}%
\begin{pgfscope}%
\pgfsetbuttcap%
\pgfsetroundjoin%
\definecolor{currentfill}{rgb}{0.000000,0.000000,0.000000}%
\pgfsetfillcolor{currentfill}%
\pgfsetlinewidth{0.602250pt}%
\definecolor{currentstroke}{rgb}{0.000000,0.000000,0.000000}%
\pgfsetstrokecolor{currentstroke}%
\pgfsetdash{}{0pt}%
\pgfsys@defobject{currentmarker}{\pgfqpoint{0.000000in}{0.000000in}}{\pgfqpoint{0.000000in}{0.027778in}}{%
\pgfpathmoveto{\pgfqpoint{0.000000in}{0.000000in}}%
\pgfpathlineto{\pgfqpoint{0.000000in}{0.027778in}}%
\pgfusepath{stroke,fill}%
}%
\begin{pgfscope}%
\pgfsys@transformshift{4.446139in}{4.627778in}%
\pgfsys@useobject{currentmarker}{}%
\end{pgfscope}%
\end{pgfscope}%
\begin{pgfscope}%
\pgfpathrectangle{\pgfqpoint{0.781944in}{2.977778in}}{\pgfqpoint{5.019444in}{1.650000in}}%
\pgfusepath{clip}%
\pgfsetrectcap%
\pgfsetroundjoin%
\pgfsetlinewidth{0.803000pt}%
\definecolor{currentstroke}{rgb}{0.690196,0.690196,0.690196}%
\pgfsetstrokecolor{currentstroke}%
\pgfsetstrokeopacity{0.300000}%
\pgfsetdash{}{0pt}%
\pgfpathmoveto{\pgfqpoint{4.496333in}{2.977778in}}%
\pgfpathlineto{\pgfqpoint{4.496333in}{4.627778in}}%
\pgfusepath{stroke}%
\end{pgfscope}%
\begin{pgfscope}%
\pgfsetbuttcap%
\pgfsetroundjoin%
\definecolor{currentfill}{rgb}{0.000000,0.000000,0.000000}%
\pgfsetfillcolor{currentfill}%
\pgfsetlinewidth{0.602250pt}%
\definecolor{currentstroke}{rgb}{0.000000,0.000000,0.000000}%
\pgfsetstrokecolor{currentstroke}%
\pgfsetdash{}{0pt}%
\pgfsys@defobject{currentmarker}{\pgfqpoint{0.000000in}{-0.027778in}}{\pgfqpoint{0.000000in}{0.000000in}}{%
\pgfpathmoveto{\pgfqpoint{0.000000in}{0.000000in}}%
\pgfpathlineto{\pgfqpoint{0.000000in}{-0.027778in}}%
\pgfusepath{stroke,fill}%
}%
\begin{pgfscope}%
\pgfsys@transformshift{4.496333in}{2.977778in}%
\pgfsys@useobject{currentmarker}{}%
\end{pgfscope}%
\end{pgfscope}%
\begin{pgfscope}%
\pgfsetbuttcap%
\pgfsetroundjoin%
\definecolor{currentfill}{rgb}{0.000000,0.000000,0.000000}%
\pgfsetfillcolor{currentfill}%
\pgfsetlinewidth{0.602250pt}%
\definecolor{currentstroke}{rgb}{0.000000,0.000000,0.000000}%
\pgfsetstrokecolor{currentstroke}%
\pgfsetdash{}{0pt}%
\pgfsys@defobject{currentmarker}{\pgfqpoint{0.000000in}{0.000000in}}{\pgfqpoint{0.000000in}{0.027778in}}{%
\pgfpathmoveto{\pgfqpoint{0.000000in}{0.000000in}}%
\pgfpathlineto{\pgfqpoint{0.000000in}{0.027778in}}%
\pgfusepath{stroke,fill}%
}%
\begin{pgfscope}%
\pgfsys@transformshift{4.496333in}{4.627778in}%
\pgfsys@useobject{currentmarker}{}%
\end{pgfscope}%
\end{pgfscope}%
\begin{pgfscope}%
\pgfpathrectangle{\pgfqpoint{0.781944in}{2.977778in}}{\pgfqpoint{5.019444in}{1.650000in}}%
\pgfusepath{clip}%
\pgfsetrectcap%
\pgfsetroundjoin%
\pgfsetlinewidth{0.803000pt}%
\definecolor{currentstroke}{rgb}{0.690196,0.690196,0.690196}%
\pgfsetstrokecolor{currentstroke}%
\pgfsetstrokeopacity{0.300000}%
\pgfsetdash{}{0pt}%
\pgfpathmoveto{\pgfqpoint{4.596722in}{2.977778in}}%
\pgfpathlineto{\pgfqpoint{4.596722in}{4.627778in}}%
\pgfusepath{stroke}%
\end{pgfscope}%
\begin{pgfscope}%
\pgfsetbuttcap%
\pgfsetroundjoin%
\definecolor{currentfill}{rgb}{0.000000,0.000000,0.000000}%
\pgfsetfillcolor{currentfill}%
\pgfsetlinewidth{0.602250pt}%
\definecolor{currentstroke}{rgb}{0.000000,0.000000,0.000000}%
\pgfsetstrokecolor{currentstroke}%
\pgfsetdash{}{0pt}%
\pgfsys@defobject{currentmarker}{\pgfqpoint{0.000000in}{-0.027778in}}{\pgfqpoint{0.000000in}{0.000000in}}{%
\pgfpathmoveto{\pgfqpoint{0.000000in}{0.000000in}}%
\pgfpathlineto{\pgfqpoint{0.000000in}{-0.027778in}}%
\pgfusepath{stroke,fill}%
}%
\begin{pgfscope}%
\pgfsys@transformshift{4.596722in}{2.977778in}%
\pgfsys@useobject{currentmarker}{}%
\end{pgfscope}%
\end{pgfscope}%
\begin{pgfscope}%
\pgfsetbuttcap%
\pgfsetroundjoin%
\definecolor{currentfill}{rgb}{0.000000,0.000000,0.000000}%
\pgfsetfillcolor{currentfill}%
\pgfsetlinewidth{0.602250pt}%
\definecolor{currentstroke}{rgb}{0.000000,0.000000,0.000000}%
\pgfsetstrokecolor{currentstroke}%
\pgfsetdash{}{0pt}%
\pgfsys@defobject{currentmarker}{\pgfqpoint{0.000000in}{0.000000in}}{\pgfqpoint{0.000000in}{0.027778in}}{%
\pgfpathmoveto{\pgfqpoint{0.000000in}{0.000000in}}%
\pgfpathlineto{\pgfqpoint{0.000000in}{0.027778in}}%
\pgfusepath{stroke,fill}%
}%
\begin{pgfscope}%
\pgfsys@transformshift{4.596722in}{4.627778in}%
\pgfsys@useobject{currentmarker}{}%
\end{pgfscope}%
\end{pgfscope}%
\begin{pgfscope}%
\pgfpathrectangle{\pgfqpoint{0.781944in}{2.977778in}}{\pgfqpoint{5.019444in}{1.650000in}}%
\pgfusepath{clip}%
\pgfsetrectcap%
\pgfsetroundjoin%
\pgfsetlinewidth{0.803000pt}%
\definecolor{currentstroke}{rgb}{0.690196,0.690196,0.690196}%
\pgfsetstrokecolor{currentstroke}%
\pgfsetstrokeopacity{0.300000}%
\pgfsetdash{}{0pt}%
\pgfpathmoveto{\pgfqpoint{4.646917in}{2.977778in}}%
\pgfpathlineto{\pgfqpoint{4.646917in}{4.627778in}}%
\pgfusepath{stroke}%
\end{pgfscope}%
\begin{pgfscope}%
\pgfsetbuttcap%
\pgfsetroundjoin%
\definecolor{currentfill}{rgb}{0.000000,0.000000,0.000000}%
\pgfsetfillcolor{currentfill}%
\pgfsetlinewidth{0.602250pt}%
\definecolor{currentstroke}{rgb}{0.000000,0.000000,0.000000}%
\pgfsetstrokecolor{currentstroke}%
\pgfsetdash{}{0pt}%
\pgfsys@defobject{currentmarker}{\pgfqpoint{0.000000in}{-0.027778in}}{\pgfqpoint{0.000000in}{0.000000in}}{%
\pgfpathmoveto{\pgfqpoint{0.000000in}{0.000000in}}%
\pgfpathlineto{\pgfqpoint{0.000000in}{-0.027778in}}%
\pgfusepath{stroke,fill}%
}%
\begin{pgfscope}%
\pgfsys@transformshift{4.646917in}{2.977778in}%
\pgfsys@useobject{currentmarker}{}%
\end{pgfscope}%
\end{pgfscope}%
\begin{pgfscope}%
\pgfsetbuttcap%
\pgfsetroundjoin%
\definecolor{currentfill}{rgb}{0.000000,0.000000,0.000000}%
\pgfsetfillcolor{currentfill}%
\pgfsetlinewidth{0.602250pt}%
\definecolor{currentstroke}{rgb}{0.000000,0.000000,0.000000}%
\pgfsetstrokecolor{currentstroke}%
\pgfsetdash{}{0pt}%
\pgfsys@defobject{currentmarker}{\pgfqpoint{0.000000in}{0.000000in}}{\pgfqpoint{0.000000in}{0.027778in}}{%
\pgfpathmoveto{\pgfqpoint{0.000000in}{0.000000in}}%
\pgfpathlineto{\pgfqpoint{0.000000in}{0.027778in}}%
\pgfusepath{stroke,fill}%
}%
\begin{pgfscope}%
\pgfsys@transformshift{4.646917in}{4.627778in}%
\pgfsys@useobject{currentmarker}{}%
\end{pgfscope}%
\end{pgfscope}%
\begin{pgfscope}%
\pgfpathrectangle{\pgfqpoint{0.781944in}{2.977778in}}{\pgfqpoint{5.019444in}{1.650000in}}%
\pgfusepath{clip}%
\pgfsetrectcap%
\pgfsetroundjoin%
\pgfsetlinewidth{0.803000pt}%
\definecolor{currentstroke}{rgb}{0.690196,0.690196,0.690196}%
\pgfsetstrokecolor{currentstroke}%
\pgfsetstrokeopacity{0.300000}%
\pgfsetdash{}{0pt}%
\pgfpathmoveto{\pgfqpoint{4.697111in}{2.977778in}}%
\pgfpathlineto{\pgfqpoint{4.697111in}{4.627778in}}%
\pgfusepath{stroke}%
\end{pgfscope}%
\begin{pgfscope}%
\pgfsetbuttcap%
\pgfsetroundjoin%
\definecolor{currentfill}{rgb}{0.000000,0.000000,0.000000}%
\pgfsetfillcolor{currentfill}%
\pgfsetlinewidth{0.602250pt}%
\definecolor{currentstroke}{rgb}{0.000000,0.000000,0.000000}%
\pgfsetstrokecolor{currentstroke}%
\pgfsetdash{}{0pt}%
\pgfsys@defobject{currentmarker}{\pgfqpoint{0.000000in}{-0.027778in}}{\pgfqpoint{0.000000in}{0.000000in}}{%
\pgfpathmoveto{\pgfqpoint{0.000000in}{0.000000in}}%
\pgfpathlineto{\pgfqpoint{0.000000in}{-0.027778in}}%
\pgfusepath{stroke,fill}%
}%
\begin{pgfscope}%
\pgfsys@transformshift{4.697111in}{2.977778in}%
\pgfsys@useobject{currentmarker}{}%
\end{pgfscope}%
\end{pgfscope}%
\begin{pgfscope}%
\pgfsetbuttcap%
\pgfsetroundjoin%
\definecolor{currentfill}{rgb}{0.000000,0.000000,0.000000}%
\pgfsetfillcolor{currentfill}%
\pgfsetlinewidth{0.602250pt}%
\definecolor{currentstroke}{rgb}{0.000000,0.000000,0.000000}%
\pgfsetstrokecolor{currentstroke}%
\pgfsetdash{}{0pt}%
\pgfsys@defobject{currentmarker}{\pgfqpoint{0.000000in}{0.000000in}}{\pgfqpoint{0.000000in}{0.027778in}}{%
\pgfpathmoveto{\pgfqpoint{0.000000in}{0.000000in}}%
\pgfpathlineto{\pgfqpoint{0.000000in}{0.027778in}}%
\pgfusepath{stroke,fill}%
}%
\begin{pgfscope}%
\pgfsys@transformshift{4.697111in}{4.627778in}%
\pgfsys@useobject{currentmarker}{}%
\end{pgfscope}%
\end{pgfscope}%
\begin{pgfscope}%
\pgfpathrectangle{\pgfqpoint{0.781944in}{2.977778in}}{\pgfqpoint{5.019444in}{1.650000in}}%
\pgfusepath{clip}%
\pgfsetrectcap%
\pgfsetroundjoin%
\pgfsetlinewidth{0.803000pt}%
\definecolor{currentstroke}{rgb}{0.690196,0.690196,0.690196}%
\pgfsetstrokecolor{currentstroke}%
\pgfsetstrokeopacity{0.300000}%
\pgfsetdash{}{0pt}%
\pgfpathmoveto{\pgfqpoint{4.747306in}{2.977778in}}%
\pgfpathlineto{\pgfqpoint{4.747306in}{4.627778in}}%
\pgfusepath{stroke}%
\end{pgfscope}%
\begin{pgfscope}%
\pgfsetbuttcap%
\pgfsetroundjoin%
\definecolor{currentfill}{rgb}{0.000000,0.000000,0.000000}%
\pgfsetfillcolor{currentfill}%
\pgfsetlinewidth{0.602250pt}%
\definecolor{currentstroke}{rgb}{0.000000,0.000000,0.000000}%
\pgfsetstrokecolor{currentstroke}%
\pgfsetdash{}{0pt}%
\pgfsys@defobject{currentmarker}{\pgfqpoint{0.000000in}{-0.027778in}}{\pgfqpoint{0.000000in}{0.000000in}}{%
\pgfpathmoveto{\pgfqpoint{0.000000in}{0.000000in}}%
\pgfpathlineto{\pgfqpoint{0.000000in}{-0.027778in}}%
\pgfusepath{stroke,fill}%
}%
\begin{pgfscope}%
\pgfsys@transformshift{4.747306in}{2.977778in}%
\pgfsys@useobject{currentmarker}{}%
\end{pgfscope}%
\end{pgfscope}%
\begin{pgfscope}%
\pgfsetbuttcap%
\pgfsetroundjoin%
\definecolor{currentfill}{rgb}{0.000000,0.000000,0.000000}%
\pgfsetfillcolor{currentfill}%
\pgfsetlinewidth{0.602250pt}%
\definecolor{currentstroke}{rgb}{0.000000,0.000000,0.000000}%
\pgfsetstrokecolor{currentstroke}%
\pgfsetdash{}{0pt}%
\pgfsys@defobject{currentmarker}{\pgfqpoint{0.000000in}{0.000000in}}{\pgfqpoint{0.000000in}{0.027778in}}{%
\pgfpathmoveto{\pgfqpoint{0.000000in}{0.000000in}}%
\pgfpathlineto{\pgfqpoint{0.000000in}{0.027778in}}%
\pgfusepath{stroke,fill}%
}%
\begin{pgfscope}%
\pgfsys@transformshift{4.747306in}{4.627778in}%
\pgfsys@useobject{currentmarker}{}%
\end{pgfscope}%
\end{pgfscope}%
\begin{pgfscope}%
\pgfpathrectangle{\pgfqpoint{0.781944in}{2.977778in}}{\pgfqpoint{5.019444in}{1.650000in}}%
\pgfusepath{clip}%
\pgfsetrectcap%
\pgfsetroundjoin%
\pgfsetlinewidth{0.803000pt}%
\definecolor{currentstroke}{rgb}{0.690196,0.690196,0.690196}%
\pgfsetstrokecolor{currentstroke}%
\pgfsetstrokeopacity{0.300000}%
\pgfsetdash{}{0pt}%
\pgfpathmoveto{\pgfqpoint{4.797500in}{2.977778in}}%
\pgfpathlineto{\pgfqpoint{4.797500in}{4.627778in}}%
\pgfusepath{stroke}%
\end{pgfscope}%
\begin{pgfscope}%
\pgfsetbuttcap%
\pgfsetroundjoin%
\definecolor{currentfill}{rgb}{0.000000,0.000000,0.000000}%
\pgfsetfillcolor{currentfill}%
\pgfsetlinewidth{0.602250pt}%
\definecolor{currentstroke}{rgb}{0.000000,0.000000,0.000000}%
\pgfsetstrokecolor{currentstroke}%
\pgfsetdash{}{0pt}%
\pgfsys@defobject{currentmarker}{\pgfqpoint{0.000000in}{-0.027778in}}{\pgfqpoint{0.000000in}{0.000000in}}{%
\pgfpathmoveto{\pgfqpoint{0.000000in}{0.000000in}}%
\pgfpathlineto{\pgfqpoint{0.000000in}{-0.027778in}}%
\pgfusepath{stroke,fill}%
}%
\begin{pgfscope}%
\pgfsys@transformshift{4.797500in}{2.977778in}%
\pgfsys@useobject{currentmarker}{}%
\end{pgfscope}%
\end{pgfscope}%
\begin{pgfscope}%
\pgfsetbuttcap%
\pgfsetroundjoin%
\definecolor{currentfill}{rgb}{0.000000,0.000000,0.000000}%
\pgfsetfillcolor{currentfill}%
\pgfsetlinewidth{0.602250pt}%
\definecolor{currentstroke}{rgb}{0.000000,0.000000,0.000000}%
\pgfsetstrokecolor{currentstroke}%
\pgfsetdash{}{0pt}%
\pgfsys@defobject{currentmarker}{\pgfqpoint{0.000000in}{0.000000in}}{\pgfqpoint{0.000000in}{0.027778in}}{%
\pgfpathmoveto{\pgfqpoint{0.000000in}{0.000000in}}%
\pgfpathlineto{\pgfqpoint{0.000000in}{0.027778in}}%
\pgfusepath{stroke,fill}%
}%
\begin{pgfscope}%
\pgfsys@transformshift{4.797500in}{4.627778in}%
\pgfsys@useobject{currentmarker}{}%
\end{pgfscope}%
\end{pgfscope}%
\begin{pgfscope}%
\pgfpathrectangle{\pgfqpoint{0.781944in}{2.977778in}}{\pgfqpoint{5.019444in}{1.650000in}}%
\pgfusepath{clip}%
\pgfsetrectcap%
\pgfsetroundjoin%
\pgfsetlinewidth{0.803000pt}%
\definecolor{currentstroke}{rgb}{0.690196,0.690196,0.690196}%
\pgfsetstrokecolor{currentstroke}%
\pgfsetstrokeopacity{0.300000}%
\pgfsetdash{}{0pt}%
\pgfpathmoveto{\pgfqpoint{4.847694in}{2.977778in}}%
\pgfpathlineto{\pgfqpoint{4.847694in}{4.627778in}}%
\pgfusepath{stroke}%
\end{pgfscope}%
\begin{pgfscope}%
\pgfsetbuttcap%
\pgfsetroundjoin%
\definecolor{currentfill}{rgb}{0.000000,0.000000,0.000000}%
\pgfsetfillcolor{currentfill}%
\pgfsetlinewidth{0.602250pt}%
\definecolor{currentstroke}{rgb}{0.000000,0.000000,0.000000}%
\pgfsetstrokecolor{currentstroke}%
\pgfsetdash{}{0pt}%
\pgfsys@defobject{currentmarker}{\pgfqpoint{0.000000in}{-0.027778in}}{\pgfqpoint{0.000000in}{0.000000in}}{%
\pgfpathmoveto{\pgfqpoint{0.000000in}{0.000000in}}%
\pgfpathlineto{\pgfqpoint{0.000000in}{-0.027778in}}%
\pgfusepath{stroke,fill}%
}%
\begin{pgfscope}%
\pgfsys@transformshift{4.847694in}{2.977778in}%
\pgfsys@useobject{currentmarker}{}%
\end{pgfscope}%
\end{pgfscope}%
\begin{pgfscope}%
\pgfsetbuttcap%
\pgfsetroundjoin%
\definecolor{currentfill}{rgb}{0.000000,0.000000,0.000000}%
\pgfsetfillcolor{currentfill}%
\pgfsetlinewidth{0.602250pt}%
\definecolor{currentstroke}{rgb}{0.000000,0.000000,0.000000}%
\pgfsetstrokecolor{currentstroke}%
\pgfsetdash{}{0pt}%
\pgfsys@defobject{currentmarker}{\pgfqpoint{0.000000in}{0.000000in}}{\pgfqpoint{0.000000in}{0.027778in}}{%
\pgfpathmoveto{\pgfqpoint{0.000000in}{0.000000in}}%
\pgfpathlineto{\pgfqpoint{0.000000in}{0.027778in}}%
\pgfusepath{stroke,fill}%
}%
\begin{pgfscope}%
\pgfsys@transformshift{4.847694in}{4.627778in}%
\pgfsys@useobject{currentmarker}{}%
\end{pgfscope}%
\end{pgfscope}%
\begin{pgfscope}%
\pgfpathrectangle{\pgfqpoint{0.781944in}{2.977778in}}{\pgfqpoint{5.019444in}{1.650000in}}%
\pgfusepath{clip}%
\pgfsetrectcap%
\pgfsetroundjoin%
\pgfsetlinewidth{0.803000pt}%
\definecolor{currentstroke}{rgb}{0.690196,0.690196,0.690196}%
\pgfsetstrokecolor{currentstroke}%
\pgfsetstrokeopacity{0.300000}%
\pgfsetdash{}{0pt}%
\pgfpathmoveto{\pgfqpoint{4.897889in}{2.977778in}}%
\pgfpathlineto{\pgfqpoint{4.897889in}{4.627778in}}%
\pgfusepath{stroke}%
\end{pgfscope}%
\begin{pgfscope}%
\pgfsetbuttcap%
\pgfsetroundjoin%
\definecolor{currentfill}{rgb}{0.000000,0.000000,0.000000}%
\pgfsetfillcolor{currentfill}%
\pgfsetlinewidth{0.602250pt}%
\definecolor{currentstroke}{rgb}{0.000000,0.000000,0.000000}%
\pgfsetstrokecolor{currentstroke}%
\pgfsetdash{}{0pt}%
\pgfsys@defobject{currentmarker}{\pgfqpoint{0.000000in}{-0.027778in}}{\pgfqpoint{0.000000in}{0.000000in}}{%
\pgfpathmoveto{\pgfqpoint{0.000000in}{0.000000in}}%
\pgfpathlineto{\pgfqpoint{0.000000in}{-0.027778in}}%
\pgfusepath{stroke,fill}%
}%
\begin{pgfscope}%
\pgfsys@transformshift{4.897889in}{2.977778in}%
\pgfsys@useobject{currentmarker}{}%
\end{pgfscope}%
\end{pgfscope}%
\begin{pgfscope}%
\pgfsetbuttcap%
\pgfsetroundjoin%
\definecolor{currentfill}{rgb}{0.000000,0.000000,0.000000}%
\pgfsetfillcolor{currentfill}%
\pgfsetlinewidth{0.602250pt}%
\definecolor{currentstroke}{rgb}{0.000000,0.000000,0.000000}%
\pgfsetstrokecolor{currentstroke}%
\pgfsetdash{}{0pt}%
\pgfsys@defobject{currentmarker}{\pgfqpoint{0.000000in}{0.000000in}}{\pgfqpoint{0.000000in}{0.027778in}}{%
\pgfpathmoveto{\pgfqpoint{0.000000in}{0.000000in}}%
\pgfpathlineto{\pgfqpoint{0.000000in}{0.027778in}}%
\pgfusepath{stroke,fill}%
}%
\begin{pgfscope}%
\pgfsys@transformshift{4.897889in}{4.627778in}%
\pgfsys@useobject{currentmarker}{}%
\end{pgfscope}%
\end{pgfscope}%
\begin{pgfscope}%
\pgfpathrectangle{\pgfqpoint{0.781944in}{2.977778in}}{\pgfqpoint{5.019444in}{1.650000in}}%
\pgfusepath{clip}%
\pgfsetrectcap%
\pgfsetroundjoin%
\pgfsetlinewidth{0.803000pt}%
\definecolor{currentstroke}{rgb}{0.690196,0.690196,0.690196}%
\pgfsetstrokecolor{currentstroke}%
\pgfsetstrokeopacity{0.300000}%
\pgfsetdash{}{0pt}%
\pgfpathmoveto{\pgfqpoint{4.948083in}{2.977778in}}%
\pgfpathlineto{\pgfqpoint{4.948083in}{4.627778in}}%
\pgfusepath{stroke}%
\end{pgfscope}%
\begin{pgfscope}%
\pgfsetbuttcap%
\pgfsetroundjoin%
\definecolor{currentfill}{rgb}{0.000000,0.000000,0.000000}%
\pgfsetfillcolor{currentfill}%
\pgfsetlinewidth{0.602250pt}%
\definecolor{currentstroke}{rgb}{0.000000,0.000000,0.000000}%
\pgfsetstrokecolor{currentstroke}%
\pgfsetdash{}{0pt}%
\pgfsys@defobject{currentmarker}{\pgfqpoint{0.000000in}{-0.027778in}}{\pgfqpoint{0.000000in}{0.000000in}}{%
\pgfpathmoveto{\pgfqpoint{0.000000in}{0.000000in}}%
\pgfpathlineto{\pgfqpoint{0.000000in}{-0.027778in}}%
\pgfusepath{stroke,fill}%
}%
\begin{pgfscope}%
\pgfsys@transformshift{4.948083in}{2.977778in}%
\pgfsys@useobject{currentmarker}{}%
\end{pgfscope}%
\end{pgfscope}%
\begin{pgfscope}%
\pgfsetbuttcap%
\pgfsetroundjoin%
\definecolor{currentfill}{rgb}{0.000000,0.000000,0.000000}%
\pgfsetfillcolor{currentfill}%
\pgfsetlinewidth{0.602250pt}%
\definecolor{currentstroke}{rgb}{0.000000,0.000000,0.000000}%
\pgfsetstrokecolor{currentstroke}%
\pgfsetdash{}{0pt}%
\pgfsys@defobject{currentmarker}{\pgfqpoint{0.000000in}{0.000000in}}{\pgfqpoint{0.000000in}{0.027778in}}{%
\pgfpathmoveto{\pgfqpoint{0.000000in}{0.000000in}}%
\pgfpathlineto{\pgfqpoint{0.000000in}{0.027778in}}%
\pgfusepath{stroke,fill}%
}%
\begin{pgfscope}%
\pgfsys@transformshift{4.948083in}{4.627778in}%
\pgfsys@useobject{currentmarker}{}%
\end{pgfscope}%
\end{pgfscope}%
\begin{pgfscope}%
\pgfpathrectangle{\pgfqpoint{0.781944in}{2.977778in}}{\pgfqpoint{5.019444in}{1.650000in}}%
\pgfusepath{clip}%
\pgfsetrectcap%
\pgfsetroundjoin%
\pgfsetlinewidth{0.803000pt}%
\definecolor{currentstroke}{rgb}{0.690196,0.690196,0.690196}%
\pgfsetstrokecolor{currentstroke}%
\pgfsetstrokeopacity{0.300000}%
\pgfsetdash{}{0pt}%
\pgfpathmoveto{\pgfqpoint{4.998278in}{2.977778in}}%
\pgfpathlineto{\pgfqpoint{4.998278in}{4.627778in}}%
\pgfusepath{stroke}%
\end{pgfscope}%
\begin{pgfscope}%
\pgfsetbuttcap%
\pgfsetroundjoin%
\definecolor{currentfill}{rgb}{0.000000,0.000000,0.000000}%
\pgfsetfillcolor{currentfill}%
\pgfsetlinewidth{0.602250pt}%
\definecolor{currentstroke}{rgb}{0.000000,0.000000,0.000000}%
\pgfsetstrokecolor{currentstroke}%
\pgfsetdash{}{0pt}%
\pgfsys@defobject{currentmarker}{\pgfqpoint{0.000000in}{-0.027778in}}{\pgfqpoint{0.000000in}{0.000000in}}{%
\pgfpathmoveto{\pgfqpoint{0.000000in}{0.000000in}}%
\pgfpathlineto{\pgfqpoint{0.000000in}{-0.027778in}}%
\pgfusepath{stroke,fill}%
}%
\begin{pgfscope}%
\pgfsys@transformshift{4.998278in}{2.977778in}%
\pgfsys@useobject{currentmarker}{}%
\end{pgfscope}%
\end{pgfscope}%
\begin{pgfscope}%
\pgfsetbuttcap%
\pgfsetroundjoin%
\definecolor{currentfill}{rgb}{0.000000,0.000000,0.000000}%
\pgfsetfillcolor{currentfill}%
\pgfsetlinewidth{0.602250pt}%
\definecolor{currentstroke}{rgb}{0.000000,0.000000,0.000000}%
\pgfsetstrokecolor{currentstroke}%
\pgfsetdash{}{0pt}%
\pgfsys@defobject{currentmarker}{\pgfqpoint{0.000000in}{0.000000in}}{\pgfqpoint{0.000000in}{0.027778in}}{%
\pgfpathmoveto{\pgfqpoint{0.000000in}{0.000000in}}%
\pgfpathlineto{\pgfqpoint{0.000000in}{0.027778in}}%
\pgfusepath{stroke,fill}%
}%
\begin{pgfscope}%
\pgfsys@transformshift{4.998278in}{4.627778in}%
\pgfsys@useobject{currentmarker}{}%
\end{pgfscope}%
\end{pgfscope}%
\begin{pgfscope}%
\pgfpathrectangle{\pgfqpoint{0.781944in}{2.977778in}}{\pgfqpoint{5.019444in}{1.650000in}}%
\pgfusepath{clip}%
\pgfsetrectcap%
\pgfsetroundjoin%
\pgfsetlinewidth{0.803000pt}%
\definecolor{currentstroke}{rgb}{0.690196,0.690196,0.690196}%
\pgfsetstrokecolor{currentstroke}%
\pgfsetstrokeopacity{0.300000}%
\pgfsetdash{}{0pt}%
\pgfpathmoveto{\pgfqpoint{5.098667in}{2.977778in}}%
\pgfpathlineto{\pgfqpoint{5.098667in}{4.627778in}}%
\pgfusepath{stroke}%
\end{pgfscope}%
\begin{pgfscope}%
\pgfsetbuttcap%
\pgfsetroundjoin%
\definecolor{currentfill}{rgb}{0.000000,0.000000,0.000000}%
\pgfsetfillcolor{currentfill}%
\pgfsetlinewidth{0.602250pt}%
\definecolor{currentstroke}{rgb}{0.000000,0.000000,0.000000}%
\pgfsetstrokecolor{currentstroke}%
\pgfsetdash{}{0pt}%
\pgfsys@defobject{currentmarker}{\pgfqpoint{0.000000in}{-0.027778in}}{\pgfqpoint{0.000000in}{0.000000in}}{%
\pgfpathmoveto{\pgfqpoint{0.000000in}{0.000000in}}%
\pgfpathlineto{\pgfqpoint{0.000000in}{-0.027778in}}%
\pgfusepath{stroke,fill}%
}%
\begin{pgfscope}%
\pgfsys@transformshift{5.098667in}{2.977778in}%
\pgfsys@useobject{currentmarker}{}%
\end{pgfscope}%
\end{pgfscope}%
\begin{pgfscope}%
\pgfsetbuttcap%
\pgfsetroundjoin%
\definecolor{currentfill}{rgb}{0.000000,0.000000,0.000000}%
\pgfsetfillcolor{currentfill}%
\pgfsetlinewidth{0.602250pt}%
\definecolor{currentstroke}{rgb}{0.000000,0.000000,0.000000}%
\pgfsetstrokecolor{currentstroke}%
\pgfsetdash{}{0pt}%
\pgfsys@defobject{currentmarker}{\pgfqpoint{0.000000in}{0.000000in}}{\pgfqpoint{0.000000in}{0.027778in}}{%
\pgfpathmoveto{\pgfqpoint{0.000000in}{0.000000in}}%
\pgfpathlineto{\pgfqpoint{0.000000in}{0.027778in}}%
\pgfusepath{stroke,fill}%
}%
\begin{pgfscope}%
\pgfsys@transformshift{5.098667in}{4.627778in}%
\pgfsys@useobject{currentmarker}{}%
\end{pgfscope}%
\end{pgfscope}%
\begin{pgfscope}%
\pgfpathrectangle{\pgfqpoint{0.781944in}{2.977778in}}{\pgfqpoint{5.019444in}{1.650000in}}%
\pgfusepath{clip}%
\pgfsetrectcap%
\pgfsetroundjoin%
\pgfsetlinewidth{0.803000pt}%
\definecolor{currentstroke}{rgb}{0.690196,0.690196,0.690196}%
\pgfsetstrokecolor{currentstroke}%
\pgfsetstrokeopacity{0.300000}%
\pgfsetdash{}{0pt}%
\pgfpathmoveto{\pgfqpoint{5.148861in}{2.977778in}}%
\pgfpathlineto{\pgfqpoint{5.148861in}{4.627778in}}%
\pgfusepath{stroke}%
\end{pgfscope}%
\begin{pgfscope}%
\pgfsetbuttcap%
\pgfsetroundjoin%
\definecolor{currentfill}{rgb}{0.000000,0.000000,0.000000}%
\pgfsetfillcolor{currentfill}%
\pgfsetlinewidth{0.602250pt}%
\definecolor{currentstroke}{rgb}{0.000000,0.000000,0.000000}%
\pgfsetstrokecolor{currentstroke}%
\pgfsetdash{}{0pt}%
\pgfsys@defobject{currentmarker}{\pgfqpoint{0.000000in}{-0.027778in}}{\pgfqpoint{0.000000in}{0.000000in}}{%
\pgfpathmoveto{\pgfqpoint{0.000000in}{0.000000in}}%
\pgfpathlineto{\pgfqpoint{0.000000in}{-0.027778in}}%
\pgfusepath{stroke,fill}%
}%
\begin{pgfscope}%
\pgfsys@transformshift{5.148861in}{2.977778in}%
\pgfsys@useobject{currentmarker}{}%
\end{pgfscope}%
\end{pgfscope}%
\begin{pgfscope}%
\pgfsetbuttcap%
\pgfsetroundjoin%
\definecolor{currentfill}{rgb}{0.000000,0.000000,0.000000}%
\pgfsetfillcolor{currentfill}%
\pgfsetlinewidth{0.602250pt}%
\definecolor{currentstroke}{rgb}{0.000000,0.000000,0.000000}%
\pgfsetstrokecolor{currentstroke}%
\pgfsetdash{}{0pt}%
\pgfsys@defobject{currentmarker}{\pgfqpoint{0.000000in}{0.000000in}}{\pgfqpoint{0.000000in}{0.027778in}}{%
\pgfpathmoveto{\pgfqpoint{0.000000in}{0.000000in}}%
\pgfpathlineto{\pgfqpoint{0.000000in}{0.027778in}}%
\pgfusepath{stroke,fill}%
}%
\begin{pgfscope}%
\pgfsys@transformshift{5.148861in}{4.627778in}%
\pgfsys@useobject{currentmarker}{}%
\end{pgfscope}%
\end{pgfscope}%
\begin{pgfscope}%
\pgfpathrectangle{\pgfqpoint{0.781944in}{2.977778in}}{\pgfqpoint{5.019444in}{1.650000in}}%
\pgfusepath{clip}%
\pgfsetrectcap%
\pgfsetroundjoin%
\pgfsetlinewidth{0.803000pt}%
\definecolor{currentstroke}{rgb}{0.690196,0.690196,0.690196}%
\pgfsetstrokecolor{currentstroke}%
\pgfsetstrokeopacity{0.300000}%
\pgfsetdash{}{0pt}%
\pgfpathmoveto{\pgfqpoint{5.199056in}{2.977778in}}%
\pgfpathlineto{\pgfqpoint{5.199056in}{4.627778in}}%
\pgfusepath{stroke}%
\end{pgfscope}%
\begin{pgfscope}%
\pgfsetbuttcap%
\pgfsetroundjoin%
\definecolor{currentfill}{rgb}{0.000000,0.000000,0.000000}%
\pgfsetfillcolor{currentfill}%
\pgfsetlinewidth{0.602250pt}%
\definecolor{currentstroke}{rgb}{0.000000,0.000000,0.000000}%
\pgfsetstrokecolor{currentstroke}%
\pgfsetdash{}{0pt}%
\pgfsys@defobject{currentmarker}{\pgfqpoint{0.000000in}{-0.027778in}}{\pgfqpoint{0.000000in}{0.000000in}}{%
\pgfpathmoveto{\pgfqpoint{0.000000in}{0.000000in}}%
\pgfpathlineto{\pgfqpoint{0.000000in}{-0.027778in}}%
\pgfusepath{stroke,fill}%
}%
\begin{pgfscope}%
\pgfsys@transformshift{5.199056in}{2.977778in}%
\pgfsys@useobject{currentmarker}{}%
\end{pgfscope}%
\end{pgfscope}%
\begin{pgfscope}%
\pgfsetbuttcap%
\pgfsetroundjoin%
\definecolor{currentfill}{rgb}{0.000000,0.000000,0.000000}%
\pgfsetfillcolor{currentfill}%
\pgfsetlinewidth{0.602250pt}%
\definecolor{currentstroke}{rgb}{0.000000,0.000000,0.000000}%
\pgfsetstrokecolor{currentstroke}%
\pgfsetdash{}{0pt}%
\pgfsys@defobject{currentmarker}{\pgfqpoint{0.000000in}{0.000000in}}{\pgfqpoint{0.000000in}{0.027778in}}{%
\pgfpathmoveto{\pgfqpoint{0.000000in}{0.000000in}}%
\pgfpathlineto{\pgfqpoint{0.000000in}{0.027778in}}%
\pgfusepath{stroke,fill}%
}%
\begin{pgfscope}%
\pgfsys@transformshift{5.199056in}{4.627778in}%
\pgfsys@useobject{currentmarker}{}%
\end{pgfscope}%
\end{pgfscope}%
\begin{pgfscope}%
\pgfpathrectangle{\pgfqpoint{0.781944in}{2.977778in}}{\pgfqpoint{5.019444in}{1.650000in}}%
\pgfusepath{clip}%
\pgfsetrectcap%
\pgfsetroundjoin%
\pgfsetlinewidth{0.803000pt}%
\definecolor{currentstroke}{rgb}{0.690196,0.690196,0.690196}%
\pgfsetstrokecolor{currentstroke}%
\pgfsetstrokeopacity{0.300000}%
\pgfsetdash{}{0pt}%
\pgfpathmoveto{\pgfqpoint{5.249250in}{2.977778in}}%
\pgfpathlineto{\pgfqpoint{5.249250in}{4.627778in}}%
\pgfusepath{stroke}%
\end{pgfscope}%
\begin{pgfscope}%
\pgfsetbuttcap%
\pgfsetroundjoin%
\definecolor{currentfill}{rgb}{0.000000,0.000000,0.000000}%
\pgfsetfillcolor{currentfill}%
\pgfsetlinewidth{0.602250pt}%
\definecolor{currentstroke}{rgb}{0.000000,0.000000,0.000000}%
\pgfsetstrokecolor{currentstroke}%
\pgfsetdash{}{0pt}%
\pgfsys@defobject{currentmarker}{\pgfqpoint{0.000000in}{-0.027778in}}{\pgfqpoint{0.000000in}{0.000000in}}{%
\pgfpathmoveto{\pgfqpoint{0.000000in}{0.000000in}}%
\pgfpathlineto{\pgfqpoint{0.000000in}{-0.027778in}}%
\pgfusepath{stroke,fill}%
}%
\begin{pgfscope}%
\pgfsys@transformshift{5.249250in}{2.977778in}%
\pgfsys@useobject{currentmarker}{}%
\end{pgfscope}%
\end{pgfscope}%
\begin{pgfscope}%
\pgfsetbuttcap%
\pgfsetroundjoin%
\definecolor{currentfill}{rgb}{0.000000,0.000000,0.000000}%
\pgfsetfillcolor{currentfill}%
\pgfsetlinewidth{0.602250pt}%
\definecolor{currentstroke}{rgb}{0.000000,0.000000,0.000000}%
\pgfsetstrokecolor{currentstroke}%
\pgfsetdash{}{0pt}%
\pgfsys@defobject{currentmarker}{\pgfqpoint{0.000000in}{0.000000in}}{\pgfqpoint{0.000000in}{0.027778in}}{%
\pgfpathmoveto{\pgfqpoint{0.000000in}{0.000000in}}%
\pgfpathlineto{\pgfqpoint{0.000000in}{0.027778in}}%
\pgfusepath{stroke,fill}%
}%
\begin{pgfscope}%
\pgfsys@transformshift{5.249250in}{4.627778in}%
\pgfsys@useobject{currentmarker}{}%
\end{pgfscope}%
\end{pgfscope}%
\begin{pgfscope}%
\pgfpathrectangle{\pgfqpoint{0.781944in}{2.977778in}}{\pgfqpoint{5.019444in}{1.650000in}}%
\pgfusepath{clip}%
\pgfsetrectcap%
\pgfsetroundjoin%
\pgfsetlinewidth{0.803000pt}%
\definecolor{currentstroke}{rgb}{0.690196,0.690196,0.690196}%
\pgfsetstrokecolor{currentstroke}%
\pgfsetstrokeopacity{0.300000}%
\pgfsetdash{}{0pt}%
\pgfpathmoveto{\pgfqpoint{5.299444in}{2.977778in}}%
\pgfpathlineto{\pgfqpoint{5.299444in}{4.627778in}}%
\pgfusepath{stroke}%
\end{pgfscope}%
\begin{pgfscope}%
\pgfsetbuttcap%
\pgfsetroundjoin%
\definecolor{currentfill}{rgb}{0.000000,0.000000,0.000000}%
\pgfsetfillcolor{currentfill}%
\pgfsetlinewidth{0.602250pt}%
\definecolor{currentstroke}{rgb}{0.000000,0.000000,0.000000}%
\pgfsetstrokecolor{currentstroke}%
\pgfsetdash{}{0pt}%
\pgfsys@defobject{currentmarker}{\pgfqpoint{0.000000in}{-0.027778in}}{\pgfqpoint{0.000000in}{0.000000in}}{%
\pgfpathmoveto{\pgfqpoint{0.000000in}{0.000000in}}%
\pgfpathlineto{\pgfqpoint{0.000000in}{-0.027778in}}%
\pgfusepath{stroke,fill}%
}%
\begin{pgfscope}%
\pgfsys@transformshift{5.299444in}{2.977778in}%
\pgfsys@useobject{currentmarker}{}%
\end{pgfscope}%
\end{pgfscope}%
\begin{pgfscope}%
\pgfsetbuttcap%
\pgfsetroundjoin%
\definecolor{currentfill}{rgb}{0.000000,0.000000,0.000000}%
\pgfsetfillcolor{currentfill}%
\pgfsetlinewidth{0.602250pt}%
\definecolor{currentstroke}{rgb}{0.000000,0.000000,0.000000}%
\pgfsetstrokecolor{currentstroke}%
\pgfsetdash{}{0pt}%
\pgfsys@defobject{currentmarker}{\pgfqpoint{0.000000in}{0.000000in}}{\pgfqpoint{0.000000in}{0.027778in}}{%
\pgfpathmoveto{\pgfqpoint{0.000000in}{0.000000in}}%
\pgfpathlineto{\pgfqpoint{0.000000in}{0.027778in}}%
\pgfusepath{stroke,fill}%
}%
\begin{pgfscope}%
\pgfsys@transformshift{5.299444in}{4.627778in}%
\pgfsys@useobject{currentmarker}{}%
\end{pgfscope}%
\end{pgfscope}%
\begin{pgfscope}%
\pgfpathrectangle{\pgfqpoint{0.781944in}{2.977778in}}{\pgfqpoint{5.019444in}{1.650000in}}%
\pgfusepath{clip}%
\pgfsetrectcap%
\pgfsetroundjoin%
\pgfsetlinewidth{0.803000pt}%
\definecolor{currentstroke}{rgb}{0.690196,0.690196,0.690196}%
\pgfsetstrokecolor{currentstroke}%
\pgfsetstrokeopacity{0.300000}%
\pgfsetdash{}{0pt}%
\pgfpathmoveto{\pgfqpoint{5.349639in}{2.977778in}}%
\pgfpathlineto{\pgfqpoint{5.349639in}{4.627778in}}%
\pgfusepath{stroke}%
\end{pgfscope}%
\begin{pgfscope}%
\pgfsetbuttcap%
\pgfsetroundjoin%
\definecolor{currentfill}{rgb}{0.000000,0.000000,0.000000}%
\pgfsetfillcolor{currentfill}%
\pgfsetlinewidth{0.602250pt}%
\definecolor{currentstroke}{rgb}{0.000000,0.000000,0.000000}%
\pgfsetstrokecolor{currentstroke}%
\pgfsetdash{}{0pt}%
\pgfsys@defobject{currentmarker}{\pgfqpoint{0.000000in}{-0.027778in}}{\pgfqpoint{0.000000in}{0.000000in}}{%
\pgfpathmoveto{\pgfqpoint{0.000000in}{0.000000in}}%
\pgfpathlineto{\pgfqpoint{0.000000in}{-0.027778in}}%
\pgfusepath{stroke,fill}%
}%
\begin{pgfscope}%
\pgfsys@transformshift{5.349639in}{2.977778in}%
\pgfsys@useobject{currentmarker}{}%
\end{pgfscope}%
\end{pgfscope}%
\begin{pgfscope}%
\pgfsetbuttcap%
\pgfsetroundjoin%
\definecolor{currentfill}{rgb}{0.000000,0.000000,0.000000}%
\pgfsetfillcolor{currentfill}%
\pgfsetlinewidth{0.602250pt}%
\definecolor{currentstroke}{rgb}{0.000000,0.000000,0.000000}%
\pgfsetstrokecolor{currentstroke}%
\pgfsetdash{}{0pt}%
\pgfsys@defobject{currentmarker}{\pgfqpoint{0.000000in}{0.000000in}}{\pgfqpoint{0.000000in}{0.027778in}}{%
\pgfpathmoveto{\pgfqpoint{0.000000in}{0.000000in}}%
\pgfpathlineto{\pgfqpoint{0.000000in}{0.027778in}}%
\pgfusepath{stroke,fill}%
}%
\begin{pgfscope}%
\pgfsys@transformshift{5.349639in}{4.627778in}%
\pgfsys@useobject{currentmarker}{}%
\end{pgfscope}%
\end{pgfscope}%
\begin{pgfscope}%
\pgfpathrectangle{\pgfqpoint{0.781944in}{2.977778in}}{\pgfqpoint{5.019444in}{1.650000in}}%
\pgfusepath{clip}%
\pgfsetrectcap%
\pgfsetroundjoin%
\pgfsetlinewidth{0.803000pt}%
\definecolor{currentstroke}{rgb}{0.690196,0.690196,0.690196}%
\pgfsetstrokecolor{currentstroke}%
\pgfsetstrokeopacity{0.300000}%
\pgfsetdash{}{0pt}%
\pgfpathmoveto{\pgfqpoint{5.399833in}{2.977778in}}%
\pgfpathlineto{\pgfqpoint{5.399833in}{4.627778in}}%
\pgfusepath{stroke}%
\end{pgfscope}%
\begin{pgfscope}%
\pgfsetbuttcap%
\pgfsetroundjoin%
\definecolor{currentfill}{rgb}{0.000000,0.000000,0.000000}%
\pgfsetfillcolor{currentfill}%
\pgfsetlinewidth{0.602250pt}%
\definecolor{currentstroke}{rgb}{0.000000,0.000000,0.000000}%
\pgfsetstrokecolor{currentstroke}%
\pgfsetdash{}{0pt}%
\pgfsys@defobject{currentmarker}{\pgfqpoint{0.000000in}{-0.027778in}}{\pgfqpoint{0.000000in}{0.000000in}}{%
\pgfpathmoveto{\pgfqpoint{0.000000in}{0.000000in}}%
\pgfpathlineto{\pgfqpoint{0.000000in}{-0.027778in}}%
\pgfusepath{stroke,fill}%
}%
\begin{pgfscope}%
\pgfsys@transformshift{5.399833in}{2.977778in}%
\pgfsys@useobject{currentmarker}{}%
\end{pgfscope}%
\end{pgfscope}%
\begin{pgfscope}%
\pgfsetbuttcap%
\pgfsetroundjoin%
\definecolor{currentfill}{rgb}{0.000000,0.000000,0.000000}%
\pgfsetfillcolor{currentfill}%
\pgfsetlinewidth{0.602250pt}%
\definecolor{currentstroke}{rgb}{0.000000,0.000000,0.000000}%
\pgfsetstrokecolor{currentstroke}%
\pgfsetdash{}{0pt}%
\pgfsys@defobject{currentmarker}{\pgfqpoint{0.000000in}{0.000000in}}{\pgfqpoint{0.000000in}{0.027778in}}{%
\pgfpathmoveto{\pgfqpoint{0.000000in}{0.000000in}}%
\pgfpathlineto{\pgfqpoint{0.000000in}{0.027778in}}%
\pgfusepath{stroke,fill}%
}%
\begin{pgfscope}%
\pgfsys@transformshift{5.399833in}{4.627778in}%
\pgfsys@useobject{currentmarker}{}%
\end{pgfscope}%
\end{pgfscope}%
\begin{pgfscope}%
\pgfpathrectangle{\pgfqpoint{0.781944in}{2.977778in}}{\pgfqpoint{5.019444in}{1.650000in}}%
\pgfusepath{clip}%
\pgfsetrectcap%
\pgfsetroundjoin%
\pgfsetlinewidth{0.803000pt}%
\definecolor{currentstroke}{rgb}{0.690196,0.690196,0.690196}%
\pgfsetstrokecolor{currentstroke}%
\pgfsetstrokeopacity{0.300000}%
\pgfsetdash{}{0pt}%
\pgfpathmoveto{\pgfqpoint{5.450028in}{2.977778in}}%
\pgfpathlineto{\pgfqpoint{5.450028in}{4.627778in}}%
\pgfusepath{stroke}%
\end{pgfscope}%
\begin{pgfscope}%
\pgfsetbuttcap%
\pgfsetroundjoin%
\definecolor{currentfill}{rgb}{0.000000,0.000000,0.000000}%
\pgfsetfillcolor{currentfill}%
\pgfsetlinewidth{0.602250pt}%
\definecolor{currentstroke}{rgb}{0.000000,0.000000,0.000000}%
\pgfsetstrokecolor{currentstroke}%
\pgfsetdash{}{0pt}%
\pgfsys@defobject{currentmarker}{\pgfqpoint{0.000000in}{-0.027778in}}{\pgfqpoint{0.000000in}{0.000000in}}{%
\pgfpathmoveto{\pgfqpoint{0.000000in}{0.000000in}}%
\pgfpathlineto{\pgfqpoint{0.000000in}{-0.027778in}}%
\pgfusepath{stroke,fill}%
}%
\begin{pgfscope}%
\pgfsys@transformshift{5.450028in}{2.977778in}%
\pgfsys@useobject{currentmarker}{}%
\end{pgfscope}%
\end{pgfscope}%
\begin{pgfscope}%
\pgfsetbuttcap%
\pgfsetroundjoin%
\definecolor{currentfill}{rgb}{0.000000,0.000000,0.000000}%
\pgfsetfillcolor{currentfill}%
\pgfsetlinewidth{0.602250pt}%
\definecolor{currentstroke}{rgb}{0.000000,0.000000,0.000000}%
\pgfsetstrokecolor{currentstroke}%
\pgfsetdash{}{0pt}%
\pgfsys@defobject{currentmarker}{\pgfqpoint{0.000000in}{0.000000in}}{\pgfqpoint{0.000000in}{0.027778in}}{%
\pgfpathmoveto{\pgfqpoint{0.000000in}{0.000000in}}%
\pgfpathlineto{\pgfqpoint{0.000000in}{0.027778in}}%
\pgfusepath{stroke,fill}%
}%
\begin{pgfscope}%
\pgfsys@transformshift{5.450028in}{4.627778in}%
\pgfsys@useobject{currentmarker}{}%
\end{pgfscope}%
\end{pgfscope}%
\begin{pgfscope}%
\pgfpathrectangle{\pgfqpoint{0.781944in}{2.977778in}}{\pgfqpoint{5.019444in}{1.650000in}}%
\pgfusepath{clip}%
\pgfsetrectcap%
\pgfsetroundjoin%
\pgfsetlinewidth{0.803000pt}%
\definecolor{currentstroke}{rgb}{0.690196,0.690196,0.690196}%
\pgfsetstrokecolor{currentstroke}%
\pgfsetstrokeopacity{0.300000}%
\pgfsetdash{}{0pt}%
\pgfpathmoveto{\pgfqpoint{5.500222in}{2.977778in}}%
\pgfpathlineto{\pgfqpoint{5.500222in}{4.627778in}}%
\pgfusepath{stroke}%
\end{pgfscope}%
\begin{pgfscope}%
\pgfsetbuttcap%
\pgfsetroundjoin%
\definecolor{currentfill}{rgb}{0.000000,0.000000,0.000000}%
\pgfsetfillcolor{currentfill}%
\pgfsetlinewidth{0.602250pt}%
\definecolor{currentstroke}{rgb}{0.000000,0.000000,0.000000}%
\pgfsetstrokecolor{currentstroke}%
\pgfsetdash{}{0pt}%
\pgfsys@defobject{currentmarker}{\pgfqpoint{0.000000in}{-0.027778in}}{\pgfqpoint{0.000000in}{0.000000in}}{%
\pgfpathmoveto{\pgfqpoint{0.000000in}{0.000000in}}%
\pgfpathlineto{\pgfqpoint{0.000000in}{-0.027778in}}%
\pgfusepath{stroke,fill}%
}%
\begin{pgfscope}%
\pgfsys@transformshift{5.500222in}{2.977778in}%
\pgfsys@useobject{currentmarker}{}%
\end{pgfscope}%
\end{pgfscope}%
\begin{pgfscope}%
\pgfsetbuttcap%
\pgfsetroundjoin%
\definecolor{currentfill}{rgb}{0.000000,0.000000,0.000000}%
\pgfsetfillcolor{currentfill}%
\pgfsetlinewidth{0.602250pt}%
\definecolor{currentstroke}{rgb}{0.000000,0.000000,0.000000}%
\pgfsetstrokecolor{currentstroke}%
\pgfsetdash{}{0pt}%
\pgfsys@defobject{currentmarker}{\pgfqpoint{0.000000in}{0.000000in}}{\pgfqpoint{0.000000in}{0.027778in}}{%
\pgfpathmoveto{\pgfqpoint{0.000000in}{0.000000in}}%
\pgfpathlineto{\pgfqpoint{0.000000in}{0.027778in}}%
\pgfusepath{stroke,fill}%
}%
\begin{pgfscope}%
\pgfsys@transformshift{5.500222in}{4.627778in}%
\pgfsys@useobject{currentmarker}{}%
\end{pgfscope}%
\end{pgfscope}%
\begin{pgfscope}%
\pgfpathrectangle{\pgfqpoint{0.781944in}{2.977778in}}{\pgfqpoint{5.019444in}{1.650000in}}%
\pgfusepath{clip}%
\pgfsetrectcap%
\pgfsetroundjoin%
\pgfsetlinewidth{0.803000pt}%
\definecolor{currentstroke}{rgb}{0.690196,0.690196,0.690196}%
\pgfsetstrokecolor{currentstroke}%
\pgfsetstrokeopacity{0.300000}%
\pgfsetdash{}{0pt}%
\pgfpathmoveto{\pgfqpoint{5.600611in}{2.977778in}}%
\pgfpathlineto{\pgfqpoint{5.600611in}{4.627778in}}%
\pgfusepath{stroke}%
\end{pgfscope}%
\begin{pgfscope}%
\pgfsetbuttcap%
\pgfsetroundjoin%
\definecolor{currentfill}{rgb}{0.000000,0.000000,0.000000}%
\pgfsetfillcolor{currentfill}%
\pgfsetlinewidth{0.602250pt}%
\definecolor{currentstroke}{rgb}{0.000000,0.000000,0.000000}%
\pgfsetstrokecolor{currentstroke}%
\pgfsetdash{}{0pt}%
\pgfsys@defobject{currentmarker}{\pgfqpoint{0.000000in}{-0.027778in}}{\pgfqpoint{0.000000in}{0.000000in}}{%
\pgfpathmoveto{\pgfqpoint{0.000000in}{0.000000in}}%
\pgfpathlineto{\pgfqpoint{0.000000in}{-0.027778in}}%
\pgfusepath{stroke,fill}%
}%
\begin{pgfscope}%
\pgfsys@transformshift{5.600611in}{2.977778in}%
\pgfsys@useobject{currentmarker}{}%
\end{pgfscope}%
\end{pgfscope}%
\begin{pgfscope}%
\pgfsetbuttcap%
\pgfsetroundjoin%
\definecolor{currentfill}{rgb}{0.000000,0.000000,0.000000}%
\pgfsetfillcolor{currentfill}%
\pgfsetlinewidth{0.602250pt}%
\definecolor{currentstroke}{rgb}{0.000000,0.000000,0.000000}%
\pgfsetstrokecolor{currentstroke}%
\pgfsetdash{}{0pt}%
\pgfsys@defobject{currentmarker}{\pgfqpoint{0.000000in}{0.000000in}}{\pgfqpoint{0.000000in}{0.027778in}}{%
\pgfpathmoveto{\pgfqpoint{0.000000in}{0.000000in}}%
\pgfpathlineto{\pgfqpoint{0.000000in}{0.027778in}}%
\pgfusepath{stroke,fill}%
}%
\begin{pgfscope}%
\pgfsys@transformshift{5.600611in}{4.627778in}%
\pgfsys@useobject{currentmarker}{}%
\end{pgfscope}%
\end{pgfscope}%
\begin{pgfscope}%
\pgfpathrectangle{\pgfqpoint{0.781944in}{2.977778in}}{\pgfqpoint{5.019444in}{1.650000in}}%
\pgfusepath{clip}%
\pgfsetrectcap%
\pgfsetroundjoin%
\pgfsetlinewidth{0.803000pt}%
\definecolor{currentstroke}{rgb}{0.690196,0.690196,0.690196}%
\pgfsetstrokecolor{currentstroke}%
\pgfsetstrokeopacity{0.300000}%
\pgfsetdash{}{0pt}%
\pgfpathmoveto{\pgfqpoint{5.650806in}{2.977778in}}%
\pgfpathlineto{\pgfqpoint{5.650806in}{4.627778in}}%
\pgfusepath{stroke}%
\end{pgfscope}%
\begin{pgfscope}%
\pgfsetbuttcap%
\pgfsetroundjoin%
\definecolor{currentfill}{rgb}{0.000000,0.000000,0.000000}%
\pgfsetfillcolor{currentfill}%
\pgfsetlinewidth{0.602250pt}%
\definecolor{currentstroke}{rgb}{0.000000,0.000000,0.000000}%
\pgfsetstrokecolor{currentstroke}%
\pgfsetdash{}{0pt}%
\pgfsys@defobject{currentmarker}{\pgfqpoint{0.000000in}{-0.027778in}}{\pgfqpoint{0.000000in}{0.000000in}}{%
\pgfpathmoveto{\pgfqpoint{0.000000in}{0.000000in}}%
\pgfpathlineto{\pgfqpoint{0.000000in}{-0.027778in}}%
\pgfusepath{stroke,fill}%
}%
\begin{pgfscope}%
\pgfsys@transformshift{5.650806in}{2.977778in}%
\pgfsys@useobject{currentmarker}{}%
\end{pgfscope}%
\end{pgfscope}%
\begin{pgfscope}%
\pgfsetbuttcap%
\pgfsetroundjoin%
\definecolor{currentfill}{rgb}{0.000000,0.000000,0.000000}%
\pgfsetfillcolor{currentfill}%
\pgfsetlinewidth{0.602250pt}%
\definecolor{currentstroke}{rgb}{0.000000,0.000000,0.000000}%
\pgfsetstrokecolor{currentstroke}%
\pgfsetdash{}{0pt}%
\pgfsys@defobject{currentmarker}{\pgfqpoint{0.000000in}{0.000000in}}{\pgfqpoint{0.000000in}{0.027778in}}{%
\pgfpathmoveto{\pgfqpoint{0.000000in}{0.000000in}}%
\pgfpathlineto{\pgfqpoint{0.000000in}{0.027778in}}%
\pgfusepath{stroke,fill}%
}%
\begin{pgfscope}%
\pgfsys@transformshift{5.650806in}{4.627778in}%
\pgfsys@useobject{currentmarker}{}%
\end{pgfscope}%
\end{pgfscope}%
\begin{pgfscope}%
\pgfpathrectangle{\pgfqpoint{0.781944in}{2.977778in}}{\pgfqpoint{5.019444in}{1.650000in}}%
\pgfusepath{clip}%
\pgfsetrectcap%
\pgfsetroundjoin%
\pgfsetlinewidth{0.803000pt}%
\definecolor{currentstroke}{rgb}{0.690196,0.690196,0.690196}%
\pgfsetstrokecolor{currentstroke}%
\pgfsetstrokeopacity{0.300000}%
\pgfsetdash{}{0pt}%
\pgfpathmoveto{\pgfqpoint{5.701000in}{2.977778in}}%
\pgfpathlineto{\pgfqpoint{5.701000in}{4.627778in}}%
\pgfusepath{stroke}%
\end{pgfscope}%
\begin{pgfscope}%
\pgfsetbuttcap%
\pgfsetroundjoin%
\definecolor{currentfill}{rgb}{0.000000,0.000000,0.000000}%
\pgfsetfillcolor{currentfill}%
\pgfsetlinewidth{0.602250pt}%
\definecolor{currentstroke}{rgb}{0.000000,0.000000,0.000000}%
\pgfsetstrokecolor{currentstroke}%
\pgfsetdash{}{0pt}%
\pgfsys@defobject{currentmarker}{\pgfqpoint{0.000000in}{-0.027778in}}{\pgfqpoint{0.000000in}{0.000000in}}{%
\pgfpathmoveto{\pgfqpoint{0.000000in}{0.000000in}}%
\pgfpathlineto{\pgfqpoint{0.000000in}{-0.027778in}}%
\pgfusepath{stroke,fill}%
}%
\begin{pgfscope}%
\pgfsys@transformshift{5.701000in}{2.977778in}%
\pgfsys@useobject{currentmarker}{}%
\end{pgfscope}%
\end{pgfscope}%
\begin{pgfscope}%
\pgfsetbuttcap%
\pgfsetroundjoin%
\definecolor{currentfill}{rgb}{0.000000,0.000000,0.000000}%
\pgfsetfillcolor{currentfill}%
\pgfsetlinewidth{0.602250pt}%
\definecolor{currentstroke}{rgb}{0.000000,0.000000,0.000000}%
\pgfsetstrokecolor{currentstroke}%
\pgfsetdash{}{0pt}%
\pgfsys@defobject{currentmarker}{\pgfqpoint{0.000000in}{0.000000in}}{\pgfqpoint{0.000000in}{0.027778in}}{%
\pgfpathmoveto{\pgfqpoint{0.000000in}{0.000000in}}%
\pgfpathlineto{\pgfqpoint{0.000000in}{0.027778in}}%
\pgfusepath{stroke,fill}%
}%
\begin{pgfscope}%
\pgfsys@transformshift{5.701000in}{4.627778in}%
\pgfsys@useobject{currentmarker}{}%
\end{pgfscope}%
\end{pgfscope}%
\begin{pgfscope}%
\pgfpathrectangle{\pgfqpoint{0.781944in}{2.977778in}}{\pgfqpoint{5.019444in}{1.650000in}}%
\pgfusepath{clip}%
\pgfsetrectcap%
\pgfsetroundjoin%
\pgfsetlinewidth{0.803000pt}%
\definecolor{currentstroke}{rgb}{0.690196,0.690196,0.690196}%
\pgfsetstrokecolor{currentstroke}%
\pgfsetstrokeopacity{0.300000}%
\pgfsetdash{}{0pt}%
\pgfpathmoveto{\pgfqpoint{5.751194in}{2.977778in}}%
\pgfpathlineto{\pgfqpoint{5.751194in}{4.627778in}}%
\pgfusepath{stroke}%
\end{pgfscope}%
\begin{pgfscope}%
\pgfsetbuttcap%
\pgfsetroundjoin%
\definecolor{currentfill}{rgb}{0.000000,0.000000,0.000000}%
\pgfsetfillcolor{currentfill}%
\pgfsetlinewidth{0.602250pt}%
\definecolor{currentstroke}{rgb}{0.000000,0.000000,0.000000}%
\pgfsetstrokecolor{currentstroke}%
\pgfsetdash{}{0pt}%
\pgfsys@defobject{currentmarker}{\pgfqpoint{0.000000in}{-0.027778in}}{\pgfqpoint{0.000000in}{0.000000in}}{%
\pgfpathmoveto{\pgfqpoint{0.000000in}{0.000000in}}%
\pgfpathlineto{\pgfqpoint{0.000000in}{-0.027778in}}%
\pgfusepath{stroke,fill}%
}%
\begin{pgfscope}%
\pgfsys@transformshift{5.751194in}{2.977778in}%
\pgfsys@useobject{currentmarker}{}%
\end{pgfscope}%
\end{pgfscope}%
\begin{pgfscope}%
\pgfsetbuttcap%
\pgfsetroundjoin%
\definecolor{currentfill}{rgb}{0.000000,0.000000,0.000000}%
\pgfsetfillcolor{currentfill}%
\pgfsetlinewidth{0.602250pt}%
\definecolor{currentstroke}{rgb}{0.000000,0.000000,0.000000}%
\pgfsetstrokecolor{currentstroke}%
\pgfsetdash{}{0pt}%
\pgfsys@defobject{currentmarker}{\pgfqpoint{0.000000in}{0.000000in}}{\pgfqpoint{0.000000in}{0.027778in}}{%
\pgfpathmoveto{\pgfqpoint{0.000000in}{0.000000in}}%
\pgfpathlineto{\pgfqpoint{0.000000in}{0.027778in}}%
\pgfusepath{stroke,fill}%
}%
\begin{pgfscope}%
\pgfsys@transformshift{5.751194in}{4.627778in}%
\pgfsys@useobject{currentmarker}{}%
\end{pgfscope}%
\end{pgfscope}%
\begin{pgfscope}%
\pgfpathrectangle{\pgfqpoint{0.781944in}{2.977778in}}{\pgfqpoint{5.019444in}{1.650000in}}%
\pgfusepath{clip}%
\pgfsetrectcap%
\pgfsetroundjoin%
\pgfsetlinewidth{0.803000pt}%
\definecolor{currentstroke}{rgb}{0.690196,0.690196,0.690196}%
\pgfsetstrokecolor{currentstroke}%
\pgfsetstrokeopacity{0.300000}%
\pgfsetdash{}{0pt}%
\pgfpathmoveto{\pgfqpoint{5.801389in}{2.977778in}}%
\pgfpathlineto{\pgfqpoint{5.801389in}{4.627778in}}%
\pgfusepath{stroke}%
\end{pgfscope}%
\begin{pgfscope}%
\pgfsetbuttcap%
\pgfsetroundjoin%
\definecolor{currentfill}{rgb}{0.000000,0.000000,0.000000}%
\pgfsetfillcolor{currentfill}%
\pgfsetlinewidth{0.602250pt}%
\definecolor{currentstroke}{rgb}{0.000000,0.000000,0.000000}%
\pgfsetstrokecolor{currentstroke}%
\pgfsetdash{}{0pt}%
\pgfsys@defobject{currentmarker}{\pgfqpoint{0.000000in}{-0.027778in}}{\pgfqpoint{0.000000in}{0.000000in}}{%
\pgfpathmoveto{\pgfqpoint{0.000000in}{0.000000in}}%
\pgfpathlineto{\pgfqpoint{0.000000in}{-0.027778in}}%
\pgfusepath{stroke,fill}%
}%
\begin{pgfscope}%
\pgfsys@transformshift{5.801389in}{2.977778in}%
\pgfsys@useobject{currentmarker}{}%
\end{pgfscope}%
\end{pgfscope}%
\begin{pgfscope}%
\pgfsetbuttcap%
\pgfsetroundjoin%
\definecolor{currentfill}{rgb}{0.000000,0.000000,0.000000}%
\pgfsetfillcolor{currentfill}%
\pgfsetlinewidth{0.602250pt}%
\definecolor{currentstroke}{rgb}{0.000000,0.000000,0.000000}%
\pgfsetstrokecolor{currentstroke}%
\pgfsetdash{}{0pt}%
\pgfsys@defobject{currentmarker}{\pgfqpoint{0.000000in}{0.000000in}}{\pgfqpoint{0.000000in}{0.027778in}}{%
\pgfpathmoveto{\pgfqpoint{0.000000in}{0.000000in}}%
\pgfpathlineto{\pgfqpoint{0.000000in}{0.027778in}}%
\pgfusepath{stroke,fill}%
}%
\begin{pgfscope}%
\pgfsys@transformshift{5.801389in}{4.627778in}%
\pgfsys@useobject{currentmarker}{}%
\end{pgfscope}%
\end{pgfscope}%
\begin{pgfscope}%
\definecolor{textcolor}{rgb}{0.000000,0.000000,0.000000}%
\pgfsetstrokecolor{textcolor}%
\pgfsetfillcolor{textcolor}%
\pgftext[x=3.291667in,y=2.701667in,,top]{\color{textcolor}\rmfamily\fontsize{10.000000}{12.000000}\selectfont t [ns]}%
\end{pgfscope}%
\begin{pgfscope}%
\pgfpathrectangle{\pgfqpoint{0.781944in}{2.977778in}}{\pgfqpoint{5.019444in}{1.650000in}}%
\pgfusepath{clip}%
\pgfsetrectcap%
\pgfsetroundjoin%
\pgfsetlinewidth{0.803000pt}%
\definecolor{currentstroke}{rgb}{0.690196,0.690196,0.690196}%
\pgfsetstrokecolor{currentstroke}%
\pgfsetstrokeopacity{0.800000}%
\pgfsetdash{}{0pt}%
\pgfpathmoveto{\pgfqpoint{0.781944in}{2.977778in}}%
\pgfpathlineto{\pgfqpoint{5.801389in}{2.977778in}}%
\pgfusepath{stroke}%
\end{pgfscope}%
\begin{pgfscope}%
\pgfsetbuttcap%
\pgfsetroundjoin%
\definecolor{currentfill}{rgb}{0.000000,0.000000,0.000000}%
\pgfsetfillcolor{currentfill}%
\pgfsetlinewidth{0.803000pt}%
\definecolor{currentstroke}{rgb}{0.000000,0.000000,0.000000}%
\pgfsetstrokecolor{currentstroke}%
\pgfsetdash{}{0pt}%
\pgfsys@defobject{currentmarker}{\pgfqpoint{-0.048611in}{0.000000in}}{\pgfqpoint{0.000000in}{0.000000in}}{%
\pgfpathmoveto{\pgfqpoint{0.000000in}{0.000000in}}%
\pgfpathlineto{\pgfqpoint{-0.048611in}{0.000000in}}%
\pgfusepath{stroke,fill}%
}%
\begin{pgfscope}%
\pgfsys@transformshift{0.781944in}{2.977778in}%
\pgfsys@useobject{currentmarker}{}%
\end{pgfscope}%
\end{pgfscope}%
\begin{pgfscope}%
\pgfsetbuttcap%
\pgfsetroundjoin%
\definecolor{currentfill}{rgb}{0.000000,0.000000,0.000000}%
\pgfsetfillcolor{currentfill}%
\pgfsetlinewidth{0.803000pt}%
\definecolor{currentstroke}{rgb}{0.000000,0.000000,0.000000}%
\pgfsetstrokecolor{currentstroke}%
\pgfsetdash{}{0pt}%
\pgfsys@defobject{currentmarker}{\pgfqpoint{0.000000in}{0.000000in}}{\pgfqpoint{0.048611in}{0.000000in}}{%
\pgfpathmoveto{\pgfqpoint{0.000000in}{0.000000in}}%
\pgfpathlineto{\pgfqpoint{0.048611in}{0.000000in}}%
\pgfusepath{stroke,fill}%
}%
\begin{pgfscope}%
\pgfsys@transformshift{5.801389in}{2.977778in}%
\pgfsys@useobject{currentmarker}{}%
\end{pgfscope}%
\end{pgfscope}%
\begin{pgfscope}%
\definecolor{textcolor}{rgb}{0.000000,0.000000,0.000000}%
\pgfsetstrokecolor{textcolor}%
\pgfsetfillcolor{textcolor}%
\pgftext[x=0.615278in,y=2.929583in,left,base]{\color{textcolor}\rmfamily\fontsize{10.000000}{12.000000}\selectfont 0}%
\end{pgfscope}%
\begin{pgfscope}%
\pgfpathrectangle{\pgfqpoint{0.781944in}{2.977778in}}{\pgfqpoint{5.019444in}{1.650000in}}%
\pgfusepath{clip}%
\pgfsetrectcap%
\pgfsetroundjoin%
\pgfsetlinewidth{0.803000pt}%
\definecolor{currentstroke}{rgb}{0.690196,0.690196,0.690196}%
\pgfsetstrokecolor{currentstroke}%
\pgfsetstrokeopacity{0.800000}%
\pgfsetdash{}{0pt}%
\pgfpathmoveto{\pgfqpoint{0.781944in}{3.196792in}}%
\pgfpathlineto{\pgfqpoint{5.801389in}{3.196792in}}%
\pgfusepath{stroke}%
\end{pgfscope}%
\begin{pgfscope}%
\pgfsetbuttcap%
\pgfsetroundjoin%
\definecolor{currentfill}{rgb}{0.000000,0.000000,0.000000}%
\pgfsetfillcolor{currentfill}%
\pgfsetlinewidth{0.803000pt}%
\definecolor{currentstroke}{rgb}{0.000000,0.000000,0.000000}%
\pgfsetstrokecolor{currentstroke}%
\pgfsetdash{}{0pt}%
\pgfsys@defobject{currentmarker}{\pgfqpoint{-0.048611in}{0.000000in}}{\pgfqpoint{0.000000in}{0.000000in}}{%
\pgfpathmoveto{\pgfqpoint{0.000000in}{0.000000in}}%
\pgfpathlineto{\pgfqpoint{-0.048611in}{0.000000in}}%
\pgfusepath{stroke,fill}%
}%
\begin{pgfscope}%
\pgfsys@transformshift{0.781944in}{3.196792in}%
\pgfsys@useobject{currentmarker}{}%
\end{pgfscope}%
\end{pgfscope}%
\begin{pgfscope}%
\pgfsetbuttcap%
\pgfsetroundjoin%
\definecolor{currentfill}{rgb}{0.000000,0.000000,0.000000}%
\pgfsetfillcolor{currentfill}%
\pgfsetlinewidth{0.803000pt}%
\definecolor{currentstroke}{rgb}{0.000000,0.000000,0.000000}%
\pgfsetstrokecolor{currentstroke}%
\pgfsetdash{}{0pt}%
\pgfsys@defobject{currentmarker}{\pgfqpoint{0.000000in}{0.000000in}}{\pgfqpoint{0.048611in}{0.000000in}}{%
\pgfpathmoveto{\pgfqpoint{0.000000in}{0.000000in}}%
\pgfpathlineto{\pgfqpoint{0.048611in}{0.000000in}}%
\pgfusepath{stroke,fill}%
}%
\begin{pgfscope}%
\pgfsys@transformshift{5.801389in}{3.196792in}%
\pgfsys@useobject{currentmarker}{}%
\end{pgfscope}%
\end{pgfscope}%
\begin{pgfscope}%
\definecolor{textcolor}{rgb}{0.000000,0.000000,0.000000}%
\pgfsetstrokecolor{textcolor}%
\pgfsetfillcolor{textcolor}%
\pgftext[x=0.476389in,y=3.148598in,left,base]{\color{textcolor}\rmfamily\fontsize{10.000000}{12.000000}\selectfont 400}%
\end{pgfscope}%
\begin{pgfscope}%
\pgfpathrectangle{\pgfqpoint{0.781944in}{2.977778in}}{\pgfqpoint{5.019444in}{1.650000in}}%
\pgfusepath{clip}%
\pgfsetrectcap%
\pgfsetroundjoin%
\pgfsetlinewidth{0.803000pt}%
\definecolor{currentstroke}{rgb}{0.690196,0.690196,0.690196}%
\pgfsetstrokecolor{currentstroke}%
\pgfsetstrokeopacity{0.800000}%
\pgfsetdash{}{0pt}%
\pgfpathmoveto{\pgfqpoint{0.781944in}{3.415807in}}%
\pgfpathlineto{\pgfqpoint{5.801389in}{3.415807in}}%
\pgfusepath{stroke}%
\end{pgfscope}%
\begin{pgfscope}%
\pgfsetbuttcap%
\pgfsetroundjoin%
\definecolor{currentfill}{rgb}{0.000000,0.000000,0.000000}%
\pgfsetfillcolor{currentfill}%
\pgfsetlinewidth{0.803000pt}%
\definecolor{currentstroke}{rgb}{0.000000,0.000000,0.000000}%
\pgfsetstrokecolor{currentstroke}%
\pgfsetdash{}{0pt}%
\pgfsys@defobject{currentmarker}{\pgfqpoint{-0.048611in}{0.000000in}}{\pgfqpoint{0.000000in}{0.000000in}}{%
\pgfpathmoveto{\pgfqpoint{0.000000in}{0.000000in}}%
\pgfpathlineto{\pgfqpoint{-0.048611in}{0.000000in}}%
\pgfusepath{stroke,fill}%
}%
\begin{pgfscope}%
\pgfsys@transformshift{0.781944in}{3.415807in}%
\pgfsys@useobject{currentmarker}{}%
\end{pgfscope}%
\end{pgfscope}%
\begin{pgfscope}%
\pgfsetbuttcap%
\pgfsetroundjoin%
\definecolor{currentfill}{rgb}{0.000000,0.000000,0.000000}%
\pgfsetfillcolor{currentfill}%
\pgfsetlinewidth{0.803000pt}%
\definecolor{currentstroke}{rgb}{0.000000,0.000000,0.000000}%
\pgfsetstrokecolor{currentstroke}%
\pgfsetdash{}{0pt}%
\pgfsys@defobject{currentmarker}{\pgfqpoint{0.000000in}{0.000000in}}{\pgfqpoint{0.048611in}{0.000000in}}{%
\pgfpathmoveto{\pgfqpoint{0.000000in}{0.000000in}}%
\pgfpathlineto{\pgfqpoint{0.048611in}{0.000000in}}%
\pgfusepath{stroke,fill}%
}%
\begin{pgfscope}%
\pgfsys@transformshift{5.801389in}{3.415807in}%
\pgfsys@useobject{currentmarker}{}%
\end{pgfscope}%
\end{pgfscope}%
\begin{pgfscope}%
\definecolor{textcolor}{rgb}{0.000000,0.000000,0.000000}%
\pgfsetstrokecolor{textcolor}%
\pgfsetfillcolor{textcolor}%
\pgftext[x=0.476389in,y=3.367612in,left,base]{\color{textcolor}\rmfamily\fontsize{10.000000}{12.000000}\selectfont 800}%
\end{pgfscope}%
\begin{pgfscope}%
\pgfpathrectangle{\pgfqpoint{0.781944in}{2.977778in}}{\pgfqpoint{5.019444in}{1.650000in}}%
\pgfusepath{clip}%
\pgfsetrectcap%
\pgfsetroundjoin%
\pgfsetlinewidth{0.803000pt}%
\definecolor{currentstroke}{rgb}{0.690196,0.690196,0.690196}%
\pgfsetstrokecolor{currentstroke}%
\pgfsetstrokeopacity{0.800000}%
\pgfsetdash{}{0pt}%
\pgfpathmoveto{\pgfqpoint{0.781944in}{3.634821in}}%
\pgfpathlineto{\pgfqpoint{5.801389in}{3.634821in}}%
\pgfusepath{stroke}%
\end{pgfscope}%
\begin{pgfscope}%
\pgfsetbuttcap%
\pgfsetroundjoin%
\definecolor{currentfill}{rgb}{0.000000,0.000000,0.000000}%
\pgfsetfillcolor{currentfill}%
\pgfsetlinewidth{0.803000pt}%
\definecolor{currentstroke}{rgb}{0.000000,0.000000,0.000000}%
\pgfsetstrokecolor{currentstroke}%
\pgfsetdash{}{0pt}%
\pgfsys@defobject{currentmarker}{\pgfqpoint{-0.048611in}{0.000000in}}{\pgfqpoint{0.000000in}{0.000000in}}{%
\pgfpathmoveto{\pgfqpoint{0.000000in}{0.000000in}}%
\pgfpathlineto{\pgfqpoint{-0.048611in}{0.000000in}}%
\pgfusepath{stroke,fill}%
}%
\begin{pgfscope}%
\pgfsys@transformshift{0.781944in}{3.634821in}%
\pgfsys@useobject{currentmarker}{}%
\end{pgfscope}%
\end{pgfscope}%
\begin{pgfscope}%
\pgfsetbuttcap%
\pgfsetroundjoin%
\definecolor{currentfill}{rgb}{0.000000,0.000000,0.000000}%
\pgfsetfillcolor{currentfill}%
\pgfsetlinewidth{0.803000pt}%
\definecolor{currentstroke}{rgb}{0.000000,0.000000,0.000000}%
\pgfsetstrokecolor{currentstroke}%
\pgfsetdash{}{0pt}%
\pgfsys@defobject{currentmarker}{\pgfqpoint{0.000000in}{0.000000in}}{\pgfqpoint{0.048611in}{0.000000in}}{%
\pgfpathmoveto{\pgfqpoint{0.000000in}{0.000000in}}%
\pgfpathlineto{\pgfqpoint{0.048611in}{0.000000in}}%
\pgfusepath{stroke,fill}%
}%
\begin{pgfscope}%
\pgfsys@transformshift{5.801389in}{3.634821in}%
\pgfsys@useobject{currentmarker}{}%
\end{pgfscope}%
\end{pgfscope}%
\begin{pgfscope}%
\definecolor{textcolor}{rgb}{0.000000,0.000000,0.000000}%
\pgfsetstrokecolor{textcolor}%
\pgfsetfillcolor{textcolor}%
\pgftext[x=0.406944in,y=3.586627in,left,base]{\color{textcolor}\rmfamily\fontsize{10.000000}{12.000000}\selectfont 1200}%
\end{pgfscope}%
\begin{pgfscope}%
\pgfpathrectangle{\pgfqpoint{0.781944in}{2.977778in}}{\pgfqpoint{5.019444in}{1.650000in}}%
\pgfusepath{clip}%
\pgfsetrectcap%
\pgfsetroundjoin%
\pgfsetlinewidth{0.803000pt}%
\definecolor{currentstroke}{rgb}{0.690196,0.690196,0.690196}%
\pgfsetstrokecolor{currentstroke}%
\pgfsetstrokeopacity{0.800000}%
\pgfsetdash{}{0pt}%
\pgfpathmoveto{\pgfqpoint{0.781944in}{3.853836in}}%
\pgfpathlineto{\pgfqpoint{5.801389in}{3.853836in}}%
\pgfusepath{stroke}%
\end{pgfscope}%
\begin{pgfscope}%
\pgfsetbuttcap%
\pgfsetroundjoin%
\definecolor{currentfill}{rgb}{0.000000,0.000000,0.000000}%
\pgfsetfillcolor{currentfill}%
\pgfsetlinewidth{0.803000pt}%
\definecolor{currentstroke}{rgb}{0.000000,0.000000,0.000000}%
\pgfsetstrokecolor{currentstroke}%
\pgfsetdash{}{0pt}%
\pgfsys@defobject{currentmarker}{\pgfqpoint{-0.048611in}{0.000000in}}{\pgfqpoint{0.000000in}{0.000000in}}{%
\pgfpathmoveto{\pgfqpoint{0.000000in}{0.000000in}}%
\pgfpathlineto{\pgfqpoint{-0.048611in}{0.000000in}}%
\pgfusepath{stroke,fill}%
}%
\begin{pgfscope}%
\pgfsys@transformshift{0.781944in}{3.853836in}%
\pgfsys@useobject{currentmarker}{}%
\end{pgfscope}%
\end{pgfscope}%
\begin{pgfscope}%
\pgfsetbuttcap%
\pgfsetroundjoin%
\definecolor{currentfill}{rgb}{0.000000,0.000000,0.000000}%
\pgfsetfillcolor{currentfill}%
\pgfsetlinewidth{0.803000pt}%
\definecolor{currentstroke}{rgb}{0.000000,0.000000,0.000000}%
\pgfsetstrokecolor{currentstroke}%
\pgfsetdash{}{0pt}%
\pgfsys@defobject{currentmarker}{\pgfqpoint{0.000000in}{0.000000in}}{\pgfqpoint{0.048611in}{0.000000in}}{%
\pgfpathmoveto{\pgfqpoint{0.000000in}{0.000000in}}%
\pgfpathlineto{\pgfqpoint{0.048611in}{0.000000in}}%
\pgfusepath{stroke,fill}%
}%
\begin{pgfscope}%
\pgfsys@transformshift{5.801389in}{3.853836in}%
\pgfsys@useobject{currentmarker}{}%
\end{pgfscope}%
\end{pgfscope}%
\begin{pgfscope}%
\definecolor{textcolor}{rgb}{0.000000,0.000000,0.000000}%
\pgfsetstrokecolor{textcolor}%
\pgfsetfillcolor{textcolor}%
\pgftext[x=0.406944in,y=3.805641in,left,base]{\color{textcolor}\rmfamily\fontsize{10.000000}{12.000000}\selectfont 1600}%
\end{pgfscope}%
\begin{pgfscope}%
\pgfpathrectangle{\pgfqpoint{0.781944in}{2.977778in}}{\pgfqpoint{5.019444in}{1.650000in}}%
\pgfusepath{clip}%
\pgfsetrectcap%
\pgfsetroundjoin%
\pgfsetlinewidth{0.803000pt}%
\definecolor{currentstroke}{rgb}{0.690196,0.690196,0.690196}%
\pgfsetstrokecolor{currentstroke}%
\pgfsetstrokeopacity{0.800000}%
\pgfsetdash{}{0pt}%
\pgfpathmoveto{\pgfqpoint{0.781944in}{4.072850in}}%
\pgfpathlineto{\pgfqpoint{5.801389in}{4.072850in}}%
\pgfusepath{stroke}%
\end{pgfscope}%
\begin{pgfscope}%
\pgfsetbuttcap%
\pgfsetroundjoin%
\definecolor{currentfill}{rgb}{0.000000,0.000000,0.000000}%
\pgfsetfillcolor{currentfill}%
\pgfsetlinewidth{0.803000pt}%
\definecolor{currentstroke}{rgb}{0.000000,0.000000,0.000000}%
\pgfsetstrokecolor{currentstroke}%
\pgfsetdash{}{0pt}%
\pgfsys@defobject{currentmarker}{\pgfqpoint{-0.048611in}{0.000000in}}{\pgfqpoint{0.000000in}{0.000000in}}{%
\pgfpathmoveto{\pgfqpoint{0.000000in}{0.000000in}}%
\pgfpathlineto{\pgfqpoint{-0.048611in}{0.000000in}}%
\pgfusepath{stroke,fill}%
}%
\begin{pgfscope}%
\pgfsys@transformshift{0.781944in}{4.072850in}%
\pgfsys@useobject{currentmarker}{}%
\end{pgfscope}%
\end{pgfscope}%
\begin{pgfscope}%
\pgfsetbuttcap%
\pgfsetroundjoin%
\definecolor{currentfill}{rgb}{0.000000,0.000000,0.000000}%
\pgfsetfillcolor{currentfill}%
\pgfsetlinewidth{0.803000pt}%
\definecolor{currentstroke}{rgb}{0.000000,0.000000,0.000000}%
\pgfsetstrokecolor{currentstroke}%
\pgfsetdash{}{0pt}%
\pgfsys@defobject{currentmarker}{\pgfqpoint{0.000000in}{0.000000in}}{\pgfqpoint{0.048611in}{0.000000in}}{%
\pgfpathmoveto{\pgfqpoint{0.000000in}{0.000000in}}%
\pgfpathlineto{\pgfqpoint{0.048611in}{0.000000in}}%
\pgfusepath{stroke,fill}%
}%
\begin{pgfscope}%
\pgfsys@transformshift{5.801389in}{4.072850in}%
\pgfsys@useobject{currentmarker}{}%
\end{pgfscope}%
\end{pgfscope}%
\begin{pgfscope}%
\definecolor{textcolor}{rgb}{0.000000,0.000000,0.000000}%
\pgfsetstrokecolor{textcolor}%
\pgfsetfillcolor{textcolor}%
\pgftext[x=0.406944in,y=4.024656in,left,base]{\color{textcolor}\rmfamily\fontsize{10.000000}{12.000000}\selectfont 2000}%
\end{pgfscope}%
\begin{pgfscope}%
\pgfpathrectangle{\pgfqpoint{0.781944in}{2.977778in}}{\pgfqpoint{5.019444in}{1.650000in}}%
\pgfusepath{clip}%
\pgfsetrectcap%
\pgfsetroundjoin%
\pgfsetlinewidth{0.803000pt}%
\definecolor{currentstroke}{rgb}{0.690196,0.690196,0.690196}%
\pgfsetstrokecolor{currentstroke}%
\pgfsetstrokeopacity{0.800000}%
\pgfsetdash{}{0pt}%
\pgfpathmoveto{\pgfqpoint{0.781944in}{4.291864in}}%
\pgfpathlineto{\pgfqpoint{5.801389in}{4.291864in}}%
\pgfusepath{stroke}%
\end{pgfscope}%
\begin{pgfscope}%
\pgfsetbuttcap%
\pgfsetroundjoin%
\definecolor{currentfill}{rgb}{0.000000,0.000000,0.000000}%
\pgfsetfillcolor{currentfill}%
\pgfsetlinewidth{0.803000pt}%
\definecolor{currentstroke}{rgb}{0.000000,0.000000,0.000000}%
\pgfsetstrokecolor{currentstroke}%
\pgfsetdash{}{0pt}%
\pgfsys@defobject{currentmarker}{\pgfqpoint{-0.048611in}{0.000000in}}{\pgfqpoint{0.000000in}{0.000000in}}{%
\pgfpathmoveto{\pgfqpoint{0.000000in}{0.000000in}}%
\pgfpathlineto{\pgfqpoint{-0.048611in}{0.000000in}}%
\pgfusepath{stroke,fill}%
}%
\begin{pgfscope}%
\pgfsys@transformshift{0.781944in}{4.291864in}%
\pgfsys@useobject{currentmarker}{}%
\end{pgfscope}%
\end{pgfscope}%
\begin{pgfscope}%
\pgfsetbuttcap%
\pgfsetroundjoin%
\definecolor{currentfill}{rgb}{0.000000,0.000000,0.000000}%
\pgfsetfillcolor{currentfill}%
\pgfsetlinewidth{0.803000pt}%
\definecolor{currentstroke}{rgb}{0.000000,0.000000,0.000000}%
\pgfsetstrokecolor{currentstroke}%
\pgfsetdash{}{0pt}%
\pgfsys@defobject{currentmarker}{\pgfqpoint{0.000000in}{0.000000in}}{\pgfqpoint{0.048611in}{0.000000in}}{%
\pgfpathmoveto{\pgfqpoint{0.000000in}{0.000000in}}%
\pgfpathlineto{\pgfqpoint{0.048611in}{0.000000in}}%
\pgfusepath{stroke,fill}%
}%
\begin{pgfscope}%
\pgfsys@transformshift{5.801389in}{4.291864in}%
\pgfsys@useobject{currentmarker}{}%
\end{pgfscope}%
\end{pgfscope}%
\begin{pgfscope}%
\definecolor{textcolor}{rgb}{0.000000,0.000000,0.000000}%
\pgfsetstrokecolor{textcolor}%
\pgfsetfillcolor{textcolor}%
\pgftext[x=0.406944in,y=4.243670in,left,base]{\color{textcolor}\rmfamily\fontsize{10.000000}{12.000000}\selectfont 2400}%
\end{pgfscope}%
\begin{pgfscope}%
\pgfpathrectangle{\pgfqpoint{0.781944in}{2.977778in}}{\pgfqpoint{5.019444in}{1.650000in}}%
\pgfusepath{clip}%
\pgfsetrectcap%
\pgfsetroundjoin%
\pgfsetlinewidth{0.803000pt}%
\definecolor{currentstroke}{rgb}{0.690196,0.690196,0.690196}%
\pgfsetstrokecolor{currentstroke}%
\pgfsetstrokeopacity{0.800000}%
\pgfsetdash{}{0pt}%
\pgfpathmoveto{\pgfqpoint{0.781944in}{4.510879in}}%
\pgfpathlineto{\pgfqpoint{5.801389in}{4.510879in}}%
\pgfusepath{stroke}%
\end{pgfscope}%
\begin{pgfscope}%
\pgfsetbuttcap%
\pgfsetroundjoin%
\definecolor{currentfill}{rgb}{0.000000,0.000000,0.000000}%
\pgfsetfillcolor{currentfill}%
\pgfsetlinewidth{0.803000pt}%
\definecolor{currentstroke}{rgb}{0.000000,0.000000,0.000000}%
\pgfsetstrokecolor{currentstroke}%
\pgfsetdash{}{0pt}%
\pgfsys@defobject{currentmarker}{\pgfqpoint{-0.048611in}{0.000000in}}{\pgfqpoint{0.000000in}{0.000000in}}{%
\pgfpathmoveto{\pgfqpoint{0.000000in}{0.000000in}}%
\pgfpathlineto{\pgfqpoint{-0.048611in}{0.000000in}}%
\pgfusepath{stroke,fill}%
}%
\begin{pgfscope}%
\pgfsys@transformshift{0.781944in}{4.510879in}%
\pgfsys@useobject{currentmarker}{}%
\end{pgfscope}%
\end{pgfscope}%
\begin{pgfscope}%
\pgfsetbuttcap%
\pgfsetroundjoin%
\definecolor{currentfill}{rgb}{0.000000,0.000000,0.000000}%
\pgfsetfillcolor{currentfill}%
\pgfsetlinewidth{0.803000pt}%
\definecolor{currentstroke}{rgb}{0.000000,0.000000,0.000000}%
\pgfsetstrokecolor{currentstroke}%
\pgfsetdash{}{0pt}%
\pgfsys@defobject{currentmarker}{\pgfqpoint{0.000000in}{0.000000in}}{\pgfqpoint{0.048611in}{0.000000in}}{%
\pgfpathmoveto{\pgfqpoint{0.000000in}{0.000000in}}%
\pgfpathlineto{\pgfqpoint{0.048611in}{0.000000in}}%
\pgfusepath{stroke,fill}%
}%
\begin{pgfscope}%
\pgfsys@transformshift{5.801389in}{4.510879in}%
\pgfsys@useobject{currentmarker}{}%
\end{pgfscope}%
\end{pgfscope}%
\begin{pgfscope}%
\definecolor{textcolor}{rgb}{0.000000,0.000000,0.000000}%
\pgfsetstrokecolor{textcolor}%
\pgfsetfillcolor{textcolor}%
\pgftext[x=0.406944in,y=4.462684in,left,base]{\color{textcolor}\rmfamily\fontsize{10.000000}{12.000000}\selectfont 2800}%
\end{pgfscope}%
\begin{pgfscope}%
\pgfpathrectangle{\pgfqpoint{0.781944in}{2.977778in}}{\pgfqpoint{5.019444in}{1.650000in}}%
\pgfusepath{clip}%
\pgfsetrectcap%
\pgfsetroundjoin%
\pgfsetlinewidth{0.803000pt}%
\definecolor{currentstroke}{rgb}{0.690196,0.690196,0.690196}%
\pgfsetstrokecolor{currentstroke}%
\pgfsetstrokeopacity{0.300000}%
\pgfsetdash{}{0pt}%
\pgfpathmoveto{\pgfqpoint{0.781944in}{2.999679in}}%
\pgfpathlineto{\pgfqpoint{5.801389in}{2.999679in}}%
\pgfusepath{stroke}%
\end{pgfscope}%
\begin{pgfscope}%
\pgfsetbuttcap%
\pgfsetroundjoin%
\definecolor{currentfill}{rgb}{0.000000,0.000000,0.000000}%
\pgfsetfillcolor{currentfill}%
\pgfsetlinewidth{0.602250pt}%
\definecolor{currentstroke}{rgb}{0.000000,0.000000,0.000000}%
\pgfsetstrokecolor{currentstroke}%
\pgfsetdash{}{0pt}%
\pgfsys@defobject{currentmarker}{\pgfqpoint{-0.027778in}{0.000000in}}{\pgfqpoint{0.000000in}{0.000000in}}{%
\pgfpathmoveto{\pgfqpoint{0.000000in}{0.000000in}}%
\pgfpathlineto{\pgfqpoint{-0.027778in}{0.000000in}}%
\pgfusepath{stroke,fill}%
}%
\begin{pgfscope}%
\pgfsys@transformshift{0.781944in}{2.999679in}%
\pgfsys@useobject{currentmarker}{}%
\end{pgfscope}%
\end{pgfscope}%
\begin{pgfscope}%
\pgfsetbuttcap%
\pgfsetroundjoin%
\definecolor{currentfill}{rgb}{0.000000,0.000000,0.000000}%
\pgfsetfillcolor{currentfill}%
\pgfsetlinewidth{0.602250pt}%
\definecolor{currentstroke}{rgb}{0.000000,0.000000,0.000000}%
\pgfsetstrokecolor{currentstroke}%
\pgfsetdash{}{0pt}%
\pgfsys@defobject{currentmarker}{\pgfqpoint{0.000000in}{0.000000in}}{\pgfqpoint{0.027778in}{0.000000in}}{%
\pgfpathmoveto{\pgfqpoint{0.000000in}{0.000000in}}%
\pgfpathlineto{\pgfqpoint{0.027778in}{0.000000in}}%
\pgfusepath{stroke,fill}%
}%
\begin{pgfscope}%
\pgfsys@transformshift{5.801389in}{2.999679in}%
\pgfsys@useobject{currentmarker}{}%
\end{pgfscope}%
\end{pgfscope}%
\begin{pgfscope}%
\pgfpathrectangle{\pgfqpoint{0.781944in}{2.977778in}}{\pgfqpoint{5.019444in}{1.650000in}}%
\pgfusepath{clip}%
\pgfsetrectcap%
\pgfsetroundjoin%
\pgfsetlinewidth{0.803000pt}%
\definecolor{currentstroke}{rgb}{0.690196,0.690196,0.690196}%
\pgfsetstrokecolor{currentstroke}%
\pgfsetstrokeopacity{0.300000}%
\pgfsetdash{}{0pt}%
\pgfpathmoveto{\pgfqpoint{0.781944in}{3.021581in}}%
\pgfpathlineto{\pgfqpoint{5.801389in}{3.021581in}}%
\pgfusepath{stroke}%
\end{pgfscope}%
\begin{pgfscope}%
\pgfsetbuttcap%
\pgfsetroundjoin%
\definecolor{currentfill}{rgb}{0.000000,0.000000,0.000000}%
\pgfsetfillcolor{currentfill}%
\pgfsetlinewidth{0.602250pt}%
\definecolor{currentstroke}{rgb}{0.000000,0.000000,0.000000}%
\pgfsetstrokecolor{currentstroke}%
\pgfsetdash{}{0pt}%
\pgfsys@defobject{currentmarker}{\pgfqpoint{-0.027778in}{0.000000in}}{\pgfqpoint{0.000000in}{0.000000in}}{%
\pgfpathmoveto{\pgfqpoint{0.000000in}{0.000000in}}%
\pgfpathlineto{\pgfqpoint{-0.027778in}{0.000000in}}%
\pgfusepath{stroke,fill}%
}%
\begin{pgfscope}%
\pgfsys@transformshift{0.781944in}{3.021581in}%
\pgfsys@useobject{currentmarker}{}%
\end{pgfscope}%
\end{pgfscope}%
\begin{pgfscope}%
\pgfsetbuttcap%
\pgfsetroundjoin%
\definecolor{currentfill}{rgb}{0.000000,0.000000,0.000000}%
\pgfsetfillcolor{currentfill}%
\pgfsetlinewidth{0.602250pt}%
\definecolor{currentstroke}{rgb}{0.000000,0.000000,0.000000}%
\pgfsetstrokecolor{currentstroke}%
\pgfsetdash{}{0pt}%
\pgfsys@defobject{currentmarker}{\pgfqpoint{0.000000in}{0.000000in}}{\pgfqpoint{0.027778in}{0.000000in}}{%
\pgfpathmoveto{\pgfqpoint{0.000000in}{0.000000in}}%
\pgfpathlineto{\pgfqpoint{0.027778in}{0.000000in}}%
\pgfusepath{stroke,fill}%
}%
\begin{pgfscope}%
\pgfsys@transformshift{5.801389in}{3.021581in}%
\pgfsys@useobject{currentmarker}{}%
\end{pgfscope}%
\end{pgfscope}%
\begin{pgfscope}%
\pgfpathrectangle{\pgfqpoint{0.781944in}{2.977778in}}{\pgfqpoint{5.019444in}{1.650000in}}%
\pgfusepath{clip}%
\pgfsetrectcap%
\pgfsetroundjoin%
\pgfsetlinewidth{0.803000pt}%
\definecolor{currentstroke}{rgb}{0.690196,0.690196,0.690196}%
\pgfsetstrokecolor{currentstroke}%
\pgfsetstrokeopacity{0.300000}%
\pgfsetdash{}{0pt}%
\pgfpathmoveto{\pgfqpoint{0.781944in}{3.043482in}}%
\pgfpathlineto{\pgfqpoint{5.801389in}{3.043482in}}%
\pgfusepath{stroke}%
\end{pgfscope}%
\begin{pgfscope}%
\pgfsetbuttcap%
\pgfsetroundjoin%
\definecolor{currentfill}{rgb}{0.000000,0.000000,0.000000}%
\pgfsetfillcolor{currentfill}%
\pgfsetlinewidth{0.602250pt}%
\definecolor{currentstroke}{rgb}{0.000000,0.000000,0.000000}%
\pgfsetstrokecolor{currentstroke}%
\pgfsetdash{}{0pt}%
\pgfsys@defobject{currentmarker}{\pgfqpoint{-0.027778in}{0.000000in}}{\pgfqpoint{0.000000in}{0.000000in}}{%
\pgfpathmoveto{\pgfqpoint{0.000000in}{0.000000in}}%
\pgfpathlineto{\pgfqpoint{-0.027778in}{0.000000in}}%
\pgfusepath{stroke,fill}%
}%
\begin{pgfscope}%
\pgfsys@transformshift{0.781944in}{3.043482in}%
\pgfsys@useobject{currentmarker}{}%
\end{pgfscope}%
\end{pgfscope}%
\begin{pgfscope}%
\pgfsetbuttcap%
\pgfsetroundjoin%
\definecolor{currentfill}{rgb}{0.000000,0.000000,0.000000}%
\pgfsetfillcolor{currentfill}%
\pgfsetlinewidth{0.602250pt}%
\definecolor{currentstroke}{rgb}{0.000000,0.000000,0.000000}%
\pgfsetstrokecolor{currentstroke}%
\pgfsetdash{}{0pt}%
\pgfsys@defobject{currentmarker}{\pgfqpoint{0.000000in}{0.000000in}}{\pgfqpoint{0.027778in}{0.000000in}}{%
\pgfpathmoveto{\pgfqpoint{0.000000in}{0.000000in}}%
\pgfpathlineto{\pgfqpoint{0.027778in}{0.000000in}}%
\pgfusepath{stroke,fill}%
}%
\begin{pgfscope}%
\pgfsys@transformshift{5.801389in}{3.043482in}%
\pgfsys@useobject{currentmarker}{}%
\end{pgfscope}%
\end{pgfscope}%
\begin{pgfscope}%
\pgfpathrectangle{\pgfqpoint{0.781944in}{2.977778in}}{\pgfqpoint{5.019444in}{1.650000in}}%
\pgfusepath{clip}%
\pgfsetrectcap%
\pgfsetroundjoin%
\pgfsetlinewidth{0.803000pt}%
\definecolor{currentstroke}{rgb}{0.690196,0.690196,0.690196}%
\pgfsetstrokecolor{currentstroke}%
\pgfsetstrokeopacity{0.300000}%
\pgfsetdash{}{0pt}%
\pgfpathmoveto{\pgfqpoint{0.781944in}{3.065384in}}%
\pgfpathlineto{\pgfqpoint{5.801389in}{3.065384in}}%
\pgfusepath{stroke}%
\end{pgfscope}%
\begin{pgfscope}%
\pgfsetbuttcap%
\pgfsetroundjoin%
\definecolor{currentfill}{rgb}{0.000000,0.000000,0.000000}%
\pgfsetfillcolor{currentfill}%
\pgfsetlinewidth{0.602250pt}%
\definecolor{currentstroke}{rgb}{0.000000,0.000000,0.000000}%
\pgfsetstrokecolor{currentstroke}%
\pgfsetdash{}{0pt}%
\pgfsys@defobject{currentmarker}{\pgfqpoint{-0.027778in}{0.000000in}}{\pgfqpoint{0.000000in}{0.000000in}}{%
\pgfpathmoveto{\pgfqpoint{0.000000in}{0.000000in}}%
\pgfpathlineto{\pgfqpoint{-0.027778in}{0.000000in}}%
\pgfusepath{stroke,fill}%
}%
\begin{pgfscope}%
\pgfsys@transformshift{0.781944in}{3.065384in}%
\pgfsys@useobject{currentmarker}{}%
\end{pgfscope}%
\end{pgfscope}%
\begin{pgfscope}%
\pgfsetbuttcap%
\pgfsetroundjoin%
\definecolor{currentfill}{rgb}{0.000000,0.000000,0.000000}%
\pgfsetfillcolor{currentfill}%
\pgfsetlinewidth{0.602250pt}%
\definecolor{currentstroke}{rgb}{0.000000,0.000000,0.000000}%
\pgfsetstrokecolor{currentstroke}%
\pgfsetdash{}{0pt}%
\pgfsys@defobject{currentmarker}{\pgfqpoint{0.000000in}{0.000000in}}{\pgfqpoint{0.027778in}{0.000000in}}{%
\pgfpathmoveto{\pgfqpoint{0.000000in}{0.000000in}}%
\pgfpathlineto{\pgfqpoint{0.027778in}{0.000000in}}%
\pgfusepath{stroke,fill}%
}%
\begin{pgfscope}%
\pgfsys@transformshift{5.801389in}{3.065384in}%
\pgfsys@useobject{currentmarker}{}%
\end{pgfscope}%
\end{pgfscope}%
\begin{pgfscope}%
\pgfpathrectangle{\pgfqpoint{0.781944in}{2.977778in}}{\pgfqpoint{5.019444in}{1.650000in}}%
\pgfusepath{clip}%
\pgfsetrectcap%
\pgfsetroundjoin%
\pgfsetlinewidth{0.803000pt}%
\definecolor{currentstroke}{rgb}{0.690196,0.690196,0.690196}%
\pgfsetstrokecolor{currentstroke}%
\pgfsetstrokeopacity{0.300000}%
\pgfsetdash{}{0pt}%
\pgfpathmoveto{\pgfqpoint{0.781944in}{3.087285in}}%
\pgfpathlineto{\pgfqpoint{5.801389in}{3.087285in}}%
\pgfusepath{stroke}%
\end{pgfscope}%
\begin{pgfscope}%
\pgfsetbuttcap%
\pgfsetroundjoin%
\definecolor{currentfill}{rgb}{0.000000,0.000000,0.000000}%
\pgfsetfillcolor{currentfill}%
\pgfsetlinewidth{0.602250pt}%
\definecolor{currentstroke}{rgb}{0.000000,0.000000,0.000000}%
\pgfsetstrokecolor{currentstroke}%
\pgfsetdash{}{0pt}%
\pgfsys@defobject{currentmarker}{\pgfqpoint{-0.027778in}{0.000000in}}{\pgfqpoint{0.000000in}{0.000000in}}{%
\pgfpathmoveto{\pgfqpoint{0.000000in}{0.000000in}}%
\pgfpathlineto{\pgfqpoint{-0.027778in}{0.000000in}}%
\pgfusepath{stroke,fill}%
}%
\begin{pgfscope}%
\pgfsys@transformshift{0.781944in}{3.087285in}%
\pgfsys@useobject{currentmarker}{}%
\end{pgfscope}%
\end{pgfscope}%
\begin{pgfscope}%
\pgfsetbuttcap%
\pgfsetroundjoin%
\definecolor{currentfill}{rgb}{0.000000,0.000000,0.000000}%
\pgfsetfillcolor{currentfill}%
\pgfsetlinewidth{0.602250pt}%
\definecolor{currentstroke}{rgb}{0.000000,0.000000,0.000000}%
\pgfsetstrokecolor{currentstroke}%
\pgfsetdash{}{0pt}%
\pgfsys@defobject{currentmarker}{\pgfqpoint{0.000000in}{0.000000in}}{\pgfqpoint{0.027778in}{0.000000in}}{%
\pgfpathmoveto{\pgfqpoint{0.000000in}{0.000000in}}%
\pgfpathlineto{\pgfqpoint{0.027778in}{0.000000in}}%
\pgfusepath{stroke,fill}%
}%
\begin{pgfscope}%
\pgfsys@transformshift{5.801389in}{3.087285in}%
\pgfsys@useobject{currentmarker}{}%
\end{pgfscope}%
\end{pgfscope}%
\begin{pgfscope}%
\pgfpathrectangle{\pgfqpoint{0.781944in}{2.977778in}}{\pgfqpoint{5.019444in}{1.650000in}}%
\pgfusepath{clip}%
\pgfsetrectcap%
\pgfsetroundjoin%
\pgfsetlinewidth{0.803000pt}%
\definecolor{currentstroke}{rgb}{0.690196,0.690196,0.690196}%
\pgfsetstrokecolor{currentstroke}%
\pgfsetstrokeopacity{0.300000}%
\pgfsetdash{}{0pt}%
\pgfpathmoveto{\pgfqpoint{0.781944in}{3.109186in}}%
\pgfpathlineto{\pgfqpoint{5.801389in}{3.109186in}}%
\pgfusepath{stroke}%
\end{pgfscope}%
\begin{pgfscope}%
\pgfsetbuttcap%
\pgfsetroundjoin%
\definecolor{currentfill}{rgb}{0.000000,0.000000,0.000000}%
\pgfsetfillcolor{currentfill}%
\pgfsetlinewidth{0.602250pt}%
\definecolor{currentstroke}{rgb}{0.000000,0.000000,0.000000}%
\pgfsetstrokecolor{currentstroke}%
\pgfsetdash{}{0pt}%
\pgfsys@defobject{currentmarker}{\pgfqpoint{-0.027778in}{0.000000in}}{\pgfqpoint{0.000000in}{0.000000in}}{%
\pgfpathmoveto{\pgfqpoint{0.000000in}{0.000000in}}%
\pgfpathlineto{\pgfqpoint{-0.027778in}{0.000000in}}%
\pgfusepath{stroke,fill}%
}%
\begin{pgfscope}%
\pgfsys@transformshift{0.781944in}{3.109186in}%
\pgfsys@useobject{currentmarker}{}%
\end{pgfscope}%
\end{pgfscope}%
\begin{pgfscope}%
\pgfsetbuttcap%
\pgfsetroundjoin%
\definecolor{currentfill}{rgb}{0.000000,0.000000,0.000000}%
\pgfsetfillcolor{currentfill}%
\pgfsetlinewidth{0.602250pt}%
\definecolor{currentstroke}{rgb}{0.000000,0.000000,0.000000}%
\pgfsetstrokecolor{currentstroke}%
\pgfsetdash{}{0pt}%
\pgfsys@defobject{currentmarker}{\pgfqpoint{0.000000in}{0.000000in}}{\pgfqpoint{0.027778in}{0.000000in}}{%
\pgfpathmoveto{\pgfqpoint{0.000000in}{0.000000in}}%
\pgfpathlineto{\pgfqpoint{0.027778in}{0.000000in}}%
\pgfusepath{stroke,fill}%
}%
\begin{pgfscope}%
\pgfsys@transformshift{5.801389in}{3.109186in}%
\pgfsys@useobject{currentmarker}{}%
\end{pgfscope}%
\end{pgfscope}%
\begin{pgfscope}%
\pgfpathrectangle{\pgfqpoint{0.781944in}{2.977778in}}{\pgfqpoint{5.019444in}{1.650000in}}%
\pgfusepath{clip}%
\pgfsetrectcap%
\pgfsetroundjoin%
\pgfsetlinewidth{0.803000pt}%
\definecolor{currentstroke}{rgb}{0.690196,0.690196,0.690196}%
\pgfsetstrokecolor{currentstroke}%
\pgfsetstrokeopacity{0.300000}%
\pgfsetdash{}{0pt}%
\pgfpathmoveto{\pgfqpoint{0.781944in}{3.131088in}}%
\pgfpathlineto{\pgfqpoint{5.801389in}{3.131088in}}%
\pgfusepath{stroke}%
\end{pgfscope}%
\begin{pgfscope}%
\pgfsetbuttcap%
\pgfsetroundjoin%
\definecolor{currentfill}{rgb}{0.000000,0.000000,0.000000}%
\pgfsetfillcolor{currentfill}%
\pgfsetlinewidth{0.602250pt}%
\definecolor{currentstroke}{rgb}{0.000000,0.000000,0.000000}%
\pgfsetstrokecolor{currentstroke}%
\pgfsetdash{}{0pt}%
\pgfsys@defobject{currentmarker}{\pgfqpoint{-0.027778in}{0.000000in}}{\pgfqpoint{0.000000in}{0.000000in}}{%
\pgfpathmoveto{\pgfqpoint{0.000000in}{0.000000in}}%
\pgfpathlineto{\pgfqpoint{-0.027778in}{0.000000in}}%
\pgfusepath{stroke,fill}%
}%
\begin{pgfscope}%
\pgfsys@transformshift{0.781944in}{3.131088in}%
\pgfsys@useobject{currentmarker}{}%
\end{pgfscope}%
\end{pgfscope}%
\begin{pgfscope}%
\pgfsetbuttcap%
\pgfsetroundjoin%
\definecolor{currentfill}{rgb}{0.000000,0.000000,0.000000}%
\pgfsetfillcolor{currentfill}%
\pgfsetlinewidth{0.602250pt}%
\definecolor{currentstroke}{rgb}{0.000000,0.000000,0.000000}%
\pgfsetstrokecolor{currentstroke}%
\pgfsetdash{}{0pt}%
\pgfsys@defobject{currentmarker}{\pgfqpoint{0.000000in}{0.000000in}}{\pgfqpoint{0.027778in}{0.000000in}}{%
\pgfpathmoveto{\pgfqpoint{0.000000in}{0.000000in}}%
\pgfpathlineto{\pgfqpoint{0.027778in}{0.000000in}}%
\pgfusepath{stroke,fill}%
}%
\begin{pgfscope}%
\pgfsys@transformshift{5.801389in}{3.131088in}%
\pgfsys@useobject{currentmarker}{}%
\end{pgfscope}%
\end{pgfscope}%
\begin{pgfscope}%
\pgfpathrectangle{\pgfqpoint{0.781944in}{2.977778in}}{\pgfqpoint{5.019444in}{1.650000in}}%
\pgfusepath{clip}%
\pgfsetrectcap%
\pgfsetroundjoin%
\pgfsetlinewidth{0.803000pt}%
\definecolor{currentstroke}{rgb}{0.690196,0.690196,0.690196}%
\pgfsetstrokecolor{currentstroke}%
\pgfsetstrokeopacity{0.300000}%
\pgfsetdash{}{0pt}%
\pgfpathmoveto{\pgfqpoint{0.781944in}{3.152989in}}%
\pgfpathlineto{\pgfqpoint{5.801389in}{3.152989in}}%
\pgfusepath{stroke}%
\end{pgfscope}%
\begin{pgfscope}%
\pgfsetbuttcap%
\pgfsetroundjoin%
\definecolor{currentfill}{rgb}{0.000000,0.000000,0.000000}%
\pgfsetfillcolor{currentfill}%
\pgfsetlinewidth{0.602250pt}%
\definecolor{currentstroke}{rgb}{0.000000,0.000000,0.000000}%
\pgfsetstrokecolor{currentstroke}%
\pgfsetdash{}{0pt}%
\pgfsys@defobject{currentmarker}{\pgfqpoint{-0.027778in}{0.000000in}}{\pgfqpoint{0.000000in}{0.000000in}}{%
\pgfpathmoveto{\pgfqpoint{0.000000in}{0.000000in}}%
\pgfpathlineto{\pgfqpoint{-0.027778in}{0.000000in}}%
\pgfusepath{stroke,fill}%
}%
\begin{pgfscope}%
\pgfsys@transformshift{0.781944in}{3.152989in}%
\pgfsys@useobject{currentmarker}{}%
\end{pgfscope}%
\end{pgfscope}%
\begin{pgfscope}%
\pgfsetbuttcap%
\pgfsetroundjoin%
\definecolor{currentfill}{rgb}{0.000000,0.000000,0.000000}%
\pgfsetfillcolor{currentfill}%
\pgfsetlinewidth{0.602250pt}%
\definecolor{currentstroke}{rgb}{0.000000,0.000000,0.000000}%
\pgfsetstrokecolor{currentstroke}%
\pgfsetdash{}{0pt}%
\pgfsys@defobject{currentmarker}{\pgfqpoint{0.000000in}{0.000000in}}{\pgfqpoint{0.027778in}{0.000000in}}{%
\pgfpathmoveto{\pgfqpoint{0.000000in}{0.000000in}}%
\pgfpathlineto{\pgfqpoint{0.027778in}{0.000000in}}%
\pgfusepath{stroke,fill}%
}%
\begin{pgfscope}%
\pgfsys@transformshift{5.801389in}{3.152989in}%
\pgfsys@useobject{currentmarker}{}%
\end{pgfscope}%
\end{pgfscope}%
\begin{pgfscope}%
\pgfpathrectangle{\pgfqpoint{0.781944in}{2.977778in}}{\pgfqpoint{5.019444in}{1.650000in}}%
\pgfusepath{clip}%
\pgfsetrectcap%
\pgfsetroundjoin%
\pgfsetlinewidth{0.803000pt}%
\definecolor{currentstroke}{rgb}{0.690196,0.690196,0.690196}%
\pgfsetstrokecolor{currentstroke}%
\pgfsetstrokeopacity{0.300000}%
\pgfsetdash{}{0pt}%
\pgfpathmoveto{\pgfqpoint{0.781944in}{3.174891in}}%
\pgfpathlineto{\pgfqpoint{5.801389in}{3.174891in}}%
\pgfusepath{stroke}%
\end{pgfscope}%
\begin{pgfscope}%
\pgfsetbuttcap%
\pgfsetroundjoin%
\definecolor{currentfill}{rgb}{0.000000,0.000000,0.000000}%
\pgfsetfillcolor{currentfill}%
\pgfsetlinewidth{0.602250pt}%
\definecolor{currentstroke}{rgb}{0.000000,0.000000,0.000000}%
\pgfsetstrokecolor{currentstroke}%
\pgfsetdash{}{0pt}%
\pgfsys@defobject{currentmarker}{\pgfqpoint{-0.027778in}{0.000000in}}{\pgfqpoint{0.000000in}{0.000000in}}{%
\pgfpathmoveto{\pgfqpoint{0.000000in}{0.000000in}}%
\pgfpathlineto{\pgfqpoint{-0.027778in}{0.000000in}}%
\pgfusepath{stroke,fill}%
}%
\begin{pgfscope}%
\pgfsys@transformshift{0.781944in}{3.174891in}%
\pgfsys@useobject{currentmarker}{}%
\end{pgfscope}%
\end{pgfscope}%
\begin{pgfscope}%
\pgfsetbuttcap%
\pgfsetroundjoin%
\definecolor{currentfill}{rgb}{0.000000,0.000000,0.000000}%
\pgfsetfillcolor{currentfill}%
\pgfsetlinewidth{0.602250pt}%
\definecolor{currentstroke}{rgb}{0.000000,0.000000,0.000000}%
\pgfsetstrokecolor{currentstroke}%
\pgfsetdash{}{0pt}%
\pgfsys@defobject{currentmarker}{\pgfqpoint{0.000000in}{0.000000in}}{\pgfqpoint{0.027778in}{0.000000in}}{%
\pgfpathmoveto{\pgfqpoint{0.000000in}{0.000000in}}%
\pgfpathlineto{\pgfqpoint{0.027778in}{0.000000in}}%
\pgfusepath{stroke,fill}%
}%
\begin{pgfscope}%
\pgfsys@transformshift{5.801389in}{3.174891in}%
\pgfsys@useobject{currentmarker}{}%
\end{pgfscope}%
\end{pgfscope}%
\begin{pgfscope}%
\pgfpathrectangle{\pgfqpoint{0.781944in}{2.977778in}}{\pgfqpoint{5.019444in}{1.650000in}}%
\pgfusepath{clip}%
\pgfsetrectcap%
\pgfsetroundjoin%
\pgfsetlinewidth{0.803000pt}%
\definecolor{currentstroke}{rgb}{0.690196,0.690196,0.690196}%
\pgfsetstrokecolor{currentstroke}%
\pgfsetstrokeopacity{0.300000}%
\pgfsetdash{}{0pt}%
\pgfpathmoveto{\pgfqpoint{0.781944in}{3.218694in}}%
\pgfpathlineto{\pgfqpoint{5.801389in}{3.218694in}}%
\pgfusepath{stroke}%
\end{pgfscope}%
\begin{pgfscope}%
\pgfsetbuttcap%
\pgfsetroundjoin%
\definecolor{currentfill}{rgb}{0.000000,0.000000,0.000000}%
\pgfsetfillcolor{currentfill}%
\pgfsetlinewidth{0.602250pt}%
\definecolor{currentstroke}{rgb}{0.000000,0.000000,0.000000}%
\pgfsetstrokecolor{currentstroke}%
\pgfsetdash{}{0pt}%
\pgfsys@defobject{currentmarker}{\pgfqpoint{-0.027778in}{0.000000in}}{\pgfqpoint{0.000000in}{0.000000in}}{%
\pgfpathmoveto{\pgfqpoint{0.000000in}{0.000000in}}%
\pgfpathlineto{\pgfqpoint{-0.027778in}{0.000000in}}%
\pgfusepath{stroke,fill}%
}%
\begin{pgfscope}%
\pgfsys@transformshift{0.781944in}{3.218694in}%
\pgfsys@useobject{currentmarker}{}%
\end{pgfscope}%
\end{pgfscope}%
\begin{pgfscope}%
\pgfsetbuttcap%
\pgfsetroundjoin%
\definecolor{currentfill}{rgb}{0.000000,0.000000,0.000000}%
\pgfsetfillcolor{currentfill}%
\pgfsetlinewidth{0.602250pt}%
\definecolor{currentstroke}{rgb}{0.000000,0.000000,0.000000}%
\pgfsetstrokecolor{currentstroke}%
\pgfsetdash{}{0pt}%
\pgfsys@defobject{currentmarker}{\pgfqpoint{0.000000in}{0.000000in}}{\pgfqpoint{0.027778in}{0.000000in}}{%
\pgfpathmoveto{\pgfqpoint{0.000000in}{0.000000in}}%
\pgfpathlineto{\pgfqpoint{0.027778in}{0.000000in}}%
\pgfusepath{stroke,fill}%
}%
\begin{pgfscope}%
\pgfsys@transformshift{5.801389in}{3.218694in}%
\pgfsys@useobject{currentmarker}{}%
\end{pgfscope}%
\end{pgfscope}%
\begin{pgfscope}%
\pgfpathrectangle{\pgfqpoint{0.781944in}{2.977778in}}{\pgfqpoint{5.019444in}{1.650000in}}%
\pgfusepath{clip}%
\pgfsetrectcap%
\pgfsetroundjoin%
\pgfsetlinewidth{0.803000pt}%
\definecolor{currentstroke}{rgb}{0.690196,0.690196,0.690196}%
\pgfsetstrokecolor{currentstroke}%
\pgfsetstrokeopacity{0.300000}%
\pgfsetdash{}{0pt}%
\pgfpathmoveto{\pgfqpoint{0.781944in}{3.240595in}}%
\pgfpathlineto{\pgfqpoint{5.801389in}{3.240595in}}%
\pgfusepath{stroke}%
\end{pgfscope}%
\begin{pgfscope}%
\pgfsetbuttcap%
\pgfsetroundjoin%
\definecolor{currentfill}{rgb}{0.000000,0.000000,0.000000}%
\pgfsetfillcolor{currentfill}%
\pgfsetlinewidth{0.602250pt}%
\definecolor{currentstroke}{rgb}{0.000000,0.000000,0.000000}%
\pgfsetstrokecolor{currentstroke}%
\pgfsetdash{}{0pt}%
\pgfsys@defobject{currentmarker}{\pgfqpoint{-0.027778in}{0.000000in}}{\pgfqpoint{0.000000in}{0.000000in}}{%
\pgfpathmoveto{\pgfqpoint{0.000000in}{0.000000in}}%
\pgfpathlineto{\pgfqpoint{-0.027778in}{0.000000in}}%
\pgfusepath{stroke,fill}%
}%
\begin{pgfscope}%
\pgfsys@transformshift{0.781944in}{3.240595in}%
\pgfsys@useobject{currentmarker}{}%
\end{pgfscope}%
\end{pgfscope}%
\begin{pgfscope}%
\pgfsetbuttcap%
\pgfsetroundjoin%
\definecolor{currentfill}{rgb}{0.000000,0.000000,0.000000}%
\pgfsetfillcolor{currentfill}%
\pgfsetlinewidth{0.602250pt}%
\definecolor{currentstroke}{rgb}{0.000000,0.000000,0.000000}%
\pgfsetstrokecolor{currentstroke}%
\pgfsetdash{}{0pt}%
\pgfsys@defobject{currentmarker}{\pgfqpoint{0.000000in}{0.000000in}}{\pgfqpoint{0.027778in}{0.000000in}}{%
\pgfpathmoveto{\pgfqpoint{0.000000in}{0.000000in}}%
\pgfpathlineto{\pgfqpoint{0.027778in}{0.000000in}}%
\pgfusepath{stroke,fill}%
}%
\begin{pgfscope}%
\pgfsys@transformshift{5.801389in}{3.240595in}%
\pgfsys@useobject{currentmarker}{}%
\end{pgfscope}%
\end{pgfscope}%
\begin{pgfscope}%
\pgfpathrectangle{\pgfqpoint{0.781944in}{2.977778in}}{\pgfqpoint{5.019444in}{1.650000in}}%
\pgfusepath{clip}%
\pgfsetrectcap%
\pgfsetroundjoin%
\pgfsetlinewidth{0.803000pt}%
\definecolor{currentstroke}{rgb}{0.690196,0.690196,0.690196}%
\pgfsetstrokecolor{currentstroke}%
\pgfsetstrokeopacity{0.300000}%
\pgfsetdash{}{0pt}%
\pgfpathmoveto{\pgfqpoint{0.781944in}{3.262497in}}%
\pgfpathlineto{\pgfqpoint{5.801389in}{3.262497in}}%
\pgfusepath{stroke}%
\end{pgfscope}%
\begin{pgfscope}%
\pgfsetbuttcap%
\pgfsetroundjoin%
\definecolor{currentfill}{rgb}{0.000000,0.000000,0.000000}%
\pgfsetfillcolor{currentfill}%
\pgfsetlinewidth{0.602250pt}%
\definecolor{currentstroke}{rgb}{0.000000,0.000000,0.000000}%
\pgfsetstrokecolor{currentstroke}%
\pgfsetdash{}{0pt}%
\pgfsys@defobject{currentmarker}{\pgfqpoint{-0.027778in}{0.000000in}}{\pgfqpoint{0.000000in}{0.000000in}}{%
\pgfpathmoveto{\pgfqpoint{0.000000in}{0.000000in}}%
\pgfpathlineto{\pgfqpoint{-0.027778in}{0.000000in}}%
\pgfusepath{stroke,fill}%
}%
\begin{pgfscope}%
\pgfsys@transformshift{0.781944in}{3.262497in}%
\pgfsys@useobject{currentmarker}{}%
\end{pgfscope}%
\end{pgfscope}%
\begin{pgfscope}%
\pgfsetbuttcap%
\pgfsetroundjoin%
\definecolor{currentfill}{rgb}{0.000000,0.000000,0.000000}%
\pgfsetfillcolor{currentfill}%
\pgfsetlinewidth{0.602250pt}%
\definecolor{currentstroke}{rgb}{0.000000,0.000000,0.000000}%
\pgfsetstrokecolor{currentstroke}%
\pgfsetdash{}{0pt}%
\pgfsys@defobject{currentmarker}{\pgfqpoint{0.000000in}{0.000000in}}{\pgfqpoint{0.027778in}{0.000000in}}{%
\pgfpathmoveto{\pgfqpoint{0.000000in}{0.000000in}}%
\pgfpathlineto{\pgfqpoint{0.027778in}{0.000000in}}%
\pgfusepath{stroke,fill}%
}%
\begin{pgfscope}%
\pgfsys@transformshift{5.801389in}{3.262497in}%
\pgfsys@useobject{currentmarker}{}%
\end{pgfscope}%
\end{pgfscope}%
\begin{pgfscope}%
\pgfpathrectangle{\pgfqpoint{0.781944in}{2.977778in}}{\pgfqpoint{5.019444in}{1.650000in}}%
\pgfusepath{clip}%
\pgfsetrectcap%
\pgfsetroundjoin%
\pgfsetlinewidth{0.803000pt}%
\definecolor{currentstroke}{rgb}{0.690196,0.690196,0.690196}%
\pgfsetstrokecolor{currentstroke}%
\pgfsetstrokeopacity{0.300000}%
\pgfsetdash{}{0pt}%
\pgfpathmoveto{\pgfqpoint{0.781944in}{3.284398in}}%
\pgfpathlineto{\pgfqpoint{5.801389in}{3.284398in}}%
\pgfusepath{stroke}%
\end{pgfscope}%
\begin{pgfscope}%
\pgfsetbuttcap%
\pgfsetroundjoin%
\definecolor{currentfill}{rgb}{0.000000,0.000000,0.000000}%
\pgfsetfillcolor{currentfill}%
\pgfsetlinewidth{0.602250pt}%
\definecolor{currentstroke}{rgb}{0.000000,0.000000,0.000000}%
\pgfsetstrokecolor{currentstroke}%
\pgfsetdash{}{0pt}%
\pgfsys@defobject{currentmarker}{\pgfqpoint{-0.027778in}{0.000000in}}{\pgfqpoint{0.000000in}{0.000000in}}{%
\pgfpathmoveto{\pgfqpoint{0.000000in}{0.000000in}}%
\pgfpathlineto{\pgfqpoint{-0.027778in}{0.000000in}}%
\pgfusepath{stroke,fill}%
}%
\begin{pgfscope}%
\pgfsys@transformshift{0.781944in}{3.284398in}%
\pgfsys@useobject{currentmarker}{}%
\end{pgfscope}%
\end{pgfscope}%
\begin{pgfscope}%
\pgfsetbuttcap%
\pgfsetroundjoin%
\definecolor{currentfill}{rgb}{0.000000,0.000000,0.000000}%
\pgfsetfillcolor{currentfill}%
\pgfsetlinewidth{0.602250pt}%
\definecolor{currentstroke}{rgb}{0.000000,0.000000,0.000000}%
\pgfsetstrokecolor{currentstroke}%
\pgfsetdash{}{0pt}%
\pgfsys@defobject{currentmarker}{\pgfqpoint{0.000000in}{0.000000in}}{\pgfqpoint{0.027778in}{0.000000in}}{%
\pgfpathmoveto{\pgfqpoint{0.000000in}{0.000000in}}%
\pgfpathlineto{\pgfqpoint{0.027778in}{0.000000in}}%
\pgfusepath{stroke,fill}%
}%
\begin{pgfscope}%
\pgfsys@transformshift{5.801389in}{3.284398in}%
\pgfsys@useobject{currentmarker}{}%
\end{pgfscope}%
\end{pgfscope}%
\begin{pgfscope}%
\pgfpathrectangle{\pgfqpoint{0.781944in}{2.977778in}}{\pgfqpoint{5.019444in}{1.650000in}}%
\pgfusepath{clip}%
\pgfsetrectcap%
\pgfsetroundjoin%
\pgfsetlinewidth{0.803000pt}%
\definecolor{currentstroke}{rgb}{0.690196,0.690196,0.690196}%
\pgfsetstrokecolor{currentstroke}%
\pgfsetstrokeopacity{0.300000}%
\pgfsetdash{}{0pt}%
\pgfpathmoveto{\pgfqpoint{0.781944in}{3.306299in}}%
\pgfpathlineto{\pgfqpoint{5.801389in}{3.306299in}}%
\pgfusepath{stroke}%
\end{pgfscope}%
\begin{pgfscope}%
\pgfsetbuttcap%
\pgfsetroundjoin%
\definecolor{currentfill}{rgb}{0.000000,0.000000,0.000000}%
\pgfsetfillcolor{currentfill}%
\pgfsetlinewidth{0.602250pt}%
\definecolor{currentstroke}{rgb}{0.000000,0.000000,0.000000}%
\pgfsetstrokecolor{currentstroke}%
\pgfsetdash{}{0pt}%
\pgfsys@defobject{currentmarker}{\pgfqpoint{-0.027778in}{0.000000in}}{\pgfqpoint{0.000000in}{0.000000in}}{%
\pgfpathmoveto{\pgfqpoint{0.000000in}{0.000000in}}%
\pgfpathlineto{\pgfqpoint{-0.027778in}{0.000000in}}%
\pgfusepath{stroke,fill}%
}%
\begin{pgfscope}%
\pgfsys@transformshift{0.781944in}{3.306299in}%
\pgfsys@useobject{currentmarker}{}%
\end{pgfscope}%
\end{pgfscope}%
\begin{pgfscope}%
\pgfsetbuttcap%
\pgfsetroundjoin%
\definecolor{currentfill}{rgb}{0.000000,0.000000,0.000000}%
\pgfsetfillcolor{currentfill}%
\pgfsetlinewidth{0.602250pt}%
\definecolor{currentstroke}{rgb}{0.000000,0.000000,0.000000}%
\pgfsetstrokecolor{currentstroke}%
\pgfsetdash{}{0pt}%
\pgfsys@defobject{currentmarker}{\pgfqpoint{0.000000in}{0.000000in}}{\pgfqpoint{0.027778in}{0.000000in}}{%
\pgfpathmoveto{\pgfqpoint{0.000000in}{0.000000in}}%
\pgfpathlineto{\pgfqpoint{0.027778in}{0.000000in}}%
\pgfusepath{stroke,fill}%
}%
\begin{pgfscope}%
\pgfsys@transformshift{5.801389in}{3.306299in}%
\pgfsys@useobject{currentmarker}{}%
\end{pgfscope}%
\end{pgfscope}%
\begin{pgfscope}%
\pgfpathrectangle{\pgfqpoint{0.781944in}{2.977778in}}{\pgfqpoint{5.019444in}{1.650000in}}%
\pgfusepath{clip}%
\pgfsetrectcap%
\pgfsetroundjoin%
\pgfsetlinewidth{0.803000pt}%
\definecolor{currentstroke}{rgb}{0.690196,0.690196,0.690196}%
\pgfsetstrokecolor{currentstroke}%
\pgfsetstrokeopacity{0.300000}%
\pgfsetdash{}{0pt}%
\pgfpathmoveto{\pgfqpoint{0.781944in}{3.328201in}}%
\pgfpathlineto{\pgfqpoint{5.801389in}{3.328201in}}%
\pgfusepath{stroke}%
\end{pgfscope}%
\begin{pgfscope}%
\pgfsetbuttcap%
\pgfsetroundjoin%
\definecolor{currentfill}{rgb}{0.000000,0.000000,0.000000}%
\pgfsetfillcolor{currentfill}%
\pgfsetlinewidth{0.602250pt}%
\definecolor{currentstroke}{rgb}{0.000000,0.000000,0.000000}%
\pgfsetstrokecolor{currentstroke}%
\pgfsetdash{}{0pt}%
\pgfsys@defobject{currentmarker}{\pgfqpoint{-0.027778in}{0.000000in}}{\pgfqpoint{0.000000in}{0.000000in}}{%
\pgfpathmoveto{\pgfqpoint{0.000000in}{0.000000in}}%
\pgfpathlineto{\pgfqpoint{-0.027778in}{0.000000in}}%
\pgfusepath{stroke,fill}%
}%
\begin{pgfscope}%
\pgfsys@transformshift{0.781944in}{3.328201in}%
\pgfsys@useobject{currentmarker}{}%
\end{pgfscope}%
\end{pgfscope}%
\begin{pgfscope}%
\pgfsetbuttcap%
\pgfsetroundjoin%
\definecolor{currentfill}{rgb}{0.000000,0.000000,0.000000}%
\pgfsetfillcolor{currentfill}%
\pgfsetlinewidth{0.602250pt}%
\definecolor{currentstroke}{rgb}{0.000000,0.000000,0.000000}%
\pgfsetstrokecolor{currentstroke}%
\pgfsetdash{}{0pt}%
\pgfsys@defobject{currentmarker}{\pgfqpoint{0.000000in}{0.000000in}}{\pgfqpoint{0.027778in}{0.000000in}}{%
\pgfpathmoveto{\pgfqpoint{0.000000in}{0.000000in}}%
\pgfpathlineto{\pgfqpoint{0.027778in}{0.000000in}}%
\pgfusepath{stroke,fill}%
}%
\begin{pgfscope}%
\pgfsys@transformshift{5.801389in}{3.328201in}%
\pgfsys@useobject{currentmarker}{}%
\end{pgfscope}%
\end{pgfscope}%
\begin{pgfscope}%
\pgfpathrectangle{\pgfqpoint{0.781944in}{2.977778in}}{\pgfqpoint{5.019444in}{1.650000in}}%
\pgfusepath{clip}%
\pgfsetrectcap%
\pgfsetroundjoin%
\pgfsetlinewidth{0.803000pt}%
\definecolor{currentstroke}{rgb}{0.690196,0.690196,0.690196}%
\pgfsetstrokecolor{currentstroke}%
\pgfsetstrokeopacity{0.300000}%
\pgfsetdash{}{0pt}%
\pgfpathmoveto{\pgfqpoint{0.781944in}{3.350102in}}%
\pgfpathlineto{\pgfqpoint{5.801389in}{3.350102in}}%
\pgfusepath{stroke}%
\end{pgfscope}%
\begin{pgfscope}%
\pgfsetbuttcap%
\pgfsetroundjoin%
\definecolor{currentfill}{rgb}{0.000000,0.000000,0.000000}%
\pgfsetfillcolor{currentfill}%
\pgfsetlinewidth{0.602250pt}%
\definecolor{currentstroke}{rgb}{0.000000,0.000000,0.000000}%
\pgfsetstrokecolor{currentstroke}%
\pgfsetdash{}{0pt}%
\pgfsys@defobject{currentmarker}{\pgfqpoint{-0.027778in}{0.000000in}}{\pgfqpoint{0.000000in}{0.000000in}}{%
\pgfpathmoveto{\pgfqpoint{0.000000in}{0.000000in}}%
\pgfpathlineto{\pgfqpoint{-0.027778in}{0.000000in}}%
\pgfusepath{stroke,fill}%
}%
\begin{pgfscope}%
\pgfsys@transformshift{0.781944in}{3.350102in}%
\pgfsys@useobject{currentmarker}{}%
\end{pgfscope}%
\end{pgfscope}%
\begin{pgfscope}%
\pgfsetbuttcap%
\pgfsetroundjoin%
\definecolor{currentfill}{rgb}{0.000000,0.000000,0.000000}%
\pgfsetfillcolor{currentfill}%
\pgfsetlinewidth{0.602250pt}%
\definecolor{currentstroke}{rgb}{0.000000,0.000000,0.000000}%
\pgfsetstrokecolor{currentstroke}%
\pgfsetdash{}{0pt}%
\pgfsys@defobject{currentmarker}{\pgfqpoint{0.000000in}{0.000000in}}{\pgfqpoint{0.027778in}{0.000000in}}{%
\pgfpathmoveto{\pgfqpoint{0.000000in}{0.000000in}}%
\pgfpathlineto{\pgfqpoint{0.027778in}{0.000000in}}%
\pgfusepath{stroke,fill}%
}%
\begin{pgfscope}%
\pgfsys@transformshift{5.801389in}{3.350102in}%
\pgfsys@useobject{currentmarker}{}%
\end{pgfscope}%
\end{pgfscope}%
\begin{pgfscope}%
\pgfpathrectangle{\pgfqpoint{0.781944in}{2.977778in}}{\pgfqpoint{5.019444in}{1.650000in}}%
\pgfusepath{clip}%
\pgfsetrectcap%
\pgfsetroundjoin%
\pgfsetlinewidth{0.803000pt}%
\definecolor{currentstroke}{rgb}{0.690196,0.690196,0.690196}%
\pgfsetstrokecolor{currentstroke}%
\pgfsetstrokeopacity{0.300000}%
\pgfsetdash{}{0pt}%
\pgfpathmoveto{\pgfqpoint{0.781944in}{3.372004in}}%
\pgfpathlineto{\pgfqpoint{5.801389in}{3.372004in}}%
\pgfusepath{stroke}%
\end{pgfscope}%
\begin{pgfscope}%
\pgfsetbuttcap%
\pgfsetroundjoin%
\definecolor{currentfill}{rgb}{0.000000,0.000000,0.000000}%
\pgfsetfillcolor{currentfill}%
\pgfsetlinewidth{0.602250pt}%
\definecolor{currentstroke}{rgb}{0.000000,0.000000,0.000000}%
\pgfsetstrokecolor{currentstroke}%
\pgfsetdash{}{0pt}%
\pgfsys@defobject{currentmarker}{\pgfqpoint{-0.027778in}{0.000000in}}{\pgfqpoint{0.000000in}{0.000000in}}{%
\pgfpathmoveto{\pgfqpoint{0.000000in}{0.000000in}}%
\pgfpathlineto{\pgfqpoint{-0.027778in}{0.000000in}}%
\pgfusepath{stroke,fill}%
}%
\begin{pgfscope}%
\pgfsys@transformshift{0.781944in}{3.372004in}%
\pgfsys@useobject{currentmarker}{}%
\end{pgfscope}%
\end{pgfscope}%
\begin{pgfscope}%
\pgfsetbuttcap%
\pgfsetroundjoin%
\definecolor{currentfill}{rgb}{0.000000,0.000000,0.000000}%
\pgfsetfillcolor{currentfill}%
\pgfsetlinewidth{0.602250pt}%
\definecolor{currentstroke}{rgb}{0.000000,0.000000,0.000000}%
\pgfsetstrokecolor{currentstroke}%
\pgfsetdash{}{0pt}%
\pgfsys@defobject{currentmarker}{\pgfqpoint{0.000000in}{0.000000in}}{\pgfqpoint{0.027778in}{0.000000in}}{%
\pgfpathmoveto{\pgfqpoint{0.000000in}{0.000000in}}%
\pgfpathlineto{\pgfqpoint{0.027778in}{0.000000in}}%
\pgfusepath{stroke,fill}%
}%
\begin{pgfscope}%
\pgfsys@transformshift{5.801389in}{3.372004in}%
\pgfsys@useobject{currentmarker}{}%
\end{pgfscope}%
\end{pgfscope}%
\begin{pgfscope}%
\pgfpathrectangle{\pgfqpoint{0.781944in}{2.977778in}}{\pgfqpoint{5.019444in}{1.650000in}}%
\pgfusepath{clip}%
\pgfsetrectcap%
\pgfsetroundjoin%
\pgfsetlinewidth{0.803000pt}%
\definecolor{currentstroke}{rgb}{0.690196,0.690196,0.690196}%
\pgfsetstrokecolor{currentstroke}%
\pgfsetstrokeopacity{0.300000}%
\pgfsetdash{}{0pt}%
\pgfpathmoveto{\pgfqpoint{0.781944in}{3.393905in}}%
\pgfpathlineto{\pgfqpoint{5.801389in}{3.393905in}}%
\pgfusepath{stroke}%
\end{pgfscope}%
\begin{pgfscope}%
\pgfsetbuttcap%
\pgfsetroundjoin%
\definecolor{currentfill}{rgb}{0.000000,0.000000,0.000000}%
\pgfsetfillcolor{currentfill}%
\pgfsetlinewidth{0.602250pt}%
\definecolor{currentstroke}{rgb}{0.000000,0.000000,0.000000}%
\pgfsetstrokecolor{currentstroke}%
\pgfsetdash{}{0pt}%
\pgfsys@defobject{currentmarker}{\pgfqpoint{-0.027778in}{0.000000in}}{\pgfqpoint{0.000000in}{0.000000in}}{%
\pgfpathmoveto{\pgfqpoint{0.000000in}{0.000000in}}%
\pgfpathlineto{\pgfqpoint{-0.027778in}{0.000000in}}%
\pgfusepath{stroke,fill}%
}%
\begin{pgfscope}%
\pgfsys@transformshift{0.781944in}{3.393905in}%
\pgfsys@useobject{currentmarker}{}%
\end{pgfscope}%
\end{pgfscope}%
\begin{pgfscope}%
\pgfsetbuttcap%
\pgfsetroundjoin%
\definecolor{currentfill}{rgb}{0.000000,0.000000,0.000000}%
\pgfsetfillcolor{currentfill}%
\pgfsetlinewidth{0.602250pt}%
\definecolor{currentstroke}{rgb}{0.000000,0.000000,0.000000}%
\pgfsetstrokecolor{currentstroke}%
\pgfsetdash{}{0pt}%
\pgfsys@defobject{currentmarker}{\pgfqpoint{0.000000in}{0.000000in}}{\pgfqpoint{0.027778in}{0.000000in}}{%
\pgfpathmoveto{\pgfqpoint{0.000000in}{0.000000in}}%
\pgfpathlineto{\pgfqpoint{0.027778in}{0.000000in}}%
\pgfusepath{stroke,fill}%
}%
\begin{pgfscope}%
\pgfsys@transformshift{5.801389in}{3.393905in}%
\pgfsys@useobject{currentmarker}{}%
\end{pgfscope}%
\end{pgfscope}%
\begin{pgfscope}%
\pgfpathrectangle{\pgfqpoint{0.781944in}{2.977778in}}{\pgfqpoint{5.019444in}{1.650000in}}%
\pgfusepath{clip}%
\pgfsetrectcap%
\pgfsetroundjoin%
\pgfsetlinewidth{0.803000pt}%
\definecolor{currentstroke}{rgb}{0.690196,0.690196,0.690196}%
\pgfsetstrokecolor{currentstroke}%
\pgfsetstrokeopacity{0.300000}%
\pgfsetdash{}{0pt}%
\pgfpathmoveto{\pgfqpoint{0.781944in}{3.437708in}}%
\pgfpathlineto{\pgfqpoint{5.801389in}{3.437708in}}%
\pgfusepath{stroke}%
\end{pgfscope}%
\begin{pgfscope}%
\pgfsetbuttcap%
\pgfsetroundjoin%
\definecolor{currentfill}{rgb}{0.000000,0.000000,0.000000}%
\pgfsetfillcolor{currentfill}%
\pgfsetlinewidth{0.602250pt}%
\definecolor{currentstroke}{rgb}{0.000000,0.000000,0.000000}%
\pgfsetstrokecolor{currentstroke}%
\pgfsetdash{}{0pt}%
\pgfsys@defobject{currentmarker}{\pgfqpoint{-0.027778in}{0.000000in}}{\pgfqpoint{0.000000in}{0.000000in}}{%
\pgfpathmoveto{\pgfqpoint{0.000000in}{0.000000in}}%
\pgfpathlineto{\pgfqpoint{-0.027778in}{0.000000in}}%
\pgfusepath{stroke,fill}%
}%
\begin{pgfscope}%
\pgfsys@transformshift{0.781944in}{3.437708in}%
\pgfsys@useobject{currentmarker}{}%
\end{pgfscope}%
\end{pgfscope}%
\begin{pgfscope}%
\pgfsetbuttcap%
\pgfsetroundjoin%
\definecolor{currentfill}{rgb}{0.000000,0.000000,0.000000}%
\pgfsetfillcolor{currentfill}%
\pgfsetlinewidth{0.602250pt}%
\definecolor{currentstroke}{rgb}{0.000000,0.000000,0.000000}%
\pgfsetstrokecolor{currentstroke}%
\pgfsetdash{}{0pt}%
\pgfsys@defobject{currentmarker}{\pgfqpoint{0.000000in}{0.000000in}}{\pgfqpoint{0.027778in}{0.000000in}}{%
\pgfpathmoveto{\pgfqpoint{0.000000in}{0.000000in}}%
\pgfpathlineto{\pgfqpoint{0.027778in}{0.000000in}}%
\pgfusepath{stroke,fill}%
}%
\begin{pgfscope}%
\pgfsys@transformshift{5.801389in}{3.437708in}%
\pgfsys@useobject{currentmarker}{}%
\end{pgfscope}%
\end{pgfscope}%
\begin{pgfscope}%
\pgfpathrectangle{\pgfqpoint{0.781944in}{2.977778in}}{\pgfqpoint{5.019444in}{1.650000in}}%
\pgfusepath{clip}%
\pgfsetrectcap%
\pgfsetroundjoin%
\pgfsetlinewidth{0.803000pt}%
\definecolor{currentstroke}{rgb}{0.690196,0.690196,0.690196}%
\pgfsetstrokecolor{currentstroke}%
\pgfsetstrokeopacity{0.300000}%
\pgfsetdash{}{0pt}%
\pgfpathmoveto{\pgfqpoint{0.781944in}{3.459610in}}%
\pgfpathlineto{\pgfqpoint{5.801389in}{3.459610in}}%
\pgfusepath{stroke}%
\end{pgfscope}%
\begin{pgfscope}%
\pgfsetbuttcap%
\pgfsetroundjoin%
\definecolor{currentfill}{rgb}{0.000000,0.000000,0.000000}%
\pgfsetfillcolor{currentfill}%
\pgfsetlinewidth{0.602250pt}%
\definecolor{currentstroke}{rgb}{0.000000,0.000000,0.000000}%
\pgfsetstrokecolor{currentstroke}%
\pgfsetdash{}{0pt}%
\pgfsys@defobject{currentmarker}{\pgfqpoint{-0.027778in}{0.000000in}}{\pgfqpoint{0.000000in}{0.000000in}}{%
\pgfpathmoveto{\pgfqpoint{0.000000in}{0.000000in}}%
\pgfpathlineto{\pgfqpoint{-0.027778in}{0.000000in}}%
\pgfusepath{stroke,fill}%
}%
\begin{pgfscope}%
\pgfsys@transformshift{0.781944in}{3.459610in}%
\pgfsys@useobject{currentmarker}{}%
\end{pgfscope}%
\end{pgfscope}%
\begin{pgfscope}%
\pgfsetbuttcap%
\pgfsetroundjoin%
\definecolor{currentfill}{rgb}{0.000000,0.000000,0.000000}%
\pgfsetfillcolor{currentfill}%
\pgfsetlinewidth{0.602250pt}%
\definecolor{currentstroke}{rgb}{0.000000,0.000000,0.000000}%
\pgfsetstrokecolor{currentstroke}%
\pgfsetdash{}{0pt}%
\pgfsys@defobject{currentmarker}{\pgfqpoint{0.000000in}{0.000000in}}{\pgfqpoint{0.027778in}{0.000000in}}{%
\pgfpathmoveto{\pgfqpoint{0.000000in}{0.000000in}}%
\pgfpathlineto{\pgfqpoint{0.027778in}{0.000000in}}%
\pgfusepath{stroke,fill}%
}%
\begin{pgfscope}%
\pgfsys@transformshift{5.801389in}{3.459610in}%
\pgfsys@useobject{currentmarker}{}%
\end{pgfscope}%
\end{pgfscope}%
\begin{pgfscope}%
\pgfpathrectangle{\pgfqpoint{0.781944in}{2.977778in}}{\pgfqpoint{5.019444in}{1.650000in}}%
\pgfusepath{clip}%
\pgfsetrectcap%
\pgfsetroundjoin%
\pgfsetlinewidth{0.803000pt}%
\definecolor{currentstroke}{rgb}{0.690196,0.690196,0.690196}%
\pgfsetstrokecolor{currentstroke}%
\pgfsetstrokeopacity{0.300000}%
\pgfsetdash{}{0pt}%
\pgfpathmoveto{\pgfqpoint{0.781944in}{3.481511in}}%
\pgfpathlineto{\pgfqpoint{5.801389in}{3.481511in}}%
\pgfusepath{stroke}%
\end{pgfscope}%
\begin{pgfscope}%
\pgfsetbuttcap%
\pgfsetroundjoin%
\definecolor{currentfill}{rgb}{0.000000,0.000000,0.000000}%
\pgfsetfillcolor{currentfill}%
\pgfsetlinewidth{0.602250pt}%
\definecolor{currentstroke}{rgb}{0.000000,0.000000,0.000000}%
\pgfsetstrokecolor{currentstroke}%
\pgfsetdash{}{0pt}%
\pgfsys@defobject{currentmarker}{\pgfqpoint{-0.027778in}{0.000000in}}{\pgfqpoint{0.000000in}{0.000000in}}{%
\pgfpathmoveto{\pgfqpoint{0.000000in}{0.000000in}}%
\pgfpathlineto{\pgfqpoint{-0.027778in}{0.000000in}}%
\pgfusepath{stroke,fill}%
}%
\begin{pgfscope}%
\pgfsys@transformshift{0.781944in}{3.481511in}%
\pgfsys@useobject{currentmarker}{}%
\end{pgfscope}%
\end{pgfscope}%
\begin{pgfscope}%
\pgfsetbuttcap%
\pgfsetroundjoin%
\definecolor{currentfill}{rgb}{0.000000,0.000000,0.000000}%
\pgfsetfillcolor{currentfill}%
\pgfsetlinewidth{0.602250pt}%
\definecolor{currentstroke}{rgb}{0.000000,0.000000,0.000000}%
\pgfsetstrokecolor{currentstroke}%
\pgfsetdash{}{0pt}%
\pgfsys@defobject{currentmarker}{\pgfqpoint{0.000000in}{0.000000in}}{\pgfqpoint{0.027778in}{0.000000in}}{%
\pgfpathmoveto{\pgfqpoint{0.000000in}{0.000000in}}%
\pgfpathlineto{\pgfqpoint{0.027778in}{0.000000in}}%
\pgfusepath{stroke,fill}%
}%
\begin{pgfscope}%
\pgfsys@transformshift{5.801389in}{3.481511in}%
\pgfsys@useobject{currentmarker}{}%
\end{pgfscope}%
\end{pgfscope}%
\begin{pgfscope}%
\pgfpathrectangle{\pgfqpoint{0.781944in}{2.977778in}}{\pgfqpoint{5.019444in}{1.650000in}}%
\pgfusepath{clip}%
\pgfsetrectcap%
\pgfsetroundjoin%
\pgfsetlinewidth{0.803000pt}%
\definecolor{currentstroke}{rgb}{0.690196,0.690196,0.690196}%
\pgfsetstrokecolor{currentstroke}%
\pgfsetstrokeopacity{0.300000}%
\pgfsetdash{}{0pt}%
\pgfpathmoveto{\pgfqpoint{0.781944in}{3.503412in}}%
\pgfpathlineto{\pgfqpoint{5.801389in}{3.503412in}}%
\pgfusepath{stroke}%
\end{pgfscope}%
\begin{pgfscope}%
\pgfsetbuttcap%
\pgfsetroundjoin%
\definecolor{currentfill}{rgb}{0.000000,0.000000,0.000000}%
\pgfsetfillcolor{currentfill}%
\pgfsetlinewidth{0.602250pt}%
\definecolor{currentstroke}{rgb}{0.000000,0.000000,0.000000}%
\pgfsetstrokecolor{currentstroke}%
\pgfsetdash{}{0pt}%
\pgfsys@defobject{currentmarker}{\pgfqpoint{-0.027778in}{0.000000in}}{\pgfqpoint{0.000000in}{0.000000in}}{%
\pgfpathmoveto{\pgfqpoint{0.000000in}{0.000000in}}%
\pgfpathlineto{\pgfqpoint{-0.027778in}{0.000000in}}%
\pgfusepath{stroke,fill}%
}%
\begin{pgfscope}%
\pgfsys@transformshift{0.781944in}{3.503412in}%
\pgfsys@useobject{currentmarker}{}%
\end{pgfscope}%
\end{pgfscope}%
\begin{pgfscope}%
\pgfsetbuttcap%
\pgfsetroundjoin%
\definecolor{currentfill}{rgb}{0.000000,0.000000,0.000000}%
\pgfsetfillcolor{currentfill}%
\pgfsetlinewidth{0.602250pt}%
\definecolor{currentstroke}{rgb}{0.000000,0.000000,0.000000}%
\pgfsetstrokecolor{currentstroke}%
\pgfsetdash{}{0pt}%
\pgfsys@defobject{currentmarker}{\pgfqpoint{0.000000in}{0.000000in}}{\pgfqpoint{0.027778in}{0.000000in}}{%
\pgfpathmoveto{\pgfqpoint{0.000000in}{0.000000in}}%
\pgfpathlineto{\pgfqpoint{0.027778in}{0.000000in}}%
\pgfusepath{stroke,fill}%
}%
\begin{pgfscope}%
\pgfsys@transformshift{5.801389in}{3.503412in}%
\pgfsys@useobject{currentmarker}{}%
\end{pgfscope}%
\end{pgfscope}%
\begin{pgfscope}%
\pgfpathrectangle{\pgfqpoint{0.781944in}{2.977778in}}{\pgfqpoint{5.019444in}{1.650000in}}%
\pgfusepath{clip}%
\pgfsetrectcap%
\pgfsetroundjoin%
\pgfsetlinewidth{0.803000pt}%
\definecolor{currentstroke}{rgb}{0.690196,0.690196,0.690196}%
\pgfsetstrokecolor{currentstroke}%
\pgfsetstrokeopacity{0.300000}%
\pgfsetdash{}{0pt}%
\pgfpathmoveto{\pgfqpoint{0.781944in}{3.525314in}}%
\pgfpathlineto{\pgfqpoint{5.801389in}{3.525314in}}%
\pgfusepath{stroke}%
\end{pgfscope}%
\begin{pgfscope}%
\pgfsetbuttcap%
\pgfsetroundjoin%
\definecolor{currentfill}{rgb}{0.000000,0.000000,0.000000}%
\pgfsetfillcolor{currentfill}%
\pgfsetlinewidth{0.602250pt}%
\definecolor{currentstroke}{rgb}{0.000000,0.000000,0.000000}%
\pgfsetstrokecolor{currentstroke}%
\pgfsetdash{}{0pt}%
\pgfsys@defobject{currentmarker}{\pgfqpoint{-0.027778in}{0.000000in}}{\pgfqpoint{0.000000in}{0.000000in}}{%
\pgfpathmoveto{\pgfqpoint{0.000000in}{0.000000in}}%
\pgfpathlineto{\pgfqpoint{-0.027778in}{0.000000in}}%
\pgfusepath{stroke,fill}%
}%
\begin{pgfscope}%
\pgfsys@transformshift{0.781944in}{3.525314in}%
\pgfsys@useobject{currentmarker}{}%
\end{pgfscope}%
\end{pgfscope}%
\begin{pgfscope}%
\pgfsetbuttcap%
\pgfsetroundjoin%
\definecolor{currentfill}{rgb}{0.000000,0.000000,0.000000}%
\pgfsetfillcolor{currentfill}%
\pgfsetlinewidth{0.602250pt}%
\definecolor{currentstroke}{rgb}{0.000000,0.000000,0.000000}%
\pgfsetstrokecolor{currentstroke}%
\pgfsetdash{}{0pt}%
\pgfsys@defobject{currentmarker}{\pgfqpoint{0.000000in}{0.000000in}}{\pgfqpoint{0.027778in}{0.000000in}}{%
\pgfpathmoveto{\pgfqpoint{0.000000in}{0.000000in}}%
\pgfpathlineto{\pgfqpoint{0.027778in}{0.000000in}}%
\pgfusepath{stroke,fill}%
}%
\begin{pgfscope}%
\pgfsys@transformshift{5.801389in}{3.525314in}%
\pgfsys@useobject{currentmarker}{}%
\end{pgfscope}%
\end{pgfscope}%
\begin{pgfscope}%
\pgfpathrectangle{\pgfqpoint{0.781944in}{2.977778in}}{\pgfqpoint{5.019444in}{1.650000in}}%
\pgfusepath{clip}%
\pgfsetrectcap%
\pgfsetroundjoin%
\pgfsetlinewidth{0.803000pt}%
\definecolor{currentstroke}{rgb}{0.690196,0.690196,0.690196}%
\pgfsetstrokecolor{currentstroke}%
\pgfsetstrokeopacity{0.300000}%
\pgfsetdash{}{0pt}%
\pgfpathmoveto{\pgfqpoint{0.781944in}{3.547215in}}%
\pgfpathlineto{\pgfqpoint{5.801389in}{3.547215in}}%
\pgfusepath{stroke}%
\end{pgfscope}%
\begin{pgfscope}%
\pgfsetbuttcap%
\pgfsetroundjoin%
\definecolor{currentfill}{rgb}{0.000000,0.000000,0.000000}%
\pgfsetfillcolor{currentfill}%
\pgfsetlinewidth{0.602250pt}%
\definecolor{currentstroke}{rgb}{0.000000,0.000000,0.000000}%
\pgfsetstrokecolor{currentstroke}%
\pgfsetdash{}{0pt}%
\pgfsys@defobject{currentmarker}{\pgfqpoint{-0.027778in}{0.000000in}}{\pgfqpoint{0.000000in}{0.000000in}}{%
\pgfpathmoveto{\pgfqpoint{0.000000in}{0.000000in}}%
\pgfpathlineto{\pgfqpoint{-0.027778in}{0.000000in}}%
\pgfusepath{stroke,fill}%
}%
\begin{pgfscope}%
\pgfsys@transformshift{0.781944in}{3.547215in}%
\pgfsys@useobject{currentmarker}{}%
\end{pgfscope}%
\end{pgfscope}%
\begin{pgfscope}%
\pgfsetbuttcap%
\pgfsetroundjoin%
\definecolor{currentfill}{rgb}{0.000000,0.000000,0.000000}%
\pgfsetfillcolor{currentfill}%
\pgfsetlinewidth{0.602250pt}%
\definecolor{currentstroke}{rgb}{0.000000,0.000000,0.000000}%
\pgfsetstrokecolor{currentstroke}%
\pgfsetdash{}{0pt}%
\pgfsys@defobject{currentmarker}{\pgfqpoint{0.000000in}{0.000000in}}{\pgfqpoint{0.027778in}{0.000000in}}{%
\pgfpathmoveto{\pgfqpoint{0.000000in}{0.000000in}}%
\pgfpathlineto{\pgfqpoint{0.027778in}{0.000000in}}%
\pgfusepath{stroke,fill}%
}%
\begin{pgfscope}%
\pgfsys@transformshift{5.801389in}{3.547215in}%
\pgfsys@useobject{currentmarker}{}%
\end{pgfscope}%
\end{pgfscope}%
\begin{pgfscope}%
\pgfpathrectangle{\pgfqpoint{0.781944in}{2.977778in}}{\pgfqpoint{5.019444in}{1.650000in}}%
\pgfusepath{clip}%
\pgfsetrectcap%
\pgfsetroundjoin%
\pgfsetlinewidth{0.803000pt}%
\definecolor{currentstroke}{rgb}{0.690196,0.690196,0.690196}%
\pgfsetstrokecolor{currentstroke}%
\pgfsetstrokeopacity{0.300000}%
\pgfsetdash{}{0pt}%
\pgfpathmoveto{\pgfqpoint{0.781944in}{3.569117in}}%
\pgfpathlineto{\pgfqpoint{5.801389in}{3.569117in}}%
\pgfusepath{stroke}%
\end{pgfscope}%
\begin{pgfscope}%
\pgfsetbuttcap%
\pgfsetroundjoin%
\definecolor{currentfill}{rgb}{0.000000,0.000000,0.000000}%
\pgfsetfillcolor{currentfill}%
\pgfsetlinewidth{0.602250pt}%
\definecolor{currentstroke}{rgb}{0.000000,0.000000,0.000000}%
\pgfsetstrokecolor{currentstroke}%
\pgfsetdash{}{0pt}%
\pgfsys@defobject{currentmarker}{\pgfqpoint{-0.027778in}{0.000000in}}{\pgfqpoint{0.000000in}{0.000000in}}{%
\pgfpathmoveto{\pgfqpoint{0.000000in}{0.000000in}}%
\pgfpathlineto{\pgfqpoint{-0.027778in}{0.000000in}}%
\pgfusepath{stroke,fill}%
}%
\begin{pgfscope}%
\pgfsys@transformshift{0.781944in}{3.569117in}%
\pgfsys@useobject{currentmarker}{}%
\end{pgfscope}%
\end{pgfscope}%
\begin{pgfscope}%
\pgfsetbuttcap%
\pgfsetroundjoin%
\definecolor{currentfill}{rgb}{0.000000,0.000000,0.000000}%
\pgfsetfillcolor{currentfill}%
\pgfsetlinewidth{0.602250pt}%
\definecolor{currentstroke}{rgb}{0.000000,0.000000,0.000000}%
\pgfsetstrokecolor{currentstroke}%
\pgfsetdash{}{0pt}%
\pgfsys@defobject{currentmarker}{\pgfqpoint{0.000000in}{0.000000in}}{\pgfqpoint{0.027778in}{0.000000in}}{%
\pgfpathmoveto{\pgfqpoint{0.000000in}{0.000000in}}%
\pgfpathlineto{\pgfqpoint{0.027778in}{0.000000in}}%
\pgfusepath{stroke,fill}%
}%
\begin{pgfscope}%
\pgfsys@transformshift{5.801389in}{3.569117in}%
\pgfsys@useobject{currentmarker}{}%
\end{pgfscope}%
\end{pgfscope}%
\begin{pgfscope}%
\pgfpathrectangle{\pgfqpoint{0.781944in}{2.977778in}}{\pgfqpoint{5.019444in}{1.650000in}}%
\pgfusepath{clip}%
\pgfsetrectcap%
\pgfsetroundjoin%
\pgfsetlinewidth{0.803000pt}%
\definecolor{currentstroke}{rgb}{0.690196,0.690196,0.690196}%
\pgfsetstrokecolor{currentstroke}%
\pgfsetstrokeopacity{0.300000}%
\pgfsetdash{}{0pt}%
\pgfpathmoveto{\pgfqpoint{0.781944in}{3.591018in}}%
\pgfpathlineto{\pgfqpoint{5.801389in}{3.591018in}}%
\pgfusepath{stroke}%
\end{pgfscope}%
\begin{pgfscope}%
\pgfsetbuttcap%
\pgfsetroundjoin%
\definecolor{currentfill}{rgb}{0.000000,0.000000,0.000000}%
\pgfsetfillcolor{currentfill}%
\pgfsetlinewidth{0.602250pt}%
\definecolor{currentstroke}{rgb}{0.000000,0.000000,0.000000}%
\pgfsetstrokecolor{currentstroke}%
\pgfsetdash{}{0pt}%
\pgfsys@defobject{currentmarker}{\pgfqpoint{-0.027778in}{0.000000in}}{\pgfqpoint{0.000000in}{0.000000in}}{%
\pgfpathmoveto{\pgfqpoint{0.000000in}{0.000000in}}%
\pgfpathlineto{\pgfqpoint{-0.027778in}{0.000000in}}%
\pgfusepath{stroke,fill}%
}%
\begin{pgfscope}%
\pgfsys@transformshift{0.781944in}{3.591018in}%
\pgfsys@useobject{currentmarker}{}%
\end{pgfscope}%
\end{pgfscope}%
\begin{pgfscope}%
\pgfsetbuttcap%
\pgfsetroundjoin%
\definecolor{currentfill}{rgb}{0.000000,0.000000,0.000000}%
\pgfsetfillcolor{currentfill}%
\pgfsetlinewidth{0.602250pt}%
\definecolor{currentstroke}{rgb}{0.000000,0.000000,0.000000}%
\pgfsetstrokecolor{currentstroke}%
\pgfsetdash{}{0pt}%
\pgfsys@defobject{currentmarker}{\pgfqpoint{0.000000in}{0.000000in}}{\pgfqpoint{0.027778in}{0.000000in}}{%
\pgfpathmoveto{\pgfqpoint{0.000000in}{0.000000in}}%
\pgfpathlineto{\pgfqpoint{0.027778in}{0.000000in}}%
\pgfusepath{stroke,fill}%
}%
\begin{pgfscope}%
\pgfsys@transformshift{5.801389in}{3.591018in}%
\pgfsys@useobject{currentmarker}{}%
\end{pgfscope}%
\end{pgfscope}%
\begin{pgfscope}%
\pgfpathrectangle{\pgfqpoint{0.781944in}{2.977778in}}{\pgfqpoint{5.019444in}{1.650000in}}%
\pgfusepath{clip}%
\pgfsetrectcap%
\pgfsetroundjoin%
\pgfsetlinewidth{0.803000pt}%
\definecolor{currentstroke}{rgb}{0.690196,0.690196,0.690196}%
\pgfsetstrokecolor{currentstroke}%
\pgfsetstrokeopacity{0.300000}%
\pgfsetdash{}{0pt}%
\pgfpathmoveto{\pgfqpoint{0.781944in}{3.612920in}}%
\pgfpathlineto{\pgfqpoint{5.801389in}{3.612920in}}%
\pgfusepath{stroke}%
\end{pgfscope}%
\begin{pgfscope}%
\pgfsetbuttcap%
\pgfsetroundjoin%
\definecolor{currentfill}{rgb}{0.000000,0.000000,0.000000}%
\pgfsetfillcolor{currentfill}%
\pgfsetlinewidth{0.602250pt}%
\definecolor{currentstroke}{rgb}{0.000000,0.000000,0.000000}%
\pgfsetstrokecolor{currentstroke}%
\pgfsetdash{}{0pt}%
\pgfsys@defobject{currentmarker}{\pgfqpoint{-0.027778in}{0.000000in}}{\pgfqpoint{0.000000in}{0.000000in}}{%
\pgfpathmoveto{\pgfqpoint{0.000000in}{0.000000in}}%
\pgfpathlineto{\pgfqpoint{-0.027778in}{0.000000in}}%
\pgfusepath{stroke,fill}%
}%
\begin{pgfscope}%
\pgfsys@transformshift{0.781944in}{3.612920in}%
\pgfsys@useobject{currentmarker}{}%
\end{pgfscope}%
\end{pgfscope}%
\begin{pgfscope}%
\pgfsetbuttcap%
\pgfsetroundjoin%
\definecolor{currentfill}{rgb}{0.000000,0.000000,0.000000}%
\pgfsetfillcolor{currentfill}%
\pgfsetlinewidth{0.602250pt}%
\definecolor{currentstroke}{rgb}{0.000000,0.000000,0.000000}%
\pgfsetstrokecolor{currentstroke}%
\pgfsetdash{}{0pt}%
\pgfsys@defobject{currentmarker}{\pgfqpoint{0.000000in}{0.000000in}}{\pgfqpoint{0.027778in}{0.000000in}}{%
\pgfpathmoveto{\pgfqpoint{0.000000in}{0.000000in}}%
\pgfpathlineto{\pgfqpoint{0.027778in}{0.000000in}}%
\pgfusepath{stroke,fill}%
}%
\begin{pgfscope}%
\pgfsys@transformshift{5.801389in}{3.612920in}%
\pgfsys@useobject{currentmarker}{}%
\end{pgfscope}%
\end{pgfscope}%
\begin{pgfscope}%
\pgfpathrectangle{\pgfqpoint{0.781944in}{2.977778in}}{\pgfqpoint{5.019444in}{1.650000in}}%
\pgfusepath{clip}%
\pgfsetrectcap%
\pgfsetroundjoin%
\pgfsetlinewidth{0.803000pt}%
\definecolor{currentstroke}{rgb}{0.690196,0.690196,0.690196}%
\pgfsetstrokecolor{currentstroke}%
\pgfsetstrokeopacity{0.300000}%
\pgfsetdash{}{0pt}%
\pgfpathmoveto{\pgfqpoint{0.781944in}{3.656723in}}%
\pgfpathlineto{\pgfqpoint{5.801389in}{3.656723in}}%
\pgfusepath{stroke}%
\end{pgfscope}%
\begin{pgfscope}%
\pgfsetbuttcap%
\pgfsetroundjoin%
\definecolor{currentfill}{rgb}{0.000000,0.000000,0.000000}%
\pgfsetfillcolor{currentfill}%
\pgfsetlinewidth{0.602250pt}%
\definecolor{currentstroke}{rgb}{0.000000,0.000000,0.000000}%
\pgfsetstrokecolor{currentstroke}%
\pgfsetdash{}{0pt}%
\pgfsys@defobject{currentmarker}{\pgfqpoint{-0.027778in}{0.000000in}}{\pgfqpoint{0.000000in}{0.000000in}}{%
\pgfpathmoveto{\pgfqpoint{0.000000in}{0.000000in}}%
\pgfpathlineto{\pgfqpoint{-0.027778in}{0.000000in}}%
\pgfusepath{stroke,fill}%
}%
\begin{pgfscope}%
\pgfsys@transformshift{0.781944in}{3.656723in}%
\pgfsys@useobject{currentmarker}{}%
\end{pgfscope}%
\end{pgfscope}%
\begin{pgfscope}%
\pgfsetbuttcap%
\pgfsetroundjoin%
\definecolor{currentfill}{rgb}{0.000000,0.000000,0.000000}%
\pgfsetfillcolor{currentfill}%
\pgfsetlinewidth{0.602250pt}%
\definecolor{currentstroke}{rgb}{0.000000,0.000000,0.000000}%
\pgfsetstrokecolor{currentstroke}%
\pgfsetdash{}{0pt}%
\pgfsys@defobject{currentmarker}{\pgfqpoint{0.000000in}{0.000000in}}{\pgfqpoint{0.027778in}{0.000000in}}{%
\pgfpathmoveto{\pgfqpoint{0.000000in}{0.000000in}}%
\pgfpathlineto{\pgfqpoint{0.027778in}{0.000000in}}%
\pgfusepath{stroke,fill}%
}%
\begin{pgfscope}%
\pgfsys@transformshift{5.801389in}{3.656723in}%
\pgfsys@useobject{currentmarker}{}%
\end{pgfscope}%
\end{pgfscope}%
\begin{pgfscope}%
\pgfpathrectangle{\pgfqpoint{0.781944in}{2.977778in}}{\pgfqpoint{5.019444in}{1.650000in}}%
\pgfusepath{clip}%
\pgfsetrectcap%
\pgfsetroundjoin%
\pgfsetlinewidth{0.803000pt}%
\definecolor{currentstroke}{rgb}{0.690196,0.690196,0.690196}%
\pgfsetstrokecolor{currentstroke}%
\pgfsetstrokeopacity{0.300000}%
\pgfsetdash{}{0pt}%
\pgfpathmoveto{\pgfqpoint{0.781944in}{3.678624in}}%
\pgfpathlineto{\pgfqpoint{5.801389in}{3.678624in}}%
\pgfusepath{stroke}%
\end{pgfscope}%
\begin{pgfscope}%
\pgfsetbuttcap%
\pgfsetroundjoin%
\definecolor{currentfill}{rgb}{0.000000,0.000000,0.000000}%
\pgfsetfillcolor{currentfill}%
\pgfsetlinewidth{0.602250pt}%
\definecolor{currentstroke}{rgb}{0.000000,0.000000,0.000000}%
\pgfsetstrokecolor{currentstroke}%
\pgfsetdash{}{0pt}%
\pgfsys@defobject{currentmarker}{\pgfqpoint{-0.027778in}{0.000000in}}{\pgfqpoint{0.000000in}{0.000000in}}{%
\pgfpathmoveto{\pgfqpoint{0.000000in}{0.000000in}}%
\pgfpathlineto{\pgfqpoint{-0.027778in}{0.000000in}}%
\pgfusepath{stroke,fill}%
}%
\begin{pgfscope}%
\pgfsys@transformshift{0.781944in}{3.678624in}%
\pgfsys@useobject{currentmarker}{}%
\end{pgfscope}%
\end{pgfscope}%
\begin{pgfscope}%
\pgfsetbuttcap%
\pgfsetroundjoin%
\definecolor{currentfill}{rgb}{0.000000,0.000000,0.000000}%
\pgfsetfillcolor{currentfill}%
\pgfsetlinewidth{0.602250pt}%
\definecolor{currentstroke}{rgb}{0.000000,0.000000,0.000000}%
\pgfsetstrokecolor{currentstroke}%
\pgfsetdash{}{0pt}%
\pgfsys@defobject{currentmarker}{\pgfqpoint{0.000000in}{0.000000in}}{\pgfqpoint{0.027778in}{0.000000in}}{%
\pgfpathmoveto{\pgfqpoint{0.000000in}{0.000000in}}%
\pgfpathlineto{\pgfqpoint{0.027778in}{0.000000in}}%
\pgfusepath{stroke,fill}%
}%
\begin{pgfscope}%
\pgfsys@transformshift{5.801389in}{3.678624in}%
\pgfsys@useobject{currentmarker}{}%
\end{pgfscope}%
\end{pgfscope}%
\begin{pgfscope}%
\pgfpathrectangle{\pgfqpoint{0.781944in}{2.977778in}}{\pgfqpoint{5.019444in}{1.650000in}}%
\pgfusepath{clip}%
\pgfsetrectcap%
\pgfsetroundjoin%
\pgfsetlinewidth{0.803000pt}%
\definecolor{currentstroke}{rgb}{0.690196,0.690196,0.690196}%
\pgfsetstrokecolor{currentstroke}%
\pgfsetstrokeopacity{0.300000}%
\pgfsetdash{}{0pt}%
\pgfpathmoveto{\pgfqpoint{0.781944in}{3.700525in}}%
\pgfpathlineto{\pgfqpoint{5.801389in}{3.700525in}}%
\pgfusepath{stroke}%
\end{pgfscope}%
\begin{pgfscope}%
\pgfsetbuttcap%
\pgfsetroundjoin%
\definecolor{currentfill}{rgb}{0.000000,0.000000,0.000000}%
\pgfsetfillcolor{currentfill}%
\pgfsetlinewidth{0.602250pt}%
\definecolor{currentstroke}{rgb}{0.000000,0.000000,0.000000}%
\pgfsetstrokecolor{currentstroke}%
\pgfsetdash{}{0pt}%
\pgfsys@defobject{currentmarker}{\pgfqpoint{-0.027778in}{0.000000in}}{\pgfqpoint{0.000000in}{0.000000in}}{%
\pgfpathmoveto{\pgfqpoint{0.000000in}{0.000000in}}%
\pgfpathlineto{\pgfqpoint{-0.027778in}{0.000000in}}%
\pgfusepath{stroke,fill}%
}%
\begin{pgfscope}%
\pgfsys@transformshift{0.781944in}{3.700525in}%
\pgfsys@useobject{currentmarker}{}%
\end{pgfscope}%
\end{pgfscope}%
\begin{pgfscope}%
\pgfsetbuttcap%
\pgfsetroundjoin%
\definecolor{currentfill}{rgb}{0.000000,0.000000,0.000000}%
\pgfsetfillcolor{currentfill}%
\pgfsetlinewidth{0.602250pt}%
\definecolor{currentstroke}{rgb}{0.000000,0.000000,0.000000}%
\pgfsetstrokecolor{currentstroke}%
\pgfsetdash{}{0pt}%
\pgfsys@defobject{currentmarker}{\pgfqpoint{0.000000in}{0.000000in}}{\pgfqpoint{0.027778in}{0.000000in}}{%
\pgfpathmoveto{\pgfqpoint{0.000000in}{0.000000in}}%
\pgfpathlineto{\pgfqpoint{0.027778in}{0.000000in}}%
\pgfusepath{stroke,fill}%
}%
\begin{pgfscope}%
\pgfsys@transformshift{5.801389in}{3.700525in}%
\pgfsys@useobject{currentmarker}{}%
\end{pgfscope}%
\end{pgfscope}%
\begin{pgfscope}%
\pgfpathrectangle{\pgfqpoint{0.781944in}{2.977778in}}{\pgfqpoint{5.019444in}{1.650000in}}%
\pgfusepath{clip}%
\pgfsetrectcap%
\pgfsetroundjoin%
\pgfsetlinewidth{0.803000pt}%
\definecolor{currentstroke}{rgb}{0.690196,0.690196,0.690196}%
\pgfsetstrokecolor{currentstroke}%
\pgfsetstrokeopacity{0.300000}%
\pgfsetdash{}{0pt}%
\pgfpathmoveto{\pgfqpoint{0.781944in}{3.722427in}}%
\pgfpathlineto{\pgfqpoint{5.801389in}{3.722427in}}%
\pgfusepath{stroke}%
\end{pgfscope}%
\begin{pgfscope}%
\pgfsetbuttcap%
\pgfsetroundjoin%
\definecolor{currentfill}{rgb}{0.000000,0.000000,0.000000}%
\pgfsetfillcolor{currentfill}%
\pgfsetlinewidth{0.602250pt}%
\definecolor{currentstroke}{rgb}{0.000000,0.000000,0.000000}%
\pgfsetstrokecolor{currentstroke}%
\pgfsetdash{}{0pt}%
\pgfsys@defobject{currentmarker}{\pgfqpoint{-0.027778in}{0.000000in}}{\pgfqpoint{0.000000in}{0.000000in}}{%
\pgfpathmoveto{\pgfqpoint{0.000000in}{0.000000in}}%
\pgfpathlineto{\pgfqpoint{-0.027778in}{0.000000in}}%
\pgfusepath{stroke,fill}%
}%
\begin{pgfscope}%
\pgfsys@transformshift{0.781944in}{3.722427in}%
\pgfsys@useobject{currentmarker}{}%
\end{pgfscope}%
\end{pgfscope}%
\begin{pgfscope}%
\pgfsetbuttcap%
\pgfsetroundjoin%
\definecolor{currentfill}{rgb}{0.000000,0.000000,0.000000}%
\pgfsetfillcolor{currentfill}%
\pgfsetlinewidth{0.602250pt}%
\definecolor{currentstroke}{rgb}{0.000000,0.000000,0.000000}%
\pgfsetstrokecolor{currentstroke}%
\pgfsetdash{}{0pt}%
\pgfsys@defobject{currentmarker}{\pgfqpoint{0.000000in}{0.000000in}}{\pgfqpoint{0.027778in}{0.000000in}}{%
\pgfpathmoveto{\pgfqpoint{0.000000in}{0.000000in}}%
\pgfpathlineto{\pgfqpoint{0.027778in}{0.000000in}}%
\pgfusepath{stroke,fill}%
}%
\begin{pgfscope}%
\pgfsys@transformshift{5.801389in}{3.722427in}%
\pgfsys@useobject{currentmarker}{}%
\end{pgfscope}%
\end{pgfscope}%
\begin{pgfscope}%
\pgfpathrectangle{\pgfqpoint{0.781944in}{2.977778in}}{\pgfqpoint{5.019444in}{1.650000in}}%
\pgfusepath{clip}%
\pgfsetrectcap%
\pgfsetroundjoin%
\pgfsetlinewidth{0.803000pt}%
\definecolor{currentstroke}{rgb}{0.690196,0.690196,0.690196}%
\pgfsetstrokecolor{currentstroke}%
\pgfsetstrokeopacity{0.300000}%
\pgfsetdash{}{0pt}%
\pgfpathmoveto{\pgfqpoint{0.781944in}{3.744328in}}%
\pgfpathlineto{\pgfqpoint{5.801389in}{3.744328in}}%
\pgfusepath{stroke}%
\end{pgfscope}%
\begin{pgfscope}%
\pgfsetbuttcap%
\pgfsetroundjoin%
\definecolor{currentfill}{rgb}{0.000000,0.000000,0.000000}%
\pgfsetfillcolor{currentfill}%
\pgfsetlinewidth{0.602250pt}%
\definecolor{currentstroke}{rgb}{0.000000,0.000000,0.000000}%
\pgfsetstrokecolor{currentstroke}%
\pgfsetdash{}{0pt}%
\pgfsys@defobject{currentmarker}{\pgfqpoint{-0.027778in}{0.000000in}}{\pgfqpoint{0.000000in}{0.000000in}}{%
\pgfpathmoveto{\pgfqpoint{0.000000in}{0.000000in}}%
\pgfpathlineto{\pgfqpoint{-0.027778in}{0.000000in}}%
\pgfusepath{stroke,fill}%
}%
\begin{pgfscope}%
\pgfsys@transformshift{0.781944in}{3.744328in}%
\pgfsys@useobject{currentmarker}{}%
\end{pgfscope}%
\end{pgfscope}%
\begin{pgfscope}%
\pgfsetbuttcap%
\pgfsetroundjoin%
\definecolor{currentfill}{rgb}{0.000000,0.000000,0.000000}%
\pgfsetfillcolor{currentfill}%
\pgfsetlinewidth{0.602250pt}%
\definecolor{currentstroke}{rgb}{0.000000,0.000000,0.000000}%
\pgfsetstrokecolor{currentstroke}%
\pgfsetdash{}{0pt}%
\pgfsys@defobject{currentmarker}{\pgfqpoint{0.000000in}{0.000000in}}{\pgfqpoint{0.027778in}{0.000000in}}{%
\pgfpathmoveto{\pgfqpoint{0.000000in}{0.000000in}}%
\pgfpathlineto{\pgfqpoint{0.027778in}{0.000000in}}%
\pgfusepath{stroke,fill}%
}%
\begin{pgfscope}%
\pgfsys@transformshift{5.801389in}{3.744328in}%
\pgfsys@useobject{currentmarker}{}%
\end{pgfscope}%
\end{pgfscope}%
\begin{pgfscope}%
\pgfpathrectangle{\pgfqpoint{0.781944in}{2.977778in}}{\pgfqpoint{5.019444in}{1.650000in}}%
\pgfusepath{clip}%
\pgfsetrectcap%
\pgfsetroundjoin%
\pgfsetlinewidth{0.803000pt}%
\definecolor{currentstroke}{rgb}{0.690196,0.690196,0.690196}%
\pgfsetstrokecolor{currentstroke}%
\pgfsetstrokeopacity{0.300000}%
\pgfsetdash{}{0pt}%
\pgfpathmoveto{\pgfqpoint{0.781944in}{3.766230in}}%
\pgfpathlineto{\pgfqpoint{5.801389in}{3.766230in}}%
\pgfusepath{stroke}%
\end{pgfscope}%
\begin{pgfscope}%
\pgfsetbuttcap%
\pgfsetroundjoin%
\definecolor{currentfill}{rgb}{0.000000,0.000000,0.000000}%
\pgfsetfillcolor{currentfill}%
\pgfsetlinewidth{0.602250pt}%
\definecolor{currentstroke}{rgb}{0.000000,0.000000,0.000000}%
\pgfsetstrokecolor{currentstroke}%
\pgfsetdash{}{0pt}%
\pgfsys@defobject{currentmarker}{\pgfqpoint{-0.027778in}{0.000000in}}{\pgfqpoint{0.000000in}{0.000000in}}{%
\pgfpathmoveto{\pgfqpoint{0.000000in}{0.000000in}}%
\pgfpathlineto{\pgfqpoint{-0.027778in}{0.000000in}}%
\pgfusepath{stroke,fill}%
}%
\begin{pgfscope}%
\pgfsys@transformshift{0.781944in}{3.766230in}%
\pgfsys@useobject{currentmarker}{}%
\end{pgfscope}%
\end{pgfscope}%
\begin{pgfscope}%
\pgfsetbuttcap%
\pgfsetroundjoin%
\definecolor{currentfill}{rgb}{0.000000,0.000000,0.000000}%
\pgfsetfillcolor{currentfill}%
\pgfsetlinewidth{0.602250pt}%
\definecolor{currentstroke}{rgb}{0.000000,0.000000,0.000000}%
\pgfsetstrokecolor{currentstroke}%
\pgfsetdash{}{0pt}%
\pgfsys@defobject{currentmarker}{\pgfqpoint{0.000000in}{0.000000in}}{\pgfqpoint{0.027778in}{0.000000in}}{%
\pgfpathmoveto{\pgfqpoint{0.000000in}{0.000000in}}%
\pgfpathlineto{\pgfqpoint{0.027778in}{0.000000in}}%
\pgfusepath{stroke,fill}%
}%
\begin{pgfscope}%
\pgfsys@transformshift{5.801389in}{3.766230in}%
\pgfsys@useobject{currentmarker}{}%
\end{pgfscope}%
\end{pgfscope}%
\begin{pgfscope}%
\pgfpathrectangle{\pgfqpoint{0.781944in}{2.977778in}}{\pgfqpoint{5.019444in}{1.650000in}}%
\pgfusepath{clip}%
\pgfsetrectcap%
\pgfsetroundjoin%
\pgfsetlinewidth{0.803000pt}%
\definecolor{currentstroke}{rgb}{0.690196,0.690196,0.690196}%
\pgfsetstrokecolor{currentstroke}%
\pgfsetstrokeopacity{0.300000}%
\pgfsetdash{}{0pt}%
\pgfpathmoveto{\pgfqpoint{0.781944in}{3.788131in}}%
\pgfpathlineto{\pgfqpoint{5.801389in}{3.788131in}}%
\pgfusepath{stroke}%
\end{pgfscope}%
\begin{pgfscope}%
\pgfsetbuttcap%
\pgfsetroundjoin%
\definecolor{currentfill}{rgb}{0.000000,0.000000,0.000000}%
\pgfsetfillcolor{currentfill}%
\pgfsetlinewidth{0.602250pt}%
\definecolor{currentstroke}{rgb}{0.000000,0.000000,0.000000}%
\pgfsetstrokecolor{currentstroke}%
\pgfsetdash{}{0pt}%
\pgfsys@defobject{currentmarker}{\pgfqpoint{-0.027778in}{0.000000in}}{\pgfqpoint{0.000000in}{0.000000in}}{%
\pgfpathmoveto{\pgfqpoint{0.000000in}{0.000000in}}%
\pgfpathlineto{\pgfqpoint{-0.027778in}{0.000000in}}%
\pgfusepath{stroke,fill}%
}%
\begin{pgfscope}%
\pgfsys@transformshift{0.781944in}{3.788131in}%
\pgfsys@useobject{currentmarker}{}%
\end{pgfscope}%
\end{pgfscope}%
\begin{pgfscope}%
\pgfsetbuttcap%
\pgfsetroundjoin%
\definecolor{currentfill}{rgb}{0.000000,0.000000,0.000000}%
\pgfsetfillcolor{currentfill}%
\pgfsetlinewidth{0.602250pt}%
\definecolor{currentstroke}{rgb}{0.000000,0.000000,0.000000}%
\pgfsetstrokecolor{currentstroke}%
\pgfsetdash{}{0pt}%
\pgfsys@defobject{currentmarker}{\pgfqpoint{0.000000in}{0.000000in}}{\pgfqpoint{0.027778in}{0.000000in}}{%
\pgfpathmoveto{\pgfqpoint{0.000000in}{0.000000in}}%
\pgfpathlineto{\pgfqpoint{0.027778in}{0.000000in}}%
\pgfusepath{stroke,fill}%
}%
\begin{pgfscope}%
\pgfsys@transformshift{5.801389in}{3.788131in}%
\pgfsys@useobject{currentmarker}{}%
\end{pgfscope}%
\end{pgfscope}%
\begin{pgfscope}%
\pgfpathrectangle{\pgfqpoint{0.781944in}{2.977778in}}{\pgfqpoint{5.019444in}{1.650000in}}%
\pgfusepath{clip}%
\pgfsetrectcap%
\pgfsetroundjoin%
\pgfsetlinewidth{0.803000pt}%
\definecolor{currentstroke}{rgb}{0.690196,0.690196,0.690196}%
\pgfsetstrokecolor{currentstroke}%
\pgfsetstrokeopacity{0.300000}%
\pgfsetdash{}{0pt}%
\pgfpathmoveto{\pgfqpoint{0.781944in}{3.810033in}}%
\pgfpathlineto{\pgfqpoint{5.801389in}{3.810033in}}%
\pgfusepath{stroke}%
\end{pgfscope}%
\begin{pgfscope}%
\pgfsetbuttcap%
\pgfsetroundjoin%
\definecolor{currentfill}{rgb}{0.000000,0.000000,0.000000}%
\pgfsetfillcolor{currentfill}%
\pgfsetlinewidth{0.602250pt}%
\definecolor{currentstroke}{rgb}{0.000000,0.000000,0.000000}%
\pgfsetstrokecolor{currentstroke}%
\pgfsetdash{}{0pt}%
\pgfsys@defobject{currentmarker}{\pgfqpoint{-0.027778in}{0.000000in}}{\pgfqpoint{0.000000in}{0.000000in}}{%
\pgfpathmoveto{\pgfqpoint{0.000000in}{0.000000in}}%
\pgfpathlineto{\pgfqpoint{-0.027778in}{0.000000in}}%
\pgfusepath{stroke,fill}%
}%
\begin{pgfscope}%
\pgfsys@transformshift{0.781944in}{3.810033in}%
\pgfsys@useobject{currentmarker}{}%
\end{pgfscope}%
\end{pgfscope}%
\begin{pgfscope}%
\pgfsetbuttcap%
\pgfsetroundjoin%
\definecolor{currentfill}{rgb}{0.000000,0.000000,0.000000}%
\pgfsetfillcolor{currentfill}%
\pgfsetlinewidth{0.602250pt}%
\definecolor{currentstroke}{rgb}{0.000000,0.000000,0.000000}%
\pgfsetstrokecolor{currentstroke}%
\pgfsetdash{}{0pt}%
\pgfsys@defobject{currentmarker}{\pgfqpoint{0.000000in}{0.000000in}}{\pgfqpoint{0.027778in}{0.000000in}}{%
\pgfpathmoveto{\pgfqpoint{0.000000in}{0.000000in}}%
\pgfpathlineto{\pgfqpoint{0.027778in}{0.000000in}}%
\pgfusepath{stroke,fill}%
}%
\begin{pgfscope}%
\pgfsys@transformshift{5.801389in}{3.810033in}%
\pgfsys@useobject{currentmarker}{}%
\end{pgfscope}%
\end{pgfscope}%
\begin{pgfscope}%
\pgfpathrectangle{\pgfqpoint{0.781944in}{2.977778in}}{\pgfqpoint{5.019444in}{1.650000in}}%
\pgfusepath{clip}%
\pgfsetrectcap%
\pgfsetroundjoin%
\pgfsetlinewidth{0.803000pt}%
\definecolor{currentstroke}{rgb}{0.690196,0.690196,0.690196}%
\pgfsetstrokecolor{currentstroke}%
\pgfsetstrokeopacity{0.300000}%
\pgfsetdash{}{0pt}%
\pgfpathmoveto{\pgfqpoint{0.781944in}{3.831934in}}%
\pgfpathlineto{\pgfqpoint{5.801389in}{3.831934in}}%
\pgfusepath{stroke}%
\end{pgfscope}%
\begin{pgfscope}%
\pgfsetbuttcap%
\pgfsetroundjoin%
\definecolor{currentfill}{rgb}{0.000000,0.000000,0.000000}%
\pgfsetfillcolor{currentfill}%
\pgfsetlinewidth{0.602250pt}%
\definecolor{currentstroke}{rgb}{0.000000,0.000000,0.000000}%
\pgfsetstrokecolor{currentstroke}%
\pgfsetdash{}{0pt}%
\pgfsys@defobject{currentmarker}{\pgfqpoint{-0.027778in}{0.000000in}}{\pgfqpoint{0.000000in}{0.000000in}}{%
\pgfpathmoveto{\pgfqpoint{0.000000in}{0.000000in}}%
\pgfpathlineto{\pgfqpoint{-0.027778in}{0.000000in}}%
\pgfusepath{stroke,fill}%
}%
\begin{pgfscope}%
\pgfsys@transformshift{0.781944in}{3.831934in}%
\pgfsys@useobject{currentmarker}{}%
\end{pgfscope}%
\end{pgfscope}%
\begin{pgfscope}%
\pgfsetbuttcap%
\pgfsetroundjoin%
\definecolor{currentfill}{rgb}{0.000000,0.000000,0.000000}%
\pgfsetfillcolor{currentfill}%
\pgfsetlinewidth{0.602250pt}%
\definecolor{currentstroke}{rgb}{0.000000,0.000000,0.000000}%
\pgfsetstrokecolor{currentstroke}%
\pgfsetdash{}{0pt}%
\pgfsys@defobject{currentmarker}{\pgfqpoint{0.000000in}{0.000000in}}{\pgfqpoint{0.027778in}{0.000000in}}{%
\pgfpathmoveto{\pgfqpoint{0.000000in}{0.000000in}}%
\pgfpathlineto{\pgfqpoint{0.027778in}{0.000000in}}%
\pgfusepath{stroke,fill}%
}%
\begin{pgfscope}%
\pgfsys@transformshift{5.801389in}{3.831934in}%
\pgfsys@useobject{currentmarker}{}%
\end{pgfscope}%
\end{pgfscope}%
\begin{pgfscope}%
\pgfpathrectangle{\pgfqpoint{0.781944in}{2.977778in}}{\pgfqpoint{5.019444in}{1.650000in}}%
\pgfusepath{clip}%
\pgfsetrectcap%
\pgfsetroundjoin%
\pgfsetlinewidth{0.803000pt}%
\definecolor{currentstroke}{rgb}{0.690196,0.690196,0.690196}%
\pgfsetstrokecolor{currentstroke}%
\pgfsetstrokeopacity{0.300000}%
\pgfsetdash{}{0pt}%
\pgfpathmoveto{\pgfqpoint{0.781944in}{3.875737in}}%
\pgfpathlineto{\pgfqpoint{5.801389in}{3.875737in}}%
\pgfusepath{stroke}%
\end{pgfscope}%
\begin{pgfscope}%
\pgfsetbuttcap%
\pgfsetroundjoin%
\definecolor{currentfill}{rgb}{0.000000,0.000000,0.000000}%
\pgfsetfillcolor{currentfill}%
\pgfsetlinewidth{0.602250pt}%
\definecolor{currentstroke}{rgb}{0.000000,0.000000,0.000000}%
\pgfsetstrokecolor{currentstroke}%
\pgfsetdash{}{0pt}%
\pgfsys@defobject{currentmarker}{\pgfqpoint{-0.027778in}{0.000000in}}{\pgfqpoint{0.000000in}{0.000000in}}{%
\pgfpathmoveto{\pgfqpoint{0.000000in}{0.000000in}}%
\pgfpathlineto{\pgfqpoint{-0.027778in}{0.000000in}}%
\pgfusepath{stroke,fill}%
}%
\begin{pgfscope}%
\pgfsys@transformshift{0.781944in}{3.875737in}%
\pgfsys@useobject{currentmarker}{}%
\end{pgfscope}%
\end{pgfscope}%
\begin{pgfscope}%
\pgfsetbuttcap%
\pgfsetroundjoin%
\definecolor{currentfill}{rgb}{0.000000,0.000000,0.000000}%
\pgfsetfillcolor{currentfill}%
\pgfsetlinewidth{0.602250pt}%
\definecolor{currentstroke}{rgb}{0.000000,0.000000,0.000000}%
\pgfsetstrokecolor{currentstroke}%
\pgfsetdash{}{0pt}%
\pgfsys@defobject{currentmarker}{\pgfqpoint{0.000000in}{0.000000in}}{\pgfqpoint{0.027778in}{0.000000in}}{%
\pgfpathmoveto{\pgfqpoint{0.000000in}{0.000000in}}%
\pgfpathlineto{\pgfqpoint{0.027778in}{0.000000in}}%
\pgfusepath{stroke,fill}%
}%
\begin{pgfscope}%
\pgfsys@transformshift{5.801389in}{3.875737in}%
\pgfsys@useobject{currentmarker}{}%
\end{pgfscope}%
\end{pgfscope}%
\begin{pgfscope}%
\pgfpathrectangle{\pgfqpoint{0.781944in}{2.977778in}}{\pgfqpoint{5.019444in}{1.650000in}}%
\pgfusepath{clip}%
\pgfsetrectcap%
\pgfsetroundjoin%
\pgfsetlinewidth{0.803000pt}%
\definecolor{currentstroke}{rgb}{0.690196,0.690196,0.690196}%
\pgfsetstrokecolor{currentstroke}%
\pgfsetstrokeopacity{0.300000}%
\pgfsetdash{}{0pt}%
\pgfpathmoveto{\pgfqpoint{0.781944in}{3.897638in}}%
\pgfpathlineto{\pgfqpoint{5.801389in}{3.897638in}}%
\pgfusepath{stroke}%
\end{pgfscope}%
\begin{pgfscope}%
\pgfsetbuttcap%
\pgfsetroundjoin%
\definecolor{currentfill}{rgb}{0.000000,0.000000,0.000000}%
\pgfsetfillcolor{currentfill}%
\pgfsetlinewidth{0.602250pt}%
\definecolor{currentstroke}{rgb}{0.000000,0.000000,0.000000}%
\pgfsetstrokecolor{currentstroke}%
\pgfsetdash{}{0pt}%
\pgfsys@defobject{currentmarker}{\pgfqpoint{-0.027778in}{0.000000in}}{\pgfqpoint{0.000000in}{0.000000in}}{%
\pgfpathmoveto{\pgfqpoint{0.000000in}{0.000000in}}%
\pgfpathlineto{\pgfqpoint{-0.027778in}{0.000000in}}%
\pgfusepath{stroke,fill}%
}%
\begin{pgfscope}%
\pgfsys@transformshift{0.781944in}{3.897638in}%
\pgfsys@useobject{currentmarker}{}%
\end{pgfscope}%
\end{pgfscope}%
\begin{pgfscope}%
\pgfsetbuttcap%
\pgfsetroundjoin%
\definecolor{currentfill}{rgb}{0.000000,0.000000,0.000000}%
\pgfsetfillcolor{currentfill}%
\pgfsetlinewidth{0.602250pt}%
\definecolor{currentstroke}{rgb}{0.000000,0.000000,0.000000}%
\pgfsetstrokecolor{currentstroke}%
\pgfsetdash{}{0pt}%
\pgfsys@defobject{currentmarker}{\pgfqpoint{0.000000in}{0.000000in}}{\pgfqpoint{0.027778in}{0.000000in}}{%
\pgfpathmoveto{\pgfqpoint{0.000000in}{0.000000in}}%
\pgfpathlineto{\pgfqpoint{0.027778in}{0.000000in}}%
\pgfusepath{stroke,fill}%
}%
\begin{pgfscope}%
\pgfsys@transformshift{5.801389in}{3.897638in}%
\pgfsys@useobject{currentmarker}{}%
\end{pgfscope}%
\end{pgfscope}%
\begin{pgfscope}%
\pgfpathrectangle{\pgfqpoint{0.781944in}{2.977778in}}{\pgfqpoint{5.019444in}{1.650000in}}%
\pgfusepath{clip}%
\pgfsetrectcap%
\pgfsetroundjoin%
\pgfsetlinewidth{0.803000pt}%
\definecolor{currentstroke}{rgb}{0.690196,0.690196,0.690196}%
\pgfsetstrokecolor{currentstroke}%
\pgfsetstrokeopacity{0.300000}%
\pgfsetdash{}{0pt}%
\pgfpathmoveto{\pgfqpoint{0.781944in}{3.919540in}}%
\pgfpathlineto{\pgfqpoint{5.801389in}{3.919540in}}%
\pgfusepath{stroke}%
\end{pgfscope}%
\begin{pgfscope}%
\pgfsetbuttcap%
\pgfsetroundjoin%
\definecolor{currentfill}{rgb}{0.000000,0.000000,0.000000}%
\pgfsetfillcolor{currentfill}%
\pgfsetlinewidth{0.602250pt}%
\definecolor{currentstroke}{rgb}{0.000000,0.000000,0.000000}%
\pgfsetstrokecolor{currentstroke}%
\pgfsetdash{}{0pt}%
\pgfsys@defobject{currentmarker}{\pgfqpoint{-0.027778in}{0.000000in}}{\pgfqpoint{0.000000in}{0.000000in}}{%
\pgfpathmoveto{\pgfqpoint{0.000000in}{0.000000in}}%
\pgfpathlineto{\pgfqpoint{-0.027778in}{0.000000in}}%
\pgfusepath{stroke,fill}%
}%
\begin{pgfscope}%
\pgfsys@transformshift{0.781944in}{3.919540in}%
\pgfsys@useobject{currentmarker}{}%
\end{pgfscope}%
\end{pgfscope}%
\begin{pgfscope}%
\pgfsetbuttcap%
\pgfsetroundjoin%
\definecolor{currentfill}{rgb}{0.000000,0.000000,0.000000}%
\pgfsetfillcolor{currentfill}%
\pgfsetlinewidth{0.602250pt}%
\definecolor{currentstroke}{rgb}{0.000000,0.000000,0.000000}%
\pgfsetstrokecolor{currentstroke}%
\pgfsetdash{}{0pt}%
\pgfsys@defobject{currentmarker}{\pgfqpoint{0.000000in}{0.000000in}}{\pgfqpoint{0.027778in}{0.000000in}}{%
\pgfpathmoveto{\pgfqpoint{0.000000in}{0.000000in}}%
\pgfpathlineto{\pgfqpoint{0.027778in}{0.000000in}}%
\pgfusepath{stroke,fill}%
}%
\begin{pgfscope}%
\pgfsys@transformshift{5.801389in}{3.919540in}%
\pgfsys@useobject{currentmarker}{}%
\end{pgfscope}%
\end{pgfscope}%
\begin{pgfscope}%
\pgfpathrectangle{\pgfqpoint{0.781944in}{2.977778in}}{\pgfqpoint{5.019444in}{1.650000in}}%
\pgfusepath{clip}%
\pgfsetrectcap%
\pgfsetroundjoin%
\pgfsetlinewidth{0.803000pt}%
\definecolor{currentstroke}{rgb}{0.690196,0.690196,0.690196}%
\pgfsetstrokecolor{currentstroke}%
\pgfsetstrokeopacity{0.300000}%
\pgfsetdash{}{0pt}%
\pgfpathmoveto{\pgfqpoint{0.781944in}{3.941441in}}%
\pgfpathlineto{\pgfqpoint{5.801389in}{3.941441in}}%
\pgfusepath{stroke}%
\end{pgfscope}%
\begin{pgfscope}%
\pgfsetbuttcap%
\pgfsetroundjoin%
\definecolor{currentfill}{rgb}{0.000000,0.000000,0.000000}%
\pgfsetfillcolor{currentfill}%
\pgfsetlinewidth{0.602250pt}%
\definecolor{currentstroke}{rgb}{0.000000,0.000000,0.000000}%
\pgfsetstrokecolor{currentstroke}%
\pgfsetdash{}{0pt}%
\pgfsys@defobject{currentmarker}{\pgfqpoint{-0.027778in}{0.000000in}}{\pgfqpoint{0.000000in}{0.000000in}}{%
\pgfpathmoveto{\pgfqpoint{0.000000in}{0.000000in}}%
\pgfpathlineto{\pgfqpoint{-0.027778in}{0.000000in}}%
\pgfusepath{stroke,fill}%
}%
\begin{pgfscope}%
\pgfsys@transformshift{0.781944in}{3.941441in}%
\pgfsys@useobject{currentmarker}{}%
\end{pgfscope}%
\end{pgfscope}%
\begin{pgfscope}%
\pgfsetbuttcap%
\pgfsetroundjoin%
\definecolor{currentfill}{rgb}{0.000000,0.000000,0.000000}%
\pgfsetfillcolor{currentfill}%
\pgfsetlinewidth{0.602250pt}%
\definecolor{currentstroke}{rgb}{0.000000,0.000000,0.000000}%
\pgfsetstrokecolor{currentstroke}%
\pgfsetdash{}{0pt}%
\pgfsys@defobject{currentmarker}{\pgfqpoint{0.000000in}{0.000000in}}{\pgfqpoint{0.027778in}{0.000000in}}{%
\pgfpathmoveto{\pgfqpoint{0.000000in}{0.000000in}}%
\pgfpathlineto{\pgfqpoint{0.027778in}{0.000000in}}%
\pgfusepath{stroke,fill}%
}%
\begin{pgfscope}%
\pgfsys@transformshift{5.801389in}{3.941441in}%
\pgfsys@useobject{currentmarker}{}%
\end{pgfscope}%
\end{pgfscope}%
\begin{pgfscope}%
\pgfpathrectangle{\pgfqpoint{0.781944in}{2.977778in}}{\pgfqpoint{5.019444in}{1.650000in}}%
\pgfusepath{clip}%
\pgfsetrectcap%
\pgfsetroundjoin%
\pgfsetlinewidth{0.803000pt}%
\definecolor{currentstroke}{rgb}{0.690196,0.690196,0.690196}%
\pgfsetstrokecolor{currentstroke}%
\pgfsetstrokeopacity{0.300000}%
\pgfsetdash{}{0pt}%
\pgfpathmoveto{\pgfqpoint{0.781944in}{3.963343in}}%
\pgfpathlineto{\pgfqpoint{5.801389in}{3.963343in}}%
\pgfusepath{stroke}%
\end{pgfscope}%
\begin{pgfscope}%
\pgfsetbuttcap%
\pgfsetroundjoin%
\definecolor{currentfill}{rgb}{0.000000,0.000000,0.000000}%
\pgfsetfillcolor{currentfill}%
\pgfsetlinewidth{0.602250pt}%
\definecolor{currentstroke}{rgb}{0.000000,0.000000,0.000000}%
\pgfsetstrokecolor{currentstroke}%
\pgfsetdash{}{0pt}%
\pgfsys@defobject{currentmarker}{\pgfqpoint{-0.027778in}{0.000000in}}{\pgfqpoint{0.000000in}{0.000000in}}{%
\pgfpathmoveto{\pgfqpoint{0.000000in}{0.000000in}}%
\pgfpathlineto{\pgfqpoint{-0.027778in}{0.000000in}}%
\pgfusepath{stroke,fill}%
}%
\begin{pgfscope}%
\pgfsys@transformshift{0.781944in}{3.963343in}%
\pgfsys@useobject{currentmarker}{}%
\end{pgfscope}%
\end{pgfscope}%
\begin{pgfscope}%
\pgfsetbuttcap%
\pgfsetroundjoin%
\definecolor{currentfill}{rgb}{0.000000,0.000000,0.000000}%
\pgfsetfillcolor{currentfill}%
\pgfsetlinewidth{0.602250pt}%
\definecolor{currentstroke}{rgb}{0.000000,0.000000,0.000000}%
\pgfsetstrokecolor{currentstroke}%
\pgfsetdash{}{0pt}%
\pgfsys@defobject{currentmarker}{\pgfqpoint{0.000000in}{0.000000in}}{\pgfqpoint{0.027778in}{0.000000in}}{%
\pgfpathmoveto{\pgfqpoint{0.000000in}{0.000000in}}%
\pgfpathlineto{\pgfqpoint{0.027778in}{0.000000in}}%
\pgfusepath{stroke,fill}%
}%
\begin{pgfscope}%
\pgfsys@transformshift{5.801389in}{3.963343in}%
\pgfsys@useobject{currentmarker}{}%
\end{pgfscope}%
\end{pgfscope}%
\begin{pgfscope}%
\pgfpathrectangle{\pgfqpoint{0.781944in}{2.977778in}}{\pgfqpoint{5.019444in}{1.650000in}}%
\pgfusepath{clip}%
\pgfsetrectcap%
\pgfsetroundjoin%
\pgfsetlinewidth{0.803000pt}%
\definecolor{currentstroke}{rgb}{0.690196,0.690196,0.690196}%
\pgfsetstrokecolor{currentstroke}%
\pgfsetstrokeopacity{0.300000}%
\pgfsetdash{}{0pt}%
\pgfpathmoveto{\pgfqpoint{0.781944in}{3.985244in}}%
\pgfpathlineto{\pgfqpoint{5.801389in}{3.985244in}}%
\pgfusepath{stroke}%
\end{pgfscope}%
\begin{pgfscope}%
\pgfsetbuttcap%
\pgfsetroundjoin%
\definecolor{currentfill}{rgb}{0.000000,0.000000,0.000000}%
\pgfsetfillcolor{currentfill}%
\pgfsetlinewidth{0.602250pt}%
\definecolor{currentstroke}{rgb}{0.000000,0.000000,0.000000}%
\pgfsetstrokecolor{currentstroke}%
\pgfsetdash{}{0pt}%
\pgfsys@defobject{currentmarker}{\pgfqpoint{-0.027778in}{0.000000in}}{\pgfqpoint{0.000000in}{0.000000in}}{%
\pgfpathmoveto{\pgfqpoint{0.000000in}{0.000000in}}%
\pgfpathlineto{\pgfqpoint{-0.027778in}{0.000000in}}%
\pgfusepath{stroke,fill}%
}%
\begin{pgfscope}%
\pgfsys@transformshift{0.781944in}{3.985244in}%
\pgfsys@useobject{currentmarker}{}%
\end{pgfscope}%
\end{pgfscope}%
\begin{pgfscope}%
\pgfsetbuttcap%
\pgfsetroundjoin%
\definecolor{currentfill}{rgb}{0.000000,0.000000,0.000000}%
\pgfsetfillcolor{currentfill}%
\pgfsetlinewidth{0.602250pt}%
\definecolor{currentstroke}{rgb}{0.000000,0.000000,0.000000}%
\pgfsetstrokecolor{currentstroke}%
\pgfsetdash{}{0pt}%
\pgfsys@defobject{currentmarker}{\pgfqpoint{0.000000in}{0.000000in}}{\pgfqpoint{0.027778in}{0.000000in}}{%
\pgfpathmoveto{\pgfqpoint{0.000000in}{0.000000in}}%
\pgfpathlineto{\pgfqpoint{0.027778in}{0.000000in}}%
\pgfusepath{stroke,fill}%
}%
\begin{pgfscope}%
\pgfsys@transformshift{5.801389in}{3.985244in}%
\pgfsys@useobject{currentmarker}{}%
\end{pgfscope}%
\end{pgfscope}%
\begin{pgfscope}%
\pgfpathrectangle{\pgfqpoint{0.781944in}{2.977778in}}{\pgfqpoint{5.019444in}{1.650000in}}%
\pgfusepath{clip}%
\pgfsetrectcap%
\pgfsetroundjoin%
\pgfsetlinewidth{0.803000pt}%
\definecolor{currentstroke}{rgb}{0.690196,0.690196,0.690196}%
\pgfsetstrokecolor{currentstroke}%
\pgfsetstrokeopacity{0.300000}%
\pgfsetdash{}{0pt}%
\pgfpathmoveto{\pgfqpoint{0.781944in}{4.007146in}}%
\pgfpathlineto{\pgfqpoint{5.801389in}{4.007146in}}%
\pgfusepath{stroke}%
\end{pgfscope}%
\begin{pgfscope}%
\pgfsetbuttcap%
\pgfsetroundjoin%
\definecolor{currentfill}{rgb}{0.000000,0.000000,0.000000}%
\pgfsetfillcolor{currentfill}%
\pgfsetlinewidth{0.602250pt}%
\definecolor{currentstroke}{rgb}{0.000000,0.000000,0.000000}%
\pgfsetstrokecolor{currentstroke}%
\pgfsetdash{}{0pt}%
\pgfsys@defobject{currentmarker}{\pgfqpoint{-0.027778in}{0.000000in}}{\pgfqpoint{0.000000in}{0.000000in}}{%
\pgfpathmoveto{\pgfqpoint{0.000000in}{0.000000in}}%
\pgfpathlineto{\pgfqpoint{-0.027778in}{0.000000in}}%
\pgfusepath{stroke,fill}%
}%
\begin{pgfscope}%
\pgfsys@transformshift{0.781944in}{4.007146in}%
\pgfsys@useobject{currentmarker}{}%
\end{pgfscope}%
\end{pgfscope}%
\begin{pgfscope}%
\pgfsetbuttcap%
\pgfsetroundjoin%
\definecolor{currentfill}{rgb}{0.000000,0.000000,0.000000}%
\pgfsetfillcolor{currentfill}%
\pgfsetlinewidth{0.602250pt}%
\definecolor{currentstroke}{rgb}{0.000000,0.000000,0.000000}%
\pgfsetstrokecolor{currentstroke}%
\pgfsetdash{}{0pt}%
\pgfsys@defobject{currentmarker}{\pgfqpoint{0.000000in}{0.000000in}}{\pgfqpoint{0.027778in}{0.000000in}}{%
\pgfpathmoveto{\pgfqpoint{0.000000in}{0.000000in}}%
\pgfpathlineto{\pgfqpoint{0.027778in}{0.000000in}}%
\pgfusepath{stroke,fill}%
}%
\begin{pgfscope}%
\pgfsys@transformshift{5.801389in}{4.007146in}%
\pgfsys@useobject{currentmarker}{}%
\end{pgfscope}%
\end{pgfscope}%
\begin{pgfscope}%
\pgfpathrectangle{\pgfqpoint{0.781944in}{2.977778in}}{\pgfqpoint{5.019444in}{1.650000in}}%
\pgfusepath{clip}%
\pgfsetrectcap%
\pgfsetroundjoin%
\pgfsetlinewidth{0.803000pt}%
\definecolor{currentstroke}{rgb}{0.690196,0.690196,0.690196}%
\pgfsetstrokecolor{currentstroke}%
\pgfsetstrokeopacity{0.300000}%
\pgfsetdash{}{0pt}%
\pgfpathmoveto{\pgfqpoint{0.781944in}{4.029047in}}%
\pgfpathlineto{\pgfqpoint{5.801389in}{4.029047in}}%
\pgfusepath{stroke}%
\end{pgfscope}%
\begin{pgfscope}%
\pgfsetbuttcap%
\pgfsetroundjoin%
\definecolor{currentfill}{rgb}{0.000000,0.000000,0.000000}%
\pgfsetfillcolor{currentfill}%
\pgfsetlinewidth{0.602250pt}%
\definecolor{currentstroke}{rgb}{0.000000,0.000000,0.000000}%
\pgfsetstrokecolor{currentstroke}%
\pgfsetdash{}{0pt}%
\pgfsys@defobject{currentmarker}{\pgfqpoint{-0.027778in}{0.000000in}}{\pgfqpoint{0.000000in}{0.000000in}}{%
\pgfpathmoveto{\pgfqpoint{0.000000in}{0.000000in}}%
\pgfpathlineto{\pgfqpoint{-0.027778in}{0.000000in}}%
\pgfusepath{stroke,fill}%
}%
\begin{pgfscope}%
\pgfsys@transformshift{0.781944in}{4.029047in}%
\pgfsys@useobject{currentmarker}{}%
\end{pgfscope}%
\end{pgfscope}%
\begin{pgfscope}%
\pgfsetbuttcap%
\pgfsetroundjoin%
\definecolor{currentfill}{rgb}{0.000000,0.000000,0.000000}%
\pgfsetfillcolor{currentfill}%
\pgfsetlinewidth{0.602250pt}%
\definecolor{currentstroke}{rgb}{0.000000,0.000000,0.000000}%
\pgfsetstrokecolor{currentstroke}%
\pgfsetdash{}{0pt}%
\pgfsys@defobject{currentmarker}{\pgfqpoint{0.000000in}{0.000000in}}{\pgfqpoint{0.027778in}{0.000000in}}{%
\pgfpathmoveto{\pgfqpoint{0.000000in}{0.000000in}}%
\pgfpathlineto{\pgfqpoint{0.027778in}{0.000000in}}%
\pgfusepath{stroke,fill}%
}%
\begin{pgfscope}%
\pgfsys@transformshift{5.801389in}{4.029047in}%
\pgfsys@useobject{currentmarker}{}%
\end{pgfscope}%
\end{pgfscope}%
\begin{pgfscope}%
\pgfpathrectangle{\pgfqpoint{0.781944in}{2.977778in}}{\pgfqpoint{5.019444in}{1.650000in}}%
\pgfusepath{clip}%
\pgfsetrectcap%
\pgfsetroundjoin%
\pgfsetlinewidth{0.803000pt}%
\definecolor{currentstroke}{rgb}{0.690196,0.690196,0.690196}%
\pgfsetstrokecolor{currentstroke}%
\pgfsetstrokeopacity{0.300000}%
\pgfsetdash{}{0pt}%
\pgfpathmoveto{\pgfqpoint{0.781944in}{4.050949in}}%
\pgfpathlineto{\pgfqpoint{5.801389in}{4.050949in}}%
\pgfusepath{stroke}%
\end{pgfscope}%
\begin{pgfscope}%
\pgfsetbuttcap%
\pgfsetroundjoin%
\definecolor{currentfill}{rgb}{0.000000,0.000000,0.000000}%
\pgfsetfillcolor{currentfill}%
\pgfsetlinewidth{0.602250pt}%
\definecolor{currentstroke}{rgb}{0.000000,0.000000,0.000000}%
\pgfsetstrokecolor{currentstroke}%
\pgfsetdash{}{0pt}%
\pgfsys@defobject{currentmarker}{\pgfqpoint{-0.027778in}{0.000000in}}{\pgfqpoint{0.000000in}{0.000000in}}{%
\pgfpathmoveto{\pgfqpoint{0.000000in}{0.000000in}}%
\pgfpathlineto{\pgfqpoint{-0.027778in}{0.000000in}}%
\pgfusepath{stroke,fill}%
}%
\begin{pgfscope}%
\pgfsys@transformshift{0.781944in}{4.050949in}%
\pgfsys@useobject{currentmarker}{}%
\end{pgfscope}%
\end{pgfscope}%
\begin{pgfscope}%
\pgfsetbuttcap%
\pgfsetroundjoin%
\definecolor{currentfill}{rgb}{0.000000,0.000000,0.000000}%
\pgfsetfillcolor{currentfill}%
\pgfsetlinewidth{0.602250pt}%
\definecolor{currentstroke}{rgb}{0.000000,0.000000,0.000000}%
\pgfsetstrokecolor{currentstroke}%
\pgfsetdash{}{0pt}%
\pgfsys@defobject{currentmarker}{\pgfqpoint{0.000000in}{0.000000in}}{\pgfqpoint{0.027778in}{0.000000in}}{%
\pgfpathmoveto{\pgfqpoint{0.000000in}{0.000000in}}%
\pgfpathlineto{\pgfqpoint{0.027778in}{0.000000in}}%
\pgfusepath{stroke,fill}%
}%
\begin{pgfscope}%
\pgfsys@transformshift{5.801389in}{4.050949in}%
\pgfsys@useobject{currentmarker}{}%
\end{pgfscope}%
\end{pgfscope}%
\begin{pgfscope}%
\pgfpathrectangle{\pgfqpoint{0.781944in}{2.977778in}}{\pgfqpoint{5.019444in}{1.650000in}}%
\pgfusepath{clip}%
\pgfsetrectcap%
\pgfsetroundjoin%
\pgfsetlinewidth{0.803000pt}%
\definecolor{currentstroke}{rgb}{0.690196,0.690196,0.690196}%
\pgfsetstrokecolor{currentstroke}%
\pgfsetstrokeopacity{0.300000}%
\pgfsetdash{}{0pt}%
\pgfpathmoveto{\pgfqpoint{0.781944in}{4.094751in}}%
\pgfpathlineto{\pgfqpoint{5.801389in}{4.094751in}}%
\pgfusepath{stroke}%
\end{pgfscope}%
\begin{pgfscope}%
\pgfsetbuttcap%
\pgfsetroundjoin%
\definecolor{currentfill}{rgb}{0.000000,0.000000,0.000000}%
\pgfsetfillcolor{currentfill}%
\pgfsetlinewidth{0.602250pt}%
\definecolor{currentstroke}{rgb}{0.000000,0.000000,0.000000}%
\pgfsetstrokecolor{currentstroke}%
\pgfsetdash{}{0pt}%
\pgfsys@defobject{currentmarker}{\pgfqpoint{-0.027778in}{0.000000in}}{\pgfqpoint{0.000000in}{0.000000in}}{%
\pgfpathmoveto{\pgfqpoint{0.000000in}{0.000000in}}%
\pgfpathlineto{\pgfqpoint{-0.027778in}{0.000000in}}%
\pgfusepath{stroke,fill}%
}%
\begin{pgfscope}%
\pgfsys@transformshift{0.781944in}{4.094751in}%
\pgfsys@useobject{currentmarker}{}%
\end{pgfscope}%
\end{pgfscope}%
\begin{pgfscope}%
\pgfsetbuttcap%
\pgfsetroundjoin%
\definecolor{currentfill}{rgb}{0.000000,0.000000,0.000000}%
\pgfsetfillcolor{currentfill}%
\pgfsetlinewidth{0.602250pt}%
\definecolor{currentstroke}{rgb}{0.000000,0.000000,0.000000}%
\pgfsetstrokecolor{currentstroke}%
\pgfsetdash{}{0pt}%
\pgfsys@defobject{currentmarker}{\pgfqpoint{0.000000in}{0.000000in}}{\pgfqpoint{0.027778in}{0.000000in}}{%
\pgfpathmoveto{\pgfqpoint{0.000000in}{0.000000in}}%
\pgfpathlineto{\pgfqpoint{0.027778in}{0.000000in}}%
\pgfusepath{stroke,fill}%
}%
\begin{pgfscope}%
\pgfsys@transformshift{5.801389in}{4.094751in}%
\pgfsys@useobject{currentmarker}{}%
\end{pgfscope}%
\end{pgfscope}%
\begin{pgfscope}%
\pgfpathrectangle{\pgfqpoint{0.781944in}{2.977778in}}{\pgfqpoint{5.019444in}{1.650000in}}%
\pgfusepath{clip}%
\pgfsetrectcap%
\pgfsetroundjoin%
\pgfsetlinewidth{0.803000pt}%
\definecolor{currentstroke}{rgb}{0.690196,0.690196,0.690196}%
\pgfsetstrokecolor{currentstroke}%
\pgfsetstrokeopacity{0.300000}%
\pgfsetdash{}{0pt}%
\pgfpathmoveto{\pgfqpoint{0.781944in}{4.116653in}}%
\pgfpathlineto{\pgfqpoint{5.801389in}{4.116653in}}%
\pgfusepath{stroke}%
\end{pgfscope}%
\begin{pgfscope}%
\pgfsetbuttcap%
\pgfsetroundjoin%
\definecolor{currentfill}{rgb}{0.000000,0.000000,0.000000}%
\pgfsetfillcolor{currentfill}%
\pgfsetlinewidth{0.602250pt}%
\definecolor{currentstroke}{rgb}{0.000000,0.000000,0.000000}%
\pgfsetstrokecolor{currentstroke}%
\pgfsetdash{}{0pt}%
\pgfsys@defobject{currentmarker}{\pgfqpoint{-0.027778in}{0.000000in}}{\pgfqpoint{0.000000in}{0.000000in}}{%
\pgfpathmoveto{\pgfqpoint{0.000000in}{0.000000in}}%
\pgfpathlineto{\pgfqpoint{-0.027778in}{0.000000in}}%
\pgfusepath{stroke,fill}%
}%
\begin{pgfscope}%
\pgfsys@transformshift{0.781944in}{4.116653in}%
\pgfsys@useobject{currentmarker}{}%
\end{pgfscope}%
\end{pgfscope}%
\begin{pgfscope}%
\pgfsetbuttcap%
\pgfsetroundjoin%
\definecolor{currentfill}{rgb}{0.000000,0.000000,0.000000}%
\pgfsetfillcolor{currentfill}%
\pgfsetlinewidth{0.602250pt}%
\definecolor{currentstroke}{rgb}{0.000000,0.000000,0.000000}%
\pgfsetstrokecolor{currentstroke}%
\pgfsetdash{}{0pt}%
\pgfsys@defobject{currentmarker}{\pgfqpoint{0.000000in}{0.000000in}}{\pgfqpoint{0.027778in}{0.000000in}}{%
\pgfpathmoveto{\pgfqpoint{0.000000in}{0.000000in}}%
\pgfpathlineto{\pgfqpoint{0.027778in}{0.000000in}}%
\pgfusepath{stroke,fill}%
}%
\begin{pgfscope}%
\pgfsys@transformshift{5.801389in}{4.116653in}%
\pgfsys@useobject{currentmarker}{}%
\end{pgfscope}%
\end{pgfscope}%
\begin{pgfscope}%
\pgfpathrectangle{\pgfqpoint{0.781944in}{2.977778in}}{\pgfqpoint{5.019444in}{1.650000in}}%
\pgfusepath{clip}%
\pgfsetrectcap%
\pgfsetroundjoin%
\pgfsetlinewidth{0.803000pt}%
\definecolor{currentstroke}{rgb}{0.690196,0.690196,0.690196}%
\pgfsetstrokecolor{currentstroke}%
\pgfsetstrokeopacity{0.300000}%
\pgfsetdash{}{0pt}%
\pgfpathmoveto{\pgfqpoint{0.781944in}{4.138554in}}%
\pgfpathlineto{\pgfqpoint{5.801389in}{4.138554in}}%
\pgfusepath{stroke}%
\end{pgfscope}%
\begin{pgfscope}%
\pgfsetbuttcap%
\pgfsetroundjoin%
\definecolor{currentfill}{rgb}{0.000000,0.000000,0.000000}%
\pgfsetfillcolor{currentfill}%
\pgfsetlinewidth{0.602250pt}%
\definecolor{currentstroke}{rgb}{0.000000,0.000000,0.000000}%
\pgfsetstrokecolor{currentstroke}%
\pgfsetdash{}{0pt}%
\pgfsys@defobject{currentmarker}{\pgfqpoint{-0.027778in}{0.000000in}}{\pgfqpoint{0.000000in}{0.000000in}}{%
\pgfpathmoveto{\pgfqpoint{0.000000in}{0.000000in}}%
\pgfpathlineto{\pgfqpoint{-0.027778in}{0.000000in}}%
\pgfusepath{stroke,fill}%
}%
\begin{pgfscope}%
\pgfsys@transformshift{0.781944in}{4.138554in}%
\pgfsys@useobject{currentmarker}{}%
\end{pgfscope}%
\end{pgfscope}%
\begin{pgfscope}%
\pgfsetbuttcap%
\pgfsetroundjoin%
\definecolor{currentfill}{rgb}{0.000000,0.000000,0.000000}%
\pgfsetfillcolor{currentfill}%
\pgfsetlinewidth{0.602250pt}%
\definecolor{currentstroke}{rgb}{0.000000,0.000000,0.000000}%
\pgfsetstrokecolor{currentstroke}%
\pgfsetdash{}{0pt}%
\pgfsys@defobject{currentmarker}{\pgfqpoint{0.000000in}{0.000000in}}{\pgfqpoint{0.027778in}{0.000000in}}{%
\pgfpathmoveto{\pgfqpoint{0.000000in}{0.000000in}}%
\pgfpathlineto{\pgfqpoint{0.027778in}{0.000000in}}%
\pgfusepath{stroke,fill}%
}%
\begin{pgfscope}%
\pgfsys@transformshift{5.801389in}{4.138554in}%
\pgfsys@useobject{currentmarker}{}%
\end{pgfscope}%
\end{pgfscope}%
\begin{pgfscope}%
\pgfpathrectangle{\pgfqpoint{0.781944in}{2.977778in}}{\pgfqpoint{5.019444in}{1.650000in}}%
\pgfusepath{clip}%
\pgfsetrectcap%
\pgfsetroundjoin%
\pgfsetlinewidth{0.803000pt}%
\definecolor{currentstroke}{rgb}{0.690196,0.690196,0.690196}%
\pgfsetstrokecolor{currentstroke}%
\pgfsetstrokeopacity{0.300000}%
\pgfsetdash{}{0pt}%
\pgfpathmoveto{\pgfqpoint{0.781944in}{4.160456in}}%
\pgfpathlineto{\pgfqpoint{5.801389in}{4.160456in}}%
\pgfusepath{stroke}%
\end{pgfscope}%
\begin{pgfscope}%
\pgfsetbuttcap%
\pgfsetroundjoin%
\definecolor{currentfill}{rgb}{0.000000,0.000000,0.000000}%
\pgfsetfillcolor{currentfill}%
\pgfsetlinewidth{0.602250pt}%
\definecolor{currentstroke}{rgb}{0.000000,0.000000,0.000000}%
\pgfsetstrokecolor{currentstroke}%
\pgfsetdash{}{0pt}%
\pgfsys@defobject{currentmarker}{\pgfqpoint{-0.027778in}{0.000000in}}{\pgfqpoint{0.000000in}{0.000000in}}{%
\pgfpathmoveto{\pgfqpoint{0.000000in}{0.000000in}}%
\pgfpathlineto{\pgfqpoint{-0.027778in}{0.000000in}}%
\pgfusepath{stroke,fill}%
}%
\begin{pgfscope}%
\pgfsys@transformshift{0.781944in}{4.160456in}%
\pgfsys@useobject{currentmarker}{}%
\end{pgfscope}%
\end{pgfscope}%
\begin{pgfscope}%
\pgfsetbuttcap%
\pgfsetroundjoin%
\definecolor{currentfill}{rgb}{0.000000,0.000000,0.000000}%
\pgfsetfillcolor{currentfill}%
\pgfsetlinewidth{0.602250pt}%
\definecolor{currentstroke}{rgb}{0.000000,0.000000,0.000000}%
\pgfsetstrokecolor{currentstroke}%
\pgfsetdash{}{0pt}%
\pgfsys@defobject{currentmarker}{\pgfqpoint{0.000000in}{0.000000in}}{\pgfqpoint{0.027778in}{0.000000in}}{%
\pgfpathmoveto{\pgfqpoint{0.000000in}{0.000000in}}%
\pgfpathlineto{\pgfqpoint{0.027778in}{0.000000in}}%
\pgfusepath{stroke,fill}%
}%
\begin{pgfscope}%
\pgfsys@transformshift{5.801389in}{4.160456in}%
\pgfsys@useobject{currentmarker}{}%
\end{pgfscope}%
\end{pgfscope}%
\begin{pgfscope}%
\pgfpathrectangle{\pgfqpoint{0.781944in}{2.977778in}}{\pgfqpoint{5.019444in}{1.650000in}}%
\pgfusepath{clip}%
\pgfsetrectcap%
\pgfsetroundjoin%
\pgfsetlinewidth{0.803000pt}%
\definecolor{currentstroke}{rgb}{0.690196,0.690196,0.690196}%
\pgfsetstrokecolor{currentstroke}%
\pgfsetstrokeopacity{0.300000}%
\pgfsetdash{}{0pt}%
\pgfpathmoveto{\pgfqpoint{0.781944in}{4.182357in}}%
\pgfpathlineto{\pgfqpoint{5.801389in}{4.182357in}}%
\pgfusepath{stroke}%
\end{pgfscope}%
\begin{pgfscope}%
\pgfsetbuttcap%
\pgfsetroundjoin%
\definecolor{currentfill}{rgb}{0.000000,0.000000,0.000000}%
\pgfsetfillcolor{currentfill}%
\pgfsetlinewidth{0.602250pt}%
\definecolor{currentstroke}{rgb}{0.000000,0.000000,0.000000}%
\pgfsetstrokecolor{currentstroke}%
\pgfsetdash{}{0pt}%
\pgfsys@defobject{currentmarker}{\pgfqpoint{-0.027778in}{0.000000in}}{\pgfqpoint{0.000000in}{0.000000in}}{%
\pgfpathmoveto{\pgfqpoint{0.000000in}{0.000000in}}%
\pgfpathlineto{\pgfqpoint{-0.027778in}{0.000000in}}%
\pgfusepath{stroke,fill}%
}%
\begin{pgfscope}%
\pgfsys@transformshift{0.781944in}{4.182357in}%
\pgfsys@useobject{currentmarker}{}%
\end{pgfscope}%
\end{pgfscope}%
\begin{pgfscope}%
\pgfsetbuttcap%
\pgfsetroundjoin%
\definecolor{currentfill}{rgb}{0.000000,0.000000,0.000000}%
\pgfsetfillcolor{currentfill}%
\pgfsetlinewidth{0.602250pt}%
\definecolor{currentstroke}{rgb}{0.000000,0.000000,0.000000}%
\pgfsetstrokecolor{currentstroke}%
\pgfsetdash{}{0pt}%
\pgfsys@defobject{currentmarker}{\pgfqpoint{0.000000in}{0.000000in}}{\pgfqpoint{0.027778in}{0.000000in}}{%
\pgfpathmoveto{\pgfqpoint{0.000000in}{0.000000in}}%
\pgfpathlineto{\pgfqpoint{0.027778in}{0.000000in}}%
\pgfusepath{stroke,fill}%
}%
\begin{pgfscope}%
\pgfsys@transformshift{5.801389in}{4.182357in}%
\pgfsys@useobject{currentmarker}{}%
\end{pgfscope}%
\end{pgfscope}%
\begin{pgfscope}%
\pgfpathrectangle{\pgfqpoint{0.781944in}{2.977778in}}{\pgfqpoint{5.019444in}{1.650000in}}%
\pgfusepath{clip}%
\pgfsetrectcap%
\pgfsetroundjoin%
\pgfsetlinewidth{0.803000pt}%
\definecolor{currentstroke}{rgb}{0.690196,0.690196,0.690196}%
\pgfsetstrokecolor{currentstroke}%
\pgfsetstrokeopacity{0.300000}%
\pgfsetdash{}{0pt}%
\pgfpathmoveto{\pgfqpoint{0.781944in}{4.204259in}}%
\pgfpathlineto{\pgfqpoint{5.801389in}{4.204259in}}%
\pgfusepath{stroke}%
\end{pgfscope}%
\begin{pgfscope}%
\pgfsetbuttcap%
\pgfsetroundjoin%
\definecolor{currentfill}{rgb}{0.000000,0.000000,0.000000}%
\pgfsetfillcolor{currentfill}%
\pgfsetlinewidth{0.602250pt}%
\definecolor{currentstroke}{rgb}{0.000000,0.000000,0.000000}%
\pgfsetstrokecolor{currentstroke}%
\pgfsetdash{}{0pt}%
\pgfsys@defobject{currentmarker}{\pgfqpoint{-0.027778in}{0.000000in}}{\pgfqpoint{0.000000in}{0.000000in}}{%
\pgfpathmoveto{\pgfqpoint{0.000000in}{0.000000in}}%
\pgfpathlineto{\pgfqpoint{-0.027778in}{0.000000in}}%
\pgfusepath{stroke,fill}%
}%
\begin{pgfscope}%
\pgfsys@transformshift{0.781944in}{4.204259in}%
\pgfsys@useobject{currentmarker}{}%
\end{pgfscope}%
\end{pgfscope}%
\begin{pgfscope}%
\pgfsetbuttcap%
\pgfsetroundjoin%
\definecolor{currentfill}{rgb}{0.000000,0.000000,0.000000}%
\pgfsetfillcolor{currentfill}%
\pgfsetlinewidth{0.602250pt}%
\definecolor{currentstroke}{rgb}{0.000000,0.000000,0.000000}%
\pgfsetstrokecolor{currentstroke}%
\pgfsetdash{}{0pt}%
\pgfsys@defobject{currentmarker}{\pgfqpoint{0.000000in}{0.000000in}}{\pgfqpoint{0.027778in}{0.000000in}}{%
\pgfpathmoveto{\pgfqpoint{0.000000in}{0.000000in}}%
\pgfpathlineto{\pgfqpoint{0.027778in}{0.000000in}}%
\pgfusepath{stroke,fill}%
}%
\begin{pgfscope}%
\pgfsys@transformshift{5.801389in}{4.204259in}%
\pgfsys@useobject{currentmarker}{}%
\end{pgfscope}%
\end{pgfscope}%
\begin{pgfscope}%
\pgfpathrectangle{\pgfqpoint{0.781944in}{2.977778in}}{\pgfqpoint{5.019444in}{1.650000in}}%
\pgfusepath{clip}%
\pgfsetrectcap%
\pgfsetroundjoin%
\pgfsetlinewidth{0.803000pt}%
\definecolor{currentstroke}{rgb}{0.690196,0.690196,0.690196}%
\pgfsetstrokecolor{currentstroke}%
\pgfsetstrokeopacity{0.300000}%
\pgfsetdash{}{0pt}%
\pgfpathmoveto{\pgfqpoint{0.781944in}{4.226160in}}%
\pgfpathlineto{\pgfqpoint{5.801389in}{4.226160in}}%
\pgfusepath{stroke}%
\end{pgfscope}%
\begin{pgfscope}%
\pgfsetbuttcap%
\pgfsetroundjoin%
\definecolor{currentfill}{rgb}{0.000000,0.000000,0.000000}%
\pgfsetfillcolor{currentfill}%
\pgfsetlinewidth{0.602250pt}%
\definecolor{currentstroke}{rgb}{0.000000,0.000000,0.000000}%
\pgfsetstrokecolor{currentstroke}%
\pgfsetdash{}{0pt}%
\pgfsys@defobject{currentmarker}{\pgfqpoint{-0.027778in}{0.000000in}}{\pgfqpoint{0.000000in}{0.000000in}}{%
\pgfpathmoveto{\pgfqpoint{0.000000in}{0.000000in}}%
\pgfpathlineto{\pgfqpoint{-0.027778in}{0.000000in}}%
\pgfusepath{stroke,fill}%
}%
\begin{pgfscope}%
\pgfsys@transformshift{0.781944in}{4.226160in}%
\pgfsys@useobject{currentmarker}{}%
\end{pgfscope}%
\end{pgfscope}%
\begin{pgfscope}%
\pgfsetbuttcap%
\pgfsetroundjoin%
\definecolor{currentfill}{rgb}{0.000000,0.000000,0.000000}%
\pgfsetfillcolor{currentfill}%
\pgfsetlinewidth{0.602250pt}%
\definecolor{currentstroke}{rgb}{0.000000,0.000000,0.000000}%
\pgfsetstrokecolor{currentstroke}%
\pgfsetdash{}{0pt}%
\pgfsys@defobject{currentmarker}{\pgfqpoint{0.000000in}{0.000000in}}{\pgfqpoint{0.027778in}{0.000000in}}{%
\pgfpathmoveto{\pgfqpoint{0.000000in}{0.000000in}}%
\pgfpathlineto{\pgfqpoint{0.027778in}{0.000000in}}%
\pgfusepath{stroke,fill}%
}%
\begin{pgfscope}%
\pgfsys@transformshift{5.801389in}{4.226160in}%
\pgfsys@useobject{currentmarker}{}%
\end{pgfscope}%
\end{pgfscope}%
\begin{pgfscope}%
\pgfpathrectangle{\pgfqpoint{0.781944in}{2.977778in}}{\pgfqpoint{5.019444in}{1.650000in}}%
\pgfusepath{clip}%
\pgfsetrectcap%
\pgfsetroundjoin%
\pgfsetlinewidth{0.803000pt}%
\definecolor{currentstroke}{rgb}{0.690196,0.690196,0.690196}%
\pgfsetstrokecolor{currentstroke}%
\pgfsetstrokeopacity{0.300000}%
\pgfsetdash{}{0pt}%
\pgfpathmoveto{\pgfqpoint{0.781944in}{4.248062in}}%
\pgfpathlineto{\pgfqpoint{5.801389in}{4.248062in}}%
\pgfusepath{stroke}%
\end{pgfscope}%
\begin{pgfscope}%
\pgfsetbuttcap%
\pgfsetroundjoin%
\definecolor{currentfill}{rgb}{0.000000,0.000000,0.000000}%
\pgfsetfillcolor{currentfill}%
\pgfsetlinewidth{0.602250pt}%
\definecolor{currentstroke}{rgb}{0.000000,0.000000,0.000000}%
\pgfsetstrokecolor{currentstroke}%
\pgfsetdash{}{0pt}%
\pgfsys@defobject{currentmarker}{\pgfqpoint{-0.027778in}{0.000000in}}{\pgfqpoint{0.000000in}{0.000000in}}{%
\pgfpathmoveto{\pgfqpoint{0.000000in}{0.000000in}}%
\pgfpathlineto{\pgfqpoint{-0.027778in}{0.000000in}}%
\pgfusepath{stroke,fill}%
}%
\begin{pgfscope}%
\pgfsys@transformshift{0.781944in}{4.248062in}%
\pgfsys@useobject{currentmarker}{}%
\end{pgfscope}%
\end{pgfscope}%
\begin{pgfscope}%
\pgfsetbuttcap%
\pgfsetroundjoin%
\definecolor{currentfill}{rgb}{0.000000,0.000000,0.000000}%
\pgfsetfillcolor{currentfill}%
\pgfsetlinewidth{0.602250pt}%
\definecolor{currentstroke}{rgb}{0.000000,0.000000,0.000000}%
\pgfsetstrokecolor{currentstroke}%
\pgfsetdash{}{0pt}%
\pgfsys@defobject{currentmarker}{\pgfqpoint{0.000000in}{0.000000in}}{\pgfqpoint{0.027778in}{0.000000in}}{%
\pgfpathmoveto{\pgfqpoint{0.000000in}{0.000000in}}%
\pgfpathlineto{\pgfqpoint{0.027778in}{0.000000in}}%
\pgfusepath{stroke,fill}%
}%
\begin{pgfscope}%
\pgfsys@transformshift{5.801389in}{4.248062in}%
\pgfsys@useobject{currentmarker}{}%
\end{pgfscope}%
\end{pgfscope}%
\begin{pgfscope}%
\pgfpathrectangle{\pgfqpoint{0.781944in}{2.977778in}}{\pgfqpoint{5.019444in}{1.650000in}}%
\pgfusepath{clip}%
\pgfsetrectcap%
\pgfsetroundjoin%
\pgfsetlinewidth{0.803000pt}%
\definecolor{currentstroke}{rgb}{0.690196,0.690196,0.690196}%
\pgfsetstrokecolor{currentstroke}%
\pgfsetstrokeopacity{0.300000}%
\pgfsetdash{}{0pt}%
\pgfpathmoveto{\pgfqpoint{0.781944in}{4.269963in}}%
\pgfpathlineto{\pgfqpoint{5.801389in}{4.269963in}}%
\pgfusepath{stroke}%
\end{pgfscope}%
\begin{pgfscope}%
\pgfsetbuttcap%
\pgfsetroundjoin%
\definecolor{currentfill}{rgb}{0.000000,0.000000,0.000000}%
\pgfsetfillcolor{currentfill}%
\pgfsetlinewidth{0.602250pt}%
\definecolor{currentstroke}{rgb}{0.000000,0.000000,0.000000}%
\pgfsetstrokecolor{currentstroke}%
\pgfsetdash{}{0pt}%
\pgfsys@defobject{currentmarker}{\pgfqpoint{-0.027778in}{0.000000in}}{\pgfqpoint{0.000000in}{0.000000in}}{%
\pgfpathmoveto{\pgfqpoint{0.000000in}{0.000000in}}%
\pgfpathlineto{\pgfqpoint{-0.027778in}{0.000000in}}%
\pgfusepath{stroke,fill}%
}%
\begin{pgfscope}%
\pgfsys@transformshift{0.781944in}{4.269963in}%
\pgfsys@useobject{currentmarker}{}%
\end{pgfscope}%
\end{pgfscope}%
\begin{pgfscope}%
\pgfsetbuttcap%
\pgfsetroundjoin%
\definecolor{currentfill}{rgb}{0.000000,0.000000,0.000000}%
\pgfsetfillcolor{currentfill}%
\pgfsetlinewidth{0.602250pt}%
\definecolor{currentstroke}{rgb}{0.000000,0.000000,0.000000}%
\pgfsetstrokecolor{currentstroke}%
\pgfsetdash{}{0pt}%
\pgfsys@defobject{currentmarker}{\pgfqpoint{0.000000in}{0.000000in}}{\pgfqpoint{0.027778in}{0.000000in}}{%
\pgfpathmoveto{\pgfqpoint{0.000000in}{0.000000in}}%
\pgfpathlineto{\pgfqpoint{0.027778in}{0.000000in}}%
\pgfusepath{stroke,fill}%
}%
\begin{pgfscope}%
\pgfsys@transformshift{5.801389in}{4.269963in}%
\pgfsys@useobject{currentmarker}{}%
\end{pgfscope}%
\end{pgfscope}%
\begin{pgfscope}%
\pgfpathrectangle{\pgfqpoint{0.781944in}{2.977778in}}{\pgfqpoint{5.019444in}{1.650000in}}%
\pgfusepath{clip}%
\pgfsetrectcap%
\pgfsetroundjoin%
\pgfsetlinewidth{0.803000pt}%
\definecolor{currentstroke}{rgb}{0.690196,0.690196,0.690196}%
\pgfsetstrokecolor{currentstroke}%
\pgfsetstrokeopacity{0.300000}%
\pgfsetdash{}{0pt}%
\pgfpathmoveto{\pgfqpoint{0.781944in}{4.313766in}}%
\pgfpathlineto{\pgfqpoint{5.801389in}{4.313766in}}%
\pgfusepath{stroke}%
\end{pgfscope}%
\begin{pgfscope}%
\pgfsetbuttcap%
\pgfsetroundjoin%
\definecolor{currentfill}{rgb}{0.000000,0.000000,0.000000}%
\pgfsetfillcolor{currentfill}%
\pgfsetlinewidth{0.602250pt}%
\definecolor{currentstroke}{rgb}{0.000000,0.000000,0.000000}%
\pgfsetstrokecolor{currentstroke}%
\pgfsetdash{}{0pt}%
\pgfsys@defobject{currentmarker}{\pgfqpoint{-0.027778in}{0.000000in}}{\pgfqpoint{0.000000in}{0.000000in}}{%
\pgfpathmoveto{\pgfqpoint{0.000000in}{0.000000in}}%
\pgfpathlineto{\pgfqpoint{-0.027778in}{0.000000in}}%
\pgfusepath{stroke,fill}%
}%
\begin{pgfscope}%
\pgfsys@transformshift{0.781944in}{4.313766in}%
\pgfsys@useobject{currentmarker}{}%
\end{pgfscope}%
\end{pgfscope}%
\begin{pgfscope}%
\pgfsetbuttcap%
\pgfsetroundjoin%
\definecolor{currentfill}{rgb}{0.000000,0.000000,0.000000}%
\pgfsetfillcolor{currentfill}%
\pgfsetlinewidth{0.602250pt}%
\definecolor{currentstroke}{rgb}{0.000000,0.000000,0.000000}%
\pgfsetstrokecolor{currentstroke}%
\pgfsetdash{}{0pt}%
\pgfsys@defobject{currentmarker}{\pgfqpoint{0.000000in}{0.000000in}}{\pgfqpoint{0.027778in}{0.000000in}}{%
\pgfpathmoveto{\pgfqpoint{0.000000in}{0.000000in}}%
\pgfpathlineto{\pgfqpoint{0.027778in}{0.000000in}}%
\pgfusepath{stroke,fill}%
}%
\begin{pgfscope}%
\pgfsys@transformshift{5.801389in}{4.313766in}%
\pgfsys@useobject{currentmarker}{}%
\end{pgfscope}%
\end{pgfscope}%
\begin{pgfscope}%
\pgfpathrectangle{\pgfqpoint{0.781944in}{2.977778in}}{\pgfqpoint{5.019444in}{1.650000in}}%
\pgfusepath{clip}%
\pgfsetrectcap%
\pgfsetroundjoin%
\pgfsetlinewidth{0.803000pt}%
\definecolor{currentstroke}{rgb}{0.690196,0.690196,0.690196}%
\pgfsetstrokecolor{currentstroke}%
\pgfsetstrokeopacity{0.300000}%
\pgfsetdash{}{0pt}%
\pgfpathmoveto{\pgfqpoint{0.781944in}{4.335667in}}%
\pgfpathlineto{\pgfqpoint{5.801389in}{4.335667in}}%
\pgfusepath{stroke}%
\end{pgfscope}%
\begin{pgfscope}%
\pgfsetbuttcap%
\pgfsetroundjoin%
\definecolor{currentfill}{rgb}{0.000000,0.000000,0.000000}%
\pgfsetfillcolor{currentfill}%
\pgfsetlinewidth{0.602250pt}%
\definecolor{currentstroke}{rgb}{0.000000,0.000000,0.000000}%
\pgfsetstrokecolor{currentstroke}%
\pgfsetdash{}{0pt}%
\pgfsys@defobject{currentmarker}{\pgfqpoint{-0.027778in}{0.000000in}}{\pgfqpoint{0.000000in}{0.000000in}}{%
\pgfpathmoveto{\pgfqpoint{0.000000in}{0.000000in}}%
\pgfpathlineto{\pgfqpoint{-0.027778in}{0.000000in}}%
\pgfusepath{stroke,fill}%
}%
\begin{pgfscope}%
\pgfsys@transformshift{0.781944in}{4.335667in}%
\pgfsys@useobject{currentmarker}{}%
\end{pgfscope}%
\end{pgfscope}%
\begin{pgfscope}%
\pgfsetbuttcap%
\pgfsetroundjoin%
\definecolor{currentfill}{rgb}{0.000000,0.000000,0.000000}%
\pgfsetfillcolor{currentfill}%
\pgfsetlinewidth{0.602250pt}%
\definecolor{currentstroke}{rgb}{0.000000,0.000000,0.000000}%
\pgfsetstrokecolor{currentstroke}%
\pgfsetdash{}{0pt}%
\pgfsys@defobject{currentmarker}{\pgfqpoint{0.000000in}{0.000000in}}{\pgfqpoint{0.027778in}{0.000000in}}{%
\pgfpathmoveto{\pgfqpoint{0.000000in}{0.000000in}}%
\pgfpathlineto{\pgfqpoint{0.027778in}{0.000000in}}%
\pgfusepath{stroke,fill}%
}%
\begin{pgfscope}%
\pgfsys@transformshift{5.801389in}{4.335667in}%
\pgfsys@useobject{currentmarker}{}%
\end{pgfscope}%
\end{pgfscope}%
\begin{pgfscope}%
\pgfpathrectangle{\pgfqpoint{0.781944in}{2.977778in}}{\pgfqpoint{5.019444in}{1.650000in}}%
\pgfusepath{clip}%
\pgfsetrectcap%
\pgfsetroundjoin%
\pgfsetlinewidth{0.803000pt}%
\definecolor{currentstroke}{rgb}{0.690196,0.690196,0.690196}%
\pgfsetstrokecolor{currentstroke}%
\pgfsetstrokeopacity{0.300000}%
\pgfsetdash{}{0pt}%
\pgfpathmoveto{\pgfqpoint{0.781944in}{4.357569in}}%
\pgfpathlineto{\pgfqpoint{5.801389in}{4.357569in}}%
\pgfusepath{stroke}%
\end{pgfscope}%
\begin{pgfscope}%
\pgfsetbuttcap%
\pgfsetroundjoin%
\definecolor{currentfill}{rgb}{0.000000,0.000000,0.000000}%
\pgfsetfillcolor{currentfill}%
\pgfsetlinewidth{0.602250pt}%
\definecolor{currentstroke}{rgb}{0.000000,0.000000,0.000000}%
\pgfsetstrokecolor{currentstroke}%
\pgfsetdash{}{0pt}%
\pgfsys@defobject{currentmarker}{\pgfqpoint{-0.027778in}{0.000000in}}{\pgfqpoint{0.000000in}{0.000000in}}{%
\pgfpathmoveto{\pgfqpoint{0.000000in}{0.000000in}}%
\pgfpathlineto{\pgfqpoint{-0.027778in}{0.000000in}}%
\pgfusepath{stroke,fill}%
}%
\begin{pgfscope}%
\pgfsys@transformshift{0.781944in}{4.357569in}%
\pgfsys@useobject{currentmarker}{}%
\end{pgfscope}%
\end{pgfscope}%
\begin{pgfscope}%
\pgfsetbuttcap%
\pgfsetroundjoin%
\definecolor{currentfill}{rgb}{0.000000,0.000000,0.000000}%
\pgfsetfillcolor{currentfill}%
\pgfsetlinewidth{0.602250pt}%
\definecolor{currentstroke}{rgb}{0.000000,0.000000,0.000000}%
\pgfsetstrokecolor{currentstroke}%
\pgfsetdash{}{0pt}%
\pgfsys@defobject{currentmarker}{\pgfqpoint{0.000000in}{0.000000in}}{\pgfqpoint{0.027778in}{0.000000in}}{%
\pgfpathmoveto{\pgfqpoint{0.000000in}{0.000000in}}%
\pgfpathlineto{\pgfqpoint{0.027778in}{0.000000in}}%
\pgfusepath{stroke,fill}%
}%
\begin{pgfscope}%
\pgfsys@transformshift{5.801389in}{4.357569in}%
\pgfsys@useobject{currentmarker}{}%
\end{pgfscope}%
\end{pgfscope}%
\begin{pgfscope}%
\pgfpathrectangle{\pgfqpoint{0.781944in}{2.977778in}}{\pgfqpoint{5.019444in}{1.650000in}}%
\pgfusepath{clip}%
\pgfsetrectcap%
\pgfsetroundjoin%
\pgfsetlinewidth{0.803000pt}%
\definecolor{currentstroke}{rgb}{0.690196,0.690196,0.690196}%
\pgfsetstrokecolor{currentstroke}%
\pgfsetstrokeopacity{0.300000}%
\pgfsetdash{}{0pt}%
\pgfpathmoveto{\pgfqpoint{0.781944in}{4.379470in}}%
\pgfpathlineto{\pgfqpoint{5.801389in}{4.379470in}}%
\pgfusepath{stroke}%
\end{pgfscope}%
\begin{pgfscope}%
\pgfsetbuttcap%
\pgfsetroundjoin%
\definecolor{currentfill}{rgb}{0.000000,0.000000,0.000000}%
\pgfsetfillcolor{currentfill}%
\pgfsetlinewidth{0.602250pt}%
\definecolor{currentstroke}{rgb}{0.000000,0.000000,0.000000}%
\pgfsetstrokecolor{currentstroke}%
\pgfsetdash{}{0pt}%
\pgfsys@defobject{currentmarker}{\pgfqpoint{-0.027778in}{0.000000in}}{\pgfqpoint{0.000000in}{0.000000in}}{%
\pgfpathmoveto{\pgfqpoint{0.000000in}{0.000000in}}%
\pgfpathlineto{\pgfqpoint{-0.027778in}{0.000000in}}%
\pgfusepath{stroke,fill}%
}%
\begin{pgfscope}%
\pgfsys@transformshift{0.781944in}{4.379470in}%
\pgfsys@useobject{currentmarker}{}%
\end{pgfscope}%
\end{pgfscope}%
\begin{pgfscope}%
\pgfsetbuttcap%
\pgfsetroundjoin%
\definecolor{currentfill}{rgb}{0.000000,0.000000,0.000000}%
\pgfsetfillcolor{currentfill}%
\pgfsetlinewidth{0.602250pt}%
\definecolor{currentstroke}{rgb}{0.000000,0.000000,0.000000}%
\pgfsetstrokecolor{currentstroke}%
\pgfsetdash{}{0pt}%
\pgfsys@defobject{currentmarker}{\pgfqpoint{0.000000in}{0.000000in}}{\pgfqpoint{0.027778in}{0.000000in}}{%
\pgfpathmoveto{\pgfqpoint{0.000000in}{0.000000in}}%
\pgfpathlineto{\pgfqpoint{0.027778in}{0.000000in}}%
\pgfusepath{stroke,fill}%
}%
\begin{pgfscope}%
\pgfsys@transformshift{5.801389in}{4.379470in}%
\pgfsys@useobject{currentmarker}{}%
\end{pgfscope}%
\end{pgfscope}%
\begin{pgfscope}%
\pgfpathrectangle{\pgfqpoint{0.781944in}{2.977778in}}{\pgfqpoint{5.019444in}{1.650000in}}%
\pgfusepath{clip}%
\pgfsetrectcap%
\pgfsetroundjoin%
\pgfsetlinewidth{0.803000pt}%
\definecolor{currentstroke}{rgb}{0.690196,0.690196,0.690196}%
\pgfsetstrokecolor{currentstroke}%
\pgfsetstrokeopacity{0.300000}%
\pgfsetdash{}{0pt}%
\pgfpathmoveto{\pgfqpoint{0.781944in}{4.401372in}}%
\pgfpathlineto{\pgfqpoint{5.801389in}{4.401372in}}%
\pgfusepath{stroke}%
\end{pgfscope}%
\begin{pgfscope}%
\pgfsetbuttcap%
\pgfsetroundjoin%
\definecolor{currentfill}{rgb}{0.000000,0.000000,0.000000}%
\pgfsetfillcolor{currentfill}%
\pgfsetlinewidth{0.602250pt}%
\definecolor{currentstroke}{rgb}{0.000000,0.000000,0.000000}%
\pgfsetstrokecolor{currentstroke}%
\pgfsetdash{}{0pt}%
\pgfsys@defobject{currentmarker}{\pgfqpoint{-0.027778in}{0.000000in}}{\pgfqpoint{0.000000in}{0.000000in}}{%
\pgfpathmoveto{\pgfqpoint{0.000000in}{0.000000in}}%
\pgfpathlineto{\pgfqpoint{-0.027778in}{0.000000in}}%
\pgfusepath{stroke,fill}%
}%
\begin{pgfscope}%
\pgfsys@transformshift{0.781944in}{4.401372in}%
\pgfsys@useobject{currentmarker}{}%
\end{pgfscope}%
\end{pgfscope}%
\begin{pgfscope}%
\pgfsetbuttcap%
\pgfsetroundjoin%
\definecolor{currentfill}{rgb}{0.000000,0.000000,0.000000}%
\pgfsetfillcolor{currentfill}%
\pgfsetlinewidth{0.602250pt}%
\definecolor{currentstroke}{rgb}{0.000000,0.000000,0.000000}%
\pgfsetstrokecolor{currentstroke}%
\pgfsetdash{}{0pt}%
\pgfsys@defobject{currentmarker}{\pgfqpoint{0.000000in}{0.000000in}}{\pgfqpoint{0.027778in}{0.000000in}}{%
\pgfpathmoveto{\pgfqpoint{0.000000in}{0.000000in}}%
\pgfpathlineto{\pgfqpoint{0.027778in}{0.000000in}}%
\pgfusepath{stroke,fill}%
}%
\begin{pgfscope}%
\pgfsys@transformshift{5.801389in}{4.401372in}%
\pgfsys@useobject{currentmarker}{}%
\end{pgfscope}%
\end{pgfscope}%
\begin{pgfscope}%
\pgfpathrectangle{\pgfqpoint{0.781944in}{2.977778in}}{\pgfqpoint{5.019444in}{1.650000in}}%
\pgfusepath{clip}%
\pgfsetrectcap%
\pgfsetroundjoin%
\pgfsetlinewidth{0.803000pt}%
\definecolor{currentstroke}{rgb}{0.690196,0.690196,0.690196}%
\pgfsetstrokecolor{currentstroke}%
\pgfsetstrokeopacity{0.300000}%
\pgfsetdash{}{0pt}%
\pgfpathmoveto{\pgfqpoint{0.781944in}{4.423273in}}%
\pgfpathlineto{\pgfqpoint{5.801389in}{4.423273in}}%
\pgfusepath{stroke}%
\end{pgfscope}%
\begin{pgfscope}%
\pgfsetbuttcap%
\pgfsetroundjoin%
\definecolor{currentfill}{rgb}{0.000000,0.000000,0.000000}%
\pgfsetfillcolor{currentfill}%
\pgfsetlinewidth{0.602250pt}%
\definecolor{currentstroke}{rgb}{0.000000,0.000000,0.000000}%
\pgfsetstrokecolor{currentstroke}%
\pgfsetdash{}{0pt}%
\pgfsys@defobject{currentmarker}{\pgfqpoint{-0.027778in}{0.000000in}}{\pgfqpoint{0.000000in}{0.000000in}}{%
\pgfpathmoveto{\pgfqpoint{0.000000in}{0.000000in}}%
\pgfpathlineto{\pgfqpoint{-0.027778in}{0.000000in}}%
\pgfusepath{stroke,fill}%
}%
\begin{pgfscope}%
\pgfsys@transformshift{0.781944in}{4.423273in}%
\pgfsys@useobject{currentmarker}{}%
\end{pgfscope}%
\end{pgfscope}%
\begin{pgfscope}%
\pgfsetbuttcap%
\pgfsetroundjoin%
\definecolor{currentfill}{rgb}{0.000000,0.000000,0.000000}%
\pgfsetfillcolor{currentfill}%
\pgfsetlinewidth{0.602250pt}%
\definecolor{currentstroke}{rgb}{0.000000,0.000000,0.000000}%
\pgfsetstrokecolor{currentstroke}%
\pgfsetdash{}{0pt}%
\pgfsys@defobject{currentmarker}{\pgfqpoint{0.000000in}{0.000000in}}{\pgfqpoint{0.027778in}{0.000000in}}{%
\pgfpathmoveto{\pgfqpoint{0.000000in}{0.000000in}}%
\pgfpathlineto{\pgfqpoint{0.027778in}{0.000000in}}%
\pgfusepath{stroke,fill}%
}%
\begin{pgfscope}%
\pgfsys@transformshift{5.801389in}{4.423273in}%
\pgfsys@useobject{currentmarker}{}%
\end{pgfscope}%
\end{pgfscope}%
\begin{pgfscope}%
\pgfpathrectangle{\pgfqpoint{0.781944in}{2.977778in}}{\pgfqpoint{5.019444in}{1.650000in}}%
\pgfusepath{clip}%
\pgfsetrectcap%
\pgfsetroundjoin%
\pgfsetlinewidth{0.803000pt}%
\definecolor{currentstroke}{rgb}{0.690196,0.690196,0.690196}%
\pgfsetstrokecolor{currentstroke}%
\pgfsetstrokeopacity{0.300000}%
\pgfsetdash{}{0pt}%
\pgfpathmoveto{\pgfqpoint{0.781944in}{4.445174in}}%
\pgfpathlineto{\pgfqpoint{5.801389in}{4.445174in}}%
\pgfusepath{stroke}%
\end{pgfscope}%
\begin{pgfscope}%
\pgfsetbuttcap%
\pgfsetroundjoin%
\definecolor{currentfill}{rgb}{0.000000,0.000000,0.000000}%
\pgfsetfillcolor{currentfill}%
\pgfsetlinewidth{0.602250pt}%
\definecolor{currentstroke}{rgb}{0.000000,0.000000,0.000000}%
\pgfsetstrokecolor{currentstroke}%
\pgfsetdash{}{0pt}%
\pgfsys@defobject{currentmarker}{\pgfqpoint{-0.027778in}{0.000000in}}{\pgfqpoint{0.000000in}{0.000000in}}{%
\pgfpathmoveto{\pgfqpoint{0.000000in}{0.000000in}}%
\pgfpathlineto{\pgfqpoint{-0.027778in}{0.000000in}}%
\pgfusepath{stroke,fill}%
}%
\begin{pgfscope}%
\pgfsys@transformshift{0.781944in}{4.445174in}%
\pgfsys@useobject{currentmarker}{}%
\end{pgfscope}%
\end{pgfscope}%
\begin{pgfscope}%
\pgfsetbuttcap%
\pgfsetroundjoin%
\definecolor{currentfill}{rgb}{0.000000,0.000000,0.000000}%
\pgfsetfillcolor{currentfill}%
\pgfsetlinewidth{0.602250pt}%
\definecolor{currentstroke}{rgb}{0.000000,0.000000,0.000000}%
\pgfsetstrokecolor{currentstroke}%
\pgfsetdash{}{0pt}%
\pgfsys@defobject{currentmarker}{\pgfqpoint{0.000000in}{0.000000in}}{\pgfqpoint{0.027778in}{0.000000in}}{%
\pgfpathmoveto{\pgfqpoint{0.000000in}{0.000000in}}%
\pgfpathlineto{\pgfqpoint{0.027778in}{0.000000in}}%
\pgfusepath{stroke,fill}%
}%
\begin{pgfscope}%
\pgfsys@transformshift{5.801389in}{4.445174in}%
\pgfsys@useobject{currentmarker}{}%
\end{pgfscope}%
\end{pgfscope}%
\begin{pgfscope}%
\pgfpathrectangle{\pgfqpoint{0.781944in}{2.977778in}}{\pgfqpoint{5.019444in}{1.650000in}}%
\pgfusepath{clip}%
\pgfsetrectcap%
\pgfsetroundjoin%
\pgfsetlinewidth{0.803000pt}%
\definecolor{currentstroke}{rgb}{0.690196,0.690196,0.690196}%
\pgfsetstrokecolor{currentstroke}%
\pgfsetstrokeopacity{0.300000}%
\pgfsetdash{}{0pt}%
\pgfpathmoveto{\pgfqpoint{0.781944in}{4.467076in}}%
\pgfpathlineto{\pgfqpoint{5.801389in}{4.467076in}}%
\pgfusepath{stroke}%
\end{pgfscope}%
\begin{pgfscope}%
\pgfsetbuttcap%
\pgfsetroundjoin%
\definecolor{currentfill}{rgb}{0.000000,0.000000,0.000000}%
\pgfsetfillcolor{currentfill}%
\pgfsetlinewidth{0.602250pt}%
\definecolor{currentstroke}{rgb}{0.000000,0.000000,0.000000}%
\pgfsetstrokecolor{currentstroke}%
\pgfsetdash{}{0pt}%
\pgfsys@defobject{currentmarker}{\pgfqpoint{-0.027778in}{0.000000in}}{\pgfqpoint{0.000000in}{0.000000in}}{%
\pgfpathmoveto{\pgfqpoint{0.000000in}{0.000000in}}%
\pgfpathlineto{\pgfqpoint{-0.027778in}{0.000000in}}%
\pgfusepath{stroke,fill}%
}%
\begin{pgfscope}%
\pgfsys@transformshift{0.781944in}{4.467076in}%
\pgfsys@useobject{currentmarker}{}%
\end{pgfscope}%
\end{pgfscope}%
\begin{pgfscope}%
\pgfsetbuttcap%
\pgfsetroundjoin%
\definecolor{currentfill}{rgb}{0.000000,0.000000,0.000000}%
\pgfsetfillcolor{currentfill}%
\pgfsetlinewidth{0.602250pt}%
\definecolor{currentstroke}{rgb}{0.000000,0.000000,0.000000}%
\pgfsetstrokecolor{currentstroke}%
\pgfsetdash{}{0pt}%
\pgfsys@defobject{currentmarker}{\pgfqpoint{0.000000in}{0.000000in}}{\pgfqpoint{0.027778in}{0.000000in}}{%
\pgfpathmoveto{\pgfqpoint{0.000000in}{0.000000in}}%
\pgfpathlineto{\pgfqpoint{0.027778in}{0.000000in}}%
\pgfusepath{stroke,fill}%
}%
\begin{pgfscope}%
\pgfsys@transformshift{5.801389in}{4.467076in}%
\pgfsys@useobject{currentmarker}{}%
\end{pgfscope}%
\end{pgfscope}%
\begin{pgfscope}%
\pgfpathrectangle{\pgfqpoint{0.781944in}{2.977778in}}{\pgfqpoint{5.019444in}{1.650000in}}%
\pgfusepath{clip}%
\pgfsetrectcap%
\pgfsetroundjoin%
\pgfsetlinewidth{0.803000pt}%
\definecolor{currentstroke}{rgb}{0.690196,0.690196,0.690196}%
\pgfsetstrokecolor{currentstroke}%
\pgfsetstrokeopacity{0.300000}%
\pgfsetdash{}{0pt}%
\pgfpathmoveto{\pgfqpoint{0.781944in}{4.488977in}}%
\pgfpathlineto{\pgfqpoint{5.801389in}{4.488977in}}%
\pgfusepath{stroke}%
\end{pgfscope}%
\begin{pgfscope}%
\pgfsetbuttcap%
\pgfsetroundjoin%
\definecolor{currentfill}{rgb}{0.000000,0.000000,0.000000}%
\pgfsetfillcolor{currentfill}%
\pgfsetlinewidth{0.602250pt}%
\definecolor{currentstroke}{rgb}{0.000000,0.000000,0.000000}%
\pgfsetstrokecolor{currentstroke}%
\pgfsetdash{}{0pt}%
\pgfsys@defobject{currentmarker}{\pgfqpoint{-0.027778in}{0.000000in}}{\pgfqpoint{0.000000in}{0.000000in}}{%
\pgfpathmoveto{\pgfqpoint{0.000000in}{0.000000in}}%
\pgfpathlineto{\pgfqpoint{-0.027778in}{0.000000in}}%
\pgfusepath{stroke,fill}%
}%
\begin{pgfscope}%
\pgfsys@transformshift{0.781944in}{4.488977in}%
\pgfsys@useobject{currentmarker}{}%
\end{pgfscope}%
\end{pgfscope}%
\begin{pgfscope}%
\pgfsetbuttcap%
\pgfsetroundjoin%
\definecolor{currentfill}{rgb}{0.000000,0.000000,0.000000}%
\pgfsetfillcolor{currentfill}%
\pgfsetlinewidth{0.602250pt}%
\definecolor{currentstroke}{rgb}{0.000000,0.000000,0.000000}%
\pgfsetstrokecolor{currentstroke}%
\pgfsetdash{}{0pt}%
\pgfsys@defobject{currentmarker}{\pgfqpoint{0.000000in}{0.000000in}}{\pgfqpoint{0.027778in}{0.000000in}}{%
\pgfpathmoveto{\pgfqpoint{0.000000in}{0.000000in}}%
\pgfpathlineto{\pgfqpoint{0.027778in}{0.000000in}}%
\pgfusepath{stroke,fill}%
}%
\begin{pgfscope}%
\pgfsys@transformshift{5.801389in}{4.488977in}%
\pgfsys@useobject{currentmarker}{}%
\end{pgfscope}%
\end{pgfscope}%
\begin{pgfscope}%
\pgfpathrectangle{\pgfqpoint{0.781944in}{2.977778in}}{\pgfqpoint{5.019444in}{1.650000in}}%
\pgfusepath{clip}%
\pgfsetrectcap%
\pgfsetroundjoin%
\pgfsetlinewidth{0.803000pt}%
\definecolor{currentstroke}{rgb}{0.690196,0.690196,0.690196}%
\pgfsetstrokecolor{currentstroke}%
\pgfsetstrokeopacity{0.300000}%
\pgfsetdash{}{0pt}%
\pgfpathmoveto{\pgfqpoint{0.781944in}{4.532780in}}%
\pgfpathlineto{\pgfqpoint{5.801389in}{4.532780in}}%
\pgfusepath{stroke}%
\end{pgfscope}%
\begin{pgfscope}%
\pgfsetbuttcap%
\pgfsetroundjoin%
\definecolor{currentfill}{rgb}{0.000000,0.000000,0.000000}%
\pgfsetfillcolor{currentfill}%
\pgfsetlinewidth{0.602250pt}%
\definecolor{currentstroke}{rgb}{0.000000,0.000000,0.000000}%
\pgfsetstrokecolor{currentstroke}%
\pgfsetdash{}{0pt}%
\pgfsys@defobject{currentmarker}{\pgfqpoint{-0.027778in}{0.000000in}}{\pgfqpoint{0.000000in}{0.000000in}}{%
\pgfpathmoveto{\pgfqpoint{0.000000in}{0.000000in}}%
\pgfpathlineto{\pgfqpoint{-0.027778in}{0.000000in}}%
\pgfusepath{stroke,fill}%
}%
\begin{pgfscope}%
\pgfsys@transformshift{0.781944in}{4.532780in}%
\pgfsys@useobject{currentmarker}{}%
\end{pgfscope}%
\end{pgfscope}%
\begin{pgfscope}%
\pgfsetbuttcap%
\pgfsetroundjoin%
\definecolor{currentfill}{rgb}{0.000000,0.000000,0.000000}%
\pgfsetfillcolor{currentfill}%
\pgfsetlinewidth{0.602250pt}%
\definecolor{currentstroke}{rgb}{0.000000,0.000000,0.000000}%
\pgfsetstrokecolor{currentstroke}%
\pgfsetdash{}{0pt}%
\pgfsys@defobject{currentmarker}{\pgfqpoint{0.000000in}{0.000000in}}{\pgfqpoint{0.027778in}{0.000000in}}{%
\pgfpathmoveto{\pgfqpoint{0.000000in}{0.000000in}}%
\pgfpathlineto{\pgfqpoint{0.027778in}{0.000000in}}%
\pgfusepath{stroke,fill}%
}%
\begin{pgfscope}%
\pgfsys@transformshift{5.801389in}{4.532780in}%
\pgfsys@useobject{currentmarker}{}%
\end{pgfscope}%
\end{pgfscope}%
\begin{pgfscope}%
\pgfpathrectangle{\pgfqpoint{0.781944in}{2.977778in}}{\pgfqpoint{5.019444in}{1.650000in}}%
\pgfusepath{clip}%
\pgfsetrectcap%
\pgfsetroundjoin%
\pgfsetlinewidth{0.803000pt}%
\definecolor{currentstroke}{rgb}{0.690196,0.690196,0.690196}%
\pgfsetstrokecolor{currentstroke}%
\pgfsetstrokeopacity{0.300000}%
\pgfsetdash{}{0pt}%
\pgfpathmoveto{\pgfqpoint{0.781944in}{4.554682in}}%
\pgfpathlineto{\pgfqpoint{5.801389in}{4.554682in}}%
\pgfusepath{stroke}%
\end{pgfscope}%
\begin{pgfscope}%
\pgfsetbuttcap%
\pgfsetroundjoin%
\definecolor{currentfill}{rgb}{0.000000,0.000000,0.000000}%
\pgfsetfillcolor{currentfill}%
\pgfsetlinewidth{0.602250pt}%
\definecolor{currentstroke}{rgb}{0.000000,0.000000,0.000000}%
\pgfsetstrokecolor{currentstroke}%
\pgfsetdash{}{0pt}%
\pgfsys@defobject{currentmarker}{\pgfqpoint{-0.027778in}{0.000000in}}{\pgfqpoint{0.000000in}{0.000000in}}{%
\pgfpathmoveto{\pgfqpoint{0.000000in}{0.000000in}}%
\pgfpathlineto{\pgfqpoint{-0.027778in}{0.000000in}}%
\pgfusepath{stroke,fill}%
}%
\begin{pgfscope}%
\pgfsys@transformshift{0.781944in}{4.554682in}%
\pgfsys@useobject{currentmarker}{}%
\end{pgfscope}%
\end{pgfscope}%
\begin{pgfscope}%
\pgfsetbuttcap%
\pgfsetroundjoin%
\definecolor{currentfill}{rgb}{0.000000,0.000000,0.000000}%
\pgfsetfillcolor{currentfill}%
\pgfsetlinewidth{0.602250pt}%
\definecolor{currentstroke}{rgb}{0.000000,0.000000,0.000000}%
\pgfsetstrokecolor{currentstroke}%
\pgfsetdash{}{0pt}%
\pgfsys@defobject{currentmarker}{\pgfqpoint{0.000000in}{0.000000in}}{\pgfqpoint{0.027778in}{0.000000in}}{%
\pgfpathmoveto{\pgfqpoint{0.000000in}{0.000000in}}%
\pgfpathlineto{\pgfqpoint{0.027778in}{0.000000in}}%
\pgfusepath{stroke,fill}%
}%
\begin{pgfscope}%
\pgfsys@transformshift{5.801389in}{4.554682in}%
\pgfsys@useobject{currentmarker}{}%
\end{pgfscope}%
\end{pgfscope}%
\begin{pgfscope}%
\pgfpathrectangle{\pgfqpoint{0.781944in}{2.977778in}}{\pgfqpoint{5.019444in}{1.650000in}}%
\pgfusepath{clip}%
\pgfsetrectcap%
\pgfsetroundjoin%
\pgfsetlinewidth{0.803000pt}%
\definecolor{currentstroke}{rgb}{0.690196,0.690196,0.690196}%
\pgfsetstrokecolor{currentstroke}%
\pgfsetstrokeopacity{0.300000}%
\pgfsetdash{}{0pt}%
\pgfpathmoveto{\pgfqpoint{0.781944in}{4.576583in}}%
\pgfpathlineto{\pgfqpoint{5.801389in}{4.576583in}}%
\pgfusepath{stroke}%
\end{pgfscope}%
\begin{pgfscope}%
\pgfsetbuttcap%
\pgfsetroundjoin%
\definecolor{currentfill}{rgb}{0.000000,0.000000,0.000000}%
\pgfsetfillcolor{currentfill}%
\pgfsetlinewidth{0.602250pt}%
\definecolor{currentstroke}{rgb}{0.000000,0.000000,0.000000}%
\pgfsetstrokecolor{currentstroke}%
\pgfsetdash{}{0pt}%
\pgfsys@defobject{currentmarker}{\pgfqpoint{-0.027778in}{0.000000in}}{\pgfqpoint{0.000000in}{0.000000in}}{%
\pgfpathmoveto{\pgfqpoint{0.000000in}{0.000000in}}%
\pgfpathlineto{\pgfqpoint{-0.027778in}{0.000000in}}%
\pgfusepath{stroke,fill}%
}%
\begin{pgfscope}%
\pgfsys@transformshift{0.781944in}{4.576583in}%
\pgfsys@useobject{currentmarker}{}%
\end{pgfscope}%
\end{pgfscope}%
\begin{pgfscope}%
\pgfsetbuttcap%
\pgfsetroundjoin%
\definecolor{currentfill}{rgb}{0.000000,0.000000,0.000000}%
\pgfsetfillcolor{currentfill}%
\pgfsetlinewidth{0.602250pt}%
\definecolor{currentstroke}{rgb}{0.000000,0.000000,0.000000}%
\pgfsetstrokecolor{currentstroke}%
\pgfsetdash{}{0pt}%
\pgfsys@defobject{currentmarker}{\pgfqpoint{0.000000in}{0.000000in}}{\pgfqpoint{0.027778in}{0.000000in}}{%
\pgfpathmoveto{\pgfqpoint{0.000000in}{0.000000in}}%
\pgfpathlineto{\pgfqpoint{0.027778in}{0.000000in}}%
\pgfusepath{stroke,fill}%
}%
\begin{pgfscope}%
\pgfsys@transformshift{5.801389in}{4.576583in}%
\pgfsys@useobject{currentmarker}{}%
\end{pgfscope}%
\end{pgfscope}%
\begin{pgfscope}%
\pgfpathrectangle{\pgfqpoint{0.781944in}{2.977778in}}{\pgfqpoint{5.019444in}{1.650000in}}%
\pgfusepath{clip}%
\pgfsetrectcap%
\pgfsetroundjoin%
\pgfsetlinewidth{0.803000pt}%
\definecolor{currentstroke}{rgb}{0.690196,0.690196,0.690196}%
\pgfsetstrokecolor{currentstroke}%
\pgfsetstrokeopacity{0.300000}%
\pgfsetdash{}{0pt}%
\pgfpathmoveto{\pgfqpoint{0.781944in}{4.598485in}}%
\pgfpathlineto{\pgfqpoint{5.801389in}{4.598485in}}%
\pgfusepath{stroke}%
\end{pgfscope}%
\begin{pgfscope}%
\pgfsetbuttcap%
\pgfsetroundjoin%
\definecolor{currentfill}{rgb}{0.000000,0.000000,0.000000}%
\pgfsetfillcolor{currentfill}%
\pgfsetlinewidth{0.602250pt}%
\definecolor{currentstroke}{rgb}{0.000000,0.000000,0.000000}%
\pgfsetstrokecolor{currentstroke}%
\pgfsetdash{}{0pt}%
\pgfsys@defobject{currentmarker}{\pgfqpoint{-0.027778in}{0.000000in}}{\pgfqpoint{0.000000in}{0.000000in}}{%
\pgfpathmoveto{\pgfqpoint{0.000000in}{0.000000in}}%
\pgfpathlineto{\pgfqpoint{-0.027778in}{0.000000in}}%
\pgfusepath{stroke,fill}%
}%
\begin{pgfscope}%
\pgfsys@transformshift{0.781944in}{4.598485in}%
\pgfsys@useobject{currentmarker}{}%
\end{pgfscope}%
\end{pgfscope}%
\begin{pgfscope}%
\pgfsetbuttcap%
\pgfsetroundjoin%
\definecolor{currentfill}{rgb}{0.000000,0.000000,0.000000}%
\pgfsetfillcolor{currentfill}%
\pgfsetlinewidth{0.602250pt}%
\definecolor{currentstroke}{rgb}{0.000000,0.000000,0.000000}%
\pgfsetstrokecolor{currentstroke}%
\pgfsetdash{}{0pt}%
\pgfsys@defobject{currentmarker}{\pgfqpoint{0.000000in}{0.000000in}}{\pgfqpoint{0.027778in}{0.000000in}}{%
\pgfpathmoveto{\pgfqpoint{0.000000in}{0.000000in}}%
\pgfpathlineto{\pgfqpoint{0.027778in}{0.000000in}}%
\pgfusepath{stroke,fill}%
}%
\begin{pgfscope}%
\pgfsys@transformshift{5.801389in}{4.598485in}%
\pgfsys@useobject{currentmarker}{}%
\end{pgfscope}%
\end{pgfscope}%
\begin{pgfscope}%
\pgfpathrectangle{\pgfqpoint{0.781944in}{2.977778in}}{\pgfqpoint{5.019444in}{1.650000in}}%
\pgfusepath{clip}%
\pgfsetrectcap%
\pgfsetroundjoin%
\pgfsetlinewidth{0.803000pt}%
\definecolor{currentstroke}{rgb}{0.690196,0.690196,0.690196}%
\pgfsetstrokecolor{currentstroke}%
\pgfsetstrokeopacity{0.300000}%
\pgfsetdash{}{0pt}%
\pgfpathmoveto{\pgfqpoint{0.781944in}{4.620386in}}%
\pgfpathlineto{\pgfqpoint{5.801389in}{4.620386in}}%
\pgfusepath{stroke}%
\end{pgfscope}%
\begin{pgfscope}%
\pgfsetbuttcap%
\pgfsetroundjoin%
\definecolor{currentfill}{rgb}{0.000000,0.000000,0.000000}%
\pgfsetfillcolor{currentfill}%
\pgfsetlinewidth{0.602250pt}%
\definecolor{currentstroke}{rgb}{0.000000,0.000000,0.000000}%
\pgfsetstrokecolor{currentstroke}%
\pgfsetdash{}{0pt}%
\pgfsys@defobject{currentmarker}{\pgfqpoint{-0.027778in}{0.000000in}}{\pgfqpoint{0.000000in}{0.000000in}}{%
\pgfpathmoveto{\pgfqpoint{0.000000in}{0.000000in}}%
\pgfpathlineto{\pgfqpoint{-0.027778in}{0.000000in}}%
\pgfusepath{stroke,fill}%
}%
\begin{pgfscope}%
\pgfsys@transformshift{0.781944in}{4.620386in}%
\pgfsys@useobject{currentmarker}{}%
\end{pgfscope}%
\end{pgfscope}%
\begin{pgfscope}%
\pgfsetbuttcap%
\pgfsetroundjoin%
\definecolor{currentfill}{rgb}{0.000000,0.000000,0.000000}%
\pgfsetfillcolor{currentfill}%
\pgfsetlinewidth{0.602250pt}%
\definecolor{currentstroke}{rgb}{0.000000,0.000000,0.000000}%
\pgfsetstrokecolor{currentstroke}%
\pgfsetdash{}{0pt}%
\pgfsys@defobject{currentmarker}{\pgfqpoint{0.000000in}{0.000000in}}{\pgfqpoint{0.027778in}{0.000000in}}{%
\pgfpathmoveto{\pgfqpoint{0.000000in}{0.000000in}}%
\pgfpathlineto{\pgfqpoint{0.027778in}{0.000000in}}%
\pgfusepath{stroke,fill}%
}%
\begin{pgfscope}%
\pgfsys@transformshift{5.801389in}{4.620386in}%
\pgfsys@useobject{currentmarker}{}%
\end{pgfscope}%
\end{pgfscope}%
\begin{pgfscope}%
\definecolor{textcolor}{rgb}{0.000000,0.000000,0.000000}%
\pgfsetstrokecolor{textcolor}%
\pgfsetfillcolor{textcolor}%
\pgftext[x=0.351389in,y=3.802778in,,bottom,rotate=90.000000]{\color{textcolor}\rmfamily\fontsize{10.000000}{12.000000}\selectfont Ereignisszahl}%
\end{pgfscope}%
\begin{pgfscope}%
\pgfpathrectangle{\pgfqpoint{0.781944in}{2.977778in}}{\pgfqpoint{5.019444in}{1.650000in}}%
\pgfusepath{clip}%
\pgfsetrectcap%
\pgfsetroundjoin%
\pgfsetlinewidth{1.505625pt}%
\definecolor{currentstroke}{rgb}{0.121569,0.466667,0.705882}%
\pgfsetstrokecolor{currentstroke}%
\pgfsetdash{}{0pt}%
\pgfpathmoveto{\pgfqpoint{0.776442in}{3.000227in}}%
\pgfpathlineto{\pgfqpoint{0.776442in}{2.998037in}}%
\pgfpathlineto{\pgfqpoint{0.782503in}{2.998037in}}%
\pgfpathlineto{\pgfqpoint{0.782503in}{2.994751in}}%
\pgfpathlineto{\pgfqpoint{0.788564in}{2.994751in}}%
\pgfpathlineto{\pgfqpoint{0.788564in}{3.000227in}}%
\pgfpathlineto{\pgfqpoint{0.794625in}{3.000227in}}%
\pgfpathlineto{\pgfqpoint{0.794625in}{2.992561in}}%
\pgfpathlineto{\pgfqpoint{0.812808in}{2.993656in}}%
\pgfpathlineto{\pgfqpoint{0.812808in}{2.997489in}}%
\pgfpathlineto{\pgfqpoint{0.824930in}{2.996942in}}%
\pgfpathlineto{\pgfqpoint{0.824930in}{2.999132in}}%
\pgfpathlineto{\pgfqpoint{0.830991in}{2.999132in}}%
\pgfpathlineto{\pgfqpoint{0.830991in}{2.996394in}}%
\pgfpathlineto{\pgfqpoint{0.837052in}{2.996394in}}%
\pgfpathlineto{\pgfqpoint{0.837052in}{3.001322in}}%
\pgfpathlineto{\pgfqpoint{0.849174in}{3.000774in}}%
\pgfpathlineto{\pgfqpoint{0.849174in}{2.997489in}}%
\pgfpathlineto{\pgfqpoint{0.861296in}{2.996942in}}%
\pgfpathlineto{\pgfqpoint{0.861296in}{2.995299in}}%
\pgfpathlineto{\pgfqpoint{0.867357in}{2.995299in}}%
\pgfpathlineto{\pgfqpoint{0.867357in}{3.001322in}}%
\pgfpathlineto{\pgfqpoint{0.873418in}{3.001322in}}%
\pgfpathlineto{\pgfqpoint{0.873418in}{2.993656in}}%
\pgfpathlineto{\pgfqpoint{0.879479in}{2.993656in}}%
\pgfpathlineto{\pgfqpoint{0.879479in}{3.000774in}}%
\pgfpathlineto{\pgfqpoint{0.885540in}{3.000774in}}%
\pgfpathlineto{\pgfqpoint{0.885540in}{2.994751in}}%
\pgfpathlineto{\pgfqpoint{0.891600in}{2.994751in}}%
\pgfpathlineto{\pgfqpoint{0.891600in}{2.998037in}}%
\pgfpathlineto{\pgfqpoint{0.897661in}{2.998037in}}%
\pgfpathlineto{\pgfqpoint{0.897661in}{3.000227in}}%
\pgfpathlineto{\pgfqpoint{0.903722in}{3.000227in}}%
\pgfpathlineto{\pgfqpoint{0.903722in}{3.004060in}}%
\pgfpathlineto{\pgfqpoint{0.909783in}{3.004060in}}%
\pgfpathlineto{\pgfqpoint{0.909783in}{2.996394in}}%
\pgfpathlineto{\pgfqpoint{0.934027in}{2.996942in}}%
\pgfpathlineto{\pgfqpoint{0.934027in}{2.992014in}}%
\pgfpathlineto{\pgfqpoint{0.940088in}{2.992014in}}%
\pgfpathlineto{\pgfqpoint{0.940088in}{2.999679in}}%
\pgfpathlineto{\pgfqpoint{0.952210in}{2.999132in}}%
\pgfpathlineto{\pgfqpoint{0.952210in}{2.993656in}}%
\pgfpathlineto{\pgfqpoint{0.958271in}{2.993656in}}%
\pgfpathlineto{\pgfqpoint{0.958271in}{2.995299in}}%
\pgfpathlineto{\pgfqpoint{0.964332in}{2.995299in}}%
\pgfpathlineto{\pgfqpoint{0.964332in}{3.008440in}}%
\pgfpathlineto{\pgfqpoint{0.970393in}{3.008440in}}%
\pgfpathlineto{\pgfqpoint{0.970393in}{2.996942in}}%
\pgfpathlineto{\pgfqpoint{0.976454in}{2.996942in}}%
\pgfpathlineto{\pgfqpoint{0.976454in}{3.002964in}}%
\pgfpathlineto{\pgfqpoint{0.982515in}{3.002964in}}%
\pgfpathlineto{\pgfqpoint{0.982515in}{2.990371in}}%
\pgfpathlineto{\pgfqpoint{0.988576in}{2.990371in}}%
\pgfpathlineto{\pgfqpoint{0.988576in}{3.004060in}}%
\pgfpathlineto{\pgfqpoint{0.994637in}{3.004060in}}%
\pgfpathlineto{\pgfqpoint{0.994637in}{2.999132in}}%
\pgfpathlineto{\pgfqpoint{1.024942in}{2.998037in}}%
\pgfpathlineto{\pgfqpoint{1.024942in}{2.992561in}}%
\pgfpathlineto{\pgfqpoint{1.031003in}{2.992561in}}%
\pgfpathlineto{\pgfqpoint{1.031003in}{3.001869in}}%
\pgfpathlineto{\pgfqpoint{1.049186in}{3.000774in}}%
\pgfpathlineto{\pgfqpoint{1.049186in}{2.993656in}}%
\pgfpathlineto{\pgfqpoint{1.055247in}{2.993656in}}%
\pgfpathlineto{\pgfqpoint{1.055247in}{3.001322in}}%
\pgfpathlineto{\pgfqpoint{1.061308in}{3.001322in}}%
\pgfpathlineto{\pgfqpoint{1.061308in}{2.997489in}}%
\pgfpathlineto{\pgfqpoint{1.067369in}{2.997489in}}%
\pgfpathlineto{\pgfqpoint{1.067369in}{2.999679in}}%
\pgfpathlineto{\pgfqpoint{1.079491in}{3.000227in}}%
\pgfpathlineto{\pgfqpoint{1.079491in}{2.993656in}}%
\pgfpathlineto{\pgfqpoint{1.085552in}{2.993656in}}%
\pgfpathlineto{\pgfqpoint{1.085552in}{3.001869in}}%
\pgfpathlineto{\pgfqpoint{1.091613in}{3.001869in}}%
\pgfpathlineto{\pgfqpoint{1.091613in}{2.998584in}}%
\pgfpathlineto{\pgfqpoint{1.097674in}{2.998584in}}%
\pgfpathlineto{\pgfqpoint{1.097674in}{3.001869in}}%
\pgfpathlineto{\pgfqpoint{1.103735in}{3.001869in}}%
\pgfpathlineto{\pgfqpoint{1.103735in}{2.994751in}}%
\pgfpathlineto{\pgfqpoint{1.109796in}{2.994751in}}%
\pgfpathlineto{\pgfqpoint{1.109796in}{2.999679in}}%
\pgfpathlineto{\pgfqpoint{1.127979in}{3.000774in}}%
\pgfpathlineto{\pgfqpoint{1.127979in}{3.008440in}}%
\pgfpathlineto{\pgfqpoint{1.134040in}{3.008440in}}%
\pgfpathlineto{\pgfqpoint{1.134040in}{2.998037in}}%
\pgfpathlineto{\pgfqpoint{1.152223in}{2.998037in}}%
\pgfpathlineto{\pgfqpoint{1.152223in}{3.007345in}}%
\pgfpathlineto{\pgfqpoint{1.158284in}{3.007345in}}%
\pgfpathlineto{\pgfqpoint{1.158284in}{2.996942in}}%
\pgfpathlineto{\pgfqpoint{1.164345in}{2.996942in}}%
\pgfpathlineto{\pgfqpoint{1.164345in}{2.994751in}}%
\pgfpathlineto{\pgfqpoint{1.170406in}{2.994751in}}%
\pgfpathlineto{\pgfqpoint{1.170406in}{3.007892in}}%
\pgfpathlineto{\pgfqpoint{1.176467in}{3.007892in}}%
\pgfpathlineto{\pgfqpoint{1.176467in}{3.000227in}}%
\pgfpathlineto{\pgfqpoint{1.182527in}{3.000227in}}%
\pgfpathlineto{\pgfqpoint{1.182527in}{3.001869in}}%
\pgfpathlineto{\pgfqpoint{1.194649in}{3.001322in}}%
\pgfpathlineto{\pgfqpoint{1.194649in}{2.999132in}}%
\pgfpathlineto{\pgfqpoint{1.200710in}{2.999132in}}%
\pgfpathlineto{\pgfqpoint{1.200710in}{3.001322in}}%
\pgfpathlineto{\pgfqpoint{1.212832in}{3.000227in}}%
\pgfpathlineto{\pgfqpoint{1.212832in}{2.999132in}}%
\pgfpathlineto{\pgfqpoint{1.218893in}{2.999132in}}%
\pgfpathlineto{\pgfqpoint{1.218893in}{3.005702in}}%
\pgfpathlineto{\pgfqpoint{1.224954in}{3.005702in}}%
\pgfpathlineto{\pgfqpoint{1.224954in}{3.002417in}}%
\pgfpathlineto{\pgfqpoint{1.231015in}{3.002417in}}%
\pgfpathlineto{\pgfqpoint{1.231015in}{2.996942in}}%
\pgfpathlineto{\pgfqpoint{1.237076in}{2.996942in}}%
\pgfpathlineto{\pgfqpoint{1.237076in}{3.006250in}}%
\pgfpathlineto{\pgfqpoint{1.243137in}{3.006250in}}%
\pgfpathlineto{\pgfqpoint{1.243137in}{3.000227in}}%
\pgfpathlineto{\pgfqpoint{1.249198in}{3.000227in}}%
\pgfpathlineto{\pgfqpoint{1.249198in}{3.007345in}}%
\pgfpathlineto{\pgfqpoint{1.255259in}{3.007345in}}%
\pgfpathlineto{\pgfqpoint{1.255259in}{2.999132in}}%
\pgfpathlineto{\pgfqpoint{1.261320in}{2.999132in}}%
\pgfpathlineto{\pgfqpoint{1.261320in}{3.006250in}}%
\pgfpathlineto{\pgfqpoint{1.267381in}{3.006250in}}%
\pgfpathlineto{\pgfqpoint{1.267381in}{2.997489in}}%
\pgfpathlineto{\pgfqpoint{1.273442in}{2.997489in}}%
\pgfpathlineto{\pgfqpoint{1.273442in}{3.006250in}}%
\pgfpathlineto{\pgfqpoint{1.279503in}{3.006250in}}%
\pgfpathlineto{\pgfqpoint{1.279503in}{3.003512in}}%
\pgfpathlineto{\pgfqpoint{1.285564in}{3.003512in}}%
\pgfpathlineto{\pgfqpoint{1.285564in}{3.005702in}}%
\pgfpathlineto{\pgfqpoint{1.291625in}{3.005702in}}%
\pgfpathlineto{\pgfqpoint{1.291625in}{3.000774in}}%
\pgfpathlineto{\pgfqpoint{1.297686in}{3.000774in}}%
\pgfpathlineto{\pgfqpoint{1.297686in}{2.995846in}}%
\pgfpathlineto{\pgfqpoint{1.303747in}{2.995846in}}%
\pgfpathlineto{\pgfqpoint{1.303747in}{3.004060in}}%
\pgfpathlineto{\pgfqpoint{1.309808in}{3.004060in}}%
\pgfpathlineto{\pgfqpoint{1.309808in}{2.998037in}}%
\pgfpathlineto{\pgfqpoint{1.327991in}{2.998037in}}%
\pgfpathlineto{\pgfqpoint{1.327991in}{3.006250in}}%
\pgfpathlineto{\pgfqpoint{1.334052in}{3.006250in}}%
\pgfpathlineto{\pgfqpoint{1.334052in}{2.999132in}}%
\pgfpathlineto{\pgfqpoint{1.346174in}{2.999679in}}%
\pgfpathlineto{\pgfqpoint{1.346174in}{3.001869in}}%
\pgfpathlineto{\pgfqpoint{1.352235in}{3.001869in}}%
\pgfpathlineto{\pgfqpoint{1.352235in}{2.995299in}}%
\pgfpathlineto{\pgfqpoint{1.358296in}{2.995299in}}%
\pgfpathlineto{\pgfqpoint{1.358296in}{3.005702in}}%
\pgfpathlineto{\pgfqpoint{1.364357in}{3.005702in}}%
\pgfpathlineto{\pgfqpoint{1.364357in}{2.997489in}}%
\pgfpathlineto{\pgfqpoint{1.370418in}{2.997489in}}%
\pgfpathlineto{\pgfqpoint{1.370418in}{3.009535in}}%
\pgfpathlineto{\pgfqpoint{1.382540in}{3.008440in}}%
\pgfpathlineto{\pgfqpoint{1.382540in}{3.011725in}}%
\pgfpathlineto{\pgfqpoint{1.388601in}{3.011725in}}%
\pgfpathlineto{\pgfqpoint{1.388601in}{2.999679in}}%
\pgfpathlineto{\pgfqpoint{1.394662in}{2.999679in}}%
\pgfpathlineto{\pgfqpoint{1.394662in}{3.003512in}}%
\pgfpathlineto{\pgfqpoint{1.400723in}{3.003512in}}%
\pgfpathlineto{\pgfqpoint{1.400723in}{2.999679in}}%
\pgfpathlineto{\pgfqpoint{1.406784in}{2.999679in}}%
\pgfpathlineto{\pgfqpoint{1.406784in}{3.002964in}}%
\pgfpathlineto{\pgfqpoint{1.412845in}{3.002964in}}%
\pgfpathlineto{\pgfqpoint{1.412845in}{3.007892in}}%
\pgfpathlineto{\pgfqpoint{1.418906in}{3.007892in}}%
\pgfpathlineto{\pgfqpoint{1.418906in}{2.998584in}}%
\pgfpathlineto{\pgfqpoint{1.424967in}{2.998584in}}%
\pgfpathlineto{\pgfqpoint{1.424967in}{3.007345in}}%
\pgfpathlineto{\pgfqpoint{1.431028in}{3.007345in}}%
\pgfpathlineto{\pgfqpoint{1.431028in}{3.000774in}}%
\pgfpathlineto{\pgfqpoint{1.437089in}{3.000774in}}%
\pgfpathlineto{\pgfqpoint{1.437089in}{3.004060in}}%
\pgfpathlineto{\pgfqpoint{1.449211in}{3.004607in}}%
\pgfpathlineto{\pgfqpoint{1.449211in}{3.007345in}}%
\pgfpathlineto{\pgfqpoint{1.461333in}{3.007892in}}%
\pgfpathlineto{\pgfqpoint{1.461333in}{3.005702in}}%
\pgfpathlineto{\pgfqpoint{1.467394in}{3.005702in}}%
\pgfpathlineto{\pgfqpoint{1.467394in}{3.013915in}}%
\pgfpathlineto{\pgfqpoint{1.473454in}{3.013915in}}%
\pgfpathlineto{\pgfqpoint{1.473454in}{3.002964in}}%
\pgfpathlineto{\pgfqpoint{1.479515in}{3.002964in}}%
\pgfpathlineto{\pgfqpoint{1.479515in}{3.000227in}}%
\pgfpathlineto{\pgfqpoint{1.485576in}{3.000227in}}%
\pgfpathlineto{\pgfqpoint{1.485576in}{3.005702in}}%
\pgfpathlineto{\pgfqpoint{1.491637in}{3.005702in}}%
\pgfpathlineto{\pgfqpoint{1.491637in}{3.012820in}}%
\pgfpathlineto{\pgfqpoint{1.497698in}{3.012820in}}%
\pgfpathlineto{\pgfqpoint{1.497698in}{3.005702in}}%
\pgfpathlineto{\pgfqpoint{1.503759in}{3.005702in}}%
\pgfpathlineto{\pgfqpoint{1.503759in}{3.002964in}}%
\pgfpathlineto{\pgfqpoint{1.509820in}{3.002964in}}%
\pgfpathlineto{\pgfqpoint{1.509820in}{3.008440in}}%
\pgfpathlineto{\pgfqpoint{1.515881in}{3.008440in}}%
\pgfpathlineto{\pgfqpoint{1.515881in}{3.001322in}}%
\pgfpathlineto{\pgfqpoint{1.521942in}{3.001322in}}%
\pgfpathlineto{\pgfqpoint{1.521942in}{3.007345in}}%
\pgfpathlineto{\pgfqpoint{1.528003in}{3.007345in}}%
\pgfpathlineto{\pgfqpoint{1.528003in}{3.003512in}}%
\pgfpathlineto{\pgfqpoint{1.534064in}{3.003512in}}%
\pgfpathlineto{\pgfqpoint{1.534064in}{3.010082in}}%
\pgfpathlineto{\pgfqpoint{1.558308in}{3.008987in}}%
\pgfpathlineto{\pgfqpoint{1.558308in}{3.004607in}}%
\pgfpathlineto{\pgfqpoint{1.564369in}{3.004607in}}%
\pgfpathlineto{\pgfqpoint{1.564369in}{3.008440in}}%
\pgfpathlineto{\pgfqpoint{1.570430in}{3.008440in}}%
\pgfpathlineto{\pgfqpoint{1.570430in}{3.006250in}}%
\pgfpathlineto{\pgfqpoint{1.576491in}{3.006250in}}%
\pgfpathlineto{\pgfqpoint{1.576491in}{3.015558in}}%
\pgfpathlineto{\pgfqpoint{1.582552in}{3.015558in}}%
\pgfpathlineto{\pgfqpoint{1.582552in}{3.005155in}}%
\pgfpathlineto{\pgfqpoint{1.588613in}{3.005155in}}%
\pgfpathlineto{\pgfqpoint{1.588613in}{3.012273in}}%
\pgfpathlineto{\pgfqpoint{1.594674in}{3.012273in}}%
\pgfpathlineto{\pgfqpoint{1.594674in}{3.006250in}}%
\pgfpathlineto{\pgfqpoint{1.600735in}{3.006250in}}%
\pgfpathlineto{\pgfqpoint{1.600735in}{3.013915in}}%
\pgfpathlineto{\pgfqpoint{1.606796in}{3.013915in}}%
\pgfpathlineto{\pgfqpoint{1.606796in}{3.009535in}}%
\pgfpathlineto{\pgfqpoint{1.612857in}{3.009535in}}%
\pgfpathlineto{\pgfqpoint{1.612857in}{3.012273in}}%
\pgfpathlineto{\pgfqpoint{1.618918in}{3.012273in}}%
\pgfpathlineto{\pgfqpoint{1.618918in}{3.010082in}}%
\pgfpathlineto{\pgfqpoint{1.624979in}{3.010082in}}%
\pgfpathlineto{\pgfqpoint{1.624979in}{3.001869in}}%
\pgfpathlineto{\pgfqpoint{1.631040in}{3.001869in}}%
\pgfpathlineto{\pgfqpoint{1.631040in}{3.013368in}}%
\pgfpathlineto{\pgfqpoint{1.643162in}{3.013368in}}%
\pgfpathlineto{\pgfqpoint{1.643162in}{3.010630in}}%
\pgfpathlineto{\pgfqpoint{1.649223in}{3.010630in}}%
\pgfpathlineto{\pgfqpoint{1.649223in}{3.001869in}}%
\pgfpathlineto{\pgfqpoint{1.655284in}{3.001869in}}%
\pgfpathlineto{\pgfqpoint{1.655284in}{3.006797in}}%
\pgfpathlineto{\pgfqpoint{1.661345in}{3.006797in}}%
\pgfpathlineto{\pgfqpoint{1.661345in}{3.002417in}}%
\pgfpathlineto{\pgfqpoint{1.667406in}{3.002417in}}%
\pgfpathlineto{\pgfqpoint{1.667406in}{3.021033in}}%
\pgfpathlineto{\pgfqpoint{1.673467in}{3.021033in}}%
\pgfpathlineto{\pgfqpoint{1.673467in}{3.005702in}}%
\pgfpathlineto{\pgfqpoint{1.679528in}{3.005702in}}%
\pgfpathlineto{\pgfqpoint{1.679528in}{3.019391in}}%
\pgfpathlineto{\pgfqpoint{1.685589in}{3.019391in}}%
\pgfpathlineto{\pgfqpoint{1.685589in}{3.016105in}}%
\pgfpathlineto{\pgfqpoint{1.691650in}{3.016105in}}%
\pgfpathlineto{\pgfqpoint{1.691650in}{3.013915in}}%
\pgfpathlineto{\pgfqpoint{1.697711in}{3.013915in}}%
\pgfpathlineto{\pgfqpoint{1.697711in}{3.016105in}}%
\pgfpathlineto{\pgfqpoint{1.703772in}{3.016105in}}%
\pgfpathlineto{\pgfqpoint{1.703772in}{3.008987in}}%
\pgfpathlineto{\pgfqpoint{1.709833in}{3.008987in}}%
\pgfpathlineto{\pgfqpoint{1.709833in}{3.019938in}}%
\pgfpathlineto{\pgfqpoint{1.715894in}{3.019938in}}%
\pgfpathlineto{\pgfqpoint{1.715894in}{3.006250in}}%
\pgfpathlineto{\pgfqpoint{1.721955in}{3.006250in}}%
\pgfpathlineto{\pgfqpoint{1.721955in}{3.014463in}}%
\pgfpathlineto{\pgfqpoint{1.728016in}{3.014463in}}%
\pgfpathlineto{\pgfqpoint{1.728016in}{3.018295in}}%
\pgfpathlineto{\pgfqpoint{1.734077in}{3.018295in}}%
\pgfpathlineto{\pgfqpoint{1.734077in}{3.023771in}}%
\pgfpathlineto{\pgfqpoint{1.740138in}{3.023771in}}%
\pgfpathlineto{\pgfqpoint{1.740138in}{3.011177in}}%
\pgfpathlineto{\pgfqpoint{1.746199in}{3.011177in}}%
\pgfpathlineto{\pgfqpoint{1.746199in}{3.008987in}}%
\pgfpathlineto{\pgfqpoint{1.752260in}{3.008987in}}%
\pgfpathlineto{\pgfqpoint{1.752260in}{3.014463in}}%
\pgfpathlineto{\pgfqpoint{1.758321in}{3.014463in}}%
\pgfpathlineto{\pgfqpoint{1.758321in}{3.022128in}}%
\pgfpathlineto{\pgfqpoint{1.764381in}{3.022128in}}%
\pgfpathlineto{\pgfqpoint{1.764381in}{3.012273in}}%
\pgfpathlineto{\pgfqpoint{1.770442in}{3.012273in}}%
\pgfpathlineto{\pgfqpoint{1.770442in}{3.015010in}}%
\pgfpathlineto{\pgfqpoint{1.776503in}{3.015010in}}%
\pgfpathlineto{\pgfqpoint{1.776503in}{3.019391in}}%
\pgfpathlineto{\pgfqpoint{1.782564in}{3.019391in}}%
\pgfpathlineto{\pgfqpoint{1.782564in}{3.016653in}}%
\pgfpathlineto{\pgfqpoint{1.788625in}{3.016653in}}%
\pgfpathlineto{\pgfqpoint{1.788625in}{3.027056in}}%
\pgfpathlineto{\pgfqpoint{1.800747in}{3.027604in}}%
\pgfpathlineto{\pgfqpoint{1.800747in}{3.009535in}}%
\pgfpathlineto{\pgfqpoint{1.806808in}{3.009535in}}%
\pgfpathlineto{\pgfqpoint{1.806808in}{3.024866in}}%
\pgfpathlineto{\pgfqpoint{1.812869in}{3.024866in}}%
\pgfpathlineto{\pgfqpoint{1.812869in}{3.019938in}}%
\pgfpathlineto{\pgfqpoint{1.818930in}{3.019938in}}%
\pgfpathlineto{\pgfqpoint{1.818930in}{3.030889in}}%
\pgfpathlineto{\pgfqpoint{1.824991in}{3.030889in}}%
\pgfpathlineto{\pgfqpoint{1.824991in}{3.017200in}}%
\pgfpathlineto{\pgfqpoint{1.831052in}{3.017200in}}%
\pgfpathlineto{\pgfqpoint{1.831052in}{3.023771in}}%
\pgfpathlineto{\pgfqpoint{1.837113in}{3.023771in}}%
\pgfpathlineto{\pgfqpoint{1.837113in}{3.016653in}}%
\pgfpathlineto{\pgfqpoint{1.843174in}{3.016653in}}%
\pgfpathlineto{\pgfqpoint{1.843174in}{3.031984in}}%
\pgfpathlineto{\pgfqpoint{1.849235in}{3.031984in}}%
\pgfpathlineto{\pgfqpoint{1.849235in}{3.028699in}}%
\pgfpathlineto{\pgfqpoint{1.855296in}{3.028699in}}%
\pgfpathlineto{\pgfqpoint{1.855296in}{3.022676in}}%
\pgfpathlineto{\pgfqpoint{1.861357in}{3.022676in}}%
\pgfpathlineto{\pgfqpoint{1.861357in}{3.024318in}}%
\pgfpathlineto{\pgfqpoint{1.867418in}{3.024318in}}%
\pgfpathlineto{\pgfqpoint{1.867418in}{3.021581in}}%
\pgfpathlineto{\pgfqpoint{1.873479in}{3.021581in}}%
\pgfpathlineto{\pgfqpoint{1.873479in}{3.031984in}}%
\pgfpathlineto{\pgfqpoint{1.879540in}{3.031984in}}%
\pgfpathlineto{\pgfqpoint{1.879540in}{3.023223in}}%
\pgfpathlineto{\pgfqpoint{1.897723in}{3.024318in}}%
\pgfpathlineto{\pgfqpoint{1.897723in}{3.030341in}}%
\pgfpathlineto{\pgfqpoint{1.903784in}{3.030341in}}%
\pgfpathlineto{\pgfqpoint{1.903784in}{3.040197in}}%
\pgfpathlineto{\pgfqpoint{1.909845in}{3.040197in}}%
\pgfpathlineto{\pgfqpoint{1.909845in}{3.018295in}}%
\pgfpathlineto{\pgfqpoint{1.915906in}{3.018295in}}%
\pgfpathlineto{\pgfqpoint{1.915906in}{3.021033in}}%
\pgfpathlineto{\pgfqpoint{1.921967in}{3.021033in}}%
\pgfpathlineto{\pgfqpoint{1.921967in}{3.024866in}}%
\pgfpathlineto{\pgfqpoint{1.928028in}{3.024866in}}%
\pgfpathlineto{\pgfqpoint{1.928028in}{3.040744in}}%
\pgfpathlineto{\pgfqpoint{1.934089in}{3.040744in}}%
\pgfpathlineto{\pgfqpoint{1.934089in}{3.024866in}}%
\pgfpathlineto{\pgfqpoint{1.940150in}{3.024866in}}%
\pgfpathlineto{\pgfqpoint{1.940150in}{3.031984in}}%
\pgfpathlineto{\pgfqpoint{1.946211in}{3.031984in}}%
\pgfpathlineto{\pgfqpoint{1.946211in}{3.027604in}}%
\pgfpathlineto{\pgfqpoint{1.952272in}{3.027604in}}%
\pgfpathlineto{\pgfqpoint{1.952272in}{3.038007in}}%
\pgfpathlineto{\pgfqpoint{1.958333in}{3.038007in}}%
\pgfpathlineto{\pgfqpoint{1.958333in}{3.027604in}}%
\pgfpathlineto{\pgfqpoint{1.964394in}{3.027604in}}%
\pgfpathlineto{\pgfqpoint{1.964394in}{3.031436in}}%
\pgfpathlineto{\pgfqpoint{1.970455in}{3.031436in}}%
\pgfpathlineto{\pgfqpoint{1.970455in}{3.025961in}}%
\pgfpathlineto{\pgfqpoint{1.976516in}{3.025961in}}%
\pgfpathlineto{\pgfqpoint{1.976516in}{3.031984in}}%
\pgfpathlineto{\pgfqpoint{1.982577in}{3.031984in}}%
\pgfpathlineto{\pgfqpoint{1.982577in}{3.028699in}}%
\pgfpathlineto{\pgfqpoint{1.988638in}{3.028699in}}%
\pgfpathlineto{\pgfqpoint{1.988638in}{3.036364in}}%
\pgfpathlineto{\pgfqpoint{1.994699in}{3.036364in}}%
\pgfpathlineto{\pgfqpoint{1.994699in}{3.042387in}}%
\pgfpathlineto{\pgfqpoint{2.000760in}{3.042387in}}%
\pgfpathlineto{\pgfqpoint{2.000760in}{3.025413in}}%
\pgfpathlineto{\pgfqpoint{2.006821in}{3.025413in}}%
\pgfpathlineto{\pgfqpoint{2.006821in}{3.050053in}}%
\pgfpathlineto{\pgfqpoint{2.012882in}{3.050053in}}%
\pgfpathlineto{\pgfqpoint{2.012882in}{3.038007in}}%
\pgfpathlineto{\pgfqpoint{2.018943in}{3.038007in}}%
\pgfpathlineto{\pgfqpoint{2.018943in}{3.045672in}}%
\pgfpathlineto{\pgfqpoint{2.025004in}{3.045672in}}%
\pgfpathlineto{\pgfqpoint{2.025004in}{3.043482in}}%
\pgfpathlineto{\pgfqpoint{2.031065in}{3.043482in}}%
\pgfpathlineto{\pgfqpoint{2.031065in}{3.030889in}}%
\pgfpathlineto{\pgfqpoint{2.037126in}{3.030889in}}%
\pgfpathlineto{\pgfqpoint{2.037126in}{3.064836in}}%
\pgfpathlineto{\pgfqpoint{2.043187in}{3.064836in}}%
\pgfpathlineto{\pgfqpoint{2.043187in}{3.037459in}}%
\pgfpathlineto{\pgfqpoint{2.049248in}{3.037459in}}%
\pgfpathlineto{\pgfqpoint{2.049248in}{3.048957in}}%
\pgfpathlineto{\pgfqpoint{2.061369in}{3.047862in}}%
\pgfpathlineto{\pgfqpoint{2.061369in}{3.053338in}}%
\pgfpathlineto{\pgfqpoint{2.067430in}{3.053338in}}%
\pgfpathlineto{\pgfqpoint{2.067430in}{3.044030in}}%
\pgfpathlineto{\pgfqpoint{2.073491in}{3.044030in}}%
\pgfpathlineto{\pgfqpoint{2.073491in}{3.054433in}}%
\pgfpathlineto{\pgfqpoint{2.079552in}{3.054433in}}%
\pgfpathlineto{\pgfqpoint{2.079552in}{3.038554in}}%
\pgfpathlineto{\pgfqpoint{2.085613in}{3.038554in}}%
\pgfpathlineto{\pgfqpoint{2.085613in}{3.048410in}}%
\pgfpathlineto{\pgfqpoint{2.097735in}{3.047315in}}%
\pgfpathlineto{\pgfqpoint{2.097735in}{3.046220in}}%
\pgfpathlineto{\pgfqpoint{2.103796in}{3.046220in}}%
\pgfpathlineto{\pgfqpoint{2.103796in}{3.070311in}}%
\pgfpathlineto{\pgfqpoint{2.109857in}{3.070311in}}%
\pgfpathlineto{\pgfqpoint{2.109857in}{3.041292in}}%
\pgfpathlineto{\pgfqpoint{2.115918in}{3.041292in}}%
\pgfpathlineto{\pgfqpoint{2.115918in}{3.050053in}}%
\pgfpathlineto{\pgfqpoint{2.128040in}{3.049505in}}%
\pgfpathlineto{\pgfqpoint{2.128040in}{3.064836in}}%
\pgfpathlineto{\pgfqpoint{2.134101in}{3.064836in}}%
\pgfpathlineto{\pgfqpoint{2.134101in}{3.048957in}}%
\pgfpathlineto{\pgfqpoint{2.140162in}{3.048957in}}%
\pgfpathlineto{\pgfqpoint{2.140162in}{3.043482in}}%
\pgfpathlineto{\pgfqpoint{2.146223in}{3.043482in}}%
\pgfpathlineto{\pgfqpoint{2.146223in}{3.070311in}}%
\pgfpathlineto{\pgfqpoint{2.152284in}{3.070311in}}%
\pgfpathlineto{\pgfqpoint{2.152284in}{3.044030in}}%
\pgfpathlineto{\pgfqpoint{2.158345in}{3.044030in}}%
\pgfpathlineto{\pgfqpoint{2.158345in}{3.067574in}}%
\pgfpathlineto{\pgfqpoint{2.164406in}{3.067574in}}%
\pgfpathlineto{\pgfqpoint{2.164406in}{3.060456in}}%
\pgfpathlineto{\pgfqpoint{2.170467in}{3.060456in}}%
\pgfpathlineto{\pgfqpoint{2.170467in}{3.064836in}}%
\pgfpathlineto{\pgfqpoint{2.176528in}{3.064836in}}%
\pgfpathlineto{\pgfqpoint{2.176528in}{3.051695in}}%
\pgfpathlineto{\pgfqpoint{2.182589in}{3.051695in}}%
\pgfpathlineto{\pgfqpoint{2.182589in}{3.065931in}}%
\pgfpathlineto{\pgfqpoint{2.194711in}{3.065931in}}%
\pgfpathlineto{\pgfqpoint{2.194711in}{3.059361in}}%
\pgfpathlineto{\pgfqpoint{2.200772in}{3.059361in}}%
\pgfpathlineto{\pgfqpoint{2.200772in}{3.074144in}}%
\pgfpathlineto{\pgfqpoint{2.212894in}{3.075239in}}%
\pgfpathlineto{\pgfqpoint{2.212894in}{3.091665in}}%
\pgfpathlineto{\pgfqpoint{2.218955in}{3.091665in}}%
\pgfpathlineto{\pgfqpoint{2.218955in}{3.070311in}}%
\pgfpathlineto{\pgfqpoint{2.225016in}{3.070311in}}%
\pgfpathlineto{\pgfqpoint{2.225016in}{3.082905in}}%
\pgfpathlineto{\pgfqpoint{2.231077in}{3.082905in}}%
\pgfpathlineto{\pgfqpoint{2.231077in}{3.068121in}}%
\pgfpathlineto{\pgfqpoint{2.237138in}{3.068121in}}%
\pgfpathlineto{\pgfqpoint{2.237138in}{3.079072in}}%
\pgfpathlineto{\pgfqpoint{2.243199in}{3.079072in}}%
\pgfpathlineto{\pgfqpoint{2.243199in}{3.096593in}}%
\pgfpathlineto{\pgfqpoint{2.249260in}{3.096593in}}%
\pgfpathlineto{\pgfqpoint{2.249260in}{3.069216in}}%
\pgfpathlineto{\pgfqpoint{2.255321in}{3.069216in}}%
\pgfpathlineto{\pgfqpoint{2.255321in}{3.083452in}}%
\pgfpathlineto{\pgfqpoint{2.261382in}{3.083452in}}%
\pgfpathlineto{\pgfqpoint{2.261382in}{3.068669in}}%
\pgfpathlineto{\pgfqpoint{2.267443in}{3.068669in}}%
\pgfpathlineto{\pgfqpoint{2.267443in}{3.084547in}}%
\pgfpathlineto{\pgfqpoint{2.273504in}{3.084547in}}%
\pgfpathlineto{\pgfqpoint{2.273504in}{3.075787in}}%
\pgfpathlineto{\pgfqpoint{2.279565in}{3.075787in}}%
\pgfpathlineto{\pgfqpoint{2.279565in}{3.103711in}}%
\pgfpathlineto{\pgfqpoint{2.285626in}{3.103711in}}%
\pgfpathlineto{\pgfqpoint{2.285626in}{3.075239in}}%
\pgfpathlineto{\pgfqpoint{2.291687in}{3.075239in}}%
\pgfpathlineto{\pgfqpoint{2.291687in}{3.096046in}}%
\pgfpathlineto{\pgfqpoint{2.303809in}{3.096046in}}%
\pgfpathlineto{\pgfqpoint{2.303809in}{3.091118in}}%
\pgfpathlineto{\pgfqpoint{2.309870in}{3.091118in}}%
\pgfpathlineto{\pgfqpoint{2.309870in}{3.101521in}}%
\pgfpathlineto{\pgfqpoint{2.315931in}{3.101521in}}%
\pgfpathlineto{\pgfqpoint{2.315931in}{3.089475in}}%
\pgfpathlineto{\pgfqpoint{2.321992in}{3.089475in}}%
\pgfpathlineto{\pgfqpoint{2.321992in}{3.099878in}}%
\pgfpathlineto{\pgfqpoint{2.328053in}{3.099878in}}%
\pgfpathlineto{\pgfqpoint{2.328053in}{3.088380in}}%
\pgfpathlineto{\pgfqpoint{2.334114in}{3.088380in}}%
\pgfpathlineto{\pgfqpoint{2.334114in}{3.097688in}}%
\pgfpathlineto{\pgfqpoint{2.340175in}{3.097688in}}%
\pgfpathlineto{\pgfqpoint{2.340175in}{3.083452in}}%
\pgfpathlineto{\pgfqpoint{2.346235in}{3.083452in}}%
\pgfpathlineto{\pgfqpoint{2.346235in}{3.116304in}}%
\pgfpathlineto{\pgfqpoint{2.352296in}{3.116304in}}%
\pgfpathlineto{\pgfqpoint{2.352296in}{3.088928in}}%
\pgfpathlineto{\pgfqpoint{2.358357in}{3.088928in}}%
\pgfpathlineto{\pgfqpoint{2.358357in}{3.119042in}}%
\pgfpathlineto{\pgfqpoint{2.364418in}{3.119042in}}%
\pgfpathlineto{\pgfqpoint{2.364418in}{3.109186in}}%
\pgfpathlineto{\pgfqpoint{2.370479in}{3.109186in}}%
\pgfpathlineto{\pgfqpoint{2.370479in}{3.090023in}}%
\pgfpathlineto{\pgfqpoint{2.376540in}{3.090023in}}%
\pgfpathlineto{\pgfqpoint{2.376540in}{3.140396in}}%
\pgfpathlineto{\pgfqpoint{2.382601in}{3.140396in}}%
\pgfpathlineto{\pgfqpoint{2.382601in}{3.094951in}}%
\pgfpathlineto{\pgfqpoint{2.388662in}{3.094951in}}%
\pgfpathlineto{\pgfqpoint{2.388662in}{3.105354in}}%
\pgfpathlineto{\pgfqpoint{2.394723in}{3.105354in}}%
\pgfpathlineto{\pgfqpoint{2.394723in}{3.096593in}}%
\pgfpathlineto{\pgfqpoint{2.400784in}{3.096593in}}%
\pgfpathlineto{\pgfqpoint{2.400784in}{3.104259in}}%
\pgfpathlineto{\pgfqpoint{2.406845in}{3.104259in}}%
\pgfpathlineto{\pgfqpoint{2.406845in}{3.114662in}}%
\pgfpathlineto{\pgfqpoint{2.412906in}{3.114662in}}%
\pgfpathlineto{\pgfqpoint{2.412906in}{3.136563in}}%
\pgfpathlineto{\pgfqpoint{2.418967in}{3.136563in}}%
\pgfpathlineto{\pgfqpoint{2.418967in}{3.139848in}}%
\pgfpathlineto{\pgfqpoint{2.425028in}{3.139848in}}%
\pgfpathlineto{\pgfqpoint{2.425028in}{3.121780in}}%
\pgfpathlineto{\pgfqpoint{2.431089in}{3.121780in}}%
\pgfpathlineto{\pgfqpoint{2.431089in}{3.144229in}}%
\pgfpathlineto{\pgfqpoint{2.437150in}{3.144229in}}%
\pgfpathlineto{\pgfqpoint{2.437150in}{3.116304in}}%
\pgfpathlineto{\pgfqpoint{2.443211in}{3.116304in}}%
\pgfpathlineto{\pgfqpoint{2.443211in}{3.132730in}}%
\pgfpathlineto{\pgfqpoint{2.449272in}{3.132730in}}%
\pgfpathlineto{\pgfqpoint{2.449272in}{3.115757in}}%
\pgfpathlineto{\pgfqpoint{2.455333in}{3.115757in}}%
\pgfpathlineto{\pgfqpoint{2.455333in}{3.146966in}}%
\pgfpathlineto{\pgfqpoint{2.461394in}{3.146966in}}%
\pgfpathlineto{\pgfqpoint{2.461394in}{3.123422in}}%
\pgfpathlineto{\pgfqpoint{2.467455in}{3.123422in}}%
\pgfpathlineto{\pgfqpoint{2.467455in}{3.145324in}}%
\pgfpathlineto{\pgfqpoint{2.473516in}{3.145324in}}%
\pgfpathlineto{\pgfqpoint{2.473516in}{3.159560in}}%
\pgfpathlineto{\pgfqpoint{2.479577in}{3.159560in}}%
\pgfpathlineto{\pgfqpoint{2.479577in}{3.121232in}}%
\pgfpathlineto{\pgfqpoint{2.485638in}{3.121232in}}%
\pgfpathlineto{\pgfqpoint{2.485638in}{3.141491in}}%
\pgfpathlineto{\pgfqpoint{2.491699in}{3.141491in}}%
\pgfpathlineto{\pgfqpoint{2.491699in}{3.143134in}}%
\pgfpathlineto{\pgfqpoint{2.497760in}{3.143134in}}%
\pgfpathlineto{\pgfqpoint{2.497760in}{3.150252in}}%
\pgfpathlineto{\pgfqpoint{2.503821in}{3.150252in}}%
\pgfpathlineto{\pgfqpoint{2.503821in}{3.142586in}}%
\pgfpathlineto{\pgfqpoint{2.509882in}{3.142586in}}%
\pgfpathlineto{\pgfqpoint{2.509882in}{3.178724in}}%
\pgfpathlineto{\pgfqpoint{2.515943in}{3.178724in}}%
\pgfpathlineto{\pgfqpoint{2.515943in}{3.126708in}}%
\pgfpathlineto{\pgfqpoint{2.522004in}{3.126708in}}%
\pgfpathlineto{\pgfqpoint{2.522004in}{3.166130in}}%
\pgfpathlineto{\pgfqpoint{2.534126in}{3.166130in}}%
\pgfpathlineto{\pgfqpoint{2.534126in}{3.126708in}}%
\pgfpathlineto{\pgfqpoint{2.540187in}{3.126708in}}%
\pgfpathlineto{\pgfqpoint{2.540187in}{3.168868in}}%
\pgfpathlineto{\pgfqpoint{2.546248in}{3.168868in}}%
\pgfpathlineto{\pgfqpoint{2.546248in}{3.149704in}}%
\pgfpathlineto{\pgfqpoint{2.552309in}{3.149704in}}%
\pgfpathlineto{\pgfqpoint{2.552309in}{3.167225in}}%
\pgfpathlineto{\pgfqpoint{2.558370in}{3.167225in}}%
\pgfpathlineto{\pgfqpoint{2.558370in}{3.141491in}}%
\pgfpathlineto{\pgfqpoint{2.564431in}{3.141491in}}%
\pgfpathlineto{\pgfqpoint{2.564431in}{3.188032in}}%
\pgfpathlineto{\pgfqpoint{2.570492in}{3.188032in}}%
\pgfpathlineto{\pgfqpoint{2.570492in}{3.160107in}}%
\pgfpathlineto{\pgfqpoint{2.576553in}{3.160107in}}%
\pgfpathlineto{\pgfqpoint{2.576553in}{3.183104in}}%
\pgfpathlineto{\pgfqpoint{2.582614in}{3.183104in}}%
\pgfpathlineto{\pgfqpoint{2.582614in}{3.184746in}}%
\pgfpathlineto{\pgfqpoint{2.588675in}{3.184746in}}%
\pgfpathlineto{\pgfqpoint{2.588675in}{3.159012in}}%
\pgfpathlineto{\pgfqpoint{2.594736in}{3.159012in}}%
\pgfpathlineto{\pgfqpoint{2.594736in}{3.172701in}}%
\pgfpathlineto{\pgfqpoint{2.600797in}{3.172701in}}%
\pgfpathlineto{\pgfqpoint{2.600797in}{3.163393in}}%
\pgfpathlineto{\pgfqpoint{2.606858in}{3.163393in}}%
\pgfpathlineto{\pgfqpoint{2.606858in}{3.205005in}}%
\pgfpathlineto{\pgfqpoint{2.612919in}{3.205005in}}%
\pgfpathlineto{\pgfqpoint{2.612919in}{3.165035in}}%
\pgfpathlineto{\pgfqpoint{2.618980in}{3.165035in}}%
\pgfpathlineto{\pgfqpoint{2.618980in}{3.203363in}}%
\pgfpathlineto{\pgfqpoint{2.625041in}{3.203363in}}%
\pgfpathlineto{\pgfqpoint{2.625041in}{3.153537in}}%
\pgfpathlineto{\pgfqpoint{2.631102in}{3.153537in}}%
\pgfpathlineto{\pgfqpoint{2.631102in}{3.218694in}}%
\pgfpathlineto{\pgfqpoint{2.637162in}{3.218694in}}%
\pgfpathlineto{\pgfqpoint{2.637162in}{3.207195in}}%
\pgfpathlineto{\pgfqpoint{2.643223in}{3.207195in}}%
\pgfpathlineto{\pgfqpoint{2.643223in}{3.172153in}}%
\pgfpathlineto{\pgfqpoint{2.649284in}{3.172153in}}%
\pgfpathlineto{\pgfqpoint{2.649284in}{3.210481in}}%
\pgfpathlineto{\pgfqpoint{2.655345in}{3.210481in}}%
\pgfpathlineto{\pgfqpoint{2.655345in}{3.183104in}}%
\pgfpathlineto{\pgfqpoint{2.661406in}{3.183104in}}%
\pgfpathlineto{\pgfqpoint{2.661406in}{3.229644in}}%
\pgfpathlineto{\pgfqpoint{2.667467in}{3.229644in}}%
\pgfpathlineto{\pgfqpoint{2.667467in}{3.194055in}}%
\pgfpathlineto{\pgfqpoint{2.673528in}{3.194055in}}%
\pgfpathlineto{\pgfqpoint{2.673528in}{3.219789in}}%
\pgfpathlineto{\pgfqpoint{2.679589in}{3.219789in}}%
\pgfpathlineto{\pgfqpoint{2.679589in}{3.185841in}}%
\pgfpathlineto{\pgfqpoint{2.685650in}{3.185841in}}%
\pgfpathlineto{\pgfqpoint{2.685650in}{3.225812in}}%
\pgfpathlineto{\pgfqpoint{2.691711in}{3.225812in}}%
\pgfpathlineto{\pgfqpoint{2.691711in}{3.196245in}}%
\pgfpathlineto{\pgfqpoint{2.697772in}{3.196245in}}%
\pgfpathlineto{\pgfqpoint{2.697772in}{3.249903in}}%
\pgfpathlineto{\pgfqpoint{2.703833in}{3.249903in}}%
\pgfpathlineto{\pgfqpoint{2.703833in}{3.224169in}}%
\pgfpathlineto{\pgfqpoint{2.709894in}{3.224169in}}%
\pgfpathlineto{\pgfqpoint{2.709894in}{3.209933in}}%
\pgfpathlineto{\pgfqpoint{2.715955in}{3.209933in}}%
\pgfpathlineto{\pgfqpoint{2.715955in}{3.264139in}}%
\pgfpathlineto{\pgfqpoint{2.722016in}{3.264139in}}%
\pgfpathlineto{\pgfqpoint{2.722016in}{3.200077in}}%
\pgfpathlineto{\pgfqpoint{2.728077in}{3.200077in}}%
\pgfpathlineto{\pgfqpoint{2.728077in}{3.234025in}}%
\pgfpathlineto{\pgfqpoint{2.734138in}{3.234025in}}%
\pgfpathlineto{\pgfqpoint{2.734138in}{3.216504in}}%
\pgfpathlineto{\pgfqpoint{2.740199in}{3.216504in}}%
\pgfpathlineto{\pgfqpoint{2.740199in}{3.244975in}}%
\pgfpathlineto{\pgfqpoint{2.746260in}{3.244975in}}%
\pgfpathlineto{\pgfqpoint{2.746260in}{3.213218in}}%
\pgfpathlineto{\pgfqpoint{2.752321in}{3.213218in}}%
\pgfpathlineto{\pgfqpoint{2.752321in}{3.243333in}}%
\pgfpathlineto{\pgfqpoint{2.758382in}{3.243333in}}%
\pgfpathlineto{\pgfqpoint{2.758382in}{3.252641in}}%
\pgfpathlineto{\pgfqpoint{2.764443in}{3.252641in}}%
\pgfpathlineto{\pgfqpoint{2.764443in}{3.213766in}}%
\pgfpathlineto{\pgfqpoint{2.770504in}{3.213766in}}%
\pgfpathlineto{\pgfqpoint{2.770504in}{3.270710in}}%
\pgfpathlineto{\pgfqpoint{2.776565in}{3.270710in}}%
\pgfpathlineto{\pgfqpoint{2.776565in}{3.254284in}}%
\pgfpathlineto{\pgfqpoint{2.782626in}{3.254284in}}%
\pgfpathlineto{\pgfqpoint{2.782626in}{3.290968in}}%
\pgfpathlineto{\pgfqpoint{2.788687in}{3.290968in}}%
\pgfpathlineto{\pgfqpoint{2.788687in}{3.220884in}}%
\pgfpathlineto{\pgfqpoint{2.794748in}{3.220884in}}%
\pgfpathlineto{\pgfqpoint{2.794748in}{3.292611in}}%
\pgfpathlineto{\pgfqpoint{2.800809in}{3.292611in}}%
\pgfpathlineto{\pgfqpoint{2.800809in}{3.238405in}}%
\pgfpathlineto{\pgfqpoint{2.806870in}{3.238405in}}%
\pgfpathlineto{\pgfqpoint{2.806870in}{3.276732in}}%
\pgfpathlineto{\pgfqpoint{2.812931in}{3.276732in}}%
\pgfpathlineto{\pgfqpoint{2.812931in}{3.281113in}}%
\pgfpathlineto{\pgfqpoint{2.818992in}{3.281113in}}%
\pgfpathlineto{\pgfqpoint{2.818992in}{3.271257in}}%
\pgfpathlineto{\pgfqpoint{2.825053in}{3.271257in}}%
\pgfpathlineto{\pgfqpoint{2.825053in}{3.321083in}}%
\pgfpathlineto{\pgfqpoint{2.831114in}{3.321083in}}%
\pgfpathlineto{\pgfqpoint{2.831114in}{3.252641in}}%
\pgfpathlineto{\pgfqpoint{2.837175in}{3.252641in}}%
\pgfpathlineto{\pgfqpoint{2.837175in}{3.313965in}}%
\pgfpathlineto{\pgfqpoint{2.843236in}{3.313965in}}%
\pgfpathlineto{\pgfqpoint{2.843236in}{3.260306in}}%
\pgfpathlineto{\pgfqpoint{2.849297in}{3.260306in}}%
\pgfpathlineto{\pgfqpoint{2.849297in}{3.316703in}}%
\pgfpathlineto{\pgfqpoint{2.855358in}{3.316703in}}%
\pgfpathlineto{\pgfqpoint{2.855358in}{3.260854in}}%
\pgfpathlineto{\pgfqpoint{2.861419in}{3.260854in}}%
\pgfpathlineto{\pgfqpoint{2.861419in}{3.328201in}}%
\pgfpathlineto{\pgfqpoint{2.867480in}{3.328201in}}%
\pgfpathlineto{\pgfqpoint{2.867480in}{3.373646in}}%
\pgfpathlineto{\pgfqpoint{2.873541in}{3.373646in}}%
\pgfpathlineto{\pgfqpoint{2.873541in}{3.283850in}}%
\pgfpathlineto{\pgfqpoint{2.879602in}{3.283850in}}%
\pgfpathlineto{\pgfqpoint{2.879602in}{3.340247in}}%
\pgfpathlineto{\pgfqpoint{2.885663in}{3.340247in}}%
\pgfpathlineto{\pgfqpoint{2.885663in}{3.303014in}}%
\pgfpathlineto{\pgfqpoint{2.891724in}{3.303014in}}%
\pgfpathlineto{\pgfqpoint{2.891724in}{3.392263in}}%
\pgfpathlineto{\pgfqpoint{2.897785in}{3.392263in}}%
\pgfpathlineto{\pgfqpoint{2.897785in}{3.314512in}}%
\pgfpathlineto{\pgfqpoint{2.903846in}{3.314512in}}%
\pgfpathlineto{\pgfqpoint{2.903846in}{3.350650in}}%
\pgfpathlineto{\pgfqpoint{2.909907in}{3.350650in}}%
\pgfpathlineto{\pgfqpoint{2.909907in}{3.321630in}}%
\pgfpathlineto{\pgfqpoint{2.915968in}{3.321630in}}%
\pgfpathlineto{\pgfqpoint{2.915968in}{3.395000in}}%
\pgfpathlineto{\pgfqpoint{2.922029in}{3.395000in}}%
\pgfpathlineto{\pgfqpoint{2.922029in}{3.368719in}}%
\pgfpathlineto{\pgfqpoint{2.928089in}{3.368719in}}%
\pgfpathlineto{\pgfqpoint{2.928089in}{3.329296in}}%
\pgfpathlineto{\pgfqpoint{2.934150in}{3.329296in}}%
\pgfpathlineto{\pgfqpoint{2.934150in}{3.442088in}}%
\pgfpathlineto{\pgfqpoint{2.940211in}{3.442088in}}%
\pgfpathlineto{\pgfqpoint{2.940211in}{3.378027in}}%
\pgfpathlineto{\pgfqpoint{2.946272in}{3.378027in}}%
\pgfpathlineto{\pgfqpoint{2.946272in}{3.428948in}}%
\pgfpathlineto{\pgfqpoint{2.952333in}{3.428948in}}%
\pgfpathlineto{\pgfqpoint{2.952333in}{3.359410in}}%
\pgfpathlineto{\pgfqpoint{2.958394in}{3.359410in}}%
\pgfpathlineto{\pgfqpoint{2.958394in}{3.473298in}}%
\pgfpathlineto{\pgfqpoint{2.964455in}{3.473298in}}%
\pgfpathlineto{\pgfqpoint{2.964455in}{3.367076in}}%
\pgfpathlineto{\pgfqpoint{2.970516in}{3.367076in}}%
\pgfpathlineto{\pgfqpoint{2.970516in}{3.454682in}}%
\pgfpathlineto{\pgfqpoint{2.976577in}{3.454682in}}%
\pgfpathlineto{\pgfqpoint{2.976577in}{3.497937in}}%
\pgfpathlineto{\pgfqpoint{2.982638in}{3.497937in}}%
\pgfpathlineto{\pgfqpoint{2.982638in}{3.382954in}}%
\pgfpathlineto{\pgfqpoint{2.988699in}{3.382954in}}%
\pgfpathlineto{\pgfqpoint{2.988699in}{3.451396in}}%
\pgfpathlineto{\pgfqpoint{2.994760in}{3.451396in}}%
\pgfpathlineto{\pgfqpoint{2.994760in}{3.416354in}}%
\pgfpathlineto{\pgfqpoint{3.000821in}{3.416354in}}%
\pgfpathlineto{\pgfqpoint{3.000821in}{3.542287in}}%
\pgfpathlineto{\pgfqpoint{3.006882in}{3.542287in}}%
\pgfpathlineto{\pgfqpoint{3.006882in}{3.455777in}}%
\pgfpathlineto{\pgfqpoint{3.012943in}{3.455777in}}%
\pgfpathlineto{\pgfqpoint{3.012943in}{3.506698in}}%
\pgfpathlineto{\pgfqpoint{3.019004in}{3.506698in}}%
\pgfpathlineto{\pgfqpoint{3.019004in}{3.465632in}}%
\pgfpathlineto{\pgfqpoint{3.025065in}{3.465632in}}%
\pgfpathlineto{\pgfqpoint{3.025065in}{3.586090in}}%
\pgfpathlineto{\pgfqpoint{3.031126in}{3.586090in}}%
\pgfpathlineto{\pgfqpoint{3.031126in}{3.528599in}}%
\pgfpathlineto{\pgfqpoint{3.037187in}{3.528599in}}%
\pgfpathlineto{\pgfqpoint{3.037187in}{3.484796in}}%
\pgfpathlineto{\pgfqpoint{3.043248in}{3.484796in}}%
\pgfpathlineto{\pgfqpoint{3.043248in}{3.610729in}}%
\pgfpathlineto{\pgfqpoint{3.049309in}{3.610729in}}%
\pgfpathlineto{\pgfqpoint{3.049309in}{3.541192in}}%
\pgfpathlineto{\pgfqpoint{3.055370in}{3.541192in}}%
\pgfpathlineto{\pgfqpoint{3.055370in}{3.568022in}}%
\pgfpathlineto{\pgfqpoint{3.061431in}{3.568022in}}%
\pgfpathlineto{\pgfqpoint{3.061431in}{3.516006in}}%
\pgfpathlineto{\pgfqpoint{3.067492in}{3.516006in}}%
\pgfpathlineto{\pgfqpoint{3.067492in}{3.678624in}}%
\pgfpathlineto{\pgfqpoint{3.073553in}{3.678624in}}%
\pgfpathlineto{\pgfqpoint{3.073553in}{3.562546in}}%
\pgfpathlineto{\pgfqpoint{3.079614in}{3.562546in}}%
\pgfpathlineto{\pgfqpoint{3.079614in}{3.620038in}}%
\pgfpathlineto{\pgfqpoint{3.085675in}{3.620038in}}%
\pgfpathlineto{\pgfqpoint{3.085675in}{3.584448in}}%
\pgfpathlineto{\pgfqpoint{3.091736in}{3.584448in}}%
\pgfpathlineto{\pgfqpoint{3.091736in}{3.692312in}}%
\pgfpathlineto{\pgfqpoint{3.097797in}{3.692312in}}%
\pgfpathlineto{\pgfqpoint{3.097797in}{3.681909in}}%
\pgfpathlineto{\pgfqpoint{3.103858in}{3.681909in}}%
\pgfpathlineto{\pgfqpoint{3.103858in}{3.586638in}}%
\pgfpathlineto{\pgfqpoint{3.109919in}{3.586638in}}%
\pgfpathlineto{\pgfqpoint{3.109919in}{3.761849in}}%
\pgfpathlineto{\pgfqpoint{3.115980in}{3.761849in}}%
\pgfpathlineto{\pgfqpoint{3.115980in}{3.652342in}}%
\pgfpathlineto{\pgfqpoint{3.122041in}{3.652342in}}%
\pgfpathlineto{\pgfqpoint{3.122041in}{3.774990in}}%
\pgfpathlineto{\pgfqpoint{3.128102in}{3.774990in}}%
\pgfpathlineto{\pgfqpoint{3.128102in}{3.662198in}}%
\pgfpathlineto{\pgfqpoint{3.134163in}{3.662198in}}%
\pgfpathlineto{\pgfqpoint{3.134163in}{3.790869in}}%
\pgfpathlineto{\pgfqpoint{3.140224in}{3.790869in}}%
\pgfpathlineto{\pgfqpoint{3.140224in}{3.681909in}}%
\pgfpathlineto{\pgfqpoint{3.146285in}{3.681909in}}%
\pgfpathlineto{\pgfqpoint{3.146285in}{3.778823in}}%
\pgfpathlineto{\pgfqpoint{3.152346in}{3.778823in}}%
\pgfpathlineto{\pgfqpoint{3.152346in}{3.851645in}}%
\pgfpathlineto{\pgfqpoint{3.158407in}{3.851645in}}%
\pgfpathlineto{\pgfqpoint{3.158407in}{3.752541in}}%
\pgfpathlineto{\pgfqpoint{3.164468in}{3.752541in}}%
\pgfpathlineto{\pgfqpoint{3.164468in}{3.897638in}}%
\pgfpathlineto{\pgfqpoint{3.170529in}{3.897638in}}%
\pgfpathlineto{\pgfqpoint{3.170529in}{3.730092in}}%
\pgfpathlineto{\pgfqpoint{3.176590in}{3.730092in}}%
\pgfpathlineto{\pgfqpoint{3.176590in}{3.938704in}}%
\pgfpathlineto{\pgfqpoint{3.182651in}{3.938704in}}%
\pgfpathlineto{\pgfqpoint{3.182651in}{3.825911in}}%
\pgfpathlineto{\pgfqpoint{3.188712in}{3.825911in}}%
\pgfpathlineto{\pgfqpoint{3.188712in}{3.962795in}}%
\pgfpathlineto{\pgfqpoint{3.194773in}{3.962795in}}%
\pgfpathlineto{\pgfqpoint{3.194773in}{3.876832in}}%
\pgfpathlineto{\pgfqpoint{3.200834in}{3.876832in}}%
\pgfpathlineto{\pgfqpoint{3.200834in}{4.017549in}}%
\pgfpathlineto{\pgfqpoint{3.206895in}{4.017549in}}%
\pgfpathlineto{\pgfqpoint{3.206895in}{3.910232in}}%
\pgfpathlineto{\pgfqpoint{3.212956in}{3.910232in}}%
\pgfpathlineto{\pgfqpoint{3.212956in}{3.899281in}}%
\pgfpathlineto{\pgfqpoint{3.219016in}{3.899281in}}%
\pgfpathlineto{\pgfqpoint{3.219016in}{4.076135in}}%
\pgfpathlineto{\pgfqpoint{3.225077in}{4.076135in}}%
\pgfpathlineto{\pgfqpoint{3.225077in}{3.913517in}}%
\pgfpathlineto{\pgfqpoint{3.231138in}{3.913517in}}%
\pgfpathlineto{\pgfqpoint{3.231138in}{4.077230in}}%
\pgfpathlineto{\pgfqpoint{3.237199in}{4.077230in}}%
\pgfpathlineto{\pgfqpoint{3.237199in}{3.981959in}}%
\pgfpathlineto{\pgfqpoint{3.243260in}{3.981959in}}%
\pgfpathlineto{\pgfqpoint{3.243260in}{4.192213in}}%
\pgfpathlineto{\pgfqpoint{3.249321in}{4.192213in}}%
\pgfpathlineto{\pgfqpoint{3.249321in}{3.952940in}}%
\pgfpathlineto{\pgfqpoint{3.255382in}{3.952940in}}%
\pgfpathlineto{\pgfqpoint{3.255382in}{4.122676in}}%
\pgfpathlineto{\pgfqpoint{3.261443in}{4.122676in}}%
\pgfpathlineto{\pgfqpoint{3.261443in}{4.232183in}}%
\pgfpathlineto{\pgfqpoint{3.267504in}{4.232183in}}%
\pgfpathlineto{\pgfqpoint{3.267504in}{4.097489in}}%
\pgfpathlineto{\pgfqpoint{3.273565in}{4.097489in}}%
\pgfpathlineto{\pgfqpoint{3.273565in}{4.177429in}}%
\pgfpathlineto{\pgfqpoint{3.279626in}{4.177429in}}%
\pgfpathlineto{\pgfqpoint{3.279626in}{4.029047in}}%
\pgfpathlineto{\pgfqpoint{3.285687in}{4.029047in}}%
\pgfpathlineto{\pgfqpoint{3.285687in}{4.350451in}}%
\pgfpathlineto{\pgfqpoint{3.291748in}{4.350451in}}%
\pgfpathlineto{\pgfqpoint{3.291748in}{4.104060in}}%
\pgfpathlineto{\pgfqpoint{3.297809in}{4.104060in}}%
\pgfpathlineto{\pgfqpoint{3.297809in}{4.259560in}}%
\pgfpathlineto{\pgfqpoint{3.303870in}{4.259560in}}%
\pgfpathlineto{\pgfqpoint{3.303870in}{4.096942in}}%
\pgfpathlineto{\pgfqpoint{3.309931in}{4.096942in}}%
\pgfpathlineto{\pgfqpoint{3.309931in}{4.360306in}}%
\pgfpathlineto{\pgfqpoint{3.315992in}{4.360306in}}%
\pgfpathlineto{\pgfqpoint{3.315992in}{4.309386in}}%
\pgfpathlineto{\pgfqpoint{3.322053in}{4.309386in}}%
\pgfpathlineto{\pgfqpoint{3.322053in}{4.229445in}}%
\pgfpathlineto{\pgfqpoint{3.328114in}{4.229445in}}%
\pgfpathlineto{\pgfqpoint{3.328114in}{4.494453in}}%
\pgfpathlineto{\pgfqpoint{3.334175in}{4.494453in}}%
\pgfpathlineto{\pgfqpoint{3.334175in}{4.229445in}}%
\pgfpathlineto{\pgfqpoint{3.340236in}{4.229445in}}%
\pgfpathlineto{\pgfqpoint{3.340236in}{4.392063in}}%
\pgfpathlineto{\pgfqpoint{3.346297in}{4.392063in}}%
\pgfpathlineto{\pgfqpoint{3.346297in}{4.235468in}}%
\pgfpathlineto{\pgfqpoint{3.352358in}{4.235468in}}%
\pgfpathlineto{\pgfqpoint{3.352358in}{4.453388in}}%
\pgfpathlineto{\pgfqpoint{3.358419in}{4.453388in}}%
\pgfpathlineto{\pgfqpoint{3.358419in}{4.219042in}}%
\pgfpathlineto{\pgfqpoint{3.364480in}{4.219042in}}%
\pgfpathlineto{\pgfqpoint{3.364480in}{4.407942in}}%
\pgfpathlineto{\pgfqpoint{3.370541in}{4.407942in}}%
\pgfpathlineto{\pgfqpoint{3.370541in}{4.467076in}}%
\pgfpathlineto{\pgfqpoint{3.376602in}{4.467076in}}%
\pgfpathlineto{\pgfqpoint{3.376602in}{4.284199in}}%
\pgfpathlineto{\pgfqpoint{3.382663in}{4.284199in}}%
\pgfpathlineto{\pgfqpoint{3.382663in}{4.433129in}}%
\pgfpathlineto{\pgfqpoint{3.388724in}{4.433129in}}%
\pgfpathlineto{\pgfqpoint{3.388724in}{4.242586in}}%
\pgfpathlineto{\pgfqpoint{3.394785in}{4.242586in}}%
\pgfpathlineto{\pgfqpoint{3.394785in}{4.404657in}}%
\pgfpathlineto{\pgfqpoint{3.400846in}{4.404657in}}%
\pgfpathlineto{\pgfqpoint{3.400846in}{4.290222in}}%
\pgfpathlineto{\pgfqpoint{3.406907in}{4.290222in}}%
\pgfpathlineto{\pgfqpoint{3.406907in}{4.461053in}}%
\pgfpathlineto{\pgfqpoint{3.412968in}{4.461053in}}%
\pgfpathlineto{\pgfqpoint{3.412968in}{4.321431in}}%
\pgfpathlineto{\pgfqpoint{3.419029in}{4.321431in}}%
\pgfpathlineto{\pgfqpoint{3.419029in}{4.517449in}}%
\pgfpathlineto{\pgfqpoint{3.425090in}{4.517449in}}%
\pgfpathlineto{\pgfqpoint{3.425090in}{4.289127in}}%
\pgfpathlineto{\pgfqpoint{3.431151in}{4.289127in}}%
\pgfpathlineto{\pgfqpoint{3.431151in}{4.419988in}}%
\pgfpathlineto{\pgfqpoint{3.437212in}{4.419988in}}%
\pgfpathlineto{\pgfqpoint{3.437212in}{4.485692in}}%
\pgfpathlineto{\pgfqpoint{3.443273in}{4.485692in}}%
\pgfpathlineto{\pgfqpoint{3.443273in}{4.219042in}}%
\pgfpathlineto{\pgfqpoint{3.449334in}{4.219042in}}%
\pgfpathlineto{\pgfqpoint{3.449334in}{4.519639in}}%
\pgfpathlineto{\pgfqpoint{3.455395in}{4.519639in}}%
\pgfpathlineto{\pgfqpoint{3.455395in}{4.225613in}}%
\pgfpathlineto{\pgfqpoint{3.461456in}{4.225613in}}%
\pgfpathlineto{\pgfqpoint{3.461456in}{4.549206in}}%
\pgfpathlineto{\pgfqpoint{3.467517in}{4.549206in}}%
\pgfpathlineto{\pgfqpoint{3.467517in}{4.190570in}}%
\pgfpathlineto{\pgfqpoint{3.473578in}{4.190570in}}%
\pgfpathlineto{\pgfqpoint{3.473578in}{4.448460in}}%
\pgfpathlineto{\pgfqpoint{3.479639in}{4.448460in}}%
\pgfpathlineto{\pgfqpoint{3.479639in}{4.211377in}}%
\pgfpathlineto{\pgfqpoint{3.485700in}{4.211377in}}%
\pgfpathlineto{\pgfqpoint{3.485700in}{4.481312in}}%
\pgfpathlineto{\pgfqpoint{3.491761in}{4.481312in}}%
\pgfpathlineto{\pgfqpoint{3.491761in}{4.396991in}}%
\pgfpathlineto{\pgfqpoint{3.497822in}{4.396991in}}%
\pgfpathlineto{\pgfqpoint{3.497822in}{4.169216in}}%
\pgfpathlineto{\pgfqpoint{3.503883in}{4.169216in}}%
\pgfpathlineto{\pgfqpoint{3.503883in}{4.430939in}}%
\pgfpathlineto{\pgfqpoint{3.509943in}{4.430939in}}%
\pgfpathlineto{\pgfqpoint{3.509943in}{4.199331in}}%
\pgfpathlineto{\pgfqpoint{3.516004in}{4.199331in}}%
\pgfpathlineto{\pgfqpoint{3.516004in}{4.273248in}}%
\pgfpathlineto{\pgfqpoint{3.522065in}{4.273248in}}%
\pgfpathlineto{\pgfqpoint{3.522065in}{4.194403in}}%
\pgfpathlineto{\pgfqpoint{3.528126in}{4.194403in}}%
\pgfpathlineto{\pgfqpoint{3.528126in}{4.339500in}}%
\pgfpathlineto{\pgfqpoint{3.534187in}{4.339500in}}%
\pgfpathlineto{\pgfqpoint{3.534187in}{4.130341in}}%
\pgfpathlineto{\pgfqpoint{3.540248in}{4.130341in}}%
\pgfpathlineto{\pgfqpoint{3.540248in}{4.344975in}}%
\pgfpathlineto{\pgfqpoint{3.546309in}{4.344975in}}%
\pgfpathlineto{\pgfqpoint{3.546309in}{4.449555in}}%
\pgfpathlineto{\pgfqpoint{3.552370in}{4.449555in}}%
\pgfpathlineto{\pgfqpoint{3.552370in}{4.118843in}}%
\pgfpathlineto{\pgfqpoint{3.558431in}{4.118843in}}%
\pgfpathlineto{\pgfqpoint{3.558431in}{4.267773in}}%
\pgfpathlineto{\pgfqpoint{3.564492in}{4.267773in}}%
\pgfpathlineto{\pgfqpoint{3.564492in}{4.094751in}}%
\pgfpathlineto{\pgfqpoint{3.570553in}{4.094751in}}%
\pgfpathlineto{\pgfqpoint{3.570553in}{4.280366in}}%
\pgfpathlineto{\pgfqpoint{3.576614in}{4.280366in}}%
\pgfpathlineto{\pgfqpoint{3.576614in}{4.048211in}}%
\pgfpathlineto{\pgfqpoint{3.582675in}{4.048211in}}%
\pgfpathlineto{\pgfqpoint{3.582675in}{4.228350in}}%
\pgfpathlineto{\pgfqpoint{3.588736in}{4.228350in}}%
\pgfpathlineto{\pgfqpoint{3.588736in}{4.047116in}}%
\pgfpathlineto{\pgfqpoint{3.594797in}{4.047116in}}%
\pgfpathlineto{\pgfqpoint{3.594797in}{4.282556in}}%
\pgfpathlineto{\pgfqpoint{3.600858in}{4.282556in}}%
\pgfpathlineto{\pgfqpoint{3.600858in}{4.134174in}}%
\pgfpathlineto{\pgfqpoint{3.606919in}{4.134174in}}%
\pgfpathlineto{\pgfqpoint{3.606919in}{3.972103in}}%
\pgfpathlineto{\pgfqpoint{3.612980in}{3.972103in}}%
\pgfpathlineto{\pgfqpoint{3.612980in}{4.159361in}}%
\pgfpathlineto{\pgfqpoint{3.619041in}{4.159361in}}%
\pgfpathlineto{\pgfqpoint{3.619041in}{3.950749in}}%
\pgfpathlineto{\pgfqpoint{3.625102in}{3.950749in}}%
\pgfpathlineto{\pgfqpoint{3.625102in}{4.062447in}}%
\pgfpathlineto{\pgfqpoint{3.631163in}{4.062447in}}%
\pgfpathlineto{\pgfqpoint{3.631163in}{3.956225in}}%
\pgfpathlineto{\pgfqpoint{3.637224in}{3.956225in}}%
\pgfpathlineto{\pgfqpoint{3.637224in}{4.073945in}}%
\pgfpathlineto{\pgfqpoint{3.643285in}{4.073945in}}%
\pgfpathlineto{\pgfqpoint{3.643285in}{3.924468in}}%
\pgfpathlineto{\pgfqpoint{3.649346in}{3.924468in}}%
\pgfpathlineto{\pgfqpoint{3.649346in}{4.009336in}}%
\pgfpathlineto{\pgfqpoint{3.655407in}{4.009336in}}%
\pgfpathlineto{\pgfqpoint{3.655407in}{4.039998in}}%
\pgfpathlineto{\pgfqpoint{3.661468in}{4.039998in}}%
\pgfpathlineto{\pgfqpoint{3.661468in}{3.803462in}}%
\pgfpathlineto{\pgfqpoint{3.667529in}{3.803462in}}%
\pgfpathlineto{\pgfqpoint{3.667529in}{3.909137in}}%
\pgfpathlineto{\pgfqpoint{3.673590in}{3.909137in}}%
\pgfpathlineto{\pgfqpoint{3.673590in}{3.763492in}}%
\pgfpathlineto{\pgfqpoint{3.679651in}{3.763492in}}%
\pgfpathlineto{\pgfqpoint{3.679651in}{3.941989in}}%
\pgfpathlineto{\pgfqpoint{3.685712in}{3.941989in}}%
\pgfpathlineto{\pgfqpoint{3.685712in}{3.738305in}}%
\pgfpathlineto{\pgfqpoint{3.691773in}{3.738305in}}%
\pgfpathlineto{\pgfqpoint{3.691773in}{3.813318in}}%
\pgfpathlineto{\pgfqpoint{3.697834in}{3.813318in}}%
\pgfpathlineto{\pgfqpoint{3.697834in}{3.744876in}}%
\pgfpathlineto{\pgfqpoint{3.703895in}{3.744876in}}%
\pgfpathlineto{\pgfqpoint{3.703895in}{3.833577in}}%
\pgfpathlineto{\pgfqpoint{3.709956in}{3.833577in}}%
\pgfpathlineto{\pgfqpoint{3.709956in}{3.779918in}}%
\pgfpathlineto{\pgfqpoint{3.716017in}{3.779918in}}%
\pgfpathlineto{\pgfqpoint{3.716017in}{3.649057in}}%
\pgfpathlineto{\pgfqpoint{3.722078in}{3.649057in}}%
\pgfpathlineto{\pgfqpoint{3.722078in}{3.776633in}}%
\pgfpathlineto{\pgfqpoint{3.728139in}{3.776633in}}%
\pgfpathlineto{\pgfqpoint{3.728139in}{3.615657in}}%
\pgfpathlineto{\pgfqpoint{3.734200in}{3.615657in}}%
\pgfpathlineto{\pgfqpoint{3.734200in}{3.696145in}}%
\pgfpathlineto{\pgfqpoint{3.740261in}{3.696145in}}%
\pgfpathlineto{\pgfqpoint{3.740261in}{3.597041in}}%
\pgfpathlineto{\pgfqpoint{3.746322in}{3.597041in}}%
\pgfpathlineto{\pgfqpoint{3.746322in}{3.690122in}}%
\pgfpathlineto{\pgfqpoint{3.752383in}{3.690122in}}%
\pgfpathlineto{\pgfqpoint{3.752383in}{3.521481in}}%
\pgfpathlineto{\pgfqpoint{3.758444in}{3.521481in}}%
\pgfpathlineto{\pgfqpoint{3.758444in}{3.636464in}}%
\pgfpathlineto{\pgfqpoint{3.764505in}{3.636464in}}%
\pgfpathlineto{\pgfqpoint{3.764505in}{3.530242in}}%
\pgfpathlineto{\pgfqpoint{3.770566in}{3.530242in}}%
\pgfpathlineto{\pgfqpoint{3.770566in}{3.620585in}}%
\pgfpathlineto{\pgfqpoint{3.776627in}{3.620585in}}%
\pgfpathlineto{\pgfqpoint{3.776627in}{3.561451in}}%
\pgfpathlineto{\pgfqpoint{3.782688in}{3.561451in}}%
\pgfpathlineto{\pgfqpoint{3.782688in}{3.523671in}}%
\pgfpathlineto{\pgfqpoint{3.788749in}{3.523671in}}%
\pgfpathlineto{\pgfqpoint{3.788749in}{3.591018in}}%
\pgfpathlineto{\pgfqpoint{3.794810in}{3.591018in}}%
\pgfpathlineto{\pgfqpoint{3.794810in}{3.483154in}}%
\pgfpathlineto{\pgfqpoint{3.800870in}{3.483154in}}%
\pgfpathlineto{\pgfqpoint{3.800870in}{3.517648in}}%
\pgfpathlineto{\pgfqpoint{3.806931in}{3.517648in}}%
\pgfpathlineto{\pgfqpoint{3.806931in}{3.443731in}}%
\pgfpathlineto{\pgfqpoint{3.812992in}{3.443731in}}%
\pgfpathlineto{\pgfqpoint{3.812992in}{3.524219in}}%
\pgfpathlineto{\pgfqpoint{3.819053in}{3.524219in}}%
\pgfpathlineto{\pgfqpoint{3.819053in}{3.416902in}}%
\pgfpathlineto{\pgfqpoint{3.825114in}{3.416902in}}%
\pgfpathlineto{\pgfqpoint{3.825114in}{3.465632in}}%
\pgfpathlineto{\pgfqpoint{3.831175in}{3.465632in}}%
\pgfpathlineto{\pgfqpoint{3.831175in}{3.493557in}}%
\pgfpathlineto{\pgfqpoint{3.837236in}{3.493557in}}%
\pgfpathlineto{\pgfqpoint{3.837236in}{3.415259in}}%
\pgfpathlineto{\pgfqpoint{3.843297in}{3.415259in}}%
\pgfpathlineto{\pgfqpoint{3.843297in}{3.444279in}}%
\pgfpathlineto{\pgfqpoint{3.849358in}{3.444279in}}%
\pgfpathlineto{\pgfqpoint{3.849358in}{3.380217in}}%
\pgfpathlineto{\pgfqpoint{3.855419in}{3.380217in}}%
\pgfpathlineto{\pgfqpoint{3.855419in}{3.414164in}}%
\pgfpathlineto{\pgfqpoint{3.861480in}{3.414164in}}%
\pgfpathlineto{\pgfqpoint{3.861480in}{3.352292in}}%
\pgfpathlineto{\pgfqpoint{3.867541in}{3.352292in}}%
\pgfpathlineto{\pgfqpoint{3.867541in}{3.407594in}}%
\pgfpathlineto{\pgfqpoint{3.873602in}{3.407594in}}%
\pgfpathlineto{\pgfqpoint{3.873602in}{3.335319in}}%
\pgfpathlineto{\pgfqpoint{3.879663in}{3.335319in}}%
\pgfpathlineto{\pgfqpoint{3.879663in}{3.395000in}}%
\pgfpathlineto{\pgfqpoint{3.885724in}{3.395000in}}%
\pgfpathlineto{\pgfqpoint{3.885724in}{3.385145in}}%
\pgfpathlineto{\pgfqpoint{3.891785in}{3.385145in}}%
\pgfpathlineto{\pgfqpoint{3.891785in}{3.311775in}}%
\pgfpathlineto{\pgfqpoint{3.897846in}{3.311775in}}%
\pgfpathlineto{\pgfqpoint{3.897846in}{3.370909in}}%
\pgfpathlineto{\pgfqpoint{3.903907in}{3.370909in}}%
\pgfpathlineto{\pgfqpoint{3.903907in}{3.299729in}}%
\pgfpathlineto{\pgfqpoint{3.909968in}{3.299729in}}%
\pgfpathlineto{\pgfqpoint{3.909968in}{3.347912in}}%
\pgfpathlineto{\pgfqpoint{3.916029in}{3.347912in}}%
\pgfpathlineto{\pgfqpoint{3.916029in}{3.302467in}}%
\pgfpathlineto{\pgfqpoint{3.922090in}{3.302467in}}%
\pgfpathlineto{\pgfqpoint{3.922090in}{3.334771in}}%
\pgfpathlineto{\pgfqpoint{3.928151in}{3.334771in}}%
\pgfpathlineto{\pgfqpoint{3.928151in}{3.281113in}}%
\pgfpathlineto{\pgfqpoint{3.934212in}{3.281113in}}%
\pgfpathlineto{\pgfqpoint{3.934212in}{3.300824in}}%
\pgfpathlineto{\pgfqpoint{3.940273in}{3.300824in}}%
\pgfpathlineto{\pgfqpoint{3.940273in}{3.317250in}}%
\pgfpathlineto{\pgfqpoint{3.946334in}{3.317250in}}%
\pgfpathlineto{\pgfqpoint{3.946334in}{3.257569in}}%
\pgfpathlineto{\pgfqpoint{3.952395in}{3.257569in}}%
\pgfpathlineto{\pgfqpoint{3.952395in}{3.306847in}}%
\pgfpathlineto{\pgfqpoint{3.958456in}{3.306847in}}%
\pgfpathlineto{\pgfqpoint{3.958456in}{3.250451in}}%
\pgfpathlineto{\pgfqpoint{3.964517in}{3.250451in}}%
\pgfpathlineto{\pgfqpoint{3.964517in}{3.289873in}}%
\pgfpathlineto{\pgfqpoint{3.970578in}{3.289873in}}%
\pgfpathlineto{\pgfqpoint{3.970578in}{3.231835in}}%
\pgfpathlineto{\pgfqpoint{3.976639in}{3.231835in}}%
\pgfpathlineto{\pgfqpoint{3.976639in}{3.273995in}}%
\pgfpathlineto{\pgfqpoint{3.982700in}{3.273995in}}%
\pgfpathlineto{\pgfqpoint{3.982700in}{3.221431in}}%
\pgfpathlineto{\pgfqpoint{3.988761in}{3.221431in}}%
\pgfpathlineto{\pgfqpoint{3.988761in}{3.265782in}}%
\pgfpathlineto{\pgfqpoint{3.994822in}{3.265782in}}%
\pgfpathlineto{\pgfqpoint{3.994822in}{3.236762in}}%
\pgfpathlineto{\pgfqpoint{4.000883in}{3.236762in}}%
\pgfpathlineto{\pgfqpoint{4.000883in}{3.225264in}}%
\pgfpathlineto{\pgfqpoint{4.006944in}{3.225264in}}%
\pgfpathlineto{\pgfqpoint{4.006944in}{3.238405in}}%
\pgfpathlineto{\pgfqpoint{4.013005in}{3.238405in}}%
\pgfpathlineto{\pgfqpoint{4.013005in}{3.225264in}}%
\pgfpathlineto{\pgfqpoint{4.025127in}{3.226359in}}%
\pgfpathlineto{\pgfqpoint{4.025127in}{3.211028in}}%
\pgfpathlineto{\pgfqpoint{4.031188in}{3.211028in}}%
\pgfpathlineto{\pgfqpoint{4.031188in}{3.223621in}}%
\pgfpathlineto{\pgfqpoint{4.037249in}{3.223621in}}%
\pgfpathlineto{\pgfqpoint{4.037249in}{3.181461in}}%
\pgfpathlineto{\pgfqpoint{4.043310in}{3.181461in}}%
\pgfpathlineto{\pgfqpoint{4.043310in}{3.186937in}}%
\pgfpathlineto{\pgfqpoint{4.049371in}{3.186937in}}%
\pgfpathlineto{\pgfqpoint{4.049371in}{3.223621in}}%
\pgfpathlineto{\pgfqpoint{4.055432in}{3.223621in}}%
\pgfpathlineto{\pgfqpoint{4.055432in}{3.182009in}}%
\pgfpathlineto{\pgfqpoint{4.061493in}{3.182009in}}%
\pgfpathlineto{\pgfqpoint{4.061493in}{3.218146in}}%
\pgfpathlineto{\pgfqpoint{4.067554in}{3.218146in}}%
\pgfpathlineto{\pgfqpoint{4.067554in}{3.184199in}}%
\pgfpathlineto{\pgfqpoint{4.073615in}{3.184199in}}%
\pgfpathlineto{\pgfqpoint{4.073615in}{3.200077in}}%
\pgfpathlineto{\pgfqpoint{4.079676in}{3.200077in}}%
\pgfpathlineto{\pgfqpoint{4.079676in}{3.157917in}}%
\pgfpathlineto{\pgfqpoint{4.085737in}{3.157917in}}%
\pgfpathlineto{\pgfqpoint{4.085737in}{3.177081in}}%
\pgfpathlineto{\pgfqpoint{4.091797in}{3.177081in}}%
\pgfpathlineto{\pgfqpoint{4.091797in}{3.152989in}}%
\pgfpathlineto{\pgfqpoint{4.097858in}{3.152989in}}%
\pgfpathlineto{\pgfqpoint{4.097858in}{3.199530in}}%
\pgfpathlineto{\pgfqpoint{4.103919in}{3.199530in}}%
\pgfpathlineto{\pgfqpoint{4.103919in}{3.162297in}}%
\pgfpathlineto{\pgfqpoint{4.109980in}{3.162297in}}%
\pgfpathlineto{\pgfqpoint{4.109980in}{3.152442in}}%
\pgfpathlineto{\pgfqpoint{4.116041in}{3.152442in}}%
\pgfpathlineto{\pgfqpoint{4.116041in}{3.178724in}}%
\pgfpathlineto{\pgfqpoint{4.122102in}{3.178724in}}%
\pgfpathlineto{\pgfqpoint{4.122102in}{3.158465in}}%
\pgfpathlineto{\pgfqpoint{4.134224in}{3.158465in}}%
\pgfpathlineto{\pgfqpoint{4.134224in}{3.148062in}}%
\pgfpathlineto{\pgfqpoint{4.140285in}{3.148062in}}%
\pgfpathlineto{\pgfqpoint{4.140285in}{3.170510in}}%
\pgfpathlineto{\pgfqpoint{4.146346in}{3.170510in}}%
\pgfpathlineto{\pgfqpoint{4.146346in}{3.136563in}}%
\pgfpathlineto{\pgfqpoint{4.152407in}{3.136563in}}%
\pgfpathlineto{\pgfqpoint{4.152407in}{3.154632in}}%
\pgfpathlineto{\pgfqpoint{4.158468in}{3.154632in}}%
\pgfpathlineto{\pgfqpoint{4.158468in}{3.131635in}}%
\pgfpathlineto{\pgfqpoint{4.164529in}{3.131635in}}%
\pgfpathlineto{\pgfqpoint{4.164529in}{3.151347in}}%
\pgfpathlineto{\pgfqpoint{4.170590in}{3.151347in}}%
\pgfpathlineto{\pgfqpoint{4.170590in}{3.138753in}}%
\pgfpathlineto{\pgfqpoint{4.176651in}{3.138753in}}%
\pgfpathlineto{\pgfqpoint{4.176651in}{3.129993in}}%
\pgfpathlineto{\pgfqpoint{4.182712in}{3.129993in}}%
\pgfpathlineto{\pgfqpoint{4.182712in}{3.136563in}}%
\pgfpathlineto{\pgfqpoint{4.188773in}{3.136563in}}%
\pgfpathlineto{\pgfqpoint{4.188773in}{3.119590in}}%
\pgfpathlineto{\pgfqpoint{4.194834in}{3.119590in}}%
\pgfpathlineto{\pgfqpoint{4.194834in}{3.128898in}}%
\pgfpathlineto{\pgfqpoint{4.200895in}{3.128898in}}%
\pgfpathlineto{\pgfqpoint{4.200895in}{3.118495in}}%
\pgfpathlineto{\pgfqpoint{4.206956in}{3.118495in}}%
\pgfpathlineto{\pgfqpoint{4.206956in}{3.124517in}}%
\pgfpathlineto{\pgfqpoint{4.213017in}{3.124517in}}%
\pgfpathlineto{\pgfqpoint{4.213017in}{3.111377in}}%
\pgfpathlineto{\pgfqpoint{4.219078in}{3.111377in}}%
\pgfpathlineto{\pgfqpoint{4.219078in}{3.132183in}}%
\pgfpathlineto{\pgfqpoint{4.225139in}{3.132183in}}%
\pgfpathlineto{\pgfqpoint{4.225139in}{3.141491in}}%
\pgfpathlineto{\pgfqpoint{4.231200in}{3.141491in}}%
\pgfpathlineto{\pgfqpoint{4.231200in}{3.102616in}}%
\pgfpathlineto{\pgfqpoint{4.237261in}{3.102616in}}%
\pgfpathlineto{\pgfqpoint{4.237261in}{3.108639in}}%
\pgfpathlineto{\pgfqpoint{4.249383in}{3.109734in}}%
\pgfpathlineto{\pgfqpoint{4.249383in}{3.114662in}}%
\pgfpathlineto{\pgfqpoint{4.255444in}{3.114662in}}%
\pgfpathlineto{\pgfqpoint{4.255444in}{3.104259in}}%
\pgfpathlineto{\pgfqpoint{4.261505in}{3.104259in}}%
\pgfpathlineto{\pgfqpoint{4.261505in}{3.106449in}}%
\pgfpathlineto{\pgfqpoint{4.267566in}{3.106449in}}%
\pgfpathlineto{\pgfqpoint{4.267566in}{3.098783in}}%
\pgfpathlineto{\pgfqpoint{4.273627in}{3.098783in}}%
\pgfpathlineto{\pgfqpoint{4.273627in}{3.122875in}}%
\pgfpathlineto{\pgfqpoint{4.279688in}{3.122875in}}%
\pgfpathlineto{\pgfqpoint{4.279688in}{3.106996in}}%
\pgfpathlineto{\pgfqpoint{4.285749in}{3.106996in}}%
\pgfpathlineto{\pgfqpoint{4.285749in}{3.079072in}}%
\pgfpathlineto{\pgfqpoint{4.291810in}{3.079072in}}%
\pgfpathlineto{\pgfqpoint{4.291810in}{3.100426in}}%
\pgfpathlineto{\pgfqpoint{4.297871in}{3.100426in}}%
\pgfpathlineto{\pgfqpoint{4.297871in}{3.097688in}}%
\pgfpathlineto{\pgfqpoint{4.303932in}{3.097688in}}%
\pgfpathlineto{\pgfqpoint{4.303932in}{3.085095in}}%
\pgfpathlineto{\pgfqpoint{4.309993in}{3.085095in}}%
\pgfpathlineto{\pgfqpoint{4.309993in}{3.075787in}}%
\pgfpathlineto{\pgfqpoint{4.316054in}{3.075787in}}%
\pgfpathlineto{\pgfqpoint{4.316054in}{3.089475in}}%
\pgfpathlineto{\pgfqpoint{4.322115in}{3.089475in}}%
\pgfpathlineto{\pgfqpoint{4.322115in}{3.073049in}}%
\pgfpathlineto{\pgfqpoint{4.328176in}{3.073049in}}%
\pgfpathlineto{\pgfqpoint{4.328176in}{3.082905in}}%
\pgfpathlineto{\pgfqpoint{4.334237in}{3.082905in}}%
\pgfpathlineto{\pgfqpoint{4.334237in}{3.109734in}}%
\pgfpathlineto{\pgfqpoint{4.340298in}{3.109734in}}%
\pgfpathlineto{\pgfqpoint{4.340298in}{3.074144in}}%
\pgfpathlineto{\pgfqpoint{4.346359in}{3.074144in}}%
\pgfpathlineto{\pgfqpoint{4.346359in}{3.090570in}}%
\pgfpathlineto{\pgfqpoint{4.352420in}{3.090570in}}%
\pgfpathlineto{\pgfqpoint{4.352420in}{3.071406in}}%
\pgfpathlineto{\pgfqpoint{4.358481in}{3.071406in}}%
\pgfpathlineto{\pgfqpoint{4.358481in}{3.088928in}}%
\pgfpathlineto{\pgfqpoint{4.364542in}{3.088928in}}%
\pgfpathlineto{\pgfqpoint{4.364542in}{3.066479in}}%
\pgfpathlineto{\pgfqpoint{4.370603in}{3.066479in}}%
\pgfpathlineto{\pgfqpoint{4.370603in}{3.090023in}}%
\pgfpathlineto{\pgfqpoint{4.376664in}{3.090023in}}%
\pgfpathlineto{\pgfqpoint{4.376664in}{3.067026in}}%
\pgfpathlineto{\pgfqpoint{4.382724in}{3.067026in}}%
\pgfpathlineto{\pgfqpoint{4.382724in}{3.087285in}}%
\pgfpathlineto{\pgfqpoint{4.388785in}{3.087285in}}%
\pgfpathlineto{\pgfqpoint{4.388785in}{3.076882in}}%
\pgfpathlineto{\pgfqpoint{4.394846in}{3.076882in}}%
\pgfpathlineto{\pgfqpoint{4.394846in}{3.060456in}}%
\pgfpathlineto{\pgfqpoint{4.400907in}{3.060456in}}%
\pgfpathlineto{\pgfqpoint{4.400907in}{3.069764in}}%
\pgfpathlineto{\pgfqpoint{4.419090in}{3.069216in}}%
\pgfpathlineto{\pgfqpoint{4.419090in}{3.068121in}}%
\pgfpathlineto{\pgfqpoint{4.425151in}{3.068121in}}%
\pgfpathlineto{\pgfqpoint{4.425151in}{3.073049in}}%
\pgfpathlineto{\pgfqpoint{4.431212in}{3.073049in}}%
\pgfpathlineto{\pgfqpoint{4.431212in}{3.061003in}}%
\pgfpathlineto{\pgfqpoint{4.437273in}{3.061003in}}%
\pgfpathlineto{\pgfqpoint{4.437273in}{3.063741in}}%
\pgfpathlineto{\pgfqpoint{4.443334in}{3.063741in}}%
\pgfpathlineto{\pgfqpoint{4.443334in}{3.047315in}}%
\pgfpathlineto{\pgfqpoint{4.449395in}{3.047315in}}%
\pgfpathlineto{\pgfqpoint{4.449395in}{3.063193in}}%
\pgfpathlineto{\pgfqpoint{4.455456in}{3.063193in}}%
\pgfpathlineto{\pgfqpoint{4.455456in}{3.061003in}}%
\pgfpathlineto{\pgfqpoint{4.461517in}{3.061003in}}%
\pgfpathlineto{\pgfqpoint{4.461517in}{3.042935in}}%
\pgfpathlineto{\pgfqpoint{4.467578in}{3.042935in}}%
\pgfpathlineto{\pgfqpoint{4.467578in}{3.070311in}}%
\pgfpathlineto{\pgfqpoint{4.473639in}{3.070311in}}%
\pgfpathlineto{\pgfqpoint{4.473639in}{3.046220in}}%
\pgfpathlineto{\pgfqpoint{4.479700in}{3.046220in}}%
\pgfpathlineto{\pgfqpoint{4.479700in}{3.053338in}}%
\pgfpathlineto{\pgfqpoint{4.485761in}{3.053338in}}%
\pgfpathlineto{\pgfqpoint{4.485761in}{3.048957in}}%
\pgfpathlineto{\pgfqpoint{4.491822in}{3.048957in}}%
\pgfpathlineto{\pgfqpoint{4.491822in}{3.056623in}}%
\pgfpathlineto{\pgfqpoint{4.497883in}{3.056623in}}%
\pgfpathlineto{\pgfqpoint{4.497883in}{3.039649in}}%
\pgfpathlineto{\pgfqpoint{4.503944in}{3.039649in}}%
\pgfpathlineto{\pgfqpoint{4.503944in}{3.058266in}}%
\pgfpathlineto{\pgfqpoint{4.516066in}{3.058266in}}%
\pgfpathlineto{\pgfqpoint{4.516066in}{3.038554in}}%
\pgfpathlineto{\pgfqpoint{4.522127in}{3.038554in}}%
\pgfpathlineto{\pgfqpoint{4.522127in}{3.058813in}}%
\pgfpathlineto{\pgfqpoint{4.528188in}{3.058813in}}%
\pgfpathlineto{\pgfqpoint{4.528188in}{3.041292in}}%
\pgfpathlineto{\pgfqpoint{4.534249in}{3.041292in}}%
\pgfpathlineto{\pgfqpoint{4.534249in}{3.061551in}}%
\pgfpathlineto{\pgfqpoint{4.540310in}{3.061551in}}%
\pgfpathlineto{\pgfqpoint{4.540310in}{3.035269in}}%
\pgfpathlineto{\pgfqpoint{4.546371in}{3.035269in}}%
\pgfpathlineto{\pgfqpoint{4.546371in}{3.067026in}}%
\pgfpathlineto{\pgfqpoint{4.552432in}{3.067026in}}%
\pgfpathlineto{\pgfqpoint{4.552432in}{3.041840in}}%
\pgfpathlineto{\pgfqpoint{4.558493in}{3.041840in}}%
\pgfpathlineto{\pgfqpoint{4.558493in}{3.061551in}}%
\pgfpathlineto{\pgfqpoint{4.564554in}{3.061551in}}%
\pgfpathlineto{\pgfqpoint{4.564554in}{3.057171in}}%
\pgfpathlineto{\pgfqpoint{4.570615in}{3.057171in}}%
\pgfpathlineto{\pgfqpoint{4.570615in}{3.039649in}}%
\pgfpathlineto{\pgfqpoint{4.582737in}{3.038554in}}%
\pgfpathlineto{\pgfqpoint{4.582737in}{3.045125in}}%
\pgfpathlineto{\pgfqpoint{4.588798in}{3.045125in}}%
\pgfpathlineto{\pgfqpoint{4.588798in}{3.038007in}}%
\pgfpathlineto{\pgfqpoint{4.594859in}{3.038007in}}%
\pgfpathlineto{\pgfqpoint{4.594859in}{3.028151in}}%
\pgfpathlineto{\pgfqpoint{4.600920in}{3.028151in}}%
\pgfpathlineto{\pgfqpoint{4.600920in}{3.051695in}}%
\pgfpathlineto{\pgfqpoint{4.606981in}{3.051695in}}%
\pgfpathlineto{\pgfqpoint{4.606981in}{3.030889in}}%
\pgfpathlineto{\pgfqpoint{4.613042in}{3.030889in}}%
\pgfpathlineto{\pgfqpoint{4.613042in}{3.046220in}}%
\pgfpathlineto{\pgfqpoint{4.625164in}{3.045672in}}%
\pgfpathlineto{\pgfqpoint{4.625164in}{3.024318in}}%
\pgfpathlineto{\pgfqpoint{4.631225in}{3.024318in}}%
\pgfpathlineto{\pgfqpoint{4.631225in}{3.042387in}}%
\pgfpathlineto{\pgfqpoint{4.637286in}{3.042387in}}%
\pgfpathlineto{\pgfqpoint{4.637286in}{3.031436in}}%
\pgfpathlineto{\pgfqpoint{4.643347in}{3.031436in}}%
\pgfpathlineto{\pgfqpoint{4.643347in}{3.041840in}}%
\pgfpathlineto{\pgfqpoint{4.649408in}{3.041840in}}%
\pgfpathlineto{\pgfqpoint{4.649408in}{3.035269in}}%
\pgfpathlineto{\pgfqpoint{4.655469in}{3.035269in}}%
\pgfpathlineto{\pgfqpoint{4.655469in}{3.041292in}}%
\pgfpathlineto{\pgfqpoint{4.661530in}{3.041292in}}%
\pgfpathlineto{\pgfqpoint{4.661530in}{3.033079in}}%
\pgfpathlineto{\pgfqpoint{4.667591in}{3.033079in}}%
\pgfpathlineto{\pgfqpoint{4.667591in}{3.052790in}}%
\pgfpathlineto{\pgfqpoint{4.673651in}{3.052790in}}%
\pgfpathlineto{\pgfqpoint{4.673651in}{3.035817in}}%
\pgfpathlineto{\pgfqpoint{4.679712in}{3.035817in}}%
\pgfpathlineto{\pgfqpoint{4.679712in}{3.039102in}}%
\pgfpathlineto{\pgfqpoint{4.685773in}{3.039102in}}%
\pgfpathlineto{\pgfqpoint{4.685773in}{3.042935in}}%
\pgfpathlineto{\pgfqpoint{4.691834in}{3.042935in}}%
\pgfpathlineto{\pgfqpoint{4.691834in}{3.029246in}}%
\pgfpathlineto{\pgfqpoint{4.703956in}{3.028151in}}%
\pgfpathlineto{\pgfqpoint{4.703956in}{3.023771in}}%
\pgfpathlineto{\pgfqpoint{4.710017in}{3.023771in}}%
\pgfpathlineto{\pgfqpoint{4.710017in}{3.051695in}}%
\pgfpathlineto{\pgfqpoint{4.716078in}{3.051695in}}%
\pgfpathlineto{\pgfqpoint{4.716078in}{3.025413in}}%
\pgfpathlineto{\pgfqpoint{4.722139in}{3.025413in}}%
\pgfpathlineto{\pgfqpoint{4.722139in}{3.035817in}}%
\pgfpathlineto{\pgfqpoint{4.728200in}{3.035817in}}%
\pgfpathlineto{\pgfqpoint{4.728200in}{3.045125in}}%
\pgfpathlineto{\pgfqpoint{4.734261in}{3.045125in}}%
\pgfpathlineto{\pgfqpoint{4.734261in}{3.031984in}}%
\pgfpathlineto{\pgfqpoint{4.758505in}{3.031436in}}%
\pgfpathlineto{\pgfqpoint{4.758505in}{3.030341in}}%
\pgfpathlineto{\pgfqpoint{4.764566in}{3.030341in}}%
\pgfpathlineto{\pgfqpoint{4.764566in}{3.036912in}}%
\pgfpathlineto{\pgfqpoint{4.770627in}{3.036912in}}%
\pgfpathlineto{\pgfqpoint{4.770627in}{3.019391in}}%
\pgfpathlineto{\pgfqpoint{4.776688in}{3.019391in}}%
\pgfpathlineto{\pgfqpoint{4.776688in}{3.032531in}}%
\pgfpathlineto{\pgfqpoint{4.782749in}{3.032531in}}%
\pgfpathlineto{\pgfqpoint{4.782749in}{3.015558in}}%
\pgfpathlineto{\pgfqpoint{4.788810in}{3.015558in}}%
\pgfpathlineto{\pgfqpoint{4.788810in}{3.028151in}}%
\pgfpathlineto{\pgfqpoint{4.794871in}{3.028151in}}%
\pgfpathlineto{\pgfqpoint{4.794871in}{3.024866in}}%
\pgfpathlineto{\pgfqpoint{4.800932in}{3.024866in}}%
\pgfpathlineto{\pgfqpoint{4.800932in}{3.021581in}}%
\pgfpathlineto{\pgfqpoint{4.813054in}{3.022676in}}%
\pgfpathlineto{\pgfqpoint{4.813054in}{3.013368in}}%
\pgfpathlineto{\pgfqpoint{4.819115in}{3.013368in}}%
\pgfpathlineto{\pgfqpoint{4.819115in}{3.017748in}}%
\pgfpathlineto{\pgfqpoint{4.825176in}{3.017748in}}%
\pgfpathlineto{\pgfqpoint{4.825176in}{3.019391in}}%
\pgfpathlineto{\pgfqpoint{4.831237in}{3.019391in}}%
\pgfpathlineto{\pgfqpoint{4.831237in}{3.024318in}}%
\pgfpathlineto{\pgfqpoint{4.837298in}{3.024318in}}%
\pgfpathlineto{\pgfqpoint{4.837298in}{3.013368in}}%
\pgfpathlineto{\pgfqpoint{4.843359in}{3.013368in}}%
\pgfpathlineto{\pgfqpoint{4.843359in}{3.020486in}}%
\pgfpathlineto{\pgfqpoint{4.849420in}{3.020486in}}%
\pgfpathlineto{\pgfqpoint{4.849420in}{3.027604in}}%
\pgfpathlineto{\pgfqpoint{4.861542in}{3.027056in}}%
\pgfpathlineto{\pgfqpoint{4.861542in}{3.019391in}}%
\pgfpathlineto{\pgfqpoint{4.873664in}{3.019391in}}%
\pgfpathlineto{\pgfqpoint{4.873664in}{3.023223in}}%
\pgfpathlineto{\pgfqpoint{4.879725in}{3.023223in}}%
\pgfpathlineto{\pgfqpoint{4.879725in}{3.018843in}}%
\pgfpathlineto{\pgfqpoint{4.885786in}{3.018843in}}%
\pgfpathlineto{\pgfqpoint{4.885786in}{3.023771in}}%
\pgfpathlineto{\pgfqpoint{4.891847in}{3.023771in}}%
\pgfpathlineto{\pgfqpoint{4.891847in}{3.015558in}}%
\pgfpathlineto{\pgfqpoint{4.897908in}{3.015558in}}%
\pgfpathlineto{\pgfqpoint{4.897908in}{3.025961in}}%
\pgfpathlineto{\pgfqpoint{4.903969in}{3.025961in}}%
\pgfpathlineto{\pgfqpoint{4.903969in}{3.023771in}}%
\pgfpathlineto{\pgfqpoint{4.910030in}{3.023771in}}%
\pgfpathlineto{\pgfqpoint{4.910030in}{3.020486in}}%
\pgfpathlineto{\pgfqpoint{4.916091in}{3.020486in}}%
\pgfpathlineto{\pgfqpoint{4.916091in}{3.018295in}}%
\pgfpathlineto{\pgfqpoint{4.922152in}{3.018295in}}%
\pgfpathlineto{\pgfqpoint{4.922152in}{3.024318in}}%
\pgfpathlineto{\pgfqpoint{4.928213in}{3.024318in}}%
\pgfpathlineto{\pgfqpoint{4.928213in}{3.021033in}}%
\pgfpathlineto{\pgfqpoint{4.934274in}{3.021033in}}%
\pgfpathlineto{\pgfqpoint{4.934274in}{3.015010in}}%
\pgfpathlineto{\pgfqpoint{4.940335in}{3.015010in}}%
\pgfpathlineto{\pgfqpoint{4.940335in}{3.021033in}}%
\pgfpathlineto{\pgfqpoint{4.946396in}{3.021033in}}%
\pgfpathlineto{\pgfqpoint{4.946396in}{3.010630in}}%
\pgfpathlineto{\pgfqpoint{4.952457in}{3.010630in}}%
\pgfpathlineto{\pgfqpoint{4.952457in}{3.023771in}}%
\pgfpathlineto{\pgfqpoint{4.958518in}{3.023771in}}%
\pgfpathlineto{\pgfqpoint{4.958518in}{3.031436in}}%
\pgfpathlineto{\pgfqpoint{4.964578in}{3.031436in}}%
\pgfpathlineto{\pgfqpoint{4.964578in}{3.011725in}}%
\pgfpathlineto{\pgfqpoint{4.970639in}{3.011725in}}%
\pgfpathlineto{\pgfqpoint{4.970639in}{3.025961in}}%
\pgfpathlineto{\pgfqpoint{4.976700in}{3.025961in}}%
\pgfpathlineto{\pgfqpoint{4.976700in}{3.012273in}}%
\pgfpathlineto{\pgfqpoint{4.982761in}{3.012273in}}%
\pgfpathlineto{\pgfqpoint{4.982761in}{3.015558in}}%
\pgfpathlineto{\pgfqpoint{4.988822in}{3.015558in}}%
\pgfpathlineto{\pgfqpoint{4.988822in}{3.004607in}}%
\pgfpathlineto{\pgfqpoint{4.994883in}{3.004607in}}%
\pgfpathlineto{\pgfqpoint{4.994883in}{3.018843in}}%
\pgfpathlineto{\pgfqpoint{5.000944in}{3.018843in}}%
\pgfpathlineto{\pgfqpoint{5.000944in}{3.006250in}}%
\pgfpathlineto{\pgfqpoint{5.007005in}{3.006250in}}%
\pgfpathlineto{\pgfqpoint{5.007005in}{3.016105in}}%
\pgfpathlineto{\pgfqpoint{5.013066in}{3.016105in}}%
\pgfpathlineto{\pgfqpoint{5.013066in}{3.010082in}}%
\pgfpathlineto{\pgfqpoint{5.019127in}{3.010082in}}%
\pgfpathlineto{\pgfqpoint{5.019127in}{3.011725in}}%
\pgfpathlineto{\pgfqpoint{5.025188in}{3.011725in}}%
\pgfpathlineto{\pgfqpoint{5.025188in}{3.019938in}}%
\pgfpathlineto{\pgfqpoint{5.031249in}{3.019938in}}%
\pgfpathlineto{\pgfqpoint{5.031249in}{3.010082in}}%
\pgfpathlineto{\pgfqpoint{5.037310in}{3.010082in}}%
\pgfpathlineto{\pgfqpoint{5.037310in}{3.019391in}}%
\pgfpathlineto{\pgfqpoint{5.043371in}{3.019391in}}%
\pgfpathlineto{\pgfqpoint{5.043371in}{3.009535in}}%
\pgfpathlineto{\pgfqpoint{5.049432in}{3.009535in}}%
\pgfpathlineto{\pgfqpoint{5.049432in}{3.020486in}}%
\pgfpathlineto{\pgfqpoint{5.055493in}{3.020486in}}%
\pgfpathlineto{\pgfqpoint{5.055493in}{3.012273in}}%
\pgfpathlineto{\pgfqpoint{5.067615in}{3.011177in}}%
\pgfpathlineto{\pgfqpoint{5.067615in}{3.010082in}}%
\pgfpathlineto{\pgfqpoint{5.073676in}{3.010082in}}%
\pgfpathlineto{\pgfqpoint{5.073676in}{3.007345in}}%
\pgfpathlineto{\pgfqpoint{5.079737in}{3.007345in}}%
\pgfpathlineto{\pgfqpoint{5.079737in}{3.011725in}}%
\pgfpathlineto{\pgfqpoint{5.085798in}{3.011725in}}%
\pgfpathlineto{\pgfqpoint{5.085798in}{3.006797in}}%
\pgfpathlineto{\pgfqpoint{5.103981in}{3.006797in}}%
\pgfpathlineto{\pgfqpoint{5.103981in}{3.011725in}}%
\pgfpathlineto{\pgfqpoint{5.110042in}{3.011725in}}%
\pgfpathlineto{\pgfqpoint{5.110042in}{3.016105in}}%
\pgfpathlineto{\pgfqpoint{5.116103in}{3.016105in}}%
\pgfpathlineto{\pgfqpoint{5.116103in}{3.010082in}}%
\pgfpathlineto{\pgfqpoint{5.122164in}{3.010082in}}%
\pgfpathlineto{\pgfqpoint{5.122164in}{3.001322in}}%
\pgfpathlineto{\pgfqpoint{5.128225in}{3.001322in}}%
\pgfpathlineto{\pgfqpoint{5.128225in}{3.006797in}}%
\pgfpathlineto{\pgfqpoint{5.134286in}{3.006797in}}%
\pgfpathlineto{\pgfqpoint{5.134286in}{3.010082in}}%
\pgfpathlineto{\pgfqpoint{5.140347in}{3.010082in}}%
\pgfpathlineto{\pgfqpoint{5.140347in}{3.000774in}}%
\pgfpathlineto{\pgfqpoint{5.152469in}{3.001869in}}%
\pgfpathlineto{\pgfqpoint{5.152469in}{3.004060in}}%
\pgfpathlineto{\pgfqpoint{5.158530in}{3.004060in}}%
\pgfpathlineto{\pgfqpoint{5.158530in}{3.008440in}}%
\pgfpathlineto{\pgfqpoint{5.164591in}{3.008440in}}%
\pgfpathlineto{\pgfqpoint{5.164591in}{3.003512in}}%
\pgfpathlineto{\pgfqpoint{5.170652in}{3.003512in}}%
\pgfpathlineto{\pgfqpoint{5.170652in}{2.997489in}}%
\pgfpathlineto{\pgfqpoint{5.176713in}{2.997489in}}%
\pgfpathlineto{\pgfqpoint{5.176713in}{2.994751in}}%
\pgfpathlineto{\pgfqpoint{5.182774in}{2.994751in}}%
\pgfpathlineto{\pgfqpoint{5.182774in}{3.008987in}}%
\pgfpathlineto{\pgfqpoint{5.188835in}{3.008987in}}%
\pgfpathlineto{\pgfqpoint{5.188835in}{3.002417in}}%
\pgfpathlineto{\pgfqpoint{5.200957in}{3.001869in}}%
\pgfpathlineto{\pgfqpoint{5.200957in}{3.006250in}}%
\pgfpathlineto{\pgfqpoint{5.207018in}{3.006250in}}%
\pgfpathlineto{\pgfqpoint{5.207018in}{3.001322in}}%
\pgfpathlineto{\pgfqpoint{5.213079in}{3.001322in}}%
\pgfpathlineto{\pgfqpoint{5.213079in}{3.009535in}}%
\pgfpathlineto{\pgfqpoint{5.219140in}{3.009535in}}%
\pgfpathlineto{\pgfqpoint{5.219140in}{2.999132in}}%
\pgfpathlineto{\pgfqpoint{5.225201in}{2.999132in}}%
\pgfpathlineto{\pgfqpoint{5.225201in}{3.008440in}}%
\pgfpathlineto{\pgfqpoint{5.231262in}{3.008440in}}%
\pgfpathlineto{\pgfqpoint{5.231262in}{2.999132in}}%
\pgfpathlineto{\pgfqpoint{5.237323in}{2.999132in}}%
\pgfpathlineto{\pgfqpoint{5.237323in}{3.005702in}}%
\pgfpathlineto{\pgfqpoint{5.243384in}{3.005702in}}%
\pgfpathlineto{\pgfqpoint{5.243384in}{3.003512in}}%
\pgfpathlineto{\pgfqpoint{5.249445in}{3.003512in}}%
\pgfpathlineto{\pgfqpoint{5.249445in}{2.999679in}}%
\pgfpathlineto{\pgfqpoint{5.255505in}{2.999679in}}%
\pgfpathlineto{\pgfqpoint{5.255505in}{3.004607in}}%
\pgfpathlineto{\pgfqpoint{5.261566in}{3.004607in}}%
\pgfpathlineto{\pgfqpoint{5.261566in}{3.000227in}}%
\pgfpathlineto{\pgfqpoint{5.267627in}{3.000227in}}%
\pgfpathlineto{\pgfqpoint{5.267627in}{3.005702in}}%
\pgfpathlineto{\pgfqpoint{5.273688in}{3.005702in}}%
\pgfpathlineto{\pgfqpoint{5.273688in}{3.002417in}}%
\pgfpathlineto{\pgfqpoint{5.279749in}{3.002417in}}%
\pgfpathlineto{\pgfqpoint{5.279749in}{3.006797in}}%
\pgfpathlineto{\pgfqpoint{5.285810in}{3.006797in}}%
\pgfpathlineto{\pgfqpoint{5.285810in}{3.000774in}}%
\pgfpathlineto{\pgfqpoint{5.297932in}{3.001322in}}%
\pgfpathlineto{\pgfqpoint{5.297932in}{3.008440in}}%
\pgfpathlineto{\pgfqpoint{5.303993in}{3.008440in}}%
\pgfpathlineto{\pgfqpoint{5.303993in}{3.002417in}}%
\pgfpathlineto{\pgfqpoint{5.310054in}{3.002417in}}%
\pgfpathlineto{\pgfqpoint{5.310054in}{2.999679in}}%
\pgfpathlineto{\pgfqpoint{5.316115in}{2.999679in}}%
\pgfpathlineto{\pgfqpoint{5.316115in}{3.002964in}}%
\pgfpathlineto{\pgfqpoint{5.322176in}{3.002964in}}%
\pgfpathlineto{\pgfqpoint{5.322176in}{3.005702in}}%
\pgfpathlineto{\pgfqpoint{5.328237in}{3.005702in}}%
\pgfpathlineto{\pgfqpoint{5.328237in}{3.004060in}}%
\pgfpathlineto{\pgfqpoint{5.334298in}{3.004060in}}%
\pgfpathlineto{\pgfqpoint{5.334298in}{3.001322in}}%
\pgfpathlineto{\pgfqpoint{5.346420in}{3.001869in}}%
\pgfpathlineto{\pgfqpoint{5.346420in}{3.004060in}}%
\pgfpathlineto{\pgfqpoint{5.352481in}{3.004060in}}%
\pgfpathlineto{\pgfqpoint{5.352481in}{3.006250in}}%
\pgfpathlineto{\pgfqpoint{5.358542in}{3.006250in}}%
\pgfpathlineto{\pgfqpoint{5.358542in}{3.002964in}}%
\pgfpathlineto{\pgfqpoint{5.364603in}{3.002964in}}%
\pgfpathlineto{\pgfqpoint{5.364603in}{3.007892in}}%
\pgfpathlineto{\pgfqpoint{5.370664in}{3.007892in}}%
\pgfpathlineto{\pgfqpoint{5.370664in}{3.000227in}}%
\pgfpathlineto{\pgfqpoint{5.376725in}{3.000227in}}%
\pgfpathlineto{\pgfqpoint{5.376725in}{3.004607in}}%
\pgfpathlineto{\pgfqpoint{5.382786in}{3.004607in}}%
\pgfpathlineto{\pgfqpoint{5.382786in}{2.992014in}}%
\pgfpathlineto{\pgfqpoint{5.388847in}{2.992014in}}%
\pgfpathlineto{\pgfqpoint{5.388847in}{3.006250in}}%
\pgfpathlineto{\pgfqpoint{5.394908in}{3.006250in}}%
\pgfpathlineto{\pgfqpoint{5.394908in}{3.004060in}}%
\pgfpathlineto{\pgfqpoint{5.400969in}{3.004060in}}%
\pgfpathlineto{\pgfqpoint{5.400969in}{2.999132in}}%
\pgfpathlineto{\pgfqpoint{5.407030in}{2.999132in}}%
\pgfpathlineto{\pgfqpoint{5.407030in}{3.001869in}}%
\pgfpathlineto{\pgfqpoint{5.413091in}{3.001869in}}%
\pgfpathlineto{\pgfqpoint{5.413091in}{2.998037in}}%
\pgfpathlineto{\pgfqpoint{5.419152in}{2.998037in}}%
\pgfpathlineto{\pgfqpoint{5.419152in}{2.999679in}}%
\pgfpathlineto{\pgfqpoint{5.425213in}{2.999679in}}%
\pgfpathlineto{\pgfqpoint{5.425213in}{2.992561in}}%
\pgfpathlineto{\pgfqpoint{5.431274in}{2.992561in}}%
\pgfpathlineto{\pgfqpoint{5.431274in}{2.999132in}}%
\pgfpathlineto{\pgfqpoint{5.437335in}{2.999132in}}%
\pgfpathlineto{\pgfqpoint{5.437335in}{3.001869in}}%
\pgfpathlineto{\pgfqpoint{5.443396in}{3.001869in}}%
\pgfpathlineto{\pgfqpoint{5.443396in}{2.999679in}}%
\pgfpathlineto{\pgfqpoint{5.455518in}{3.000774in}}%
\pgfpathlineto{\pgfqpoint{5.455518in}{2.998037in}}%
\pgfpathlineto{\pgfqpoint{5.461579in}{2.998037in}}%
\pgfpathlineto{\pgfqpoint{5.461579in}{2.994751in}}%
\pgfpathlineto{\pgfqpoint{5.467640in}{2.994751in}}%
\pgfpathlineto{\pgfqpoint{5.467640in}{3.003512in}}%
\pgfpathlineto{\pgfqpoint{5.473701in}{3.003512in}}%
\pgfpathlineto{\pgfqpoint{5.473701in}{3.012820in}}%
\pgfpathlineto{\pgfqpoint{5.479762in}{3.012820in}}%
\pgfpathlineto{\pgfqpoint{5.479762in}{2.993656in}}%
\pgfpathlineto{\pgfqpoint{5.485823in}{2.993656in}}%
\pgfpathlineto{\pgfqpoint{5.485823in}{3.001322in}}%
\pgfpathlineto{\pgfqpoint{5.491884in}{3.001322in}}%
\pgfpathlineto{\pgfqpoint{5.491884in}{2.995846in}}%
\pgfpathlineto{\pgfqpoint{5.504006in}{2.994751in}}%
\pgfpathlineto{\pgfqpoint{5.504006in}{2.990919in}}%
\pgfpathlineto{\pgfqpoint{5.510067in}{2.990919in}}%
\pgfpathlineto{\pgfqpoint{5.510067in}{2.997489in}}%
\pgfpathlineto{\pgfqpoint{5.522189in}{2.996394in}}%
\pgfpathlineto{\pgfqpoint{5.522189in}{2.994204in}}%
\pgfpathlineto{\pgfqpoint{5.540372in}{2.994751in}}%
\pgfpathlineto{\pgfqpoint{5.540372in}{2.998584in}}%
\pgfpathlineto{\pgfqpoint{5.546432in}{2.998584in}}%
\pgfpathlineto{\pgfqpoint{5.546432in}{3.005155in}}%
\pgfpathlineto{\pgfqpoint{5.552493in}{3.005155in}}%
\pgfpathlineto{\pgfqpoint{5.552493in}{2.996394in}}%
\pgfpathlineto{\pgfqpoint{5.558554in}{2.996394in}}%
\pgfpathlineto{\pgfqpoint{5.558554in}{2.990919in}}%
\pgfpathlineto{\pgfqpoint{5.564615in}{2.990919in}}%
\pgfpathlineto{\pgfqpoint{5.564615in}{3.002417in}}%
\pgfpathlineto{\pgfqpoint{5.570676in}{3.002417in}}%
\pgfpathlineto{\pgfqpoint{5.570676in}{2.990919in}}%
\pgfpathlineto{\pgfqpoint{5.576737in}{2.990919in}}%
\pgfpathlineto{\pgfqpoint{5.576737in}{2.996394in}}%
\pgfpathlineto{\pgfqpoint{5.582798in}{2.996394in}}%
\pgfpathlineto{\pgfqpoint{5.582798in}{2.994204in}}%
\pgfpathlineto{\pgfqpoint{5.600981in}{2.993656in}}%
\pgfpathlineto{\pgfqpoint{5.600981in}{2.988728in}}%
\pgfpathlineto{\pgfqpoint{5.607042in}{2.988728in}}%
\pgfpathlineto{\pgfqpoint{5.607042in}{2.999679in}}%
\pgfpathlineto{\pgfqpoint{5.613103in}{2.999679in}}%
\pgfpathlineto{\pgfqpoint{5.613103in}{2.994751in}}%
\pgfpathlineto{\pgfqpoint{5.619164in}{2.994751in}}%
\pgfpathlineto{\pgfqpoint{5.619164in}{3.002417in}}%
\pgfpathlineto{\pgfqpoint{5.625225in}{3.002417in}}%
\pgfpathlineto{\pgfqpoint{5.625225in}{2.991466in}}%
\pgfpathlineto{\pgfqpoint{5.631286in}{2.991466in}}%
\pgfpathlineto{\pgfqpoint{5.631286in}{2.993109in}}%
\pgfpathlineto{\pgfqpoint{5.637347in}{2.993109in}}%
\pgfpathlineto{\pgfqpoint{5.637347in}{3.008440in}}%
\pgfpathlineto{\pgfqpoint{5.643408in}{3.008440in}}%
\pgfpathlineto{\pgfqpoint{5.643408in}{2.994204in}}%
\pgfpathlineto{\pgfqpoint{5.649469in}{2.994204in}}%
\pgfpathlineto{\pgfqpoint{5.649469in}{3.001869in}}%
\pgfpathlineto{\pgfqpoint{5.655530in}{3.001869in}}%
\pgfpathlineto{\pgfqpoint{5.655530in}{2.998037in}}%
\pgfpathlineto{\pgfqpoint{5.661591in}{2.998037in}}%
\pgfpathlineto{\pgfqpoint{5.661591in}{2.994204in}}%
\pgfpathlineto{\pgfqpoint{5.667652in}{2.994204in}}%
\pgfpathlineto{\pgfqpoint{5.667652in}{3.001322in}}%
\pgfpathlineto{\pgfqpoint{5.679774in}{3.000227in}}%
\pgfpathlineto{\pgfqpoint{5.679774in}{2.990919in}}%
\pgfpathlineto{\pgfqpoint{5.685835in}{2.990919in}}%
\pgfpathlineto{\pgfqpoint{5.685835in}{3.000227in}}%
\pgfpathlineto{\pgfqpoint{5.697957in}{2.999132in}}%
\pgfpathlineto{\pgfqpoint{5.697957in}{2.995846in}}%
\pgfpathlineto{\pgfqpoint{5.704018in}{2.995846in}}%
\pgfpathlineto{\pgfqpoint{5.704018in}{2.993656in}}%
\pgfpathlineto{\pgfqpoint{5.716140in}{2.992561in}}%
\pgfpathlineto{\pgfqpoint{5.716140in}{2.998037in}}%
\pgfpathlineto{\pgfqpoint{5.734323in}{2.998037in}}%
\pgfpathlineto{\pgfqpoint{5.734323in}{2.992561in}}%
\pgfpathlineto{\pgfqpoint{5.740384in}{2.992561in}}%
\pgfpathlineto{\pgfqpoint{5.740384in}{2.994204in}}%
\pgfpathlineto{\pgfqpoint{5.746445in}{2.994204in}}%
\pgfpathlineto{\pgfqpoint{5.746445in}{2.996942in}}%
\pgfpathlineto{\pgfqpoint{5.752506in}{2.996942in}}%
\pgfpathlineto{\pgfqpoint{5.752506in}{2.995299in}}%
\pgfpathlineto{\pgfqpoint{5.758567in}{2.995299in}}%
\pgfpathlineto{\pgfqpoint{5.758567in}{2.993109in}}%
\pgfpathlineto{\pgfqpoint{5.764628in}{2.993109in}}%
\pgfpathlineto{\pgfqpoint{5.764628in}{2.995299in}}%
\pgfpathlineto{\pgfqpoint{5.770689in}{2.995299in}}%
\pgfpathlineto{\pgfqpoint{5.770689in}{2.991466in}}%
\pgfpathlineto{\pgfqpoint{5.776750in}{2.991466in}}%
\pgfpathlineto{\pgfqpoint{5.776750in}{2.988728in}}%
\pgfpathlineto{\pgfqpoint{5.782811in}{2.988728in}}%
\pgfpathlineto{\pgfqpoint{5.782811in}{2.995299in}}%
\pgfpathlineto{\pgfqpoint{5.794933in}{2.995299in}}%
\pgfpathlineto{\pgfqpoint{5.794933in}{3.000774in}}%
\pgfpathlineto{\pgfqpoint{5.800994in}{3.000774in}}%
\pgfpathlineto{\pgfqpoint{5.800994in}{2.990919in}}%
\pgfpathlineto{\pgfqpoint{5.807055in}{2.990919in}}%
\pgfpathlineto{\pgfqpoint{5.807055in}{2.990919in}}%
\pgfusepath{stroke}%
\end{pgfscope}%
\begin{pgfscope}%
\pgfpathrectangle{\pgfqpoint{0.781944in}{2.977778in}}{\pgfqpoint{5.019444in}{1.650000in}}%
\pgfusepath{clip}%
\pgfsetrectcap%
\pgfsetroundjoin%
\pgfsetlinewidth{1.505625pt}%
\definecolor{currentstroke}{rgb}{1.000000,0.498039,0.054902}%
\pgfsetstrokecolor{currentstroke}%
\pgfsetdash{}{0pt}%
\pgfpathmoveto{\pgfqpoint{0.776442in}{3.046767in}}%
\pgfpathlineto{\pgfqpoint{0.776442in}{3.041840in}}%
\pgfpathlineto{\pgfqpoint{0.782503in}{3.041840in}}%
\pgfpathlineto{\pgfqpoint{0.782503in}{3.024866in}}%
\pgfpathlineto{\pgfqpoint{0.788564in}{3.024866in}}%
\pgfpathlineto{\pgfqpoint{0.788564in}{3.042387in}}%
\pgfpathlineto{\pgfqpoint{0.794625in}{3.042387in}}%
\pgfpathlineto{\pgfqpoint{0.794625in}{3.021033in}}%
\pgfpathlineto{\pgfqpoint{0.800686in}{3.021033in}}%
\pgfpathlineto{\pgfqpoint{0.800686in}{3.038554in}}%
\pgfpathlineto{\pgfqpoint{0.806747in}{3.038554in}}%
\pgfpathlineto{\pgfqpoint{0.806747in}{3.029794in}}%
\pgfpathlineto{\pgfqpoint{0.812808in}{3.029794in}}%
\pgfpathlineto{\pgfqpoint{0.812808in}{3.050053in}}%
\pgfpathlineto{\pgfqpoint{0.818869in}{3.050053in}}%
\pgfpathlineto{\pgfqpoint{0.818869in}{3.032531in}}%
\pgfpathlineto{\pgfqpoint{0.824930in}{3.032531in}}%
\pgfpathlineto{\pgfqpoint{0.824930in}{3.042387in}}%
\pgfpathlineto{\pgfqpoint{0.830991in}{3.042387in}}%
\pgfpathlineto{\pgfqpoint{0.830991in}{3.038007in}}%
\pgfpathlineto{\pgfqpoint{0.837052in}{3.038007in}}%
\pgfpathlineto{\pgfqpoint{0.837052in}{3.027604in}}%
\pgfpathlineto{\pgfqpoint{0.843113in}{3.027604in}}%
\pgfpathlineto{\pgfqpoint{0.843113in}{3.039649in}}%
\pgfpathlineto{\pgfqpoint{0.849174in}{3.039649in}}%
\pgfpathlineto{\pgfqpoint{0.849174in}{3.029794in}}%
\pgfpathlineto{\pgfqpoint{0.855235in}{3.029794in}}%
\pgfpathlineto{\pgfqpoint{0.855235in}{3.038554in}}%
\pgfpathlineto{\pgfqpoint{0.867357in}{3.038007in}}%
\pgfpathlineto{\pgfqpoint{0.867357in}{3.053338in}}%
\pgfpathlineto{\pgfqpoint{0.873418in}{3.053338in}}%
\pgfpathlineto{\pgfqpoint{0.873418in}{3.028151in}}%
\pgfpathlineto{\pgfqpoint{0.879479in}{3.028151in}}%
\pgfpathlineto{\pgfqpoint{0.879479in}{3.047862in}}%
\pgfpathlineto{\pgfqpoint{0.885540in}{3.047862in}}%
\pgfpathlineto{\pgfqpoint{0.885540in}{3.043482in}}%
\pgfpathlineto{\pgfqpoint{0.897661in}{3.043482in}}%
\pgfpathlineto{\pgfqpoint{0.897661in}{3.033079in}}%
\pgfpathlineto{\pgfqpoint{0.903722in}{3.033079in}}%
\pgfpathlineto{\pgfqpoint{0.903722in}{3.031436in}}%
\pgfpathlineto{\pgfqpoint{0.909783in}{3.031436in}}%
\pgfpathlineto{\pgfqpoint{0.909783in}{3.044030in}}%
\pgfpathlineto{\pgfqpoint{0.915844in}{3.044030in}}%
\pgfpathlineto{\pgfqpoint{0.915844in}{3.036912in}}%
\pgfpathlineto{\pgfqpoint{0.921905in}{3.036912in}}%
\pgfpathlineto{\pgfqpoint{0.921905in}{3.051148in}}%
\pgfpathlineto{\pgfqpoint{0.927966in}{3.051148in}}%
\pgfpathlineto{\pgfqpoint{0.927966in}{3.036912in}}%
\pgfpathlineto{\pgfqpoint{0.934027in}{3.036912in}}%
\pgfpathlineto{\pgfqpoint{0.934027in}{3.042387in}}%
\pgfpathlineto{\pgfqpoint{0.940088in}{3.042387in}}%
\pgfpathlineto{\pgfqpoint{0.940088in}{3.049505in}}%
\pgfpathlineto{\pgfqpoint{0.946149in}{3.049505in}}%
\pgfpathlineto{\pgfqpoint{0.946149in}{3.023223in}}%
\pgfpathlineto{\pgfqpoint{0.952210in}{3.023223in}}%
\pgfpathlineto{\pgfqpoint{0.952210in}{3.035269in}}%
\pgfpathlineto{\pgfqpoint{0.958271in}{3.035269in}}%
\pgfpathlineto{\pgfqpoint{0.958271in}{3.029246in}}%
\pgfpathlineto{\pgfqpoint{0.964332in}{3.029246in}}%
\pgfpathlineto{\pgfqpoint{0.964332in}{3.039649in}}%
\pgfpathlineto{\pgfqpoint{0.970393in}{3.039649in}}%
\pgfpathlineto{\pgfqpoint{0.970393in}{3.036364in}}%
\pgfpathlineto{\pgfqpoint{0.976454in}{3.036364in}}%
\pgfpathlineto{\pgfqpoint{0.976454in}{3.028151in}}%
\pgfpathlineto{\pgfqpoint{0.982515in}{3.028151in}}%
\pgfpathlineto{\pgfqpoint{0.982515in}{3.034722in}}%
\pgfpathlineto{\pgfqpoint{0.988576in}{3.034722in}}%
\pgfpathlineto{\pgfqpoint{0.988576in}{3.037459in}}%
\pgfpathlineto{\pgfqpoint{0.994637in}{3.037459in}}%
\pgfpathlineto{\pgfqpoint{0.994637in}{3.040197in}}%
\pgfpathlineto{\pgfqpoint{1.000698in}{3.040197in}}%
\pgfpathlineto{\pgfqpoint{1.000698in}{3.035269in}}%
\pgfpathlineto{\pgfqpoint{1.006759in}{3.035269in}}%
\pgfpathlineto{\pgfqpoint{1.006759in}{3.039649in}}%
\pgfpathlineto{\pgfqpoint{1.018881in}{3.040744in}}%
\pgfpathlineto{\pgfqpoint{1.018881in}{3.031436in}}%
\pgfpathlineto{\pgfqpoint{1.024942in}{3.031436in}}%
\pgfpathlineto{\pgfqpoint{1.024942in}{3.039102in}}%
\pgfpathlineto{\pgfqpoint{1.031003in}{3.039102in}}%
\pgfpathlineto{\pgfqpoint{1.031003in}{3.057718in}}%
\pgfpathlineto{\pgfqpoint{1.037064in}{3.057718in}}%
\pgfpathlineto{\pgfqpoint{1.037064in}{3.038554in}}%
\pgfpathlineto{\pgfqpoint{1.043125in}{3.038554in}}%
\pgfpathlineto{\pgfqpoint{1.043125in}{3.049505in}}%
\pgfpathlineto{\pgfqpoint{1.049186in}{3.049505in}}%
\pgfpathlineto{\pgfqpoint{1.049186in}{3.033626in}}%
\pgfpathlineto{\pgfqpoint{1.055247in}{3.033626in}}%
\pgfpathlineto{\pgfqpoint{1.055247in}{3.047315in}}%
\pgfpathlineto{\pgfqpoint{1.061308in}{3.047315in}}%
\pgfpathlineto{\pgfqpoint{1.061308in}{3.043482in}}%
\pgfpathlineto{\pgfqpoint{1.067369in}{3.043482in}}%
\pgfpathlineto{\pgfqpoint{1.067369in}{3.032531in}}%
\pgfpathlineto{\pgfqpoint{1.073430in}{3.032531in}}%
\pgfpathlineto{\pgfqpoint{1.073430in}{3.040744in}}%
\pgfpathlineto{\pgfqpoint{1.079491in}{3.040744in}}%
\pgfpathlineto{\pgfqpoint{1.079491in}{3.033626in}}%
\pgfpathlineto{\pgfqpoint{1.085552in}{3.033626in}}%
\pgfpathlineto{\pgfqpoint{1.085552in}{3.040744in}}%
\pgfpathlineto{\pgfqpoint{1.097674in}{3.040744in}}%
\pgfpathlineto{\pgfqpoint{1.097674in}{3.057171in}}%
\pgfpathlineto{\pgfqpoint{1.103735in}{3.057171in}}%
\pgfpathlineto{\pgfqpoint{1.103735in}{3.033079in}}%
\pgfpathlineto{\pgfqpoint{1.109796in}{3.033079in}}%
\pgfpathlineto{\pgfqpoint{1.109796in}{3.046220in}}%
\pgfpathlineto{\pgfqpoint{1.121918in}{3.045125in}}%
\pgfpathlineto{\pgfqpoint{1.121918in}{3.042387in}}%
\pgfpathlineto{\pgfqpoint{1.127979in}{3.042387in}}%
\pgfpathlineto{\pgfqpoint{1.127979in}{3.051148in}}%
\pgfpathlineto{\pgfqpoint{1.140101in}{3.050053in}}%
\pgfpathlineto{\pgfqpoint{1.140101in}{3.049505in}}%
\pgfpathlineto{\pgfqpoint{1.146162in}{3.049505in}}%
\pgfpathlineto{\pgfqpoint{1.146162in}{3.039649in}}%
\pgfpathlineto{\pgfqpoint{1.152223in}{3.039649in}}%
\pgfpathlineto{\pgfqpoint{1.152223in}{3.046220in}}%
\pgfpathlineto{\pgfqpoint{1.158284in}{3.046220in}}%
\pgfpathlineto{\pgfqpoint{1.158284in}{3.040197in}}%
\pgfpathlineto{\pgfqpoint{1.164345in}{3.040197in}}%
\pgfpathlineto{\pgfqpoint{1.164345in}{3.054980in}}%
\pgfpathlineto{\pgfqpoint{1.170406in}{3.054980in}}%
\pgfpathlineto{\pgfqpoint{1.170406in}{3.050053in}}%
\pgfpathlineto{\pgfqpoint{1.176467in}{3.050053in}}%
\pgfpathlineto{\pgfqpoint{1.176467in}{3.041292in}}%
\pgfpathlineto{\pgfqpoint{1.182527in}{3.041292in}}%
\pgfpathlineto{\pgfqpoint{1.182527in}{3.046220in}}%
\pgfpathlineto{\pgfqpoint{1.188588in}{3.046220in}}%
\pgfpathlineto{\pgfqpoint{1.188588in}{3.033626in}}%
\pgfpathlineto{\pgfqpoint{1.194649in}{3.033626in}}%
\pgfpathlineto{\pgfqpoint{1.194649in}{3.051148in}}%
\pgfpathlineto{\pgfqpoint{1.200710in}{3.051148in}}%
\pgfpathlineto{\pgfqpoint{1.200710in}{3.036364in}}%
\pgfpathlineto{\pgfqpoint{1.206771in}{3.036364in}}%
\pgfpathlineto{\pgfqpoint{1.206771in}{3.063193in}}%
\pgfpathlineto{\pgfqpoint{1.212832in}{3.063193in}}%
\pgfpathlineto{\pgfqpoint{1.212832in}{3.034174in}}%
\pgfpathlineto{\pgfqpoint{1.218893in}{3.034174in}}%
\pgfpathlineto{\pgfqpoint{1.218893in}{3.049505in}}%
\pgfpathlineto{\pgfqpoint{1.224954in}{3.049505in}}%
\pgfpathlineto{\pgfqpoint{1.224954in}{3.052790in}}%
\pgfpathlineto{\pgfqpoint{1.231015in}{3.052790in}}%
\pgfpathlineto{\pgfqpoint{1.231015in}{3.042387in}}%
\pgfpathlineto{\pgfqpoint{1.237076in}{3.042387in}}%
\pgfpathlineto{\pgfqpoint{1.237076in}{3.049505in}}%
\pgfpathlineto{\pgfqpoint{1.243137in}{3.049505in}}%
\pgfpathlineto{\pgfqpoint{1.243137in}{3.035817in}}%
\pgfpathlineto{\pgfqpoint{1.249198in}{3.035817in}}%
\pgfpathlineto{\pgfqpoint{1.249198in}{3.054980in}}%
\pgfpathlineto{\pgfqpoint{1.255259in}{3.054980in}}%
\pgfpathlineto{\pgfqpoint{1.255259in}{3.042935in}}%
\pgfpathlineto{\pgfqpoint{1.261320in}{3.042935in}}%
\pgfpathlineto{\pgfqpoint{1.261320in}{3.059361in}}%
\pgfpathlineto{\pgfqpoint{1.267381in}{3.059361in}}%
\pgfpathlineto{\pgfqpoint{1.267381in}{3.041292in}}%
\pgfpathlineto{\pgfqpoint{1.273442in}{3.041292in}}%
\pgfpathlineto{\pgfqpoint{1.273442in}{3.066479in}}%
\pgfpathlineto{\pgfqpoint{1.279503in}{3.066479in}}%
\pgfpathlineto{\pgfqpoint{1.279503in}{3.054980in}}%
\pgfpathlineto{\pgfqpoint{1.285564in}{3.054980in}}%
\pgfpathlineto{\pgfqpoint{1.285564in}{3.041840in}}%
\pgfpathlineto{\pgfqpoint{1.291625in}{3.041840in}}%
\pgfpathlineto{\pgfqpoint{1.291625in}{3.054980in}}%
\pgfpathlineto{\pgfqpoint{1.297686in}{3.054980in}}%
\pgfpathlineto{\pgfqpoint{1.297686in}{3.033626in}}%
\pgfpathlineto{\pgfqpoint{1.303747in}{3.033626in}}%
\pgfpathlineto{\pgfqpoint{1.303747in}{3.054980in}}%
\pgfpathlineto{\pgfqpoint{1.309808in}{3.054980in}}%
\pgfpathlineto{\pgfqpoint{1.309808in}{3.050053in}}%
\pgfpathlineto{\pgfqpoint{1.315869in}{3.050053in}}%
\pgfpathlineto{\pgfqpoint{1.315869in}{3.058813in}}%
\pgfpathlineto{\pgfqpoint{1.321930in}{3.058813in}}%
\pgfpathlineto{\pgfqpoint{1.321930in}{3.041292in}}%
\pgfpathlineto{\pgfqpoint{1.327991in}{3.041292in}}%
\pgfpathlineto{\pgfqpoint{1.327991in}{3.046220in}}%
\pgfpathlineto{\pgfqpoint{1.334052in}{3.046220in}}%
\pgfpathlineto{\pgfqpoint{1.334052in}{3.036912in}}%
\pgfpathlineto{\pgfqpoint{1.340113in}{3.036912in}}%
\pgfpathlineto{\pgfqpoint{1.340113in}{3.045125in}}%
\pgfpathlineto{\pgfqpoint{1.346174in}{3.045125in}}%
\pgfpathlineto{\pgfqpoint{1.346174in}{3.042935in}}%
\pgfpathlineto{\pgfqpoint{1.352235in}{3.042935in}}%
\pgfpathlineto{\pgfqpoint{1.352235in}{3.041292in}}%
\pgfpathlineto{\pgfqpoint{1.358296in}{3.041292in}}%
\pgfpathlineto{\pgfqpoint{1.358296in}{3.058813in}}%
\pgfpathlineto{\pgfqpoint{1.364357in}{3.058813in}}%
\pgfpathlineto{\pgfqpoint{1.364357in}{3.044577in}}%
\pgfpathlineto{\pgfqpoint{1.370418in}{3.044577in}}%
\pgfpathlineto{\pgfqpoint{1.370418in}{3.059908in}}%
\pgfpathlineto{\pgfqpoint{1.376479in}{3.059908in}}%
\pgfpathlineto{\pgfqpoint{1.376479in}{3.029794in}}%
\pgfpathlineto{\pgfqpoint{1.382540in}{3.029794in}}%
\pgfpathlineto{\pgfqpoint{1.382540in}{3.064836in}}%
\pgfpathlineto{\pgfqpoint{1.388601in}{3.064836in}}%
\pgfpathlineto{\pgfqpoint{1.388601in}{3.052790in}}%
\pgfpathlineto{\pgfqpoint{1.394662in}{3.052790in}}%
\pgfpathlineto{\pgfqpoint{1.394662in}{3.056623in}}%
\pgfpathlineto{\pgfqpoint{1.400723in}{3.056623in}}%
\pgfpathlineto{\pgfqpoint{1.400723in}{3.065384in}}%
\pgfpathlineto{\pgfqpoint{1.406784in}{3.065384in}}%
\pgfpathlineto{\pgfqpoint{1.406784in}{3.054433in}}%
\pgfpathlineto{\pgfqpoint{1.418906in}{3.054433in}}%
\pgfpathlineto{\pgfqpoint{1.418906in}{3.059361in}}%
\pgfpathlineto{\pgfqpoint{1.424967in}{3.059361in}}%
\pgfpathlineto{\pgfqpoint{1.424967in}{3.061551in}}%
\pgfpathlineto{\pgfqpoint{1.431028in}{3.061551in}}%
\pgfpathlineto{\pgfqpoint{1.431028in}{3.046220in}}%
\pgfpathlineto{\pgfqpoint{1.437089in}{3.046220in}}%
\pgfpathlineto{\pgfqpoint{1.437089in}{3.060456in}}%
\pgfpathlineto{\pgfqpoint{1.443150in}{3.060456in}}%
\pgfpathlineto{\pgfqpoint{1.443150in}{3.055528in}}%
\pgfpathlineto{\pgfqpoint{1.449211in}{3.055528in}}%
\pgfpathlineto{\pgfqpoint{1.449211in}{3.061003in}}%
\pgfpathlineto{\pgfqpoint{1.467394in}{3.059908in}}%
\pgfpathlineto{\pgfqpoint{1.467394in}{3.067574in}}%
\pgfpathlineto{\pgfqpoint{1.473454in}{3.067574in}}%
\pgfpathlineto{\pgfqpoint{1.473454in}{3.053338in}}%
\pgfpathlineto{\pgfqpoint{1.479515in}{3.053338in}}%
\pgfpathlineto{\pgfqpoint{1.479515in}{3.056075in}}%
\pgfpathlineto{\pgfqpoint{1.485576in}{3.056075in}}%
\pgfpathlineto{\pgfqpoint{1.485576in}{3.052243in}}%
\pgfpathlineto{\pgfqpoint{1.491637in}{3.052243in}}%
\pgfpathlineto{\pgfqpoint{1.491637in}{3.068669in}}%
\pgfpathlineto{\pgfqpoint{1.497698in}{3.068669in}}%
\pgfpathlineto{\pgfqpoint{1.497698in}{3.046767in}}%
\pgfpathlineto{\pgfqpoint{1.503759in}{3.046767in}}%
\pgfpathlineto{\pgfqpoint{1.503759in}{3.053885in}}%
\pgfpathlineto{\pgfqpoint{1.509820in}{3.053885in}}%
\pgfpathlineto{\pgfqpoint{1.509820in}{3.081262in}}%
\pgfpathlineto{\pgfqpoint{1.515881in}{3.081262in}}%
\pgfpathlineto{\pgfqpoint{1.515881in}{3.064836in}}%
\pgfpathlineto{\pgfqpoint{1.528003in}{3.065384in}}%
\pgfpathlineto{\pgfqpoint{1.528003in}{3.054980in}}%
\pgfpathlineto{\pgfqpoint{1.534064in}{3.054980in}}%
\pgfpathlineto{\pgfqpoint{1.534064in}{3.063741in}}%
\pgfpathlineto{\pgfqpoint{1.540125in}{3.063741in}}%
\pgfpathlineto{\pgfqpoint{1.540125in}{3.052243in}}%
\pgfpathlineto{\pgfqpoint{1.546186in}{3.052243in}}%
\pgfpathlineto{\pgfqpoint{1.546186in}{3.067026in}}%
\pgfpathlineto{\pgfqpoint{1.564369in}{3.068121in}}%
\pgfpathlineto{\pgfqpoint{1.564369in}{3.071954in}}%
\pgfpathlineto{\pgfqpoint{1.570430in}{3.071954in}}%
\pgfpathlineto{\pgfqpoint{1.570430in}{3.059361in}}%
\pgfpathlineto{\pgfqpoint{1.576491in}{3.059361in}}%
\pgfpathlineto{\pgfqpoint{1.576491in}{3.088380in}}%
\pgfpathlineto{\pgfqpoint{1.582552in}{3.088380in}}%
\pgfpathlineto{\pgfqpoint{1.582552in}{3.059908in}}%
\pgfpathlineto{\pgfqpoint{1.588613in}{3.059908in}}%
\pgfpathlineto{\pgfqpoint{1.588613in}{3.081262in}}%
\pgfpathlineto{\pgfqpoint{1.594674in}{3.081262in}}%
\pgfpathlineto{\pgfqpoint{1.594674in}{3.065384in}}%
\pgfpathlineto{\pgfqpoint{1.600735in}{3.065384in}}%
\pgfpathlineto{\pgfqpoint{1.600735in}{3.071406in}}%
\pgfpathlineto{\pgfqpoint{1.606796in}{3.071406in}}%
\pgfpathlineto{\pgfqpoint{1.606796in}{3.059908in}}%
\pgfpathlineto{\pgfqpoint{1.612857in}{3.059908in}}%
\pgfpathlineto{\pgfqpoint{1.612857in}{3.089475in}}%
\pgfpathlineto{\pgfqpoint{1.618918in}{3.089475in}}%
\pgfpathlineto{\pgfqpoint{1.618918in}{3.071406in}}%
\pgfpathlineto{\pgfqpoint{1.631040in}{3.070311in}}%
\pgfpathlineto{\pgfqpoint{1.631040in}{3.081262in}}%
\pgfpathlineto{\pgfqpoint{1.637101in}{3.081262in}}%
\pgfpathlineto{\pgfqpoint{1.637101in}{3.067026in}}%
\pgfpathlineto{\pgfqpoint{1.643162in}{3.067026in}}%
\pgfpathlineto{\pgfqpoint{1.643162in}{3.070311in}}%
\pgfpathlineto{\pgfqpoint{1.649223in}{3.070311in}}%
\pgfpathlineto{\pgfqpoint{1.649223in}{3.071954in}}%
\pgfpathlineto{\pgfqpoint{1.655284in}{3.071954in}}%
\pgfpathlineto{\pgfqpoint{1.655284in}{3.068121in}}%
\pgfpathlineto{\pgfqpoint{1.661345in}{3.068121in}}%
\pgfpathlineto{\pgfqpoint{1.661345in}{3.061551in}}%
\pgfpathlineto{\pgfqpoint{1.667406in}{3.061551in}}%
\pgfpathlineto{\pgfqpoint{1.667406in}{3.069764in}}%
\pgfpathlineto{\pgfqpoint{1.673467in}{3.069764in}}%
\pgfpathlineto{\pgfqpoint{1.673467in}{3.078524in}}%
\pgfpathlineto{\pgfqpoint{1.679528in}{3.078524in}}%
\pgfpathlineto{\pgfqpoint{1.679528in}{3.076882in}}%
\pgfpathlineto{\pgfqpoint{1.685589in}{3.076882in}}%
\pgfpathlineto{\pgfqpoint{1.685589in}{3.094951in}}%
\pgfpathlineto{\pgfqpoint{1.691650in}{3.094951in}}%
\pgfpathlineto{\pgfqpoint{1.691650in}{3.063193in}}%
\pgfpathlineto{\pgfqpoint{1.697711in}{3.063193in}}%
\pgfpathlineto{\pgfqpoint{1.697711in}{3.086737in}}%
\pgfpathlineto{\pgfqpoint{1.703772in}{3.086737in}}%
\pgfpathlineto{\pgfqpoint{1.703772in}{3.077429in}}%
\pgfpathlineto{\pgfqpoint{1.709833in}{3.077429in}}%
\pgfpathlineto{\pgfqpoint{1.709833in}{3.089475in}}%
\pgfpathlineto{\pgfqpoint{1.715894in}{3.089475in}}%
\pgfpathlineto{\pgfqpoint{1.715894in}{3.079619in}}%
\pgfpathlineto{\pgfqpoint{1.721955in}{3.079619in}}%
\pgfpathlineto{\pgfqpoint{1.721955in}{3.095498in}}%
\pgfpathlineto{\pgfqpoint{1.728016in}{3.095498in}}%
\pgfpathlineto{\pgfqpoint{1.728016in}{3.066479in}}%
\pgfpathlineto{\pgfqpoint{1.734077in}{3.066479in}}%
\pgfpathlineto{\pgfqpoint{1.734077in}{3.084000in}}%
\pgfpathlineto{\pgfqpoint{1.740138in}{3.084000in}}%
\pgfpathlineto{\pgfqpoint{1.740138in}{3.082357in}}%
\pgfpathlineto{\pgfqpoint{1.746199in}{3.082357in}}%
\pgfpathlineto{\pgfqpoint{1.746199in}{3.069764in}}%
\pgfpathlineto{\pgfqpoint{1.752260in}{3.069764in}}%
\pgfpathlineto{\pgfqpoint{1.752260in}{3.095498in}}%
\pgfpathlineto{\pgfqpoint{1.758321in}{3.095498in}}%
\pgfpathlineto{\pgfqpoint{1.758321in}{3.061003in}}%
\pgfpathlineto{\pgfqpoint{1.764381in}{3.061003in}}%
\pgfpathlineto{\pgfqpoint{1.764381in}{3.095498in}}%
\pgfpathlineto{\pgfqpoint{1.770442in}{3.095498in}}%
\pgfpathlineto{\pgfqpoint{1.770442in}{3.067574in}}%
\pgfpathlineto{\pgfqpoint{1.776503in}{3.067574in}}%
\pgfpathlineto{\pgfqpoint{1.776503in}{3.099331in}}%
\pgfpathlineto{\pgfqpoint{1.782564in}{3.099331in}}%
\pgfpathlineto{\pgfqpoint{1.782564in}{3.092213in}}%
\pgfpathlineto{\pgfqpoint{1.788625in}{3.092213in}}%
\pgfpathlineto{\pgfqpoint{1.788625in}{3.093855in}}%
\pgfpathlineto{\pgfqpoint{1.794686in}{3.093855in}}%
\pgfpathlineto{\pgfqpoint{1.794686in}{3.108639in}}%
\pgfpathlineto{\pgfqpoint{1.800747in}{3.108639in}}%
\pgfpathlineto{\pgfqpoint{1.800747in}{3.084000in}}%
\pgfpathlineto{\pgfqpoint{1.806808in}{3.084000in}}%
\pgfpathlineto{\pgfqpoint{1.806808in}{3.098783in}}%
\pgfpathlineto{\pgfqpoint{1.812869in}{3.098783in}}%
\pgfpathlineto{\pgfqpoint{1.812869in}{3.088380in}}%
\pgfpathlineto{\pgfqpoint{1.818930in}{3.088380in}}%
\pgfpathlineto{\pgfqpoint{1.818930in}{3.093308in}}%
\pgfpathlineto{\pgfqpoint{1.824991in}{3.093308in}}%
\pgfpathlineto{\pgfqpoint{1.824991in}{3.076882in}}%
\pgfpathlineto{\pgfqpoint{1.831052in}{3.076882in}}%
\pgfpathlineto{\pgfqpoint{1.831052in}{3.085095in}}%
\pgfpathlineto{\pgfqpoint{1.837113in}{3.085095in}}%
\pgfpathlineto{\pgfqpoint{1.837113in}{3.096046in}}%
\pgfpathlineto{\pgfqpoint{1.843174in}{3.096046in}}%
\pgfpathlineto{\pgfqpoint{1.843174in}{3.107544in}}%
\pgfpathlineto{\pgfqpoint{1.849235in}{3.107544in}}%
\pgfpathlineto{\pgfqpoint{1.849235in}{3.085095in}}%
\pgfpathlineto{\pgfqpoint{1.855296in}{3.085095in}}%
\pgfpathlineto{\pgfqpoint{1.855296in}{3.081810in}}%
\pgfpathlineto{\pgfqpoint{1.861357in}{3.081810in}}%
\pgfpathlineto{\pgfqpoint{1.861357in}{3.105901in}}%
\pgfpathlineto{\pgfqpoint{1.867418in}{3.105901in}}%
\pgfpathlineto{\pgfqpoint{1.867418in}{3.091665in}}%
\pgfpathlineto{\pgfqpoint{1.873479in}{3.091665in}}%
\pgfpathlineto{\pgfqpoint{1.873479in}{3.109734in}}%
\pgfpathlineto{\pgfqpoint{1.879540in}{3.109734in}}%
\pgfpathlineto{\pgfqpoint{1.879540in}{3.111924in}}%
\pgfpathlineto{\pgfqpoint{1.885601in}{3.111924in}}%
\pgfpathlineto{\pgfqpoint{1.885601in}{3.102616in}}%
\pgfpathlineto{\pgfqpoint{1.891662in}{3.102616in}}%
\pgfpathlineto{\pgfqpoint{1.891662in}{3.104259in}}%
\pgfpathlineto{\pgfqpoint{1.897723in}{3.104259in}}%
\pgfpathlineto{\pgfqpoint{1.897723in}{3.106449in}}%
\pgfpathlineto{\pgfqpoint{1.903784in}{3.106449in}}%
\pgfpathlineto{\pgfqpoint{1.903784in}{3.110282in}}%
\pgfpathlineto{\pgfqpoint{1.909845in}{3.110282in}}%
\pgfpathlineto{\pgfqpoint{1.909845in}{3.075787in}}%
\pgfpathlineto{\pgfqpoint{1.915906in}{3.075787in}}%
\pgfpathlineto{\pgfqpoint{1.915906in}{3.111377in}}%
\pgfpathlineto{\pgfqpoint{1.921967in}{3.111377in}}%
\pgfpathlineto{\pgfqpoint{1.921967in}{3.103164in}}%
\pgfpathlineto{\pgfqpoint{1.928028in}{3.103164in}}%
\pgfpathlineto{\pgfqpoint{1.928028in}{3.127255in}}%
\pgfpathlineto{\pgfqpoint{1.934089in}{3.127255in}}%
\pgfpathlineto{\pgfqpoint{1.934089in}{3.090023in}}%
\pgfpathlineto{\pgfqpoint{1.940150in}{3.090023in}}%
\pgfpathlineto{\pgfqpoint{1.940150in}{3.109186in}}%
\pgfpathlineto{\pgfqpoint{1.946211in}{3.109186in}}%
\pgfpathlineto{\pgfqpoint{1.946211in}{3.092213in}}%
\pgfpathlineto{\pgfqpoint{1.952272in}{3.092213in}}%
\pgfpathlineto{\pgfqpoint{1.952272in}{3.127803in}}%
\pgfpathlineto{\pgfqpoint{1.958333in}{3.127803in}}%
\pgfpathlineto{\pgfqpoint{1.958333in}{3.117399in}}%
\pgfpathlineto{\pgfqpoint{1.964394in}{3.117399in}}%
\pgfpathlineto{\pgfqpoint{1.964394in}{3.090023in}}%
\pgfpathlineto{\pgfqpoint{1.970455in}{3.090023in}}%
\pgfpathlineto{\pgfqpoint{1.970455in}{3.134373in}}%
\pgfpathlineto{\pgfqpoint{1.976516in}{3.134373in}}%
\pgfpathlineto{\pgfqpoint{1.976516in}{3.096046in}}%
\pgfpathlineto{\pgfqpoint{1.982577in}{3.096046in}}%
\pgfpathlineto{\pgfqpoint{1.982577in}{3.111924in}}%
\pgfpathlineto{\pgfqpoint{1.988638in}{3.111924in}}%
\pgfpathlineto{\pgfqpoint{1.988638in}{3.097688in}}%
\pgfpathlineto{\pgfqpoint{1.994699in}{3.097688in}}%
\pgfpathlineto{\pgfqpoint{1.994699in}{3.109734in}}%
\pgfpathlineto{\pgfqpoint{2.000760in}{3.109734in}}%
\pgfpathlineto{\pgfqpoint{2.000760in}{3.098236in}}%
\pgfpathlineto{\pgfqpoint{2.006821in}{3.098236in}}%
\pgfpathlineto{\pgfqpoint{2.006821in}{3.119590in}}%
\pgfpathlineto{\pgfqpoint{2.012882in}{3.119590in}}%
\pgfpathlineto{\pgfqpoint{2.012882in}{3.098236in}}%
\pgfpathlineto{\pgfqpoint{2.018943in}{3.098236in}}%
\pgfpathlineto{\pgfqpoint{2.018943in}{3.132183in}}%
\pgfpathlineto{\pgfqpoint{2.025004in}{3.132183in}}%
\pgfpathlineto{\pgfqpoint{2.025004in}{3.126708in}}%
\pgfpathlineto{\pgfqpoint{2.031065in}{3.126708in}}%
\pgfpathlineto{\pgfqpoint{2.031065in}{3.120685in}}%
\pgfpathlineto{\pgfqpoint{2.037126in}{3.120685in}}%
\pgfpathlineto{\pgfqpoint{2.037126in}{3.157370in}}%
\pgfpathlineto{\pgfqpoint{2.043187in}{3.157370in}}%
\pgfpathlineto{\pgfqpoint{2.043187in}{3.113567in}}%
\pgfpathlineto{\pgfqpoint{2.049248in}{3.113567in}}%
\pgfpathlineto{\pgfqpoint{2.049248in}{3.137658in}}%
\pgfpathlineto{\pgfqpoint{2.055308in}{3.137658in}}%
\pgfpathlineto{\pgfqpoint{2.055308in}{3.100973in}}%
\pgfpathlineto{\pgfqpoint{2.061369in}{3.100973in}}%
\pgfpathlineto{\pgfqpoint{2.061369in}{3.131635in}}%
\pgfpathlineto{\pgfqpoint{2.067430in}{3.131635in}}%
\pgfpathlineto{\pgfqpoint{2.067430in}{3.101521in}}%
\pgfpathlineto{\pgfqpoint{2.073491in}{3.101521in}}%
\pgfpathlineto{\pgfqpoint{2.073491in}{3.128898in}}%
\pgfpathlineto{\pgfqpoint{2.079552in}{3.128898in}}%
\pgfpathlineto{\pgfqpoint{2.079552in}{3.148062in}}%
\pgfpathlineto{\pgfqpoint{2.085613in}{3.148062in}}%
\pgfpathlineto{\pgfqpoint{2.085613in}{3.126160in}}%
\pgfpathlineto{\pgfqpoint{2.097735in}{3.125613in}}%
\pgfpathlineto{\pgfqpoint{2.097735in}{3.096046in}}%
\pgfpathlineto{\pgfqpoint{2.103796in}{3.096046in}}%
\pgfpathlineto{\pgfqpoint{2.103796in}{3.142586in}}%
\pgfpathlineto{\pgfqpoint{2.109857in}{3.142586in}}%
\pgfpathlineto{\pgfqpoint{2.109857in}{3.109734in}}%
\pgfpathlineto{\pgfqpoint{2.115918in}{3.109734in}}%
\pgfpathlineto{\pgfqpoint{2.115918in}{3.126160in}}%
\pgfpathlineto{\pgfqpoint{2.121979in}{3.126160in}}%
\pgfpathlineto{\pgfqpoint{2.121979in}{3.131088in}}%
\pgfpathlineto{\pgfqpoint{2.128040in}{3.131088in}}%
\pgfpathlineto{\pgfqpoint{2.128040in}{3.154084in}}%
\pgfpathlineto{\pgfqpoint{2.134101in}{3.154084in}}%
\pgfpathlineto{\pgfqpoint{2.134101in}{3.135468in}}%
\pgfpathlineto{\pgfqpoint{2.140162in}{3.135468in}}%
\pgfpathlineto{\pgfqpoint{2.140162in}{3.129445in}}%
\pgfpathlineto{\pgfqpoint{2.146223in}{3.129445in}}%
\pgfpathlineto{\pgfqpoint{2.146223in}{3.166130in}}%
\pgfpathlineto{\pgfqpoint{2.152284in}{3.166130in}}%
\pgfpathlineto{\pgfqpoint{2.152284in}{3.104259in}}%
\pgfpathlineto{\pgfqpoint{2.158345in}{3.104259in}}%
\pgfpathlineto{\pgfqpoint{2.158345in}{3.150799in}}%
\pgfpathlineto{\pgfqpoint{2.164406in}{3.150799in}}%
\pgfpathlineto{\pgfqpoint{2.164406in}{3.133826in}}%
\pgfpathlineto{\pgfqpoint{2.170467in}{3.133826in}}%
\pgfpathlineto{\pgfqpoint{2.170467in}{3.157917in}}%
\pgfpathlineto{\pgfqpoint{2.176528in}{3.157917in}}%
\pgfpathlineto{\pgfqpoint{2.176528in}{3.120137in}}%
\pgfpathlineto{\pgfqpoint{2.182589in}{3.120137in}}%
\pgfpathlineto{\pgfqpoint{2.182589in}{3.152989in}}%
\pgfpathlineto{\pgfqpoint{2.188650in}{3.152989in}}%
\pgfpathlineto{\pgfqpoint{2.188650in}{3.164488in}}%
\pgfpathlineto{\pgfqpoint{2.194711in}{3.164488in}}%
\pgfpathlineto{\pgfqpoint{2.194711in}{3.129445in}}%
\pgfpathlineto{\pgfqpoint{2.200772in}{3.129445in}}%
\pgfpathlineto{\pgfqpoint{2.200772in}{3.149157in}}%
\pgfpathlineto{\pgfqpoint{2.206833in}{3.149157in}}%
\pgfpathlineto{\pgfqpoint{2.206833in}{3.133826in}}%
\pgfpathlineto{\pgfqpoint{2.212894in}{3.133826in}}%
\pgfpathlineto{\pgfqpoint{2.212894in}{3.150799in}}%
\pgfpathlineto{\pgfqpoint{2.218955in}{3.150799in}}%
\pgfpathlineto{\pgfqpoint{2.218955in}{3.140944in}}%
\pgfpathlineto{\pgfqpoint{2.225016in}{3.140944in}}%
\pgfpathlineto{\pgfqpoint{2.225016in}{3.136563in}}%
\pgfpathlineto{\pgfqpoint{2.231077in}{3.136563in}}%
\pgfpathlineto{\pgfqpoint{2.231077in}{3.126708in}}%
\pgfpathlineto{\pgfqpoint{2.237138in}{3.126708in}}%
\pgfpathlineto{\pgfqpoint{2.237138in}{3.161202in}}%
\pgfpathlineto{\pgfqpoint{2.243199in}{3.161202in}}%
\pgfpathlineto{\pgfqpoint{2.243199in}{3.155727in}}%
\pgfpathlineto{\pgfqpoint{2.249260in}{3.155727in}}%
\pgfpathlineto{\pgfqpoint{2.249260in}{3.141491in}}%
\pgfpathlineto{\pgfqpoint{2.255321in}{3.141491in}}%
\pgfpathlineto{\pgfqpoint{2.255321in}{3.167773in}}%
\pgfpathlineto{\pgfqpoint{2.261382in}{3.167773in}}%
\pgfpathlineto{\pgfqpoint{2.261382in}{3.157917in}}%
\pgfpathlineto{\pgfqpoint{2.267443in}{3.157917in}}%
\pgfpathlineto{\pgfqpoint{2.267443in}{3.182556in}}%
\pgfpathlineto{\pgfqpoint{2.273504in}{3.182556in}}%
\pgfpathlineto{\pgfqpoint{2.273504in}{3.126708in}}%
\pgfpathlineto{\pgfqpoint{2.279565in}{3.126708in}}%
\pgfpathlineto{\pgfqpoint{2.279565in}{3.194602in}}%
\pgfpathlineto{\pgfqpoint{2.285626in}{3.194602in}}%
\pgfpathlineto{\pgfqpoint{2.285626in}{3.142586in}}%
\pgfpathlineto{\pgfqpoint{2.291687in}{3.142586in}}%
\pgfpathlineto{\pgfqpoint{2.291687in}{3.180366in}}%
\pgfpathlineto{\pgfqpoint{2.297748in}{3.180366in}}%
\pgfpathlineto{\pgfqpoint{2.297748in}{3.185294in}}%
\pgfpathlineto{\pgfqpoint{2.303809in}{3.185294in}}%
\pgfpathlineto{\pgfqpoint{2.303809in}{3.144229in}}%
\pgfpathlineto{\pgfqpoint{2.309870in}{3.144229in}}%
\pgfpathlineto{\pgfqpoint{2.309870in}{3.168868in}}%
\pgfpathlineto{\pgfqpoint{2.315931in}{3.168868in}}%
\pgfpathlineto{\pgfqpoint{2.315931in}{3.138206in}}%
\pgfpathlineto{\pgfqpoint{2.321992in}{3.138206in}}%
\pgfpathlineto{\pgfqpoint{2.321992in}{3.177081in}}%
\pgfpathlineto{\pgfqpoint{2.328053in}{3.177081in}}%
\pgfpathlineto{\pgfqpoint{2.328053in}{3.145871in}}%
\pgfpathlineto{\pgfqpoint{2.334114in}{3.145871in}}%
\pgfpathlineto{\pgfqpoint{2.334114in}{3.165583in}}%
\pgfpathlineto{\pgfqpoint{2.340175in}{3.165583in}}%
\pgfpathlineto{\pgfqpoint{2.340175in}{3.136563in}}%
\pgfpathlineto{\pgfqpoint{2.346235in}{3.136563in}}%
\pgfpathlineto{\pgfqpoint{2.346235in}{3.183651in}}%
\pgfpathlineto{\pgfqpoint{2.352296in}{3.183651in}}%
\pgfpathlineto{\pgfqpoint{2.352296in}{3.162845in}}%
\pgfpathlineto{\pgfqpoint{2.358357in}{3.162845in}}%
\pgfpathlineto{\pgfqpoint{2.358357in}{3.206100in}}%
\pgfpathlineto{\pgfqpoint{2.364418in}{3.206100in}}%
\pgfpathlineto{\pgfqpoint{2.364418in}{3.172153in}}%
\pgfpathlineto{\pgfqpoint{2.370479in}{3.172153in}}%
\pgfpathlineto{\pgfqpoint{2.370479in}{3.152442in}}%
\pgfpathlineto{\pgfqpoint{2.376540in}{3.152442in}}%
\pgfpathlineto{\pgfqpoint{2.376540in}{3.174891in}}%
\pgfpathlineto{\pgfqpoint{2.382601in}{3.174891in}}%
\pgfpathlineto{\pgfqpoint{2.382601in}{3.157370in}}%
\pgfpathlineto{\pgfqpoint{2.388662in}{3.157370in}}%
\pgfpathlineto{\pgfqpoint{2.388662in}{3.205553in}}%
\pgfpathlineto{\pgfqpoint{2.394723in}{3.205553in}}%
\pgfpathlineto{\pgfqpoint{2.394723in}{3.154084in}}%
\pgfpathlineto{\pgfqpoint{2.400784in}{3.154084in}}%
\pgfpathlineto{\pgfqpoint{2.400784in}{3.190222in}}%
\pgfpathlineto{\pgfqpoint{2.406845in}{3.190222in}}%
\pgfpathlineto{\pgfqpoint{2.406845in}{3.168320in}}%
\pgfpathlineto{\pgfqpoint{2.412906in}{3.168320in}}%
\pgfpathlineto{\pgfqpoint{2.412906in}{3.195697in}}%
\pgfpathlineto{\pgfqpoint{2.418967in}{3.195697in}}%
\pgfpathlineto{\pgfqpoint{2.418967in}{3.163940in}}%
\pgfpathlineto{\pgfqpoint{2.425028in}{3.163940in}}%
\pgfpathlineto{\pgfqpoint{2.425028in}{3.180366in}}%
\pgfpathlineto{\pgfqpoint{2.431089in}{3.180366in}}%
\pgfpathlineto{\pgfqpoint{2.431089in}{3.206100in}}%
\pgfpathlineto{\pgfqpoint{2.437150in}{3.206100in}}%
\pgfpathlineto{\pgfqpoint{2.437150in}{3.179819in}}%
\pgfpathlineto{\pgfqpoint{2.443211in}{3.179819in}}%
\pgfpathlineto{\pgfqpoint{2.443211in}{3.198435in}}%
\pgfpathlineto{\pgfqpoint{2.449272in}{3.198435in}}%
\pgfpathlineto{\pgfqpoint{2.449272in}{3.171606in}}%
\pgfpathlineto{\pgfqpoint{2.455333in}{3.171606in}}%
\pgfpathlineto{\pgfqpoint{2.455333in}{3.219789in}}%
\pgfpathlineto{\pgfqpoint{2.461394in}{3.219789in}}%
\pgfpathlineto{\pgfqpoint{2.461394in}{3.177081in}}%
\pgfpathlineto{\pgfqpoint{2.467455in}{3.177081in}}%
\pgfpathlineto{\pgfqpoint{2.467455in}{3.191864in}}%
\pgfpathlineto{\pgfqpoint{2.473516in}{3.191864in}}%
\pgfpathlineto{\pgfqpoint{2.473516in}{3.249356in}}%
\pgfpathlineto{\pgfqpoint{2.479577in}{3.249356in}}%
\pgfpathlineto{\pgfqpoint{2.479577in}{3.173248in}}%
\pgfpathlineto{\pgfqpoint{2.485638in}{3.173248in}}%
\pgfpathlineto{\pgfqpoint{2.485638in}{3.187484in}}%
\pgfpathlineto{\pgfqpoint{2.491699in}{3.187484in}}%
\pgfpathlineto{\pgfqpoint{2.491699in}{3.174343in}}%
\pgfpathlineto{\pgfqpoint{2.497760in}{3.174343in}}%
\pgfpathlineto{\pgfqpoint{2.497760in}{3.225264in}}%
\pgfpathlineto{\pgfqpoint{2.503821in}{3.225264in}}%
\pgfpathlineto{\pgfqpoint{2.503821in}{3.185294in}}%
\pgfpathlineto{\pgfqpoint{2.509882in}{3.185294in}}%
\pgfpathlineto{\pgfqpoint{2.509882in}{3.189127in}}%
\pgfpathlineto{\pgfqpoint{2.515943in}{3.189127in}}%
\pgfpathlineto{\pgfqpoint{2.515943in}{3.165583in}}%
\pgfpathlineto{\pgfqpoint{2.522004in}{3.165583in}}%
\pgfpathlineto{\pgfqpoint{2.522004in}{3.231835in}}%
\pgfpathlineto{\pgfqpoint{2.528065in}{3.231835in}}%
\pgfpathlineto{\pgfqpoint{2.528065in}{3.200077in}}%
\pgfpathlineto{\pgfqpoint{2.534126in}{3.200077in}}%
\pgfpathlineto{\pgfqpoint{2.534126in}{3.190222in}}%
\pgfpathlineto{\pgfqpoint{2.540187in}{3.190222in}}%
\pgfpathlineto{\pgfqpoint{2.540187in}{3.245523in}}%
\pgfpathlineto{\pgfqpoint{2.546248in}{3.245523in}}%
\pgfpathlineto{\pgfqpoint{2.546248in}{3.188579in}}%
\pgfpathlineto{\pgfqpoint{2.552309in}{3.188579in}}%
\pgfpathlineto{\pgfqpoint{2.552309in}{3.209386in}}%
\pgfpathlineto{\pgfqpoint{2.558370in}{3.209386in}}%
\pgfpathlineto{\pgfqpoint{2.558370in}{3.201173in}}%
\pgfpathlineto{\pgfqpoint{2.564431in}{3.201173in}}%
\pgfpathlineto{\pgfqpoint{2.564431in}{3.225812in}}%
\pgfpathlineto{\pgfqpoint{2.570492in}{3.225812in}}%
\pgfpathlineto{\pgfqpoint{2.570492in}{3.212671in}}%
\pgfpathlineto{\pgfqpoint{2.576553in}{3.212671in}}%
\pgfpathlineto{\pgfqpoint{2.576553in}{3.225812in}}%
\pgfpathlineto{\pgfqpoint{2.582614in}{3.225812in}}%
\pgfpathlineto{\pgfqpoint{2.582614in}{3.242785in}}%
\pgfpathlineto{\pgfqpoint{2.588675in}{3.242785in}}%
\pgfpathlineto{\pgfqpoint{2.588675in}{3.188579in}}%
\pgfpathlineto{\pgfqpoint{2.594736in}{3.188579in}}%
\pgfpathlineto{\pgfqpoint{2.594736in}{3.250998in}}%
\pgfpathlineto{\pgfqpoint{2.600797in}{3.250998in}}%
\pgfpathlineto{\pgfqpoint{2.600797in}{3.213218in}}%
\pgfpathlineto{\pgfqpoint{2.606858in}{3.213218in}}%
\pgfpathlineto{\pgfqpoint{2.606858in}{3.248808in}}%
\pgfpathlineto{\pgfqpoint{2.612919in}{3.248808in}}%
\pgfpathlineto{\pgfqpoint{2.612919in}{3.198435in}}%
\pgfpathlineto{\pgfqpoint{2.618980in}{3.198435in}}%
\pgfpathlineto{\pgfqpoint{2.618980in}{3.254284in}}%
\pgfpathlineto{\pgfqpoint{2.625041in}{3.254284in}}%
\pgfpathlineto{\pgfqpoint{2.625041in}{3.201173in}}%
\pgfpathlineto{\pgfqpoint{2.631102in}{3.201173in}}%
\pgfpathlineto{\pgfqpoint{2.631102in}{3.259759in}}%
\pgfpathlineto{\pgfqpoint{2.637162in}{3.259759in}}%
\pgfpathlineto{\pgfqpoint{2.637162in}{3.247166in}}%
\pgfpathlineto{\pgfqpoint{2.643223in}{3.247166in}}%
\pgfpathlineto{\pgfqpoint{2.643223in}{3.214861in}}%
\pgfpathlineto{\pgfqpoint{2.649284in}{3.214861in}}%
\pgfpathlineto{\pgfqpoint{2.649284in}{3.261949in}}%
\pgfpathlineto{\pgfqpoint{2.655345in}{3.261949in}}%
\pgfpathlineto{\pgfqpoint{2.655345in}{3.203910in}}%
\pgfpathlineto{\pgfqpoint{2.661406in}{3.203910in}}%
\pgfpathlineto{\pgfqpoint{2.661406in}{3.244975in}}%
\pgfpathlineto{\pgfqpoint{2.667467in}{3.244975in}}%
\pgfpathlineto{\pgfqpoint{2.667467in}{3.209386in}}%
\pgfpathlineto{\pgfqpoint{2.673528in}{3.209386in}}%
\pgfpathlineto{\pgfqpoint{2.673528in}{3.295349in}}%
\pgfpathlineto{\pgfqpoint{2.679589in}{3.295349in}}%
\pgfpathlineto{\pgfqpoint{2.679589in}{3.230192in}}%
\pgfpathlineto{\pgfqpoint{2.685650in}{3.230192in}}%
\pgfpathlineto{\pgfqpoint{2.685650in}{3.276732in}}%
\pgfpathlineto{\pgfqpoint{2.691711in}{3.276732in}}%
\pgfpathlineto{\pgfqpoint{2.691711in}{3.229097in}}%
\pgfpathlineto{\pgfqpoint{2.697772in}{3.229097in}}%
\pgfpathlineto{\pgfqpoint{2.697772in}{3.296991in}}%
\pgfpathlineto{\pgfqpoint{2.703833in}{3.296991in}}%
\pgfpathlineto{\pgfqpoint{2.703833in}{3.257569in}}%
\pgfpathlineto{\pgfqpoint{2.709894in}{3.257569in}}%
\pgfpathlineto{\pgfqpoint{2.709894in}{3.251546in}}%
\pgfpathlineto{\pgfqpoint{2.715955in}{3.251546in}}%
\pgfpathlineto{\pgfqpoint{2.715955in}{3.282755in}}%
\pgfpathlineto{\pgfqpoint{2.722016in}{3.282755in}}%
\pgfpathlineto{\pgfqpoint{2.722016in}{3.252093in}}%
\pgfpathlineto{\pgfqpoint{2.728077in}{3.252093in}}%
\pgfpathlineto{\pgfqpoint{2.728077in}{3.308490in}}%
\pgfpathlineto{\pgfqpoint{2.734138in}{3.308490in}}%
\pgfpathlineto{\pgfqpoint{2.734138in}{3.250998in}}%
\pgfpathlineto{\pgfqpoint{2.740199in}{3.250998in}}%
\pgfpathlineto{\pgfqpoint{2.740199in}{3.283850in}}%
\pgfpathlineto{\pgfqpoint{2.746260in}{3.283850in}}%
\pgfpathlineto{\pgfqpoint{2.746260in}{3.238405in}}%
\pgfpathlineto{\pgfqpoint{2.752321in}{3.238405in}}%
\pgfpathlineto{\pgfqpoint{2.752321in}{3.299181in}}%
\pgfpathlineto{\pgfqpoint{2.758382in}{3.299181in}}%
\pgfpathlineto{\pgfqpoint{2.758382in}{3.306847in}}%
\pgfpathlineto{\pgfqpoint{2.764443in}{3.306847in}}%
\pgfpathlineto{\pgfqpoint{2.764443in}{3.264687in}}%
\pgfpathlineto{\pgfqpoint{2.770504in}{3.264687in}}%
\pgfpathlineto{\pgfqpoint{2.770504in}{3.298634in}}%
\pgfpathlineto{\pgfqpoint{2.776565in}{3.298634in}}%
\pgfpathlineto{\pgfqpoint{2.776565in}{3.268519in}}%
\pgfpathlineto{\pgfqpoint{2.782626in}{3.268519in}}%
\pgfpathlineto{\pgfqpoint{2.782626in}{3.363243in}}%
\pgfpathlineto{\pgfqpoint{2.788687in}{3.363243in}}%
\pgfpathlineto{\pgfqpoint{2.788687in}{3.282755in}}%
\pgfpathlineto{\pgfqpoint{2.794748in}{3.282755in}}%
\pgfpathlineto{\pgfqpoint{2.794748in}{3.329843in}}%
\pgfpathlineto{\pgfqpoint{2.800809in}{3.329843in}}%
\pgfpathlineto{\pgfqpoint{2.800809in}{3.283850in}}%
\pgfpathlineto{\pgfqpoint{2.806870in}{3.283850in}}%
\pgfpathlineto{\pgfqpoint{2.806870in}{3.306847in}}%
\pgfpathlineto{\pgfqpoint{2.812931in}{3.306847in}}%
\pgfpathlineto{\pgfqpoint{2.812931in}{3.348460in}}%
\pgfpathlineto{\pgfqpoint{2.818992in}{3.348460in}}%
\pgfpathlineto{\pgfqpoint{2.818992in}{3.280018in}}%
\pgfpathlineto{\pgfqpoint{2.825053in}{3.280018in}}%
\pgfpathlineto{\pgfqpoint{2.825053in}{3.354483in}}%
\pgfpathlineto{\pgfqpoint{2.831114in}{3.354483in}}%
\pgfpathlineto{\pgfqpoint{2.831114in}{3.281660in}}%
\pgfpathlineto{\pgfqpoint{2.837175in}{3.281660in}}%
\pgfpathlineto{\pgfqpoint{2.837175in}{3.311227in}}%
\pgfpathlineto{\pgfqpoint{2.843236in}{3.311227in}}%
\pgfpathlineto{\pgfqpoint{2.843236in}{3.275090in}}%
\pgfpathlineto{\pgfqpoint{2.849297in}{3.275090in}}%
\pgfpathlineto{\pgfqpoint{2.849297in}{3.393905in}}%
\pgfpathlineto{\pgfqpoint{2.855358in}{3.393905in}}%
\pgfpathlineto{\pgfqpoint{2.855358in}{3.304657in}}%
\pgfpathlineto{\pgfqpoint{2.861419in}{3.304657in}}%
\pgfpathlineto{\pgfqpoint{2.861419in}{3.357220in}}%
\pgfpathlineto{\pgfqpoint{2.867480in}{3.357220in}}%
\pgfpathlineto{\pgfqpoint{2.867480in}{3.376384in}}%
\pgfpathlineto{\pgfqpoint{2.873541in}{3.376384in}}%
\pgfpathlineto{\pgfqpoint{2.873541in}{3.324368in}}%
\pgfpathlineto{\pgfqpoint{2.879602in}{3.324368in}}%
\pgfpathlineto{\pgfqpoint{2.879602in}{3.352292in}}%
\pgfpathlineto{\pgfqpoint{2.885663in}{3.352292in}}%
\pgfpathlineto{\pgfqpoint{2.885663in}{3.346270in}}%
\pgfpathlineto{\pgfqpoint{2.891724in}{3.346270in}}%
\pgfpathlineto{\pgfqpoint{2.891724in}{3.389525in}}%
\pgfpathlineto{\pgfqpoint{2.897785in}{3.389525in}}%
\pgfpathlineto{\pgfqpoint{2.897785in}{3.339152in}}%
\pgfpathlineto{\pgfqpoint{2.903846in}{3.339152in}}%
\pgfpathlineto{\pgfqpoint{2.903846in}{3.379669in}}%
\pgfpathlineto{\pgfqpoint{2.909907in}{3.379669in}}%
\pgfpathlineto{\pgfqpoint{2.909907in}{3.335866in}}%
\pgfpathlineto{\pgfqpoint{2.915968in}{3.335866in}}%
\pgfpathlineto{\pgfqpoint{2.915968in}{3.445921in}}%
\pgfpathlineto{\pgfqpoint{2.922029in}{3.445921in}}%
\pgfpathlineto{\pgfqpoint{2.922029in}{3.406499in}}%
\pgfpathlineto{\pgfqpoint{2.928089in}{3.406499in}}%
\pgfpathlineto{\pgfqpoint{2.928089in}{3.357768in}}%
\pgfpathlineto{\pgfqpoint{2.934150in}{3.357768in}}%
\pgfpathlineto{\pgfqpoint{2.934150in}{3.416354in}}%
\pgfpathlineto{\pgfqpoint{2.940211in}{3.416354in}}%
\pgfpathlineto{\pgfqpoint{2.940211in}{3.368171in}}%
\pgfpathlineto{\pgfqpoint{2.946272in}{3.368171in}}%
\pgfpathlineto{\pgfqpoint{2.946272in}{3.432780in}}%
\pgfpathlineto{\pgfqpoint{2.952333in}{3.432780in}}%
\pgfpathlineto{\pgfqpoint{2.952333in}{3.380217in}}%
\pgfpathlineto{\pgfqpoint{2.958394in}{3.380217in}}%
\pgfpathlineto{\pgfqpoint{2.958394in}{3.461252in}}%
\pgfpathlineto{\pgfqpoint{2.964455in}{3.461252in}}%
\pgfpathlineto{\pgfqpoint{2.964455in}{3.387335in}}%
\pgfpathlineto{\pgfqpoint{2.970516in}{3.387335in}}%
\pgfpathlineto{\pgfqpoint{2.970516in}{3.483701in}}%
\pgfpathlineto{\pgfqpoint{2.976577in}{3.483701in}}%
\pgfpathlineto{\pgfqpoint{2.976577in}{3.496842in}}%
\pgfpathlineto{\pgfqpoint{2.982638in}{3.496842in}}%
\pgfpathlineto{\pgfqpoint{2.982638in}{3.386787in}}%
\pgfpathlineto{\pgfqpoint{2.988699in}{3.386787in}}%
\pgfpathlineto{\pgfqpoint{2.988699in}{3.444279in}}%
\pgfpathlineto{\pgfqpoint{2.994760in}{3.444279in}}%
\pgfpathlineto{\pgfqpoint{2.994760in}{3.410331in}}%
\pgfpathlineto{\pgfqpoint{3.000821in}{3.410331in}}%
\pgfpathlineto{\pgfqpoint{3.000821in}{3.508888in}}%
\pgfpathlineto{\pgfqpoint{3.006882in}{3.508888in}}%
\pgfpathlineto{\pgfqpoint{3.006882in}{3.417449in}}%
\pgfpathlineto{\pgfqpoint{3.012943in}{3.417449in}}%
\pgfpathlineto{\pgfqpoint{3.012943in}{3.480416in}}%
\pgfpathlineto{\pgfqpoint{3.019004in}{3.480416in}}%
\pgfpathlineto{\pgfqpoint{3.019004in}{3.427305in}}%
\pgfpathlineto{\pgfqpoint{3.025065in}{3.427305in}}%
\pgfpathlineto{\pgfqpoint{3.025065in}{3.520386in}}%
\pgfpathlineto{\pgfqpoint{3.031126in}{3.520386in}}%
\pgfpathlineto{\pgfqpoint{3.031126in}{3.500675in}}%
\pgfpathlineto{\pgfqpoint{3.037187in}{3.500675in}}%
\pgfpathlineto{\pgfqpoint{3.037187in}{3.431138in}}%
\pgfpathlineto{\pgfqpoint{3.043248in}{3.431138in}}%
\pgfpathlineto{\pgfqpoint{3.043248in}{3.556523in}}%
\pgfpathlineto{\pgfqpoint{3.049309in}{3.556523in}}%
\pgfpathlineto{\pgfqpoint{3.049309in}{3.447016in}}%
\pgfpathlineto{\pgfqpoint{3.055370in}{3.447016in}}%
\pgfpathlineto{\pgfqpoint{3.055370in}{3.528599in}}%
\pgfpathlineto{\pgfqpoint{3.061431in}{3.528599in}}%
\pgfpathlineto{\pgfqpoint{3.061431in}{3.437161in}}%
\pgfpathlineto{\pgfqpoint{3.067492in}{3.437161in}}%
\pgfpathlineto{\pgfqpoint{3.067492in}{3.567474in}}%
\pgfpathlineto{\pgfqpoint{3.073553in}{3.567474in}}%
\pgfpathlineto{\pgfqpoint{3.073553in}{3.467823in}}%
\pgfpathlineto{\pgfqpoint{3.079614in}{3.467823in}}%
\pgfpathlineto{\pgfqpoint{3.079614in}{3.550501in}}%
\pgfpathlineto{\pgfqpoint{3.085675in}{3.550501in}}%
\pgfpathlineto{\pgfqpoint{3.085675in}{3.510530in}}%
\pgfpathlineto{\pgfqpoint{3.091736in}{3.510530in}}%
\pgfpathlineto{\pgfqpoint{3.091736in}{3.575140in}}%
\pgfpathlineto{\pgfqpoint{3.097797in}{3.575140in}}%
\pgfpathlineto{\pgfqpoint{3.097797in}{3.561451in}}%
\pgfpathlineto{\pgfqpoint{3.103858in}{3.561451in}}%
\pgfpathlineto{\pgfqpoint{3.103858in}{3.453039in}}%
\pgfpathlineto{\pgfqpoint{3.109919in}{3.453039in}}%
\pgfpathlineto{\pgfqpoint{3.109919in}{3.609087in}}%
\pgfpathlineto{\pgfqpoint{3.115980in}{3.609087in}}%
\pgfpathlineto{\pgfqpoint{3.115980in}{3.534074in}}%
\pgfpathlineto{\pgfqpoint{3.122041in}{3.534074in}}%
\pgfpathlineto{\pgfqpoint{3.122041in}{3.583353in}}%
\pgfpathlineto{\pgfqpoint{3.128102in}{3.583353in}}%
\pgfpathlineto{\pgfqpoint{3.128102in}{3.493009in}}%
\pgfpathlineto{\pgfqpoint{3.134163in}{3.493009in}}%
\pgfpathlineto{\pgfqpoint{3.134163in}{3.604707in}}%
\pgfpathlineto{\pgfqpoint{3.140224in}{3.604707in}}%
\pgfpathlineto{\pgfqpoint{3.140224in}{3.535717in}}%
\pgfpathlineto{\pgfqpoint{3.146285in}{3.535717in}}%
\pgfpathlineto{\pgfqpoint{3.146285in}{3.558166in}}%
\pgfpathlineto{\pgfqpoint{3.152346in}{3.558166in}}%
\pgfpathlineto{\pgfqpoint{3.152346in}{3.628251in}}%
\pgfpathlineto{\pgfqpoint{3.158407in}{3.628251in}}%
\pgfpathlineto{\pgfqpoint{3.158407in}{3.549953in}}%
\pgfpathlineto{\pgfqpoint{3.164468in}{3.549953in}}%
\pgfpathlineto{\pgfqpoint{3.164468in}{3.606897in}}%
\pgfpathlineto{\pgfqpoint{3.170529in}{3.606897in}}%
\pgfpathlineto{\pgfqpoint{3.170529in}{3.550501in}}%
\pgfpathlineto{\pgfqpoint{3.176590in}{3.550501in}}%
\pgfpathlineto{\pgfqpoint{3.176590in}{3.668768in}}%
\pgfpathlineto{\pgfqpoint{3.182651in}{3.668768in}}%
\pgfpathlineto{\pgfqpoint{3.182651in}{3.566379in}}%
\pgfpathlineto{\pgfqpoint{3.188712in}{3.566379in}}%
\pgfpathlineto{\pgfqpoint{3.188712in}{3.620038in}}%
\pgfpathlineto{\pgfqpoint{3.194773in}{3.620038in}}%
\pgfpathlineto{\pgfqpoint{3.194773in}{3.574592in}}%
\pgfpathlineto{\pgfqpoint{3.200834in}{3.574592in}}%
\pgfpathlineto{\pgfqpoint{3.200834in}{3.674791in}}%
\pgfpathlineto{\pgfqpoint{3.206895in}{3.674791in}}%
\pgfpathlineto{\pgfqpoint{3.206895in}{3.643034in}}%
\pgfpathlineto{\pgfqpoint{3.212956in}{3.643034in}}%
\pgfpathlineto{\pgfqpoint{3.212956in}{3.610729in}}%
\pgfpathlineto{\pgfqpoint{3.219016in}{3.610729in}}%
\pgfpathlineto{\pgfqpoint{3.219016in}{3.713119in}}%
\pgfpathlineto{\pgfqpoint{3.225077in}{3.713119in}}%
\pgfpathlineto{\pgfqpoint{3.225077in}{3.621680in}}%
\pgfpathlineto{\pgfqpoint{3.231138in}{3.621680in}}%
\pgfpathlineto{\pgfqpoint{3.231138in}{3.697240in}}%
\pgfpathlineto{\pgfqpoint{3.237199in}{3.697240in}}%
\pgfpathlineto{\pgfqpoint{3.237199in}{3.601969in}}%
\pgfpathlineto{\pgfqpoint{3.243260in}{3.601969in}}%
\pgfpathlineto{\pgfqpoint{3.243260in}{3.727355in}}%
\pgfpathlineto{\pgfqpoint{3.249321in}{3.727355in}}%
\pgfpathlineto{\pgfqpoint{3.249321in}{3.612920in}}%
\pgfpathlineto{\pgfqpoint{3.255382in}{3.612920in}}%
\pgfpathlineto{\pgfqpoint{3.255382in}{3.709286in}}%
\pgfpathlineto{\pgfqpoint{3.261443in}{3.709286in}}%
\pgfpathlineto{\pgfqpoint{3.261443in}{3.732283in}}%
\pgfpathlineto{\pgfqpoint{3.267504in}{3.732283in}}%
\pgfpathlineto{\pgfqpoint{3.267504in}{3.647962in}}%
\pgfpathlineto{\pgfqpoint{3.273565in}{3.647962in}}%
\pgfpathlineto{\pgfqpoint{3.273565in}{3.750899in}}%
\pgfpathlineto{\pgfqpoint{3.279626in}{3.750899in}}%
\pgfpathlineto{\pgfqpoint{3.279626in}{3.599231in}}%
\pgfpathlineto{\pgfqpoint{3.285687in}{3.599231in}}%
\pgfpathlineto{\pgfqpoint{3.285687in}{3.821531in}}%
\pgfpathlineto{\pgfqpoint{3.291748in}{3.821531in}}%
\pgfpathlineto{\pgfqpoint{3.291748in}{3.640296in}}%
\pgfpathlineto{\pgfqpoint{3.297809in}{3.640296in}}%
\pgfpathlineto{\pgfqpoint{3.297809in}{3.748161in}}%
\pgfpathlineto{\pgfqpoint{3.303870in}{3.748161in}}%
\pgfpathlineto{\pgfqpoint{3.303870in}{3.661103in}}%
\pgfpathlineto{\pgfqpoint{3.309931in}{3.661103in}}%
\pgfpathlineto{\pgfqpoint{3.309931in}{3.767325in}}%
\pgfpathlineto{\pgfqpoint{3.315992in}{3.767325in}}%
\pgfpathlineto{\pgfqpoint{3.315992in}{3.739400in}}%
\pgfpathlineto{\pgfqpoint{3.322053in}{3.739400in}}%
\pgfpathlineto{\pgfqpoint{3.322053in}{3.649605in}}%
\pgfpathlineto{\pgfqpoint{3.328114in}{3.649605in}}%
\pgfpathlineto{\pgfqpoint{3.328114in}{3.797987in}}%
\pgfpathlineto{\pgfqpoint{3.334175in}{3.797987in}}%
\pgfpathlineto{\pgfqpoint{3.334175in}{3.633178in}}%
\pgfpathlineto{\pgfqpoint{3.340236in}{3.633178in}}%
\pgfpathlineto{\pgfqpoint{3.340236in}{3.731735in}}%
\pgfpathlineto{\pgfqpoint{3.346297in}{3.731735in}}%
\pgfpathlineto{\pgfqpoint{3.346297in}{3.689027in}}%
\pgfpathlineto{\pgfqpoint{3.352358in}{3.689027in}}%
\pgfpathlineto{\pgfqpoint{3.352358in}{3.831387in}}%
\pgfpathlineto{\pgfqpoint{3.358419in}{3.831387in}}%
\pgfpathlineto{\pgfqpoint{3.358419in}{3.627156in}}%
\pgfpathlineto{\pgfqpoint{3.364480in}{3.627156in}}%
\pgfpathlineto{\pgfqpoint{3.364480in}{3.767325in}}%
\pgfpathlineto{\pgfqpoint{3.370541in}{3.767325in}}%
\pgfpathlineto{\pgfqpoint{3.370541in}{3.814413in}}%
\pgfpathlineto{\pgfqpoint{3.376602in}{3.814413in}}%
\pgfpathlineto{\pgfqpoint{3.376602in}{3.664388in}}%
\pgfpathlineto{\pgfqpoint{3.382663in}{3.664388in}}%
\pgfpathlineto{\pgfqpoint{3.382663in}{3.723522in}}%
\pgfpathlineto{\pgfqpoint{3.388724in}{3.723522in}}%
\pgfpathlineto{\pgfqpoint{3.388724in}{3.653437in}}%
\pgfpathlineto{\pgfqpoint{3.394785in}{3.653437in}}%
\pgfpathlineto{\pgfqpoint{3.394785in}{3.724069in}}%
\pgfpathlineto{\pgfqpoint{3.400846in}{3.724069in}}%
\pgfpathlineto{\pgfqpoint{3.400846in}{3.652890in}}%
\pgfpathlineto{\pgfqpoint{3.406907in}{3.652890in}}%
\pgfpathlineto{\pgfqpoint{3.406907in}{3.761302in}}%
\pgfpathlineto{\pgfqpoint{3.412968in}{3.761302in}}%
\pgfpathlineto{\pgfqpoint{3.412968in}{3.637559in}}%
\pgfpathlineto{\pgfqpoint{3.419029in}{3.637559in}}%
\pgfpathlineto{\pgfqpoint{3.419029in}{3.743781in}}%
\pgfpathlineto{\pgfqpoint{3.425090in}{3.743781in}}%
\pgfpathlineto{\pgfqpoint{3.425090in}{3.594851in}}%
\pgfpathlineto{\pgfqpoint{3.431151in}{3.594851in}}%
\pgfpathlineto{\pgfqpoint{3.431151in}{3.670411in}}%
\pgfpathlineto{\pgfqpoint{3.437212in}{3.670411in}}%
\pgfpathlineto{\pgfqpoint{3.437212in}{3.765682in}}%
\pgfpathlineto{\pgfqpoint{3.443273in}{3.765682in}}%
\pgfpathlineto{\pgfqpoint{3.443273in}{3.652342in}}%
\pgfpathlineto{\pgfqpoint{3.449334in}{3.652342in}}%
\pgfpathlineto{\pgfqpoint{3.449334in}{3.729545in}}%
\pgfpathlineto{\pgfqpoint{3.455395in}{3.729545in}}%
\pgfpathlineto{\pgfqpoint{3.455395in}{3.632083in}}%
\pgfpathlineto{\pgfqpoint{3.461456in}{3.632083in}}%
\pgfpathlineto{\pgfqpoint{3.461456in}{3.771158in}}%
\pgfpathlineto{\pgfqpoint{3.467517in}{3.771158in}}%
\pgfpathlineto{\pgfqpoint{3.467517in}{3.607444in}}%
\pgfpathlineto{\pgfqpoint{3.473578in}{3.607444in}}%
\pgfpathlineto{\pgfqpoint{3.473578in}{3.701620in}}%
\pgfpathlineto{\pgfqpoint{3.479639in}{3.701620in}}%
\pgfpathlineto{\pgfqpoint{3.479639in}{3.616205in}}%
\pgfpathlineto{\pgfqpoint{3.485700in}{3.616205in}}%
\pgfpathlineto{\pgfqpoint{3.485700in}{3.708738in}}%
\pgfpathlineto{\pgfqpoint{3.491761in}{3.708738in}}%
\pgfpathlineto{\pgfqpoint{3.491761in}{3.671506in}}%
\pgfpathlineto{\pgfqpoint{3.497822in}{3.671506in}}%
\pgfpathlineto{\pgfqpoint{3.497822in}{3.597589in}}%
\pgfpathlineto{\pgfqpoint{3.503883in}{3.597589in}}%
\pgfpathlineto{\pgfqpoint{3.503883in}{3.693407in}}%
\pgfpathlineto{\pgfqpoint{3.509943in}{3.693407in}}%
\pgfpathlineto{\pgfqpoint{3.509943in}{3.583900in}}%
\pgfpathlineto{\pgfqpoint{3.516004in}{3.583900in}}%
\pgfpathlineto{\pgfqpoint{3.516004in}{3.661650in}}%
\pgfpathlineto{\pgfqpoint{3.522065in}{3.661650in}}%
\pgfpathlineto{\pgfqpoint{3.522065in}{3.560904in}}%
\pgfpathlineto{\pgfqpoint{3.528126in}{3.560904in}}%
\pgfpathlineto{\pgfqpoint{3.528126in}{3.690670in}}%
\pgfpathlineto{\pgfqpoint{3.534187in}{3.690670in}}%
\pgfpathlineto{\pgfqpoint{3.534187in}{3.559261in}}%
\pgfpathlineto{\pgfqpoint{3.540248in}{3.559261in}}%
\pgfpathlineto{\pgfqpoint{3.540248in}{3.653437in}}%
\pgfpathlineto{\pgfqpoint{3.546309in}{3.653437in}}%
\pgfpathlineto{\pgfqpoint{3.546309in}{3.716404in}}%
\pgfpathlineto{\pgfqpoint{3.552370in}{3.716404in}}%
\pgfpathlineto{\pgfqpoint{3.552370in}{3.548858in}}%
\pgfpathlineto{\pgfqpoint{3.558431in}{3.548858in}}%
\pgfpathlineto{\pgfqpoint{3.558431in}{3.635916in}}%
\pgfpathlineto{\pgfqpoint{3.564492in}{3.635916in}}%
\pgfpathlineto{\pgfqpoint{3.564492in}{3.549405in}}%
\pgfpathlineto{\pgfqpoint{3.570553in}{3.549405in}}%
\pgfpathlineto{\pgfqpoint{3.570553in}{3.641939in}}%
\pgfpathlineto{\pgfqpoint{3.576614in}{3.641939in}}%
\pgfpathlineto{\pgfqpoint{3.576614in}{3.528052in}}%
\pgfpathlineto{\pgfqpoint{3.582675in}{3.528052in}}%
\pgfpathlineto{\pgfqpoint{3.582675in}{3.616205in}}%
\pgfpathlineto{\pgfqpoint{3.588736in}{3.616205in}}%
\pgfpathlineto{\pgfqpoint{3.588736in}{3.529147in}}%
\pgfpathlineto{\pgfqpoint{3.594797in}{3.529147in}}%
\pgfpathlineto{\pgfqpoint{3.594797in}{3.617847in}}%
\pgfpathlineto{\pgfqpoint{3.600858in}{3.617847in}}%
\pgfpathlineto{\pgfqpoint{3.600858in}{3.612920in}}%
\pgfpathlineto{\pgfqpoint{3.606919in}{3.612920in}}%
\pgfpathlineto{\pgfqpoint{3.606919in}{3.534074in}}%
\pgfpathlineto{\pgfqpoint{3.612980in}{3.534074in}}%
\pgfpathlineto{\pgfqpoint{3.612980in}{3.649057in}}%
\pgfpathlineto{\pgfqpoint{3.619041in}{3.649057in}}%
\pgfpathlineto{\pgfqpoint{3.619041in}{3.485344in}}%
\pgfpathlineto{\pgfqpoint{3.625102in}{3.485344in}}%
\pgfpathlineto{\pgfqpoint{3.625102in}{3.564736in}}%
\pgfpathlineto{\pgfqpoint{3.631163in}{3.564736in}}%
\pgfpathlineto{\pgfqpoint{3.631163in}{3.542287in}}%
\pgfpathlineto{\pgfqpoint{3.637224in}{3.542287in}}%
\pgfpathlineto{\pgfqpoint{3.637224in}{3.524219in}}%
\pgfpathlineto{\pgfqpoint{3.643285in}{3.524219in}}%
\pgfpathlineto{\pgfqpoint{3.643285in}{3.504507in}}%
\pgfpathlineto{\pgfqpoint{3.649346in}{3.504507in}}%
\pgfpathlineto{\pgfqpoint{3.649346in}{3.577877in}}%
\pgfpathlineto{\pgfqpoint{3.655407in}{3.577877in}}%
\pgfpathlineto{\pgfqpoint{3.655407in}{3.537360in}}%
\pgfpathlineto{\pgfqpoint{3.661468in}{3.537360in}}%
\pgfpathlineto{\pgfqpoint{3.661468in}{3.476583in}}%
\pgfpathlineto{\pgfqpoint{3.667529in}{3.476583in}}%
\pgfpathlineto{\pgfqpoint{3.667529in}{3.519839in}}%
\pgfpathlineto{\pgfqpoint{3.673590in}{3.519839in}}%
\pgfpathlineto{\pgfqpoint{3.673590in}{3.469465in}}%
\pgfpathlineto{\pgfqpoint{3.679651in}{3.469465in}}%
\pgfpathlineto{\pgfqpoint{3.679651in}{3.572402in}}%
\pgfpathlineto{\pgfqpoint{3.685712in}{3.572402in}}%
\pgfpathlineto{\pgfqpoint{3.685712in}{3.482606in}}%
\pgfpathlineto{\pgfqpoint{3.691773in}{3.482606in}}%
\pgfpathlineto{\pgfqpoint{3.691773in}{3.491367in}}%
\pgfpathlineto{\pgfqpoint{3.697834in}{3.491367in}}%
\pgfpathlineto{\pgfqpoint{3.697834in}{3.457419in}}%
\pgfpathlineto{\pgfqpoint{3.703895in}{3.457419in}}%
\pgfpathlineto{\pgfqpoint{3.703895in}{3.506698in}}%
\pgfpathlineto{\pgfqpoint{3.709956in}{3.506698in}}%
\pgfpathlineto{\pgfqpoint{3.709956in}{3.450301in}}%
\pgfpathlineto{\pgfqpoint{3.716017in}{3.450301in}}%
\pgfpathlineto{\pgfqpoint{3.716017in}{3.387335in}}%
\pgfpathlineto{\pgfqpoint{3.722078in}{3.387335in}}%
\pgfpathlineto{\pgfqpoint{3.722078in}{3.491367in}}%
\pgfpathlineto{\pgfqpoint{3.728139in}{3.491367in}}%
\pgfpathlineto{\pgfqpoint{3.728139in}{3.386787in}}%
\pgfpathlineto{\pgfqpoint{3.734200in}{3.386787in}}%
\pgfpathlineto{\pgfqpoint{3.734200in}{3.470560in}}%
\pgfpathlineto{\pgfqpoint{3.740261in}{3.470560in}}%
\pgfpathlineto{\pgfqpoint{3.740261in}{3.380217in}}%
\pgfpathlineto{\pgfqpoint{3.746322in}{3.380217in}}%
\pgfpathlineto{\pgfqpoint{3.746322in}{3.463990in}}%
\pgfpathlineto{\pgfqpoint{3.752383in}{3.463990in}}%
\pgfpathlineto{\pgfqpoint{3.752383in}{3.362696in}}%
\pgfpathlineto{\pgfqpoint{3.758444in}{3.362696in}}%
\pgfpathlineto{\pgfqpoint{3.758444in}{3.431138in}}%
\pgfpathlineto{\pgfqpoint{3.764505in}{3.431138in}}%
\pgfpathlineto{\pgfqpoint{3.764505in}{3.342984in}}%
\pgfpathlineto{\pgfqpoint{3.770566in}{3.342984in}}%
\pgfpathlineto{\pgfqpoint{3.770566in}{3.443731in}}%
\pgfpathlineto{\pgfqpoint{3.776627in}{3.443731in}}%
\pgfpathlineto{\pgfqpoint{3.776627in}{3.405951in}}%
\pgfpathlineto{\pgfqpoint{3.782688in}{3.405951in}}%
\pgfpathlineto{\pgfqpoint{3.782688in}{3.333676in}}%
\pgfpathlineto{\pgfqpoint{3.788749in}{3.333676in}}%
\pgfpathlineto{\pgfqpoint{3.788749in}{3.381859in}}%
\pgfpathlineto{\pgfqpoint{3.794810in}{3.381859in}}%
\pgfpathlineto{\pgfqpoint{3.794810in}{3.332581in}}%
\pgfpathlineto{\pgfqpoint{3.800870in}{3.332581in}}%
\pgfpathlineto{\pgfqpoint{3.800870in}{3.378574in}}%
\pgfpathlineto{\pgfqpoint{3.806931in}{3.378574in}}%
\pgfpathlineto{\pgfqpoint{3.806931in}{3.335866in}}%
\pgfpathlineto{\pgfqpoint{3.812992in}{3.335866in}}%
\pgfpathlineto{\pgfqpoint{3.812992in}{3.373646in}}%
\pgfpathlineto{\pgfqpoint{3.819053in}{3.373646in}}%
\pgfpathlineto{\pgfqpoint{3.819053in}{3.296991in}}%
\pgfpathlineto{\pgfqpoint{3.825114in}{3.296991in}}%
\pgfpathlineto{\pgfqpoint{3.825114in}{3.345722in}}%
\pgfpathlineto{\pgfqpoint{3.831175in}{3.345722in}}%
\pgfpathlineto{\pgfqpoint{3.831175in}{3.382407in}}%
\pgfpathlineto{\pgfqpoint{3.837236in}{3.382407in}}%
\pgfpathlineto{\pgfqpoint{3.837236in}{3.302467in}}%
\pgfpathlineto{\pgfqpoint{3.843297in}{3.302467in}}%
\pgfpathlineto{\pgfqpoint{3.843297in}{3.341889in}}%
\pgfpathlineto{\pgfqpoint{3.849358in}{3.341889in}}%
\pgfpathlineto{\pgfqpoint{3.849358in}{3.303014in}}%
\pgfpathlineto{\pgfqpoint{3.855419in}{3.303014in}}%
\pgfpathlineto{\pgfqpoint{3.855419in}{3.357768in}}%
\pgfpathlineto{\pgfqpoint{3.861480in}{3.357768in}}%
\pgfpathlineto{\pgfqpoint{3.861480in}{3.277280in}}%
\pgfpathlineto{\pgfqpoint{3.867541in}{3.277280in}}%
\pgfpathlineto{\pgfqpoint{3.867541in}{3.316155in}}%
\pgfpathlineto{\pgfqpoint{3.873602in}{3.316155in}}%
\pgfpathlineto{\pgfqpoint{3.873602in}{3.272352in}}%
\pgfpathlineto{\pgfqpoint{3.879663in}{3.272352in}}%
\pgfpathlineto{\pgfqpoint{3.879663in}{3.316703in}}%
\pgfpathlineto{\pgfqpoint{3.885724in}{3.316703in}}%
\pgfpathlineto{\pgfqpoint{3.885724in}{3.287683in}}%
\pgfpathlineto{\pgfqpoint{3.891785in}{3.287683in}}%
\pgfpathlineto{\pgfqpoint{3.891785in}{3.253736in}}%
\pgfpathlineto{\pgfqpoint{3.897846in}{3.253736in}}%
\pgfpathlineto{\pgfqpoint{3.897846in}{3.282755in}}%
\pgfpathlineto{\pgfqpoint{3.903907in}{3.282755in}}%
\pgfpathlineto{\pgfqpoint{3.903907in}{3.255926in}}%
\pgfpathlineto{\pgfqpoint{3.909968in}{3.255926in}}%
\pgfpathlineto{\pgfqpoint{3.909968in}{3.279470in}}%
\pgfpathlineto{\pgfqpoint{3.916029in}{3.279470in}}%
\pgfpathlineto{\pgfqpoint{3.916029in}{3.254284in}}%
\pgfpathlineto{\pgfqpoint{3.922090in}{3.254284in}}%
\pgfpathlineto{\pgfqpoint{3.922090in}{3.283850in}}%
\pgfpathlineto{\pgfqpoint{3.928151in}{3.283850in}}%
\pgfpathlineto{\pgfqpoint{3.928151in}{3.232382in}}%
\pgfpathlineto{\pgfqpoint{3.934212in}{3.232382in}}%
\pgfpathlineto{\pgfqpoint{3.934212in}{3.288231in}}%
\pgfpathlineto{\pgfqpoint{3.940273in}{3.288231in}}%
\pgfpathlineto{\pgfqpoint{3.940273in}{3.282208in}}%
\pgfpathlineto{\pgfqpoint{3.946334in}{3.282208in}}%
\pgfpathlineto{\pgfqpoint{3.946334in}{3.213218in}}%
\pgfpathlineto{\pgfqpoint{3.952395in}{3.213218in}}%
\pgfpathlineto{\pgfqpoint{3.952395in}{3.277280in}}%
\pgfpathlineto{\pgfqpoint{3.958456in}{3.277280in}}%
\pgfpathlineto{\pgfqpoint{3.958456in}{3.219241in}}%
\pgfpathlineto{\pgfqpoint{3.964517in}{3.219241in}}%
\pgfpathlineto{\pgfqpoint{3.964517in}{3.279470in}}%
\pgfpathlineto{\pgfqpoint{3.970578in}{3.279470in}}%
\pgfpathlineto{\pgfqpoint{3.970578in}{3.210481in}}%
\pgfpathlineto{\pgfqpoint{3.976639in}{3.210481in}}%
\pgfpathlineto{\pgfqpoint{3.976639in}{3.237857in}}%
\pgfpathlineto{\pgfqpoint{3.982700in}{3.237857in}}%
\pgfpathlineto{\pgfqpoint{3.982700in}{3.218146in}}%
\pgfpathlineto{\pgfqpoint{3.988761in}{3.218146in}}%
\pgfpathlineto{\pgfqpoint{3.988761in}{3.242785in}}%
\pgfpathlineto{\pgfqpoint{4.000883in}{3.242785in}}%
\pgfpathlineto{\pgfqpoint{4.000883in}{3.213218in}}%
\pgfpathlineto{\pgfqpoint{4.006944in}{3.213218in}}%
\pgfpathlineto{\pgfqpoint{4.006944in}{3.229644in}}%
\pgfpathlineto{\pgfqpoint{4.013005in}{3.229644in}}%
\pgfpathlineto{\pgfqpoint{4.013005in}{3.190222in}}%
\pgfpathlineto{\pgfqpoint{4.019066in}{3.190222in}}%
\pgfpathlineto{\pgfqpoint{4.019066in}{3.233477in}}%
\pgfpathlineto{\pgfqpoint{4.025127in}{3.233477in}}%
\pgfpathlineto{\pgfqpoint{4.025127in}{3.200625in}}%
\pgfpathlineto{\pgfqpoint{4.031188in}{3.200625in}}%
\pgfpathlineto{\pgfqpoint{4.031188in}{3.237857in}}%
\pgfpathlineto{\pgfqpoint{4.037249in}{3.237857in}}%
\pgfpathlineto{\pgfqpoint{4.037249in}{3.196792in}}%
\pgfpathlineto{\pgfqpoint{4.043310in}{3.196792in}}%
\pgfpathlineto{\pgfqpoint{4.043310in}{3.237310in}}%
\pgfpathlineto{\pgfqpoint{4.049371in}{3.237310in}}%
\pgfpathlineto{\pgfqpoint{4.049371in}{3.204458in}}%
\pgfpathlineto{\pgfqpoint{4.055432in}{3.204458in}}%
\pgfpathlineto{\pgfqpoint{4.055432in}{3.197340in}}%
\pgfpathlineto{\pgfqpoint{4.061493in}{3.197340in}}%
\pgfpathlineto{\pgfqpoint{4.061493in}{3.240048in}}%
\pgfpathlineto{\pgfqpoint{4.067554in}{3.240048in}}%
\pgfpathlineto{\pgfqpoint{4.067554in}{3.203910in}}%
\pgfpathlineto{\pgfqpoint{4.073615in}{3.203910in}}%
\pgfpathlineto{\pgfqpoint{4.073615in}{3.215408in}}%
\pgfpathlineto{\pgfqpoint{4.079676in}{3.215408in}}%
\pgfpathlineto{\pgfqpoint{4.079676in}{3.173796in}}%
\pgfpathlineto{\pgfqpoint{4.085737in}{3.173796in}}%
\pgfpathlineto{\pgfqpoint{4.085737in}{3.203363in}}%
\pgfpathlineto{\pgfqpoint{4.091797in}{3.203363in}}%
\pgfpathlineto{\pgfqpoint{4.091797in}{3.173248in}}%
\pgfpathlineto{\pgfqpoint{4.097858in}{3.173248in}}%
\pgfpathlineto{\pgfqpoint{4.097858in}{3.208838in}}%
\pgfpathlineto{\pgfqpoint{4.103919in}{3.208838in}}%
\pgfpathlineto{\pgfqpoint{4.103919in}{3.185841in}}%
\pgfpathlineto{\pgfqpoint{4.109980in}{3.185841in}}%
\pgfpathlineto{\pgfqpoint{4.109980in}{3.197340in}}%
\pgfpathlineto{\pgfqpoint{4.116041in}{3.197340in}}%
\pgfpathlineto{\pgfqpoint{4.116041in}{3.206648in}}%
\pgfpathlineto{\pgfqpoint{4.122102in}{3.206648in}}%
\pgfpathlineto{\pgfqpoint{4.122102in}{3.161202in}}%
\pgfpathlineto{\pgfqpoint{4.128163in}{3.161202in}}%
\pgfpathlineto{\pgfqpoint{4.128163in}{3.195697in}}%
\pgfpathlineto{\pgfqpoint{4.134224in}{3.195697in}}%
\pgfpathlineto{\pgfqpoint{4.134224in}{3.179819in}}%
\pgfpathlineto{\pgfqpoint{4.140285in}{3.179819in}}%
\pgfpathlineto{\pgfqpoint{4.140285in}{3.216504in}}%
\pgfpathlineto{\pgfqpoint{4.146346in}{3.216504in}}%
\pgfpathlineto{\pgfqpoint{4.146346in}{3.146966in}}%
\pgfpathlineto{\pgfqpoint{4.152407in}{3.146966in}}%
\pgfpathlineto{\pgfqpoint{4.152407in}{3.208838in}}%
\pgfpathlineto{\pgfqpoint{4.158468in}{3.208838in}}%
\pgfpathlineto{\pgfqpoint{4.158468in}{3.156275in}}%
\pgfpathlineto{\pgfqpoint{4.164529in}{3.156275in}}%
\pgfpathlineto{\pgfqpoint{4.164529in}{3.169963in}}%
\pgfpathlineto{\pgfqpoint{4.170590in}{3.169963in}}%
\pgfpathlineto{\pgfqpoint{4.170590in}{3.179271in}}%
\pgfpathlineto{\pgfqpoint{4.176651in}{3.179271in}}%
\pgfpathlineto{\pgfqpoint{4.176651in}{3.141491in}}%
\pgfpathlineto{\pgfqpoint{4.182712in}{3.141491in}}%
\pgfpathlineto{\pgfqpoint{4.182712in}{3.178724in}}%
\pgfpathlineto{\pgfqpoint{4.188773in}{3.178724in}}%
\pgfpathlineto{\pgfqpoint{4.188773in}{3.142586in}}%
\pgfpathlineto{\pgfqpoint{4.194834in}{3.142586in}}%
\pgfpathlineto{\pgfqpoint{4.194834in}{3.186937in}}%
\pgfpathlineto{\pgfqpoint{4.200895in}{3.186937in}}%
\pgfpathlineto{\pgfqpoint{4.200895in}{3.144776in}}%
\pgfpathlineto{\pgfqpoint{4.206956in}{3.144776in}}%
\pgfpathlineto{\pgfqpoint{4.206956in}{3.186937in}}%
\pgfpathlineto{\pgfqpoint{4.213017in}{3.186937in}}%
\pgfpathlineto{\pgfqpoint{4.213017in}{3.139301in}}%
\pgfpathlineto{\pgfqpoint{4.219078in}{3.139301in}}%
\pgfpathlineto{\pgfqpoint{4.219078in}{3.151347in}}%
\pgfpathlineto{\pgfqpoint{4.225139in}{3.151347in}}%
\pgfpathlineto{\pgfqpoint{4.225139in}{3.158465in}}%
\pgfpathlineto{\pgfqpoint{4.231200in}{3.158465in}}%
\pgfpathlineto{\pgfqpoint{4.231200in}{3.128350in}}%
\pgfpathlineto{\pgfqpoint{4.237261in}{3.128350in}}%
\pgfpathlineto{\pgfqpoint{4.237261in}{3.138206in}}%
\pgfpathlineto{\pgfqpoint{4.243322in}{3.138206in}}%
\pgfpathlineto{\pgfqpoint{4.243322in}{3.126160in}}%
\pgfpathlineto{\pgfqpoint{4.249383in}{3.126160in}}%
\pgfpathlineto{\pgfqpoint{4.249383in}{3.173796in}}%
\pgfpathlineto{\pgfqpoint{4.255444in}{3.173796in}}%
\pgfpathlineto{\pgfqpoint{4.255444in}{3.114662in}}%
\pgfpathlineto{\pgfqpoint{4.261505in}{3.114662in}}%
\pgfpathlineto{\pgfqpoint{4.261505in}{3.163393in}}%
\pgfpathlineto{\pgfqpoint{4.267566in}{3.163393in}}%
\pgfpathlineto{\pgfqpoint{4.267566in}{3.142586in}}%
\pgfpathlineto{\pgfqpoint{4.273627in}{3.142586in}}%
\pgfpathlineto{\pgfqpoint{4.273627in}{3.166678in}}%
\pgfpathlineto{\pgfqpoint{4.279688in}{3.166678in}}%
\pgfpathlineto{\pgfqpoint{4.279688in}{3.145871in}}%
\pgfpathlineto{\pgfqpoint{4.285749in}{3.145871in}}%
\pgfpathlineto{\pgfqpoint{4.285749in}{3.132730in}}%
\pgfpathlineto{\pgfqpoint{4.291810in}{3.132730in}}%
\pgfpathlineto{\pgfqpoint{4.291810in}{3.139848in}}%
\pgfpathlineto{\pgfqpoint{4.297871in}{3.139848in}}%
\pgfpathlineto{\pgfqpoint{4.297871in}{3.118495in}}%
\pgfpathlineto{\pgfqpoint{4.303932in}{3.118495in}}%
\pgfpathlineto{\pgfqpoint{4.303932in}{3.127255in}}%
\pgfpathlineto{\pgfqpoint{4.309993in}{3.127255in}}%
\pgfpathlineto{\pgfqpoint{4.309993in}{3.114114in}}%
\pgfpathlineto{\pgfqpoint{4.316054in}{3.114114in}}%
\pgfpathlineto{\pgfqpoint{4.316054in}{3.145324in}}%
\pgfpathlineto{\pgfqpoint{4.322115in}{3.145324in}}%
\pgfpathlineto{\pgfqpoint{4.322115in}{3.097688in}}%
\pgfpathlineto{\pgfqpoint{4.328176in}{3.097688in}}%
\pgfpathlineto{\pgfqpoint{4.328176in}{3.120137in}}%
\pgfpathlineto{\pgfqpoint{4.334237in}{3.120137in}}%
\pgfpathlineto{\pgfqpoint{4.334237in}{3.134921in}}%
\pgfpathlineto{\pgfqpoint{4.340298in}{3.134921in}}%
\pgfpathlineto{\pgfqpoint{4.340298in}{3.108639in}}%
\pgfpathlineto{\pgfqpoint{4.346359in}{3.108639in}}%
\pgfpathlineto{\pgfqpoint{4.346359in}{3.122327in}}%
\pgfpathlineto{\pgfqpoint{4.352420in}{3.122327in}}%
\pgfpathlineto{\pgfqpoint{4.352420in}{3.098783in}}%
\pgfpathlineto{\pgfqpoint{4.358481in}{3.098783in}}%
\pgfpathlineto{\pgfqpoint{4.358481in}{3.121780in}}%
\pgfpathlineto{\pgfqpoint{4.364542in}{3.121780in}}%
\pgfpathlineto{\pgfqpoint{4.364542in}{3.103164in}}%
\pgfpathlineto{\pgfqpoint{4.370603in}{3.103164in}}%
\pgfpathlineto{\pgfqpoint{4.370603in}{3.115209in}}%
\pgfpathlineto{\pgfqpoint{4.376664in}{3.115209in}}%
\pgfpathlineto{\pgfqpoint{4.376664in}{3.087833in}}%
\pgfpathlineto{\pgfqpoint{4.382724in}{3.087833in}}%
\pgfpathlineto{\pgfqpoint{4.382724in}{3.117399in}}%
\pgfpathlineto{\pgfqpoint{4.388785in}{3.117399in}}%
\pgfpathlineto{\pgfqpoint{4.388785in}{3.106996in}}%
\pgfpathlineto{\pgfqpoint{4.394846in}{3.106996in}}%
\pgfpathlineto{\pgfqpoint{4.394846in}{3.102068in}}%
\pgfpathlineto{\pgfqpoint{4.400907in}{3.102068in}}%
\pgfpathlineto{\pgfqpoint{4.400907in}{3.104806in}}%
\pgfpathlineto{\pgfqpoint{4.406968in}{3.104806in}}%
\pgfpathlineto{\pgfqpoint{4.406968in}{3.079619in}}%
\pgfpathlineto{\pgfqpoint{4.413029in}{3.079619in}}%
\pgfpathlineto{\pgfqpoint{4.413029in}{3.102616in}}%
\pgfpathlineto{\pgfqpoint{4.419090in}{3.102616in}}%
\pgfpathlineto{\pgfqpoint{4.419090in}{3.089475in}}%
\pgfpathlineto{\pgfqpoint{4.425151in}{3.089475in}}%
\pgfpathlineto{\pgfqpoint{4.425151in}{3.114662in}}%
\pgfpathlineto{\pgfqpoint{4.431212in}{3.114662in}}%
\pgfpathlineto{\pgfqpoint{4.431212in}{3.087833in}}%
\pgfpathlineto{\pgfqpoint{4.437273in}{3.087833in}}%
\pgfpathlineto{\pgfqpoint{4.437273in}{3.114114in}}%
\pgfpathlineto{\pgfqpoint{4.443334in}{3.114114in}}%
\pgfpathlineto{\pgfqpoint{4.443334in}{3.082905in}}%
\pgfpathlineto{\pgfqpoint{4.449395in}{3.082905in}}%
\pgfpathlineto{\pgfqpoint{4.449395in}{3.099878in}}%
\pgfpathlineto{\pgfqpoint{4.455456in}{3.099878in}}%
\pgfpathlineto{\pgfqpoint{4.455456in}{3.087285in}}%
\pgfpathlineto{\pgfqpoint{4.461517in}{3.087285in}}%
\pgfpathlineto{\pgfqpoint{4.461517in}{3.073597in}}%
\pgfpathlineto{\pgfqpoint{4.467578in}{3.073597in}}%
\pgfpathlineto{\pgfqpoint{4.467578in}{3.085642in}}%
\pgfpathlineto{\pgfqpoint{4.473639in}{3.085642in}}%
\pgfpathlineto{\pgfqpoint{4.473639in}{3.068669in}}%
\pgfpathlineto{\pgfqpoint{4.479700in}{3.068669in}}%
\pgfpathlineto{\pgfqpoint{4.479700in}{3.092213in}}%
\pgfpathlineto{\pgfqpoint{4.485761in}{3.092213in}}%
\pgfpathlineto{\pgfqpoint{4.485761in}{3.083452in}}%
\pgfpathlineto{\pgfqpoint{4.491822in}{3.083452in}}%
\pgfpathlineto{\pgfqpoint{4.491822in}{3.105901in}}%
\pgfpathlineto{\pgfqpoint{4.497883in}{3.105901in}}%
\pgfpathlineto{\pgfqpoint{4.497883in}{3.065931in}}%
\pgfpathlineto{\pgfqpoint{4.503944in}{3.065931in}}%
\pgfpathlineto{\pgfqpoint{4.503944in}{3.081810in}}%
\pgfpathlineto{\pgfqpoint{4.510005in}{3.081810in}}%
\pgfpathlineto{\pgfqpoint{4.510005in}{3.090570in}}%
\pgfpathlineto{\pgfqpoint{4.516066in}{3.090570in}}%
\pgfpathlineto{\pgfqpoint{4.516066in}{3.062646in}}%
\pgfpathlineto{\pgfqpoint{4.522127in}{3.062646in}}%
\pgfpathlineto{\pgfqpoint{4.522127in}{3.086190in}}%
\pgfpathlineto{\pgfqpoint{4.528188in}{3.086190in}}%
\pgfpathlineto{\pgfqpoint{4.528188in}{3.069216in}}%
\pgfpathlineto{\pgfqpoint{4.534249in}{3.069216in}}%
\pgfpathlineto{\pgfqpoint{4.534249in}{3.097141in}}%
\pgfpathlineto{\pgfqpoint{4.540310in}{3.097141in}}%
\pgfpathlineto{\pgfqpoint{4.540310in}{3.064288in}}%
\pgfpathlineto{\pgfqpoint{4.546371in}{3.064288in}}%
\pgfpathlineto{\pgfqpoint{4.546371in}{3.073049in}}%
\pgfpathlineto{\pgfqpoint{4.552432in}{3.073049in}}%
\pgfpathlineto{\pgfqpoint{4.552432in}{3.070311in}}%
\pgfpathlineto{\pgfqpoint{4.558493in}{3.070311in}}%
\pgfpathlineto{\pgfqpoint{4.558493in}{3.084000in}}%
\pgfpathlineto{\pgfqpoint{4.564554in}{3.084000in}}%
\pgfpathlineto{\pgfqpoint{4.564554in}{3.079072in}}%
\pgfpathlineto{\pgfqpoint{4.570615in}{3.079072in}}%
\pgfpathlineto{\pgfqpoint{4.570615in}{3.068121in}}%
\pgfpathlineto{\pgfqpoint{4.582737in}{3.068121in}}%
\pgfpathlineto{\pgfqpoint{4.582737in}{3.074144in}}%
\pgfpathlineto{\pgfqpoint{4.594859in}{3.074144in}}%
\pgfpathlineto{\pgfqpoint{4.594859in}{3.069216in}}%
\pgfpathlineto{\pgfqpoint{4.600920in}{3.069216in}}%
\pgfpathlineto{\pgfqpoint{4.600920in}{3.081810in}}%
\pgfpathlineto{\pgfqpoint{4.606981in}{3.081810in}}%
\pgfpathlineto{\pgfqpoint{4.606981in}{3.072502in}}%
\pgfpathlineto{\pgfqpoint{4.613042in}{3.072502in}}%
\pgfpathlineto{\pgfqpoint{4.613042in}{3.077977in}}%
\pgfpathlineto{\pgfqpoint{4.625164in}{3.079072in}}%
\pgfpathlineto{\pgfqpoint{4.625164in}{3.069764in}}%
\pgfpathlineto{\pgfqpoint{4.631225in}{3.069764in}}%
\pgfpathlineto{\pgfqpoint{4.631225in}{3.075787in}}%
\pgfpathlineto{\pgfqpoint{4.637286in}{3.075787in}}%
\pgfpathlineto{\pgfqpoint{4.637286in}{3.048410in}}%
\pgfpathlineto{\pgfqpoint{4.643347in}{3.048410in}}%
\pgfpathlineto{\pgfqpoint{4.643347in}{3.073597in}}%
\pgfpathlineto{\pgfqpoint{4.649408in}{3.073597in}}%
\pgfpathlineto{\pgfqpoint{4.649408in}{3.068121in}}%
\pgfpathlineto{\pgfqpoint{4.655469in}{3.068121in}}%
\pgfpathlineto{\pgfqpoint{4.655469in}{3.061003in}}%
\pgfpathlineto{\pgfqpoint{4.661530in}{3.061003in}}%
\pgfpathlineto{\pgfqpoint{4.661530in}{3.057171in}}%
\pgfpathlineto{\pgfqpoint{4.667591in}{3.057171in}}%
\pgfpathlineto{\pgfqpoint{4.667591in}{3.064836in}}%
\pgfpathlineto{\pgfqpoint{4.679712in}{3.065931in}}%
\pgfpathlineto{\pgfqpoint{4.679712in}{3.048957in}}%
\pgfpathlineto{\pgfqpoint{4.685773in}{3.048957in}}%
\pgfpathlineto{\pgfqpoint{4.685773in}{3.059908in}}%
\pgfpathlineto{\pgfqpoint{4.691834in}{3.059908in}}%
\pgfpathlineto{\pgfqpoint{4.691834in}{3.058266in}}%
\pgfpathlineto{\pgfqpoint{4.697895in}{3.058266in}}%
\pgfpathlineto{\pgfqpoint{4.697895in}{3.056623in}}%
\pgfpathlineto{\pgfqpoint{4.703956in}{3.056623in}}%
\pgfpathlineto{\pgfqpoint{4.703956in}{3.047315in}}%
\pgfpathlineto{\pgfqpoint{4.710017in}{3.047315in}}%
\pgfpathlineto{\pgfqpoint{4.710017in}{3.061551in}}%
\pgfpathlineto{\pgfqpoint{4.716078in}{3.061551in}}%
\pgfpathlineto{\pgfqpoint{4.716078in}{3.045125in}}%
\pgfpathlineto{\pgfqpoint{4.722139in}{3.045125in}}%
\pgfpathlineto{\pgfqpoint{4.722139in}{3.054433in}}%
\pgfpathlineto{\pgfqpoint{4.728200in}{3.054433in}}%
\pgfpathlineto{\pgfqpoint{4.728200in}{3.058813in}}%
\pgfpathlineto{\pgfqpoint{4.734261in}{3.058813in}}%
\pgfpathlineto{\pgfqpoint{4.734261in}{3.039102in}}%
\pgfpathlineto{\pgfqpoint{4.740322in}{3.039102in}}%
\pgfpathlineto{\pgfqpoint{4.740322in}{3.053885in}}%
\pgfpathlineto{\pgfqpoint{4.746383in}{3.053885in}}%
\pgfpathlineto{\pgfqpoint{4.746383in}{3.048957in}}%
\pgfpathlineto{\pgfqpoint{4.752444in}{3.048957in}}%
\pgfpathlineto{\pgfqpoint{4.752444in}{3.067574in}}%
\pgfpathlineto{\pgfqpoint{4.758505in}{3.067574in}}%
\pgfpathlineto{\pgfqpoint{4.758505in}{3.042387in}}%
\pgfpathlineto{\pgfqpoint{4.764566in}{3.042387in}}%
\pgfpathlineto{\pgfqpoint{4.764566in}{3.051695in}}%
\pgfpathlineto{\pgfqpoint{4.770627in}{3.051695in}}%
\pgfpathlineto{\pgfqpoint{4.770627in}{3.046220in}}%
\pgfpathlineto{\pgfqpoint{4.776688in}{3.046220in}}%
\pgfpathlineto{\pgfqpoint{4.776688in}{3.052243in}}%
\pgfpathlineto{\pgfqpoint{4.782749in}{3.052243in}}%
\pgfpathlineto{\pgfqpoint{4.782749in}{3.042935in}}%
\pgfpathlineto{\pgfqpoint{4.788810in}{3.042935in}}%
\pgfpathlineto{\pgfqpoint{4.788810in}{3.045125in}}%
\pgfpathlineto{\pgfqpoint{4.794871in}{3.045125in}}%
\pgfpathlineto{\pgfqpoint{4.794871in}{3.051148in}}%
\pgfpathlineto{\pgfqpoint{4.800932in}{3.051148in}}%
\pgfpathlineto{\pgfqpoint{4.800932in}{3.039102in}}%
\pgfpathlineto{\pgfqpoint{4.813054in}{3.040197in}}%
\pgfpathlineto{\pgfqpoint{4.813054in}{3.034174in}}%
\pgfpathlineto{\pgfqpoint{4.819115in}{3.034174in}}%
\pgfpathlineto{\pgfqpoint{4.819115in}{3.045672in}}%
\pgfpathlineto{\pgfqpoint{4.825176in}{3.045672in}}%
\pgfpathlineto{\pgfqpoint{4.825176in}{3.033079in}}%
\pgfpathlineto{\pgfqpoint{4.831237in}{3.033079in}}%
\pgfpathlineto{\pgfqpoint{4.831237in}{3.045125in}}%
\pgfpathlineto{\pgfqpoint{4.837298in}{3.045125in}}%
\pgfpathlineto{\pgfqpoint{4.837298in}{3.040744in}}%
\pgfpathlineto{\pgfqpoint{4.843359in}{3.040744in}}%
\pgfpathlineto{\pgfqpoint{4.843359in}{3.047315in}}%
\pgfpathlineto{\pgfqpoint{4.849420in}{3.047315in}}%
\pgfpathlineto{\pgfqpoint{4.849420in}{3.044577in}}%
\pgfpathlineto{\pgfqpoint{4.855481in}{3.044577in}}%
\pgfpathlineto{\pgfqpoint{4.855481in}{3.034174in}}%
\pgfpathlineto{\pgfqpoint{4.861542in}{3.034174in}}%
\pgfpathlineto{\pgfqpoint{4.861542in}{3.045125in}}%
\pgfpathlineto{\pgfqpoint{4.867603in}{3.045125in}}%
\pgfpathlineto{\pgfqpoint{4.867603in}{3.025961in}}%
\pgfpathlineto{\pgfqpoint{4.873664in}{3.025961in}}%
\pgfpathlineto{\pgfqpoint{4.873664in}{3.040744in}}%
\pgfpathlineto{\pgfqpoint{4.879725in}{3.040744in}}%
\pgfpathlineto{\pgfqpoint{4.879725in}{3.025413in}}%
\pgfpathlineto{\pgfqpoint{4.885786in}{3.025413in}}%
\pgfpathlineto{\pgfqpoint{4.885786in}{3.051695in}}%
\pgfpathlineto{\pgfqpoint{4.891847in}{3.051695in}}%
\pgfpathlineto{\pgfqpoint{4.891847in}{3.028699in}}%
\pgfpathlineto{\pgfqpoint{4.897908in}{3.028699in}}%
\pgfpathlineto{\pgfqpoint{4.897908in}{3.041840in}}%
\pgfpathlineto{\pgfqpoint{4.903969in}{3.041840in}}%
\pgfpathlineto{\pgfqpoint{4.903969in}{3.054980in}}%
\pgfpathlineto{\pgfqpoint{4.910030in}{3.054980in}}%
\pgfpathlineto{\pgfqpoint{4.910030in}{3.028699in}}%
\pgfpathlineto{\pgfqpoint{4.916091in}{3.028699in}}%
\pgfpathlineto{\pgfqpoint{4.916091in}{3.039649in}}%
\pgfpathlineto{\pgfqpoint{4.922152in}{3.039649in}}%
\pgfpathlineto{\pgfqpoint{4.922152in}{3.030341in}}%
\pgfpathlineto{\pgfqpoint{4.928213in}{3.030341in}}%
\pgfpathlineto{\pgfqpoint{4.928213in}{3.052790in}}%
\pgfpathlineto{\pgfqpoint{4.934274in}{3.052790in}}%
\pgfpathlineto{\pgfqpoint{4.934274in}{3.034722in}}%
\pgfpathlineto{\pgfqpoint{4.940335in}{3.034722in}}%
\pgfpathlineto{\pgfqpoint{4.940335in}{3.038554in}}%
\pgfpathlineto{\pgfqpoint{4.946396in}{3.038554in}}%
\pgfpathlineto{\pgfqpoint{4.946396in}{3.034174in}}%
\pgfpathlineto{\pgfqpoint{4.952457in}{3.034174in}}%
\pgfpathlineto{\pgfqpoint{4.952457in}{3.036364in}}%
\pgfpathlineto{\pgfqpoint{4.958518in}{3.036364in}}%
\pgfpathlineto{\pgfqpoint{4.958518in}{3.040197in}}%
\pgfpathlineto{\pgfqpoint{4.964578in}{3.040197in}}%
\pgfpathlineto{\pgfqpoint{4.964578in}{3.028699in}}%
\pgfpathlineto{\pgfqpoint{4.970639in}{3.028699in}}%
\pgfpathlineto{\pgfqpoint{4.970639in}{3.040197in}}%
\pgfpathlineto{\pgfqpoint{4.976700in}{3.040197in}}%
\pgfpathlineto{\pgfqpoint{4.976700in}{3.025413in}}%
\pgfpathlineto{\pgfqpoint{4.982761in}{3.025413in}}%
\pgfpathlineto{\pgfqpoint{4.982761in}{3.033626in}}%
\pgfpathlineto{\pgfqpoint{4.988822in}{3.033626in}}%
\pgfpathlineto{\pgfqpoint{4.988822in}{3.019938in}}%
\pgfpathlineto{\pgfqpoint{4.994883in}{3.019938in}}%
\pgfpathlineto{\pgfqpoint{4.994883in}{3.032531in}}%
\pgfpathlineto{\pgfqpoint{5.000944in}{3.032531in}}%
\pgfpathlineto{\pgfqpoint{5.000944in}{3.017748in}}%
\pgfpathlineto{\pgfqpoint{5.007005in}{3.017748in}}%
\pgfpathlineto{\pgfqpoint{5.007005in}{3.028151in}}%
\pgfpathlineto{\pgfqpoint{5.013066in}{3.028151in}}%
\pgfpathlineto{\pgfqpoint{5.013066in}{3.034174in}}%
\pgfpathlineto{\pgfqpoint{5.019127in}{3.034174in}}%
\pgfpathlineto{\pgfqpoint{5.019127in}{3.028151in}}%
\pgfpathlineto{\pgfqpoint{5.025188in}{3.028151in}}%
\pgfpathlineto{\pgfqpoint{5.025188in}{3.037459in}}%
\pgfpathlineto{\pgfqpoint{5.031249in}{3.037459in}}%
\pgfpathlineto{\pgfqpoint{5.031249in}{3.025413in}}%
\pgfpathlineto{\pgfqpoint{5.043371in}{3.024318in}}%
\pgfpathlineto{\pgfqpoint{5.043371in}{3.031436in}}%
\pgfpathlineto{\pgfqpoint{5.049432in}{3.031436in}}%
\pgfpathlineto{\pgfqpoint{5.049432in}{3.034174in}}%
\pgfpathlineto{\pgfqpoint{5.055493in}{3.034174in}}%
\pgfpathlineto{\pgfqpoint{5.055493in}{3.007345in}}%
\pgfpathlineto{\pgfqpoint{5.061554in}{3.007345in}}%
\pgfpathlineto{\pgfqpoint{5.061554in}{3.025413in}}%
\pgfpathlineto{\pgfqpoint{5.067615in}{3.025413in}}%
\pgfpathlineto{\pgfqpoint{5.067615in}{3.028151in}}%
\pgfpathlineto{\pgfqpoint{5.073676in}{3.028151in}}%
\pgfpathlineto{\pgfqpoint{5.073676in}{3.022128in}}%
\pgfpathlineto{\pgfqpoint{5.079737in}{3.022128in}}%
\pgfpathlineto{\pgfqpoint{5.079737in}{3.033626in}}%
\pgfpathlineto{\pgfqpoint{5.085798in}{3.033626in}}%
\pgfpathlineto{\pgfqpoint{5.085798in}{3.022128in}}%
\pgfpathlineto{\pgfqpoint{5.091859in}{3.022128in}}%
\pgfpathlineto{\pgfqpoint{5.091859in}{3.033079in}}%
\pgfpathlineto{\pgfqpoint{5.097920in}{3.033079in}}%
\pgfpathlineto{\pgfqpoint{5.097920in}{3.015558in}}%
\pgfpathlineto{\pgfqpoint{5.103981in}{3.015558in}}%
\pgfpathlineto{\pgfqpoint{5.103981in}{3.024866in}}%
\pgfpathlineto{\pgfqpoint{5.110042in}{3.024866in}}%
\pgfpathlineto{\pgfqpoint{5.110042in}{3.022676in}}%
\pgfpathlineto{\pgfqpoint{5.122164in}{3.023771in}}%
\pgfpathlineto{\pgfqpoint{5.122164in}{3.016653in}}%
\pgfpathlineto{\pgfqpoint{5.128225in}{3.016653in}}%
\pgfpathlineto{\pgfqpoint{5.128225in}{3.022676in}}%
\pgfpathlineto{\pgfqpoint{5.134286in}{3.022676in}}%
\pgfpathlineto{\pgfqpoint{5.134286in}{3.018843in}}%
\pgfpathlineto{\pgfqpoint{5.140347in}{3.018843in}}%
\pgfpathlineto{\pgfqpoint{5.140347in}{3.016105in}}%
\pgfpathlineto{\pgfqpoint{5.146408in}{3.016105in}}%
\pgfpathlineto{\pgfqpoint{5.146408in}{3.028699in}}%
\pgfpathlineto{\pgfqpoint{5.152469in}{3.028699in}}%
\pgfpathlineto{\pgfqpoint{5.152469in}{3.019391in}}%
\pgfpathlineto{\pgfqpoint{5.164591in}{3.018295in}}%
\pgfpathlineto{\pgfqpoint{5.164591in}{3.011177in}}%
\pgfpathlineto{\pgfqpoint{5.170652in}{3.011177in}}%
\pgfpathlineto{\pgfqpoint{5.170652in}{3.012820in}}%
\pgfpathlineto{\pgfqpoint{5.176713in}{3.012820in}}%
\pgfpathlineto{\pgfqpoint{5.176713in}{3.007892in}}%
\pgfpathlineto{\pgfqpoint{5.182774in}{3.007892in}}%
\pgfpathlineto{\pgfqpoint{5.182774in}{3.025413in}}%
\pgfpathlineto{\pgfqpoint{5.188835in}{3.025413in}}%
\pgfpathlineto{\pgfqpoint{5.188835in}{3.028151in}}%
\pgfpathlineto{\pgfqpoint{5.194896in}{3.028151in}}%
\pgfpathlineto{\pgfqpoint{5.194896in}{3.021581in}}%
\pgfpathlineto{\pgfqpoint{5.200957in}{3.021581in}}%
\pgfpathlineto{\pgfqpoint{5.200957in}{3.024866in}}%
\pgfpathlineto{\pgfqpoint{5.207018in}{3.024866in}}%
\pgfpathlineto{\pgfqpoint{5.207018in}{3.026508in}}%
\pgfpathlineto{\pgfqpoint{5.213079in}{3.026508in}}%
\pgfpathlineto{\pgfqpoint{5.213079in}{3.019391in}}%
\pgfpathlineto{\pgfqpoint{5.219140in}{3.019391in}}%
\pgfpathlineto{\pgfqpoint{5.219140in}{3.008987in}}%
\pgfpathlineto{\pgfqpoint{5.225201in}{3.008987in}}%
\pgfpathlineto{\pgfqpoint{5.225201in}{3.011725in}}%
\pgfpathlineto{\pgfqpoint{5.231262in}{3.011725in}}%
\pgfpathlineto{\pgfqpoint{5.231262in}{3.013368in}}%
\pgfpathlineto{\pgfqpoint{5.237323in}{3.013368in}}%
\pgfpathlineto{\pgfqpoint{5.237323in}{3.022128in}}%
\pgfpathlineto{\pgfqpoint{5.243384in}{3.022128in}}%
\pgfpathlineto{\pgfqpoint{5.243384in}{3.006797in}}%
\pgfpathlineto{\pgfqpoint{5.249445in}{3.006797in}}%
\pgfpathlineto{\pgfqpoint{5.249445in}{3.012273in}}%
\pgfpathlineto{\pgfqpoint{5.255505in}{3.012273in}}%
\pgfpathlineto{\pgfqpoint{5.255505in}{3.024866in}}%
\pgfpathlineto{\pgfqpoint{5.261566in}{3.024866in}}%
\pgfpathlineto{\pgfqpoint{5.261566in}{3.020486in}}%
\pgfpathlineto{\pgfqpoint{5.267627in}{3.020486in}}%
\pgfpathlineto{\pgfqpoint{5.267627in}{3.013915in}}%
\pgfpathlineto{\pgfqpoint{5.279749in}{3.015010in}}%
\pgfpathlineto{\pgfqpoint{5.279749in}{3.018843in}}%
\pgfpathlineto{\pgfqpoint{5.285810in}{3.018843in}}%
\pgfpathlineto{\pgfqpoint{5.285810in}{3.009535in}}%
\pgfpathlineto{\pgfqpoint{5.291871in}{3.009535in}}%
\pgfpathlineto{\pgfqpoint{5.291871in}{3.019938in}}%
\pgfpathlineto{\pgfqpoint{5.297932in}{3.019938in}}%
\pgfpathlineto{\pgfqpoint{5.297932in}{3.027056in}}%
\pgfpathlineto{\pgfqpoint{5.303993in}{3.027056in}}%
\pgfpathlineto{\pgfqpoint{5.303993in}{3.011177in}}%
\pgfpathlineto{\pgfqpoint{5.310054in}{3.011177in}}%
\pgfpathlineto{\pgfqpoint{5.310054in}{3.012820in}}%
\pgfpathlineto{\pgfqpoint{5.316115in}{3.012820in}}%
\pgfpathlineto{\pgfqpoint{5.316115in}{3.018843in}}%
\pgfpathlineto{\pgfqpoint{5.322176in}{3.018843in}}%
\pgfpathlineto{\pgfqpoint{5.322176in}{3.016653in}}%
\pgfpathlineto{\pgfqpoint{5.328237in}{3.016653in}}%
\pgfpathlineto{\pgfqpoint{5.328237in}{3.013915in}}%
\pgfpathlineto{\pgfqpoint{5.334298in}{3.013915in}}%
\pgfpathlineto{\pgfqpoint{5.334298in}{3.019938in}}%
\pgfpathlineto{\pgfqpoint{5.340359in}{3.019938in}}%
\pgfpathlineto{\pgfqpoint{5.340359in}{3.014463in}}%
\pgfpathlineto{\pgfqpoint{5.346420in}{3.014463in}}%
\pgfpathlineto{\pgfqpoint{5.346420in}{3.012820in}}%
\pgfpathlineto{\pgfqpoint{5.364603in}{3.011725in}}%
\pgfpathlineto{\pgfqpoint{5.364603in}{3.019938in}}%
\pgfpathlineto{\pgfqpoint{5.370664in}{3.019938in}}%
\pgfpathlineto{\pgfqpoint{5.370664in}{3.009535in}}%
\pgfpathlineto{\pgfqpoint{5.382786in}{3.009535in}}%
\pgfpathlineto{\pgfqpoint{5.382786in}{3.014463in}}%
\pgfpathlineto{\pgfqpoint{5.394908in}{3.015558in}}%
\pgfpathlineto{\pgfqpoint{5.394908in}{3.011725in}}%
\pgfpathlineto{\pgfqpoint{5.400969in}{3.011725in}}%
\pgfpathlineto{\pgfqpoint{5.400969in}{3.013915in}}%
\pgfpathlineto{\pgfqpoint{5.407030in}{3.013915in}}%
\pgfpathlineto{\pgfqpoint{5.407030in}{3.016105in}}%
\pgfpathlineto{\pgfqpoint{5.413091in}{3.016105in}}%
\pgfpathlineto{\pgfqpoint{5.413091in}{3.013915in}}%
\pgfpathlineto{\pgfqpoint{5.419152in}{3.013915in}}%
\pgfpathlineto{\pgfqpoint{5.419152in}{3.010630in}}%
\pgfpathlineto{\pgfqpoint{5.425213in}{3.010630in}}%
\pgfpathlineto{\pgfqpoint{5.425213in}{3.006797in}}%
\pgfpathlineto{\pgfqpoint{5.431274in}{3.006797in}}%
\pgfpathlineto{\pgfqpoint{5.431274in}{3.022676in}}%
\pgfpathlineto{\pgfqpoint{5.437335in}{3.022676in}}%
\pgfpathlineto{\pgfqpoint{5.437335in}{3.011725in}}%
\pgfpathlineto{\pgfqpoint{5.455518in}{3.012273in}}%
\pgfpathlineto{\pgfqpoint{5.455518in}{3.010082in}}%
\pgfpathlineto{\pgfqpoint{5.473701in}{3.008987in}}%
\pgfpathlineto{\pgfqpoint{5.473701in}{3.006250in}}%
\pgfpathlineto{\pgfqpoint{5.485823in}{3.006797in}}%
\pgfpathlineto{\pgfqpoint{5.485823in}{3.008440in}}%
\pgfpathlineto{\pgfqpoint{5.497945in}{3.007892in}}%
\pgfpathlineto{\pgfqpoint{5.497945in}{3.004607in}}%
\pgfpathlineto{\pgfqpoint{5.504006in}{3.004607in}}%
\pgfpathlineto{\pgfqpoint{5.504006in}{3.006797in}}%
\pgfpathlineto{\pgfqpoint{5.510067in}{3.006797in}}%
\pgfpathlineto{\pgfqpoint{5.510067in}{3.016653in}}%
\pgfpathlineto{\pgfqpoint{5.516128in}{3.016653in}}%
\pgfpathlineto{\pgfqpoint{5.516128in}{3.009535in}}%
\pgfpathlineto{\pgfqpoint{5.522189in}{3.009535in}}%
\pgfpathlineto{\pgfqpoint{5.522189in}{3.005702in}}%
\pgfpathlineto{\pgfqpoint{5.528250in}{3.005702in}}%
\pgfpathlineto{\pgfqpoint{5.528250in}{3.010082in}}%
\pgfpathlineto{\pgfqpoint{5.534311in}{3.010082in}}%
\pgfpathlineto{\pgfqpoint{5.534311in}{2.997489in}}%
\pgfpathlineto{\pgfqpoint{5.540372in}{2.997489in}}%
\pgfpathlineto{\pgfqpoint{5.540372in}{3.014463in}}%
\pgfpathlineto{\pgfqpoint{5.546432in}{3.014463in}}%
\pgfpathlineto{\pgfqpoint{5.546432in}{3.004607in}}%
\pgfpathlineto{\pgfqpoint{5.558554in}{3.005702in}}%
\pgfpathlineto{\pgfqpoint{5.558554in}{3.008987in}}%
\pgfpathlineto{\pgfqpoint{5.564615in}{3.008987in}}%
\pgfpathlineto{\pgfqpoint{5.564615in}{3.011725in}}%
\pgfpathlineto{\pgfqpoint{5.570676in}{3.011725in}}%
\pgfpathlineto{\pgfqpoint{5.570676in}{3.006797in}}%
\pgfpathlineto{\pgfqpoint{5.582798in}{3.006250in}}%
\pgfpathlineto{\pgfqpoint{5.582798in}{3.008440in}}%
\pgfpathlineto{\pgfqpoint{5.588859in}{3.008440in}}%
\pgfpathlineto{\pgfqpoint{5.588859in}{3.012820in}}%
\pgfpathlineto{\pgfqpoint{5.600981in}{3.012820in}}%
\pgfpathlineto{\pgfqpoint{5.600981in}{2.997489in}}%
\pgfpathlineto{\pgfqpoint{5.607042in}{2.997489in}}%
\pgfpathlineto{\pgfqpoint{5.607042in}{3.016653in}}%
\pgfpathlineto{\pgfqpoint{5.613103in}{3.016653in}}%
\pgfpathlineto{\pgfqpoint{5.613103in}{2.998584in}}%
\pgfpathlineto{\pgfqpoint{5.619164in}{2.998584in}}%
\pgfpathlineto{\pgfqpoint{5.619164in}{3.007892in}}%
\pgfpathlineto{\pgfqpoint{5.625225in}{3.007892in}}%
\pgfpathlineto{\pgfqpoint{5.625225in}{3.005702in}}%
\pgfpathlineto{\pgfqpoint{5.637347in}{3.005702in}}%
\pgfpathlineto{\pgfqpoint{5.637347in}{3.003512in}}%
\pgfpathlineto{\pgfqpoint{5.643408in}{3.003512in}}%
\pgfpathlineto{\pgfqpoint{5.643408in}{3.001869in}}%
\pgfpathlineto{\pgfqpoint{5.649469in}{3.001869in}}%
\pgfpathlineto{\pgfqpoint{5.649469in}{3.012273in}}%
\pgfpathlineto{\pgfqpoint{5.655530in}{3.012273in}}%
\pgfpathlineto{\pgfqpoint{5.655530in}{3.008440in}}%
\pgfpathlineto{\pgfqpoint{5.661591in}{3.008440in}}%
\pgfpathlineto{\pgfqpoint{5.661591in}{3.006797in}}%
\pgfpathlineto{\pgfqpoint{5.667652in}{3.006797in}}%
\pgfpathlineto{\pgfqpoint{5.667652in}{3.008987in}}%
\pgfpathlineto{\pgfqpoint{5.673713in}{3.008987in}}%
\pgfpathlineto{\pgfqpoint{5.673713in}{3.005702in}}%
\pgfpathlineto{\pgfqpoint{5.679774in}{3.005702in}}%
\pgfpathlineto{\pgfqpoint{5.679774in}{3.004060in}}%
\pgfpathlineto{\pgfqpoint{5.685835in}{3.004060in}}%
\pgfpathlineto{\pgfqpoint{5.685835in}{3.010630in}}%
\pgfpathlineto{\pgfqpoint{5.691896in}{3.010630in}}%
\pgfpathlineto{\pgfqpoint{5.691896in}{3.003512in}}%
\pgfpathlineto{\pgfqpoint{5.697957in}{3.003512in}}%
\pgfpathlineto{\pgfqpoint{5.697957in}{3.000227in}}%
\pgfpathlineto{\pgfqpoint{5.704018in}{3.000227in}}%
\pgfpathlineto{\pgfqpoint{5.704018in}{3.005155in}}%
\pgfpathlineto{\pgfqpoint{5.710079in}{3.005155in}}%
\pgfpathlineto{\pgfqpoint{5.710079in}{3.007345in}}%
\pgfpathlineto{\pgfqpoint{5.722201in}{3.006250in}}%
\pgfpathlineto{\pgfqpoint{5.722201in}{3.012273in}}%
\pgfpathlineto{\pgfqpoint{5.728262in}{3.012273in}}%
\pgfpathlineto{\pgfqpoint{5.728262in}{3.004060in}}%
\pgfpathlineto{\pgfqpoint{5.740384in}{3.004060in}}%
\pgfpathlineto{\pgfqpoint{5.740384in}{3.010082in}}%
\pgfpathlineto{\pgfqpoint{5.746445in}{3.010082in}}%
\pgfpathlineto{\pgfqpoint{5.746445in}{3.001869in}}%
\pgfpathlineto{\pgfqpoint{5.752506in}{3.001869in}}%
\pgfpathlineto{\pgfqpoint{5.752506in}{2.999132in}}%
\pgfpathlineto{\pgfqpoint{5.758567in}{2.999132in}}%
\pgfpathlineto{\pgfqpoint{5.758567in}{3.010082in}}%
\pgfpathlineto{\pgfqpoint{5.764628in}{3.010082in}}%
\pgfpathlineto{\pgfqpoint{5.764628in}{3.002417in}}%
\pgfpathlineto{\pgfqpoint{5.770689in}{3.002417in}}%
\pgfpathlineto{\pgfqpoint{5.770689in}{3.008440in}}%
\pgfpathlineto{\pgfqpoint{5.776750in}{3.008440in}}%
\pgfpathlineto{\pgfqpoint{5.776750in}{3.003512in}}%
\pgfpathlineto{\pgfqpoint{5.782811in}{3.003512in}}%
\pgfpathlineto{\pgfqpoint{5.782811in}{3.006250in}}%
\pgfpathlineto{\pgfqpoint{5.788872in}{3.006250in}}%
\pgfpathlineto{\pgfqpoint{5.788872in}{3.000774in}}%
\pgfpathlineto{\pgfqpoint{5.800994in}{3.000774in}}%
\pgfpathlineto{\pgfqpoint{5.800994in}{3.003512in}}%
\pgfpathlineto{\pgfqpoint{5.807055in}{3.003512in}}%
\pgfpathlineto{\pgfqpoint{5.807055in}{3.003512in}}%
\pgfusepath{stroke}%
\end{pgfscope}%
\begin{pgfscope}%
\pgfpathrectangle{\pgfqpoint{0.781944in}{2.977778in}}{\pgfqpoint{5.019444in}{1.650000in}}%
\pgfusepath{clip}%
\pgfsetrectcap%
\pgfsetroundjoin%
\pgfsetlinewidth{1.505625pt}%
\definecolor{currentstroke}{rgb}{0.172549,0.627451,0.172549}%
\pgfsetstrokecolor{currentstroke}%
\pgfsetdash{}{0pt}%
\pgfpathmoveto{\pgfqpoint{0.776442in}{3.001869in}}%
\pgfpathlineto{\pgfqpoint{0.776442in}{2.993109in}}%
\pgfpathlineto{\pgfqpoint{0.782503in}{2.993109in}}%
\pgfpathlineto{\pgfqpoint{0.782503in}{2.995299in}}%
\pgfpathlineto{\pgfqpoint{0.806747in}{2.995846in}}%
\pgfpathlineto{\pgfqpoint{0.806747in}{2.998037in}}%
\pgfpathlineto{\pgfqpoint{0.830991in}{2.998037in}}%
\pgfpathlineto{\pgfqpoint{0.830991in}{2.999679in}}%
\pgfpathlineto{\pgfqpoint{0.837052in}{2.999679in}}%
\pgfpathlineto{\pgfqpoint{0.837052in}{2.994751in}}%
\pgfpathlineto{\pgfqpoint{0.849174in}{2.995846in}}%
\pgfpathlineto{\pgfqpoint{0.849174in}{3.000774in}}%
\pgfpathlineto{\pgfqpoint{0.855235in}{3.000774in}}%
\pgfpathlineto{\pgfqpoint{0.855235in}{3.006250in}}%
\pgfpathlineto{\pgfqpoint{0.861296in}{3.006250in}}%
\pgfpathlineto{\pgfqpoint{0.861296in}{2.994751in}}%
\pgfpathlineto{\pgfqpoint{0.867357in}{2.994751in}}%
\pgfpathlineto{\pgfqpoint{0.867357in}{2.998037in}}%
\pgfpathlineto{\pgfqpoint{0.873418in}{2.998037in}}%
\pgfpathlineto{\pgfqpoint{0.873418in}{2.995846in}}%
\pgfpathlineto{\pgfqpoint{0.879479in}{2.995846in}}%
\pgfpathlineto{\pgfqpoint{0.879479in}{2.999679in}}%
\pgfpathlineto{\pgfqpoint{0.885540in}{2.999679in}}%
\pgfpathlineto{\pgfqpoint{0.885540in}{2.996942in}}%
\pgfpathlineto{\pgfqpoint{0.897661in}{2.995846in}}%
\pgfpathlineto{\pgfqpoint{0.897661in}{2.999679in}}%
\pgfpathlineto{\pgfqpoint{0.903722in}{2.999679in}}%
\pgfpathlineto{\pgfqpoint{0.903722in}{2.994204in}}%
\pgfpathlineto{\pgfqpoint{0.909783in}{2.994204in}}%
\pgfpathlineto{\pgfqpoint{0.909783in}{2.995846in}}%
\pgfpathlineto{\pgfqpoint{0.915844in}{2.995846in}}%
\pgfpathlineto{\pgfqpoint{0.915844in}{2.994204in}}%
\pgfpathlineto{\pgfqpoint{0.921905in}{2.994204in}}%
\pgfpathlineto{\pgfqpoint{0.921905in}{2.997489in}}%
\pgfpathlineto{\pgfqpoint{0.927966in}{2.997489in}}%
\pgfpathlineto{\pgfqpoint{0.927966in}{2.991466in}}%
\pgfpathlineto{\pgfqpoint{0.940088in}{2.992014in}}%
\pgfpathlineto{\pgfqpoint{0.940088in}{3.000227in}}%
\pgfpathlineto{\pgfqpoint{0.946149in}{3.000227in}}%
\pgfpathlineto{\pgfqpoint{0.946149in}{2.992561in}}%
\pgfpathlineto{\pgfqpoint{0.952210in}{2.992561in}}%
\pgfpathlineto{\pgfqpoint{0.952210in}{2.999679in}}%
\pgfpathlineto{\pgfqpoint{0.958271in}{2.999679in}}%
\pgfpathlineto{\pgfqpoint{0.958271in}{2.991466in}}%
\pgfpathlineto{\pgfqpoint{0.964332in}{2.991466in}}%
\pgfpathlineto{\pgfqpoint{0.964332in}{2.998584in}}%
\pgfpathlineto{\pgfqpoint{0.970393in}{2.998584in}}%
\pgfpathlineto{\pgfqpoint{0.970393in}{3.002964in}}%
\pgfpathlineto{\pgfqpoint{0.976454in}{3.002964in}}%
\pgfpathlineto{\pgfqpoint{0.976454in}{2.999132in}}%
\pgfpathlineto{\pgfqpoint{0.982515in}{2.999132in}}%
\pgfpathlineto{\pgfqpoint{0.982515in}{2.995846in}}%
\pgfpathlineto{\pgfqpoint{0.988576in}{2.995846in}}%
\pgfpathlineto{\pgfqpoint{0.988576in}{3.000774in}}%
\pgfpathlineto{\pgfqpoint{0.994637in}{3.000774in}}%
\pgfpathlineto{\pgfqpoint{0.994637in}{2.990371in}}%
\pgfpathlineto{\pgfqpoint{1.000698in}{2.990371in}}%
\pgfpathlineto{\pgfqpoint{1.000698in}{2.999132in}}%
\pgfpathlineto{\pgfqpoint{1.006759in}{2.999132in}}%
\pgfpathlineto{\pgfqpoint{1.006759in}{3.001322in}}%
\pgfpathlineto{\pgfqpoint{1.012820in}{3.001322in}}%
\pgfpathlineto{\pgfqpoint{1.012820in}{2.993656in}}%
\pgfpathlineto{\pgfqpoint{1.024942in}{2.993656in}}%
\pgfpathlineto{\pgfqpoint{1.024942in}{2.995299in}}%
\pgfpathlineto{\pgfqpoint{1.037064in}{2.996394in}}%
\pgfpathlineto{\pgfqpoint{1.037064in}{2.993109in}}%
\pgfpathlineto{\pgfqpoint{1.043125in}{2.993109in}}%
\pgfpathlineto{\pgfqpoint{1.043125in}{3.001869in}}%
\pgfpathlineto{\pgfqpoint{1.049186in}{3.001869in}}%
\pgfpathlineto{\pgfqpoint{1.049186in}{2.999132in}}%
\pgfpathlineto{\pgfqpoint{1.055247in}{2.999132in}}%
\pgfpathlineto{\pgfqpoint{1.055247in}{2.996394in}}%
\pgfpathlineto{\pgfqpoint{1.061308in}{2.996394in}}%
\pgfpathlineto{\pgfqpoint{1.061308in}{2.994751in}}%
\pgfpathlineto{\pgfqpoint{1.067369in}{2.994751in}}%
\pgfpathlineto{\pgfqpoint{1.067369in}{3.000774in}}%
\pgfpathlineto{\pgfqpoint{1.073430in}{3.000774in}}%
\pgfpathlineto{\pgfqpoint{1.073430in}{3.006250in}}%
\pgfpathlineto{\pgfqpoint{1.079491in}{3.006250in}}%
\pgfpathlineto{\pgfqpoint{1.079491in}{2.994204in}}%
\pgfpathlineto{\pgfqpoint{1.085552in}{2.994204in}}%
\pgfpathlineto{\pgfqpoint{1.085552in}{2.997489in}}%
\pgfpathlineto{\pgfqpoint{1.091613in}{2.997489in}}%
\pgfpathlineto{\pgfqpoint{1.091613in}{2.994204in}}%
\pgfpathlineto{\pgfqpoint{1.097674in}{2.994204in}}%
\pgfpathlineto{\pgfqpoint{1.097674in}{3.004607in}}%
\pgfpathlineto{\pgfqpoint{1.103735in}{3.004607in}}%
\pgfpathlineto{\pgfqpoint{1.103735in}{2.995299in}}%
\pgfpathlineto{\pgfqpoint{1.115857in}{2.994204in}}%
\pgfpathlineto{\pgfqpoint{1.115857in}{2.999679in}}%
\pgfpathlineto{\pgfqpoint{1.121918in}{2.999679in}}%
\pgfpathlineto{\pgfqpoint{1.121918in}{2.996942in}}%
\pgfpathlineto{\pgfqpoint{1.134040in}{2.996942in}}%
\pgfpathlineto{\pgfqpoint{1.134040in}{2.993109in}}%
\pgfpathlineto{\pgfqpoint{1.140101in}{2.993109in}}%
\pgfpathlineto{\pgfqpoint{1.140101in}{2.999679in}}%
\pgfpathlineto{\pgfqpoint{1.146162in}{2.999679in}}%
\pgfpathlineto{\pgfqpoint{1.146162in}{2.995846in}}%
\pgfpathlineto{\pgfqpoint{1.152223in}{2.995846in}}%
\pgfpathlineto{\pgfqpoint{1.152223in}{2.993109in}}%
\pgfpathlineto{\pgfqpoint{1.158284in}{2.993109in}}%
\pgfpathlineto{\pgfqpoint{1.158284in}{2.997489in}}%
\pgfpathlineto{\pgfqpoint{1.164345in}{2.997489in}}%
\pgfpathlineto{\pgfqpoint{1.164345in}{2.999132in}}%
\pgfpathlineto{\pgfqpoint{1.170406in}{2.999132in}}%
\pgfpathlineto{\pgfqpoint{1.170406in}{3.003512in}}%
\pgfpathlineto{\pgfqpoint{1.176467in}{3.003512in}}%
\pgfpathlineto{\pgfqpoint{1.176467in}{2.993109in}}%
\pgfpathlineto{\pgfqpoint{1.182527in}{2.993109in}}%
\pgfpathlineto{\pgfqpoint{1.182527in}{3.001322in}}%
\pgfpathlineto{\pgfqpoint{1.194649in}{3.000227in}}%
\pgfpathlineto{\pgfqpoint{1.194649in}{2.995846in}}%
\pgfpathlineto{\pgfqpoint{1.200710in}{2.995846in}}%
\pgfpathlineto{\pgfqpoint{1.200710in}{2.999679in}}%
\pgfpathlineto{\pgfqpoint{1.206771in}{2.999679in}}%
\pgfpathlineto{\pgfqpoint{1.206771in}{3.005702in}}%
\pgfpathlineto{\pgfqpoint{1.212832in}{3.005702in}}%
\pgfpathlineto{\pgfqpoint{1.212832in}{3.001322in}}%
\pgfpathlineto{\pgfqpoint{1.224954in}{3.002417in}}%
\pgfpathlineto{\pgfqpoint{1.224954in}{3.002964in}}%
\pgfpathlineto{\pgfqpoint{1.231015in}{3.002964in}}%
\pgfpathlineto{\pgfqpoint{1.231015in}{2.995299in}}%
\pgfpathlineto{\pgfqpoint{1.237076in}{2.995299in}}%
\pgfpathlineto{\pgfqpoint{1.237076in}{3.004060in}}%
\pgfpathlineto{\pgfqpoint{1.243137in}{3.004060in}}%
\pgfpathlineto{\pgfqpoint{1.243137in}{3.001322in}}%
\pgfpathlineto{\pgfqpoint{1.249198in}{3.001322in}}%
\pgfpathlineto{\pgfqpoint{1.249198in}{2.990371in}}%
\pgfpathlineto{\pgfqpoint{1.255259in}{2.990371in}}%
\pgfpathlineto{\pgfqpoint{1.255259in}{3.000227in}}%
\pgfpathlineto{\pgfqpoint{1.267381in}{3.001322in}}%
\pgfpathlineto{\pgfqpoint{1.267381in}{2.996942in}}%
\pgfpathlineto{\pgfqpoint{1.273442in}{2.996942in}}%
\pgfpathlineto{\pgfqpoint{1.273442in}{3.000227in}}%
\pgfpathlineto{\pgfqpoint{1.285564in}{2.999132in}}%
\pgfpathlineto{\pgfqpoint{1.285564in}{2.994751in}}%
\pgfpathlineto{\pgfqpoint{1.297686in}{2.995846in}}%
\pgfpathlineto{\pgfqpoint{1.297686in}{2.997489in}}%
\pgfpathlineto{\pgfqpoint{1.303747in}{2.997489in}}%
\pgfpathlineto{\pgfqpoint{1.303747in}{3.001869in}}%
\pgfpathlineto{\pgfqpoint{1.321930in}{3.002964in}}%
\pgfpathlineto{\pgfqpoint{1.321930in}{2.995846in}}%
\pgfpathlineto{\pgfqpoint{1.327991in}{2.995846in}}%
\pgfpathlineto{\pgfqpoint{1.327991in}{3.001869in}}%
\pgfpathlineto{\pgfqpoint{1.334052in}{3.001869in}}%
\pgfpathlineto{\pgfqpoint{1.334052in}{2.998584in}}%
\pgfpathlineto{\pgfqpoint{1.340113in}{2.998584in}}%
\pgfpathlineto{\pgfqpoint{1.340113in}{3.005155in}}%
\pgfpathlineto{\pgfqpoint{1.346174in}{3.005155in}}%
\pgfpathlineto{\pgfqpoint{1.346174in}{3.001869in}}%
\pgfpathlineto{\pgfqpoint{1.352235in}{3.001869in}}%
\pgfpathlineto{\pgfqpoint{1.352235in}{2.998584in}}%
\pgfpathlineto{\pgfqpoint{1.358296in}{2.998584in}}%
\pgfpathlineto{\pgfqpoint{1.358296in}{3.005155in}}%
\pgfpathlineto{\pgfqpoint{1.364357in}{3.005155in}}%
\pgfpathlineto{\pgfqpoint{1.364357in}{2.998037in}}%
\pgfpathlineto{\pgfqpoint{1.370418in}{2.998037in}}%
\pgfpathlineto{\pgfqpoint{1.370418in}{3.001322in}}%
\pgfpathlineto{\pgfqpoint{1.376479in}{3.001322in}}%
\pgfpathlineto{\pgfqpoint{1.376479in}{2.997489in}}%
\pgfpathlineto{\pgfqpoint{1.382540in}{2.997489in}}%
\pgfpathlineto{\pgfqpoint{1.382540in}{3.004607in}}%
\pgfpathlineto{\pgfqpoint{1.388601in}{3.004607in}}%
\pgfpathlineto{\pgfqpoint{1.388601in}{3.002964in}}%
\pgfpathlineto{\pgfqpoint{1.400723in}{3.003512in}}%
\pgfpathlineto{\pgfqpoint{1.400723in}{3.005155in}}%
\pgfpathlineto{\pgfqpoint{1.406784in}{3.005155in}}%
\pgfpathlineto{\pgfqpoint{1.406784in}{2.995299in}}%
\pgfpathlineto{\pgfqpoint{1.412845in}{2.995299in}}%
\pgfpathlineto{\pgfqpoint{1.412845in}{3.000227in}}%
\pgfpathlineto{\pgfqpoint{1.424967in}{2.999679in}}%
\pgfpathlineto{\pgfqpoint{1.424967in}{3.005155in}}%
\pgfpathlineto{\pgfqpoint{1.437089in}{3.004060in}}%
\pgfpathlineto{\pgfqpoint{1.437089in}{3.001322in}}%
\pgfpathlineto{\pgfqpoint{1.443150in}{3.001322in}}%
\pgfpathlineto{\pgfqpoint{1.443150in}{2.997489in}}%
\pgfpathlineto{\pgfqpoint{1.455272in}{2.998584in}}%
\pgfpathlineto{\pgfqpoint{1.455272in}{3.001869in}}%
\pgfpathlineto{\pgfqpoint{1.461333in}{3.001869in}}%
\pgfpathlineto{\pgfqpoint{1.461333in}{2.999132in}}%
\pgfpathlineto{\pgfqpoint{1.467394in}{2.999132in}}%
\pgfpathlineto{\pgfqpoint{1.467394in}{3.007892in}}%
\pgfpathlineto{\pgfqpoint{1.473454in}{3.007892in}}%
\pgfpathlineto{\pgfqpoint{1.473454in}{3.000227in}}%
\pgfpathlineto{\pgfqpoint{1.479515in}{3.000227in}}%
\pgfpathlineto{\pgfqpoint{1.479515in}{3.003512in}}%
\pgfpathlineto{\pgfqpoint{1.485576in}{3.003512in}}%
\pgfpathlineto{\pgfqpoint{1.485576in}{3.001322in}}%
\pgfpathlineto{\pgfqpoint{1.491637in}{3.001322in}}%
\pgfpathlineto{\pgfqpoint{1.491637in}{3.003512in}}%
\pgfpathlineto{\pgfqpoint{1.497698in}{3.003512in}}%
\pgfpathlineto{\pgfqpoint{1.497698in}{3.000227in}}%
\pgfpathlineto{\pgfqpoint{1.503759in}{3.000227in}}%
\pgfpathlineto{\pgfqpoint{1.503759in}{3.009535in}}%
\pgfpathlineto{\pgfqpoint{1.515881in}{3.009535in}}%
\pgfpathlineto{\pgfqpoint{1.515881in}{3.004607in}}%
\pgfpathlineto{\pgfqpoint{1.546186in}{3.003512in}}%
\pgfpathlineto{\pgfqpoint{1.546186in}{3.001322in}}%
\pgfpathlineto{\pgfqpoint{1.552247in}{3.001322in}}%
\pgfpathlineto{\pgfqpoint{1.552247in}{3.005702in}}%
\pgfpathlineto{\pgfqpoint{1.558308in}{3.005702in}}%
\pgfpathlineto{\pgfqpoint{1.558308in}{3.004060in}}%
\pgfpathlineto{\pgfqpoint{1.576491in}{3.005155in}}%
\pgfpathlineto{\pgfqpoint{1.576491in}{3.001322in}}%
\pgfpathlineto{\pgfqpoint{1.588613in}{3.001322in}}%
\pgfpathlineto{\pgfqpoint{1.588613in}{3.005155in}}%
\pgfpathlineto{\pgfqpoint{1.594674in}{3.005155in}}%
\pgfpathlineto{\pgfqpoint{1.594674in}{3.008440in}}%
\pgfpathlineto{\pgfqpoint{1.600735in}{3.008440in}}%
\pgfpathlineto{\pgfqpoint{1.600735in}{3.005155in}}%
\pgfpathlineto{\pgfqpoint{1.612857in}{3.005702in}}%
\pgfpathlineto{\pgfqpoint{1.612857in}{3.009535in}}%
\pgfpathlineto{\pgfqpoint{1.624979in}{3.008987in}}%
\pgfpathlineto{\pgfqpoint{1.624979in}{3.004060in}}%
\pgfpathlineto{\pgfqpoint{1.631040in}{3.004060in}}%
\pgfpathlineto{\pgfqpoint{1.631040in}{3.016653in}}%
\pgfpathlineto{\pgfqpoint{1.637101in}{3.016653in}}%
\pgfpathlineto{\pgfqpoint{1.637101in}{3.003512in}}%
\pgfpathlineto{\pgfqpoint{1.643162in}{3.003512in}}%
\pgfpathlineto{\pgfqpoint{1.643162in}{3.013915in}}%
\pgfpathlineto{\pgfqpoint{1.649223in}{3.013915in}}%
\pgfpathlineto{\pgfqpoint{1.649223in}{3.006250in}}%
\pgfpathlineto{\pgfqpoint{1.655284in}{3.006250in}}%
\pgfpathlineto{\pgfqpoint{1.655284in}{3.010082in}}%
\pgfpathlineto{\pgfqpoint{1.661345in}{3.010082in}}%
\pgfpathlineto{\pgfqpoint{1.661345in}{3.000774in}}%
\pgfpathlineto{\pgfqpoint{1.667406in}{3.000774in}}%
\pgfpathlineto{\pgfqpoint{1.667406in}{3.010082in}}%
\pgfpathlineto{\pgfqpoint{1.673467in}{3.010082in}}%
\pgfpathlineto{\pgfqpoint{1.673467in}{3.003512in}}%
\pgfpathlineto{\pgfqpoint{1.679528in}{3.003512in}}%
\pgfpathlineto{\pgfqpoint{1.679528in}{3.007892in}}%
\pgfpathlineto{\pgfqpoint{1.685589in}{3.007892in}}%
\pgfpathlineto{\pgfqpoint{1.685589in}{3.009535in}}%
\pgfpathlineto{\pgfqpoint{1.691650in}{3.009535in}}%
\pgfpathlineto{\pgfqpoint{1.691650in}{3.006250in}}%
\pgfpathlineto{\pgfqpoint{1.697711in}{3.006250in}}%
\pgfpathlineto{\pgfqpoint{1.697711in}{3.011177in}}%
\pgfpathlineto{\pgfqpoint{1.703772in}{3.011177in}}%
\pgfpathlineto{\pgfqpoint{1.703772in}{2.999679in}}%
\pgfpathlineto{\pgfqpoint{1.709833in}{2.999679in}}%
\pgfpathlineto{\pgfqpoint{1.709833in}{3.023223in}}%
\pgfpathlineto{\pgfqpoint{1.715894in}{3.023223in}}%
\pgfpathlineto{\pgfqpoint{1.715894in}{3.013915in}}%
\pgfpathlineto{\pgfqpoint{1.721955in}{3.013915in}}%
\pgfpathlineto{\pgfqpoint{1.721955in}{3.015558in}}%
\pgfpathlineto{\pgfqpoint{1.728016in}{3.015558in}}%
\pgfpathlineto{\pgfqpoint{1.728016in}{3.010082in}}%
\pgfpathlineto{\pgfqpoint{1.734077in}{3.010082in}}%
\pgfpathlineto{\pgfqpoint{1.734077in}{3.015010in}}%
\pgfpathlineto{\pgfqpoint{1.740138in}{3.015010in}}%
\pgfpathlineto{\pgfqpoint{1.740138in}{3.008987in}}%
\pgfpathlineto{\pgfqpoint{1.746199in}{3.008987in}}%
\pgfpathlineto{\pgfqpoint{1.746199in}{3.005155in}}%
\pgfpathlineto{\pgfqpoint{1.752260in}{3.005155in}}%
\pgfpathlineto{\pgfqpoint{1.752260in}{3.014463in}}%
\pgfpathlineto{\pgfqpoint{1.758321in}{3.014463in}}%
\pgfpathlineto{\pgfqpoint{1.758321in}{3.001869in}}%
\pgfpathlineto{\pgfqpoint{1.764381in}{3.001869in}}%
\pgfpathlineto{\pgfqpoint{1.764381in}{3.023771in}}%
\pgfpathlineto{\pgfqpoint{1.770442in}{3.023771in}}%
\pgfpathlineto{\pgfqpoint{1.770442in}{3.008440in}}%
\pgfpathlineto{\pgfqpoint{1.776503in}{3.008440in}}%
\pgfpathlineto{\pgfqpoint{1.776503in}{3.018295in}}%
\pgfpathlineto{\pgfqpoint{1.788625in}{3.017748in}}%
\pgfpathlineto{\pgfqpoint{1.788625in}{3.014463in}}%
\pgfpathlineto{\pgfqpoint{1.800747in}{3.014463in}}%
\pgfpathlineto{\pgfqpoint{1.800747in}{3.008987in}}%
\pgfpathlineto{\pgfqpoint{1.806808in}{3.008987in}}%
\pgfpathlineto{\pgfqpoint{1.806808in}{3.016653in}}%
\pgfpathlineto{\pgfqpoint{1.812869in}{3.016653in}}%
\pgfpathlineto{\pgfqpoint{1.812869in}{3.009535in}}%
\pgfpathlineto{\pgfqpoint{1.818930in}{3.009535in}}%
\pgfpathlineto{\pgfqpoint{1.818930in}{3.019938in}}%
\pgfpathlineto{\pgfqpoint{1.824991in}{3.019938in}}%
\pgfpathlineto{\pgfqpoint{1.824991in}{3.010630in}}%
\pgfpathlineto{\pgfqpoint{1.843174in}{3.009535in}}%
\pgfpathlineto{\pgfqpoint{1.843174in}{3.022128in}}%
\pgfpathlineto{\pgfqpoint{1.849235in}{3.022128in}}%
\pgfpathlineto{\pgfqpoint{1.849235in}{3.010630in}}%
\pgfpathlineto{\pgfqpoint{1.867418in}{3.010082in}}%
\pgfpathlineto{\pgfqpoint{1.867418in}{3.017748in}}%
\pgfpathlineto{\pgfqpoint{1.879540in}{3.017748in}}%
\pgfpathlineto{\pgfqpoint{1.879540in}{3.009535in}}%
\pgfpathlineto{\pgfqpoint{1.885601in}{3.009535in}}%
\pgfpathlineto{\pgfqpoint{1.885601in}{3.029246in}}%
\pgfpathlineto{\pgfqpoint{1.891662in}{3.029246in}}%
\pgfpathlineto{\pgfqpoint{1.891662in}{3.012820in}}%
\pgfpathlineto{\pgfqpoint{1.897723in}{3.012820in}}%
\pgfpathlineto{\pgfqpoint{1.897723in}{3.011177in}}%
\pgfpathlineto{\pgfqpoint{1.903784in}{3.011177in}}%
\pgfpathlineto{\pgfqpoint{1.903784in}{3.031984in}}%
\pgfpathlineto{\pgfqpoint{1.909845in}{3.031984in}}%
\pgfpathlineto{\pgfqpoint{1.909845in}{3.016105in}}%
\pgfpathlineto{\pgfqpoint{1.921967in}{3.015010in}}%
\pgfpathlineto{\pgfqpoint{1.921967in}{3.013915in}}%
\pgfpathlineto{\pgfqpoint{1.928028in}{3.013915in}}%
\pgfpathlineto{\pgfqpoint{1.928028in}{3.028699in}}%
\pgfpathlineto{\pgfqpoint{1.934089in}{3.028699in}}%
\pgfpathlineto{\pgfqpoint{1.934089in}{3.017200in}}%
\pgfpathlineto{\pgfqpoint{1.940150in}{3.017200in}}%
\pgfpathlineto{\pgfqpoint{1.940150in}{3.020486in}}%
\pgfpathlineto{\pgfqpoint{1.946211in}{3.020486in}}%
\pgfpathlineto{\pgfqpoint{1.946211in}{3.022128in}}%
\pgfpathlineto{\pgfqpoint{1.958333in}{3.022128in}}%
\pgfpathlineto{\pgfqpoint{1.958333in}{3.026508in}}%
\pgfpathlineto{\pgfqpoint{1.964394in}{3.026508in}}%
\pgfpathlineto{\pgfqpoint{1.964394in}{3.010630in}}%
\pgfpathlineto{\pgfqpoint{1.970455in}{3.010630in}}%
\pgfpathlineto{\pgfqpoint{1.970455in}{3.035817in}}%
\pgfpathlineto{\pgfqpoint{1.976516in}{3.035817in}}%
\pgfpathlineto{\pgfqpoint{1.976516in}{3.018295in}}%
\pgfpathlineto{\pgfqpoint{1.982577in}{3.018295in}}%
\pgfpathlineto{\pgfqpoint{1.982577in}{3.032531in}}%
\pgfpathlineto{\pgfqpoint{1.988638in}{3.032531in}}%
\pgfpathlineto{\pgfqpoint{1.988638in}{3.022676in}}%
\pgfpathlineto{\pgfqpoint{1.994699in}{3.022676in}}%
\pgfpathlineto{\pgfqpoint{1.994699in}{3.035269in}}%
\pgfpathlineto{\pgfqpoint{2.000760in}{3.035269in}}%
\pgfpathlineto{\pgfqpoint{2.000760in}{3.024318in}}%
\pgfpathlineto{\pgfqpoint{2.006821in}{3.024318in}}%
\pgfpathlineto{\pgfqpoint{2.006821in}{3.022676in}}%
\pgfpathlineto{\pgfqpoint{2.018943in}{3.021581in}}%
\pgfpathlineto{\pgfqpoint{2.018943in}{3.034174in}}%
\pgfpathlineto{\pgfqpoint{2.025004in}{3.034174in}}%
\pgfpathlineto{\pgfqpoint{2.025004in}{3.032531in}}%
\pgfpathlineto{\pgfqpoint{2.031065in}{3.032531in}}%
\pgfpathlineto{\pgfqpoint{2.031065in}{3.019391in}}%
\pgfpathlineto{\pgfqpoint{2.037126in}{3.019391in}}%
\pgfpathlineto{\pgfqpoint{2.037126in}{3.028699in}}%
\pgfpathlineto{\pgfqpoint{2.043187in}{3.028699in}}%
\pgfpathlineto{\pgfqpoint{2.043187in}{3.024318in}}%
\pgfpathlineto{\pgfqpoint{2.049248in}{3.024318in}}%
\pgfpathlineto{\pgfqpoint{2.049248in}{3.035269in}}%
\pgfpathlineto{\pgfqpoint{2.055308in}{3.035269in}}%
\pgfpathlineto{\pgfqpoint{2.055308in}{3.027056in}}%
\pgfpathlineto{\pgfqpoint{2.061369in}{3.027056in}}%
\pgfpathlineto{\pgfqpoint{2.061369in}{3.045672in}}%
\pgfpathlineto{\pgfqpoint{2.067430in}{3.045672in}}%
\pgfpathlineto{\pgfqpoint{2.067430in}{3.015010in}}%
\pgfpathlineto{\pgfqpoint{2.073491in}{3.015010in}}%
\pgfpathlineto{\pgfqpoint{2.073491in}{3.045672in}}%
\pgfpathlineto{\pgfqpoint{2.079552in}{3.045672in}}%
\pgfpathlineto{\pgfqpoint{2.079552in}{3.041840in}}%
\pgfpathlineto{\pgfqpoint{2.085613in}{3.041840in}}%
\pgfpathlineto{\pgfqpoint{2.085613in}{3.028699in}}%
\pgfpathlineto{\pgfqpoint{2.091674in}{3.028699in}}%
\pgfpathlineto{\pgfqpoint{2.091674in}{3.031984in}}%
\pgfpathlineto{\pgfqpoint{2.097735in}{3.031984in}}%
\pgfpathlineto{\pgfqpoint{2.097735in}{3.027604in}}%
\pgfpathlineto{\pgfqpoint{2.103796in}{3.027604in}}%
\pgfpathlineto{\pgfqpoint{2.103796in}{3.039102in}}%
\pgfpathlineto{\pgfqpoint{2.109857in}{3.039102in}}%
\pgfpathlineto{\pgfqpoint{2.109857in}{3.035817in}}%
\pgfpathlineto{\pgfqpoint{2.115918in}{3.035817in}}%
\pgfpathlineto{\pgfqpoint{2.115918in}{3.040197in}}%
\pgfpathlineto{\pgfqpoint{2.121979in}{3.040197in}}%
\pgfpathlineto{\pgfqpoint{2.121979in}{3.030341in}}%
\pgfpathlineto{\pgfqpoint{2.128040in}{3.030341in}}%
\pgfpathlineto{\pgfqpoint{2.128040in}{3.038007in}}%
\pgfpathlineto{\pgfqpoint{2.134101in}{3.038007in}}%
\pgfpathlineto{\pgfqpoint{2.134101in}{3.044577in}}%
\pgfpathlineto{\pgfqpoint{2.140162in}{3.044577in}}%
\pgfpathlineto{\pgfqpoint{2.140162in}{3.030341in}}%
\pgfpathlineto{\pgfqpoint{2.146223in}{3.030341in}}%
\pgfpathlineto{\pgfqpoint{2.146223in}{3.044030in}}%
\pgfpathlineto{\pgfqpoint{2.152284in}{3.044030in}}%
\pgfpathlineto{\pgfqpoint{2.152284in}{3.034722in}}%
\pgfpathlineto{\pgfqpoint{2.158345in}{3.034722in}}%
\pgfpathlineto{\pgfqpoint{2.158345in}{3.047862in}}%
\pgfpathlineto{\pgfqpoint{2.164406in}{3.047862in}}%
\pgfpathlineto{\pgfqpoint{2.164406in}{3.021033in}}%
\pgfpathlineto{\pgfqpoint{2.170467in}{3.021033in}}%
\pgfpathlineto{\pgfqpoint{2.170467in}{3.050053in}}%
\pgfpathlineto{\pgfqpoint{2.176528in}{3.050053in}}%
\pgfpathlineto{\pgfqpoint{2.176528in}{3.039649in}}%
\pgfpathlineto{\pgfqpoint{2.182589in}{3.039649in}}%
\pgfpathlineto{\pgfqpoint{2.182589in}{3.045672in}}%
\pgfpathlineto{\pgfqpoint{2.188650in}{3.045672in}}%
\pgfpathlineto{\pgfqpoint{2.188650in}{3.053338in}}%
\pgfpathlineto{\pgfqpoint{2.194711in}{3.053338in}}%
\pgfpathlineto{\pgfqpoint{2.194711in}{3.036364in}}%
\pgfpathlineto{\pgfqpoint{2.200772in}{3.036364in}}%
\pgfpathlineto{\pgfqpoint{2.200772in}{3.047315in}}%
\pgfpathlineto{\pgfqpoint{2.206833in}{3.047315in}}%
\pgfpathlineto{\pgfqpoint{2.206833in}{3.034722in}}%
\pgfpathlineto{\pgfqpoint{2.212894in}{3.034722in}}%
\pgfpathlineto{\pgfqpoint{2.212894in}{3.048410in}}%
\pgfpathlineto{\pgfqpoint{2.218955in}{3.048410in}}%
\pgfpathlineto{\pgfqpoint{2.218955in}{3.050600in}}%
\pgfpathlineto{\pgfqpoint{2.225016in}{3.050600in}}%
\pgfpathlineto{\pgfqpoint{2.225016in}{3.044030in}}%
\pgfpathlineto{\pgfqpoint{2.237138in}{3.044030in}}%
\pgfpathlineto{\pgfqpoint{2.237138in}{3.051148in}}%
\pgfpathlineto{\pgfqpoint{2.249260in}{3.050600in}}%
\pgfpathlineto{\pgfqpoint{2.249260in}{3.043482in}}%
\pgfpathlineto{\pgfqpoint{2.255321in}{3.043482in}}%
\pgfpathlineto{\pgfqpoint{2.255321in}{3.053885in}}%
\pgfpathlineto{\pgfqpoint{2.261382in}{3.053885in}}%
\pgfpathlineto{\pgfqpoint{2.261382in}{3.046767in}}%
\pgfpathlineto{\pgfqpoint{2.267443in}{3.046767in}}%
\pgfpathlineto{\pgfqpoint{2.267443in}{3.065931in}}%
\pgfpathlineto{\pgfqpoint{2.273504in}{3.065931in}}%
\pgfpathlineto{\pgfqpoint{2.273504in}{3.059361in}}%
\pgfpathlineto{\pgfqpoint{2.279565in}{3.059361in}}%
\pgfpathlineto{\pgfqpoint{2.279565in}{3.064836in}}%
\pgfpathlineto{\pgfqpoint{2.285626in}{3.064836in}}%
\pgfpathlineto{\pgfqpoint{2.285626in}{3.051695in}}%
\pgfpathlineto{\pgfqpoint{2.291687in}{3.051695in}}%
\pgfpathlineto{\pgfqpoint{2.291687in}{3.055528in}}%
\pgfpathlineto{\pgfqpoint{2.297748in}{3.055528in}}%
\pgfpathlineto{\pgfqpoint{2.297748in}{3.060456in}}%
\pgfpathlineto{\pgfqpoint{2.303809in}{3.060456in}}%
\pgfpathlineto{\pgfqpoint{2.303809in}{3.054980in}}%
\pgfpathlineto{\pgfqpoint{2.309870in}{3.054980in}}%
\pgfpathlineto{\pgfqpoint{2.309870in}{3.071406in}}%
\pgfpathlineto{\pgfqpoint{2.315931in}{3.071406in}}%
\pgfpathlineto{\pgfqpoint{2.315931in}{3.064288in}}%
\pgfpathlineto{\pgfqpoint{2.321992in}{3.064288in}}%
\pgfpathlineto{\pgfqpoint{2.321992in}{3.079072in}}%
\pgfpathlineto{\pgfqpoint{2.328053in}{3.079072in}}%
\pgfpathlineto{\pgfqpoint{2.328053in}{3.059908in}}%
\pgfpathlineto{\pgfqpoint{2.334114in}{3.059908in}}%
\pgfpathlineto{\pgfqpoint{2.334114in}{3.079072in}}%
\pgfpathlineto{\pgfqpoint{2.340175in}{3.079072in}}%
\pgfpathlineto{\pgfqpoint{2.340175in}{3.051148in}}%
\pgfpathlineto{\pgfqpoint{2.346235in}{3.051148in}}%
\pgfpathlineto{\pgfqpoint{2.346235in}{3.077429in}}%
\pgfpathlineto{\pgfqpoint{2.352296in}{3.077429in}}%
\pgfpathlineto{\pgfqpoint{2.352296in}{3.058266in}}%
\pgfpathlineto{\pgfqpoint{2.358357in}{3.058266in}}%
\pgfpathlineto{\pgfqpoint{2.358357in}{3.076334in}}%
\pgfpathlineto{\pgfqpoint{2.364418in}{3.076334in}}%
\pgfpathlineto{\pgfqpoint{2.364418in}{3.083452in}}%
\pgfpathlineto{\pgfqpoint{2.370479in}{3.083452in}}%
\pgfpathlineto{\pgfqpoint{2.370479in}{3.061551in}}%
\pgfpathlineto{\pgfqpoint{2.376540in}{3.061551in}}%
\pgfpathlineto{\pgfqpoint{2.376540in}{3.086190in}}%
\pgfpathlineto{\pgfqpoint{2.382601in}{3.086190in}}%
\pgfpathlineto{\pgfqpoint{2.382601in}{3.062098in}}%
\pgfpathlineto{\pgfqpoint{2.388662in}{3.062098in}}%
\pgfpathlineto{\pgfqpoint{2.388662in}{3.077977in}}%
\pgfpathlineto{\pgfqpoint{2.394723in}{3.077977in}}%
\pgfpathlineto{\pgfqpoint{2.394723in}{3.067574in}}%
\pgfpathlineto{\pgfqpoint{2.400784in}{3.067574in}}%
\pgfpathlineto{\pgfqpoint{2.400784in}{3.081262in}}%
\pgfpathlineto{\pgfqpoint{2.406845in}{3.081262in}}%
\pgfpathlineto{\pgfqpoint{2.406845in}{3.084000in}}%
\pgfpathlineto{\pgfqpoint{2.412906in}{3.084000in}}%
\pgfpathlineto{\pgfqpoint{2.412906in}{3.088380in}}%
\pgfpathlineto{\pgfqpoint{2.418967in}{3.088380in}}%
\pgfpathlineto{\pgfqpoint{2.418967in}{3.099331in}}%
\pgfpathlineto{\pgfqpoint{2.425028in}{3.099331in}}%
\pgfpathlineto{\pgfqpoint{2.425028in}{3.075239in}}%
\pgfpathlineto{\pgfqpoint{2.431089in}{3.075239in}}%
\pgfpathlineto{\pgfqpoint{2.431089in}{3.094403in}}%
\pgfpathlineto{\pgfqpoint{2.437150in}{3.094403in}}%
\pgfpathlineto{\pgfqpoint{2.437150in}{3.077429in}}%
\pgfpathlineto{\pgfqpoint{2.443211in}{3.077429in}}%
\pgfpathlineto{\pgfqpoint{2.443211in}{3.102068in}}%
\pgfpathlineto{\pgfqpoint{2.449272in}{3.102068in}}%
\pgfpathlineto{\pgfqpoint{2.449272in}{3.089475in}}%
\pgfpathlineto{\pgfqpoint{2.455333in}{3.089475in}}%
\pgfpathlineto{\pgfqpoint{2.455333in}{3.110282in}}%
\pgfpathlineto{\pgfqpoint{2.461394in}{3.110282in}}%
\pgfpathlineto{\pgfqpoint{2.461394in}{3.092213in}}%
\pgfpathlineto{\pgfqpoint{2.467455in}{3.092213in}}%
\pgfpathlineto{\pgfqpoint{2.467455in}{3.096593in}}%
\pgfpathlineto{\pgfqpoint{2.473516in}{3.096593in}}%
\pgfpathlineto{\pgfqpoint{2.473516in}{3.112472in}}%
\pgfpathlineto{\pgfqpoint{2.479577in}{3.112472in}}%
\pgfpathlineto{\pgfqpoint{2.479577in}{3.087285in}}%
\pgfpathlineto{\pgfqpoint{2.485638in}{3.087285in}}%
\pgfpathlineto{\pgfqpoint{2.485638in}{3.104259in}}%
\pgfpathlineto{\pgfqpoint{2.491699in}{3.104259in}}%
\pgfpathlineto{\pgfqpoint{2.491699in}{3.088380in}}%
\pgfpathlineto{\pgfqpoint{2.497760in}{3.088380in}}%
\pgfpathlineto{\pgfqpoint{2.497760in}{3.100973in}}%
\pgfpathlineto{\pgfqpoint{2.503821in}{3.100973in}}%
\pgfpathlineto{\pgfqpoint{2.503821in}{3.095498in}}%
\pgfpathlineto{\pgfqpoint{2.509882in}{3.095498in}}%
\pgfpathlineto{\pgfqpoint{2.509882in}{3.088928in}}%
\pgfpathlineto{\pgfqpoint{2.515943in}{3.088928in}}%
\pgfpathlineto{\pgfqpoint{2.515943in}{3.094403in}}%
\pgfpathlineto{\pgfqpoint{2.522004in}{3.094403in}}%
\pgfpathlineto{\pgfqpoint{2.522004in}{3.109734in}}%
\pgfpathlineto{\pgfqpoint{2.534126in}{3.109186in}}%
\pgfpathlineto{\pgfqpoint{2.534126in}{3.083452in}}%
\pgfpathlineto{\pgfqpoint{2.540187in}{3.083452in}}%
\pgfpathlineto{\pgfqpoint{2.540187in}{3.108091in}}%
\pgfpathlineto{\pgfqpoint{2.546248in}{3.108091in}}%
\pgfpathlineto{\pgfqpoint{2.546248in}{3.105354in}}%
\pgfpathlineto{\pgfqpoint{2.552309in}{3.105354in}}%
\pgfpathlineto{\pgfqpoint{2.552309in}{3.110282in}}%
\pgfpathlineto{\pgfqpoint{2.558370in}{3.110282in}}%
\pgfpathlineto{\pgfqpoint{2.558370in}{3.095498in}}%
\pgfpathlineto{\pgfqpoint{2.564431in}{3.095498in}}%
\pgfpathlineto{\pgfqpoint{2.564431in}{3.137111in}}%
\pgfpathlineto{\pgfqpoint{2.570492in}{3.137111in}}%
\pgfpathlineto{\pgfqpoint{2.570492in}{3.090570in}}%
\pgfpathlineto{\pgfqpoint{2.576553in}{3.090570in}}%
\pgfpathlineto{\pgfqpoint{2.576553in}{3.137111in}}%
\pgfpathlineto{\pgfqpoint{2.582614in}{3.137111in}}%
\pgfpathlineto{\pgfqpoint{2.582614in}{3.120685in}}%
\pgfpathlineto{\pgfqpoint{2.588675in}{3.120685in}}%
\pgfpathlineto{\pgfqpoint{2.588675in}{3.097688in}}%
\pgfpathlineto{\pgfqpoint{2.594736in}{3.097688in}}%
\pgfpathlineto{\pgfqpoint{2.594736in}{3.115757in}}%
\pgfpathlineto{\pgfqpoint{2.600797in}{3.115757in}}%
\pgfpathlineto{\pgfqpoint{2.600797in}{3.103164in}}%
\pgfpathlineto{\pgfqpoint{2.606858in}{3.103164in}}%
\pgfpathlineto{\pgfqpoint{2.606858in}{3.118495in}}%
\pgfpathlineto{\pgfqpoint{2.612919in}{3.118495in}}%
\pgfpathlineto{\pgfqpoint{2.612919in}{3.104259in}}%
\pgfpathlineto{\pgfqpoint{2.618980in}{3.104259in}}%
\pgfpathlineto{\pgfqpoint{2.618980in}{3.121780in}}%
\pgfpathlineto{\pgfqpoint{2.625041in}{3.121780in}}%
\pgfpathlineto{\pgfqpoint{2.625041in}{3.107544in}}%
\pgfpathlineto{\pgfqpoint{2.631102in}{3.107544in}}%
\pgfpathlineto{\pgfqpoint{2.631102in}{3.135468in}}%
\pgfpathlineto{\pgfqpoint{2.637162in}{3.135468in}}%
\pgfpathlineto{\pgfqpoint{2.637162in}{3.126708in}}%
\pgfpathlineto{\pgfqpoint{2.643223in}{3.126708in}}%
\pgfpathlineto{\pgfqpoint{2.643223in}{3.120685in}}%
\pgfpathlineto{\pgfqpoint{2.649284in}{3.120685in}}%
\pgfpathlineto{\pgfqpoint{2.649284in}{3.152989in}}%
\pgfpathlineto{\pgfqpoint{2.655345in}{3.152989in}}%
\pgfpathlineto{\pgfqpoint{2.655345in}{3.121780in}}%
\pgfpathlineto{\pgfqpoint{2.661406in}{3.121780in}}%
\pgfpathlineto{\pgfqpoint{2.661406in}{3.143681in}}%
\pgfpathlineto{\pgfqpoint{2.667467in}{3.143681in}}%
\pgfpathlineto{\pgfqpoint{2.667467in}{3.124517in}}%
\pgfpathlineto{\pgfqpoint{2.673528in}{3.124517in}}%
\pgfpathlineto{\pgfqpoint{2.673528in}{3.151347in}}%
\pgfpathlineto{\pgfqpoint{2.679589in}{3.151347in}}%
\pgfpathlineto{\pgfqpoint{2.679589in}{3.141491in}}%
\pgfpathlineto{\pgfqpoint{2.685650in}{3.141491in}}%
\pgfpathlineto{\pgfqpoint{2.685650in}{3.151894in}}%
\pgfpathlineto{\pgfqpoint{2.691711in}{3.151894in}}%
\pgfpathlineto{\pgfqpoint{2.691711in}{3.127255in}}%
\pgfpathlineto{\pgfqpoint{2.697772in}{3.127255in}}%
\pgfpathlineto{\pgfqpoint{2.697772in}{3.154632in}}%
\pgfpathlineto{\pgfqpoint{2.703833in}{3.154632in}}%
\pgfpathlineto{\pgfqpoint{2.703833in}{3.138206in}}%
\pgfpathlineto{\pgfqpoint{2.709894in}{3.138206in}}%
\pgfpathlineto{\pgfqpoint{2.709894in}{3.126160in}}%
\pgfpathlineto{\pgfqpoint{2.715955in}{3.126160in}}%
\pgfpathlineto{\pgfqpoint{2.715955in}{3.175438in}}%
\pgfpathlineto{\pgfqpoint{2.722016in}{3.175438in}}%
\pgfpathlineto{\pgfqpoint{2.722016in}{3.126708in}}%
\pgfpathlineto{\pgfqpoint{2.728077in}{3.126708in}}%
\pgfpathlineto{\pgfqpoint{2.728077in}{3.164488in}}%
\pgfpathlineto{\pgfqpoint{2.734138in}{3.164488in}}%
\pgfpathlineto{\pgfqpoint{2.734138in}{3.127803in}}%
\pgfpathlineto{\pgfqpoint{2.740199in}{3.127803in}}%
\pgfpathlineto{\pgfqpoint{2.740199in}{3.161750in}}%
\pgfpathlineto{\pgfqpoint{2.746260in}{3.161750in}}%
\pgfpathlineto{\pgfqpoint{2.746260in}{3.144776in}}%
\pgfpathlineto{\pgfqpoint{2.752321in}{3.144776in}}%
\pgfpathlineto{\pgfqpoint{2.752321in}{3.152442in}}%
\pgfpathlineto{\pgfqpoint{2.758382in}{3.152442in}}%
\pgfpathlineto{\pgfqpoint{2.758382in}{3.166130in}}%
\pgfpathlineto{\pgfqpoint{2.764443in}{3.166130in}}%
\pgfpathlineto{\pgfqpoint{2.764443in}{3.147514in}}%
\pgfpathlineto{\pgfqpoint{2.770504in}{3.147514in}}%
\pgfpathlineto{\pgfqpoint{2.770504in}{3.166678in}}%
\pgfpathlineto{\pgfqpoint{2.776565in}{3.166678in}}%
\pgfpathlineto{\pgfqpoint{2.776565in}{3.146419in}}%
\pgfpathlineto{\pgfqpoint{2.782626in}{3.146419in}}%
\pgfpathlineto{\pgfqpoint{2.782626in}{3.184746in}}%
\pgfpathlineto{\pgfqpoint{2.788687in}{3.184746in}}%
\pgfpathlineto{\pgfqpoint{2.788687in}{3.157370in}}%
\pgfpathlineto{\pgfqpoint{2.794748in}{3.157370in}}%
\pgfpathlineto{\pgfqpoint{2.794748in}{3.190222in}}%
\pgfpathlineto{\pgfqpoint{2.800809in}{3.190222in}}%
\pgfpathlineto{\pgfqpoint{2.800809in}{3.143681in}}%
\pgfpathlineto{\pgfqpoint{2.806870in}{3.143681in}}%
\pgfpathlineto{\pgfqpoint{2.806870in}{3.178724in}}%
\pgfpathlineto{\pgfqpoint{2.812931in}{3.178724in}}%
\pgfpathlineto{\pgfqpoint{2.812931in}{3.181461in}}%
\pgfpathlineto{\pgfqpoint{2.818992in}{3.181461in}}%
\pgfpathlineto{\pgfqpoint{2.818992in}{3.149704in}}%
\pgfpathlineto{\pgfqpoint{2.825053in}{3.149704in}}%
\pgfpathlineto{\pgfqpoint{2.825053in}{3.217599in}}%
\pgfpathlineto{\pgfqpoint{2.831114in}{3.217599in}}%
\pgfpathlineto{\pgfqpoint{2.831114in}{3.169415in}}%
\pgfpathlineto{\pgfqpoint{2.837175in}{3.169415in}}%
\pgfpathlineto{\pgfqpoint{2.837175in}{3.185294in}}%
\pgfpathlineto{\pgfqpoint{2.843236in}{3.185294in}}%
\pgfpathlineto{\pgfqpoint{2.843236in}{3.175986in}}%
\pgfpathlineto{\pgfqpoint{2.849297in}{3.175986in}}%
\pgfpathlineto{\pgfqpoint{2.849297in}{3.215956in}}%
\pgfpathlineto{\pgfqpoint{2.855358in}{3.215956in}}%
\pgfpathlineto{\pgfqpoint{2.855358in}{3.171606in}}%
\pgfpathlineto{\pgfqpoint{2.861419in}{3.171606in}}%
\pgfpathlineto{\pgfqpoint{2.861419in}{3.182556in}}%
\pgfpathlineto{\pgfqpoint{2.867480in}{3.182556in}}%
\pgfpathlineto{\pgfqpoint{2.867480in}{3.223621in}}%
\pgfpathlineto{\pgfqpoint{2.873541in}{3.223621in}}%
\pgfpathlineto{\pgfqpoint{2.873541in}{3.170510in}}%
\pgfpathlineto{\pgfqpoint{2.879602in}{3.170510in}}%
\pgfpathlineto{\pgfqpoint{2.879602in}{3.201720in}}%
\pgfpathlineto{\pgfqpoint{2.891724in}{3.200625in}}%
\pgfpathlineto{\pgfqpoint{2.891724in}{3.236762in}}%
\pgfpathlineto{\pgfqpoint{2.897785in}{3.236762in}}%
\pgfpathlineto{\pgfqpoint{2.897785in}{3.174343in}}%
\pgfpathlineto{\pgfqpoint{2.903846in}{3.174343in}}%
\pgfpathlineto{\pgfqpoint{2.903846in}{3.209386in}}%
\pgfpathlineto{\pgfqpoint{2.909907in}{3.209386in}}%
\pgfpathlineto{\pgfqpoint{2.909907in}{3.194602in}}%
\pgfpathlineto{\pgfqpoint{2.915968in}{3.194602in}}%
\pgfpathlineto{\pgfqpoint{2.915968in}{3.253736in}}%
\pgfpathlineto{\pgfqpoint{2.922029in}{3.253736in}}%
\pgfpathlineto{\pgfqpoint{2.922029in}{3.226907in}}%
\pgfpathlineto{\pgfqpoint{2.928089in}{3.226907in}}%
\pgfpathlineto{\pgfqpoint{2.928089in}{3.207195in}}%
\pgfpathlineto{\pgfqpoint{2.934150in}{3.207195in}}%
\pgfpathlineto{\pgfqpoint{2.934150in}{3.238952in}}%
\pgfpathlineto{\pgfqpoint{2.940211in}{3.238952in}}%
\pgfpathlineto{\pgfqpoint{2.940211in}{3.189674in}}%
\pgfpathlineto{\pgfqpoint{2.946272in}{3.189674in}}%
\pgfpathlineto{\pgfqpoint{2.946272in}{3.226359in}}%
\pgfpathlineto{\pgfqpoint{2.952333in}{3.226359in}}%
\pgfpathlineto{\pgfqpoint{2.952333in}{3.224717in}}%
\pgfpathlineto{\pgfqpoint{2.958394in}{3.224717in}}%
\pgfpathlineto{\pgfqpoint{2.958394in}{3.264139in}}%
\pgfpathlineto{\pgfqpoint{2.964455in}{3.264139in}}%
\pgfpathlineto{\pgfqpoint{2.964455in}{3.198982in}}%
\pgfpathlineto{\pgfqpoint{2.970516in}{3.198982in}}%
\pgfpathlineto{\pgfqpoint{2.970516in}{3.249903in}}%
\pgfpathlineto{\pgfqpoint{2.976577in}{3.249903in}}%
\pgfpathlineto{\pgfqpoint{2.976577in}{3.259759in}}%
\pgfpathlineto{\pgfqpoint{2.982638in}{3.259759in}}%
\pgfpathlineto{\pgfqpoint{2.982638in}{3.206648in}}%
\pgfpathlineto{\pgfqpoint{2.988699in}{3.206648in}}%
\pgfpathlineto{\pgfqpoint{2.988699in}{3.284398in}}%
\pgfpathlineto{\pgfqpoint{2.994760in}{3.284398in}}%
\pgfpathlineto{\pgfqpoint{2.994760in}{3.241143in}}%
\pgfpathlineto{\pgfqpoint{3.000821in}{3.241143in}}%
\pgfpathlineto{\pgfqpoint{3.000821in}{3.296991in}}%
\pgfpathlineto{\pgfqpoint{3.006882in}{3.296991in}}%
\pgfpathlineto{\pgfqpoint{3.006882in}{3.240595in}}%
\pgfpathlineto{\pgfqpoint{3.012943in}{3.240595in}}%
\pgfpathlineto{\pgfqpoint{3.012943in}{3.273995in}}%
\pgfpathlineto{\pgfqpoint{3.019004in}{3.273995in}}%
\pgfpathlineto{\pgfqpoint{3.019004in}{3.242785in}}%
\pgfpathlineto{\pgfqpoint{3.025065in}{3.242785in}}%
\pgfpathlineto{\pgfqpoint{3.025065in}{3.311775in}}%
\pgfpathlineto{\pgfqpoint{3.031126in}{3.311775in}}%
\pgfpathlineto{\pgfqpoint{3.031126in}{3.290968in}}%
\pgfpathlineto{\pgfqpoint{3.037187in}{3.290968in}}%
\pgfpathlineto{\pgfqpoint{3.037187in}{3.258116in}}%
\pgfpathlineto{\pgfqpoint{3.043248in}{3.258116in}}%
\pgfpathlineto{\pgfqpoint{3.043248in}{3.287683in}}%
\pgfpathlineto{\pgfqpoint{3.049309in}{3.287683in}}%
\pgfpathlineto{\pgfqpoint{3.049309in}{3.253736in}}%
\pgfpathlineto{\pgfqpoint{3.055370in}{3.253736in}}%
\pgfpathlineto{\pgfqpoint{3.055370in}{3.302467in}}%
\pgfpathlineto{\pgfqpoint{3.061431in}{3.302467in}}%
\pgfpathlineto{\pgfqpoint{3.061431in}{3.270710in}}%
\pgfpathlineto{\pgfqpoint{3.067492in}{3.270710in}}%
\pgfpathlineto{\pgfqpoint{3.067492in}{3.309585in}}%
\pgfpathlineto{\pgfqpoint{3.073553in}{3.309585in}}%
\pgfpathlineto{\pgfqpoint{3.073553in}{3.275090in}}%
\pgfpathlineto{\pgfqpoint{3.079614in}{3.275090in}}%
\pgfpathlineto{\pgfqpoint{3.079614in}{3.324916in}}%
\pgfpathlineto{\pgfqpoint{3.085675in}{3.324916in}}%
\pgfpathlineto{\pgfqpoint{3.085675in}{3.294801in}}%
\pgfpathlineto{\pgfqpoint{3.091736in}{3.294801in}}%
\pgfpathlineto{\pgfqpoint{3.091736in}{3.356673in}}%
\pgfpathlineto{\pgfqpoint{3.097797in}{3.356673in}}%
\pgfpathlineto{\pgfqpoint{3.097797in}{3.334224in}}%
\pgfpathlineto{\pgfqpoint{3.103858in}{3.334224in}}%
\pgfpathlineto{\pgfqpoint{3.103858in}{3.284946in}}%
\pgfpathlineto{\pgfqpoint{3.109919in}{3.284946in}}%
\pgfpathlineto{\pgfqpoint{3.109919in}{3.366528in}}%
\pgfpathlineto{\pgfqpoint{3.115980in}{3.366528in}}%
\pgfpathlineto{\pgfqpoint{3.115980in}{3.327653in}}%
\pgfpathlineto{\pgfqpoint{3.122041in}{3.327653in}}%
\pgfpathlineto{\pgfqpoint{3.122041in}{3.364338in}}%
\pgfpathlineto{\pgfqpoint{3.128102in}{3.364338in}}%
\pgfpathlineto{\pgfqpoint{3.128102in}{3.319440in}}%
\pgfpathlineto{\pgfqpoint{3.134163in}{3.319440in}}%
\pgfpathlineto{\pgfqpoint{3.134163in}{3.399928in}}%
\pgfpathlineto{\pgfqpoint{3.140224in}{3.399928in}}%
\pgfpathlineto{\pgfqpoint{3.140224in}{3.338604in}}%
\pgfpathlineto{\pgfqpoint{3.146285in}{3.338604in}}%
\pgfpathlineto{\pgfqpoint{3.146285in}{3.388977in}}%
\pgfpathlineto{\pgfqpoint{3.152346in}{3.388977in}}%
\pgfpathlineto{\pgfqpoint{3.152346in}{3.417997in}}%
\pgfpathlineto{\pgfqpoint{3.158407in}{3.417997in}}%
\pgfpathlineto{\pgfqpoint{3.158407in}{3.359958in}}%
\pgfpathlineto{\pgfqpoint{3.164468in}{3.359958in}}%
\pgfpathlineto{\pgfqpoint{3.164468in}{3.414164in}}%
\pgfpathlineto{\pgfqpoint{3.170529in}{3.414164in}}%
\pgfpathlineto{\pgfqpoint{3.170529in}{3.328748in}}%
\pgfpathlineto{\pgfqpoint{3.176590in}{3.328748in}}%
\pgfpathlineto{\pgfqpoint{3.176590in}{3.432233in}}%
\pgfpathlineto{\pgfqpoint{3.182651in}{3.432233in}}%
\pgfpathlineto{\pgfqpoint{3.182651in}{3.392810in}}%
\pgfpathlineto{\pgfqpoint{3.188712in}{3.392810in}}%
\pgfpathlineto{\pgfqpoint{3.188712in}{3.430043in}}%
\pgfpathlineto{\pgfqpoint{3.194773in}{3.430043in}}%
\pgfpathlineto{\pgfqpoint{3.194773in}{3.387335in}}%
\pgfpathlineto{\pgfqpoint{3.200834in}{3.387335in}}%
\pgfpathlineto{\pgfqpoint{3.200834in}{3.471108in}}%
\pgfpathlineto{\pgfqpoint{3.206895in}{3.471108in}}%
\pgfpathlineto{\pgfqpoint{3.206895in}{3.462347in}}%
\pgfpathlineto{\pgfqpoint{3.212956in}{3.462347in}}%
\pgfpathlineto{\pgfqpoint{3.212956in}{3.376384in}}%
\pgfpathlineto{\pgfqpoint{3.219016in}{3.376384in}}%
\pgfpathlineto{\pgfqpoint{3.219016in}{3.506698in}}%
\pgfpathlineto{\pgfqpoint{3.225077in}{3.506698in}}%
\pgfpathlineto{\pgfqpoint{3.225077in}{3.430590in}}%
\pgfpathlineto{\pgfqpoint{3.231138in}{3.430590in}}%
\pgfpathlineto{\pgfqpoint{3.231138in}{3.468918in}}%
\pgfpathlineto{\pgfqpoint{3.237199in}{3.468918in}}%
\pgfpathlineto{\pgfqpoint{3.237199in}{3.449754in}}%
\pgfpathlineto{\pgfqpoint{3.243260in}{3.449754in}}%
\pgfpathlineto{\pgfqpoint{3.243260in}{3.554333in}}%
\pgfpathlineto{\pgfqpoint{3.249321in}{3.554333in}}%
\pgfpathlineto{\pgfqpoint{3.249321in}{3.467823in}}%
\pgfpathlineto{\pgfqpoint{3.255382in}{3.467823in}}%
\pgfpathlineto{\pgfqpoint{3.255382in}{3.517648in}}%
\pgfpathlineto{\pgfqpoint{3.261443in}{3.517648in}}%
\pgfpathlineto{\pgfqpoint{3.261443in}{3.570759in}}%
\pgfpathlineto{\pgfqpoint{3.267504in}{3.570759in}}%
\pgfpathlineto{\pgfqpoint{3.267504in}{3.509435in}}%
\pgfpathlineto{\pgfqpoint{3.273565in}{3.509435in}}%
\pgfpathlineto{\pgfqpoint{3.273565in}{3.545573in}}%
\pgfpathlineto{\pgfqpoint{3.279626in}{3.545573in}}%
\pgfpathlineto{\pgfqpoint{3.279626in}{3.484796in}}%
\pgfpathlineto{\pgfqpoint{3.285687in}{3.484796in}}%
\pgfpathlineto{\pgfqpoint{3.285687in}{3.594303in}}%
\pgfpathlineto{\pgfqpoint{3.291748in}{3.594303in}}%
\pgfpathlineto{\pgfqpoint{3.291748in}{3.497937in}}%
\pgfpathlineto{\pgfqpoint{3.297809in}{3.497937in}}%
\pgfpathlineto{\pgfqpoint{3.297809in}{3.597589in}}%
\pgfpathlineto{\pgfqpoint{3.303870in}{3.597589in}}%
\pgfpathlineto{\pgfqpoint{3.303870in}{3.532979in}}%
\pgfpathlineto{\pgfqpoint{3.309931in}{3.532979in}}%
\pgfpathlineto{\pgfqpoint{3.309931in}{3.629346in}}%
\pgfpathlineto{\pgfqpoint{3.315992in}{3.629346in}}%
\pgfpathlineto{\pgfqpoint{3.315992in}{3.624418in}}%
\pgfpathlineto{\pgfqpoint{3.322053in}{3.624418in}}%
\pgfpathlineto{\pgfqpoint{3.322053in}{3.546668in}}%
\pgfpathlineto{\pgfqpoint{3.328114in}{3.546668in}}%
\pgfpathlineto{\pgfqpoint{3.328114in}{3.681909in}}%
\pgfpathlineto{\pgfqpoint{3.334175in}{3.681909in}}%
\pgfpathlineto{\pgfqpoint{3.334175in}{3.546668in}}%
\pgfpathlineto{\pgfqpoint{3.340236in}{3.546668in}}%
\pgfpathlineto{\pgfqpoint{3.340236in}{3.638106in}}%
\pgfpathlineto{\pgfqpoint{3.346297in}{3.638106in}}%
\pgfpathlineto{\pgfqpoint{3.346297in}{3.571854in}}%
\pgfpathlineto{\pgfqpoint{3.352358in}{3.571854in}}%
\pgfpathlineto{\pgfqpoint{3.352358in}{3.705453in}}%
\pgfpathlineto{\pgfqpoint{3.358419in}{3.705453in}}%
\pgfpathlineto{\pgfqpoint{3.358419in}{3.574045in}}%
\pgfpathlineto{\pgfqpoint{3.364480in}{3.574045in}}%
\pgfpathlineto{\pgfqpoint{3.364480in}{3.671506in}}%
\pgfpathlineto{\pgfqpoint{3.370541in}{3.671506in}}%
\pgfpathlineto{\pgfqpoint{3.370541in}{3.713119in}}%
\pgfpathlineto{\pgfqpoint{3.376602in}{3.713119in}}%
\pgfpathlineto{\pgfqpoint{3.376602in}{3.577330in}}%
\pgfpathlineto{\pgfqpoint{3.382663in}{3.577330in}}%
\pgfpathlineto{\pgfqpoint{3.382663in}{3.676981in}}%
\pgfpathlineto{\pgfqpoint{3.388724in}{3.676981in}}%
\pgfpathlineto{\pgfqpoint{3.388724in}{3.604707in}}%
\pgfpathlineto{\pgfqpoint{3.394785in}{3.604707in}}%
\pgfpathlineto{\pgfqpoint{3.394785in}{3.725165in}}%
\pgfpathlineto{\pgfqpoint{3.400846in}{3.725165in}}%
\pgfpathlineto{\pgfqpoint{3.400846in}{3.600874in}}%
\pgfpathlineto{\pgfqpoint{3.406907in}{3.600874in}}%
\pgfpathlineto{\pgfqpoint{3.406907in}{3.727355in}}%
\pgfpathlineto{\pgfqpoint{3.412968in}{3.727355in}}%
\pgfpathlineto{\pgfqpoint{3.412968in}{3.642487in}}%
\pgfpathlineto{\pgfqpoint{3.419029in}{3.642487in}}%
\pgfpathlineto{\pgfqpoint{3.419029in}{3.762945in}}%
\pgfpathlineto{\pgfqpoint{3.425090in}{3.762945in}}%
\pgfpathlineto{\pgfqpoint{3.425090in}{3.632083in}}%
\pgfpathlineto{\pgfqpoint{3.431151in}{3.632083in}}%
\pgfpathlineto{\pgfqpoint{3.431151in}{3.733925in}}%
\pgfpathlineto{\pgfqpoint{3.437212in}{3.733925in}}%
\pgfpathlineto{\pgfqpoint{3.437212in}{3.727902in}}%
\pgfpathlineto{\pgfqpoint{3.443273in}{3.727902in}}%
\pgfpathlineto{\pgfqpoint{3.443273in}{3.621133in}}%
\pgfpathlineto{\pgfqpoint{3.449334in}{3.621133in}}%
\pgfpathlineto{\pgfqpoint{3.449334in}{3.698335in}}%
\pgfpathlineto{\pgfqpoint{3.455395in}{3.698335in}}%
\pgfpathlineto{\pgfqpoint{3.455395in}{3.614015in}}%
\pgfpathlineto{\pgfqpoint{3.461456in}{3.614015in}}%
\pgfpathlineto{\pgfqpoint{3.461456in}{3.789226in}}%
\pgfpathlineto{\pgfqpoint{3.467517in}{3.789226in}}%
\pgfpathlineto{\pgfqpoint{3.467517in}{3.643034in}}%
\pgfpathlineto{\pgfqpoint{3.473578in}{3.643034in}}%
\pgfpathlineto{\pgfqpoint{3.473578in}{3.731735in}}%
\pgfpathlineto{\pgfqpoint{3.479639in}{3.731735in}}%
\pgfpathlineto{\pgfqpoint{3.479639in}{3.684099in}}%
\pgfpathlineto{\pgfqpoint{3.485700in}{3.684099in}}%
\pgfpathlineto{\pgfqpoint{3.485700in}{3.773348in}}%
\pgfpathlineto{\pgfqpoint{3.491761in}{3.773348in}}%
\pgfpathlineto{\pgfqpoint{3.491761in}{3.738853in}}%
\pgfpathlineto{\pgfqpoint{3.497822in}{3.738853in}}%
\pgfpathlineto{\pgfqpoint{3.497822in}{3.669863in}}%
\pgfpathlineto{\pgfqpoint{3.503883in}{3.669863in}}%
\pgfpathlineto{\pgfqpoint{3.503883in}{3.834124in}}%
\pgfpathlineto{\pgfqpoint{3.509943in}{3.834124in}}%
\pgfpathlineto{\pgfqpoint{3.509943in}{3.646319in}}%
\pgfpathlineto{\pgfqpoint{3.516004in}{3.646319in}}%
\pgfpathlineto{\pgfqpoint{3.516004in}{3.745423in}}%
\pgfpathlineto{\pgfqpoint{3.522065in}{3.745423in}}%
\pgfpathlineto{\pgfqpoint{3.522065in}{3.688480in}}%
\pgfpathlineto{\pgfqpoint{3.528126in}{3.688480in}}%
\pgfpathlineto{\pgfqpoint{3.528126in}{3.823173in}}%
\pgfpathlineto{\pgfqpoint{3.534187in}{3.823173in}}%
\pgfpathlineto{\pgfqpoint{3.534187in}{3.696693in}}%
\pgfpathlineto{\pgfqpoint{3.540248in}{3.696693in}}%
\pgfpathlineto{\pgfqpoint{3.540248in}{3.790321in}}%
\pgfpathlineto{\pgfqpoint{3.546309in}{3.790321in}}%
\pgfpathlineto{\pgfqpoint{3.546309in}{3.831387in}}%
\pgfpathlineto{\pgfqpoint{3.552370in}{3.831387in}}%
\pgfpathlineto{\pgfqpoint{3.552370in}{3.718047in}}%
\pgfpathlineto{\pgfqpoint{3.558431in}{3.718047in}}%
\pgfpathlineto{\pgfqpoint{3.558431in}{3.748709in}}%
\pgfpathlineto{\pgfqpoint{3.564492in}{3.748709in}}%
\pgfpathlineto{\pgfqpoint{3.564492in}{3.685742in}}%
\pgfpathlineto{\pgfqpoint{3.570553in}{3.685742in}}%
\pgfpathlineto{\pgfqpoint{3.570553in}{3.826459in}}%
\pgfpathlineto{\pgfqpoint{3.576614in}{3.826459in}}%
\pgfpathlineto{\pgfqpoint{3.576614in}{3.672601in}}%
\pgfpathlineto{\pgfqpoint{3.582675in}{3.672601in}}%
\pgfpathlineto{\pgfqpoint{3.582675in}{3.774443in}}%
\pgfpathlineto{\pgfqpoint{3.588736in}{3.774443in}}%
\pgfpathlineto{\pgfqpoint{3.588736in}{3.692312in}}%
\pgfpathlineto{\pgfqpoint{3.594797in}{3.692312in}}%
\pgfpathlineto{\pgfqpoint{3.594797in}{3.885045in}}%
\pgfpathlineto{\pgfqpoint{3.600858in}{3.885045in}}%
\pgfpathlineto{\pgfqpoint{3.600858in}{3.871357in}}%
\pgfpathlineto{\pgfqpoint{3.606919in}{3.871357in}}%
\pgfpathlineto{\pgfqpoint{3.606919in}{3.693955in}}%
\pgfpathlineto{\pgfqpoint{3.612980in}{3.693955in}}%
\pgfpathlineto{\pgfqpoint{3.612980in}{3.864786in}}%
\pgfpathlineto{\pgfqpoint{3.619041in}{3.864786in}}%
\pgfpathlineto{\pgfqpoint{3.619041in}{3.725712in}}%
\pgfpathlineto{\pgfqpoint{3.625102in}{3.725712in}}%
\pgfpathlineto{\pgfqpoint{3.625102in}{3.775538in}}%
\pgfpathlineto{\pgfqpoint{3.631163in}{3.775538in}}%
\pgfpathlineto{\pgfqpoint{3.631163in}{3.713666in}}%
\pgfpathlineto{\pgfqpoint{3.637224in}{3.713666in}}%
\pgfpathlineto{\pgfqpoint{3.637224in}{3.869167in}}%
\pgfpathlineto{\pgfqpoint{3.643285in}{3.869167in}}%
\pgfpathlineto{\pgfqpoint{3.643285in}{3.714214in}}%
\pgfpathlineto{\pgfqpoint{3.649346in}{3.714214in}}%
\pgfpathlineto{\pgfqpoint{3.649346in}{3.831934in}}%
\pgfpathlineto{\pgfqpoint{3.655407in}{3.831934in}}%
\pgfpathlineto{\pgfqpoint{3.655407in}{3.830291in}}%
\pgfpathlineto{\pgfqpoint{3.661468in}{3.830291in}}%
\pgfpathlineto{\pgfqpoint{3.661468in}{3.659460in}}%
\pgfpathlineto{\pgfqpoint{3.667529in}{3.659460in}}%
\pgfpathlineto{\pgfqpoint{3.667529in}{3.785941in}}%
\pgfpathlineto{\pgfqpoint{3.673590in}{3.785941in}}%
\pgfpathlineto{\pgfqpoint{3.673590in}{3.660555in}}%
\pgfpathlineto{\pgfqpoint{3.679651in}{3.660555in}}%
\pgfpathlineto{\pgfqpoint{3.679651in}{3.803462in}}%
\pgfpathlineto{\pgfqpoint{3.685712in}{3.803462in}}%
\pgfpathlineto{\pgfqpoint{3.685712in}{3.655627in}}%
\pgfpathlineto{\pgfqpoint{3.691773in}{3.655627in}}%
\pgfpathlineto{\pgfqpoint{3.691773in}{3.744876in}}%
\pgfpathlineto{\pgfqpoint{3.697834in}{3.744876in}}%
\pgfpathlineto{\pgfqpoint{3.697834in}{3.680814in}}%
\pgfpathlineto{\pgfqpoint{3.703895in}{3.680814in}}%
\pgfpathlineto{\pgfqpoint{3.703895in}{3.784846in}}%
\pgfpathlineto{\pgfqpoint{3.709956in}{3.784846in}}%
\pgfpathlineto{\pgfqpoint{3.709956in}{3.759659in}}%
\pgfpathlineto{\pgfqpoint{3.716017in}{3.759659in}}%
\pgfpathlineto{\pgfqpoint{3.716017in}{3.626608in}}%
\pgfpathlineto{\pgfqpoint{3.722078in}{3.626608in}}%
\pgfpathlineto{\pgfqpoint{3.722078in}{3.758564in}}%
\pgfpathlineto{\pgfqpoint{3.728139in}{3.758564in}}%
\pgfpathlineto{\pgfqpoint{3.728139in}{3.643034in}}%
\pgfpathlineto{\pgfqpoint{3.734200in}{3.643034in}}%
\pgfpathlineto{\pgfqpoint{3.734200in}{3.705453in}}%
\pgfpathlineto{\pgfqpoint{3.740261in}{3.705453in}}%
\pgfpathlineto{\pgfqpoint{3.740261in}{3.609087in}}%
\pgfpathlineto{\pgfqpoint{3.746322in}{3.609087in}}%
\pgfpathlineto{\pgfqpoint{3.746322in}{3.709286in}}%
\pgfpathlineto{\pgfqpoint{3.752383in}{3.709286in}}%
\pgfpathlineto{\pgfqpoint{3.752383in}{3.558166in}}%
\pgfpathlineto{\pgfqpoint{3.758444in}{3.558166in}}%
\pgfpathlineto{\pgfqpoint{3.758444in}{3.674244in}}%
\pgfpathlineto{\pgfqpoint{3.764505in}{3.674244in}}%
\pgfpathlineto{\pgfqpoint{3.764505in}{3.607444in}}%
\pgfpathlineto{\pgfqpoint{3.770566in}{3.607444in}}%
\pgfpathlineto{\pgfqpoint{3.770566in}{3.674244in}}%
\pgfpathlineto{\pgfqpoint{3.776627in}{3.674244in}}%
\pgfpathlineto{\pgfqpoint{3.776627in}{3.632083in}}%
\pgfpathlineto{\pgfqpoint{3.782688in}{3.632083in}}%
\pgfpathlineto{\pgfqpoint{3.782688in}{3.586090in}}%
\pgfpathlineto{\pgfqpoint{3.788749in}{3.586090in}}%
\pgfpathlineto{\pgfqpoint{3.788749in}{3.639201in}}%
\pgfpathlineto{\pgfqpoint{3.794810in}{3.639201in}}%
\pgfpathlineto{\pgfqpoint{3.794810in}{3.539002in}}%
\pgfpathlineto{\pgfqpoint{3.800870in}{3.539002in}}%
\pgfpathlineto{\pgfqpoint{3.800870in}{3.622228in}}%
\pgfpathlineto{\pgfqpoint{3.806931in}{3.622228in}}%
\pgfpathlineto{\pgfqpoint{3.806931in}{3.540097in}}%
\pgfpathlineto{\pgfqpoint{3.812992in}{3.540097in}}%
\pgfpathlineto{\pgfqpoint{3.812992in}{3.630441in}}%
\pgfpathlineto{\pgfqpoint{3.819053in}{3.630441in}}%
\pgfpathlineto{\pgfqpoint{3.819053in}{3.535170in}}%
\pgfpathlineto{\pgfqpoint{3.825114in}{3.535170in}}%
\pgfpathlineto{\pgfqpoint{3.825114in}{3.581163in}}%
\pgfpathlineto{\pgfqpoint{3.831175in}{3.581163in}}%
\pgfpathlineto{\pgfqpoint{3.831175in}{3.604159in}}%
\pgfpathlineto{\pgfqpoint{3.837236in}{3.604159in}}%
\pgfpathlineto{\pgfqpoint{3.837236in}{3.512173in}}%
\pgfpathlineto{\pgfqpoint{3.843297in}{3.512173in}}%
\pgfpathlineto{\pgfqpoint{3.843297in}{3.558166in}}%
\pgfpathlineto{\pgfqpoint{3.849358in}{3.558166in}}%
\pgfpathlineto{\pgfqpoint{3.849358in}{3.511625in}}%
\pgfpathlineto{\pgfqpoint{3.855419in}{3.511625in}}%
\pgfpathlineto{\pgfqpoint{3.855419in}{3.625513in}}%
\pgfpathlineto{\pgfqpoint{3.861480in}{3.625513in}}%
\pgfpathlineto{\pgfqpoint{3.861480in}{3.517648in}}%
\pgfpathlineto{\pgfqpoint{3.867541in}{3.517648in}}%
\pgfpathlineto{\pgfqpoint{3.867541in}{3.566379in}}%
\pgfpathlineto{\pgfqpoint{3.873602in}{3.566379in}}%
\pgfpathlineto{\pgfqpoint{3.873602in}{3.502865in}}%
\pgfpathlineto{\pgfqpoint{3.879663in}{3.502865in}}%
\pgfpathlineto{\pgfqpoint{3.879663in}{3.564736in}}%
\pgfpathlineto{\pgfqpoint{3.885724in}{3.564736in}}%
\pgfpathlineto{\pgfqpoint{3.885724in}{3.539550in}}%
\pgfpathlineto{\pgfqpoint{3.891785in}{3.539550in}}%
\pgfpathlineto{\pgfqpoint{3.891785in}{3.483701in}}%
\pgfpathlineto{\pgfqpoint{3.897846in}{3.483701in}}%
\pgfpathlineto{\pgfqpoint{3.897846in}{3.564736in}}%
\pgfpathlineto{\pgfqpoint{3.903907in}{3.564736in}}%
\pgfpathlineto{\pgfqpoint{3.903907in}{3.449206in}}%
\pgfpathlineto{\pgfqpoint{3.909968in}{3.449206in}}%
\pgfpathlineto{\pgfqpoint{3.909968in}{3.474941in}}%
\pgfpathlineto{\pgfqpoint{3.916029in}{3.474941in}}%
\pgfpathlineto{\pgfqpoint{3.916029in}{3.425662in}}%
\pgfpathlineto{\pgfqpoint{3.922090in}{3.425662in}}%
\pgfpathlineto{\pgfqpoint{3.922090in}{3.517101in}}%
\pgfpathlineto{\pgfqpoint{3.928151in}{3.517101in}}%
\pgfpathlineto{\pgfqpoint{3.928151in}{3.424567in}}%
\pgfpathlineto{\pgfqpoint{3.934212in}{3.424567in}}%
\pgfpathlineto{\pgfqpoint{3.934212in}{3.471108in}}%
\pgfpathlineto{\pgfqpoint{3.940273in}{3.471108in}}%
\pgfpathlineto{\pgfqpoint{3.940273in}{3.486986in}}%
\pgfpathlineto{\pgfqpoint{3.946334in}{3.486986in}}%
\pgfpathlineto{\pgfqpoint{3.946334in}{3.411426in}}%
\pgfpathlineto{\pgfqpoint{3.952395in}{3.411426in}}%
\pgfpathlineto{\pgfqpoint{3.952395in}{3.450301in}}%
\pgfpathlineto{\pgfqpoint{3.958456in}{3.450301in}}%
\pgfpathlineto{\pgfqpoint{3.958456in}{3.414712in}}%
\pgfpathlineto{\pgfqpoint{3.964517in}{3.414712in}}%
\pgfpathlineto{\pgfqpoint{3.964517in}{3.479321in}}%
\pgfpathlineto{\pgfqpoint{3.970578in}{3.479321in}}%
\pgfpathlineto{\pgfqpoint{3.970578in}{3.389525in}}%
\pgfpathlineto{\pgfqpoint{3.976639in}{3.389525in}}%
\pgfpathlineto{\pgfqpoint{3.976639in}{3.449206in}}%
\pgfpathlineto{\pgfqpoint{3.982700in}{3.449206in}}%
\pgfpathlineto{\pgfqpoint{3.982700in}{3.396643in}}%
\pgfpathlineto{\pgfqpoint{3.988761in}{3.396643in}}%
\pgfpathlineto{\pgfqpoint{3.988761in}{3.451396in}}%
\pgfpathlineto{\pgfqpoint{3.994822in}{3.451396in}}%
\pgfpathlineto{\pgfqpoint{3.994822in}{3.407046in}}%
\pgfpathlineto{\pgfqpoint{4.000883in}{3.407046in}}%
\pgfpathlineto{\pgfqpoint{4.000883in}{3.352840in}}%
\pgfpathlineto{\pgfqpoint{4.006944in}{3.352840in}}%
\pgfpathlineto{\pgfqpoint{4.006944in}{3.443731in}}%
\pgfpathlineto{\pgfqpoint{4.013005in}{3.443731in}}%
\pgfpathlineto{\pgfqpoint{4.013005in}{3.364338in}}%
\pgfpathlineto{\pgfqpoint{4.019066in}{3.364338in}}%
\pgfpathlineto{\pgfqpoint{4.019066in}{3.427305in}}%
\pgfpathlineto{\pgfqpoint{4.025127in}{3.427305in}}%
\pgfpathlineto{\pgfqpoint{4.025127in}{3.349007in}}%
\pgfpathlineto{\pgfqpoint{4.031188in}{3.349007in}}%
\pgfpathlineto{\pgfqpoint{4.031188in}{3.416902in}}%
\pgfpathlineto{\pgfqpoint{4.037249in}{3.416902in}}%
\pgfpathlineto{\pgfqpoint{4.037249in}{3.334224in}}%
\pgfpathlineto{\pgfqpoint{4.043310in}{3.334224in}}%
\pgfpathlineto{\pgfqpoint{4.043310in}{3.396095in}}%
\pgfpathlineto{\pgfqpoint{4.049371in}{3.396095in}}%
\pgfpathlineto{\pgfqpoint{4.049371in}{3.429495in}}%
\pgfpathlineto{\pgfqpoint{4.055432in}{3.429495in}}%
\pgfpathlineto{\pgfqpoint{4.055432in}{3.361053in}}%
\pgfpathlineto{\pgfqpoint{4.061493in}{3.361053in}}%
\pgfpathlineto{\pgfqpoint{4.061493in}{3.374741in}}%
\pgfpathlineto{\pgfqpoint{4.067554in}{3.374741in}}%
\pgfpathlineto{\pgfqpoint{4.067554in}{3.318893in}}%
\pgfpathlineto{\pgfqpoint{4.073615in}{3.318893in}}%
\pgfpathlineto{\pgfqpoint{4.073615in}{3.402666in}}%
\pgfpathlineto{\pgfqpoint{4.079676in}{3.402666in}}%
\pgfpathlineto{\pgfqpoint{4.079676in}{3.345722in}}%
\pgfpathlineto{\pgfqpoint{4.085737in}{3.345722in}}%
\pgfpathlineto{\pgfqpoint{4.085737in}{3.381859in}}%
\pgfpathlineto{\pgfqpoint{4.091797in}{3.381859in}}%
\pgfpathlineto{\pgfqpoint{4.091797in}{3.299181in}}%
\pgfpathlineto{\pgfqpoint{4.097858in}{3.299181in}}%
\pgfpathlineto{\pgfqpoint{4.097858in}{3.386787in}}%
\pgfpathlineto{\pgfqpoint{4.103919in}{3.386787in}}%
\pgfpathlineto{\pgfqpoint{4.103919in}{3.316155in}}%
\pgfpathlineto{\pgfqpoint{4.109980in}{3.316155in}}%
\pgfpathlineto{\pgfqpoint{4.109980in}{3.324916in}}%
\pgfpathlineto{\pgfqpoint{4.116041in}{3.324916in}}%
\pgfpathlineto{\pgfqpoint{4.116041in}{3.391168in}}%
\pgfpathlineto{\pgfqpoint{4.122102in}{3.391168in}}%
\pgfpathlineto{\pgfqpoint{4.122102in}{3.291516in}}%
\pgfpathlineto{\pgfqpoint{4.128163in}{3.291516in}}%
\pgfpathlineto{\pgfqpoint{4.128163in}{3.324916in}}%
\pgfpathlineto{\pgfqpoint{4.134224in}{3.324916in}}%
\pgfpathlineto{\pgfqpoint{4.134224in}{3.302467in}}%
\pgfpathlineto{\pgfqpoint{4.140285in}{3.302467in}}%
\pgfpathlineto{\pgfqpoint{4.140285in}{3.345722in}}%
\pgfpathlineto{\pgfqpoint{4.146346in}{3.345722in}}%
\pgfpathlineto{\pgfqpoint{4.146346in}{3.300277in}}%
\pgfpathlineto{\pgfqpoint{4.152407in}{3.300277in}}%
\pgfpathlineto{\pgfqpoint{4.152407in}{3.318345in}}%
\pgfpathlineto{\pgfqpoint{4.158468in}{3.318345in}}%
\pgfpathlineto{\pgfqpoint{4.158468in}{3.281113in}}%
\pgfpathlineto{\pgfqpoint{4.164529in}{3.281113in}}%
\pgfpathlineto{\pgfqpoint{4.164529in}{3.331486in}}%
\pgfpathlineto{\pgfqpoint{4.170590in}{3.331486in}}%
\pgfpathlineto{\pgfqpoint{4.170590in}{3.284946in}}%
\pgfpathlineto{\pgfqpoint{4.176651in}{3.284946in}}%
\pgfpathlineto{\pgfqpoint{4.176651in}{3.261949in}}%
\pgfpathlineto{\pgfqpoint{4.182712in}{3.261949in}}%
\pgfpathlineto{\pgfqpoint{4.182712in}{3.294254in}}%
\pgfpathlineto{\pgfqpoint{4.188773in}{3.294254in}}%
\pgfpathlineto{\pgfqpoint{4.188773in}{3.261401in}}%
\pgfpathlineto{\pgfqpoint{4.194834in}{3.261401in}}%
\pgfpathlineto{\pgfqpoint{4.194834in}{3.290421in}}%
\pgfpathlineto{\pgfqpoint{4.200895in}{3.290421in}}%
\pgfpathlineto{\pgfqpoint{4.200895in}{3.247713in}}%
\pgfpathlineto{\pgfqpoint{4.206956in}{3.247713in}}%
\pgfpathlineto{\pgfqpoint{4.206956in}{3.272900in}}%
\pgfpathlineto{\pgfqpoint{4.213017in}{3.272900in}}%
\pgfpathlineto{\pgfqpoint{4.213017in}{3.228002in}}%
\pgfpathlineto{\pgfqpoint{4.219078in}{3.228002in}}%
\pgfpathlineto{\pgfqpoint{4.219078in}{3.267424in}}%
\pgfpathlineto{\pgfqpoint{4.225139in}{3.267424in}}%
\pgfpathlineto{\pgfqpoint{4.225139in}{3.277828in}}%
\pgfpathlineto{\pgfqpoint{4.231200in}{3.277828in}}%
\pgfpathlineto{\pgfqpoint{4.231200in}{3.228002in}}%
\pgfpathlineto{\pgfqpoint{4.237261in}{3.228002in}}%
\pgfpathlineto{\pgfqpoint{4.237261in}{3.269067in}}%
\pgfpathlineto{\pgfqpoint{4.243322in}{3.269067in}}%
\pgfpathlineto{\pgfqpoint{4.243322in}{3.237857in}}%
\pgfpathlineto{\pgfqpoint{4.249383in}{3.237857in}}%
\pgfpathlineto{\pgfqpoint{4.249383in}{3.280018in}}%
\pgfpathlineto{\pgfqpoint{4.255444in}{3.280018in}}%
\pgfpathlineto{\pgfqpoint{4.255444in}{3.237310in}}%
\pgfpathlineto{\pgfqpoint{4.261505in}{3.237310in}}%
\pgfpathlineto{\pgfqpoint{4.261505in}{3.258664in}}%
\pgfpathlineto{\pgfqpoint{4.267566in}{3.258664in}}%
\pgfpathlineto{\pgfqpoint{4.267566in}{3.234572in}}%
\pgfpathlineto{\pgfqpoint{4.273627in}{3.234572in}}%
\pgfpathlineto{\pgfqpoint{4.273627in}{3.253736in}}%
\pgfpathlineto{\pgfqpoint{4.279688in}{3.253736in}}%
\pgfpathlineto{\pgfqpoint{4.279688in}{3.236215in}}%
\pgfpathlineto{\pgfqpoint{4.285749in}{3.236215in}}%
\pgfpathlineto{\pgfqpoint{4.285749in}{3.213218in}}%
\pgfpathlineto{\pgfqpoint{4.291810in}{3.213218in}}%
\pgfpathlineto{\pgfqpoint{4.291810in}{3.242238in}}%
\pgfpathlineto{\pgfqpoint{4.297871in}{3.242238in}}%
\pgfpathlineto{\pgfqpoint{4.297871in}{3.202268in}}%
\pgfpathlineto{\pgfqpoint{4.303932in}{3.202268in}}%
\pgfpathlineto{\pgfqpoint{4.303932in}{3.218146in}}%
\pgfpathlineto{\pgfqpoint{4.309993in}{3.218146in}}%
\pgfpathlineto{\pgfqpoint{4.309993in}{3.200077in}}%
\pgfpathlineto{\pgfqpoint{4.316054in}{3.200077in}}%
\pgfpathlineto{\pgfqpoint{4.316054in}{3.218146in}}%
\pgfpathlineto{\pgfqpoint{4.322115in}{3.218146in}}%
\pgfpathlineto{\pgfqpoint{4.322115in}{3.178176in}}%
\pgfpathlineto{\pgfqpoint{4.328176in}{3.178176in}}%
\pgfpathlineto{\pgfqpoint{4.328176in}{3.223074in}}%
\pgfpathlineto{\pgfqpoint{4.340298in}{3.224169in}}%
\pgfpathlineto{\pgfqpoint{4.340298in}{3.168320in}}%
\pgfpathlineto{\pgfqpoint{4.346359in}{3.168320in}}%
\pgfpathlineto{\pgfqpoint{4.346359in}{3.215956in}}%
\pgfpathlineto{\pgfqpoint{4.352420in}{3.215956in}}%
\pgfpathlineto{\pgfqpoint{4.352420in}{3.162845in}}%
\pgfpathlineto{\pgfqpoint{4.358481in}{3.162845in}}%
\pgfpathlineto{\pgfqpoint{4.358481in}{3.198982in}}%
\pgfpathlineto{\pgfqpoint{4.364542in}{3.198982in}}%
\pgfpathlineto{\pgfqpoint{4.364542in}{3.169963in}}%
\pgfpathlineto{\pgfqpoint{4.370603in}{3.169963in}}%
\pgfpathlineto{\pgfqpoint{4.370603in}{3.180914in}}%
\pgfpathlineto{\pgfqpoint{4.376664in}{3.180914in}}%
\pgfpathlineto{\pgfqpoint{4.376664in}{3.165583in}}%
\pgfpathlineto{\pgfqpoint{4.382724in}{3.165583in}}%
\pgfpathlineto{\pgfqpoint{4.382724in}{3.209386in}}%
\pgfpathlineto{\pgfqpoint{4.388785in}{3.209386in}}%
\pgfpathlineto{\pgfqpoint{4.388785in}{3.199530in}}%
\pgfpathlineto{\pgfqpoint{4.394846in}{3.199530in}}%
\pgfpathlineto{\pgfqpoint{4.394846in}{3.150799in}}%
\pgfpathlineto{\pgfqpoint{4.400907in}{3.150799in}}%
\pgfpathlineto{\pgfqpoint{4.400907in}{3.207195in}}%
\pgfpathlineto{\pgfqpoint{4.406968in}{3.207195in}}%
\pgfpathlineto{\pgfqpoint{4.406968in}{3.162845in}}%
\pgfpathlineto{\pgfqpoint{4.413029in}{3.162845in}}%
\pgfpathlineto{\pgfqpoint{4.413029in}{3.194602in}}%
\pgfpathlineto{\pgfqpoint{4.419090in}{3.194602in}}%
\pgfpathlineto{\pgfqpoint{4.419090in}{3.157917in}}%
\pgfpathlineto{\pgfqpoint{4.425151in}{3.157917in}}%
\pgfpathlineto{\pgfqpoint{4.425151in}{3.178724in}}%
\pgfpathlineto{\pgfqpoint{4.431212in}{3.178724in}}%
\pgfpathlineto{\pgfqpoint{4.431212in}{3.148062in}}%
\pgfpathlineto{\pgfqpoint{4.437273in}{3.148062in}}%
\pgfpathlineto{\pgfqpoint{4.437273in}{3.158465in}}%
\pgfpathlineto{\pgfqpoint{4.443334in}{3.158465in}}%
\pgfpathlineto{\pgfqpoint{4.443334in}{3.162297in}}%
\pgfpathlineto{\pgfqpoint{4.449395in}{3.162297in}}%
\pgfpathlineto{\pgfqpoint{4.449395in}{3.166130in}}%
\pgfpathlineto{\pgfqpoint{4.455456in}{3.166130in}}%
\pgfpathlineto{\pgfqpoint{4.455456in}{3.167773in}}%
\pgfpathlineto{\pgfqpoint{4.461517in}{3.167773in}}%
\pgfpathlineto{\pgfqpoint{4.461517in}{3.125065in}}%
\pgfpathlineto{\pgfqpoint{4.467578in}{3.125065in}}%
\pgfpathlineto{\pgfqpoint{4.467578in}{3.171058in}}%
\pgfpathlineto{\pgfqpoint{4.473639in}{3.171058in}}%
\pgfpathlineto{\pgfqpoint{4.473639in}{3.146419in}}%
\pgfpathlineto{\pgfqpoint{4.479700in}{3.146419in}}%
\pgfpathlineto{\pgfqpoint{4.479700in}{3.156822in}}%
\pgfpathlineto{\pgfqpoint{4.485761in}{3.156822in}}%
\pgfpathlineto{\pgfqpoint{4.485761in}{3.129993in}}%
\pgfpathlineto{\pgfqpoint{4.491822in}{3.129993in}}%
\pgfpathlineto{\pgfqpoint{4.491822in}{3.168320in}}%
\pgfpathlineto{\pgfqpoint{4.497883in}{3.168320in}}%
\pgfpathlineto{\pgfqpoint{4.497883in}{3.137111in}}%
\pgfpathlineto{\pgfqpoint{4.503944in}{3.137111in}}%
\pgfpathlineto{\pgfqpoint{4.503944in}{3.151894in}}%
\pgfpathlineto{\pgfqpoint{4.510005in}{3.151894in}}%
\pgfpathlineto{\pgfqpoint{4.510005in}{3.163393in}}%
\pgfpathlineto{\pgfqpoint{4.516066in}{3.163393in}}%
\pgfpathlineto{\pgfqpoint{4.516066in}{3.125613in}}%
\pgfpathlineto{\pgfqpoint{4.522127in}{3.125613in}}%
\pgfpathlineto{\pgfqpoint{4.522127in}{3.155727in}}%
\pgfpathlineto{\pgfqpoint{4.528188in}{3.155727in}}%
\pgfpathlineto{\pgfqpoint{4.528188in}{3.134373in}}%
\pgfpathlineto{\pgfqpoint{4.534249in}{3.134373in}}%
\pgfpathlineto{\pgfqpoint{4.534249in}{3.163940in}}%
\pgfpathlineto{\pgfqpoint{4.540310in}{3.163940in}}%
\pgfpathlineto{\pgfqpoint{4.540310in}{3.126708in}}%
\pgfpathlineto{\pgfqpoint{4.546371in}{3.126708in}}%
\pgfpathlineto{\pgfqpoint{4.546371in}{3.137658in}}%
\pgfpathlineto{\pgfqpoint{4.552432in}{3.137658in}}%
\pgfpathlineto{\pgfqpoint{4.552432in}{3.126160in}}%
\pgfpathlineto{\pgfqpoint{4.558493in}{3.126160in}}%
\pgfpathlineto{\pgfqpoint{4.558493in}{3.168320in}}%
\pgfpathlineto{\pgfqpoint{4.564554in}{3.168320in}}%
\pgfpathlineto{\pgfqpoint{4.564554in}{3.136016in}}%
\pgfpathlineto{\pgfqpoint{4.570615in}{3.136016in}}%
\pgfpathlineto{\pgfqpoint{4.570615in}{3.113019in}}%
\pgfpathlineto{\pgfqpoint{4.576676in}{3.113019in}}%
\pgfpathlineto{\pgfqpoint{4.576676in}{3.133826in}}%
\pgfpathlineto{\pgfqpoint{4.582737in}{3.133826in}}%
\pgfpathlineto{\pgfqpoint{4.582737in}{3.116304in}}%
\pgfpathlineto{\pgfqpoint{4.588798in}{3.116304in}}%
\pgfpathlineto{\pgfqpoint{4.588798in}{3.120137in}}%
\pgfpathlineto{\pgfqpoint{4.594859in}{3.120137in}}%
\pgfpathlineto{\pgfqpoint{4.594859in}{3.111924in}}%
\pgfpathlineto{\pgfqpoint{4.600920in}{3.111924in}}%
\pgfpathlineto{\pgfqpoint{4.600920in}{3.161750in}}%
\pgfpathlineto{\pgfqpoint{4.606981in}{3.161750in}}%
\pgfpathlineto{\pgfqpoint{4.606981in}{3.127803in}}%
\pgfpathlineto{\pgfqpoint{4.613042in}{3.127803in}}%
\pgfpathlineto{\pgfqpoint{4.613042in}{3.154084in}}%
\pgfpathlineto{\pgfqpoint{4.619103in}{3.154084in}}%
\pgfpathlineto{\pgfqpoint{4.619103in}{3.150799in}}%
\pgfpathlineto{\pgfqpoint{4.625164in}{3.150799in}}%
\pgfpathlineto{\pgfqpoint{4.625164in}{3.134373in}}%
\pgfpathlineto{\pgfqpoint{4.631225in}{3.134373in}}%
\pgfpathlineto{\pgfqpoint{4.631225in}{3.139301in}}%
\pgfpathlineto{\pgfqpoint{4.637286in}{3.139301in}}%
\pgfpathlineto{\pgfqpoint{4.637286in}{3.122875in}}%
\pgfpathlineto{\pgfqpoint{4.643347in}{3.122875in}}%
\pgfpathlineto{\pgfqpoint{4.643347in}{3.142039in}}%
\pgfpathlineto{\pgfqpoint{4.649408in}{3.142039in}}%
\pgfpathlineto{\pgfqpoint{4.649408in}{3.115209in}}%
\pgfpathlineto{\pgfqpoint{4.655469in}{3.115209in}}%
\pgfpathlineto{\pgfqpoint{4.655469in}{3.149704in}}%
\pgfpathlineto{\pgfqpoint{4.661530in}{3.149704in}}%
\pgfpathlineto{\pgfqpoint{4.661530in}{3.113567in}}%
\pgfpathlineto{\pgfqpoint{4.667591in}{3.113567in}}%
\pgfpathlineto{\pgfqpoint{4.667591in}{3.139848in}}%
\pgfpathlineto{\pgfqpoint{4.673651in}{3.139848in}}%
\pgfpathlineto{\pgfqpoint{4.673651in}{3.141491in}}%
\pgfpathlineto{\pgfqpoint{4.679712in}{3.141491in}}%
\pgfpathlineto{\pgfqpoint{4.679712in}{3.120137in}}%
\pgfpathlineto{\pgfqpoint{4.685773in}{3.120137in}}%
\pgfpathlineto{\pgfqpoint{4.685773in}{3.148609in}}%
\pgfpathlineto{\pgfqpoint{4.691834in}{3.148609in}}%
\pgfpathlineto{\pgfqpoint{4.691834in}{3.106449in}}%
\pgfpathlineto{\pgfqpoint{4.697895in}{3.106449in}}%
\pgfpathlineto{\pgfqpoint{4.697895in}{3.132183in}}%
\pgfpathlineto{\pgfqpoint{4.703956in}{3.132183in}}%
\pgfpathlineto{\pgfqpoint{4.703956in}{3.110282in}}%
\pgfpathlineto{\pgfqpoint{4.710017in}{3.110282in}}%
\pgfpathlineto{\pgfqpoint{4.710017in}{3.156822in}}%
\pgfpathlineto{\pgfqpoint{4.716078in}{3.156822in}}%
\pgfpathlineto{\pgfqpoint{4.716078in}{3.099878in}}%
\pgfpathlineto{\pgfqpoint{4.722139in}{3.099878in}}%
\pgfpathlineto{\pgfqpoint{4.722139in}{3.133826in}}%
\pgfpathlineto{\pgfqpoint{4.728200in}{3.133826in}}%
\pgfpathlineto{\pgfqpoint{4.728200in}{3.160107in}}%
\pgfpathlineto{\pgfqpoint{4.734261in}{3.160107in}}%
\pgfpathlineto{\pgfqpoint{4.734261in}{3.112472in}}%
\pgfpathlineto{\pgfqpoint{4.740322in}{3.112472in}}%
\pgfpathlineto{\pgfqpoint{4.740322in}{3.128898in}}%
\pgfpathlineto{\pgfqpoint{4.746383in}{3.128898in}}%
\pgfpathlineto{\pgfqpoint{4.746383in}{3.111924in}}%
\pgfpathlineto{\pgfqpoint{4.752444in}{3.111924in}}%
\pgfpathlineto{\pgfqpoint{4.752444in}{3.122327in}}%
\pgfpathlineto{\pgfqpoint{4.758505in}{3.122327in}}%
\pgfpathlineto{\pgfqpoint{4.758505in}{3.114114in}}%
\pgfpathlineto{\pgfqpoint{4.764566in}{3.114114in}}%
\pgfpathlineto{\pgfqpoint{4.764566in}{3.119042in}}%
\pgfpathlineto{\pgfqpoint{4.770627in}{3.119042in}}%
\pgfpathlineto{\pgfqpoint{4.770627in}{3.104806in}}%
\pgfpathlineto{\pgfqpoint{4.776688in}{3.104806in}}%
\pgfpathlineto{\pgfqpoint{4.776688in}{3.125065in}}%
\pgfpathlineto{\pgfqpoint{4.782749in}{3.125065in}}%
\pgfpathlineto{\pgfqpoint{4.782749in}{3.099878in}}%
\pgfpathlineto{\pgfqpoint{4.788810in}{3.099878in}}%
\pgfpathlineto{\pgfqpoint{4.788810in}{3.127803in}}%
\pgfpathlineto{\pgfqpoint{4.794871in}{3.127803in}}%
\pgfpathlineto{\pgfqpoint{4.794871in}{3.134921in}}%
\pgfpathlineto{\pgfqpoint{4.800932in}{3.134921in}}%
\pgfpathlineto{\pgfqpoint{4.800932in}{3.098236in}}%
\pgfpathlineto{\pgfqpoint{4.806993in}{3.098236in}}%
\pgfpathlineto{\pgfqpoint{4.806993in}{3.117947in}}%
\pgfpathlineto{\pgfqpoint{4.813054in}{3.117947in}}%
\pgfpathlineto{\pgfqpoint{4.813054in}{3.101521in}}%
\pgfpathlineto{\pgfqpoint{4.819115in}{3.101521in}}%
\pgfpathlineto{\pgfqpoint{4.819115in}{3.131088in}}%
\pgfpathlineto{\pgfqpoint{4.825176in}{3.131088in}}%
\pgfpathlineto{\pgfqpoint{4.825176in}{3.103164in}}%
\pgfpathlineto{\pgfqpoint{4.837298in}{3.103164in}}%
\pgfpathlineto{\pgfqpoint{4.837298in}{3.100973in}}%
\pgfpathlineto{\pgfqpoint{4.843359in}{3.100973in}}%
\pgfpathlineto{\pgfqpoint{4.843359in}{3.113019in}}%
\pgfpathlineto{\pgfqpoint{4.849420in}{3.113019in}}%
\pgfpathlineto{\pgfqpoint{4.849420in}{3.115209in}}%
\pgfpathlineto{\pgfqpoint{4.855481in}{3.115209in}}%
\pgfpathlineto{\pgfqpoint{4.855481in}{3.100426in}}%
\pgfpathlineto{\pgfqpoint{4.861542in}{3.100426in}}%
\pgfpathlineto{\pgfqpoint{4.861542in}{3.133278in}}%
\pgfpathlineto{\pgfqpoint{4.867603in}{3.133278in}}%
\pgfpathlineto{\pgfqpoint{4.867603in}{3.076882in}}%
\pgfpathlineto{\pgfqpoint{4.873664in}{3.076882in}}%
\pgfpathlineto{\pgfqpoint{4.873664in}{3.090023in}}%
\pgfpathlineto{\pgfqpoint{4.879725in}{3.090023in}}%
\pgfpathlineto{\pgfqpoint{4.879725in}{3.093855in}}%
\pgfpathlineto{\pgfqpoint{4.885786in}{3.093855in}}%
\pgfpathlineto{\pgfqpoint{4.885786in}{3.139301in}}%
\pgfpathlineto{\pgfqpoint{4.891847in}{3.139301in}}%
\pgfpathlineto{\pgfqpoint{4.891847in}{3.081810in}}%
\pgfpathlineto{\pgfqpoint{4.897908in}{3.081810in}}%
\pgfpathlineto{\pgfqpoint{4.897908in}{3.106996in}}%
\pgfpathlineto{\pgfqpoint{4.903969in}{3.106996in}}%
\pgfpathlineto{\pgfqpoint{4.903969in}{3.122327in}}%
\pgfpathlineto{\pgfqpoint{4.910030in}{3.122327in}}%
\pgfpathlineto{\pgfqpoint{4.910030in}{3.087833in}}%
\pgfpathlineto{\pgfqpoint{4.916091in}{3.087833in}}%
\pgfpathlineto{\pgfqpoint{4.916091in}{3.113567in}}%
\pgfpathlineto{\pgfqpoint{4.922152in}{3.113567in}}%
\pgfpathlineto{\pgfqpoint{4.922152in}{3.077977in}}%
\pgfpathlineto{\pgfqpoint{4.928213in}{3.077977in}}%
\pgfpathlineto{\pgfqpoint{4.928213in}{3.121780in}}%
\pgfpathlineto{\pgfqpoint{4.934274in}{3.121780in}}%
\pgfpathlineto{\pgfqpoint{4.934274in}{3.097688in}}%
\pgfpathlineto{\pgfqpoint{4.940335in}{3.097688in}}%
\pgfpathlineto{\pgfqpoint{4.940335in}{3.113567in}}%
\pgfpathlineto{\pgfqpoint{4.946396in}{3.113567in}}%
\pgfpathlineto{\pgfqpoint{4.946396in}{3.101521in}}%
\pgfpathlineto{\pgfqpoint{4.952457in}{3.101521in}}%
\pgfpathlineto{\pgfqpoint{4.952457in}{3.103164in}}%
\pgfpathlineto{\pgfqpoint{4.958518in}{3.103164in}}%
\pgfpathlineto{\pgfqpoint{4.958518in}{3.106996in}}%
\pgfpathlineto{\pgfqpoint{4.964578in}{3.106996in}}%
\pgfpathlineto{\pgfqpoint{4.964578in}{3.105354in}}%
\pgfpathlineto{\pgfqpoint{4.970639in}{3.105354in}}%
\pgfpathlineto{\pgfqpoint{4.970639in}{3.137658in}}%
\pgfpathlineto{\pgfqpoint{4.976700in}{3.137658in}}%
\pgfpathlineto{\pgfqpoint{4.976700in}{3.085095in}}%
\pgfpathlineto{\pgfqpoint{4.982761in}{3.085095in}}%
\pgfpathlineto{\pgfqpoint{4.982761in}{3.076882in}}%
\pgfpathlineto{\pgfqpoint{4.988822in}{3.076882in}}%
\pgfpathlineto{\pgfqpoint{4.988822in}{3.082905in}}%
\pgfpathlineto{\pgfqpoint{4.994883in}{3.082905in}}%
\pgfpathlineto{\pgfqpoint{4.994883in}{3.117399in}}%
\pgfpathlineto{\pgfqpoint{5.000944in}{3.117399in}}%
\pgfpathlineto{\pgfqpoint{5.000944in}{3.087833in}}%
\pgfpathlineto{\pgfqpoint{5.007005in}{3.087833in}}%
\pgfpathlineto{\pgfqpoint{5.007005in}{3.111377in}}%
\pgfpathlineto{\pgfqpoint{5.013066in}{3.111377in}}%
\pgfpathlineto{\pgfqpoint{5.013066in}{3.109186in}}%
\pgfpathlineto{\pgfqpoint{5.019127in}{3.109186in}}%
\pgfpathlineto{\pgfqpoint{5.019127in}{3.092760in}}%
\pgfpathlineto{\pgfqpoint{5.031249in}{3.092213in}}%
\pgfpathlineto{\pgfqpoint{5.031249in}{3.087285in}}%
\pgfpathlineto{\pgfqpoint{5.037310in}{3.087285in}}%
\pgfpathlineto{\pgfqpoint{5.037310in}{3.095498in}}%
\pgfpathlineto{\pgfqpoint{5.043371in}{3.095498in}}%
\pgfpathlineto{\pgfqpoint{5.043371in}{3.069216in}}%
\pgfpathlineto{\pgfqpoint{5.049432in}{3.069216in}}%
\pgfpathlineto{\pgfqpoint{5.049432in}{3.097141in}}%
\pgfpathlineto{\pgfqpoint{5.055493in}{3.097141in}}%
\pgfpathlineto{\pgfqpoint{5.055493in}{3.074144in}}%
\pgfpathlineto{\pgfqpoint{5.061554in}{3.074144in}}%
\pgfpathlineto{\pgfqpoint{5.061554in}{3.100973in}}%
\pgfpathlineto{\pgfqpoint{5.067615in}{3.100973in}}%
\pgfpathlineto{\pgfqpoint{5.067615in}{3.103164in}}%
\pgfpathlineto{\pgfqpoint{5.073676in}{3.103164in}}%
\pgfpathlineto{\pgfqpoint{5.073676in}{3.069764in}}%
\pgfpathlineto{\pgfqpoint{5.079737in}{3.069764in}}%
\pgfpathlineto{\pgfqpoint{5.079737in}{3.112472in}}%
\pgfpathlineto{\pgfqpoint{5.085798in}{3.112472in}}%
\pgfpathlineto{\pgfqpoint{5.085798in}{3.082357in}}%
\pgfpathlineto{\pgfqpoint{5.091859in}{3.082357in}}%
\pgfpathlineto{\pgfqpoint{5.091859in}{3.091665in}}%
\pgfpathlineto{\pgfqpoint{5.097920in}{3.091665in}}%
\pgfpathlineto{\pgfqpoint{5.097920in}{3.075787in}}%
\pgfpathlineto{\pgfqpoint{5.103981in}{3.075787in}}%
\pgfpathlineto{\pgfqpoint{5.103981in}{3.097688in}}%
\pgfpathlineto{\pgfqpoint{5.110042in}{3.097688in}}%
\pgfpathlineto{\pgfqpoint{5.110042in}{3.090023in}}%
\pgfpathlineto{\pgfqpoint{5.116103in}{3.090023in}}%
\pgfpathlineto{\pgfqpoint{5.116103in}{3.082905in}}%
\pgfpathlineto{\pgfqpoint{5.122164in}{3.082905in}}%
\pgfpathlineto{\pgfqpoint{5.122164in}{3.073049in}}%
\pgfpathlineto{\pgfqpoint{5.128225in}{3.073049in}}%
\pgfpathlineto{\pgfqpoint{5.128225in}{3.099878in}}%
\pgfpathlineto{\pgfqpoint{5.134286in}{3.099878in}}%
\pgfpathlineto{\pgfqpoint{5.134286in}{3.088380in}}%
\pgfpathlineto{\pgfqpoint{5.140347in}{3.088380in}}%
\pgfpathlineto{\pgfqpoint{5.140347in}{3.073049in}}%
\pgfpathlineto{\pgfqpoint{5.146408in}{3.073049in}}%
\pgfpathlineto{\pgfqpoint{5.146408in}{3.082357in}}%
\pgfpathlineto{\pgfqpoint{5.152469in}{3.082357in}}%
\pgfpathlineto{\pgfqpoint{5.152469in}{3.086190in}}%
\pgfpathlineto{\pgfqpoint{5.158530in}{3.086190in}}%
\pgfpathlineto{\pgfqpoint{5.158530in}{3.083452in}}%
\pgfpathlineto{\pgfqpoint{5.164591in}{3.083452in}}%
\pgfpathlineto{\pgfqpoint{5.164591in}{3.077977in}}%
\pgfpathlineto{\pgfqpoint{5.170652in}{3.077977in}}%
\pgfpathlineto{\pgfqpoint{5.170652in}{3.082357in}}%
\pgfpathlineto{\pgfqpoint{5.182774in}{3.081810in}}%
\pgfpathlineto{\pgfqpoint{5.182774in}{3.079619in}}%
\pgfpathlineto{\pgfqpoint{5.188835in}{3.079619in}}%
\pgfpathlineto{\pgfqpoint{5.188835in}{3.085095in}}%
\pgfpathlineto{\pgfqpoint{5.194896in}{3.085095in}}%
\pgfpathlineto{\pgfqpoint{5.194896in}{3.067574in}}%
\pgfpathlineto{\pgfqpoint{5.200957in}{3.067574in}}%
\pgfpathlineto{\pgfqpoint{5.200957in}{3.073597in}}%
\pgfpathlineto{\pgfqpoint{5.207018in}{3.073597in}}%
\pgfpathlineto{\pgfqpoint{5.207018in}{3.075787in}}%
\pgfpathlineto{\pgfqpoint{5.213079in}{3.075787in}}%
\pgfpathlineto{\pgfqpoint{5.213079in}{3.081262in}}%
\pgfpathlineto{\pgfqpoint{5.219140in}{3.081262in}}%
\pgfpathlineto{\pgfqpoint{5.219140in}{3.066479in}}%
\pgfpathlineto{\pgfqpoint{5.225201in}{3.066479in}}%
\pgfpathlineto{\pgfqpoint{5.225201in}{3.074144in}}%
\pgfpathlineto{\pgfqpoint{5.231262in}{3.074144in}}%
\pgfpathlineto{\pgfqpoint{5.231262in}{3.059908in}}%
\pgfpathlineto{\pgfqpoint{5.237323in}{3.059908in}}%
\pgfpathlineto{\pgfqpoint{5.237323in}{3.073597in}}%
\pgfpathlineto{\pgfqpoint{5.243384in}{3.073597in}}%
\pgfpathlineto{\pgfqpoint{5.243384in}{3.080715in}}%
\pgfpathlineto{\pgfqpoint{5.249445in}{3.080715in}}%
\pgfpathlineto{\pgfqpoint{5.249445in}{3.070859in}}%
\pgfpathlineto{\pgfqpoint{5.255505in}{3.070859in}}%
\pgfpathlineto{\pgfqpoint{5.255505in}{3.084547in}}%
\pgfpathlineto{\pgfqpoint{5.261566in}{3.084547in}}%
\pgfpathlineto{\pgfqpoint{5.261566in}{3.071954in}}%
\pgfpathlineto{\pgfqpoint{5.267627in}{3.071954in}}%
\pgfpathlineto{\pgfqpoint{5.267627in}{3.077977in}}%
\pgfpathlineto{\pgfqpoint{5.273688in}{3.077977in}}%
\pgfpathlineto{\pgfqpoint{5.273688in}{3.074692in}}%
\pgfpathlineto{\pgfqpoint{5.285810in}{3.075239in}}%
\pgfpathlineto{\pgfqpoint{5.285810in}{3.058813in}}%
\pgfpathlineto{\pgfqpoint{5.291871in}{3.058813in}}%
\pgfpathlineto{\pgfqpoint{5.291871in}{3.068669in}}%
\pgfpathlineto{\pgfqpoint{5.297932in}{3.068669in}}%
\pgfpathlineto{\pgfqpoint{5.297932in}{3.076334in}}%
\pgfpathlineto{\pgfqpoint{5.303993in}{3.076334in}}%
\pgfpathlineto{\pgfqpoint{5.303993in}{3.063193in}}%
\pgfpathlineto{\pgfqpoint{5.310054in}{3.063193in}}%
\pgfpathlineto{\pgfqpoint{5.310054in}{3.074144in}}%
\pgfpathlineto{\pgfqpoint{5.316115in}{3.074144in}}%
\pgfpathlineto{\pgfqpoint{5.316115in}{3.059361in}}%
\pgfpathlineto{\pgfqpoint{5.322176in}{3.059361in}}%
\pgfpathlineto{\pgfqpoint{5.322176in}{3.077977in}}%
\pgfpathlineto{\pgfqpoint{5.328237in}{3.077977in}}%
\pgfpathlineto{\pgfqpoint{5.328237in}{3.062098in}}%
\pgfpathlineto{\pgfqpoint{5.334298in}{3.062098in}}%
\pgfpathlineto{\pgfqpoint{5.334298in}{3.079619in}}%
\pgfpathlineto{\pgfqpoint{5.340359in}{3.079619in}}%
\pgfpathlineto{\pgfqpoint{5.340359in}{3.052790in}}%
\pgfpathlineto{\pgfqpoint{5.346420in}{3.052790in}}%
\pgfpathlineto{\pgfqpoint{5.346420in}{3.089475in}}%
\pgfpathlineto{\pgfqpoint{5.352481in}{3.089475in}}%
\pgfpathlineto{\pgfqpoint{5.352481in}{3.086190in}}%
\pgfpathlineto{\pgfqpoint{5.358542in}{3.086190in}}%
\pgfpathlineto{\pgfqpoint{5.358542in}{3.062098in}}%
\pgfpathlineto{\pgfqpoint{5.364603in}{3.062098in}}%
\pgfpathlineto{\pgfqpoint{5.364603in}{3.072502in}}%
\pgfpathlineto{\pgfqpoint{5.370664in}{3.072502in}}%
\pgfpathlineto{\pgfqpoint{5.370664in}{3.057718in}}%
\pgfpathlineto{\pgfqpoint{5.376725in}{3.057718in}}%
\pgfpathlineto{\pgfqpoint{5.376725in}{3.062646in}}%
\pgfpathlineto{\pgfqpoint{5.382786in}{3.062646in}}%
\pgfpathlineto{\pgfqpoint{5.382786in}{3.067026in}}%
\pgfpathlineto{\pgfqpoint{5.388847in}{3.067026in}}%
\pgfpathlineto{\pgfqpoint{5.388847in}{3.074692in}}%
\pgfpathlineto{\pgfqpoint{5.394908in}{3.074692in}}%
\pgfpathlineto{\pgfqpoint{5.394908in}{3.043482in}}%
\pgfpathlineto{\pgfqpoint{5.400969in}{3.043482in}}%
\pgfpathlineto{\pgfqpoint{5.400969in}{3.061551in}}%
\pgfpathlineto{\pgfqpoint{5.407030in}{3.061551in}}%
\pgfpathlineto{\pgfqpoint{5.407030in}{3.056623in}}%
\pgfpathlineto{\pgfqpoint{5.413091in}{3.056623in}}%
\pgfpathlineto{\pgfqpoint{5.413091in}{3.044577in}}%
\pgfpathlineto{\pgfqpoint{5.419152in}{3.044577in}}%
\pgfpathlineto{\pgfqpoint{5.419152in}{3.066479in}}%
\pgfpathlineto{\pgfqpoint{5.425213in}{3.066479in}}%
\pgfpathlineto{\pgfqpoint{5.425213in}{3.055528in}}%
\pgfpathlineto{\pgfqpoint{5.431274in}{3.055528in}}%
\pgfpathlineto{\pgfqpoint{5.431274in}{3.065931in}}%
\pgfpathlineto{\pgfqpoint{5.437335in}{3.065931in}}%
\pgfpathlineto{\pgfqpoint{5.437335in}{3.050053in}}%
\pgfpathlineto{\pgfqpoint{5.443396in}{3.050053in}}%
\pgfpathlineto{\pgfqpoint{5.443396in}{3.069764in}}%
\pgfpathlineto{\pgfqpoint{5.449457in}{3.069764in}}%
\pgfpathlineto{\pgfqpoint{5.449457in}{3.045672in}}%
\pgfpathlineto{\pgfqpoint{5.455518in}{3.045672in}}%
\pgfpathlineto{\pgfqpoint{5.455518in}{3.056623in}}%
\pgfpathlineto{\pgfqpoint{5.461579in}{3.056623in}}%
\pgfpathlineto{\pgfqpoint{5.461579in}{3.047862in}}%
\pgfpathlineto{\pgfqpoint{5.467640in}{3.047862in}}%
\pgfpathlineto{\pgfqpoint{5.467640in}{3.072502in}}%
\pgfpathlineto{\pgfqpoint{5.473701in}{3.072502in}}%
\pgfpathlineto{\pgfqpoint{5.473701in}{3.053885in}}%
\pgfpathlineto{\pgfqpoint{5.479762in}{3.053885in}}%
\pgfpathlineto{\pgfqpoint{5.479762in}{3.049505in}}%
\pgfpathlineto{\pgfqpoint{5.485823in}{3.049505in}}%
\pgfpathlineto{\pgfqpoint{5.485823in}{3.052790in}}%
\pgfpathlineto{\pgfqpoint{5.491884in}{3.052790in}}%
\pgfpathlineto{\pgfqpoint{5.491884in}{3.054980in}}%
\pgfpathlineto{\pgfqpoint{5.497945in}{3.054980in}}%
\pgfpathlineto{\pgfqpoint{5.497945in}{3.069216in}}%
\pgfpathlineto{\pgfqpoint{5.504006in}{3.069216in}}%
\pgfpathlineto{\pgfqpoint{5.504006in}{3.039102in}}%
\pgfpathlineto{\pgfqpoint{5.510067in}{3.039102in}}%
\pgfpathlineto{\pgfqpoint{5.510067in}{3.067574in}}%
\pgfpathlineto{\pgfqpoint{5.516128in}{3.067574in}}%
\pgfpathlineto{\pgfqpoint{5.516128in}{3.059361in}}%
\pgfpathlineto{\pgfqpoint{5.522189in}{3.059361in}}%
\pgfpathlineto{\pgfqpoint{5.522189in}{3.056075in}}%
\pgfpathlineto{\pgfqpoint{5.528250in}{3.056075in}}%
\pgfpathlineto{\pgfqpoint{5.528250in}{3.062646in}}%
\pgfpathlineto{\pgfqpoint{5.534311in}{3.062646in}}%
\pgfpathlineto{\pgfqpoint{5.534311in}{3.047862in}}%
\pgfpathlineto{\pgfqpoint{5.540372in}{3.047862in}}%
\pgfpathlineto{\pgfqpoint{5.540372in}{3.054980in}}%
\pgfpathlineto{\pgfqpoint{5.546432in}{3.054980in}}%
\pgfpathlineto{\pgfqpoint{5.546432in}{3.051695in}}%
\pgfpathlineto{\pgfqpoint{5.552493in}{3.051695in}}%
\pgfpathlineto{\pgfqpoint{5.552493in}{3.055528in}}%
\pgfpathlineto{\pgfqpoint{5.558554in}{3.055528in}}%
\pgfpathlineto{\pgfqpoint{5.558554in}{3.040744in}}%
\pgfpathlineto{\pgfqpoint{5.564615in}{3.040744in}}%
\pgfpathlineto{\pgfqpoint{5.564615in}{3.061003in}}%
\pgfpathlineto{\pgfqpoint{5.570676in}{3.061003in}}%
\pgfpathlineto{\pgfqpoint{5.570676in}{3.036364in}}%
\pgfpathlineto{\pgfqpoint{5.576737in}{3.036364in}}%
\pgfpathlineto{\pgfqpoint{5.576737in}{3.055528in}}%
\pgfpathlineto{\pgfqpoint{5.582798in}{3.055528in}}%
\pgfpathlineto{\pgfqpoint{5.582798in}{3.051148in}}%
\pgfpathlineto{\pgfqpoint{5.600981in}{3.051148in}}%
\pgfpathlineto{\pgfqpoint{5.600981in}{3.047862in}}%
\pgfpathlineto{\pgfqpoint{5.607042in}{3.047862in}}%
\pgfpathlineto{\pgfqpoint{5.607042in}{3.063193in}}%
\pgfpathlineto{\pgfqpoint{5.613103in}{3.063193in}}%
\pgfpathlineto{\pgfqpoint{5.613103in}{3.051148in}}%
\pgfpathlineto{\pgfqpoint{5.619164in}{3.051148in}}%
\pgfpathlineto{\pgfqpoint{5.619164in}{3.057718in}}%
\pgfpathlineto{\pgfqpoint{5.625225in}{3.057718in}}%
\pgfpathlineto{\pgfqpoint{5.625225in}{3.047315in}}%
\pgfpathlineto{\pgfqpoint{5.631286in}{3.047315in}}%
\pgfpathlineto{\pgfqpoint{5.631286in}{3.042387in}}%
\pgfpathlineto{\pgfqpoint{5.637347in}{3.042387in}}%
\pgfpathlineto{\pgfqpoint{5.637347in}{3.052243in}}%
\pgfpathlineto{\pgfqpoint{5.643408in}{3.052243in}}%
\pgfpathlineto{\pgfqpoint{5.643408in}{3.040197in}}%
\pgfpathlineto{\pgfqpoint{5.649469in}{3.040197in}}%
\pgfpathlineto{\pgfqpoint{5.649469in}{3.048410in}}%
\pgfpathlineto{\pgfqpoint{5.655530in}{3.048410in}}%
\pgfpathlineto{\pgfqpoint{5.655530in}{3.036364in}}%
\pgfpathlineto{\pgfqpoint{5.661591in}{3.036364in}}%
\pgfpathlineto{\pgfqpoint{5.661591in}{3.062646in}}%
\pgfpathlineto{\pgfqpoint{5.667652in}{3.062646in}}%
\pgfpathlineto{\pgfqpoint{5.667652in}{3.037459in}}%
\pgfpathlineto{\pgfqpoint{5.673713in}{3.037459in}}%
\pgfpathlineto{\pgfqpoint{5.673713in}{3.055528in}}%
\pgfpathlineto{\pgfqpoint{5.679774in}{3.055528in}}%
\pgfpathlineto{\pgfqpoint{5.679774in}{3.025961in}}%
\pgfpathlineto{\pgfqpoint{5.685835in}{3.025961in}}%
\pgfpathlineto{\pgfqpoint{5.685835in}{3.049505in}}%
\pgfpathlineto{\pgfqpoint{5.691896in}{3.049505in}}%
\pgfpathlineto{\pgfqpoint{5.691896in}{3.052243in}}%
\pgfpathlineto{\pgfqpoint{5.697957in}{3.052243in}}%
\pgfpathlineto{\pgfqpoint{5.697957in}{3.037459in}}%
\pgfpathlineto{\pgfqpoint{5.704018in}{3.037459in}}%
\pgfpathlineto{\pgfqpoint{5.704018in}{3.046767in}}%
\pgfpathlineto{\pgfqpoint{5.710079in}{3.046767in}}%
\pgfpathlineto{\pgfqpoint{5.710079in}{3.042387in}}%
\pgfpathlineto{\pgfqpoint{5.716140in}{3.042387in}}%
\pgfpathlineto{\pgfqpoint{5.716140in}{3.046767in}}%
\pgfpathlineto{\pgfqpoint{5.722201in}{3.046767in}}%
\pgfpathlineto{\pgfqpoint{5.722201in}{3.035817in}}%
\pgfpathlineto{\pgfqpoint{5.728262in}{3.035817in}}%
\pgfpathlineto{\pgfqpoint{5.728262in}{3.048957in}}%
\pgfpathlineto{\pgfqpoint{5.734323in}{3.048957in}}%
\pgfpathlineto{\pgfqpoint{5.734323in}{3.031436in}}%
\pgfpathlineto{\pgfqpoint{5.740384in}{3.031436in}}%
\pgfpathlineto{\pgfqpoint{5.740384in}{3.045672in}}%
\pgfpathlineto{\pgfqpoint{5.746445in}{3.045672in}}%
\pgfpathlineto{\pgfqpoint{5.746445in}{3.041292in}}%
\pgfpathlineto{\pgfqpoint{5.758567in}{3.040744in}}%
\pgfpathlineto{\pgfqpoint{5.758567in}{3.044030in}}%
\pgfpathlineto{\pgfqpoint{5.764628in}{3.044030in}}%
\pgfpathlineto{\pgfqpoint{5.764628in}{3.040744in}}%
\pgfpathlineto{\pgfqpoint{5.770689in}{3.040744in}}%
\pgfpathlineto{\pgfqpoint{5.770689in}{3.046220in}}%
\pgfpathlineto{\pgfqpoint{5.776750in}{3.046220in}}%
\pgfpathlineto{\pgfqpoint{5.776750in}{3.035817in}}%
\pgfpathlineto{\pgfqpoint{5.782811in}{3.035817in}}%
\pgfpathlineto{\pgfqpoint{5.782811in}{3.047315in}}%
\pgfpathlineto{\pgfqpoint{5.788872in}{3.047315in}}%
\pgfpathlineto{\pgfqpoint{5.788872in}{3.034174in}}%
\pgfpathlineto{\pgfqpoint{5.794933in}{3.034174in}}%
\pgfpathlineto{\pgfqpoint{5.794933in}{3.035817in}}%
\pgfpathlineto{\pgfqpoint{5.807055in}{3.034722in}}%
\pgfpathlineto{\pgfqpoint{5.807055in}{3.034722in}}%
\pgfusepath{stroke}%
\end{pgfscope}%
\begin{pgfscope}%
\pgfsetrectcap%
\pgfsetmiterjoin%
\pgfsetlinewidth{0.803000pt}%
\definecolor{currentstroke}{rgb}{0.000000,0.000000,0.000000}%
\pgfsetstrokecolor{currentstroke}%
\pgfsetdash{}{0pt}%
\pgfpathmoveto{\pgfqpoint{0.781944in}{2.977778in}}%
\pgfpathlineto{\pgfqpoint{0.781944in}{4.627778in}}%
\pgfusepath{stroke}%
\end{pgfscope}%
\begin{pgfscope}%
\pgfsetrectcap%
\pgfsetmiterjoin%
\pgfsetlinewidth{0.803000pt}%
\definecolor{currentstroke}{rgb}{0.000000,0.000000,0.000000}%
\pgfsetstrokecolor{currentstroke}%
\pgfsetdash{}{0pt}%
\pgfpathmoveto{\pgfqpoint{5.801389in}{2.977778in}}%
\pgfpathlineto{\pgfqpoint{5.801389in}{4.627778in}}%
\pgfusepath{stroke}%
\end{pgfscope}%
\begin{pgfscope}%
\pgfsetrectcap%
\pgfsetmiterjoin%
\pgfsetlinewidth{0.803000pt}%
\definecolor{currentstroke}{rgb}{0.000000,0.000000,0.000000}%
\pgfsetstrokecolor{currentstroke}%
\pgfsetdash{}{0pt}%
\pgfpathmoveto{\pgfqpoint{0.781944in}{2.977778in}}%
\pgfpathlineto{\pgfqpoint{5.801389in}{2.977778in}}%
\pgfusepath{stroke}%
\end{pgfscope}%
\begin{pgfscope}%
\pgfsetrectcap%
\pgfsetmiterjoin%
\pgfsetlinewidth{0.803000pt}%
\definecolor{currentstroke}{rgb}{0.000000,0.000000,0.000000}%
\pgfsetstrokecolor{currentstroke}%
\pgfsetdash{}{0pt}%
\pgfpathmoveto{\pgfqpoint{0.781944in}{4.627778in}}%
\pgfpathlineto{\pgfqpoint{5.801389in}{4.627778in}}%
\pgfusepath{stroke}%
\end{pgfscope}%
\begin{pgfscope}%
\definecolor{textcolor}{rgb}{0.000000,0.000000,0.000000}%
\pgfsetstrokecolor{textcolor}%
\pgfsetfillcolor{textcolor}%
\pgftext[x=3.291667in,y=4.711111in,,base]{\color{textcolor}\rmfamily\fontsize{12.000000}{14.400000}\selectfont Koinzidenzzeitspektrum}%
\end{pgfscope}%
\begin{pgfscope}%
\pgfsetbuttcap%
\pgfsetmiterjoin%
\definecolor{currentfill}{rgb}{1.000000,1.000000,1.000000}%
\pgfsetfillcolor{currentfill}%
\pgfsetfillopacity{0.800000}%
\pgfsetlinewidth{1.003750pt}%
\definecolor{currentstroke}{rgb}{0.800000,0.800000,0.800000}%
\pgfsetstrokecolor{currentstroke}%
\pgfsetstrokeopacity{0.800000}%
\pgfsetdash{}{0pt}%
\pgfpathmoveto{\pgfqpoint{4.855972in}{3.935834in}}%
\pgfpathlineto{\pgfqpoint{5.704167in}{3.935834in}}%
\pgfpathquadraticcurveto{\pgfqpoint{5.731944in}{3.935834in}}{\pgfqpoint{5.731944in}{3.963612in}}%
\pgfpathlineto{\pgfqpoint{5.731944in}{4.530556in}}%
\pgfpathquadraticcurveto{\pgfqpoint{5.731944in}{4.558333in}}{\pgfqpoint{5.704167in}{4.558333in}}%
\pgfpathlineto{\pgfqpoint{4.855972in}{4.558333in}}%
\pgfpathquadraticcurveto{\pgfqpoint{4.828194in}{4.558333in}}{\pgfqpoint{4.828194in}{4.530556in}}%
\pgfpathlineto{\pgfqpoint{4.828194in}{3.963612in}}%
\pgfpathquadraticcurveto{\pgfqpoint{4.828194in}{3.935834in}}{\pgfqpoint{4.855972in}{3.935834in}}%
\pgfpathclose%
\pgfusepath{stroke,fill}%
\end{pgfscope}%
\begin{pgfscope}%
\pgfsetrectcap%
\pgfsetroundjoin%
\pgfsetlinewidth{1.505625pt}%
\definecolor{currentstroke}{rgb}{0.121569,0.466667,0.705882}%
\pgfsetstrokecolor{currentstroke}%
\pgfsetdash{}{0pt}%
\pgfpathmoveto{\pgfqpoint{4.883750in}{4.454167in}}%
\pgfpathlineto{\pgfqpoint{5.161528in}{4.454167in}}%
\pgfusepath{stroke}%
\end{pgfscope}%
\begin{pgfscope}%
\definecolor{textcolor}{rgb}{0.000000,0.000000,0.000000}%
\pgfsetstrokecolor{textcolor}%
\pgfsetfillcolor{textcolor}%
\pgftext[x=5.272639in,y=4.405556in,left,base]{\color{textcolor}\rmfamily\fontsize{10.000000}{12.000000}\selectfont Mitte}%
\end{pgfscope}%
\begin{pgfscope}%
\pgfsetrectcap%
\pgfsetroundjoin%
\pgfsetlinewidth{1.505625pt}%
\definecolor{currentstroke}{rgb}{1.000000,0.498039,0.054902}%
\pgfsetstrokecolor{currentstroke}%
\pgfsetdash{}{0pt}%
\pgfpathmoveto{\pgfqpoint{4.883750in}{4.260556in}}%
\pgfpathlineto{\pgfqpoint{5.161528in}{4.260556in}}%
\pgfusepath{stroke}%
\end{pgfscope}%
\begin{pgfscope}%
\definecolor{textcolor}{rgb}{0.000000,0.000000,0.000000}%
\pgfsetstrokecolor{textcolor}%
\pgfsetfillcolor{textcolor}%
\pgftext[x=5.272639in,y=4.211945in,left,base]{\color{textcolor}\rmfamily\fontsize{10.000000}{12.000000}\selectfont Links}%
\end{pgfscope}%
\begin{pgfscope}%
\pgfsetrectcap%
\pgfsetroundjoin%
\pgfsetlinewidth{1.505625pt}%
\definecolor{currentstroke}{rgb}{0.172549,0.627451,0.172549}%
\pgfsetstrokecolor{currentstroke}%
\pgfsetdash{}{0pt}%
\pgfpathmoveto{\pgfqpoint{4.883750in}{4.066945in}}%
\pgfpathlineto{\pgfqpoint{5.161528in}{4.066945in}}%
\pgfusepath{stroke}%
\end{pgfscope}%
\begin{pgfscope}%
\definecolor{textcolor}{rgb}{0.000000,0.000000,0.000000}%
\pgfsetstrokecolor{textcolor}%
\pgfsetfillcolor{textcolor}%
\pgftext[x=5.272639in,y=4.018334in,left,base]{\color{textcolor}\rmfamily\fontsize{10.000000}{12.000000}\selectfont Rechts}%
\end{pgfscope}%
\begin{pgfscope}%
\pgfsetbuttcap%
\pgfsetmiterjoin%
\definecolor{currentfill}{rgb}{1.000000,1.000000,1.000000}%
\pgfsetfillcolor{currentfill}%
\pgfsetlinewidth{0.000000pt}%
\definecolor{currentstroke}{rgb}{0.000000,0.000000,0.000000}%
\pgfsetstrokecolor{currentstroke}%
\pgfsetstrokeopacity{0.000000}%
\pgfsetdash{}{0pt}%
\pgfpathmoveto{\pgfqpoint{0.781944in}{0.552778in}}%
\pgfpathlineto{\pgfqpoint{2.920660in}{0.552778in}}%
\pgfpathlineto{\pgfqpoint{2.920660in}{2.202778in}}%
\pgfpathlineto{\pgfqpoint{0.781944in}{2.202778in}}%
\pgfpathclose%
\pgfusepath{fill}%
\end{pgfscope}%
\begin{pgfscope}%
\pgfpathrectangle{\pgfqpoint{0.781944in}{0.552778in}}{\pgfqpoint{2.138715in}{1.650000in}}%
\pgfusepath{clip}%
\pgfsetrectcap%
\pgfsetroundjoin%
\pgfsetlinewidth{0.803000pt}%
\definecolor{currentstroke}{rgb}{0.690196,0.690196,0.690196}%
\pgfsetstrokecolor{currentstroke}%
\pgfsetstrokeopacity{0.800000}%
\pgfsetdash{}{0pt}%
\pgfpathmoveto{\pgfqpoint{0.995816in}{0.552778in}}%
\pgfpathlineto{\pgfqpoint{0.995816in}{2.202778in}}%
\pgfusepath{stroke}%
\end{pgfscope}%
\begin{pgfscope}%
\pgfsetbuttcap%
\pgfsetroundjoin%
\definecolor{currentfill}{rgb}{0.000000,0.000000,0.000000}%
\pgfsetfillcolor{currentfill}%
\pgfsetlinewidth{0.803000pt}%
\definecolor{currentstroke}{rgb}{0.000000,0.000000,0.000000}%
\pgfsetstrokecolor{currentstroke}%
\pgfsetdash{}{0pt}%
\pgfsys@defobject{currentmarker}{\pgfqpoint{0.000000in}{-0.048611in}}{\pgfqpoint{0.000000in}{0.000000in}}{%
\pgfpathmoveto{\pgfqpoint{0.000000in}{0.000000in}}%
\pgfpathlineto{\pgfqpoint{0.000000in}{-0.048611in}}%
\pgfusepath{stroke,fill}%
}%
\begin{pgfscope}%
\pgfsys@transformshift{0.995816in}{0.552778in}%
\pgfsys@useobject{currentmarker}{}%
\end{pgfscope}%
\end{pgfscope}%
\begin{pgfscope}%
\pgfsetbuttcap%
\pgfsetroundjoin%
\definecolor{currentfill}{rgb}{0.000000,0.000000,0.000000}%
\pgfsetfillcolor{currentfill}%
\pgfsetlinewidth{0.803000pt}%
\definecolor{currentstroke}{rgb}{0.000000,0.000000,0.000000}%
\pgfsetstrokecolor{currentstroke}%
\pgfsetdash{}{0pt}%
\pgfsys@defobject{currentmarker}{\pgfqpoint{0.000000in}{0.000000in}}{\pgfqpoint{0.000000in}{0.048611in}}{%
\pgfpathmoveto{\pgfqpoint{0.000000in}{0.000000in}}%
\pgfpathlineto{\pgfqpoint{0.000000in}{0.048611in}}%
\pgfusepath{stroke,fill}%
}%
\begin{pgfscope}%
\pgfsys@transformshift{0.995816in}{2.202778in}%
\pgfsys@useobject{currentmarker}{}%
\end{pgfscope}%
\end{pgfscope}%
\begin{pgfscope}%
\definecolor{textcolor}{rgb}{0.000000,0.000000,0.000000}%
\pgfsetstrokecolor{textcolor}%
\pgfsetfillcolor{textcolor}%
\pgftext[x=0.995816in,y=0.455556in,,top]{\color{textcolor}\rmfamily\fontsize{10.000000}{12.000000}\selectfont 200}%
\end{pgfscope}%
\begin{pgfscope}%
\pgfpathrectangle{\pgfqpoint{0.781944in}{0.552778in}}{\pgfqpoint{2.138715in}{1.650000in}}%
\pgfusepath{clip}%
\pgfsetrectcap%
\pgfsetroundjoin%
\pgfsetlinewidth{0.803000pt}%
\definecolor{currentstroke}{rgb}{0.690196,0.690196,0.690196}%
\pgfsetstrokecolor{currentstroke}%
\pgfsetstrokeopacity{0.800000}%
\pgfsetdash{}{0pt}%
\pgfpathmoveto{\pgfqpoint{1.423559in}{0.552778in}}%
\pgfpathlineto{\pgfqpoint{1.423559in}{2.202778in}}%
\pgfusepath{stroke}%
\end{pgfscope}%
\begin{pgfscope}%
\pgfsetbuttcap%
\pgfsetroundjoin%
\definecolor{currentfill}{rgb}{0.000000,0.000000,0.000000}%
\pgfsetfillcolor{currentfill}%
\pgfsetlinewidth{0.803000pt}%
\definecolor{currentstroke}{rgb}{0.000000,0.000000,0.000000}%
\pgfsetstrokecolor{currentstroke}%
\pgfsetdash{}{0pt}%
\pgfsys@defobject{currentmarker}{\pgfqpoint{0.000000in}{-0.048611in}}{\pgfqpoint{0.000000in}{0.000000in}}{%
\pgfpathmoveto{\pgfqpoint{0.000000in}{0.000000in}}%
\pgfpathlineto{\pgfqpoint{0.000000in}{-0.048611in}}%
\pgfusepath{stroke,fill}%
}%
\begin{pgfscope}%
\pgfsys@transformshift{1.423559in}{0.552778in}%
\pgfsys@useobject{currentmarker}{}%
\end{pgfscope}%
\end{pgfscope}%
\begin{pgfscope}%
\pgfsetbuttcap%
\pgfsetroundjoin%
\definecolor{currentfill}{rgb}{0.000000,0.000000,0.000000}%
\pgfsetfillcolor{currentfill}%
\pgfsetlinewidth{0.803000pt}%
\definecolor{currentstroke}{rgb}{0.000000,0.000000,0.000000}%
\pgfsetstrokecolor{currentstroke}%
\pgfsetdash{}{0pt}%
\pgfsys@defobject{currentmarker}{\pgfqpoint{0.000000in}{0.000000in}}{\pgfqpoint{0.000000in}{0.048611in}}{%
\pgfpathmoveto{\pgfqpoint{0.000000in}{0.000000in}}%
\pgfpathlineto{\pgfqpoint{0.000000in}{0.048611in}}%
\pgfusepath{stroke,fill}%
}%
\begin{pgfscope}%
\pgfsys@transformshift{1.423559in}{2.202778in}%
\pgfsys@useobject{currentmarker}{}%
\end{pgfscope}%
\end{pgfscope}%
\begin{pgfscope}%
\definecolor{textcolor}{rgb}{0.000000,0.000000,0.000000}%
\pgfsetstrokecolor{textcolor}%
\pgfsetfillcolor{textcolor}%
\pgftext[x=1.423559in,y=0.455556in,,top]{\color{textcolor}\rmfamily\fontsize{10.000000}{12.000000}\selectfont 400}%
\end{pgfscope}%
\begin{pgfscope}%
\pgfpathrectangle{\pgfqpoint{0.781944in}{0.552778in}}{\pgfqpoint{2.138715in}{1.650000in}}%
\pgfusepath{clip}%
\pgfsetrectcap%
\pgfsetroundjoin%
\pgfsetlinewidth{0.803000pt}%
\definecolor{currentstroke}{rgb}{0.690196,0.690196,0.690196}%
\pgfsetstrokecolor{currentstroke}%
\pgfsetstrokeopacity{0.800000}%
\pgfsetdash{}{0pt}%
\pgfpathmoveto{\pgfqpoint{1.851302in}{0.552778in}}%
\pgfpathlineto{\pgfqpoint{1.851302in}{2.202778in}}%
\pgfusepath{stroke}%
\end{pgfscope}%
\begin{pgfscope}%
\pgfsetbuttcap%
\pgfsetroundjoin%
\definecolor{currentfill}{rgb}{0.000000,0.000000,0.000000}%
\pgfsetfillcolor{currentfill}%
\pgfsetlinewidth{0.803000pt}%
\definecolor{currentstroke}{rgb}{0.000000,0.000000,0.000000}%
\pgfsetstrokecolor{currentstroke}%
\pgfsetdash{}{0pt}%
\pgfsys@defobject{currentmarker}{\pgfqpoint{0.000000in}{-0.048611in}}{\pgfqpoint{0.000000in}{0.000000in}}{%
\pgfpathmoveto{\pgfqpoint{0.000000in}{0.000000in}}%
\pgfpathlineto{\pgfqpoint{0.000000in}{-0.048611in}}%
\pgfusepath{stroke,fill}%
}%
\begin{pgfscope}%
\pgfsys@transformshift{1.851302in}{0.552778in}%
\pgfsys@useobject{currentmarker}{}%
\end{pgfscope}%
\end{pgfscope}%
\begin{pgfscope}%
\pgfsetbuttcap%
\pgfsetroundjoin%
\definecolor{currentfill}{rgb}{0.000000,0.000000,0.000000}%
\pgfsetfillcolor{currentfill}%
\pgfsetlinewidth{0.803000pt}%
\definecolor{currentstroke}{rgb}{0.000000,0.000000,0.000000}%
\pgfsetstrokecolor{currentstroke}%
\pgfsetdash{}{0pt}%
\pgfsys@defobject{currentmarker}{\pgfqpoint{0.000000in}{0.000000in}}{\pgfqpoint{0.000000in}{0.048611in}}{%
\pgfpathmoveto{\pgfqpoint{0.000000in}{0.000000in}}%
\pgfpathlineto{\pgfqpoint{0.000000in}{0.048611in}}%
\pgfusepath{stroke,fill}%
}%
\begin{pgfscope}%
\pgfsys@transformshift{1.851302in}{2.202778in}%
\pgfsys@useobject{currentmarker}{}%
\end{pgfscope}%
\end{pgfscope}%
\begin{pgfscope}%
\definecolor{textcolor}{rgb}{0.000000,0.000000,0.000000}%
\pgfsetstrokecolor{textcolor}%
\pgfsetfillcolor{textcolor}%
\pgftext[x=1.851302in,y=0.455556in,,top]{\color{textcolor}\rmfamily\fontsize{10.000000}{12.000000}\selectfont 600}%
\end{pgfscope}%
\begin{pgfscope}%
\pgfpathrectangle{\pgfqpoint{0.781944in}{0.552778in}}{\pgfqpoint{2.138715in}{1.650000in}}%
\pgfusepath{clip}%
\pgfsetrectcap%
\pgfsetroundjoin%
\pgfsetlinewidth{0.803000pt}%
\definecolor{currentstroke}{rgb}{0.690196,0.690196,0.690196}%
\pgfsetstrokecolor{currentstroke}%
\pgfsetstrokeopacity{0.800000}%
\pgfsetdash{}{0pt}%
\pgfpathmoveto{\pgfqpoint{2.279045in}{0.552778in}}%
\pgfpathlineto{\pgfqpoint{2.279045in}{2.202778in}}%
\pgfusepath{stroke}%
\end{pgfscope}%
\begin{pgfscope}%
\pgfsetbuttcap%
\pgfsetroundjoin%
\definecolor{currentfill}{rgb}{0.000000,0.000000,0.000000}%
\pgfsetfillcolor{currentfill}%
\pgfsetlinewidth{0.803000pt}%
\definecolor{currentstroke}{rgb}{0.000000,0.000000,0.000000}%
\pgfsetstrokecolor{currentstroke}%
\pgfsetdash{}{0pt}%
\pgfsys@defobject{currentmarker}{\pgfqpoint{0.000000in}{-0.048611in}}{\pgfqpoint{0.000000in}{0.000000in}}{%
\pgfpathmoveto{\pgfqpoint{0.000000in}{0.000000in}}%
\pgfpathlineto{\pgfqpoint{0.000000in}{-0.048611in}}%
\pgfusepath{stroke,fill}%
}%
\begin{pgfscope}%
\pgfsys@transformshift{2.279045in}{0.552778in}%
\pgfsys@useobject{currentmarker}{}%
\end{pgfscope}%
\end{pgfscope}%
\begin{pgfscope}%
\pgfsetbuttcap%
\pgfsetroundjoin%
\definecolor{currentfill}{rgb}{0.000000,0.000000,0.000000}%
\pgfsetfillcolor{currentfill}%
\pgfsetlinewidth{0.803000pt}%
\definecolor{currentstroke}{rgb}{0.000000,0.000000,0.000000}%
\pgfsetstrokecolor{currentstroke}%
\pgfsetdash{}{0pt}%
\pgfsys@defobject{currentmarker}{\pgfqpoint{0.000000in}{0.000000in}}{\pgfqpoint{0.000000in}{0.048611in}}{%
\pgfpathmoveto{\pgfqpoint{0.000000in}{0.000000in}}%
\pgfpathlineto{\pgfqpoint{0.000000in}{0.048611in}}%
\pgfusepath{stroke,fill}%
}%
\begin{pgfscope}%
\pgfsys@transformshift{2.279045in}{2.202778in}%
\pgfsys@useobject{currentmarker}{}%
\end{pgfscope}%
\end{pgfscope}%
\begin{pgfscope}%
\definecolor{textcolor}{rgb}{0.000000,0.000000,0.000000}%
\pgfsetstrokecolor{textcolor}%
\pgfsetfillcolor{textcolor}%
\pgftext[x=2.279045in,y=0.455556in,,top]{\color{textcolor}\rmfamily\fontsize{10.000000}{12.000000}\selectfont 800}%
\end{pgfscope}%
\begin{pgfscope}%
\pgfpathrectangle{\pgfqpoint{0.781944in}{0.552778in}}{\pgfqpoint{2.138715in}{1.650000in}}%
\pgfusepath{clip}%
\pgfsetrectcap%
\pgfsetroundjoin%
\pgfsetlinewidth{0.803000pt}%
\definecolor{currentstroke}{rgb}{0.690196,0.690196,0.690196}%
\pgfsetstrokecolor{currentstroke}%
\pgfsetstrokeopacity{0.800000}%
\pgfsetdash{}{0pt}%
\pgfpathmoveto{\pgfqpoint{2.706788in}{0.552778in}}%
\pgfpathlineto{\pgfqpoint{2.706788in}{2.202778in}}%
\pgfusepath{stroke}%
\end{pgfscope}%
\begin{pgfscope}%
\pgfsetbuttcap%
\pgfsetroundjoin%
\definecolor{currentfill}{rgb}{0.000000,0.000000,0.000000}%
\pgfsetfillcolor{currentfill}%
\pgfsetlinewidth{0.803000pt}%
\definecolor{currentstroke}{rgb}{0.000000,0.000000,0.000000}%
\pgfsetstrokecolor{currentstroke}%
\pgfsetdash{}{0pt}%
\pgfsys@defobject{currentmarker}{\pgfqpoint{0.000000in}{-0.048611in}}{\pgfqpoint{0.000000in}{0.000000in}}{%
\pgfpathmoveto{\pgfqpoint{0.000000in}{0.000000in}}%
\pgfpathlineto{\pgfqpoint{0.000000in}{-0.048611in}}%
\pgfusepath{stroke,fill}%
}%
\begin{pgfscope}%
\pgfsys@transformshift{2.706788in}{0.552778in}%
\pgfsys@useobject{currentmarker}{}%
\end{pgfscope}%
\end{pgfscope}%
\begin{pgfscope}%
\pgfsetbuttcap%
\pgfsetroundjoin%
\definecolor{currentfill}{rgb}{0.000000,0.000000,0.000000}%
\pgfsetfillcolor{currentfill}%
\pgfsetlinewidth{0.803000pt}%
\definecolor{currentstroke}{rgb}{0.000000,0.000000,0.000000}%
\pgfsetstrokecolor{currentstroke}%
\pgfsetdash{}{0pt}%
\pgfsys@defobject{currentmarker}{\pgfqpoint{0.000000in}{0.000000in}}{\pgfqpoint{0.000000in}{0.048611in}}{%
\pgfpathmoveto{\pgfqpoint{0.000000in}{0.000000in}}%
\pgfpathlineto{\pgfqpoint{0.000000in}{0.048611in}}%
\pgfusepath{stroke,fill}%
}%
\begin{pgfscope}%
\pgfsys@transformshift{2.706788in}{2.202778in}%
\pgfsys@useobject{currentmarker}{}%
\end{pgfscope}%
\end{pgfscope}%
\begin{pgfscope}%
\definecolor{textcolor}{rgb}{0.000000,0.000000,0.000000}%
\pgfsetstrokecolor{textcolor}%
\pgfsetfillcolor{textcolor}%
\pgftext[x=2.706788in,y=0.455556in,,top]{\color{textcolor}\rmfamily\fontsize{10.000000}{12.000000}\selectfont 1000}%
\end{pgfscope}%
\begin{pgfscope}%
\pgfpathrectangle{\pgfqpoint{0.781944in}{0.552778in}}{\pgfqpoint{2.138715in}{1.650000in}}%
\pgfusepath{clip}%
\pgfsetrectcap%
\pgfsetroundjoin%
\pgfsetlinewidth{0.803000pt}%
\definecolor{currentstroke}{rgb}{0.690196,0.690196,0.690196}%
\pgfsetstrokecolor{currentstroke}%
\pgfsetstrokeopacity{0.300000}%
\pgfsetdash{}{0pt}%
\pgfpathmoveto{\pgfqpoint{0.781944in}{0.552778in}}%
\pgfpathlineto{\pgfqpoint{0.781944in}{2.202778in}}%
\pgfusepath{stroke}%
\end{pgfscope}%
\begin{pgfscope}%
\pgfsetbuttcap%
\pgfsetroundjoin%
\definecolor{currentfill}{rgb}{0.000000,0.000000,0.000000}%
\pgfsetfillcolor{currentfill}%
\pgfsetlinewidth{0.602250pt}%
\definecolor{currentstroke}{rgb}{0.000000,0.000000,0.000000}%
\pgfsetstrokecolor{currentstroke}%
\pgfsetdash{}{0pt}%
\pgfsys@defobject{currentmarker}{\pgfqpoint{0.000000in}{-0.027778in}}{\pgfqpoint{0.000000in}{0.000000in}}{%
\pgfpathmoveto{\pgfqpoint{0.000000in}{0.000000in}}%
\pgfpathlineto{\pgfqpoint{0.000000in}{-0.027778in}}%
\pgfusepath{stroke,fill}%
}%
\begin{pgfscope}%
\pgfsys@transformshift{0.781944in}{0.552778in}%
\pgfsys@useobject{currentmarker}{}%
\end{pgfscope}%
\end{pgfscope}%
\begin{pgfscope}%
\pgfsetbuttcap%
\pgfsetroundjoin%
\definecolor{currentfill}{rgb}{0.000000,0.000000,0.000000}%
\pgfsetfillcolor{currentfill}%
\pgfsetlinewidth{0.602250pt}%
\definecolor{currentstroke}{rgb}{0.000000,0.000000,0.000000}%
\pgfsetstrokecolor{currentstroke}%
\pgfsetdash{}{0pt}%
\pgfsys@defobject{currentmarker}{\pgfqpoint{0.000000in}{0.000000in}}{\pgfqpoint{0.000000in}{0.027778in}}{%
\pgfpathmoveto{\pgfqpoint{0.000000in}{0.000000in}}%
\pgfpathlineto{\pgfqpoint{0.000000in}{0.027778in}}%
\pgfusepath{stroke,fill}%
}%
\begin{pgfscope}%
\pgfsys@transformshift{0.781944in}{2.202778in}%
\pgfsys@useobject{currentmarker}{}%
\end{pgfscope}%
\end{pgfscope}%
\begin{pgfscope}%
\pgfpathrectangle{\pgfqpoint{0.781944in}{0.552778in}}{\pgfqpoint{2.138715in}{1.650000in}}%
\pgfusepath{clip}%
\pgfsetrectcap%
\pgfsetroundjoin%
\pgfsetlinewidth{0.803000pt}%
\definecolor{currentstroke}{rgb}{0.690196,0.690196,0.690196}%
\pgfsetstrokecolor{currentstroke}%
\pgfsetstrokeopacity{0.300000}%
\pgfsetdash{}{0pt}%
\pgfpathmoveto{\pgfqpoint{0.824719in}{0.552778in}}%
\pgfpathlineto{\pgfqpoint{0.824719in}{2.202778in}}%
\pgfusepath{stroke}%
\end{pgfscope}%
\begin{pgfscope}%
\pgfsetbuttcap%
\pgfsetroundjoin%
\definecolor{currentfill}{rgb}{0.000000,0.000000,0.000000}%
\pgfsetfillcolor{currentfill}%
\pgfsetlinewidth{0.602250pt}%
\definecolor{currentstroke}{rgb}{0.000000,0.000000,0.000000}%
\pgfsetstrokecolor{currentstroke}%
\pgfsetdash{}{0pt}%
\pgfsys@defobject{currentmarker}{\pgfqpoint{0.000000in}{-0.027778in}}{\pgfqpoint{0.000000in}{0.000000in}}{%
\pgfpathmoveto{\pgfqpoint{0.000000in}{0.000000in}}%
\pgfpathlineto{\pgfqpoint{0.000000in}{-0.027778in}}%
\pgfusepath{stroke,fill}%
}%
\begin{pgfscope}%
\pgfsys@transformshift{0.824719in}{0.552778in}%
\pgfsys@useobject{currentmarker}{}%
\end{pgfscope}%
\end{pgfscope}%
\begin{pgfscope}%
\pgfsetbuttcap%
\pgfsetroundjoin%
\definecolor{currentfill}{rgb}{0.000000,0.000000,0.000000}%
\pgfsetfillcolor{currentfill}%
\pgfsetlinewidth{0.602250pt}%
\definecolor{currentstroke}{rgb}{0.000000,0.000000,0.000000}%
\pgfsetstrokecolor{currentstroke}%
\pgfsetdash{}{0pt}%
\pgfsys@defobject{currentmarker}{\pgfqpoint{0.000000in}{0.000000in}}{\pgfqpoint{0.000000in}{0.027778in}}{%
\pgfpathmoveto{\pgfqpoint{0.000000in}{0.000000in}}%
\pgfpathlineto{\pgfqpoint{0.000000in}{0.027778in}}%
\pgfusepath{stroke,fill}%
}%
\begin{pgfscope}%
\pgfsys@transformshift{0.824719in}{2.202778in}%
\pgfsys@useobject{currentmarker}{}%
\end{pgfscope}%
\end{pgfscope}%
\begin{pgfscope}%
\pgfpathrectangle{\pgfqpoint{0.781944in}{0.552778in}}{\pgfqpoint{2.138715in}{1.650000in}}%
\pgfusepath{clip}%
\pgfsetrectcap%
\pgfsetroundjoin%
\pgfsetlinewidth{0.803000pt}%
\definecolor{currentstroke}{rgb}{0.690196,0.690196,0.690196}%
\pgfsetstrokecolor{currentstroke}%
\pgfsetstrokeopacity{0.300000}%
\pgfsetdash{}{0pt}%
\pgfpathmoveto{\pgfqpoint{0.867493in}{0.552778in}}%
\pgfpathlineto{\pgfqpoint{0.867493in}{2.202778in}}%
\pgfusepath{stroke}%
\end{pgfscope}%
\begin{pgfscope}%
\pgfsetbuttcap%
\pgfsetroundjoin%
\definecolor{currentfill}{rgb}{0.000000,0.000000,0.000000}%
\pgfsetfillcolor{currentfill}%
\pgfsetlinewidth{0.602250pt}%
\definecolor{currentstroke}{rgb}{0.000000,0.000000,0.000000}%
\pgfsetstrokecolor{currentstroke}%
\pgfsetdash{}{0pt}%
\pgfsys@defobject{currentmarker}{\pgfqpoint{0.000000in}{-0.027778in}}{\pgfqpoint{0.000000in}{0.000000in}}{%
\pgfpathmoveto{\pgfqpoint{0.000000in}{0.000000in}}%
\pgfpathlineto{\pgfqpoint{0.000000in}{-0.027778in}}%
\pgfusepath{stroke,fill}%
}%
\begin{pgfscope}%
\pgfsys@transformshift{0.867493in}{0.552778in}%
\pgfsys@useobject{currentmarker}{}%
\end{pgfscope}%
\end{pgfscope}%
\begin{pgfscope}%
\pgfsetbuttcap%
\pgfsetroundjoin%
\definecolor{currentfill}{rgb}{0.000000,0.000000,0.000000}%
\pgfsetfillcolor{currentfill}%
\pgfsetlinewidth{0.602250pt}%
\definecolor{currentstroke}{rgb}{0.000000,0.000000,0.000000}%
\pgfsetstrokecolor{currentstroke}%
\pgfsetdash{}{0pt}%
\pgfsys@defobject{currentmarker}{\pgfqpoint{0.000000in}{0.000000in}}{\pgfqpoint{0.000000in}{0.027778in}}{%
\pgfpathmoveto{\pgfqpoint{0.000000in}{0.000000in}}%
\pgfpathlineto{\pgfqpoint{0.000000in}{0.027778in}}%
\pgfusepath{stroke,fill}%
}%
\begin{pgfscope}%
\pgfsys@transformshift{0.867493in}{2.202778in}%
\pgfsys@useobject{currentmarker}{}%
\end{pgfscope}%
\end{pgfscope}%
\begin{pgfscope}%
\pgfpathrectangle{\pgfqpoint{0.781944in}{0.552778in}}{\pgfqpoint{2.138715in}{1.650000in}}%
\pgfusepath{clip}%
\pgfsetrectcap%
\pgfsetroundjoin%
\pgfsetlinewidth{0.803000pt}%
\definecolor{currentstroke}{rgb}{0.690196,0.690196,0.690196}%
\pgfsetstrokecolor{currentstroke}%
\pgfsetstrokeopacity{0.300000}%
\pgfsetdash{}{0pt}%
\pgfpathmoveto{\pgfqpoint{0.910267in}{0.552778in}}%
\pgfpathlineto{\pgfqpoint{0.910267in}{2.202778in}}%
\pgfusepath{stroke}%
\end{pgfscope}%
\begin{pgfscope}%
\pgfsetbuttcap%
\pgfsetroundjoin%
\definecolor{currentfill}{rgb}{0.000000,0.000000,0.000000}%
\pgfsetfillcolor{currentfill}%
\pgfsetlinewidth{0.602250pt}%
\definecolor{currentstroke}{rgb}{0.000000,0.000000,0.000000}%
\pgfsetstrokecolor{currentstroke}%
\pgfsetdash{}{0pt}%
\pgfsys@defobject{currentmarker}{\pgfqpoint{0.000000in}{-0.027778in}}{\pgfqpoint{0.000000in}{0.000000in}}{%
\pgfpathmoveto{\pgfqpoint{0.000000in}{0.000000in}}%
\pgfpathlineto{\pgfqpoint{0.000000in}{-0.027778in}}%
\pgfusepath{stroke,fill}%
}%
\begin{pgfscope}%
\pgfsys@transformshift{0.910267in}{0.552778in}%
\pgfsys@useobject{currentmarker}{}%
\end{pgfscope}%
\end{pgfscope}%
\begin{pgfscope}%
\pgfsetbuttcap%
\pgfsetroundjoin%
\definecolor{currentfill}{rgb}{0.000000,0.000000,0.000000}%
\pgfsetfillcolor{currentfill}%
\pgfsetlinewidth{0.602250pt}%
\definecolor{currentstroke}{rgb}{0.000000,0.000000,0.000000}%
\pgfsetstrokecolor{currentstroke}%
\pgfsetdash{}{0pt}%
\pgfsys@defobject{currentmarker}{\pgfqpoint{0.000000in}{0.000000in}}{\pgfqpoint{0.000000in}{0.027778in}}{%
\pgfpathmoveto{\pgfqpoint{0.000000in}{0.000000in}}%
\pgfpathlineto{\pgfqpoint{0.000000in}{0.027778in}}%
\pgfusepath{stroke,fill}%
}%
\begin{pgfscope}%
\pgfsys@transformshift{0.910267in}{2.202778in}%
\pgfsys@useobject{currentmarker}{}%
\end{pgfscope}%
\end{pgfscope}%
\begin{pgfscope}%
\pgfpathrectangle{\pgfqpoint{0.781944in}{0.552778in}}{\pgfqpoint{2.138715in}{1.650000in}}%
\pgfusepath{clip}%
\pgfsetrectcap%
\pgfsetroundjoin%
\pgfsetlinewidth{0.803000pt}%
\definecolor{currentstroke}{rgb}{0.690196,0.690196,0.690196}%
\pgfsetstrokecolor{currentstroke}%
\pgfsetstrokeopacity{0.300000}%
\pgfsetdash{}{0pt}%
\pgfpathmoveto{\pgfqpoint{0.953042in}{0.552778in}}%
\pgfpathlineto{\pgfqpoint{0.953042in}{2.202778in}}%
\pgfusepath{stroke}%
\end{pgfscope}%
\begin{pgfscope}%
\pgfsetbuttcap%
\pgfsetroundjoin%
\definecolor{currentfill}{rgb}{0.000000,0.000000,0.000000}%
\pgfsetfillcolor{currentfill}%
\pgfsetlinewidth{0.602250pt}%
\definecolor{currentstroke}{rgb}{0.000000,0.000000,0.000000}%
\pgfsetstrokecolor{currentstroke}%
\pgfsetdash{}{0pt}%
\pgfsys@defobject{currentmarker}{\pgfqpoint{0.000000in}{-0.027778in}}{\pgfqpoint{0.000000in}{0.000000in}}{%
\pgfpathmoveto{\pgfqpoint{0.000000in}{0.000000in}}%
\pgfpathlineto{\pgfqpoint{0.000000in}{-0.027778in}}%
\pgfusepath{stroke,fill}%
}%
\begin{pgfscope}%
\pgfsys@transformshift{0.953042in}{0.552778in}%
\pgfsys@useobject{currentmarker}{}%
\end{pgfscope}%
\end{pgfscope}%
\begin{pgfscope}%
\pgfsetbuttcap%
\pgfsetroundjoin%
\definecolor{currentfill}{rgb}{0.000000,0.000000,0.000000}%
\pgfsetfillcolor{currentfill}%
\pgfsetlinewidth{0.602250pt}%
\definecolor{currentstroke}{rgb}{0.000000,0.000000,0.000000}%
\pgfsetstrokecolor{currentstroke}%
\pgfsetdash{}{0pt}%
\pgfsys@defobject{currentmarker}{\pgfqpoint{0.000000in}{0.000000in}}{\pgfqpoint{0.000000in}{0.027778in}}{%
\pgfpathmoveto{\pgfqpoint{0.000000in}{0.000000in}}%
\pgfpathlineto{\pgfqpoint{0.000000in}{0.027778in}}%
\pgfusepath{stroke,fill}%
}%
\begin{pgfscope}%
\pgfsys@transformshift{0.953042in}{2.202778in}%
\pgfsys@useobject{currentmarker}{}%
\end{pgfscope}%
\end{pgfscope}%
\begin{pgfscope}%
\pgfpathrectangle{\pgfqpoint{0.781944in}{0.552778in}}{\pgfqpoint{2.138715in}{1.650000in}}%
\pgfusepath{clip}%
\pgfsetrectcap%
\pgfsetroundjoin%
\pgfsetlinewidth{0.803000pt}%
\definecolor{currentstroke}{rgb}{0.690196,0.690196,0.690196}%
\pgfsetstrokecolor{currentstroke}%
\pgfsetstrokeopacity{0.300000}%
\pgfsetdash{}{0pt}%
\pgfpathmoveto{\pgfqpoint{1.038590in}{0.552778in}}%
\pgfpathlineto{\pgfqpoint{1.038590in}{2.202778in}}%
\pgfusepath{stroke}%
\end{pgfscope}%
\begin{pgfscope}%
\pgfsetbuttcap%
\pgfsetroundjoin%
\definecolor{currentfill}{rgb}{0.000000,0.000000,0.000000}%
\pgfsetfillcolor{currentfill}%
\pgfsetlinewidth{0.602250pt}%
\definecolor{currentstroke}{rgb}{0.000000,0.000000,0.000000}%
\pgfsetstrokecolor{currentstroke}%
\pgfsetdash{}{0pt}%
\pgfsys@defobject{currentmarker}{\pgfqpoint{0.000000in}{-0.027778in}}{\pgfqpoint{0.000000in}{0.000000in}}{%
\pgfpathmoveto{\pgfqpoint{0.000000in}{0.000000in}}%
\pgfpathlineto{\pgfqpoint{0.000000in}{-0.027778in}}%
\pgfusepath{stroke,fill}%
}%
\begin{pgfscope}%
\pgfsys@transformshift{1.038590in}{0.552778in}%
\pgfsys@useobject{currentmarker}{}%
\end{pgfscope}%
\end{pgfscope}%
\begin{pgfscope}%
\pgfsetbuttcap%
\pgfsetroundjoin%
\definecolor{currentfill}{rgb}{0.000000,0.000000,0.000000}%
\pgfsetfillcolor{currentfill}%
\pgfsetlinewidth{0.602250pt}%
\definecolor{currentstroke}{rgb}{0.000000,0.000000,0.000000}%
\pgfsetstrokecolor{currentstroke}%
\pgfsetdash{}{0pt}%
\pgfsys@defobject{currentmarker}{\pgfqpoint{0.000000in}{0.000000in}}{\pgfqpoint{0.000000in}{0.027778in}}{%
\pgfpathmoveto{\pgfqpoint{0.000000in}{0.000000in}}%
\pgfpathlineto{\pgfqpoint{0.000000in}{0.027778in}}%
\pgfusepath{stroke,fill}%
}%
\begin{pgfscope}%
\pgfsys@transformshift{1.038590in}{2.202778in}%
\pgfsys@useobject{currentmarker}{}%
\end{pgfscope}%
\end{pgfscope}%
\begin{pgfscope}%
\pgfpathrectangle{\pgfqpoint{0.781944in}{0.552778in}}{\pgfqpoint{2.138715in}{1.650000in}}%
\pgfusepath{clip}%
\pgfsetrectcap%
\pgfsetroundjoin%
\pgfsetlinewidth{0.803000pt}%
\definecolor{currentstroke}{rgb}{0.690196,0.690196,0.690196}%
\pgfsetstrokecolor{currentstroke}%
\pgfsetstrokeopacity{0.300000}%
\pgfsetdash{}{0pt}%
\pgfpathmoveto{\pgfqpoint{1.081365in}{0.552778in}}%
\pgfpathlineto{\pgfqpoint{1.081365in}{2.202778in}}%
\pgfusepath{stroke}%
\end{pgfscope}%
\begin{pgfscope}%
\pgfsetbuttcap%
\pgfsetroundjoin%
\definecolor{currentfill}{rgb}{0.000000,0.000000,0.000000}%
\pgfsetfillcolor{currentfill}%
\pgfsetlinewidth{0.602250pt}%
\definecolor{currentstroke}{rgb}{0.000000,0.000000,0.000000}%
\pgfsetstrokecolor{currentstroke}%
\pgfsetdash{}{0pt}%
\pgfsys@defobject{currentmarker}{\pgfqpoint{0.000000in}{-0.027778in}}{\pgfqpoint{0.000000in}{0.000000in}}{%
\pgfpathmoveto{\pgfqpoint{0.000000in}{0.000000in}}%
\pgfpathlineto{\pgfqpoint{0.000000in}{-0.027778in}}%
\pgfusepath{stroke,fill}%
}%
\begin{pgfscope}%
\pgfsys@transformshift{1.081365in}{0.552778in}%
\pgfsys@useobject{currentmarker}{}%
\end{pgfscope}%
\end{pgfscope}%
\begin{pgfscope}%
\pgfsetbuttcap%
\pgfsetroundjoin%
\definecolor{currentfill}{rgb}{0.000000,0.000000,0.000000}%
\pgfsetfillcolor{currentfill}%
\pgfsetlinewidth{0.602250pt}%
\definecolor{currentstroke}{rgb}{0.000000,0.000000,0.000000}%
\pgfsetstrokecolor{currentstroke}%
\pgfsetdash{}{0pt}%
\pgfsys@defobject{currentmarker}{\pgfqpoint{0.000000in}{0.000000in}}{\pgfqpoint{0.000000in}{0.027778in}}{%
\pgfpathmoveto{\pgfqpoint{0.000000in}{0.000000in}}%
\pgfpathlineto{\pgfqpoint{0.000000in}{0.027778in}}%
\pgfusepath{stroke,fill}%
}%
\begin{pgfscope}%
\pgfsys@transformshift{1.081365in}{2.202778in}%
\pgfsys@useobject{currentmarker}{}%
\end{pgfscope}%
\end{pgfscope}%
\begin{pgfscope}%
\pgfpathrectangle{\pgfqpoint{0.781944in}{0.552778in}}{\pgfqpoint{2.138715in}{1.650000in}}%
\pgfusepath{clip}%
\pgfsetrectcap%
\pgfsetroundjoin%
\pgfsetlinewidth{0.803000pt}%
\definecolor{currentstroke}{rgb}{0.690196,0.690196,0.690196}%
\pgfsetstrokecolor{currentstroke}%
\pgfsetstrokeopacity{0.300000}%
\pgfsetdash{}{0pt}%
\pgfpathmoveto{\pgfqpoint{1.124139in}{0.552778in}}%
\pgfpathlineto{\pgfqpoint{1.124139in}{2.202778in}}%
\pgfusepath{stroke}%
\end{pgfscope}%
\begin{pgfscope}%
\pgfsetbuttcap%
\pgfsetroundjoin%
\definecolor{currentfill}{rgb}{0.000000,0.000000,0.000000}%
\pgfsetfillcolor{currentfill}%
\pgfsetlinewidth{0.602250pt}%
\definecolor{currentstroke}{rgb}{0.000000,0.000000,0.000000}%
\pgfsetstrokecolor{currentstroke}%
\pgfsetdash{}{0pt}%
\pgfsys@defobject{currentmarker}{\pgfqpoint{0.000000in}{-0.027778in}}{\pgfqpoint{0.000000in}{0.000000in}}{%
\pgfpathmoveto{\pgfqpoint{0.000000in}{0.000000in}}%
\pgfpathlineto{\pgfqpoint{0.000000in}{-0.027778in}}%
\pgfusepath{stroke,fill}%
}%
\begin{pgfscope}%
\pgfsys@transformshift{1.124139in}{0.552778in}%
\pgfsys@useobject{currentmarker}{}%
\end{pgfscope}%
\end{pgfscope}%
\begin{pgfscope}%
\pgfsetbuttcap%
\pgfsetroundjoin%
\definecolor{currentfill}{rgb}{0.000000,0.000000,0.000000}%
\pgfsetfillcolor{currentfill}%
\pgfsetlinewidth{0.602250pt}%
\definecolor{currentstroke}{rgb}{0.000000,0.000000,0.000000}%
\pgfsetstrokecolor{currentstroke}%
\pgfsetdash{}{0pt}%
\pgfsys@defobject{currentmarker}{\pgfqpoint{0.000000in}{0.000000in}}{\pgfqpoint{0.000000in}{0.027778in}}{%
\pgfpathmoveto{\pgfqpoint{0.000000in}{0.000000in}}%
\pgfpathlineto{\pgfqpoint{0.000000in}{0.027778in}}%
\pgfusepath{stroke,fill}%
}%
\begin{pgfscope}%
\pgfsys@transformshift{1.124139in}{2.202778in}%
\pgfsys@useobject{currentmarker}{}%
\end{pgfscope}%
\end{pgfscope}%
\begin{pgfscope}%
\pgfpathrectangle{\pgfqpoint{0.781944in}{0.552778in}}{\pgfqpoint{2.138715in}{1.650000in}}%
\pgfusepath{clip}%
\pgfsetrectcap%
\pgfsetroundjoin%
\pgfsetlinewidth{0.803000pt}%
\definecolor{currentstroke}{rgb}{0.690196,0.690196,0.690196}%
\pgfsetstrokecolor{currentstroke}%
\pgfsetstrokeopacity{0.300000}%
\pgfsetdash{}{0pt}%
\pgfpathmoveto{\pgfqpoint{1.166913in}{0.552778in}}%
\pgfpathlineto{\pgfqpoint{1.166913in}{2.202778in}}%
\pgfusepath{stroke}%
\end{pgfscope}%
\begin{pgfscope}%
\pgfsetbuttcap%
\pgfsetroundjoin%
\definecolor{currentfill}{rgb}{0.000000,0.000000,0.000000}%
\pgfsetfillcolor{currentfill}%
\pgfsetlinewidth{0.602250pt}%
\definecolor{currentstroke}{rgb}{0.000000,0.000000,0.000000}%
\pgfsetstrokecolor{currentstroke}%
\pgfsetdash{}{0pt}%
\pgfsys@defobject{currentmarker}{\pgfqpoint{0.000000in}{-0.027778in}}{\pgfqpoint{0.000000in}{0.000000in}}{%
\pgfpathmoveto{\pgfqpoint{0.000000in}{0.000000in}}%
\pgfpathlineto{\pgfqpoint{0.000000in}{-0.027778in}}%
\pgfusepath{stroke,fill}%
}%
\begin{pgfscope}%
\pgfsys@transformshift{1.166913in}{0.552778in}%
\pgfsys@useobject{currentmarker}{}%
\end{pgfscope}%
\end{pgfscope}%
\begin{pgfscope}%
\pgfsetbuttcap%
\pgfsetroundjoin%
\definecolor{currentfill}{rgb}{0.000000,0.000000,0.000000}%
\pgfsetfillcolor{currentfill}%
\pgfsetlinewidth{0.602250pt}%
\definecolor{currentstroke}{rgb}{0.000000,0.000000,0.000000}%
\pgfsetstrokecolor{currentstroke}%
\pgfsetdash{}{0pt}%
\pgfsys@defobject{currentmarker}{\pgfqpoint{0.000000in}{0.000000in}}{\pgfqpoint{0.000000in}{0.027778in}}{%
\pgfpathmoveto{\pgfqpoint{0.000000in}{0.000000in}}%
\pgfpathlineto{\pgfqpoint{0.000000in}{0.027778in}}%
\pgfusepath{stroke,fill}%
}%
\begin{pgfscope}%
\pgfsys@transformshift{1.166913in}{2.202778in}%
\pgfsys@useobject{currentmarker}{}%
\end{pgfscope}%
\end{pgfscope}%
\begin{pgfscope}%
\pgfpathrectangle{\pgfqpoint{0.781944in}{0.552778in}}{\pgfqpoint{2.138715in}{1.650000in}}%
\pgfusepath{clip}%
\pgfsetrectcap%
\pgfsetroundjoin%
\pgfsetlinewidth{0.803000pt}%
\definecolor{currentstroke}{rgb}{0.690196,0.690196,0.690196}%
\pgfsetstrokecolor{currentstroke}%
\pgfsetstrokeopacity{0.300000}%
\pgfsetdash{}{0pt}%
\pgfpathmoveto{\pgfqpoint{1.209687in}{0.552778in}}%
\pgfpathlineto{\pgfqpoint{1.209687in}{2.202778in}}%
\pgfusepath{stroke}%
\end{pgfscope}%
\begin{pgfscope}%
\pgfsetbuttcap%
\pgfsetroundjoin%
\definecolor{currentfill}{rgb}{0.000000,0.000000,0.000000}%
\pgfsetfillcolor{currentfill}%
\pgfsetlinewidth{0.602250pt}%
\definecolor{currentstroke}{rgb}{0.000000,0.000000,0.000000}%
\pgfsetstrokecolor{currentstroke}%
\pgfsetdash{}{0pt}%
\pgfsys@defobject{currentmarker}{\pgfqpoint{0.000000in}{-0.027778in}}{\pgfqpoint{0.000000in}{0.000000in}}{%
\pgfpathmoveto{\pgfqpoint{0.000000in}{0.000000in}}%
\pgfpathlineto{\pgfqpoint{0.000000in}{-0.027778in}}%
\pgfusepath{stroke,fill}%
}%
\begin{pgfscope}%
\pgfsys@transformshift{1.209687in}{0.552778in}%
\pgfsys@useobject{currentmarker}{}%
\end{pgfscope}%
\end{pgfscope}%
\begin{pgfscope}%
\pgfsetbuttcap%
\pgfsetroundjoin%
\definecolor{currentfill}{rgb}{0.000000,0.000000,0.000000}%
\pgfsetfillcolor{currentfill}%
\pgfsetlinewidth{0.602250pt}%
\definecolor{currentstroke}{rgb}{0.000000,0.000000,0.000000}%
\pgfsetstrokecolor{currentstroke}%
\pgfsetdash{}{0pt}%
\pgfsys@defobject{currentmarker}{\pgfqpoint{0.000000in}{0.000000in}}{\pgfqpoint{0.000000in}{0.027778in}}{%
\pgfpathmoveto{\pgfqpoint{0.000000in}{0.000000in}}%
\pgfpathlineto{\pgfqpoint{0.000000in}{0.027778in}}%
\pgfusepath{stroke,fill}%
}%
\begin{pgfscope}%
\pgfsys@transformshift{1.209687in}{2.202778in}%
\pgfsys@useobject{currentmarker}{}%
\end{pgfscope}%
\end{pgfscope}%
\begin{pgfscope}%
\pgfpathrectangle{\pgfqpoint{0.781944in}{0.552778in}}{\pgfqpoint{2.138715in}{1.650000in}}%
\pgfusepath{clip}%
\pgfsetrectcap%
\pgfsetroundjoin%
\pgfsetlinewidth{0.803000pt}%
\definecolor{currentstroke}{rgb}{0.690196,0.690196,0.690196}%
\pgfsetstrokecolor{currentstroke}%
\pgfsetstrokeopacity{0.300000}%
\pgfsetdash{}{0pt}%
\pgfpathmoveto{\pgfqpoint{1.252462in}{0.552778in}}%
\pgfpathlineto{\pgfqpoint{1.252462in}{2.202778in}}%
\pgfusepath{stroke}%
\end{pgfscope}%
\begin{pgfscope}%
\pgfsetbuttcap%
\pgfsetroundjoin%
\definecolor{currentfill}{rgb}{0.000000,0.000000,0.000000}%
\pgfsetfillcolor{currentfill}%
\pgfsetlinewidth{0.602250pt}%
\definecolor{currentstroke}{rgb}{0.000000,0.000000,0.000000}%
\pgfsetstrokecolor{currentstroke}%
\pgfsetdash{}{0pt}%
\pgfsys@defobject{currentmarker}{\pgfqpoint{0.000000in}{-0.027778in}}{\pgfqpoint{0.000000in}{0.000000in}}{%
\pgfpathmoveto{\pgfqpoint{0.000000in}{0.000000in}}%
\pgfpathlineto{\pgfqpoint{0.000000in}{-0.027778in}}%
\pgfusepath{stroke,fill}%
}%
\begin{pgfscope}%
\pgfsys@transformshift{1.252462in}{0.552778in}%
\pgfsys@useobject{currentmarker}{}%
\end{pgfscope}%
\end{pgfscope}%
\begin{pgfscope}%
\pgfsetbuttcap%
\pgfsetroundjoin%
\definecolor{currentfill}{rgb}{0.000000,0.000000,0.000000}%
\pgfsetfillcolor{currentfill}%
\pgfsetlinewidth{0.602250pt}%
\definecolor{currentstroke}{rgb}{0.000000,0.000000,0.000000}%
\pgfsetstrokecolor{currentstroke}%
\pgfsetdash{}{0pt}%
\pgfsys@defobject{currentmarker}{\pgfqpoint{0.000000in}{0.000000in}}{\pgfqpoint{0.000000in}{0.027778in}}{%
\pgfpathmoveto{\pgfqpoint{0.000000in}{0.000000in}}%
\pgfpathlineto{\pgfqpoint{0.000000in}{0.027778in}}%
\pgfusepath{stroke,fill}%
}%
\begin{pgfscope}%
\pgfsys@transformshift{1.252462in}{2.202778in}%
\pgfsys@useobject{currentmarker}{}%
\end{pgfscope}%
\end{pgfscope}%
\begin{pgfscope}%
\pgfpathrectangle{\pgfqpoint{0.781944in}{0.552778in}}{\pgfqpoint{2.138715in}{1.650000in}}%
\pgfusepath{clip}%
\pgfsetrectcap%
\pgfsetroundjoin%
\pgfsetlinewidth{0.803000pt}%
\definecolor{currentstroke}{rgb}{0.690196,0.690196,0.690196}%
\pgfsetstrokecolor{currentstroke}%
\pgfsetstrokeopacity{0.300000}%
\pgfsetdash{}{0pt}%
\pgfpathmoveto{\pgfqpoint{1.295236in}{0.552778in}}%
\pgfpathlineto{\pgfqpoint{1.295236in}{2.202778in}}%
\pgfusepath{stroke}%
\end{pgfscope}%
\begin{pgfscope}%
\pgfsetbuttcap%
\pgfsetroundjoin%
\definecolor{currentfill}{rgb}{0.000000,0.000000,0.000000}%
\pgfsetfillcolor{currentfill}%
\pgfsetlinewidth{0.602250pt}%
\definecolor{currentstroke}{rgb}{0.000000,0.000000,0.000000}%
\pgfsetstrokecolor{currentstroke}%
\pgfsetdash{}{0pt}%
\pgfsys@defobject{currentmarker}{\pgfqpoint{0.000000in}{-0.027778in}}{\pgfqpoint{0.000000in}{0.000000in}}{%
\pgfpathmoveto{\pgfqpoint{0.000000in}{0.000000in}}%
\pgfpathlineto{\pgfqpoint{0.000000in}{-0.027778in}}%
\pgfusepath{stroke,fill}%
}%
\begin{pgfscope}%
\pgfsys@transformshift{1.295236in}{0.552778in}%
\pgfsys@useobject{currentmarker}{}%
\end{pgfscope}%
\end{pgfscope}%
\begin{pgfscope}%
\pgfsetbuttcap%
\pgfsetroundjoin%
\definecolor{currentfill}{rgb}{0.000000,0.000000,0.000000}%
\pgfsetfillcolor{currentfill}%
\pgfsetlinewidth{0.602250pt}%
\definecolor{currentstroke}{rgb}{0.000000,0.000000,0.000000}%
\pgfsetstrokecolor{currentstroke}%
\pgfsetdash{}{0pt}%
\pgfsys@defobject{currentmarker}{\pgfqpoint{0.000000in}{0.000000in}}{\pgfqpoint{0.000000in}{0.027778in}}{%
\pgfpathmoveto{\pgfqpoint{0.000000in}{0.000000in}}%
\pgfpathlineto{\pgfqpoint{0.000000in}{0.027778in}}%
\pgfusepath{stroke,fill}%
}%
\begin{pgfscope}%
\pgfsys@transformshift{1.295236in}{2.202778in}%
\pgfsys@useobject{currentmarker}{}%
\end{pgfscope}%
\end{pgfscope}%
\begin{pgfscope}%
\pgfpathrectangle{\pgfqpoint{0.781944in}{0.552778in}}{\pgfqpoint{2.138715in}{1.650000in}}%
\pgfusepath{clip}%
\pgfsetrectcap%
\pgfsetroundjoin%
\pgfsetlinewidth{0.803000pt}%
\definecolor{currentstroke}{rgb}{0.690196,0.690196,0.690196}%
\pgfsetstrokecolor{currentstroke}%
\pgfsetstrokeopacity{0.300000}%
\pgfsetdash{}{0pt}%
\pgfpathmoveto{\pgfqpoint{1.338010in}{0.552778in}}%
\pgfpathlineto{\pgfqpoint{1.338010in}{2.202778in}}%
\pgfusepath{stroke}%
\end{pgfscope}%
\begin{pgfscope}%
\pgfsetbuttcap%
\pgfsetroundjoin%
\definecolor{currentfill}{rgb}{0.000000,0.000000,0.000000}%
\pgfsetfillcolor{currentfill}%
\pgfsetlinewidth{0.602250pt}%
\definecolor{currentstroke}{rgb}{0.000000,0.000000,0.000000}%
\pgfsetstrokecolor{currentstroke}%
\pgfsetdash{}{0pt}%
\pgfsys@defobject{currentmarker}{\pgfqpoint{0.000000in}{-0.027778in}}{\pgfqpoint{0.000000in}{0.000000in}}{%
\pgfpathmoveto{\pgfqpoint{0.000000in}{0.000000in}}%
\pgfpathlineto{\pgfqpoint{0.000000in}{-0.027778in}}%
\pgfusepath{stroke,fill}%
}%
\begin{pgfscope}%
\pgfsys@transformshift{1.338010in}{0.552778in}%
\pgfsys@useobject{currentmarker}{}%
\end{pgfscope}%
\end{pgfscope}%
\begin{pgfscope}%
\pgfsetbuttcap%
\pgfsetroundjoin%
\definecolor{currentfill}{rgb}{0.000000,0.000000,0.000000}%
\pgfsetfillcolor{currentfill}%
\pgfsetlinewidth{0.602250pt}%
\definecolor{currentstroke}{rgb}{0.000000,0.000000,0.000000}%
\pgfsetstrokecolor{currentstroke}%
\pgfsetdash{}{0pt}%
\pgfsys@defobject{currentmarker}{\pgfqpoint{0.000000in}{0.000000in}}{\pgfqpoint{0.000000in}{0.027778in}}{%
\pgfpathmoveto{\pgfqpoint{0.000000in}{0.000000in}}%
\pgfpathlineto{\pgfqpoint{0.000000in}{0.027778in}}%
\pgfusepath{stroke,fill}%
}%
\begin{pgfscope}%
\pgfsys@transformshift{1.338010in}{2.202778in}%
\pgfsys@useobject{currentmarker}{}%
\end{pgfscope}%
\end{pgfscope}%
\begin{pgfscope}%
\pgfpathrectangle{\pgfqpoint{0.781944in}{0.552778in}}{\pgfqpoint{2.138715in}{1.650000in}}%
\pgfusepath{clip}%
\pgfsetrectcap%
\pgfsetroundjoin%
\pgfsetlinewidth{0.803000pt}%
\definecolor{currentstroke}{rgb}{0.690196,0.690196,0.690196}%
\pgfsetstrokecolor{currentstroke}%
\pgfsetstrokeopacity{0.300000}%
\pgfsetdash{}{0pt}%
\pgfpathmoveto{\pgfqpoint{1.380785in}{0.552778in}}%
\pgfpathlineto{\pgfqpoint{1.380785in}{2.202778in}}%
\pgfusepath{stroke}%
\end{pgfscope}%
\begin{pgfscope}%
\pgfsetbuttcap%
\pgfsetroundjoin%
\definecolor{currentfill}{rgb}{0.000000,0.000000,0.000000}%
\pgfsetfillcolor{currentfill}%
\pgfsetlinewidth{0.602250pt}%
\definecolor{currentstroke}{rgb}{0.000000,0.000000,0.000000}%
\pgfsetstrokecolor{currentstroke}%
\pgfsetdash{}{0pt}%
\pgfsys@defobject{currentmarker}{\pgfqpoint{0.000000in}{-0.027778in}}{\pgfqpoint{0.000000in}{0.000000in}}{%
\pgfpathmoveto{\pgfqpoint{0.000000in}{0.000000in}}%
\pgfpathlineto{\pgfqpoint{0.000000in}{-0.027778in}}%
\pgfusepath{stroke,fill}%
}%
\begin{pgfscope}%
\pgfsys@transformshift{1.380785in}{0.552778in}%
\pgfsys@useobject{currentmarker}{}%
\end{pgfscope}%
\end{pgfscope}%
\begin{pgfscope}%
\pgfsetbuttcap%
\pgfsetroundjoin%
\definecolor{currentfill}{rgb}{0.000000,0.000000,0.000000}%
\pgfsetfillcolor{currentfill}%
\pgfsetlinewidth{0.602250pt}%
\definecolor{currentstroke}{rgb}{0.000000,0.000000,0.000000}%
\pgfsetstrokecolor{currentstroke}%
\pgfsetdash{}{0pt}%
\pgfsys@defobject{currentmarker}{\pgfqpoint{0.000000in}{0.000000in}}{\pgfqpoint{0.000000in}{0.027778in}}{%
\pgfpathmoveto{\pgfqpoint{0.000000in}{0.000000in}}%
\pgfpathlineto{\pgfqpoint{0.000000in}{0.027778in}}%
\pgfusepath{stroke,fill}%
}%
\begin{pgfscope}%
\pgfsys@transformshift{1.380785in}{2.202778in}%
\pgfsys@useobject{currentmarker}{}%
\end{pgfscope}%
\end{pgfscope}%
\begin{pgfscope}%
\pgfpathrectangle{\pgfqpoint{0.781944in}{0.552778in}}{\pgfqpoint{2.138715in}{1.650000in}}%
\pgfusepath{clip}%
\pgfsetrectcap%
\pgfsetroundjoin%
\pgfsetlinewidth{0.803000pt}%
\definecolor{currentstroke}{rgb}{0.690196,0.690196,0.690196}%
\pgfsetstrokecolor{currentstroke}%
\pgfsetstrokeopacity{0.300000}%
\pgfsetdash{}{0pt}%
\pgfpathmoveto{\pgfqpoint{1.466333in}{0.552778in}}%
\pgfpathlineto{\pgfqpoint{1.466333in}{2.202778in}}%
\pgfusepath{stroke}%
\end{pgfscope}%
\begin{pgfscope}%
\pgfsetbuttcap%
\pgfsetroundjoin%
\definecolor{currentfill}{rgb}{0.000000,0.000000,0.000000}%
\pgfsetfillcolor{currentfill}%
\pgfsetlinewidth{0.602250pt}%
\definecolor{currentstroke}{rgb}{0.000000,0.000000,0.000000}%
\pgfsetstrokecolor{currentstroke}%
\pgfsetdash{}{0pt}%
\pgfsys@defobject{currentmarker}{\pgfqpoint{0.000000in}{-0.027778in}}{\pgfqpoint{0.000000in}{0.000000in}}{%
\pgfpathmoveto{\pgfqpoint{0.000000in}{0.000000in}}%
\pgfpathlineto{\pgfqpoint{0.000000in}{-0.027778in}}%
\pgfusepath{stroke,fill}%
}%
\begin{pgfscope}%
\pgfsys@transformshift{1.466333in}{0.552778in}%
\pgfsys@useobject{currentmarker}{}%
\end{pgfscope}%
\end{pgfscope}%
\begin{pgfscope}%
\pgfsetbuttcap%
\pgfsetroundjoin%
\definecolor{currentfill}{rgb}{0.000000,0.000000,0.000000}%
\pgfsetfillcolor{currentfill}%
\pgfsetlinewidth{0.602250pt}%
\definecolor{currentstroke}{rgb}{0.000000,0.000000,0.000000}%
\pgfsetstrokecolor{currentstroke}%
\pgfsetdash{}{0pt}%
\pgfsys@defobject{currentmarker}{\pgfqpoint{0.000000in}{0.000000in}}{\pgfqpoint{0.000000in}{0.027778in}}{%
\pgfpathmoveto{\pgfqpoint{0.000000in}{0.000000in}}%
\pgfpathlineto{\pgfqpoint{0.000000in}{0.027778in}}%
\pgfusepath{stroke,fill}%
}%
\begin{pgfscope}%
\pgfsys@transformshift{1.466333in}{2.202778in}%
\pgfsys@useobject{currentmarker}{}%
\end{pgfscope}%
\end{pgfscope}%
\begin{pgfscope}%
\pgfpathrectangle{\pgfqpoint{0.781944in}{0.552778in}}{\pgfqpoint{2.138715in}{1.650000in}}%
\pgfusepath{clip}%
\pgfsetrectcap%
\pgfsetroundjoin%
\pgfsetlinewidth{0.803000pt}%
\definecolor{currentstroke}{rgb}{0.690196,0.690196,0.690196}%
\pgfsetstrokecolor{currentstroke}%
\pgfsetstrokeopacity{0.300000}%
\pgfsetdash{}{0pt}%
\pgfpathmoveto{\pgfqpoint{1.509108in}{0.552778in}}%
\pgfpathlineto{\pgfqpoint{1.509108in}{2.202778in}}%
\pgfusepath{stroke}%
\end{pgfscope}%
\begin{pgfscope}%
\pgfsetbuttcap%
\pgfsetroundjoin%
\definecolor{currentfill}{rgb}{0.000000,0.000000,0.000000}%
\pgfsetfillcolor{currentfill}%
\pgfsetlinewidth{0.602250pt}%
\definecolor{currentstroke}{rgb}{0.000000,0.000000,0.000000}%
\pgfsetstrokecolor{currentstroke}%
\pgfsetdash{}{0pt}%
\pgfsys@defobject{currentmarker}{\pgfqpoint{0.000000in}{-0.027778in}}{\pgfqpoint{0.000000in}{0.000000in}}{%
\pgfpathmoveto{\pgfqpoint{0.000000in}{0.000000in}}%
\pgfpathlineto{\pgfqpoint{0.000000in}{-0.027778in}}%
\pgfusepath{stroke,fill}%
}%
\begin{pgfscope}%
\pgfsys@transformshift{1.509108in}{0.552778in}%
\pgfsys@useobject{currentmarker}{}%
\end{pgfscope}%
\end{pgfscope}%
\begin{pgfscope}%
\pgfsetbuttcap%
\pgfsetroundjoin%
\definecolor{currentfill}{rgb}{0.000000,0.000000,0.000000}%
\pgfsetfillcolor{currentfill}%
\pgfsetlinewidth{0.602250pt}%
\definecolor{currentstroke}{rgb}{0.000000,0.000000,0.000000}%
\pgfsetstrokecolor{currentstroke}%
\pgfsetdash{}{0pt}%
\pgfsys@defobject{currentmarker}{\pgfqpoint{0.000000in}{0.000000in}}{\pgfqpoint{0.000000in}{0.027778in}}{%
\pgfpathmoveto{\pgfqpoint{0.000000in}{0.000000in}}%
\pgfpathlineto{\pgfqpoint{0.000000in}{0.027778in}}%
\pgfusepath{stroke,fill}%
}%
\begin{pgfscope}%
\pgfsys@transformshift{1.509108in}{2.202778in}%
\pgfsys@useobject{currentmarker}{}%
\end{pgfscope}%
\end{pgfscope}%
\begin{pgfscope}%
\pgfpathrectangle{\pgfqpoint{0.781944in}{0.552778in}}{\pgfqpoint{2.138715in}{1.650000in}}%
\pgfusepath{clip}%
\pgfsetrectcap%
\pgfsetroundjoin%
\pgfsetlinewidth{0.803000pt}%
\definecolor{currentstroke}{rgb}{0.690196,0.690196,0.690196}%
\pgfsetstrokecolor{currentstroke}%
\pgfsetstrokeopacity{0.300000}%
\pgfsetdash{}{0pt}%
\pgfpathmoveto{\pgfqpoint{1.551882in}{0.552778in}}%
\pgfpathlineto{\pgfqpoint{1.551882in}{2.202778in}}%
\pgfusepath{stroke}%
\end{pgfscope}%
\begin{pgfscope}%
\pgfsetbuttcap%
\pgfsetroundjoin%
\definecolor{currentfill}{rgb}{0.000000,0.000000,0.000000}%
\pgfsetfillcolor{currentfill}%
\pgfsetlinewidth{0.602250pt}%
\definecolor{currentstroke}{rgb}{0.000000,0.000000,0.000000}%
\pgfsetstrokecolor{currentstroke}%
\pgfsetdash{}{0pt}%
\pgfsys@defobject{currentmarker}{\pgfqpoint{0.000000in}{-0.027778in}}{\pgfqpoint{0.000000in}{0.000000in}}{%
\pgfpathmoveto{\pgfqpoint{0.000000in}{0.000000in}}%
\pgfpathlineto{\pgfqpoint{0.000000in}{-0.027778in}}%
\pgfusepath{stroke,fill}%
}%
\begin{pgfscope}%
\pgfsys@transformshift{1.551882in}{0.552778in}%
\pgfsys@useobject{currentmarker}{}%
\end{pgfscope}%
\end{pgfscope}%
\begin{pgfscope}%
\pgfsetbuttcap%
\pgfsetroundjoin%
\definecolor{currentfill}{rgb}{0.000000,0.000000,0.000000}%
\pgfsetfillcolor{currentfill}%
\pgfsetlinewidth{0.602250pt}%
\definecolor{currentstroke}{rgb}{0.000000,0.000000,0.000000}%
\pgfsetstrokecolor{currentstroke}%
\pgfsetdash{}{0pt}%
\pgfsys@defobject{currentmarker}{\pgfqpoint{0.000000in}{0.000000in}}{\pgfqpoint{0.000000in}{0.027778in}}{%
\pgfpathmoveto{\pgfqpoint{0.000000in}{0.000000in}}%
\pgfpathlineto{\pgfqpoint{0.000000in}{0.027778in}}%
\pgfusepath{stroke,fill}%
}%
\begin{pgfscope}%
\pgfsys@transformshift{1.551882in}{2.202778in}%
\pgfsys@useobject{currentmarker}{}%
\end{pgfscope}%
\end{pgfscope}%
\begin{pgfscope}%
\pgfpathrectangle{\pgfqpoint{0.781944in}{0.552778in}}{\pgfqpoint{2.138715in}{1.650000in}}%
\pgfusepath{clip}%
\pgfsetrectcap%
\pgfsetroundjoin%
\pgfsetlinewidth{0.803000pt}%
\definecolor{currentstroke}{rgb}{0.690196,0.690196,0.690196}%
\pgfsetstrokecolor{currentstroke}%
\pgfsetstrokeopacity{0.300000}%
\pgfsetdash{}{0pt}%
\pgfpathmoveto{\pgfqpoint{1.594656in}{0.552778in}}%
\pgfpathlineto{\pgfqpoint{1.594656in}{2.202778in}}%
\pgfusepath{stroke}%
\end{pgfscope}%
\begin{pgfscope}%
\pgfsetbuttcap%
\pgfsetroundjoin%
\definecolor{currentfill}{rgb}{0.000000,0.000000,0.000000}%
\pgfsetfillcolor{currentfill}%
\pgfsetlinewidth{0.602250pt}%
\definecolor{currentstroke}{rgb}{0.000000,0.000000,0.000000}%
\pgfsetstrokecolor{currentstroke}%
\pgfsetdash{}{0pt}%
\pgfsys@defobject{currentmarker}{\pgfqpoint{0.000000in}{-0.027778in}}{\pgfqpoint{0.000000in}{0.000000in}}{%
\pgfpathmoveto{\pgfqpoint{0.000000in}{0.000000in}}%
\pgfpathlineto{\pgfqpoint{0.000000in}{-0.027778in}}%
\pgfusepath{stroke,fill}%
}%
\begin{pgfscope}%
\pgfsys@transformshift{1.594656in}{0.552778in}%
\pgfsys@useobject{currentmarker}{}%
\end{pgfscope}%
\end{pgfscope}%
\begin{pgfscope}%
\pgfsetbuttcap%
\pgfsetroundjoin%
\definecolor{currentfill}{rgb}{0.000000,0.000000,0.000000}%
\pgfsetfillcolor{currentfill}%
\pgfsetlinewidth{0.602250pt}%
\definecolor{currentstroke}{rgb}{0.000000,0.000000,0.000000}%
\pgfsetstrokecolor{currentstroke}%
\pgfsetdash{}{0pt}%
\pgfsys@defobject{currentmarker}{\pgfqpoint{0.000000in}{0.000000in}}{\pgfqpoint{0.000000in}{0.027778in}}{%
\pgfpathmoveto{\pgfqpoint{0.000000in}{0.000000in}}%
\pgfpathlineto{\pgfqpoint{0.000000in}{0.027778in}}%
\pgfusepath{stroke,fill}%
}%
\begin{pgfscope}%
\pgfsys@transformshift{1.594656in}{2.202778in}%
\pgfsys@useobject{currentmarker}{}%
\end{pgfscope}%
\end{pgfscope}%
\begin{pgfscope}%
\pgfpathrectangle{\pgfqpoint{0.781944in}{0.552778in}}{\pgfqpoint{2.138715in}{1.650000in}}%
\pgfusepath{clip}%
\pgfsetrectcap%
\pgfsetroundjoin%
\pgfsetlinewidth{0.803000pt}%
\definecolor{currentstroke}{rgb}{0.690196,0.690196,0.690196}%
\pgfsetstrokecolor{currentstroke}%
\pgfsetstrokeopacity{0.300000}%
\pgfsetdash{}{0pt}%
\pgfpathmoveto{\pgfqpoint{1.637431in}{0.552778in}}%
\pgfpathlineto{\pgfqpoint{1.637431in}{2.202778in}}%
\pgfusepath{stroke}%
\end{pgfscope}%
\begin{pgfscope}%
\pgfsetbuttcap%
\pgfsetroundjoin%
\definecolor{currentfill}{rgb}{0.000000,0.000000,0.000000}%
\pgfsetfillcolor{currentfill}%
\pgfsetlinewidth{0.602250pt}%
\definecolor{currentstroke}{rgb}{0.000000,0.000000,0.000000}%
\pgfsetstrokecolor{currentstroke}%
\pgfsetdash{}{0pt}%
\pgfsys@defobject{currentmarker}{\pgfqpoint{0.000000in}{-0.027778in}}{\pgfqpoint{0.000000in}{0.000000in}}{%
\pgfpathmoveto{\pgfqpoint{0.000000in}{0.000000in}}%
\pgfpathlineto{\pgfqpoint{0.000000in}{-0.027778in}}%
\pgfusepath{stroke,fill}%
}%
\begin{pgfscope}%
\pgfsys@transformshift{1.637431in}{0.552778in}%
\pgfsys@useobject{currentmarker}{}%
\end{pgfscope}%
\end{pgfscope}%
\begin{pgfscope}%
\pgfsetbuttcap%
\pgfsetroundjoin%
\definecolor{currentfill}{rgb}{0.000000,0.000000,0.000000}%
\pgfsetfillcolor{currentfill}%
\pgfsetlinewidth{0.602250pt}%
\definecolor{currentstroke}{rgb}{0.000000,0.000000,0.000000}%
\pgfsetstrokecolor{currentstroke}%
\pgfsetdash{}{0pt}%
\pgfsys@defobject{currentmarker}{\pgfqpoint{0.000000in}{0.000000in}}{\pgfqpoint{0.000000in}{0.027778in}}{%
\pgfpathmoveto{\pgfqpoint{0.000000in}{0.000000in}}%
\pgfpathlineto{\pgfqpoint{0.000000in}{0.027778in}}%
\pgfusepath{stroke,fill}%
}%
\begin{pgfscope}%
\pgfsys@transformshift{1.637431in}{2.202778in}%
\pgfsys@useobject{currentmarker}{}%
\end{pgfscope}%
\end{pgfscope}%
\begin{pgfscope}%
\pgfpathrectangle{\pgfqpoint{0.781944in}{0.552778in}}{\pgfqpoint{2.138715in}{1.650000in}}%
\pgfusepath{clip}%
\pgfsetrectcap%
\pgfsetroundjoin%
\pgfsetlinewidth{0.803000pt}%
\definecolor{currentstroke}{rgb}{0.690196,0.690196,0.690196}%
\pgfsetstrokecolor{currentstroke}%
\pgfsetstrokeopacity{0.300000}%
\pgfsetdash{}{0pt}%
\pgfpathmoveto{\pgfqpoint{1.680205in}{0.552778in}}%
\pgfpathlineto{\pgfqpoint{1.680205in}{2.202778in}}%
\pgfusepath{stroke}%
\end{pgfscope}%
\begin{pgfscope}%
\pgfsetbuttcap%
\pgfsetroundjoin%
\definecolor{currentfill}{rgb}{0.000000,0.000000,0.000000}%
\pgfsetfillcolor{currentfill}%
\pgfsetlinewidth{0.602250pt}%
\definecolor{currentstroke}{rgb}{0.000000,0.000000,0.000000}%
\pgfsetstrokecolor{currentstroke}%
\pgfsetdash{}{0pt}%
\pgfsys@defobject{currentmarker}{\pgfqpoint{0.000000in}{-0.027778in}}{\pgfqpoint{0.000000in}{0.000000in}}{%
\pgfpathmoveto{\pgfqpoint{0.000000in}{0.000000in}}%
\pgfpathlineto{\pgfqpoint{0.000000in}{-0.027778in}}%
\pgfusepath{stroke,fill}%
}%
\begin{pgfscope}%
\pgfsys@transformshift{1.680205in}{0.552778in}%
\pgfsys@useobject{currentmarker}{}%
\end{pgfscope}%
\end{pgfscope}%
\begin{pgfscope}%
\pgfsetbuttcap%
\pgfsetroundjoin%
\definecolor{currentfill}{rgb}{0.000000,0.000000,0.000000}%
\pgfsetfillcolor{currentfill}%
\pgfsetlinewidth{0.602250pt}%
\definecolor{currentstroke}{rgb}{0.000000,0.000000,0.000000}%
\pgfsetstrokecolor{currentstroke}%
\pgfsetdash{}{0pt}%
\pgfsys@defobject{currentmarker}{\pgfqpoint{0.000000in}{0.000000in}}{\pgfqpoint{0.000000in}{0.027778in}}{%
\pgfpathmoveto{\pgfqpoint{0.000000in}{0.000000in}}%
\pgfpathlineto{\pgfqpoint{0.000000in}{0.027778in}}%
\pgfusepath{stroke,fill}%
}%
\begin{pgfscope}%
\pgfsys@transformshift{1.680205in}{2.202778in}%
\pgfsys@useobject{currentmarker}{}%
\end{pgfscope}%
\end{pgfscope}%
\begin{pgfscope}%
\pgfpathrectangle{\pgfqpoint{0.781944in}{0.552778in}}{\pgfqpoint{2.138715in}{1.650000in}}%
\pgfusepath{clip}%
\pgfsetrectcap%
\pgfsetroundjoin%
\pgfsetlinewidth{0.803000pt}%
\definecolor{currentstroke}{rgb}{0.690196,0.690196,0.690196}%
\pgfsetstrokecolor{currentstroke}%
\pgfsetstrokeopacity{0.300000}%
\pgfsetdash{}{0pt}%
\pgfpathmoveto{\pgfqpoint{1.722979in}{0.552778in}}%
\pgfpathlineto{\pgfqpoint{1.722979in}{2.202778in}}%
\pgfusepath{stroke}%
\end{pgfscope}%
\begin{pgfscope}%
\pgfsetbuttcap%
\pgfsetroundjoin%
\definecolor{currentfill}{rgb}{0.000000,0.000000,0.000000}%
\pgfsetfillcolor{currentfill}%
\pgfsetlinewidth{0.602250pt}%
\definecolor{currentstroke}{rgb}{0.000000,0.000000,0.000000}%
\pgfsetstrokecolor{currentstroke}%
\pgfsetdash{}{0pt}%
\pgfsys@defobject{currentmarker}{\pgfqpoint{0.000000in}{-0.027778in}}{\pgfqpoint{0.000000in}{0.000000in}}{%
\pgfpathmoveto{\pgfqpoint{0.000000in}{0.000000in}}%
\pgfpathlineto{\pgfqpoint{0.000000in}{-0.027778in}}%
\pgfusepath{stroke,fill}%
}%
\begin{pgfscope}%
\pgfsys@transformshift{1.722979in}{0.552778in}%
\pgfsys@useobject{currentmarker}{}%
\end{pgfscope}%
\end{pgfscope}%
\begin{pgfscope}%
\pgfsetbuttcap%
\pgfsetroundjoin%
\definecolor{currentfill}{rgb}{0.000000,0.000000,0.000000}%
\pgfsetfillcolor{currentfill}%
\pgfsetlinewidth{0.602250pt}%
\definecolor{currentstroke}{rgb}{0.000000,0.000000,0.000000}%
\pgfsetstrokecolor{currentstroke}%
\pgfsetdash{}{0pt}%
\pgfsys@defobject{currentmarker}{\pgfqpoint{0.000000in}{0.000000in}}{\pgfqpoint{0.000000in}{0.027778in}}{%
\pgfpathmoveto{\pgfqpoint{0.000000in}{0.000000in}}%
\pgfpathlineto{\pgfqpoint{0.000000in}{0.027778in}}%
\pgfusepath{stroke,fill}%
}%
\begin{pgfscope}%
\pgfsys@transformshift{1.722979in}{2.202778in}%
\pgfsys@useobject{currentmarker}{}%
\end{pgfscope}%
\end{pgfscope}%
\begin{pgfscope}%
\pgfpathrectangle{\pgfqpoint{0.781944in}{0.552778in}}{\pgfqpoint{2.138715in}{1.650000in}}%
\pgfusepath{clip}%
\pgfsetrectcap%
\pgfsetroundjoin%
\pgfsetlinewidth{0.803000pt}%
\definecolor{currentstroke}{rgb}{0.690196,0.690196,0.690196}%
\pgfsetstrokecolor{currentstroke}%
\pgfsetstrokeopacity{0.300000}%
\pgfsetdash{}{0pt}%
\pgfpathmoveto{\pgfqpoint{1.765753in}{0.552778in}}%
\pgfpathlineto{\pgfqpoint{1.765753in}{2.202778in}}%
\pgfusepath{stroke}%
\end{pgfscope}%
\begin{pgfscope}%
\pgfsetbuttcap%
\pgfsetroundjoin%
\definecolor{currentfill}{rgb}{0.000000,0.000000,0.000000}%
\pgfsetfillcolor{currentfill}%
\pgfsetlinewidth{0.602250pt}%
\definecolor{currentstroke}{rgb}{0.000000,0.000000,0.000000}%
\pgfsetstrokecolor{currentstroke}%
\pgfsetdash{}{0pt}%
\pgfsys@defobject{currentmarker}{\pgfqpoint{0.000000in}{-0.027778in}}{\pgfqpoint{0.000000in}{0.000000in}}{%
\pgfpathmoveto{\pgfqpoint{0.000000in}{0.000000in}}%
\pgfpathlineto{\pgfqpoint{0.000000in}{-0.027778in}}%
\pgfusepath{stroke,fill}%
}%
\begin{pgfscope}%
\pgfsys@transformshift{1.765753in}{0.552778in}%
\pgfsys@useobject{currentmarker}{}%
\end{pgfscope}%
\end{pgfscope}%
\begin{pgfscope}%
\pgfsetbuttcap%
\pgfsetroundjoin%
\definecolor{currentfill}{rgb}{0.000000,0.000000,0.000000}%
\pgfsetfillcolor{currentfill}%
\pgfsetlinewidth{0.602250pt}%
\definecolor{currentstroke}{rgb}{0.000000,0.000000,0.000000}%
\pgfsetstrokecolor{currentstroke}%
\pgfsetdash{}{0pt}%
\pgfsys@defobject{currentmarker}{\pgfqpoint{0.000000in}{0.000000in}}{\pgfqpoint{0.000000in}{0.027778in}}{%
\pgfpathmoveto{\pgfqpoint{0.000000in}{0.000000in}}%
\pgfpathlineto{\pgfqpoint{0.000000in}{0.027778in}}%
\pgfusepath{stroke,fill}%
}%
\begin{pgfscope}%
\pgfsys@transformshift{1.765753in}{2.202778in}%
\pgfsys@useobject{currentmarker}{}%
\end{pgfscope}%
\end{pgfscope}%
\begin{pgfscope}%
\pgfpathrectangle{\pgfqpoint{0.781944in}{0.552778in}}{\pgfqpoint{2.138715in}{1.650000in}}%
\pgfusepath{clip}%
\pgfsetrectcap%
\pgfsetroundjoin%
\pgfsetlinewidth{0.803000pt}%
\definecolor{currentstroke}{rgb}{0.690196,0.690196,0.690196}%
\pgfsetstrokecolor{currentstroke}%
\pgfsetstrokeopacity{0.300000}%
\pgfsetdash{}{0pt}%
\pgfpathmoveto{\pgfqpoint{1.808528in}{0.552778in}}%
\pgfpathlineto{\pgfqpoint{1.808528in}{2.202778in}}%
\pgfusepath{stroke}%
\end{pgfscope}%
\begin{pgfscope}%
\pgfsetbuttcap%
\pgfsetroundjoin%
\definecolor{currentfill}{rgb}{0.000000,0.000000,0.000000}%
\pgfsetfillcolor{currentfill}%
\pgfsetlinewidth{0.602250pt}%
\definecolor{currentstroke}{rgb}{0.000000,0.000000,0.000000}%
\pgfsetstrokecolor{currentstroke}%
\pgfsetdash{}{0pt}%
\pgfsys@defobject{currentmarker}{\pgfqpoint{0.000000in}{-0.027778in}}{\pgfqpoint{0.000000in}{0.000000in}}{%
\pgfpathmoveto{\pgfqpoint{0.000000in}{0.000000in}}%
\pgfpathlineto{\pgfqpoint{0.000000in}{-0.027778in}}%
\pgfusepath{stroke,fill}%
}%
\begin{pgfscope}%
\pgfsys@transformshift{1.808528in}{0.552778in}%
\pgfsys@useobject{currentmarker}{}%
\end{pgfscope}%
\end{pgfscope}%
\begin{pgfscope}%
\pgfsetbuttcap%
\pgfsetroundjoin%
\definecolor{currentfill}{rgb}{0.000000,0.000000,0.000000}%
\pgfsetfillcolor{currentfill}%
\pgfsetlinewidth{0.602250pt}%
\definecolor{currentstroke}{rgb}{0.000000,0.000000,0.000000}%
\pgfsetstrokecolor{currentstroke}%
\pgfsetdash{}{0pt}%
\pgfsys@defobject{currentmarker}{\pgfqpoint{0.000000in}{0.000000in}}{\pgfqpoint{0.000000in}{0.027778in}}{%
\pgfpathmoveto{\pgfqpoint{0.000000in}{0.000000in}}%
\pgfpathlineto{\pgfqpoint{0.000000in}{0.027778in}}%
\pgfusepath{stroke,fill}%
}%
\begin{pgfscope}%
\pgfsys@transformshift{1.808528in}{2.202778in}%
\pgfsys@useobject{currentmarker}{}%
\end{pgfscope}%
\end{pgfscope}%
\begin{pgfscope}%
\pgfpathrectangle{\pgfqpoint{0.781944in}{0.552778in}}{\pgfqpoint{2.138715in}{1.650000in}}%
\pgfusepath{clip}%
\pgfsetrectcap%
\pgfsetroundjoin%
\pgfsetlinewidth{0.803000pt}%
\definecolor{currentstroke}{rgb}{0.690196,0.690196,0.690196}%
\pgfsetstrokecolor{currentstroke}%
\pgfsetstrokeopacity{0.300000}%
\pgfsetdash{}{0pt}%
\pgfpathmoveto{\pgfqpoint{1.894076in}{0.552778in}}%
\pgfpathlineto{\pgfqpoint{1.894076in}{2.202778in}}%
\pgfusepath{stroke}%
\end{pgfscope}%
\begin{pgfscope}%
\pgfsetbuttcap%
\pgfsetroundjoin%
\definecolor{currentfill}{rgb}{0.000000,0.000000,0.000000}%
\pgfsetfillcolor{currentfill}%
\pgfsetlinewidth{0.602250pt}%
\definecolor{currentstroke}{rgb}{0.000000,0.000000,0.000000}%
\pgfsetstrokecolor{currentstroke}%
\pgfsetdash{}{0pt}%
\pgfsys@defobject{currentmarker}{\pgfqpoint{0.000000in}{-0.027778in}}{\pgfqpoint{0.000000in}{0.000000in}}{%
\pgfpathmoveto{\pgfqpoint{0.000000in}{0.000000in}}%
\pgfpathlineto{\pgfqpoint{0.000000in}{-0.027778in}}%
\pgfusepath{stroke,fill}%
}%
\begin{pgfscope}%
\pgfsys@transformshift{1.894076in}{0.552778in}%
\pgfsys@useobject{currentmarker}{}%
\end{pgfscope}%
\end{pgfscope}%
\begin{pgfscope}%
\pgfsetbuttcap%
\pgfsetroundjoin%
\definecolor{currentfill}{rgb}{0.000000,0.000000,0.000000}%
\pgfsetfillcolor{currentfill}%
\pgfsetlinewidth{0.602250pt}%
\definecolor{currentstroke}{rgb}{0.000000,0.000000,0.000000}%
\pgfsetstrokecolor{currentstroke}%
\pgfsetdash{}{0pt}%
\pgfsys@defobject{currentmarker}{\pgfqpoint{0.000000in}{0.000000in}}{\pgfqpoint{0.000000in}{0.027778in}}{%
\pgfpathmoveto{\pgfqpoint{0.000000in}{0.000000in}}%
\pgfpathlineto{\pgfqpoint{0.000000in}{0.027778in}}%
\pgfusepath{stroke,fill}%
}%
\begin{pgfscope}%
\pgfsys@transformshift{1.894076in}{2.202778in}%
\pgfsys@useobject{currentmarker}{}%
\end{pgfscope}%
\end{pgfscope}%
\begin{pgfscope}%
\pgfpathrectangle{\pgfqpoint{0.781944in}{0.552778in}}{\pgfqpoint{2.138715in}{1.650000in}}%
\pgfusepath{clip}%
\pgfsetrectcap%
\pgfsetroundjoin%
\pgfsetlinewidth{0.803000pt}%
\definecolor{currentstroke}{rgb}{0.690196,0.690196,0.690196}%
\pgfsetstrokecolor{currentstroke}%
\pgfsetstrokeopacity{0.300000}%
\pgfsetdash{}{0pt}%
\pgfpathmoveto{\pgfqpoint{1.936851in}{0.552778in}}%
\pgfpathlineto{\pgfqpoint{1.936851in}{2.202778in}}%
\pgfusepath{stroke}%
\end{pgfscope}%
\begin{pgfscope}%
\pgfsetbuttcap%
\pgfsetroundjoin%
\definecolor{currentfill}{rgb}{0.000000,0.000000,0.000000}%
\pgfsetfillcolor{currentfill}%
\pgfsetlinewidth{0.602250pt}%
\definecolor{currentstroke}{rgb}{0.000000,0.000000,0.000000}%
\pgfsetstrokecolor{currentstroke}%
\pgfsetdash{}{0pt}%
\pgfsys@defobject{currentmarker}{\pgfqpoint{0.000000in}{-0.027778in}}{\pgfqpoint{0.000000in}{0.000000in}}{%
\pgfpathmoveto{\pgfqpoint{0.000000in}{0.000000in}}%
\pgfpathlineto{\pgfqpoint{0.000000in}{-0.027778in}}%
\pgfusepath{stroke,fill}%
}%
\begin{pgfscope}%
\pgfsys@transformshift{1.936851in}{0.552778in}%
\pgfsys@useobject{currentmarker}{}%
\end{pgfscope}%
\end{pgfscope}%
\begin{pgfscope}%
\pgfsetbuttcap%
\pgfsetroundjoin%
\definecolor{currentfill}{rgb}{0.000000,0.000000,0.000000}%
\pgfsetfillcolor{currentfill}%
\pgfsetlinewidth{0.602250pt}%
\definecolor{currentstroke}{rgb}{0.000000,0.000000,0.000000}%
\pgfsetstrokecolor{currentstroke}%
\pgfsetdash{}{0pt}%
\pgfsys@defobject{currentmarker}{\pgfqpoint{0.000000in}{0.000000in}}{\pgfqpoint{0.000000in}{0.027778in}}{%
\pgfpathmoveto{\pgfqpoint{0.000000in}{0.000000in}}%
\pgfpathlineto{\pgfqpoint{0.000000in}{0.027778in}}%
\pgfusepath{stroke,fill}%
}%
\begin{pgfscope}%
\pgfsys@transformshift{1.936851in}{2.202778in}%
\pgfsys@useobject{currentmarker}{}%
\end{pgfscope}%
\end{pgfscope}%
\begin{pgfscope}%
\pgfpathrectangle{\pgfqpoint{0.781944in}{0.552778in}}{\pgfqpoint{2.138715in}{1.650000in}}%
\pgfusepath{clip}%
\pgfsetrectcap%
\pgfsetroundjoin%
\pgfsetlinewidth{0.803000pt}%
\definecolor{currentstroke}{rgb}{0.690196,0.690196,0.690196}%
\pgfsetstrokecolor{currentstroke}%
\pgfsetstrokeopacity{0.300000}%
\pgfsetdash{}{0pt}%
\pgfpathmoveto{\pgfqpoint{1.979625in}{0.552778in}}%
\pgfpathlineto{\pgfqpoint{1.979625in}{2.202778in}}%
\pgfusepath{stroke}%
\end{pgfscope}%
\begin{pgfscope}%
\pgfsetbuttcap%
\pgfsetroundjoin%
\definecolor{currentfill}{rgb}{0.000000,0.000000,0.000000}%
\pgfsetfillcolor{currentfill}%
\pgfsetlinewidth{0.602250pt}%
\definecolor{currentstroke}{rgb}{0.000000,0.000000,0.000000}%
\pgfsetstrokecolor{currentstroke}%
\pgfsetdash{}{0pt}%
\pgfsys@defobject{currentmarker}{\pgfqpoint{0.000000in}{-0.027778in}}{\pgfqpoint{0.000000in}{0.000000in}}{%
\pgfpathmoveto{\pgfqpoint{0.000000in}{0.000000in}}%
\pgfpathlineto{\pgfqpoint{0.000000in}{-0.027778in}}%
\pgfusepath{stroke,fill}%
}%
\begin{pgfscope}%
\pgfsys@transformshift{1.979625in}{0.552778in}%
\pgfsys@useobject{currentmarker}{}%
\end{pgfscope}%
\end{pgfscope}%
\begin{pgfscope}%
\pgfsetbuttcap%
\pgfsetroundjoin%
\definecolor{currentfill}{rgb}{0.000000,0.000000,0.000000}%
\pgfsetfillcolor{currentfill}%
\pgfsetlinewidth{0.602250pt}%
\definecolor{currentstroke}{rgb}{0.000000,0.000000,0.000000}%
\pgfsetstrokecolor{currentstroke}%
\pgfsetdash{}{0pt}%
\pgfsys@defobject{currentmarker}{\pgfqpoint{0.000000in}{0.000000in}}{\pgfqpoint{0.000000in}{0.027778in}}{%
\pgfpathmoveto{\pgfqpoint{0.000000in}{0.000000in}}%
\pgfpathlineto{\pgfqpoint{0.000000in}{0.027778in}}%
\pgfusepath{stroke,fill}%
}%
\begin{pgfscope}%
\pgfsys@transformshift{1.979625in}{2.202778in}%
\pgfsys@useobject{currentmarker}{}%
\end{pgfscope}%
\end{pgfscope}%
\begin{pgfscope}%
\pgfpathrectangle{\pgfqpoint{0.781944in}{0.552778in}}{\pgfqpoint{2.138715in}{1.650000in}}%
\pgfusepath{clip}%
\pgfsetrectcap%
\pgfsetroundjoin%
\pgfsetlinewidth{0.803000pt}%
\definecolor{currentstroke}{rgb}{0.690196,0.690196,0.690196}%
\pgfsetstrokecolor{currentstroke}%
\pgfsetstrokeopacity{0.300000}%
\pgfsetdash{}{0pt}%
\pgfpathmoveto{\pgfqpoint{2.022399in}{0.552778in}}%
\pgfpathlineto{\pgfqpoint{2.022399in}{2.202778in}}%
\pgfusepath{stroke}%
\end{pgfscope}%
\begin{pgfscope}%
\pgfsetbuttcap%
\pgfsetroundjoin%
\definecolor{currentfill}{rgb}{0.000000,0.000000,0.000000}%
\pgfsetfillcolor{currentfill}%
\pgfsetlinewidth{0.602250pt}%
\definecolor{currentstroke}{rgb}{0.000000,0.000000,0.000000}%
\pgfsetstrokecolor{currentstroke}%
\pgfsetdash{}{0pt}%
\pgfsys@defobject{currentmarker}{\pgfqpoint{0.000000in}{-0.027778in}}{\pgfqpoint{0.000000in}{0.000000in}}{%
\pgfpathmoveto{\pgfqpoint{0.000000in}{0.000000in}}%
\pgfpathlineto{\pgfqpoint{0.000000in}{-0.027778in}}%
\pgfusepath{stroke,fill}%
}%
\begin{pgfscope}%
\pgfsys@transformshift{2.022399in}{0.552778in}%
\pgfsys@useobject{currentmarker}{}%
\end{pgfscope}%
\end{pgfscope}%
\begin{pgfscope}%
\pgfsetbuttcap%
\pgfsetroundjoin%
\definecolor{currentfill}{rgb}{0.000000,0.000000,0.000000}%
\pgfsetfillcolor{currentfill}%
\pgfsetlinewidth{0.602250pt}%
\definecolor{currentstroke}{rgb}{0.000000,0.000000,0.000000}%
\pgfsetstrokecolor{currentstroke}%
\pgfsetdash{}{0pt}%
\pgfsys@defobject{currentmarker}{\pgfqpoint{0.000000in}{0.000000in}}{\pgfqpoint{0.000000in}{0.027778in}}{%
\pgfpathmoveto{\pgfqpoint{0.000000in}{0.000000in}}%
\pgfpathlineto{\pgfqpoint{0.000000in}{0.027778in}}%
\pgfusepath{stroke,fill}%
}%
\begin{pgfscope}%
\pgfsys@transformshift{2.022399in}{2.202778in}%
\pgfsys@useobject{currentmarker}{}%
\end{pgfscope}%
\end{pgfscope}%
\begin{pgfscope}%
\pgfpathrectangle{\pgfqpoint{0.781944in}{0.552778in}}{\pgfqpoint{2.138715in}{1.650000in}}%
\pgfusepath{clip}%
\pgfsetrectcap%
\pgfsetroundjoin%
\pgfsetlinewidth{0.803000pt}%
\definecolor{currentstroke}{rgb}{0.690196,0.690196,0.690196}%
\pgfsetstrokecolor{currentstroke}%
\pgfsetstrokeopacity{0.300000}%
\pgfsetdash{}{0pt}%
\pgfpathmoveto{\pgfqpoint{2.065174in}{0.552778in}}%
\pgfpathlineto{\pgfqpoint{2.065174in}{2.202778in}}%
\pgfusepath{stroke}%
\end{pgfscope}%
\begin{pgfscope}%
\pgfsetbuttcap%
\pgfsetroundjoin%
\definecolor{currentfill}{rgb}{0.000000,0.000000,0.000000}%
\pgfsetfillcolor{currentfill}%
\pgfsetlinewidth{0.602250pt}%
\definecolor{currentstroke}{rgb}{0.000000,0.000000,0.000000}%
\pgfsetstrokecolor{currentstroke}%
\pgfsetdash{}{0pt}%
\pgfsys@defobject{currentmarker}{\pgfqpoint{0.000000in}{-0.027778in}}{\pgfqpoint{0.000000in}{0.000000in}}{%
\pgfpathmoveto{\pgfqpoint{0.000000in}{0.000000in}}%
\pgfpathlineto{\pgfqpoint{0.000000in}{-0.027778in}}%
\pgfusepath{stroke,fill}%
}%
\begin{pgfscope}%
\pgfsys@transformshift{2.065174in}{0.552778in}%
\pgfsys@useobject{currentmarker}{}%
\end{pgfscope}%
\end{pgfscope}%
\begin{pgfscope}%
\pgfsetbuttcap%
\pgfsetroundjoin%
\definecolor{currentfill}{rgb}{0.000000,0.000000,0.000000}%
\pgfsetfillcolor{currentfill}%
\pgfsetlinewidth{0.602250pt}%
\definecolor{currentstroke}{rgb}{0.000000,0.000000,0.000000}%
\pgfsetstrokecolor{currentstroke}%
\pgfsetdash{}{0pt}%
\pgfsys@defobject{currentmarker}{\pgfqpoint{0.000000in}{0.000000in}}{\pgfqpoint{0.000000in}{0.027778in}}{%
\pgfpathmoveto{\pgfqpoint{0.000000in}{0.000000in}}%
\pgfpathlineto{\pgfqpoint{0.000000in}{0.027778in}}%
\pgfusepath{stroke,fill}%
}%
\begin{pgfscope}%
\pgfsys@transformshift{2.065174in}{2.202778in}%
\pgfsys@useobject{currentmarker}{}%
\end{pgfscope}%
\end{pgfscope}%
\begin{pgfscope}%
\pgfpathrectangle{\pgfqpoint{0.781944in}{0.552778in}}{\pgfqpoint{2.138715in}{1.650000in}}%
\pgfusepath{clip}%
\pgfsetrectcap%
\pgfsetroundjoin%
\pgfsetlinewidth{0.803000pt}%
\definecolor{currentstroke}{rgb}{0.690196,0.690196,0.690196}%
\pgfsetstrokecolor{currentstroke}%
\pgfsetstrokeopacity{0.300000}%
\pgfsetdash{}{0pt}%
\pgfpathmoveto{\pgfqpoint{2.107948in}{0.552778in}}%
\pgfpathlineto{\pgfqpoint{2.107948in}{2.202778in}}%
\pgfusepath{stroke}%
\end{pgfscope}%
\begin{pgfscope}%
\pgfsetbuttcap%
\pgfsetroundjoin%
\definecolor{currentfill}{rgb}{0.000000,0.000000,0.000000}%
\pgfsetfillcolor{currentfill}%
\pgfsetlinewidth{0.602250pt}%
\definecolor{currentstroke}{rgb}{0.000000,0.000000,0.000000}%
\pgfsetstrokecolor{currentstroke}%
\pgfsetdash{}{0pt}%
\pgfsys@defobject{currentmarker}{\pgfqpoint{0.000000in}{-0.027778in}}{\pgfqpoint{0.000000in}{0.000000in}}{%
\pgfpathmoveto{\pgfqpoint{0.000000in}{0.000000in}}%
\pgfpathlineto{\pgfqpoint{0.000000in}{-0.027778in}}%
\pgfusepath{stroke,fill}%
}%
\begin{pgfscope}%
\pgfsys@transformshift{2.107948in}{0.552778in}%
\pgfsys@useobject{currentmarker}{}%
\end{pgfscope}%
\end{pgfscope}%
\begin{pgfscope}%
\pgfsetbuttcap%
\pgfsetroundjoin%
\definecolor{currentfill}{rgb}{0.000000,0.000000,0.000000}%
\pgfsetfillcolor{currentfill}%
\pgfsetlinewidth{0.602250pt}%
\definecolor{currentstroke}{rgb}{0.000000,0.000000,0.000000}%
\pgfsetstrokecolor{currentstroke}%
\pgfsetdash{}{0pt}%
\pgfsys@defobject{currentmarker}{\pgfqpoint{0.000000in}{0.000000in}}{\pgfqpoint{0.000000in}{0.027778in}}{%
\pgfpathmoveto{\pgfqpoint{0.000000in}{0.000000in}}%
\pgfpathlineto{\pgfqpoint{0.000000in}{0.027778in}}%
\pgfusepath{stroke,fill}%
}%
\begin{pgfscope}%
\pgfsys@transformshift{2.107948in}{2.202778in}%
\pgfsys@useobject{currentmarker}{}%
\end{pgfscope}%
\end{pgfscope}%
\begin{pgfscope}%
\pgfpathrectangle{\pgfqpoint{0.781944in}{0.552778in}}{\pgfqpoint{2.138715in}{1.650000in}}%
\pgfusepath{clip}%
\pgfsetrectcap%
\pgfsetroundjoin%
\pgfsetlinewidth{0.803000pt}%
\definecolor{currentstroke}{rgb}{0.690196,0.690196,0.690196}%
\pgfsetstrokecolor{currentstroke}%
\pgfsetstrokeopacity{0.300000}%
\pgfsetdash{}{0pt}%
\pgfpathmoveto{\pgfqpoint{2.150722in}{0.552778in}}%
\pgfpathlineto{\pgfqpoint{2.150722in}{2.202778in}}%
\pgfusepath{stroke}%
\end{pgfscope}%
\begin{pgfscope}%
\pgfsetbuttcap%
\pgfsetroundjoin%
\definecolor{currentfill}{rgb}{0.000000,0.000000,0.000000}%
\pgfsetfillcolor{currentfill}%
\pgfsetlinewidth{0.602250pt}%
\definecolor{currentstroke}{rgb}{0.000000,0.000000,0.000000}%
\pgfsetstrokecolor{currentstroke}%
\pgfsetdash{}{0pt}%
\pgfsys@defobject{currentmarker}{\pgfqpoint{0.000000in}{-0.027778in}}{\pgfqpoint{0.000000in}{0.000000in}}{%
\pgfpathmoveto{\pgfqpoint{0.000000in}{0.000000in}}%
\pgfpathlineto{\pgfqpoint{0.000000in}{-0.027778in}}%
\pgfusepath{stroke,fill}%
}%
\begin{pgfscope}%
\pgfsys@transformshift{2.150722in}{0.552778in}%
\pgfsys@useobject{currentmarker}{}%
\end{pgfscope}%
\end{pgfscope}%
\begin{pgfscope}%
\pgfsetbuttcap%
\pgfsetroundjoin%
\definecolor{currentfill}{rgb}{0.000000,0.000000,0.000000}%
\pgfsetfillcolor{currentfill}%
\pgfsetlinewidth{0.602250pt}%
\definecolor{currentstroke}{rgb}{0.000000,0.000000,0.000000}%
\pgfsetstrokecolor{currentstroke}%
\pgfsetdash{}{0pt}%
\pgfsys@defobject{currentmarker}{\pgfqpoint{0.000000in}{0.000000in}}{\pgfqpoint{0.000000in}{0.027778in}}{%
\pgfpathmoveto{\pgfqpoint{0.000000in}{0.000000in}}%
\pgfpathlineto{\pgfqpoint{0.000000in}{0.027778in}}%
\pgfusepath{stroke,fill}%
}%
\begin{pgfscope}%
\pgfsys@transformshift{2.150722in}{2.202778in}%
\pgfsys@useobject{currentmarker}{}%
\end{pgfscope}%
\end{pgfscope}%
\begin{pgfscope}%
\pgfpathrectangle{\pgfqpoint{0.781944in}{0.552778in}}{\pgfqpoint{2.138715in}{1.650000in}}%
\pgfusepath{clip}%
\pgfsetrectcap%
\pgfsetroundjoin%
\pgfsetlinewidth{0.803000pt}%
\definecolor{currentstroke}{rgb}{0.690196,0.690196,0.690196}%
\pgfsetstrokecolor{currentstroke}%
\pgfsetstrokeopacity{0.300000}%
\pgfsetdash{}{0pt}%
\pgfpathmoveto{\pgfqpoint{2.193497in}{0.552778in}}%
\pgfpathlineto{\pgfqpoint{2.193497in}{2.202778in}}%
\pgfusepath{stroke}%
\end{pgfscope}%
\begin{pgfscope}%
\pgfsetbuttcap%
\pgfsetroundjoin%
\definecolor{currentfill}{rgb}{0.000000,0.000000,0.000000}%
\pgfsetfillcolor{currentfill}%
\pgfsetlinewidth{0.602250pt}%
\definecolor{currentstroke}{rgb}{0.000000,0.000000,0.000000}%
\pgfsetstrokecolor{currentstroke}%
\pgfsetdash{}{0pt}%
\pgfsys@defobject{currentmarker}{\pgfqpoint{0.000000in}{-0.027778in}}{\pgfqpoint{0.000000in}{0.000000in}}{%
\pgfpathmoveto{\pgfqpoint{0.000000in}{0.000000in}}%
\pgfpathlineto{\pgfqpoint{0.000000in}{-0.027778in}}%
\pgfusepath{stroke,fill}%
}%
\begin{pgfscope}%
\pgfsys@transformshift{2.193497in}{0.552778in}%
\pgfsys@useobject{currentmarker}{}%
\end{pgfscope}%
\end{pgfscope}%
\begin{pgfscope}%
\pgfsetbuttcap%
\pgfsetroundjoin%
\definecolor{currentfill}{rgb}{0.000000,0.000000,0.000000}%
\pgfsetfillcolor{currentfill}%
\pgfsetlinewidth{0.602250pt}%
\definecolor{currentstroke}{rgb}{0.000000,0.000000,0.000000}%
\pgfsetstrokecolor{currentstroke}%
\pgfsetdash{}{0pt}%
\pgfsys@defobject{currentmarker}{\pgfqpoint{0.000000in}{0.000000in}}{\pgfqpoint{0.000000in}{0.027778in}}{%
\pgfpathmoveto{\pgfqpoint{0.000000in}{0.000000in}}%
\pgfpathlineto{\pgfqpoint{0.000000in}{0.027778in}}%
\pgfusepath{stroke,fill}%
}%
\begin{pgfscope}%
\pgfsys@transformshift{2.193497in}{2.202778in}%
\pgfsys@useobject{currentmarker}{}%
\end{pgfscope}%
\end{pgfscope}%
\begin{pgfscope}%
\pgfpathrectangle{\pgfqpoint{0.781944in}{0.552778in}}{\pgfqpoint{2.138715in}{1.650000in}}%
\pgfusepath{clip}%
\pgfsetrectcap%
\pgfsetroundjoin%
\pgfsetlinewidth{0.803000pt}%
\definecolor{currentstroke}{rgb}{0.690196,0.690196,0.690196}%
\pgfsetstrokecolor{currentstroke}%
\pgfsetstrokeopacity{0.300000}%
\pgfsetdash{}{0pt}%
\pgfpathmoveto{\pgfqpoint{2.236271in}{0.552778in}}%
\pgfpathlineto{\pgfqpoint{2.236271in}{2.202778in}}%
\pgfusepath{stroke}%
\end{pgfscope}%
\begin{pgfscope}%
\pgfsetbuttcap%
\pgfsetroundjoin%
\definecolor{currentfill}{rgb}{0.000000,0.000000,0.000000}%
\pgfsetfillcolor{currentfill}%
\pgfsetlinewidth{0.602250pt}%
\definecolor{currentstroke}{rgb}{0.000000,0.000000,0.000000}%
\pgfsetstrokecolor{currentstroke}%
\pgfsetdash{}{0pt}%
\pgfsys@defobject{currentmarker}{\pgfqpoint{0.000000in}{-0.027778in}}{\pgfqpoint{0.000000in}{0.000000in}}{%
\pgfpathmoveto{\pgfqpoint{0.000000in}{0.000000in}}%
\pgfpathlineto{\pgfqpoint{0.000000in}{-0.027778in}}%
\pgfusepath{stroke,fill}%
}%
\begin{pgfscope}%
\pgfsys@transformshift{2.236271in}{0.552778in}%
\pgfsys@useobject{currentmarker}{}%
\end{pgfscope}%
\end{pgfscope}%
\begin{pgfscope}%
\pgfsetbuttcap%
\pgfsetroundjoin%
\definecolor{currentfill}{rgb}{0.000000,0.000000,0.000000}%
\pgfsetfillcolor{currentfill}%
\pgfsetlinewidth{0.602250pt}%
\definecolor{currentstroke}{rgb}{0.000000,0.000000,0.000000}%
\pgfsetstrokecolor{currentstroke}%
\pgfsetdash{}{0pt}%
\pgfsys@defobject{currentmarker}{\pgfqpoint{0.000000in}{0.000000in}}{\pgfqpoint{0.000000in}{0.027778in}}{%
\pgfpathmoveto{\pgfqpoint{0.000000in}{0.000000in}}%
\pgfpathlineto{\pgfqpoint{0.000000in}{0.027778in}}%
\pgfusepath{stroke,fill}%
}%
\begin{pgfscope}%
\pgfsys@transformshift{2.236271in}{2.202778in}%
\pgfsys@useobject{currentmarker}{}%
\end{pgfscope}%
\end{pgfscope}%
\begin{pgfscope}%
\pgfpathrectangle{\pgfqpoint{0.781944in}{0.552778in}}{\pgfqpoint{2.138715in}{1.650000in}}%
\pgfusepath{clip}%
\pgfsetrectcap%
\pgfsetroundjoin%
\pgfsetlinewidth{0.803000pt}%
\definecolor{currentstroke}{rgb}{0.690196,0.690196,0.690196}%
\pgfsetstrokecolor{currentstroke}%
\pgfsetstrokeopacity{0.300000}%
\pgfsetdash{}{0pt}%
\pgfpathmoveto{\pgfqpoint{2.321819in}{0.552778in}}%
\pgfpathlineto{\pgfqpoint{2.321819in}{2.202778in}}%
\pgfusepath{stroke}%
\end{pgfscope}%
\begin{pgfscope}%
\pgfsetbuttcap%
\pgfsetroundjoin%
\definecolor{currentfill}{rgb}{0.000000,0.000000,0.000000}%
\pgfsetfillcolor{currentfill}%
\pgfsetlinewidth{0.602250pt}%
\definecolor{currentstroke}{rgb}{0.000000,0.000000,0.000000}%
\pgfsetstrokecolor{currentstroke}%
\pgfsetdash{}{0pt}%
\pgfsys@defobject{currentmarker}{\pgfqpoint{0.000000in}{-0.027778in}}{\pgfqpoint{0.000000in}{0.000000in}}{%
\pgfpathmoveto{\pgfqpoint{0.000000in}{0.000000in}}%
\pgfpathlineto{\pgfqpoint{0.000000in}{-0.027778in}}%
\pgfusepath{stroke,fill}%
}%
\begin{pgfscope}%
\pgfsys@transformshift{2.321819in}{0.552778in}%
\pgfsys@useobject{currentmarker}{}%
\end{pgfscope}%
\end{pgfscope}%
\begin{pgfscope}%
\pgfsetbuttcap%
\pgfsetroundjoin%
\definecolor{currentfill}{rgb}{0.000000,0.000000,0.000000}%
\pgfsetfillcolor{currentfill}%
\pgfsetlinewidth{0.602250pt}%
\definecolor{currentstroke}{rgb}{0.000000,0.000000,0.000000}%
\pgfsetstrokecolor{currentstroke}%
\pgfsetdash{}{0pt}%
\pgfsys@defobject{currentmarker}{\pgfqpoint{0.000000in}{0.000000in}}{\pgfqpoint{0.000000in}{0.027778in}}{%
\pgfpathmoveto{\pgfqpoint{0.000000in}{0.000000in}}%
\pgfpathlineto{\pgfqpoint{0.000000in}{0.027778in}}%
\pgfusepath{stroke,fill}%
}%
\begin{pgfscope}%
\pgfsys@transformshift{2.321819in}{2.202778in}%
\pgfsys@useobject{currentmarker}{}%
\end{pgfscope}%
\end{pgfscope}%
\begin{pgfscope}%
\pgfpathrectangle{\pgfqpoint{0.781944in}{0.552778in}}{\pgfqpoint{2.138715in}{1.650000in}}%
\pgfusepath{clip}%
\pgfsetrectcap%
\pgfsetroundjoin%
\pgfsetlinewidth{0.803000pt}%
\definecolor{currentstroke}{rgb}{0.690196,0.690196,0.690196}%
\pgfsetstrokecolor{currentstroke}%
\pgfsetstrokeopacity{0.300000}%
\pgfsetdash{}{0pt}%
\pgfpathmoveto{\pgfqpoint{2.364594in}{0.552778in}}%
\pgfpathlineto{\pgfqpoint{2.364594in}{2.202778in}}%
\pgfusepath{stroke}%
\end{pgfscope}%
\begin{pgfscope}%
\pgfsetbuttcap%
\pgfsetroundjoin%
\definecolor{currentfill}{rgb}{0.000000,0.000000,0.000000}%
\pgfsetfillcolor{currentfill}%
\pgfsetlinewidth{0.602250pt}%
\definecolor{currentstroke}{rgb}{0.000000,0.000000,0.000000}%
\pgfsetstrokecolor{currentstroke}%
\pgfsetdash{}{0pt}%
\pgfsys@defobject{currentmarker}{\pgfqpoint{0.000000in}{-0.027778in}}{\pgfqpoint{0.000000in}{0.000000in}}{%
\pgfpathmoveto{\pgfqpoint{0.000000in}{0.000000in}}%
\pgfpathlineto{\pgfqpoint{0.000000in}{-0.027778in}}%
\pgfusepath{stroke,fill}%
}%
\begin{pgfscope}%
\pgfsys@transformshift{2.364594in}{0.552778in}%
\pgfsys@useobject{currentmarker}{}%
\end{pgfscope}%
\end{pgfscope}%
\begin{pgfscope}%
\pgfsetbuttcap%
\pgfsetroundjoin%
\definecolor{currentfill}{rgb}{0.000000,0.000000,0.000000}%
\pgfsetfillcolor{currentfill}%
\pgfsetlinewidth{0.602250pt}%
\definecolor{currentstroke}{rgb}{0.000000,0.000000,0.000000}%
\pgfsetstrokecolor{currentstroke}%
\pgfsetdash{}{0pt}%
\pgfsys@defobject{currentmarker}{\pgfqpoint{0.000000in}{0.000000in}}{\pgfqpoint{0.000000in}{0.027778in}}{%
\pgfpathmoveto{\pgfqpoint{0.000000in}{0.000000in}}%
\pgfpathlineto{\pgfqpoint{0.000000in}{0.027778in}}%
\pgfusepath{stroke,fill}%
}%
\begin{pgfscope}%
\pgfsys@transformshift{2.364594in}{2.202778in}%
\pgfsys@useobject{currentmarker}{}%
\end{pgfscope}%
\end{pgfscope}%
\begin{pgfscope}%
\pgfpathrectangle{\pgfqpoint{0.781944in}{0.552778in}}{\pgfqpoint{2.138715in}{1.650000in}}%
\pgfusepath{clip}%
\pgfsetrectcap%
\pgfsetroundjoin%
\pgfsetlinewidth{0.803000pt}%
\definecolor{currentstroke}{rgb}{0.690196,0.690196,0.690196}%
\pgfsetstrokecolor{currentstroke}%
\pgfsetstrokeopacity{0.300000}%
\pgfsetdash{}{0pt}%
\pgfpathmoveto{\pgfqpoint{2.407368in}{0.552778in}}%
\pgfpathlineto{\pgfqpoint{2.407368in}{2.202778in}}%
\pgfusepath{stroke}%
\end{pgfscope}%
\begin{pgfscope}%
\pgfsetbuttcap%
\pgfsetroundjoin%
\definecolor{currentfill}{rgb}{0.000000,0.000000,0.000000}%
\pgfsetfillcolor{currentfill}%
\pgfsetlinewidth{0.602250pt}%
\definecolor{currentstroke}{rgb}{0.000000,0.000000,0.000000}%
\pgfsetstrokecolor{currentstroke}%
\pgfsetdash{}{0pt}%
\pgfsys@defobject{currentmarker}{\pgfqpoint{0.000000in}{-0.027778in}}{\pgfqpoint{0.000000in}{0.000000in}}{%
\pgfpathmoveto{\pgfqpoint{0.000000in}{0.000000in}}%
\pgfpathlineto{\pgfqpoint{0.000000in}{-0.027778in}}%
\pgfusepath{stroke,fill}%
}%
\begin{pgfscope}%
\pgfsys@transformshift{2.407368in}{0.552778in}%
\pgfsys@useobject{currentmarker}{}%
\end{pgfscope}%
\end{pgfscope}%
\begin{pgfscope}%
\pgfsetbuttcap%
\pgfsetroundjoin%
\definecolor{currentfill}{rgb}{0.000000,0.000000,0.000000}%
\pgfsetfillcolor{currentfill}%
\pgfsetlinewidth{0.602250pt}%
\definecolor{currentstroke}{rgb}{0.000000,0.000000,0.000000}%
\pgfsetstrokecolor{currentstroke}%
\pgfsetdash{}{0pt}%
\pgfsys@defobject{currentmarker}{\pgfqpoint{0.000000in}{0.000000in}}{\pgfqpoint{0.000000in}{0.027778in}}{%
\pgfpathmoveto{\pgfqpoint{0.000000in}{0.000000in}}%
\pgfpathlineto{\pgfqpoint{0.000000in}{0.027778in}}%
\pgfusepath{stroke,fill}%
}%
\begin{pgfscope}%
\pgfsys@transformshift{2.407368in}{2.202778in}%
\pgfsys@useobject{currentmarker}{}%
\end{pgfscope}%
\end{pgfscope}%
\begin{pgfscope}%
\pgfpathrectangle{\pgfqpoint{0.781944in}{0.552778in}}{\pgfqpoint{2.138715in}{1.650000in}}%
\pgfusepath{clip}%
\pgfsetrectcap%
\pgfsetroundjoin%
\pgfsetlinewidth{0.803000pt}%
\definecolor{currentstroke}{rgb}{0.690196,0.690196,0.690196}%
\pgfsetstrokecolor{currentstroke}%
\pgfsetstrokeopacity{0.300000}%
\pgfsetdash{}{0pt}%
\pgfpathmoveto{\pgfqpoint{2.450142in}{0.552778in}}%
\pgfpathlineto{\pgfqpoint{2.450142in}{2.202778in}}%
\pgfusepath{stroke}%
\end{pgfscope}%
\begin{pgfscope}%
\pgfsetbuttcap%
\pgfsetroundjoin%
\definecolor{currentfill}{rgb}{0.000000,0.000000,0.000000}%
\pgfsetfillcolor{currentfill}%
\pgfsetlinewidth{0.602250pt}%
\definecolor{currentstroke}{rgb}{0.000000,0.000000,0.000000}%
\pgfsetstrokecolor{currentstroke}%
\pgfsetdash{}{0pt}%
\pgfsys@defobject{currentmarker}{\pgfqpoint{0.000000in}{-0.027778in}}{\pgfqpoint{0.000000in}{0.000000in}}{%
\pgfpathmoveto{\pgfqpoint{0.000000in}{0.000000in}}%
\pgfpathlineto{\pgfqpoint{0.000000in}{-0.027778in}}%
\pgfusepath{stroke,fill}%
}%
\begin{pgfscope}%
\pgfsys@transformshift{2.450142in}{0.552778in}%
\pgfsys@useobject{currentmarker}{}%
\end{pgfscope}%
\end{pgfscope}%
\begin{pgfscope}%
\pgfsetbuttcap%
\pgfsetroundjoin%
\definecolor{currentfill}{rgb}{0.000000,0.000000,0.000000}%
\pgfsetfillcolor{currentfill}%
\pgfsetlinewidth{0.602250pt}%
\definecolor{currentstroke}{rgb}{0.000000,0.000000,0.000000}%
\pgfsetstrokecolor{currentstroke}%
\pgfsetdash{}{0pt}%
\pgfsys@defobject{currentmarker}{\pgfqpoint{0.000000in}{0.000000in}}{\pgfqpoint{0.000000in}{0.027778in}}{%
\pgfpathmoveto{\pgfqpoint{0.000000in}{0.000000in}}%
\pgfpathlineto{\pgfqpoint{0.000000in}{0.027778in}}%
\pgfusepath{stroke,fill}%
}%
\begin{pgfscope}%
\pgfsys@transformshift{2.450142in}{2.202778in}%
\pgfsys@useobject{currentmarker}{}%
\end{pgfscope}%
\end{pgfscope}%
\begin{pgfscope}%
\pgfpathrectangle{\pgfqpoint{0.781944in}{0.552778in}}{\pgfqpoint{2.138715in}{1.650000in}}%
\pgfusepath{clip}%
\pgfsetrectcap%
\pgfsetroundjoin%
\pgfsetlinewidth{0.803000pt}%
\definecolor{currentstroke}{rgb}{0.690196,0.690196,0.690196}%
\pgfsetstrokecolor{currentstroke}%
\pgfsetstrokeopacity{0.300000}%
\pgfsetdash{}{0pt}%
\pgfpathmoveto{\pgfqpoint{2.492917in}{0.552778in}}%
\pgfpathlineto{\pgfqpoint{2.492917in}{2.202778in}}%
\pgfusepath{stroke}%
\end{pgfscope}%
\begin{pgfscope}%
\pgfsetbuttcap%
\pgfsetroundjoin%
\definecolor{currentfill}{rgb}{0.000000,0.000000,0.000000}%
\pgfsetfillcolor{currentfill}%
\pgfsetlinewidth{0.602250pt}%
\definecolor{currentstroke}{rgb}{0.000000,0.000000,0.000000}%
\pgfsetstrokecolor{currentstroke}%
\pgfsetdash{}{0pt}%
\pgfsys@defobject{currentmarker}{\pgfqpoint{0.000000in}{-0.027778in}}{\pgfqpoint{0.000000in}{0.000000in}}{%
\pgfpathmoveto{\pgfqpoint{0.000000in}{0.000000in}}%
\pgfpathlineto{\pgfqpoint{0.000000in}{-0.027778in}}%
\pgfusepath{stroke,fill}%
}%
\begin{pgfscope}%
\pgfsys@transformshift{2.492917in}{0.552778in}%
\pgfsys@useobject{currentmarker}{}%
\end{pgfscope}%
\end{pgfscope}%
\begin{pgfscope}%
\pgfsetbuttcap%
\pgfsetroundjoin%
\definecolor{currentfill}{rgb}{0.000000,0.000000,0.000000}%
\pgfsetfillcolor{currentfill}%
\pgfsetlinewidth{0.602250pt}%
\definecolor{currentstroke}{rgb}{0.000000,0.000000,0.000000}%
\pgfsetstrokecolor{currentstroke}%
\pgfsetdash{}{0pt}%
\pgfsys@defobject{currentmarker}{\pgfqpoint{0.000000in}{0.000000in}}{\pgfqpoint{0.000000in}{0.027778in}}{%
\pgfpathmoveto{\pgfqpoint{0.000000in}{0.000000in}}%
\pgfpathlineto{\pgfqpoint{0.000000in}{0.027778in}}%
\pgfusepath{stroke,fill}%
}%
\begin{pgfscope}%
\pgfsys@transformshift{2.492917in}{2.202778in}%
\pgfsys@useobject{currentmarker}{}%
\end{pgfscope}%
\end{pgfscope}%
\begin{pgfscope}%
\pgfpathrectangle{\pgfqpoint{0.781944in}{0.552778in}}{\pgfqpoint{2.138715in}{1.650000in}}%
\pgfusepath{clip}%
\pgfsetrectcap%
\pgfsetroundjoin%
\pgfsetlinewidth{0.803000pt}%
\definecolor{currentstroke}{rgb}{0.690196,0.690196,0.690196}%
\pgfsetstrokecolor{currentstroke}%
\pgfsetstrokeopacity{0.300000}%
\pgfsetdash{}{0pt}%
\pgfpathmoveto{\pgfqpoint{2.535691in}{0.552778in}}%
\pgfpathlineto{\pgfqpoint{2.535691in}{2.202778in}}%
\pgfusepath{stroke}%
\end{pgfscope}%
\begin{pgfscope}%
\pgfsetbuttcap%
\pgfsetroundjoin%
\definecolor{currentfill}{rgb}{0.000000,0.000000,0.000000}%
\pgfsetfillcolor{currentfill}%
\pgfsetlinewidth{0.602250pt}%
\definecolor{currentstroke}{rgb}{0.000000,0.000000,0.000000}%
\pgfsetstrokecolor{currentstroke}%
\pgfsetdash{}{0pt}%
\pgfsys@defobject{currentmarker}{\pgfqpoint{0.000000in}{-0.027778in}}{\pgfqpoint{0.000000in}{0.000000in}}{%
\pgfpathmoveto{\pgfqpoint{0.000000in}{0.000000in}}%
\pgfpathlineto{\pgfqpoint{0.000000in}{-0.027778in}}%
\pgfusepath{stroke,fill}%
}%
\begin{pgfscope}%
\pgfsys@transformshift{2.535691in}{0.552778in}%
\pgfsys@useobject{currentmarker}{}%
\end{pgfscope}%
\end{pgfscope}%
\begin{pgfscope}%
\pgfsetbuttcap%
\pgfsetroundjoin%
\definecolor{currentfill}{rgb}{0.000000,0.000000,0.000000}%
\pgfsetfillcolor{currentfill}%
\pgfsetlinewidth{0.602250pt}%
\definecolor{currentstroke}{rgb}{0.000000,0.000000,0.000000}%
\pgfsetstrokecolor{currentstroke}%
\pgfsetdash{}{0pt}%
\pgfsys@defobject{currentmarker}{\pgfqpoint{0.000000in}{0.000000in}}{\pgfqpoint{0.000000in}{0.027778in}}{%
\pgfpathmoveto{\pgfqpoint{0.000000in}{0.000000in}}%
\pgfpathlineto{\pgfqpoint{0.000000in}{0.027778in}}%
\pgfusepath{stroke,fill}%
}%
\begin{pgfscope}%
\pgfsys@transformshift{2.535691in}{2.202778in}%
\pgfsys@useobject{currentmarker}{}%
\end{pgfscope}%
\end{pgfscope}%
\begin{pgfscope}%
\pgfpathrectangle{\pgfqpoint{0.781944in}{0.552778in}}{\pgfqpoint{2.138715in}{1.650000in}}%
\pgfusepath{clip}%
\pgfsetrectcap%
\pgfsetroundjoin%
\pgfsetlinewidth{0.803000pt}%
\definecolor{currentstroke}{rgb}{0.690196,0.690196,0.690196}%
\pgfsetstrokecolor{currentstroke}%
\pgfsetstrokeopacity{0.300000}%
\pgfsetdash{}{0pt}%
\pgfpathmoveto{\pgfqpoint{2.578465in}{0.552778in}}%
\pgfpathlineto{\pgfqpoint{2.578465in}{2.202778in}}%
\pgfusepath{stroke}%
\end{pgfscope}%
\begin{pgfscope}%
\pgfsetbuttcap%
\pgfsetroundjoin%
\definecolor{currentfill}{rgb}{0.000000,0.000000,0.000000}%
\pgfsetfillcolor{currentfill}%
\pgfsetlinewidth{0.602250pt}%
\definecolor{currentstroke}{rgb}{0.000000,0.000000,0.000000}%
\pgfsetstrokecolor{currentstroke}%
\pgfsetdash{}{0pt}%
\pgfsys@defobject{currentmarker}{\pgfqpoint{0.000000in}{-0.027778in}}{\pgfqpoint{0.000000in}{0.000000in}}{%
\pgfpathmoveto{\pgfqpoint{0.000000in}{0.000000in}}%
\pgfpathlineto{\pgfqpoint{0.000000in}{-0.027778in}}%
\pgfusepath{stroke,fill}%
}%
\begin{pgfscope}%
\pgfsys@transformshift{2.578465in}{0.552778in}%
\pgfsys@useobject{currentmarker}{}%
\end{pgfscope}%
\end{pgfscope}%
\begin{pgfscope}%
\pgfsetbuttcap%
\pgfsetroundjoin%
\definecolor{currentfill}{rgb}{0.000000,0.000000,0.000000}%
\pgfsetfillcolor{currentfill}%
\pgfsetlinewidth{0.602250pt}%
\definecolor{currentstroke}{rgb}{0.000000,0.000000,0.000000}%
\pgfsetstrokecolor{currentstroke}%
\pgfsetdash{}{0pt}%
\pgfsys@defobject{currentmarker}{\pgfqpoint{0.000000in}{0.000000in}}{\pgfqpoint{0.000000in}{0.027778in}}{%
\pgfpathmoveto{\pgfqpoint{0.000000in}{0.000000in}}%
\pgfpathlineto{\pgfqpoint{0.000000in}{0.027778in}}%
\pgfusepath{stroke,fill}%
}%
\begin{pgfscope}%
\pgfsys@transformshift{2.578465in}{2.202778in}%
\pgfsys@useobject{currentmarker}{}%
\end{pgfscope}%
\end{pgfscope}%
\begin{pgfscope}%
\pgfpathrectangle{\pgfqpoint{0.781944in}{0.552778in}}{\pgfqpoint{2.138715in}{1.650000in}}%
\pgfusepath{clip}%
\pgfsetrectcap%
\pgfsetroundjoin%
\pgfsetlinewidth{0.803000pt}%
\definecolor{currentstroke}{rgb}{0.690196,0.690196,0.690196}%
\pgfsetstrokecolor{currentstroke}%
\pgfsetstrokeopacity{0.300000}%
\pgfsetdash{}{0pt}%
\pgfpathmoveto{\pgfqpoint{2.621240in}{0.552778in}}%
\pgfpathlineto{\pgfqpoint{2.621240in}{2.202778in}}%
\pgfusepath{stroke}%
\end{pgfscope}%
\begin{pgfscope}%
\pgfsetbuttcap%
\pgfsetroundjoin%
\definecolor{currentfill}{rgb}{0.000000,0.000000,0.000000}%
\pgfsetfillcolor{currentfill}%
\pgfsetlinewidth{0.602250pt}%
\definecolor{currentstroke}{rgb}{0.000000,0.000000,0.000000}%
\pgfsetstrokecolor{currentstroke}%
\pgfsetdash{}{0pt}%
\pgfsys@defobject{currentmarker}{\pgfqpoint{0.000000in}{-0.027778in}}{\pgfqpoint{0.000000in}{0.000000in}}{%
\pgfpathmoveto{\pgfqpoint{0.000000in}{0.000000in}}%
\pgfpathlineto{\pgfqpoint{0.000000in}{-0.027778in}}%
\pgfusepath{stroke,fill}%
}%
\begin{pgfscope}%
\pgfsys@transformshift{2.621240in}{0.552778in}%
\pgfsys@useobject{currentmarker}{}%
\end{pgfscope}%
\end{pgfscope}%
\begin{pgfscope}%
\pgfsetbuttcap%
\pgfsetroundjoin%
\definecolor{currentfill}{rgb}{0.000000,0.000000,0.000000}%
\pgfsetfillcolor{currentfill}%
\pgfsetlinewidth{0.602250pt}%
\definecolor{currentstroke}{rgb}{0.000000,0.000000,0.000000}%
\pgfsetstrokecolor{currentstroke}%
\pgfsetdash{}{0pt}%
\pgfsys@defobject{currentmarker}{\pgfqpoint{0.000000in}{0.000000in}}{\pgfqpoint{0.000000in}{0.027778in}}{%
\pgfpathmoveto{\pgfqpoint{0.000000in}{0.000000in}}%
\pgfpathlineto{\pgfqpoint{0.000000in}{0.027778in}}%
\pgfusepath{stroke,fill}%
}%
\begin{pgfscope}%
\pgfsys@transformshift{2.621240in}{2.202778in}%
\pgfsys@useobject{currentmarker}{}%
\end{pgfscope}%
\end{pgfscope}%
\begin{pgfscope}%
\pgfpathrectangle{\pgfqpoint{0.781944in}{0.552778in}}{\pgfqpoint{2.138715in}{1.650000in}}%
\pgfusepath{clip}%
\pgfsetrectcap%
\pgfsetroundjoin%
\pgfsetlinewidth{0.803000pt}%
\definecolor{currentstroke}{rgb}{0.690196,0.690196,0.690196}%
\pgfsetstrokecolor{currentstroke}%
\pgfsetstrokeopacity{0.300000}%
\pgfsetdash{}{0pt}%
\pgfpathmoveto{\pgfqpoint{2.664014in}{0.552778in}}%
\pgfpathlineto{\pgfqpoint{2.664014in}{2.202778in}}%
\pgfusepath{stroke}%
\end{pgfscope}%
\begin{pgfscope}%
\pgfsetbuttcap%
\pgfsetroundjoin%
\definecolor{currentfill}{rgb}{0.000000,0.000000,0.000000}%
\pgfsetfillcolor{currentfill}%
\pgfsetlinewidth{0.602250pt}%
\definecolor{currentstroke}{rgb}{0.000000,0.000000,0.000000}%
\pgfsetstrokecolor{currentstroke}%
\pgfsetdash{}{0pt}%
\pgfsys@defobject{currentmarker}{\pgfqpoint{0.000000in}{-0.027778in}}{\pgfqpoint{0.000000in}{0.000000in}}{%
\pgfpathmoveto{\pgfqpoint{0.000000in}{0.000000in}}%
\pgfpathlineto{\pgfqpoint{0.000000in}{-0.027778in}}%
\pgfusepath{stroke,fill}%
}%
\begin{pgfscope}%
\pgfsys@transformshift{2.664014in}{0.552778in}%
\pgfsys@useobject{currentmarker}{}%
\end{pgfscope}%
\end{pgfscope}%
\begin{pgfscope}%
\pgfsetbuttcap%
\pgfsetroundjoin%
\definecolor{currentfill}{rgb}{0.000000,0.000000,0.000000}%
\pgfsetfillcolor{currentfill}%
\pgfsetlinewidth{0.602250pt}%
\definecolor{currentstroke}{rgb}{0.000000,0.000000,0.000000}%
\pgfsetstrokecolor{currentstroke}%
\pgfsetdash{}{0pt}%
\pgfsys@defobject{currentmarker}{\pgfqpoint{0.000000in}{0.000000in}}{\pgfqpoint{0.000000in}{0.027778in}}{%
\pgfpathmoveto{\pgfqpoint{0.000000in}{0.000000in}}%
\pgfpathlineto{\pgfqpoint{0.000000in}{0.027778in}}%
\pgfusepath{stroke,fill}%
}%
\begin{pgfscope}%
\pgfsys@transformshift{2.664014in}{2.202778in}%
\pgfsys@useobject{currentmarker}{}%
\end{pgfscope}%
\end{pgfscope}%
\begin{pgfscope}%
\pgfpathrectangle{\pgfqpoint{0.781944in}{0.552778in}}{\pgfqpoint{2.138715in}{1.650000in}}%
\pgfusepath{clip}%
\pgfsetrectcap%
\pgfsetroundjoin%
\pgfsetlinewidth{0.803000pt}%
\definecolor{currentstroke}{rgb}{0.690196,0.690196,0.690196}%
\pgfsetstrokecolor{currentstroke}%
\pgfsetstrokeopacity{0.300000}%
\pgfsetdash{}{0pt}%
\pgfpathmoveto{\pgfqpoint{2.749563in}{0.552778in}}%
\pgfpathlineto{\pgfqpoint{2.749563in}{2.202778in}}%
\pgfusepath{stroke}%
\end{pgfscope}%
\begin{pgfscope}%
\pgfsetbuttcap%
\pgfsetroundjoin%
\definecolor{currentfill}{rgb}{0.000000,0.000000,0.000000}%
\pgfsetfillcolor{currentfill}%
\pgfsetlinewidth{0.602250pt}%
\definecolor{currentstroke}{rgb}{0.000000,0.000000,0.000000}%
\pgfsetstrokecolor{currentstroke}%
\pgfsetdash{}{0pt}%
\pgfsys@defobject{currentmarker}{\pgfqpoint{0.000000in}{-0.027778in}}{\pgfqpoint{0.000000in}{0.000000in}}{%
\pgfpathmoveto{\pgfqpoint{0.000000in}{0.000000in}}%
\pgfpathlineto{\pgfqpoint{0.000000in}{-0.027778in}}%
\pgfusepath{stroke,fill}%
}%
\begin{pgfscope}%
\pgfsys@transformshift{2.749563in}{0.552778in}%
\pgfsys@useobject{currentmarker}{}%
\end{pgfscope}%
\end{pgfscope}%
\begin{pgfscope}%
\pgfsetbuttcap%
\pgfsetroundjoin%
\definecolor{currentfill}{rgb}{0.000000,0.000000,0.000000}%
\pgfsetfillcolor{currentfill}%
\pgfsetlinewidth{0.602250pt}%
\definecolor{currentstroke}{rgb}{0.000000,0.000000,0.000000}%
\pgfsetstrokecolor{currentstroke}%
\pgfsetdash{}{0pt}%
\pgfsys@defobject{currentmarker}{\pgfqpoint{0.000000in}{0.000000in}}{\pgfqpoint{0.000000in}{0.027778in}}{%
\pgfpathmoveto{\pgfqpoint{0.000000in}{0.000000in}}%
\pgfpathlineto{\pgfqpoint{0.000000in}{0.027778in}}%
\pgfusepath{stroke,fill}%
}%
\begin{pgfscope}%
\pgfsys@transformshift{2.749563in}{2.202778in}%
\pgfsys@useobject{currentmarker}{}%
\end{pgfscope}%
\end{pgfscope}%
\begin{pgfscope}%
\pgfpathrectangle{\pgfqpoint{0.781944in}{0.552778in}}{\pgfqpoint{2.138715in}{1.650000in}}%
\pgfusepath{clip}%
\pgfsetrectcap%
\pgfsetroundjoin%
\pgfsetlinewidth{0.803000pt}%
\definecolor{currentstroke}{rgb}{0.690196,0.690196,0.690196}%
\pgfsetstrokecolor{currentstroke}%
\pgfsetstrokeopacity{0.300000}%
\pgfsetdash{}{0pt}%
\pgfpathmoveto{\pgfqpoint{2.792337in}{0.552778in}}%
\pgfpathlineto{\pgfqpoint{2.792337in}{2.202778in}}%
\pgfusepath{stroke}%
\end{pgfscope}%
\begin{pgfscope}%
\pgfsetbuttcap%
\pgfsetroundjoin%
\definecolor{currentfill}{rgb}{0.000000,0.000000,0.000000}%
\pgfsetfillcolor{currentfill}%
\pgfsetlinewidth{0.602250pt}%
\definecolor{currentstroke}{rgb}{0.000000,0.000000,0.000000}%
\pgfsetstrokecolor{currentstroke}%
\pgfsetdash{}{0pt}%
\pgfsys@defobject{currentmarker}{\pgfqpoint{0.000000in}{-0.027778in}}{\pgfqpoint{0.000000in}{0.000000in}}{%
\pgfpathmoveto{\pgfqpoint{0.000000in}{0.000000in}}%
\pgfpathlineto{\pgfqpoint{0.000000in}{-0.027778in}}%
\pgfusepath{stroke,fill}%
}%
\begin{pgfscope}%
\pgfsys@transformshift{2.792337in}{0.552778in}%
\pgfsys@useobject{currentmarker}{}%
\end{pgfscope}%
\end{pgfscope}%
\begin{pgfscope}%
\pgfsetbuttcap%
\pgfsetroundjoin%
\definecolor{currentfill}{rgb}{0.000000,0.000000,0.000000}%
\pgfsetfillcolor{currentfill}%
\pgfsetlinewidth{0.602250pt}%
\definecolor{currentstroke}{rgb}{0.000000,0.000000,0.000000}%
\pgfsetstrokecolor{currentstroke}%
\pgfsetdash{}{0pt}%
\pgfsys@defobject{currentmarker}{\pgfqpoint{0.000000in}{0.000000in}}{\pgfqpoint{0.000000in}{0.027778in}}{%
\pgfpathmoveto{\pgfqpoint{0.000000in}{0.000000in}}%
\pgfpathlineto{\pgfqpoint{0.000000in}{0.027778in}}%
\pgfusepath{stroke,fill}%
}%
\begin{pgfscope}%
\pgfsys@transformshift{2.792337in}{2.202778in}%
\pgfsys@useobject{currentmarker}{}%
\end{pgfscope}%
\end{pgfscope}%
\begin{pgfscope}%
\pgfpathrectangle{\pgfqpoint{0.781944in}{0.552778in}}{\pgfqpoint{2.138715in}{1.650000in}}%
\pgfusepath{clip}%
\pgfsetrectcap%
\pgfsetroundjoin%
\pgfsetlinewidth{0.803000pt}%
\definecolor{currentstroke}{rgb}{0.690196,0.690196,0.690196}%
\pgfsetstrokecolor{currentstroke}%
\pgfsetstrokeopacity{0.300000}%
\pgfsetdash{}{0pt}%
\pgfpathmoveto{\pgfqpoint{2.835111in}{0.552778in}}%
\pgfpathlineto{\pgfqpoint{2.835111in}{2.202778in}}%
\pgfusepath{stroke}%
\end{pgfscope}%
\begin{pgfscope}%
\pgfsetbuttcap%
\pgfsetroundjoin%
\definecolor{currentfill}{rgb}{0.000000,0.000000,0.000000}%
\pgfsetfillcolor{currentfill}%
\pgfsetlinewidth{0.602250pt}%
\definecolor{currentstroke}{rgb}{0.000000,0.000000,0.000000}%
\pgfsetstrokecolor{currentstroke}%
\pgfsetdash{}{0pt}%
\pgfsys@defobject{currentmarker}{\pgfqpoint{0.000000in}{-0.027778in}}{\pgfqpoint{0.000000in}{0.000000in}}{%
\pgfpathmoveto{\pgfqpoint{0.000000in}{0.000000in}}%
\pgfpathlineto{\pgfqpoint{0.000000in}{-0.027778in}}%
\pgfusepath{stroke,fill}%
}%
\begin{pgfscope}%
\pgfsys@transformshift{2.835111in}{0.552778in}%
\pgfsys@useobject{currentmarker}{}%
\end{pgfscope}%
\end{pgfscope}%
\begin{pgfscope}%
\pgfsetbuttcap%
\pgfsetroundjoin%
\definecolor{currentfill}{rgb}{0.000000,0.000000,0.000000}%
\pgfsetfillcolor{currentfill}%
\pgfsetlinewidth{0.602250pt}%
\definecolor{currentstroke}{rgb}{0.000000,0.000000,0.000000}%
\pgfsetstrokecolor{currentstroke}%
\pgfsetdash{}{0pt}%
\pgfsys@defobject{currentmarker}{\pgfqpoint{0.000000in}{0.000000in}}{\pgfqpoint{0.000000in}{0.027778in}}{%
\pgfpathmoveto{\pgfqpoint{0.000000in}{0.000000in}}%
\pgfpathlineto{\pgfqpoint{0.000000in}{0.027778in}}%
\pgfusepath{stroke,fill}%
}%
\begin{pgfscope}%
\pgfsys@transformshift{2.835111in}{2.202778in}%
\pgfsys@useobject{currentmarker}{}%
\end{pgfscope}%
\end{pgfscope}%
\begin{pgfscope}%
\pgfpathrectangle{\pgfqpoint{0.781944in}{0.552778in}}{\pgfqpoint{2.138715in}{1.650000in}}%
\pgfusepath{clip}%
\pgfsetrectcap%
\pgfsetroundjoin%
\pgfsetlinewidth{0.803000pt}%
\definecolor{currentstroke}{rgb}{0.690196,0.690196,0.690196}%
\pgfsetstrokecolor{currentstroke}%
\pgfsetstrokeopacity{0.300000}%
\pgfsetdash{}{0pt}%
\pgfpathmoveto{\pgfqpoint{2.877885in}{0.552778in}}%
\pgfpathlineto{\pgfqpoint{2.877885in}{2.202778in}}%
\pgfusepath{stroke}%
\end{pgfscope}%
\begin{pgfscope}%
\pgfsetbuttcap%
\pgfsetroundjoin%
\definecolor{currentfill}{rgb}{0.000000,0.000000,0.000000}%
\pgfsetfillcolor{currentfill}%
\pgfsetlinewidth{0.602250pt}%
\definecolor{currentstroke}{rgb}{0.000000,0.000000,0.000000}%
\pgfsetstrokecolor{currentstroke}%
\pgfsetdash{}{0pt}%
\pgfsys@defobject{currentmarker}{\pgfqpoint{0.000000in}{-0.027778in}}{\pgfqpoint{0.000000in}{0.000000in}}{%
\pgfpathmoveto{\pgfqpoint{0.000000in}{0.000000in}}%
\pgfpathlineto{\pgfqpoint{0.000000in}{-0.027778in}}%
\pgfusepath{stroke,fill}%
}%
\begin{pgfscope}%
\pgfsys@transformshift{2.877885in}{0.552778in}%
\pgfsys@useobject{currentmarker}{}%
\end{pgfscope}%
\end{pgfscope}%
\begin{pgfscope}%
\pgfsetbuttcap%
\pgfsetroundjoin%
\definecolor{currentfill}{rgb}{0.000000,0.000000,0.000000}%
\pgfsetfillcolor{currentfill}%
\pgfsetlinewidth{0.602250pt}%
\definecolor{currentstroke}{rgb}{0.000000,0.000000,0.000000}%
\pgfsetstrokecolor{currentstroke}%
\pgfsetdash{}{0pt}%
\pgfsys@defobject{currentmarker}{\pgfqpoint{0.000000in}{0.000000in}}{\pgfqpoint{0.000000in}{0.027778in}}{%
\pgfpathmoveto{\pgfqpoint{0.000000in}{0.000000in}}%
\pgfpathlineto{\pgfqpoint{0.000000in}{0.027778in}}%
\pgfusepath{stroke,fill}%
}%
\begin{pgfscope}%
\pgfsys@transformshift{2.877885in}{2.202778in}%
\pgfsys@useobject{currentmarker}{}%
\end{pgfscope}%
\end{pgfscope}%
\begin{pgfscope}%
\pgfpathrectangle{\pgfqpoint{0.781944in}{0.552778in}}{\pgfqpoint{2.138715in}{1.650000in}}%
\pgfusepath{clip}%
\pgfsetrectcap%
\pgfsetroundjoin%
\pgfsetlinewidth{0.803000pt}%
\definecolor{currentstroke}{rgb}{0.690196,0.690196,0.690196}%
\pgfsetstrokecolor{currentstroke}%
\pgfsetstrokeopacity{0.300000}%
\pgfsetdash{}{0pt}%
\pgfpathmoveto{\pgfqpoint{2.920660in}{0.552778in}}%
\pgfpathlineto{\pgfqpoint{2.920660in}{2.202778in}}%
\pgfusepath{stroke}%
\end{pgfscope}%
\begin{pgfscope}%
\pgfsetbuttcap%
\pgfsetroundjoin%
\definecolor{currentfill}{rgb}{0.000000,0.000000,0.000000}%
\pgfsetfillcolor{currentfill}%
\pgfsetlinewidth{0.602250pt}%
\definecolor{currentstroke}{rgb}{0.000000,0.000000,0.000000}%
\pgfsetstrokecolor{currentstroke}%
\pgfsetdash{}{0pt}%
\pgfsys@defobject{currentmarker}{\pgfqpoint{0.000000in}{-0.027778in}}{\pgfqpoint{0.000000in}{0.000000in}}{%
\pgfpathmoveto{\pgfqpoint{0.000000in}{0.000000in}}%
\pgfpathlineto{\pgfqpoint{0.000000in}{-0.027778in}}%
\pgfusepath{stroke,fill}%
}%
\begin{pgfscope}%
\pgfsys@transformshift{2.920660in}{0.552778in}%
\pgfsys@useobject{currentmarker}{}%
\end{pgfscope}%
\end{pgfscope}%
\begin{pgfscope}%
\pgfsetbuttcap%
\pgfsetroundjoin%
\definecolor{currentfill}{rgb}{0.000000,0.000000,0.000000}%
\pgfsetfillcolor{currentfill}%
\pgfsetlinewidth{0.602250pt}%
\definecolor{currentstroke}{rgb}{0.000000,0.000000,0.000000}%
\pgfsetstrokecolor{currentstroke}%
\pgfsetdash{}{0pt}%
\pgfsys@defobject{currentmarker}{\pgfqpoint{0.000000in}{0.000000in}}{\pgfqpoint{0.000000in}{0.027778in}}{%
\pgfpathmoveto{\pgfqpoint{0.000000in}{0.000000in}}%
\pgfpathlineto{\pgfqpoint{0.000000in}{0.027778in}}%
\pgfusepath{stroke,fill}%
}%
\begin{pgfscope}%
\pgfsys@transformshift{2.920660in}{2.202778in}%
\pgfsys@useobject{currentmarker}{}%
\end{pgfscope}%
\end{pgfscope}%
\begin{pgfscope}%
\definecolor{textcolor}{rgb}{0.000000,0.000000,0.000000}%
\pgfsetstrokecolor{textcolor}%
\pgfsetfillcolor{textcolor}%
\pgftext[x=1.851302in,y=0.276667in,,top]{\color{textcolor}\rmfamily\fontsize{10.000000}{12.000000}\selectfont Energie [keV]}%
\end{pgfscope}%
\begin{pgfscope}%
\pgfpathrectangle{\pgfqpoint{0.781944in}{0.552778in}}{\pgfqpoint{2.138715in}{1.650000in}}%
\pgfusepath{clip}%
\pgfsetrectcap%
\pgfsetroundjoin%
\pgfsetlinewidth{0.803000pt}%
\definecolor{currentstroke}{rgb}{0.690196,0.690196,0.690196}%
\pgfsetstrokecolor{currentstroke}%
\pgfsetstrokeopacity{0.800000}%
\pgfsetdash{}{0pt}%
\pgfpathmoveto{\pgfqpoint{0.781944in}{0.627778in}}%
\pgfpathlineto{\pgfqpoint{2.920660in}{0.627778in}}%
\pgfusepath{stroke}%
\end{pgfscope}%
\begin{pgfscope}%
\pgfsetbuttcap%
\pgfsetroundjoin%
\definecolor{currentfill}{rgb}{0.000000,0.000000,0.000000}%
\pgfsetfillcolor{currentfill}%
\pgfsetlinewidth{0.803000pt}%
\definecolor{currentstroke}{rgb}{0.000000,0.000000,0.000000}%
\pgfsetstrokecolor{currentstroke}%
\pgfsetdash{}{0pt}%
\pgfsys@defobject{currentmarker}{\pgfqpoint{-0.048611in}{0.000000in}}{\pgfqpoint{0.000000in}{0.000000in}}{%
\pgfpathmoveto{\pgfqpoint{0.000000in}{0.000000in}}%
\pgfpathlineto{\pgfqpoint{-0.048611in}{0.000000in}}%
\pgfusepath{stroke,fill}%
}%
\begin{pgfscope}%
\pgfsys@transformshift{0.781944in}{0.627778in}%
\pgfsys@useobject{currentmarker}{}%
\end{pgfscope}%
\end{pgfscope}%
\begin{pgfscope}%
\pgfsetbuttcap%
\pgfsetroundjoin%
\definecolor{currentfill}{rgb}{0.000000,0.000000,0.000000}%
\pgfsetfillcolor{currentfill}%
\pgfsetlinewidth{0.803000pt}%
\definecolor{currentstroke}{rgb}{0.000000,0.000000,0.000000}%
\pgfsetstrokecolor{currentstroke}%
\pgfsetdash{}{0pt}%
\pgfsys@defobject{currentmarker}{\pgfqpoint{0.000000in}{0.000000in}}{\pgfqpoint{0.048611in}{0.000000in}}{%
\pgfpathmoveto{\pgfqpoint{0.000000in}{0.000000in}}%
\pgfpathlineto{\pgfqpoint{0.048611in}{0.000000in}}%
\pgfusepath{stroke,fill}%
}%
\begin{pgfscope}%
\pgfsys@transformshift{2.920660in}{0.627778in}%
\pgfsys@useobject{currentmarker}{}%
\end{pgfscope}%
\end{pgfscope}%
\begin{pgfscope}%
\definecolor{textcolor}{rgb}{0.000000,0.000000,0.000000}%
\pgfsetstrokecolor{textcolor}%
\pgfsetfillcolor{textcolor}%
\pgftext[x=0.615278in,y=0.579583in,left,base]{\color{textcolor}\rmfamily\fontsize{10.000000}{12.000000}\selectfont 0}%
\end{pgfscope}%
\begin{pgfscope}%
\pgfpathrectangle{\pgfqpoint{0.781944in}{0.552778in}}{\pgfqpoint{2.138715in}{1.650000in}}%
\pgfusepath{clip}%
\pgfsetrectcap%
\pgfsetroundjoin%
\pgfsetlinewidth{0.803000pt}%
\definecolor{currentstroke}{rgb}{0.690196,0.690196,0.690196}%
\pgfsetstrokecolor{currentstroke}%
\pgfsetstrokeopacity{0.800000}%
\pgfsetdash{}{0pt}%
\pgfpathmoveto{\pgfqpoint{0.781944in}{1.045896in}}%
\pgfpathlineto{\pgfqpoint{2.920660in}{1.045896in}}%
\pgfusepath{stroke}%
\end{pgfscope}%
\begin{pgfscope}%
\pgfsetbuttcap%
\pgfsetroundjoin%
\definecolor{currentfill}{rgb}{0.000000,0.000000,0.000000}%
\pgfsetfillcolor{currentfill}%
\pgfsetlinewidth{0.803000pt}%
\definecolor{currentstroke}{rgb}{0.000000,0.000000,0.000000}%
\pgfsetstrokecolor{currentstroke}%
\pgfsetdash{}{0pt}%
\pgfsys@defobject{currentmarker}{\pgfqpoint{-0.048611in}{0.000000in}}{\pgfqpoint{0.000000in}{0.000000in}}{%
\pgfpathmoveto{\pgfqpoint{0.000000in}{0.000000in}}%
\pgfpathlineto{\pgfqpoint{-0.048611in}{0.000000in}}%
\pgfusepath{stroke,fill}%
}%
\begin{pgfscope}%
\pgfsys@transformshift{0.781944in}{1.045896in}%
\pgfsys@useobject{currentmarker}{}%
\end{pgfscope}%
\end{pgfscope}%
\begin{pgfscope}%
\pgfsetbuttcap%
\pgfsetroundjoin%
\definecolor{currentfill}{rgb}{0.000000,0.000000,0.000000}%
\pgfsetfillcolor{currentfill}%
\pgfsetlinewidth{0.803000pt}%
\definecolor{currentstroke}{rgb}{0.000000,0.000000,0.000000}%
\pgfsetstrokecolor{currentstroke}%
\pgfsetdash{}{0pt}%
\pgfsys@defobject{currentmarker}{\pgfqpoint{0.000000in}{0.000000in}}{\pgfqpoint{0.048611in}{0.000000in}}{%
\pgfpathmoveto{\pgfqpoint{0.000000in}{0.000000in}}%
\pgfpathlineto{\pgfqpoint{0.048611in}{0.000000in}}%
\pgfusepath{stroke,fill}%
}%
\begin{pgfscope}%
\pgfsys@transformshift{2.920660in}{1.045896in}%
\pgfsys@useobject{currentmarker}{}%
\end{pgfscope}%
\end{pgfscope}%
\begin{pgfscope}%
\definecolor{textcolor}{rgb}{0.000000,0.000000,0.000000}%
\pgfsetstrokecolor{textcolor}%
\pgfsetfillcolor{textcolor}%
\pgftext[x=0.545833in,y=0.997702in,left,base]{\color{textcolor}\rmfamily\fontsize{10.000000}{12.000000}\selectfont 80}%
\end{pgfscope}%
\begin{pgfscope}%
\pgfpathrectangle{\pgfqpoint{0.781944in}{0.552778in}}{\pgfqpoint{2.138715in}{1.650000in}}%
\pgfusepath{clip}%
\pgfsetrectcap%
\pgfsetroundjoin%
\pgfsetlinewidth{0.803000pt}%
\definecolor{currentstroke}{rgb}{0.690196,0.690196,0.690196}%
\pgfsetstrokecolor{currentstroke}%
\pgfsetstrokeopacity{0.800000}%
\pgfsetdash{}{0pt}%
\pgfpathmoveto{\pgfqpoint{0.781944in}{1.464015in}}%
\pgfpathlineto{\pgfqpoint{2.920660in}{1.464015in}}%
\pgfusepath{stroke}%
\end{pgfscope}%
\begin{pgfscope}%
\pgfsetbuttcap%
\pgfsetroundjoin%
\definecolor{currentfill}{rgb}{0.000000,0.000000,0.000000}%
\pgfsetfillcolor{currentfill}%
\pgfsetlinewidth{0.803000pt}%
\definecolor{currentstroke}{rgb}{0.000000,0.000000,0.000000}%
\pgfsetstrokecolor{currentstroke}%
\pgfsetdash{}{0pt}%
\pgfsys@defobject{currentmarker}{\pgfqpoint{-0.048611in}{0.000000in}}{\pgfqpoint{0.000000in}{0.000000in}}{%
\pgfpathmoveto{\pgfqpoint{0.000000in}{0.000000in}}%
\pgfpathlineto{\pgfqpoint{-0.048611in}{0.000000in}}%
\pgfusepath{stroke,fill}%
}%
\begin{pgfscope}%
\pgfsys@transformshift{0.781944in}{1.464015in}%
\pgfsys@useobject{currentmarker}{}%
\end{pgfscope}%
\end{pgfscope}%
\begin{pgfscope}%
\pgfsetbuttcap%
\pgfsetroundjoin%
\definecolor{currentfill}{rgb}{0.000000,0.000000,0.000000}%
\pgfsetfillcolor{currentfill}%
\pgfsetlinewidth{0.803000pt}%
\definecolor{currentstroke}{rgb}{0.000000,0.000000,0.000000}%
\pgfsetstrokecolor{currentstroke}%
\pgfsetdash{}{0pt}%
\pgfsys@defobject{currentmarker}{\pgfqpoint{0.000000in}{0.000000in}}{\pgfqpoint{0.048611in}{0.000000in}}{%
\pgfpathmoveto{\pgfqpoint{0.000000in}{0.000000in}}%
\pgfpathlineto{\pgfqpoint{0.048611in}{0.000000in}}%
\pgfusepath{stroke,fill}%
}%
\begin{pgfscope}%
\pgfsys@transformshift{2.920660in}{1.464015in}%
\pgfsys@useobject{currentmarker}{}%
\end{pgfscope}%
\end{pgfscope}%
\begin{pgfscope}%
\definecolor{textcolor}{rgb}{0.000000,0.000000,0.000000}%
\pgfsetstrokecolor{textcolor}%
\pgfsetfillcolor{textcolor}%
\pgftext[x=0.476389in,y=1.415820in,left,base]{\color{textcolor}\rmfamily\fontsize{10.000000}{12.000000}\selectfont 160}%
\end{pgfscope}%
\begin{pgfscope}%
\pgfpathrectangle{\pgfqpoint{0.781944in}{0.552778in}}{\pgfqpoint{2.138715in}{1.650000in}}%
\pgfusepath{clip}%
\pgfsetrectcap%
\pgfsetroundjoin%
\pgfsetlinewidth{0.803000pt}%
\definecolor{currentstroke}{rgb}{0.690196,0.690196,0.690196}%
\pgfsetstrokecolor{currentstroke}%
\pgfsetstrokeopacity{0.800000}%
\pgfsetdash{}{0pt}%
\pgfpathmoveto{\pgfqpoint{0.781944in}{1.882133in}}%
\pgfpathlineto{\pgfqpoint{2.920660in}{1.882133in}}%
\pgfusepath{stroke}%
\end{pgfscope}%
\begin{pgfscope}%
\pgfsetbuttcap%
\pgfsetroundjoin%
\definecolor{currentfill}{rgb}{0.000000,0.000000,0.000000}%
\pgfsetfillcolor{currentfill}%
\pgfsetlinewidth{0.803000pt}%
\definecolor{currentstroke}{rgb}{0.000000,0.000000,0.000000}%
\pgfsetstrokecolor{currentstroke}%
\pgfsetdash{}{0pt}%
\pgfsys@defobject{currentmarker}{\pgfqpoint{-0.048611in}{0.000000in}}{\pgfqpoint{0.000000in}{0.000000in}}{%
\pgfpathmoveto{\pgfqpoint{0.000000in}{0.000000in}}%
\pgfpathlineto{\pgfqpoint{-0.048611in}{0.000000in}}%
\pgfusepath{stroke,fill}%
}%
\begin{pgfscope}%
\pgfsys@transformshift{0.781944in}{1.882133in}%
\pgfsys@useobject{currentmarker}{}%
\end{pgfscope}%
\end{pgfscope}%
\begin{pgfscope}%
\pgfsetbuttcap%
\pgfsetroundjoin%
\definecolor{currentfill}{rgb}{0.000000,0.000000,0.000000}%
\pgfsetfillcolor{currentfill}%
\pgfsetlinewidth{0.803000pt}%
\definecolor{currentstroke}{rgb}{0.000000,0.000000,0.000000}%
\pgfsetstrokecolor{currentstroke}%
\pgfsetdash{}{0pt}%
\pgfsys@defobject{currentmarker}{\pgfqpoint{0.000000in}{0.000000in}}{\pgfqpoint{0.048611in}{0.000000in}}{%
\pgfpathmoveto{\pgfqpoint{0.000000in}{0.000000in}}%
\pgfpathlineto{\pgfqpoint{0.048611in}{0.000000in}}%
\pgfusepath{stroke,fill}%
}%
\begin{pgfscope}%
\pgfsys@transformshift{2.920660in}{1.882133in}%
\pgfsys@useobject{currentmarker}{}%
\end{pgfscope}%
\end{pgfscope}%
\begin{pgfscope}%
\definecolor{textcolor}{rgb}{0.000000,0.000000,0.000000}%
\pgfsetstrokecolor{textcolor}%
\pgfsetfillcolor{textcolor}%
\pgftext[x=0.476389in,y=1.833939in,left,base]{\color{textcolor}\rmfamily\fontsize{10.000000}{12.000000}\selectfont 240}%
\end{pgfscope}%
\begin{pgfscope}%
\pgfpathrectangle{\pgfqpoint{0.781944in}{0.552778in}}{\pgfqpoint{2.138715in}{1.650000in}}%
\pgfusepath{clip}%
\pgfsetrectcap%
\pgfsetroundjoin%
\pgfsetlinewidth{0.803000pt}%
\definecolor{currentstroke}{rgb}{0.690196,0.690196,0.690196}%
\pgfsetstrokecolor{currentstroke}%
\pgfsetstrokeopacity{0.300000}%
\pgfsetdash{}{0pt}%
\pgfpathmoveto{\pgfqpoint{0.781944in}{0.585966in}}%
\pgfpathlineto{\pgfqpoint{2.920660in}{0.585966in}}%
\pgfusepath{stroke}%
\end{pgfscope}%
\begin{pgfscope}%
\pgfsetbuttcap%
\pgfsetroundjoin%
\definecolor{currentfill}{rgb}{0.000000,0.000000,0.000000}%
\pgfsetfillcolor{currentfill}%
\pgfsetlinewidth{0.602250pt}%
\definecolor{currentstroke}{rgb}{0.000000,0.000000,0.000000}%
\pgfsetstrokecolor{currentstroke}%
\pgfsetdash{}{0pt}%
\pgfsys@defobject{currentmarker}{\pgfqpoint{-0.027778in}{0.000000in}}{\pgfqpoint{0.000000in}{0.000000in}}{%
\pgfpathmoveto{\pgfqpoint{0.000000in}{0.000000in}}%
\pgfpathlineto{\pgfqpoint{-0.027778in}{0.000000in}}%
\pgfusepath{stroke,fill}%
}%
\begin{pgfscope}%
\pgfsys@transformshift{0.781944in}{0.585966in}%
\pgfsys@useobject{currentmarker}{}%
\end{pgfscope}%
\end{pgfscope}%
\begin{pgfscope}%
\pgfsetbuttcap%
\pgfsetroundjoin%
\definecolor{currentfill}{rgb}{0.000000,0.000000,0.000000}%
\pgfsetfillcolor{currentfill}%
\pgfsetlinewidth{0.602250pt}%
\definecolor{currentstroke}{rgb}{0.000000,0.000000,0.000000}%
\pgfsetstrokecolor{currentstroke}%
\pgfsetdash{}{0pt}%
\pgfsys@defobject{currentmarker}{\pgfqpoint{0.000000in}{0.000000in}}{\pgfqpoint{0.027778in}{0.000000in}}{%
\pgfpathmoveto{\pgfqpoint{0.000000in}{0.000000in}}%
\pgfpathlineto{\pgfqpoint{0.027778in}{0.000000in}}%
\pgfusepath{stroke,fill}%
}%
\begin{pgfscope}%
\pgfsys@transformshift{2.920660in}{0.585966in}%
\pgfsys@useobject{currentmarker}{}%
\end{pgfscope}%
\end{pgfscope}%
\begin{pgfscope}%
\pgfpathrectangle{\pgfqpoint{0.781944in}{0.552778in}}{\pgfqpoint{2.138715in}{1.650000in}}%
\pgfusepath{clip}%
\pgfsetrectcap%
\pgfsetroundjoin%
\pgfsetlinewidth{0.803000pt}%
\definecolor{currentstroke}{rgb}{0.690196,0.690196,0.690196}%
\pgfsetstrokecolor{currentstroke}%
\pgfsetstrokeopacity{0.300000}%
\pgfsetdash{}{0pt}%
\pgfpathmoveto{\pgfqpoint{0.781944in}{0.669590in}}%
\pgfpathlineto{\pgfqpoint{2.920660in}{0.669590in}}%
\pgfusepath{stroke}%
\end{pgfscope}%
\begin{pgfscope}%
\pgfsetbuttcap%
\pgfsetroundjoin%
\definecolor{currentfill}{rgb}{0.000000,0.000000,0.000000}%
\pgfsetfillcolor{currentfill}%
\pgfsetlinewidth{0.602250pt}%
\definecolor{currentstroke}{rgb}{0.000000,0.000000,0.000000}%
\pgfsetstrokecolor{currentstroke}%
\pgfsetdash{}{0pt}%
\pgfsys@defobject{currentmarker}{\pgfqpoint{-0.027778in}{0.000000in}}{\pgfqpoint{0.000000in}{0.000000in}}{%
\pgfpathmoveto{\pgfqpoint{0.000000in}{0.000000in}}%
\pgfpathlineto{\pgfqpoint{-0.027778in}{0.000000in}}%
\pgfusepath{stroke,fill}%
}%
\begin{pgfscope}%
\pgfsys@transformshift{0.781944in}{0.669590in}%
\pgfsys@useobject{currentmarker}{}%
\end{pgfscope}%
\end{pgfscope}%
\begin{pgfscope}%
\pgfsetbuttcap%
\pgfsetroundjoin%
\definecolor{currentfill}{rgb}{0.000000,0.000000,0.000000}%
\pgfsetfillcolor{currentfill}%
\pgfsetlinewidth{0.602250pt}%
\definecolor{currentstroke}{rgb}{0.000000,0.000000,0.000000}%
\pgfsetstrokecolor{currentstroke}%
\pgfsetdash{}{0pt}%
\pgfsys@defobject{currentmarker}{\pgfqpoint{0.000000in}{0.000000in}}{\pgfqpoint{0.027778in}{0.000000in}}{%
\pgfpathmoveto{\pgfqpoint{0.000000in}{0.000000in}}%
\pgfpathlineto{\pgfqpoint{0.027778in}{0.000000in}}%
\pgfusepath{stroke,fill}%
}%
\begin{pgfscope}%
\pgfsys@transformshift{2.920660in}{0.669590in}%
\pgfsys@useobject{currentmarker}{}%
\end{pgfscope}%
\end{pgfscope}%
\begin{pgfscope}%
\pgfpathrectangle{\pgfqpoint{0.781944in}{0.552778in}}{\pgfqpoint{2.138715in}{1.650000in}}%
\pgfusepath{clip}%
\pgfsetrectcap%
\pgfsetroundjoin%
\pgfsetlinewidth{0.803000pt}%
\definecolor{currentstroke}{rgb}{0.690196,0.690196,0.690196}%
\pgfsetstrokecolor{currentstroke}%
\pgfsetstrokeopacity{0.300000}%
\pgfsetdash{}{0pt}%
\pgfpathmoveto{\pgfqpoint{0.781944in}{0.711401in}}%
\pgfpathlineto{\pgfqpoint{2.920660in}{0.711401in}}%
\pgfusepath{stroke}%
\end{pgfscope}%
\begin{pgfscope}%
\pgfsetbuttcap%
\pgfsetroundjoin%
\definecolor{currentfill}{rgb}{0.000000,0.000000,0.000000}%
\pgfsetfillcolor{currentfill}%
\pgfsetlinewidth{0.602250pt}%
\definecolor{currentstroke}{rgb}{0.000000,0.000000,0.000000}%
\pgfsetstrokecolor{currentstroke}%
\pgfsetdash{}{0pt}%
\pgfsys@defobject{currentmarker}{\pgfqpoint{-0.027778in}{0.000000in}}{\pgfqpoint{0.000000in}{0.000000in}}{%
\pgfpathmoveto{\pgfqpoint{0.000000in}{0.000000in}}%
\pgfpathlineto{\pgfqpoint{-0.027778in}{0.000000in}}%
\pgfusepath{stroke,fill}%
}%
\begin{pgfscope}%
\pgfsys@transformshift{0.781944in}{0.711401in}%
\pgfsys@useobject{currentmarker}{}%
\end{pgfscope}%
\end{pgfscope}%
\begin{pgfscope}%
\pgfsetbuttcap%
\pgfsetroundjoin%
\definecolor{currentfill}{rgb}{0.000000,0.000000,0.000000}%
\pgfsetfillcolor{currentfill}%
\pgfsetlinewidth{0.602250pt}%
\definecolor{currentstroke}{rgb}{0.000000,0.000000,0.000000}%
\pgfsetstrokecolor{currentstroke}%
\pgfsetdash{}{0pt}%
\pgfsys@defobject{currentmarker}{\pgfqpoint{0.000000in}{0.000000in}}{\pgfqpoint{0.027778in}{0.000000in}}{%
\pgfpathmoveto{\pgfqpoint{0.000000in}{0.000000in}}%
\pgfpathlineto{\pgfqpoint{0.027778in}{0.000000in}}%
\pgfusepath{stroke,fill}%
}%
\begin{pgfscope}%
\pgfsys@transformshift{2.920660in}{0.711401in}%
\pgfsys@useobject{currentmarker}{}%
\end{pgfscope}%
\end{pgfscope}%
\begin{pgfscope}%
\pgfpathrectangle{\pgfqpoint{0.781944in}{0.552778in}}{\pgfqpoint{2.138715in}{1.650000in}}%
\pgfusepath{clip}%
\pgfsetrectcap%
\pgfsetroundjoin%
\pgfsetlinewidth{0.803000pt}%
\definecolor{currentstroke}{rgb}{0.690196,0.690196,0.690196}%
\pgfsetstrokecolor{currentstroke}%
\pgfsetstrokeopacity{0.300000}%
\pgfsetdash{}{0pt}%
\pgfpathmoveto{\pgfqpoint{0.781944in}{0.753213in}}%
\pgfpathlineto{\pgfqpoint{2.920660in}{0.753213in}}%
\pgfusepath{stroke}%
\end{pgfscope}%
\begin{pgfscope}%
\pgfsetbuttcap%
\pgfsetroundjoin%
\definecolor{currentfill}{rgb}{0.000000,0.000000,0.000000}%
\pgfsetfillcolor{currentfill}%
\pgfsetlinewidth{0.602250pt}%
\definecolor{currentstroke}{rgb}{0.000000,0.000000,0.000000}%
\pgfsetstrokecolor{currentstroke}%
\pgfsetdash{}{0pt}%
\pgfsys@defobject{currentmarker}{\pgfqpoint{-0.027778in}{0.000000in}}{\pgfqpoint{0.000000in}{0.000000in}}{%
\pgfpathmoveto{\pgfqpoint{0.000000in}{0.000000in}}%
\pgfpathlineto{\pgfqpoint{-0.027778in}{0.000000in}}%
\pgfusepath{stroke,fill}%
}%
\begin{pgfscope}%
\pgfsys@transformshift{0.781944in}{0.753213in}%
\pgfsys@useobject{currentmarker}{}%
\end{pgfscope}%
\end{pgfscope}%
\begin{pgfscope}%
\pgfsetbuttcap%
\pgfsetroundjoin%
\definecolor{currentfill}{rgb}{0.000000,0.000000,0.000000}%
\pgfsetfillcolor{currentfill}%
\pgfsetlinewidth{0.602250pt}%
\definecolor{currentstroke}{rgb}{0.000000,0.000000,0.000000}%
\pgfsetstrokecolor{currentstroke}%
\pgfsetdash{}{0pt}%
\pgfsys@defobject{currentmarker}{\pgfqpoint{0.000000in}{0.000000in}}{\pgfqpoint{0.027778in}{0.000000in}}{%
\pgfpathmoveto{\pgfqpoint{0.000000in}{0.000000in}}%
\pgfpathlineto{\pgfqpoint{0.027778in}{0.000000in}}%
\pgfusepath{stroke,fill}%
}%
\begin{pgfscope}%
\pgfsys@transformshift{2.920660in}{0.753213in}%
\pgfsys@useobject{currentmarker}{}%
\end{pgfscope}%
\end{pgfscope}%
\begin{pgfscope}%
\pgfpathrectangle{\pgfqpoint{0.781944in}{0.552778in}}{\pgfqpoint{2.138715in}{1.650000in}}%
\pgfusepath{clip}%
\pgfsetrectcap%
\pgfsetroundjoin%
\pgfsetlinewidth{0.803000pt}%
\definecolor{currentstroke}{rgb}{0.690196,0.690196,0.690196}%
\pgfsetstrokecolor{currentstroke}%
\pgfsetstrokeopacity{0.300000}%
\pgfsetdash{}{0pt}%
\pgfpathmoveto{\pgfqpoint{0.781944in}{0.795025in}}%
\pgfpathlineto{\pgfqpoint{2.920660in}{0.795025in}}%
\pgfusepath{stroke}%
\end{pgfscope}%
\begin{pgfscope}%
\pgfsetbuttcap%
\pgfsetroundjoin%
\definecolor{currentfill}{rgb}{0.000000,0.000000,0.000000}%
\pgfsetfillcolor{currentfill}%
\pgfsetlinewidth{0.602250pt}%
\definecolor{currentstroke}{rgb}{0.000000,0.000000,0.000000}%
\pgfsetstrokecolor{currentstroke}%
\pgfsetdash{}{0pt}%
\pgfsys@defobject{currentmarker}{\pgfqpoint{-0.027778in}{0.000000in}}{\pgfqpoint{0.000000in}{0.000000in}}{%
\pgfpathmoveto{\pgfqpoint{0.000000in}{0.000000in}}%
\pgfpathlineto{\pgfqpoint{-0.027778in}{0.000000in}}%
\pgfusepath{stroke,fill}%
}%
\begin{pgfscope}%
\pgfsys@transformshift{0.781944in}{0.795025in}%
\pgfsys@useobject{currentmarker}{}%
\end{pgfscope}%
\end{pgfscope}%
\begin{pgfscope}%
\pgfsetbuttcap%
\pgfsetroundjoin%
\definecolor{currentfill}{rgb}{0.000000,0.000000,0.000000}%
\pgfsetfillcolor{currentfill}%
\pgfsetlinewidth{0.602250pt}%
\definecolor{currentstroke}{rgb}{0.000000,0.000000,0.000000}%
\pgfsetstrokecolor{currentstroke}%
\pgfsetdash{}{0pt}%
\pgfsys@defobject{currentmarker}{\pgfqpoint{0.000000in}{0.000000in}}{\pgfqpoint{0.027778in}{0.000000in}}{%
\pgfpathmoveto{\pgfqpoint{0.000000in}{0.000000in}}%
\pgfpathlineto{\pgfqpoint{0.027778in}{0.000000in}}%
\pgfusepath{stroke,fill}%
}%
\begin{pgfscope}%
\pgfsys@transformshift{2.920660in}{0.795025in}%
\pgfsys@useobject{currentmarker}{}%
\end{pgfscope}%
\end{pgfscope}%
\begin{pgfscope}%
\pgfpathrectangle{\pgfqpoint{0.781944in}{0.552778in}}{\pgfqpoint{2.138715in}{1.650000in}}%
\pgfusepath{clip}%
\pgfsetrectcap%
\pgfsetroundjoin%
\pgfsetlinewidth{0.803000pt}%
\definecolor{currentstroke}{rgb}{0.690196,0.690196,0.690196}%
\pgfsetstrokecolor{currentstroke}%
\pgfsetstrokeopacity{0.300000}%
\pgfsetdash{}{0pt}%
\pgfpathmoveto{\pgfqpoint{0.781944in}{0.836837in}}%
\pgfpathlineto{\pgfqpoint{2.920660in}{0.836837in}}%
\pgfusepath{stroke}%
\end{pgfscope}%
\begin{pgfscope}%
\pgfsetbuttcap%
\pgfsetroundjoin%
\definecolor{currentfill}{rgb}{0.000000,0.000000,0.000000}%
\pgfsetfillcolor{currentfill}%
\pgfsetlinewidth{0.602250pt}%
\definecolor{currentstroke}{rgb}{0.000000,0.000000,0.000000}%
\pgfsetstrokecolor{currentstroke}%
\pgfsetdash{}{0pt}%
\pgfsys@defobject{currentmarker}{\pgfqpoint{-0.027778in}{0.000000in}}{\pgfqpoint{0.000000in}{0.000000in}}{%
\pgfpathmoveto{\pgfqpoint{0.000000in}{0.000000in}}%
\pgfpathlineto{\pgfqpoint{-0.027778in}{0.000000in}}%
\pgfusepath{stroke,fill}%
}%
\begin{pgfscope}%
\pgfsys@transformshift{0.781944in}{0.836837in}%
\pgfsys@useobject{currentmarker}{}%
\end{pgfscope}%
\end{pgfscope}%
\begin{pgfscope}%
\pgfsetbuttcap%
\pgfsetroundjoin%
\definecolor{currentfill}{rgb}{0.000000,0.000000,0.000000}%
\pgfsetfillcolor{currentfill}%
\pgfsetlinewidth{0.602250pt}%
\definecolor{currentstroke}{rgb}{0.000000,0.000000,0.000000}%
\pgfsetstrokecolor{currentstroke}%
\pgfsetdash{}{0pt}%
\pgfsys@defobject{currentmarker}{\pgfqpoint{0.000000in}{0.000000in}}{\pgfqpoint{0.027778in}{0.000000in}}{%
\pgfpathmoveto{\pgfqpoint{0.000000in}{0.000000in}}%
\pgfpathlineto{\pgfqpoint{0.027778in}{0.000000in}}%
\pgfusepath{stroke,fill}%
}%
\begin{pgfscope}%
\pgfsys@transformshift{2.920660in}{0.836837in}%
\pgfsys@useobject{currentmarker}{}%
\end{pgfscope}%
\end{pgfscope}%
\begin{pgfscope}%
\pgfpathrectangle{\pgfqpoint{0.781944in}{0.552778in}}{\pgfqpoint{2.138715in}{1.650000in}}%
\pgfusepath{clip}%
\pgfsetrectcap%
\pgfsetroundjoin%
\pgfsetlinewidth{0.803000pt}%
\definecolor{currentstroke}{rgb}{0.690196,0.690196,0.690196}%
\pgfsetstrokecolor{currentstroke}%
\pgfsetstrokeopacity{0.300000}%
\pgfsetdash{}{0pt}%
\pgfpathmoveto{\pgfqpoint{0.781944in}{0.878649in}}%
\pgfpathlineto{\pgfqpoint{2.920660in}{0.878649in}}%
\pgfusepath{stroke}%
\end{pgfscope}%
\begin{pgfscope}%
\pgfsetbuttcap%
\pgfsetroundjoin%
\definecolor{currentfill}{rgb}{0.000000,0.000000,0.000000}%
\pgfsetfillcolor{currentfill}%
\pgfsetlinewidth{0.602250pt}%
\definecolor{currentstroke}{rgb}{0.000000,0.000000,0.000000}%
\pgfsetstrokecolor{currentstroke}%
\pgfsetdash{}{0pt}%
\pgfsys@defobject{currentmarker}{\pgfqpoint{-0.027778in}{0.000000in}}{\pgfqpoint{0.000000in}{0.000000in}}{%
\pgfpathmoveto{\pgfqpoint{0.000000in}{0.000000in}}%
\pgfpathlineto{\pgfqpoint{-0.027778in}{0.000000in}}%
\pgfusepath{stroke,fill}%
}%
\begin{pgfscope}%
\pgfsys@transformshift{0.781944in}{0.878649in}%
\pgfsys@useobject{currentmarker}{}%
\end{pgfscope}%
\end{pgfscope}%
\begin{pgfscope}%
\pgfsetbuttcap%
\pgfsetroundjoin%
\definecolor{currentfill}{rgb}{0.000000,0.000000,0.000000}%
\pgfsetfillcolor{currentfill}%
\pgfsetlinewidth{0.602250pt}%
\definecolor{currentstroke}{rgb}{0.000000,0.000000,0.000000}%
\pgfsetstrokecolor{currentstroke}%
\pgfsetdash{}{0pt}%
\pgfsys@defobject{currentmarker}{\pgfqpoint{0.000000in}{0.000000in}}{\pgfqpoint{0.027778in}{0.000000in}}{%
\pgfpathmoveto{\pgfqpoint{0.000000in}{0.000000in}}%
\pgfpathlineto{\pgfqpoint{0.027778in}{0.000000in}}%
\pgfusepath{stroke,fill}%
}%
\begin{pgfscope}%
\pgfsys@transformshift{2.920660in}{0.878649in}%
\pgfsys@useobject{currentmarker}{}%
\end{pgfscope}%
\end{pgfscope}%
\begin{pgfscope}%
\pgfpathrectangle{\pgfqpoint{0.781944in}{0.552778in}}{\pgfqpoint{2.138715in}{1.650000in}}%
\pgfusepath{clip}%
\pgfsetrectcap%
\pgfsetroundjoin%
\pgfsetlinewidth{0.803000pt}%
\definecolor{currentstroke}{rgb}{0.690196,0.690196,0.690196}%
\pgfsetstrokecolor{currentstroke}%
\pgfsetstrokeopacity{0.300000}%
\pgfsetdash{}{0pt}%
\pgfpathmoveto{\pgfqpoint{0.781944in}{0.920461in}}%
\pgfpathlineto{\pgfqpoint{2.920660in}{0.920461in}}%
\pgfusepath{stroke}%
\end{pgfscope}%
\begin{pgfscope}%
\pgfsetbuttcap%
\pgfsetroundjoin%
\definecolor{currentfill}{rgb}{0.000000,0.000000,0.000000}%
\pgfsetfillcolor{currentfill}%
\pgfsetlinewidth{0.602250pt}%
\definecolor{currentstroke}{rgb}{0.000000,0.000000,0.000000}%
\pgfsetstrokecolor{currentstroke}%
\pgfsetdash{}{0pt}%
\pgfsys@defobject{currentmarker}{\pgfqpoint{-0.027778in}{0.000000in}}{\pgfqpoint{0.000000in}{0.000000in}}{%
\pgfpathmoveto{\pgfqpoint{0.000000in}{0.000000in}}%
\pgfpathlineto{\pgfqpoint{-0.027778in}{0.000000in}}%
\pgfusepath{stroke,fill}%
}%
\begin{pgfscope}%
\pgfsys@transformshift{0.781944in}{0.920461in}%
\pgfsys@useobject{currentmarker}{}%
\end{pgfscope}%
\end{pgfscope}%
\begin{pgfscope}%
\pgfsetbuttcap%
\pgfsetroundjoin%
\definecolor{currentfill}{rgb}{0.000000,0.000000,0.000000}%
\pgfsetfillcolor{currentfill}%
\pgfsetlinewidth{0.602250pt}%
\definecolor{currentstroke}{rgb}{0.000000,0.000000,0.000000}%
\pgfsetstrokecolor{currentstroke}%
\pgfsetdash{}{0pt}%
\pgfsys@defobject{currentmarker}{\pgfqpoint{0.000000in}{0.000000in}}{\pgfqpoint{0.027778in}{0.000000in}}{%
\pgfpathmoveto{\pgfqpoint{0.000000in}{0.000000in}}%
\pgfpathlineto{\pgfqpoint{0.027778in}{0.000000in}}%
\pgfusepath{stroke,fill}%
}%
\begin{pgfscope}%
\pgfsys@transformshift{2.920660in}{0.920461in}%
\pgfsys@useobject{currentmarker}{}%
\end{pgfscope}%
\end{pgfscope}%
\begin{pgfscope}%
\pgfpathrectangle{\pgfqpoint{0.781944in}{0.552778in}}{\pgfqpoint{2.138715in}{1.650000in}}%
\pgfusepath{clip}%
\pgfsetrectcap%
\pgfsetroundjoin%
\pgfsetlinewidth{0.803000pt}%
\definecolor{currentstroke}{rgb}{0.690196,0.690196,0.690196}%
\pgfsetstrokecolor{currentstroke}%
\pgfsetstrokeopacity{0.300000}%
\pgfsetdash{}{0pt}%
\pgfpathmoveto{\pgfqpoint{0.781944in}{0.962273in}}%
\pgfpathlineto{\pgfqpoint{2.920660in}{0.962273in}}%
\pgfusepath{stroke}%
\end{pgfscope}%
\begin{pgfscope}%
\pgfsetbuttcap%
\pgfsetroundjoin%
\definecolor{currentfill}{rgb}{0.000000,0.000000,0.000000}%
\pgfsetfillcolor{currentfill}%
\pgfsetlinewidth{0.602250pt}%
\definecolor{currentstroke}{rgb}{0.000000,0.000000,0.000000}%
\pgfsetstrokecolor{currentstroke}%
\pgfsetdash{}{0pt}%
\pgfsys@defobject{currentmarker}{\pgfqpoint{-0.027778in}{0.000000in}}{\pgfqpoint{0.000000in}{0.000000in}}{%
\pgfpathmoveto{\pgfqpoint{0.000000in}{0.000000in}}%
\pgfpathlineto{\pgfqpoint{-0.027778in}{0.000000in}}%
\pgfusepath{stroke,fill}%
}%
\begin{pgfscope}%
\pgfsys@transformshift{0.781944in}{0.962273in}%
\pgfsys@useobject{currentmarker}{}%
\end{pgfscope}%
\end{pgfscope}%
\begin{pgfscope}%
\pgfsetbuttcap%
\pgfsetroundjoin%
\definecolor{currentfill}{rgb}{0.000000,0.000000,0.000000}%
\pgfsetfillcolor{currentfill}%
\pgfsetlinewidth{0.602250pt}%
\definecolor{currentstroke}{rgb}{0.000000,0.000000,0.000000}%
\pgfsetstrokecolor{currentstroke}%
\pgfsetdash{}{0pt}%
\pgfsys@defobject{currentmarker}{\pgfqpoint{0.000000in}{0.000000in}}{\pgfqpoint{0.027778in}{0.000000in}}{%
\pgfpathmoveto{\pgfqpoint{0.000000in}{0.000000in}}%
\pgfpathlineto{\pgfqpoint{0.027778in}{0.000000in}}%
\pgfusepath{stroke,fill}%
}%
\begin{pgfscope}%
\pgfsys@transformshift{2.920660in}{0.962273in}%
\pgfsys@useobject{currentmarker}{}%
\end{pgfscope}%
\end{pgfscope}%
\begin{pgfscope}%
\pgfpathrectangle{\pgfqpoint{0.781944in}{0.552778in}}{\pgfqpoint{2.138715in}{1.650000in}}%
\pgfusepath{clip}%
\pgfsetrectcap%
\pgfsetroundjoin%
\pgfsetlinewidth{0.803000pt}%
\definecolor{currentstroke}{rgb}{0.690196,0.690196,0.690196}%
\pgfsetstrokecolor{currentstroke}%
\pgfsetstrokeopacity{0.300000}%
\pgfsetdash{}{0pt}%
\pgfpathmoveto{\pgfqpoint{0.781944in}{1.004084in}}%
\pgfpathlineto{\pgfqpoint{2.920660in}{1.004084in}}%
\pgfusepath{stroke}%
\end{pgfscope}%
\begin{pgfscope}%
\pgfsetbuttcap%
\pgfsetroundjoin%
\definecolor{currentfill}{rgb}{0.000000,0.000000,0.000000}%
\pgfsetfillcolor{currentfill}%
\pgfsetlinewidth{0.602250pt}%
\definecolor{currentstroke}{rgb}{0.000000,0.000000,0.000000}%
\pgfsetstrokecolor{currentstroke}%
\pgfsetdash{}{0pt}%
\pgfsys@defobject{currentmarker}{\pgfqpoint{-0.027778in}{0.000000in}}{\pgfqpoint{0.000000in}{0.000000in}}{%
\pgfpathmoveto{\pgfqpoint{0.000000in}{0.000000in}}%
\pgfpathlineto{\pgfqpoint{-0.027778in}{0.000000in}}%
\pgfusepath{stroke,fill}%
}%
\begin{pgfscope}%
\pgfsys@transformshift{0.781944in}{1.004084in}%
\pgfsys@useobject{currentmarker}{}%
\end{pgfscope}%
\end{pgfscope}%
\begin{pgfscope}%
\pgfsetbuttcap%
\pgfsetroundjoin%
\definecolor{currentfill}{rgb}{0.000000,0.000000,0.000000}%
\pgfsetfillcolor{currentfill}%
\pgfsetlinewidth{0.602250pt}%
\definecolor{currentstroke}{rgb}{0.000000,0.000000,0.000000}%
\pgfsetstrokecolor{currentstroke}%
\pgfsetdash{}{0pt}%
\pgfsys@defobject{currentmarker}{\pgfqpoint{0.000000in}{0.000000in}}{\pgfqpoint{0.027778in}{0.000000in}}{%
\pgfpathmoveto{\pgfqpoint{0.000000in}{0.000000in}}%
\pgfpathlineto{\pgfqpoint{0.027778in}{0.000000in}}%
\pgfusepath{stroke,fill}%
}%
\begin{pgfscope}%
\pgfsys@transformshift{2.920660in}{1.004084in}%
\pgfsys@useobject{currentmarker}{}%
\end{pgfscope}%
\end{pgfscope}%
\begin{pgfscope}%
\pgfpathrectangle{\pgfqpoint{0.781944in}{0.552778in}}{\pgfqpoint{2.138715in}{1.650000in}}%
\pgfusepath{clip}%
\pgfsetrectcap%
\pgfsetroundjoin%
\pgfsetlinewidth{0.803000pt}%
\definecolor{currentstroke}{rgb}{0.690196,0.690196,0.690196}%
\pgfsetstrokecolor{currentstroke}%
\pgfsetstrokeopacity{0.300000}%
\pgfsetdash{}{0pt}%
\pgfpathmoveto{\pgfqpoint{0.781944in}{1.087708in}}%
\pgfpathlineto{\pgfqpoint{2.920660in}{1.087708in}}%
\pgfusepath{stroke}%
\end{pgfscope}%
\begin{pgfscope}%
\pgfsetbuttcap%
\pgfsetroundjoin%
\definecolor{currentfill}{rgb}{0.000000,0.000000,0.000000}%
\pgfsetfillcolor{currentfill}%
\pgfsetlinewidth{0.602250pt}%
\definecolor{currentstroke}{rgb}{0.000000,0.000000,0.000000}%
\pgfsetstrokecolor{currentstroke}%
\pgfsetdash{}{0pt}%
\pgfsys@defobject{currentmarker}{\pgfqpoint{-0.027778in}{0.000000in}}{\pgfqpoint{0.000000in}{0.000000in}}{%
\pgfpathmoveto{\pgfqpoint{0.000000in}{0.000000in}}%
\pgfpathlineto{\pgfqpoint{-0.027778in}{0.000000in}}%
\pgfusepath{stroke,fill}%
}%
\begin{pgfscope}%
\pgfsys@transformshift{0.781944in}{1.087708in}%
\pgfsys@useobject{currentmarker}{}%
\end{pgfscope}%
\end{pgfscope}%
\begin{pgfscope}%
\pgfsetbuttcap%
\pgfsetroundjoin%
\definecolor{currentfill}{rgb}{0.000000,0.000000,0.000000}%
\pgfsetfillcolor{currentfill}%
\pgfsetlinewidth{0.602250pt}%
\definecolor{currentstroke}{rgb}{0.000000,0.000000,0.000000}%
\pgfsetstrokecolor{currentstroke}%
\pgfsetdash{}{0pt}%
\pgfsys@defobject{currentmarker}{\pgfqpoint{0.000000in}{0.000000in}}{\pgfqpoint{0.027778in}{0.000000in}}{%
\pgfpathmoveto{\pgfqpoint{0.000000in}{0.000000in}}%
\pgfpathlineto{\pgfqpoint{0.027778in}{0.000000in}}%
\pgfusepath{stroke,fill}%
}%
\begin{pgfscope}%
\pgfsys@transformshift{2.920660in}{1.087708in}%
\pgfsys@useobject{currentmarker}{}%
\end{pgfscope}%
\end{pgfscope}%
\begin{pgfscope}%
\pgfpathrectangle{\pgfqpoint{0.781944in}{0.552778in}}{\pgfqpoint{2.138715in}{1.650000in}}%
\pgfusepath{clip}%
\pgfsetrectcap%
\pgfsetroundjoin%
\pgfsetlinewidth{0.803000pt}%
\definecolor{currentstroke}{rgb}{0.690196,0.690196,0.690196}%
\pgfsetstrokecolor{currentstroke}%
\pgfsetstrokeopacity{0.300000}%
\pgfsetdash{}{0pt}%
\pgfpathmoveto{\pgfqpoint{0.781944in}{1.129520in}}%
\pgfpathlineto{\pgfqpoint{2.920660in}{1.129520in}}%
\pgfusepath{stroke}%
\end{pgfscope}%
\begin{pgfscope}%
\pgfsetbuttcap%
\pgfsetroundjoin%
\definecolor{currentfill}{rgb}{0.000000,0.000000,0.000000}%
\pgfsetfillcolor{currentfill}%
\pgfsetlinewidth{0.602250pt}%
\definecolor{currentstroke}{rgb}{0.000000,0.000000,0.000000}%
\pgfsetstrokecolor{currentstroke}%
\pgfsetdash{}{0pt}%
\pgfsys@defobject{currentmarker}{\pgfqpoint{-0.027778in}{0.000000in}}{\pgfqpoint{0.000000in}{0.000000in}}{%
\pgfpathmoveto{\pgfqpoint{0.000000in}{0.000000in}}%
\pgfpathlineto{\pgfqpoint{-0.027778in}{0.000000in}}%
\pgfusepath{stroke,fill}%
}%
\begin{pgfscope}%
\pgfsys@transformshift{0.781944in}{1.129520in}%
\pgfsys@useobject{currentmarker}{}%
\end{pgfscope}%
\end{pgfscope}%
\begin{pgfscope}%
\pgfsetbuttcap%
\pgfsetroundjoin%
\definecolor{currentfill}{rgb}{0.000000,0.000000,0.000000}%
\pgfsetfillcolor{currentfill}%
\pgfsetlinewidth{0.602250pt}%
\definecolor{currentstroke}{rgb}{0.000000,0.000000,0.000000}%
\pgfsetstrokecolor{currentstroke}%
\pgfsetdash{}{0pt}%
\pgfsys@defobject{currentmarker}{\pgfqpoint{0.000000in}{0.000000in}}{\pgfqpoint{0.027778in}{0.000000in}}{%
\pgfpathmoveto{\pgfqpoint{0.000000in}{0.000000in}}%
\pgfpathlineto{\pgfqpoint{0.027778in}{0.000000in}}%
\pgfusepath{stroke,fill}%
}%
\begin{pgfscope}%
\pgfsys@transformshift{2.920660in}{1.129520in}%
\pgfsys@useobject{currentmarker}{}%
\end{pgfscope}%
\end{pgfscope}%
\begin{pgfscope}%
\pgfpathrectangle{\pgfqpoint{0.781944in}{0.552778in}}{\pgfqpoint{2.138715in}{1.650000in}}%
\pgfusepath{clip}%
\pgfsetrectcap%
\pgfsetroundjoin%
\pgfsetlinewidth{0.803000pt}%
\definecolor{currentstroke}{rgb}{0.690196,0.690196,0.690196}%
\pgfsetstrokecolor{currentstroke}%
\pgfsetstrokeopacity{0.300000}%
\pgfsetdash{}{0pt}%
\pgfpathmoveto{\pgfqpoint{0.781944in}{1.171332in}}%
\pgfpathlineto{\pgfqpoint{2.920660in}{1.171332in}}%
\pgfusepath{stroke}%
\end{pgfscope}%
\begin{pgfscope}%
\pgfsetbuttcap%
\pgfsetroundjoin%
\definecolor{currentfill}{rgb}{0.000000,0.000000,0.000000}%
\pgfsetfillcolor{currentfill}%
\pgfsetlinewidth{0.602250pt}%
\definecolor{currentstroke}{rgb}{0.000000,0.000000,0.000000}%
\pgfsetstrokecolor{currentstroke}%
\pgfsetdash{}{0pt}%
\pgfsys@defobject{currentmarker}{\pgfqpoint{-0.027778in}{0.000000in}}{\pgfqpoint{0.000000in}{0.000000in}}{%
\pgfpathmoveto{\pgfqpoint{0.000000in}{0.000000in}}%
\pgfpathlineto{\pgfqpoint{-0.027778in}{0.000000in}}%
\pgfusepath{stroke,fill}%
}%
\begin{pgfscope}%
\pgfsys@transformshift{0.781944in}{1.171332in}%
\pgfsys@useobject{currentmarker}{}%
\end{pgfscope}%
\end{pgfscope}%
\begin{pgfscope}%
\pgfsetbuttcap%
\pgfsetroundjoin%
\definecolor{currentfill}{rgb}{0.000000,0.000000,0.000000}%
\pgfsetfillcolor{currentfill}%
\pgfsetlinewidth{0.602250pt}%
\definecolor{currentstroke}{rgb}{0.000000,0.000000,0.000000}%
\pgfsetstrokecolor{currentstroke}%
\pgfsetdash{}{0pt}%
\pgfsys@defobject{currentmarker}{\pgfqpoint{0.000000in}{0.000000in}}{\pgfqpoint{0.027778in}{0.000000in}}{%
\pgfpathmoveto{\pgfqpoint{0.000000in}{0.000000in}}%
\pgfpathlineto{\pgfqpoint{0.027778in}{0.000000in}}%
\pgfusepath{stroke,fill}%
}%
\begin{pgfscope}%
\pgfsys@transformshift{2.920660in}{1.171332in}%
\pgfsys@useobject{currentmarker}{}%
\end{pgfscope}%
\end{pgfscope}%
\begin{pgfscope}%
\pgfpathrectangle{\pgfqpoint{0.781944in}{0.552778in}}{\pgfqpoint{2.138715in}{1.650000in}}%
\pgfusepath{clip}%
\pgfsetrectcap%
\pgfsetroundjoin%
\pgfsetlinewidth{0.803000pt}%
\definecolor{currentstroke}{rgb}{0.690196,0.690196,0.690196}%
\pgfsetstrokecolor{currentstroke}%
\pgfsetstrokeopacity{0.300000}%
\pgfsetdash{}{0pt}%
\pgfpathmoveto{\pgfqpoint{0.781944in}{1.213144in}}%
\pgfpathlineto{\pgfqpoint{2.920660in}{1.213144in}}%
\pgfusepath{stroke}%
\end{pgfscope}%
\begin{pgfscope}%
\pgfsetbuttcap%
\pgfsetroundjoin%
\definecolor{currentfill}{rgb}{0.000000,0.000000,0.000000}%
\pgfsetfillcolor{currentfill}%
\pgfsetlinewidth{0.602250pt}%
\definecolor{currentstroke}{rgb}{0.000000,0.000000,0.000000}%
\pgfsetstrokecolor{currentstroke}%
\pgfsetdash{}{0pt}%
\pgfsys@defobject{currentmarker}{\pgfqpoint{-0.027778in}{0.000000in}}{\pgfqpoint{0.000000in}{0.000000in}}{%
\pgfpathmoveto{\pgfqpoint{0.000000in}{0.000000in}}%
\pgfpathlineto{\pgfqpoint{-0.027778in}{0.000000in}}%
\pgfusepath{stroke,fill}%
}%
\begin{pgfscope}%
\pgfsys@transformshift{0.781944in}{1.213144in}%
\pgfsys@useobject{currentmarker}{}%
\end{pgfscope}%
\end{pgfscope}%
\begin{pgfscope}%
\pgfsetbuttcap%
\pgfsetroundjoin%
\definecolor{currentfill}{rgb}{0.000000,0.000000,0.000000}%
\pgfsetfillcolor{currentfill}%
\pgfsetlinewidth{0.602250pt}%
\definecolor{currentstroke}{rgb}{0.000000,0.000000,0.000000}%
\pgfsetstrokecolor{currentstroke}%
\pgfsetdash{}{0pt}%
\pgfsys@defobject{currentmarker}{\pgfqpoint{0.000000in}{0.000000in}}{\pgfqpoint{0.027778in}{0.000000in}}{%
\pgfpathmoveto{\pgfqpoint{0.000000in}{0.000000in}}%
\pgfpathlineto{\pgfqpoint{0.027778in}{0.000000in}}%
\pgfusepath{stroke,fill}%
}%
\begin{pgfscope}%
\pgfsys@transformshift{2.920660in}{1.213144in}%
\pgfsys@useobject{currentmarker}{}%
\end{pgfscope}%
\end{pgfscope}%
\begin{pgfscope}%
\pgfpathrectangle{\pgfqpoint{0.781944in}{0.552778in}}{\pgfqpoint{2.138715in}{1.650000in}}%
\pgfusepath{clip}%
\pgfsetrectcap%
\pgfsetroundjoin%
\pgfsetlinewidth{0.803000pt}%
\definecolor{currentstroke}{rgb}{0.690196,0.690196,0.690196}%
\pgfsetstrokecolor{currentstroke}%
\pgfsetstrokeopacity{0.300000}%
\pgfsetdash{}{0pt}%
\pgfpathmoveto{\pgfqpoint{0.781944in}{1.254955in}}%
\pgfpathlineto{\pgfqpoint{2.920660in}{1.254955in}}%
\pgfusepath{stroke}%
\end{pgfscope}%
\begin{pgfscope}%
\pgfsetbuttcap%
\pgfsetroundjoin%
\definecolor{currentfill}{rgb}{0.000000,0.000000,0.000000}%
\pgfsetfillcolor{currentfill}%
\pgfsetlinewidth{0.602250pt}%
\definecolor{currentstroke}{rgb}{0.000000,0.000000,0.000000}%
\pgfsetstrokecolor{currentstroke}%
\pgfsetdash{}{0pt}%
\pgfsys@defobject{currentmarker}{\pgfqpoint{-0.027778in}{0.000000in}}{\pgfqpoint{0.000000in}{0.000000in}}{%
\pgfpathmoveto{\pgfqpoint{0.000000in}{0.000000in}}%
\pgfpathlineto{\pgfqpoint{-0.027778in}{0.000000in}}%
\pgfusepath{stroke,fill}%
}%
\begin{pgfscope}%
\pgfsys@transformshift{0.781944in}{1.254955in}%
\pgfsys@useobject{currentmarker}{}%
\end{pgfscope}%
\end{pgfscope}%
\begin{pgfscope}%
\pgfsetbuttcap%
\pgfsetroundjoin%
\definecolor{currentfill}{rgb}{0.000000,0.000000,0.000000}%
\pgfsetfillcolor{currentfill}%
\pgfsetlinewidth{0.602250pt}%
\definecolor{currentstroke}{rgb}{0.000000,0.000000,0.000000}%
\pgfsetstrokecolor{currentstroke}%
\pgfsetdash{}{0pt}%
\pgfsys@defobject{currentmarker}{\pgfqpoint{0.000000in}{0.000000in}}{\pgfqpoint{0.027778in}{0.000000in}}{%
\pgfpathmoveto{\pgfqpoint{0.000000in}{0.000000in}}%
\pgfpathlineto{\pgfqpoint{0.027778in}{0.000000in}}%
\pgfusepath{stroke,fill}%
}%
\begin{pgfscope}%
\pgfsys@transformshift{2.920660in}{1.254955in}%
\pgfsys@useobject{currentmarker}{}%
\end{pgfscope}%
\end{pgfscope}%
\begin{pgfscope}%
\pgfpathrectangle{\pgfqpoint{0.781944in}{0.552778in}}{\pgfqpoint{2.138715in}{1.650000in}}%
\pgfusepath{clip}%
\pgfsetrectcap%
\pgfsetroundjoin%
\pgfsetlinewidth{0.803000pt}%
\definecolor{currentstroke}{rgb}{0.690196,0.690196,0.690196}%
\pgfsetstrokecolor{currentstroke}%
\pgfsetstrokeopacity{0.300000}%
\pgfsetdash{}{0pt}%
\pgfpathmoveto{\pgfqpoint{0.781944in}{1.296767in}}%
\pgfpathlineto{\pgfqpoint{2.920660in}{1.296767in}}%
\pgfusepath{stroke}%
\end{pgfscope}%
\begin{pgfscope}%
\pgfsetbuttcap%
\pgfsetroundjoin%
\definecolor{currentfill}{rgb}{0.000000,0.000000,0.000000}%
\pgfsetfillcolor{currentfill}%
\pgfsetlinewidth{0.602250pt}%
\definecolor{currentstroke}{rgb}{0.000000,0.000000,0.000000}%
\pgfsetstrokecolor{currentstroke}%
\pgfsetdash{}{0pt}%
\pgfsys@defobject{currentmarker}{\pgfqpoint{-0.027778in}{0.000000in}}{\pgfqpoint{0.000000in}{0.000000in}}{%
\pgfpathmoveto{\pgfqpoint{0.000000in}{0.000000in}}%
\pgfpathlineto{\pgfqpoint{-0.027778in}{0.000000in}}%
\pgfusepath{stroke,fill}%
}%
\begin{pgfscope}%
\pgfsys@transformshift{0.781944in}{1.296767in}%
\pgfsys@useobject{currentmarker}{}%
\end{pgfscope}%
\end{pgfscope}%
\begin{pgfscope}%
\pgfsetbuttcap%
\pgfsetroundjoin%
\definecolor{currentfill}{rgb}{0.000000,0.000000,0.000000}%
\pgfsetfillcolor{currentfill}%
\pgfsetlinewidth{0.602250pt}%
\definecolor{currentstroke}{rgb}{0.000000,0.000000,0.000000}%
\pgfsetstrokecolor{currentstroke}%
\pgfsetdash{}{0pt}%
\pgfsys@defobject{currentmarker}{\pgfqpoint{0.000000in}{0.000000in}}{\pgfqpoint{0.027778in}{0.000000in}}{%
\pgfpathmoveto{\pgfqpoint{0.000000in}{0.000000in}}%
\pgfpathlineto{\pgfqpoint{0.027778in}{0.000000in}}%
\pgfusepath{stroke,fill}%
}%
\begin{pgfscope}%
\pgfsys@transformshift{2.920660in}{1.296767in}%
\pgfsys@useobject{currentmarker}{}%
\end{pgfscope}%
\end{pgfscope}%
\begin{pgfscope}%
\pgfpathrectangle{\pgfqpoint{0.781944in}{0.552778in}}{\pgfqpoint{2.138715in}{1.650000in}}%
\pgfusepath{clip}%
\pgfsetrectcap%
\pgfsetroundjoin%
\pgfsetlinewidth{0.803000pt}%
\definecolor{currentstroke}{rgb}{0.690196,0.690196,0.690196}%
\pgfsetstrokecolor{currentstroke}%
\pgfsetstrokeopacity{0.300000}%
\pgfsetdash{}{0pt}%
\pgfpathmoveto{\pgfqpoint{0.781944in}{1.338579in}}%
\pgfpathlineto{\pgfqpoint{2.920660in}{1.338579in}}%
\pgfusepath{stroke}%
\end{pgfscope}%
\begin{pgfscope}%
\pgfsetbuttcap%
\pgfsetroundjoin%
\definecolor{currentfill}{rgb}{0.000000,0.000000,0.000000}%
\pgfsetfillcolor{currentfill}%
\pgfsetlinewidth{0.602250pt}%
\definecolor{currentstroke}{rgb}{0.000000,0.000000,0.000000}%
\pgfsetstrokecolor{currentstroke}%
\pgfsetdash{}{0pt}%
\pgfsys@defobject{currentmarker}{\pgfqpoint{-0.027778in}{0.000000in}}{\pgfqpoint{0.000000in}{0.000000in}}{%
\pgfpathmoveto{\pgfqpoint{0.000000in}{0.000000in}}%
\pgfpathlineto{\pgfqpoint{-0.027778in}{0.000000in}}%
\pgfusepath{stroke,fill}%
}%
\begin{pgfscope}%
\pgfsys@transformshift{0.781944in}{1.338579in}%
\pgfsys@useobject{currentmarker}{}%
\end{pgfscope}%
\end{pgfscope}%
\begin{pgfscope}%
\pgfsetbuttcap%
\pgfsetroundjoin%
\definecolor{currentfill}{rgb}{0.000000,0.000000,0.000000}%
\pgfsetfillcolor{currentfill}%
\pgfsetlinewidth{0.602250pt}%
\definecolor{currentstroke}{rgb}{0.000000,0.000000,0.000000}%
\pgfsetstrokecolor{currentstroke}%
\pgfsetdash{}{0pt}%
\pgfsys@defobject{currentmarker}{\pgfqpoint{0.000000in}{0.000000in}}{\pgfqpoint{0.027778in}{0.000000in}}{%
\pgfpathmoveto{\pgfqpoint{0.000000in}{0.000000in}}%
\pgfpathlineto{\pgfqpoint{0.027778in}{0.000000in}}%
\pgfusepath{stroke,fill}%
}%
\begin{pgfscope}%
\pgfsys@transformshift{2.920660in}{1.338579in}%
\pgfsys@useobject{currentmarker}{}%
\end{pgfscope}%
\end{pgfscope}%
\begin{pgfscope}%
\pgfpathrectangle{\pgfqpoint{0.781944in}{0.552778in}}{\pgfqpoint{2.138715in}{1.650000in}}%
\pgfusepath{clip}%
\pgfsetrectcap%
\pgfsetroundjoin%
\pgfsetlinewidth{0.803000pt}%
\definecolor{currentstroke}{rgb}{0.690196,0.690196,0.690196}%
\pgfsetstrokecolor{currentstroke}%
\pgfsetstrokeopacity{0.300000}%
\pgfsetdash{}{0pt}%
\pgfpathmoveto{\pgfqpoint{0.781944in}{1.380391in}}%
\pgfpathlineto{\pgfqpoint{2.920660in}{1.380391in}}%
\pgfusepath{stroke}%
\end{pgfscope}%
\begin{pgfscope}%
\pgfsetbuttcap%
\pgfsetroundjoin%
\definecolor{currentfill}{rgb}{0.000000,0.000000,0.000000}%
\pgfsetfillcolor{currentfill}%
\pgfsetlinewidth{0.602250pt}%
\definecolor{currentstroke}{rgb}{0.000000,0.000000,0.000000}%
\pgfsetstrokecolor{currentstroke}%
\pgfsetdash{}{0pt}%
\pgfsys@defobject{currentmarker}{\pgfqpoint{-0.027778in}{0.000000in}}{\pgfqpoint{0.000000in}{0.000000in}}{%
\pgfpathmoveto{\pgfqpoint{0.000000in}{0.000000in}}%
\pgfpathlineto{\pgfqpoint{-0.027778in}{0.000000in}}%
\pgfusepath{stroke,fill}%
}%
\begin{pgfscope}%
\pgfsys@transformshift{0.781944in}{1.380391in}%
\pgfsys@useobject{currentmarker}{}%
\end{pgfscope}%
\end{pgfscope}%
\begin{pgfscope}%
\pgfsetbuttcap%
\pgfsetroundjoin%
\definecolor{currentfill}{rgb}{0.000000,0.000000,0.000000}%
\pgfsetfillcolor{currentfill}%
\pgfsetlinewidth{0.602250pt}%
\definecolor{currentstroke}{rgb}{0.000000,0.000000,0.000000}%
\pgfsetstrokecolor{currentstroke}%
\pgfsetdash{}{0pt}%
\pgfsys@defobject{currentmarker}{\pgfqpoint{0.000000in}{0.000000in}}{\pgfqpoint{0.027778in}{0.000000in}}{%
\pgfpathmoveto{\pgfqpoint{0.000000in}{0.000000in}}%
\pgfpathlineto{\pgfqpoint{0.027778in}{0.000000in}}%
\pgfusepath{stroke,fill}%
}%
\begin{pgfscope}%
\pgfsys@transformshift{2.920660in}{1.380391in}%
\pgfsys@useobject{currentmarker}{}%
\end{pgfscope}%
\end{pgfscope}%
\begin{pgfscope}%
\pgfpathrectangle{\pgfqpoint{0.781944in}{0.552778in}}{\pgfqpoint{2.138715in}{1.650000in}}%
\pgfusepath{clip}%
\pgfsetrectcap%
\pgfsetroundjoin%
\pgfsetlinewidth{0.803000pt}%
\definecolor{currentstroke}{rgb}{0.690196,0.690196,0.690196}%
\pgfsetstrokecolor{currentstroke}%
\pgfsetstrokeopacity{0.300000}%
\pgfsetdash{}{0pt}%
\pgfpathmoveto{\pgfqpoint{0.781944in}{1.422203in}}%
\pgfpathlineto{\pgfqpoint{2.920660in}{1.422203in}}%
\pgfusepath{stroke}%
\end{pgfscope}%
\begin{pgfscope}%
\pgfsetbuttcap%
\pgfsetroundjoin%
\definecolor{currentfill}{rgb}{0.000000,0.000000,0.000000}%
\pgfsetfillcolor{currentfill}%
\pgfsetlinewidth{0.602250pt}%
\definecolor{currentstroke}{rgb}{0.000000,0.000000,0.000000}%
\pgfsetstrokecolor{currentstroke}%
\pgfsetdash{}{0pt}%
\pgfsys@defobject{currentmarker}{\pgfqpoint{-0.027778in}{0.000000in}}{\pgfqpoint{0.000000in}{0.000000in}}{%
\pgfpathmoveto{\pgfqpoint{0.000000in}{0.000000in}}%
\pgfpathlineto{\pgfqpoint{-0.027778in}{0.000000in}}%
\pgfusepath{stroke,fill}%
}%
\begin{pgfscope}%
\pgfsys@transformshift{0.781944in}{1.422203in}%
\pgfsys@useobject{currentmarker}{}%
\end{pgfscope}%
\end{pgfscope}%
\begin{pgfscope}%
\pgfsetbuttcap%
\pgfsetroundjoin%
\definecolor{currentfill}{rgb}{0.000000,0.000000,0.000000}%
\pgfsetfillcolor{currentfill}%
\pgfsetlinewidth{0.602250pt}%
\definecolor{currentstroke}{rgb}{0.000000,0.000000,0.000000}%
\pgfsetstrokecolor{currentstroke}%
\pgfsetdash{}{0pt}%
\pgfsys@defobject{currentmarker}{\pgfqpoint{0.000000in}{0.000000in}}{\pgfqpoint{0.027778in}{0.000000in}}{%
\pgfpathmoveto{\pgfqpoint{0.000000in}{0.000000in}}%
\pgfpathlineto{\pgfqpoint{0.027778in}{0.000000in}}%
\pgfusepath{stroke,fill}%
}%
\begin{pgfscope}%
\pgfsys@transformshift{2.920660in}{1.422203in}%
\pgfsys@useobject{currentmarker}{}%
\end{pgfscope}%
\end{pgfscope}%
\begin{pgfscope}%
\pgfpathrectangle{\pgfqpoint{0.781944in}{0.552778in}}{\pgfqpoint{2.138715in}{1.650000in}}%
\pgfusepath{clip}%
\pgfsetrectcap%
\pgfsetroundjoin%
\pgfsetlinewidth{0.803000pt}%
\definecolor{currentstroke}{rgb}{0.690196,0.690196,0.690196}%
\pgfsetstrokecolor{currentstroke}%
\pgfsetstrokeopacity{0.300000}%
\pgfsetdash{}{0pt}%
\pgfpathmoveto{\pgfqpoint{0.781944in}{1.505827in}}%
\pgfpathlineto{\pgfqpoint{2.920660in}{1.505827in}}%
\pgfusepath{stroke}%
\end{pgfscope}%
\begin{pgfscope}%
\pgfsetbuttcap%
\pgfsetroundjoin%
\definecolor{currentfill}{rgb}{0.000000,0.000000,0.000000}%
\pgfsetfillcolor{currentfill}%
\pgfsetlinewidth{0.602250pt}%
\definecolor{currentstroke}{rgb}{0.000000,0.000000,0.000000}%
\pgfsetstrokecolor{currentstroke}%
\pgfsetdash{}{0pt}%
\pgfsys@defobject{currentmarker}{\pgfqpoint{-0.027778in}{0.000000in}}{\pgfqpoint{0.000000in}{0.000000in}}{%
\pgfpathmoveto{\pgfqpoint{0.000000in}{0.000000in}}%
\pgfpathlineto{\pgfqpoint{-0.027778in}{0.000000in}}%
\pgfusepath{stroke,fill}%
}%
\begin{pgfscope}%
\pgfsys@transformshift{0.781944in}{1.505827in}%
\pgfsys@useobject{currentmarker}{}%
\end{pgfscope}%
\end{pgfscope}%
\begin{pgfscope}%
\pgfsetbuttcap%
\pgfsetroundjoin%
\definecolor{currentfill}{rgb}{0.000000,0.000000,0.000000}%
\pgfsetfillcolor{currentfill}%
\pgfsetlinewidth{0.602250pt}%
\definecolor{currentstroke}{rgb}{0.000000,0.000000,0.000000}%
\pgfsetstrokecolor{currentstroke}%
\pgfsetdash{}{0pt}%
\pgfsys@defobject{currentmarker}{\pgfqpoint{0.000000in}{0.000000in}}{\pgfqpoint{0.027778in}{0.000000in}}{%
\pgfpathmoveto{\pgfqpoint{0.000000in}{0.000000in}}%
\pgfpathlineto{\pgfqpoint{0.027778in}{0.000000in}}%
\pgfusepath{stroke,fill}%
}%
\begin{pgfscope}%
\pgfsys@transformshift{2.920660in}{1.505827in}%
\pgfsys@useobject{currentmarker}{}%
\end{pgfscope}%
\end{pgfscope}%
\begin{pgfscope}%
\pgfpathrectangle{\pgfqpoint{0.781944in}{0.552778in}}{\pgfqpoint{2.138715in}{1.650000in}}%
\pgfusepath{clip}%
\pgfsetrectcap%
\pgfsetroundjoin%
\pgfsetlinewidth{0.803000pt}%
\definecolor{currentstroke}{rgb}{0.690196,0.690196,0.690196}%
\pgfsetstrokecolor{currentstroke}%
\pgfsetstrokeopacity{0.300000}%
\pgfsetdash{}{0pt}%
\pgfpathmoveto{\pgfqpoint{0.781944in}{1.547638in}}%
\pgfpathlineto{\pgfqpoint{2.920660in}{1.547638in}}%
\pgfusepath{stroke}%
\end{pgfscope}%
\begin{pgfscope}%
\pgfsetbuttcap%
\pgfsetroundjoin%
\definecolor{currentfill}{rgb}{0.000000,0.000000,0.000000}%
\pgfsetfillcolor{currentfill}%
\pgfsetlinewidth{0.602250pt}%
\definecolor{currentstroke}{rgb}{0.000000,0.000000,0.000000}%
\pgfsetstrokecolor{currentstroke}%
\pgfsetdash{}{0pt}%
\pgfsys@defobject{currentmarker}{\pgfqpoint{-0.027778in}{0.000000in}}{\pgfqpoint{0.000000in}{0.000000in}}{%
\pgfpathmoveto{\pgfqpoint{0.000000in}{0.000000in}}%
\pgfpathlineto{\pgfqpoint{-0.027778in}{0.000000in}}%
\pgfusepath{stroke,fill}%
}%
\begin{pgfscope}%
\pgfsys@transformshift{0.781944in}{1.547638in}%
\pgfsys@useobject{currentmarker}{}%
\end{pgfscope}%
\end{pgfscope}%
\begin{pgfscope}%
\pgfsetbuttcap%
\pgfsetroundjoin%
\definecolor{currentfill}{rgb}{0.000000,0.000000,0.000000}%
\pgfsetfillcolor{currentfill}%
\pgfsetlinewidth{0.602250pt}%
\definecolor{currentstroke}{rgb}{0.000000,0.000000,0.000000}%
\pgfsetstrokecolor{currentstroke}%
\pgfsetdash{}{0pt}%
\pgfsys@defobject{currentmarker}{\pgfqpoint{0.000000in}{0.000000in}}{\pgfqpoint{0.027778in}{0.000000in}}{%
\pgfpathmoveto{\pgfqpoint{0.000000in}{0.000000in}}%
\pgfpathlineto{\pgfqpoint{0.027778in}{0.000000in}}%
\pgfusepath{stroke,fill}%
}%
\begin{pgfscope}%
\pgfsys@transformshift{2.920660in}{1.547638in}%
\pgfsys@useobject{currentmarker}{}%
\end{pgfscope}%
\end{pgfscope}%
\begin{pgfscope}%
\pgfpathrectangle{\pgfqpoint{0.781944in}{0.552778in}}{\pgfqpoint{2.138715in}{1.650000in}}%
\pgfusepath{clip}%
\pgfsetrectcap%
\pgfsetroundjoin%
\pgfsetlinewidth{0.803000pt}%
\definecolor{currentstroke}{rgb}{0.690196,0.690196,0.690196}%
\pgfsetstrokecolor{currentstroke}%
\pgfsetstrokeopacity{0.300000}%
\pgfsetdash{}{0pt}%
\pgfpathmoveto{\pgfqpoint{0.781944in}{1.589450in}}%
\pgfpathlineto{\pgfqpoint{2.920660in}{1.589450in}}%
\pgfusepath{stroke}%
\end{pgfscope}%
\begin{pgfscope}%
\pgfsetbuttcap%
\pgfsetroundjoin%
\definecolor{currentfill}{rgb}{0.000000,0.000000,0.000000}%
\pgfsetfillcolor{currentfill}%
\pgfsetlinewidth{0.602250pt}%
\definecolor{currentstroke}{rgb}{0.000000,0.000000,0.000000}%
\pgfsetstrokecolor{currentstroke}%
\pgfsetdash{}{0pt}%
\pgfsys@defobject{currentmarker}{\pgfqpoint{-0.027778in}{0.000000in}}{\pgfqpoint{0.000000in}{0.000000in}}{%
\pgfpathmoveto{\pgfqpoint{0.000000in}{0.000000in}}%
\pgfpathlineto{\pgfqpoint{-0.027778in}{0.000000in}}%
\pgfusepath{stroke,fill}%
}%
\begin{pgfscope}%
\pgfsys@transformshift{0.781944in}{1.589450in}%
\pgfsys@useobject{currentmarker}{}%
\end{pgfscope}%
\end{pgfscope}%
\begin{pgfscope}%
\pgfsetbuttcap%
\pgfsetroundjoin%
\definecolor{currentfill}{rgb}{0.000000,0.000000,0.000000}%
\pgfsetfillcolor{currentfill}%
\pgfsetlinewidth{0.602250pt}%
\definecolor{currentstroke}{rgb}{0.000000,0.000000,0.000000}%
\pgfsetstrokecolor{currentstroke}%
\pgfsetdash{}{0pt}%
\pgfsys@defobject{currentmarker}{\pgfqpoint{0.000000in}{0.000000in}}{\pgfqpoint{0.027778in}{0.000000in}}{%
\pgfpathmoveto{\pgfqpoint{0.000000in}{0.000000in}}%
\pgfpathlineto{\pgfqpoint{0.027778in}{0.000000in}}%
\pgfusepath{stroke,fill}%
}%
\begin{pgfscope}%
\pgfsys@transformshift{2.920660in}{1.589450in}%
\pgfsys@useobject{currentmarker}{}%
\end{pgfscope}%
\end{pgfscope}%
\begin{pgfscope}%
\pgfpathrectangle{\pgfqpoint{0.781944in}{0.552778in}}{\pgfqpoint{2.138715in}{1.650000in}}%
\pgfusepath{clip}%
\pgfsetrectcap%
\pgfsetroundjoin%
\pgfsetlinewidth{0.803000pt}%
\definecolor{currentstroke}{rgb}{0.690196,0.690196,0.690196}%
\pgfsetstrokecolor{currentstroke}%
\pgfsetstrokeopacity{0.300000}%
\pgfsetdash{}{0pt}%
\pgfpathmoveto{\pgfqpoint{0.781944in}{1.631262in}}%
\pgfpathlineto{\pgfqpoint{2.920660in}{1.631262in}}%
\pgfusepath{stroke}%
\end{pgfscope}%
\begin{pgfscope}%
\pgfsetbuttcap%
\pgfsetroundjoin%
\definecolor{currentfill}{rgb}{0.000000,0.000000,0.000000}%
\pgfsetfillcolor{currentfill}%
\pgfsetlinewidth{0.602250pt}%
\definecolor{currentstroke}{rgb}{0.000000,0.000000,0.000000}%
\pgfsetstrokecolor{currentstroke}%
\pgfsetdash{}{0pt}%
\pgfsys@defobject{currentmarker}{\pgfqpoint{-0.027778in}{0.000000in}}{\pgfqpoint{0.000000in}{0.000000in}}{%
\pgfpathmoveto{\pgfqpoint{0.000000in}{0.000000in}}%
\pgfpathlineto{\pgfqpoint{-0.027778in}{0.000000in}}%
\pgfusepath{stroke,fill}%
}%
\begin{pgfscope}%
\pgfsys@transformshift{0.781944in}{1.631262in}%
\pgfsys@useobject{currentmarker}{}%
\end{pgfscope}%
\end{pgfscope}%
\begin{pgfscope}%
\pgfsetbuttcap%
\pgfsetroundjoin%
\definecolor{currentfill}{rgb}{0.000000,0.000000,0.000000}%
\pgfsetfillcolor{currentfill}%
\pgfsetlinewidth{0.602250pt}%
\definecolor{currentstroke}{rgb}{0.000000,0.000000,0.000000}%
\pgfsetstrokecolor{currentstroke}%
\pgfsetdash{}{0pt}%
\pgfsys@defobject{currentmarker}{\pgfqpoint{0.000000in}{0.000000in}}{\pgfqpoint{0.027778in}{0.000000in}}{%
\pgfpathmoveto{\pgfqpoint{0.000000in}{0.000000in}}%
\pgfpathlineto{\pgfqpoint{0.027778in}{0.000000in}}%
\pgfusepath{stroke,fill}%
}%
\begin{pgfscope}%
\pgfsys@transformshift{2.920660in}{1.631262in}%
\pgfsys@useobject{currentmarker}{}%
\end{pgfscope}%
\end{pgfscope}%
\begin{pgfscope}%
\pgfpathrectangle{\pgfqpoint{0.781944in}{0.552778in}}{\pgfqpoint{2.138715in}{1.650000in}}%
\pgfusepath{clip}%
\pgfsetrectcap%
\pgfsetroundjoin%
\pgfsetlinewidth{0.803000pt}%
\definecolor{currentstroke}{rgb}{0.690196,0.690196,0.690196}%
\pgfsetstrokecolor{currentstroke}%
\pgfsetstrokeopacity{0.300000}%
\pgfsetdash{}{0pt}%
\pgfpathmoveto{\pgfqpoint{0.781944in}{1.673074in}}%
\pgfpathlineto{\pgfqpoint{2.920660in}{1.673074in}}%
\pgfusepath{stroke}%
\end{pgfscope}%
\begin{pgfscope}%
\pgfsetbuttcap%
\pgfsetroundjoin%
\definecolor{currentfill}{rgb}{0.000000,0.000000,0.000000}%
\pgfsetfillcolor{currentfill}%
\pgfsetlinewidth{0.602250pt}%
\definecolor{currentstroke}{rgb}{0.000000,0.000000,0.000000}%
\pgfsetstrokecolor{currentstroke}%
\pgfsetdash{}{0pt}%
\pgfsys@defobject{currentmarker}{\pgfqpoint{-0.027778in}{0.000000in}}{\pgfqpoint{0.000000in}{0.000000in}}{%
\pgfpathmoveto{\pgfqpoint{0.000000in}{0.000000in}}%
\pgfpathlineto{\pgfqpoint{-0.027778in}{0.000000in}}%
\pgfusepath{stroke,fill}%
}%
\begin{pgfscope}%
\pgfsys@transformshift{0.781944in}{1.673074in}%
\pgfsys@useobject{currentmarker}{}%
\end{pgfscope}%
\end{pgfscope}%
\begin{pgfscope}%
\pgfsetbuttcap%
\pgfsetroundjoin%
\definecolor{currentfill}{rgb}{0.000000,0.000000,0.000000}%
\pgfsetfillcolor{currentfill}%
\pgfsetlinewidth{0.602250pt}%
\definecolor{currentstroke}{rgb}{0.000000,0.000000,0.000000}%
\pgfsetstrokecolor{currentstroke}%
\pgfsetdash{}{0pt}%
\pgfsys@defobject{currentmarker}{\pgfqpoint{0.000000in}{0.000000in}}{\pgfqpoint{0.027778in}{0.000000in}}{%
\pgfpathmoveto{\pgfqpoint{0.000000in}{0.000000in}}%
\pgfpathlineto{\pgfqpoint{0.027778in}{0.000000in}}%
\pgfusepath{stroke,fill}%
}%
\begin{pgfscope}%
\pgfsys@transformshift{2.920660in}{1.673074in}%
\pgfsys@useobject{currentmarker}{}%
\end{pgfscope}%
\end{pgfscope}%
\begin{pgfscope}%
\pgfpathrectangle{\pgfqpoint{0.781944in}{0.552778in}}{\pgfqpoint{2.138715in}{1.650000in}}%
\pgfusepath{clip}%
\pgfsetrectcap%
\pgfsetroundjoin%
\pgfsetlinewidth{0.803000pt}%
\definecolor{currentstroke}{rgb}{0.690196,0.690196,0.690196}%
\pgfsetstrokecolor{currentstroke}%
\pgfsetstrokeopacity{0.300000}%
\pgfsetdash{}{0pt}%
\pgfpathmoveto{\pgfqpoint{0.781944in}{1.714886in}}%
\pgfpathlineto{\pgfqpoint{2.920660in}{1.714886in}}%
\pgfusepath{stroke}%
\end{pgfscope}%
\begin{pgfscope}%
\pgfsetbuttcap%
\pgfsetroundjoin%
\definecolor{currentfill}{rgb}{0.000000,0.000000,0.000000}%
\pgfsetfillcolor{currentfill}%
\pgfsetlinewidth{0.602250pt}%
\definecolor{currentstroke}{rgb}{0.000000,0.000000,0.000000}%
\pgfsetstrokecolor{currentstroke}%
\pgfsetdash{}{0pt}%
\pgfsys@defobject{currentmarker}{\pgfqpoint{-0.027778in}{0.000000in}}{\pgfqpoint{0.000000in}{0.000000in}}{%
\pgfpathmoveto{\pgfqpoint{0.000000in}{0.000000in}}%
\pgfpathlineto{\pgfqpoint{-0.027778in}{0.000000in}}%
\pgfusepath{stroke,fill}%
}%
\begin{pgfscope}%
\pgfsys@transformshift{0.781944in}{1.714886in}%
\pgfsys@useobject{currentmarker}{}%
\end{pgfscope}%
\end{pgfscope}%
\begin{pgfscope}%
\pgfsetbuttcap%
\pgfsetroundjoin%
\definecolor{currentfill}{rgb}{0.000000,0.000000,0.000000}%
\pgfsetfillcolor{currentfill}%
\pgfsetlinewidth{0.602250pt}%
\definecolor{currentstroke}{rgb}{0.000000,0.000000,0.000000}%
\pgfsetstrokecolor{currentstroke}%
\pgfsetdash{}{0pt}%
\pgfsys@defobject{currentmarker}{\pgfqpoint{0.000000in}{0.000000in}}{\pgfqpoint{0.027778in}{0.000000in}}{%
\pgfpathmoveto{\pgfqpoint{0.000000in}{0.000000in}}%
\pgfpathlineto{\pgfqpoint{0.027778in}{0.000000in}}%
\pgfusepath{stroke,fill}%
}%
\begin{pgfscope}%
\pgfsys@transformshift{2.920660in}{1.714886in}%
\pgfsys@useobject{currentmarker}{}%
\end{pgfscope}%
\end{pgfscope}%
\begin{pgfscope}%
\pgfpathrectangle{\pgfqpoint{0.781944in}{0.552778in}}{\pgfqpoint{2.138715in}{1.650000in}}%
\pgfusepath{clip}%
\pgfsetrectcap%
\pgfsetroundjoin%
\pgfsetlinewidth{0.803000pt}%
\definecolor{currentstroke}{rgb}{0.690196,0.690196,0.690196}%
\pgfsetstrokecolor{currentstroke}%
\pgfsetstrokeopacity{0.300000}%
\pgfsetdash{}{0pt}%
\pgfpathmoveto{\pgfqpoint{0.781944in}{1.756698in}}%
\pgfpathlineto{\pgfqpoint{2.920660in}{1.756698in}}%
\pgfusepath{stroke}%
\end{pgfscope}%
\begin{pgfscope}%
\pgfsetbuttcap%
\pgfsetroundjoin%
\definecolor{currentfill}{rgb}{0.000000,0.000000,0.000000}%
\pgfsetfillcolor{currentfill}%
\pgfsetlinewidth{0.602250pt}%
\definecolor{currentstroke}{rgb}{0.000000,0.000000,0.000000}%
\pgfsetstrokecolor{currentstroke}%
\pgfsetdash{}{0pt}%
\pgfsys@defobject{currentmarker}{\pgfqpoint{-0.027778in}{0.000000in}}{\pgfqpoint{0.000000in}{0.000000in}}{%
\pgfpathmoveto{\pgfqpoint{0.000000in}{0.000000in}}%
\pgfpathlineto{\pgfqpoint{-0.027778in}{0.000000in}}%
\pgfusepath{stroke,fill}%
}%
\begin{pgfscope}%
\pgfsys@transformshift{0.781944in}{1.756698in}%
\pgfsys@useobject{currentmarker}{}%
\end{pgfscope}%
\end{pgfscope}%
\begin{pgfscope}%
\pgfsetbuttcap%
\pgfsetroundjoin%
\definecolor{currentfill}{rgb}{0.000000,0.000000,0.000000}%
\pgfsetfillcolor{currentfill}%
\pgfsetlinewidth{0.602250pt}%
\definecolor{currentstroke}{rgb}{0.000000,0.000000,0.000000}%
\pgfsetstrokecolor{currentstroke}%
\pgfsetdash{}{0pt}%
\pgfsys@defobject{currentmarker}{\pgfqpoint{0.000000in}{0.000000in}}{\pgfqpoint{0.027778in}{0.000000in}}{%
\pgfpathmoveto{\pgfqpoint{0.000000in}{0.000000in}}%
\pgfpathlineto{\pgfqpoint{0.027778in}{0.000000in}}%
\pgfusepath{stroke,fill}%
}%
\begin{pgfscope}%
\pgfsys@transformshift{2.920660in}{1.756698in}%
\pgfsys@useobject{currentmarker}{}%
\end{pgfscope}%
\end{pgfscope}%
\begin{pgfscope}%
\pgfpathrectangle{\pgfqpoint{0.781944in}{0.552778in}}{\pgfqpoint{2.138715in}{1.650000in}}%
\pgfusepath{clip}%
\pgfsetrectcap%
\pgfsetroundjoin%
\pgfsetlinewidth{0.803000pt}%
\definecolor{currentstroke}{rgb}{0.690196,0.690196,0.690196}%
\pgfsetstrokecolor{currentstroke}%
\pgfsetstrokeopacity{0.300000}%
\pgfsetdash{}{0pt}%
\pgfpathmoveto{\pgfqpoint{0.781944in}{1.798509in}}%
\pgfpathlineto{\pgfqpoint{2.920660in}{1.798509in}}%
\pgfusepath{stroke}%
\end{pgfscope}%
\begin{pgfscope}%
\pgfsetbuttcap%
\pgfsetroundjoin%
\definecolor{currentfill}{rgb}{0.000000,0.000000,0.000000}%
\pgfsetfillcolor{currentfill}%
\pgfsetlinewidth{0.602250pt}%
\definecolor{currentstroke}{rgb}{0.000000,0.000000,0.000000}%
\pgfsetstrokecolor{currentstroke}%
\pgfsetdash{}{0pt}%
\pgfsys@defobject{currentmarker}{\pgfqpoint{-0.027778in}{0.000000in}}{\pgfqpoint{0.000000in}{0.000000in}}{%
\pgfpathmoveto{\pgfqpoint{0.000000in}{0.000000in}}%
\pgfpathlineto{\pgfqpoint{-0.027778in}{0.000000in}}%
\pgfusepath{stroke,fill}%
}%
\begin{pgfscope}%
\pgfsys@transformshift{0.781944in}{1.798509in}%
\pgfsys@useobject{currentmarker}{}%
\end{pgfscope}%
\end{pgfscope}%
\begin{pgfscope}%
\pgfsetbuttcap%
\pgfsetroundjoin%
\definecolor{currentfill}{rgb}{0.000000,0.000000,0.000000}%
\pgfsetfillcolor{currentfill}%
\pgfsetlinewidth{0.602250pt}%
\definecolor{currentstroke}{rgb}{0.000000,0.000000,0.000000}%
\pgfsetstrokecolor{currentstroke}%
\pgfsetdash{}{0pt}%
\pgfsys@defobject{currentmarker}{\pgfqpoint{0.000000in}{0.000000in}}{\pgfqpoint{0.027778in}{0.000000in}}{%
\pgfpathmoveto{\pgfqpoint{0.000000in}{0.000000in}}%
\pgfpathlineto{\pgfqpoint{0.027778in}{0.000000in}}%
\pgfusepath{stroke,fill}%
}%
\begin{pgfscope}%
\pgfsys@transformshift{2.920660in}{1.798509in}%
\pgfsys@useobject{currentmarker}{}%
\end{pgfscope}%
\end{pgfscope}%
\begin{pgfscope}%
\pgfpathrectangle{\pgfqpoint{0.781944in}{0.552778in}}{\pgfqpoint{2.138715in}{1.650000in}}%
\pgfusepath{clip}%
\pgfsetrectcap%
\pgfsetroundjoin%
\pgfsetlinewidth{0.803000pt}%
\definecolor{currentstroke}{rgb}{0.690196,0.690196,0.690196}%
\pgfsetstrokecolor{currentstroke}%
\pgfsetstrokeopacity{0.300000}%
\pgfsetdash{}{0pt}%
\pgfpathmoveto{\pgfqpoint{0.781944in}{1.840321in}}%
\pgfpathlineto{\pgfqpoint{2.920660in}{1.840321in}}%
\pgfusepath{stroke}%
\end{pgfscope}%
\begin{pgfscope}%
\pgfsetbuttcap%
\pgfsetroundjoin%
\definecolor{currentfill}{rgb}{0.000000,0.000000,0.000000}%
\pgfsetfillcolor{currentfill}%
\pgfsetlinewidth{0.602250pt}%
\definecolor{currentstroke}{rgb}{0.000000,0.000000,0.000000}%
\pgfsetstrokecolor{currentstroke}%
\pgfsetdash{}{0pt}%
\pgfsys@defobject{currentmarker}{\pgfqpoint{-0.027778in}{0.000000in}}{\pgfqpoint{0.000000in}{0.000000in}}{%
\pgfpathmoveto{\pgfqpoint{0.000000in}{0.000000in}}%
\pgfpathlineto{\pgfqpoint{-0.027778in}{0.000000in}}%
\pgfusepath{stroke,fill}%
}%
\begin{pgfscope}%
\pgfsys@transformshift{0.781944in}{1.840321in}%
\pgfsys@useobject{currentmarker}{}%
\end{pgfscope}%
\end{pgfscope}%
\begin{pgfscope}%
\pgfsetbuttcap%
\pgfsetroundjoin%
\definecolor{currentfill}{rgb}{0.000000,0.000000,0.000000}%
\pgfsetfillcolor{currentfill}%
\pgfsetlinewidth{0.602250pt}%
\definecolor{currentstroke}{rgb}{0.000000,0.000000,0.000000}%
\pgfsetstrokecolor{currentstroke}%
\pgfsetdash{}{0pt}%
\pgfsys@defobject{currentmarker}{\pgfqpoint{0.000000in}{0.000000in}}{\pgfqpoint{0.027778in}{0.000000in}}{%
\pgfpathmoveto{\pgfqpoint{0.000000in}{0.000000in}}%
\pgfpathlineto{\pgfqpoint{0.027778in}{0.000000in}}%
\pgfusepath{stroke,fill}%
}%
\begin{pgfscope}%
\pgfsys@transformshift{2.920660in}{1.840321in}%
\pgfsys@useobject{currentmarker}{}%
\end{pgfscope}%
\end{pgfscope}%
\begin{pgfscope}%
\pgfpathrectangle{\pgfqpoint{0.781944in}{0.552778in}}{\pgfqpoint{2.138715in}{1.650000in}}%
\pgfusepath{clip}%
\pgfsetrectcap%
\pgfsetroundjoin%
\pgfsetlinewidth{0.803000pt}%
\definecolor{currentstroke}{rgb}{0.690196,0.690196,0.690196}%
\pgfsetstrokecolor{currentstroke}%
\pgfsetstrokeopacity{0.300000}%
\pgfsetdash{}{0pt}%
\pgfpathmoveto{\pgfqpoint{0.781944in}{1.923945in}}%
\pgfpathlineto{\pgfqpoint{2.920660in}{1.923945in}}%
\pgfusepath{stroke}%
\end{pgfscope}%
\begin{pgfscope}%
\pgfsetbuttcap%
\pgfsetroundjoin%
\definecolor{currentfill}{rgb}{0.000000,0.000000,0.000000}%
\pgfsetfillcolor{currentfill}%
\pgfsetlinewidth{0.602250pt}%
\definecolor{currentstroke}{rgb}{0.000000,0.000000,0.000000}%
\pgfsetstrokecolor{currentstroke}%
\pgfsetdash{}{0pt}%
\pgfsys@defobject{currentmarker}{\pgfqpoint{-0.027778in}{0.000000in}}{\pgfqpoint{0.000000in}{0.000000in}}{%
\pgfpathmoveto{\pgfqpoint{0.000000in}{0.000000in}}%
\pgfpathlineto{\pgfqpoint{-0.027778in}{0.000000in}}%
\pgfusepath{stroke,fill}%
}%
\begin{pgfscope}%
\pgfsys@transformshift{0.781944in}{1.923945in}%
\pgfsys@useobject{currentmarker}{}%
\end{pgfscope}%
\end{pgfscope}%
\begin{pgfscope}%
\pgfsetbuttcap%
\pgfsetroundjoin%
\definecolor{currentfill}{rgb}{0.000000,0.000000,0.000000}%
\pgfsetfillcolor{currentfill}%
\pgfsetlinewidth{0.602250pt}%
\definecolor{currentstroke}{rgb}{0.000000,0.000000,0.000000}%
\pgfsetstrokecolor{currentstroke}%
\pgfsetdash{}{0pt}%
\pgfsys@defobject{currentmarker}{\pgfqpoint{0.000000in}{0.000000in}}{\pgfqpoint{0.027778in}{0.000000in}}{%
\pgfpathmoveto{\pgfqpoint{0.000000in}{0.000000in}}%
\pgfpathlineto{\pgfqpoint{0.027778in}{0.000000in}}%
\pgfusepath{stroke,fill}%
}%
\begin{pgfscope}%
\pgfsys@transformshift{2.920660in}{1.923945in}%
\pgfsys@useobject{currentmarker}{}%
\end{pgfscope}%
\end{pgfscope}%
\begin{pgfscope}%
\pgfpathrectangle{\pgfqpoint{0.781944in}{0.552778in}}{\pgfqpoint{2.138715in}{1.650000in}}%
\pgfusepath{clip}%
\pgfsetrectcap%
\pgfsetroundjoin%
\pgfsetlinewidth{0.803000pt}%
\definecolor{currentstroke}{rgb}{0.690196,0.690196,0.690196}%
\pgfsetstrokecolor{currentstroke}%
\pgfsetstrokeopacity{0.300000}%
\pgfsetdash{}{0pt}%
\pgfpathmoveto{\pgfqpoint{0.781944in}{1.965757in}}%
\pgfpathlineto{\pgfqpoint{2.920660in}{1.965757in}}%
\pgfusepath{stroke}%
\end{pgfscope}%
\begin{pgfscope}%
\pgfsetbuttcap%
\pgfsetroundjoin%
\definecolor{currentfill}{rgb}{0.000000,0.000000,0.000000}%
\pgfsetfillcolor{currentfill}%
\pgfsetlinewidth{0.602250pt}%
\definecolor{currentstroke}{rgb}{0.000000,0.000000,0.000000}%
\pgfsetstrokecolor{currentstroke}%
\pgfsetdash{}{0pt}%
\pgfsys@defobject{currentmarker}{\pgfqpoint{-0.027778in}{0.000000in}}{\pgfqpoint{0.000000in}{0.000000in}}{%
\pgfpathmoveto{\pgfqpoint{0.000000in}{0.000000in}}%
\pgfpathlineto{\pgfqpoint{-0.027778in}{0.000000in}}%
\pgfusepath{stroke,fill}%
}%
\begin{pgfscope}%
\pgfsys@transformshift{0.781944in}{1.965757in}%
\pgfsys@useobject{currentmarker}{}%
\end{pgfscope}%
\end{pgfscope}%
\begin{pgfscope}%
\pgfsetbuttcap%
\pgfsetroundjoin%
\definecolor{currentfill}{rgb}{0.000000,0.000000,0.000000}%
\pgfsetfillcolor{currentfill}%
\pgfsetlinewidth{0.602250pt}%
\definecolor{currentstroke}{rgb}{0.000000,0.000000,0.000000}%
\pgfsetstrokecolor{currentstroke}%
\pgfsetdash{}{0pt}%
\pgfsys@defobject{currentmarker}{\pgfqpoint{0.000000in}{0.000000in}}{\pgfqpoint{0.027778in}{0.000000in}}{%
\pgfpathmoveto{\pgfqpoint{0.000000in}{0.000000in}}%
\pgfpathlineto{\pgfqpoint{0.027778in}{0.000000in}}%
\pgfusepath{stroke,fill}%
}%
\begin{pgfscope}%
\pgfsys@transformshift{2.920660in}{1.965757in}%
\pgfsys@useobject{currentmarker}{}%
\end{pgfscope}%
\end{pgfscope}%
\begin{pgfscope}%
\pgfpathrectangle{\pgfqpoint{0.781944in}{0.552778in}}{\pgfqpoint{2.138715in}{1.650000in}}%
\pgfusepath{clip}%
\pgfsetrectcap%
\pgfsetroundjoin%
\pgfsetlinewidth{0.803000pt}%
\definecolor{currentstroke}{rgb}{0.690196,0.690196,0.690196}%
\pgfsetstrokecolor{currentstroke}%
\pgfsetstrokeopacity{0.300000}%
\pgfsetdash{}{0pt}%
\pgfpathmoveto{\pgfqpoint{0.781944in}{2.007569in}}%
\pgfpathlineto{\pgfqpoint{2.920660in}{2.007569in}}%
\pgfusepath{stroke}%
\end{pgfscope}%
\begin{pgfscope}%
\pgfsetbuttcap%
\pgfsetroundjoin%
\definecolor{currentfill}{rgb}{0.000000,0.000000,0.000000}%
\pgfsetfillcolor{currentfill}%
\pgfsetlinewidth{0.602250pt}%
\definecolor{currentstroke}{rgb}{0.000000,0.000000,0.000000}%
\pgfsetstrokecolor{currentstroke}%
\pgfsetdash{}{0pt}%
\pgfsys@defobject{currentmarker}{\pgfqpoint{-0.027778in}{0.000000in}}{\pgfqpoint{0.000000in}{0.000000in}}{%
\pgfpathmoveto{\pgfqpoint{0.000000in}{0.000000in}}%
\pgfpathlineto{\pgfqpoint{-0.027778in}{0.000000in}}%
\pgfusepath{stroke,fill}%
}%
\begin{pgfscope}%
\pgfsys@transformshift{0.781944in}{2.007569in}%
\pgfsys@useobject{currentmarker}{}%
\end{pgfscope}%
\end{pgfscope}%
\begin{pgfscope}%
\pgfsetbuttcap%
\pgfsetroundjoin%
\definecolor{currentfill}{rgb}{0.000000,0.000000,0.000000}%
\pgfsetfillcolor{currentfill}%
\pgfsetlinewidth{0.602250pt}%
\definecolor{currentstroke}{rgb}{0.000000,0.000000,0.000000}%
\pgfsetstrokecolor{currentstroke}%
\pgfsetdash{}{0pt}%
\pgfsys@defobject{currentmarker}{\pgfqpoint{0.000000in}{0.000000in}}{\pgfqpoint{0.027778in}{0.000000in}}{%
\pgfpathmoveto{\pgfqpoint{0.000000in}{0.000000in}}%
\pgfpathlineto{\pgfqpoint{0.027778in}{0.000000in}}%
\pgfusepath{stroke,fill}%
}%
\begin{pgfscope}%
\pgfsys@transformshift{2.920660in}{2.007569in}%
\pgfsys@useobject{currentmarker}{}%
\end{pgfscope}%
\end{pgfscope}%
\begin{pgfscope}%
\pgfpathrectangle{\pgfqpoint{0.781944in}{0.552778in}}{\pgfqpoint{2.138715in}{1.650000in}}%
\pgfusepath{clip}%
\pgfsetrectcap%
\pgfsetroundjoin%
\pgfsetlinewidth{0.803000pt}%
\definecolor{currentstroke}{rgb}{0.690196,0.690196,0.690196}%
\pgfsetstrokecolor{currentstroke}%
\pgfsetstrokeopacity{0.300000}%
\pgfsetdash{}{0pt}%
\pgfpathmoveto{\pgfqpoint{0.781944in}{2.049381in}}%
\pgfpathlineto{\pgfqpoint{2.920660in}{2.049381in}}%
\pgfusepath{stroke}%
\end{pgfscope}%
\begin{pgfscope}%
\pgfsetbuttcap%
\pgfsetroundjoin%
\definecolor{currentfill}{rgb}{0.000000,0.000000,0.000000}%
\pgfsetfillcolor{currentfill}%
\pgfsetlinewidth{0.602250pt}%
\definecolor{currentstroke}{rgb}{0.000000,0.000000,0.000000}%
\pgfsetstrokecolor{currentstroke}%
\pgfsetdash{}{0pt}%
\pgfsys@defobject{currentmarker}{\pgfqpoint{-0.027778in}{0.000000in}}{\pgfqpoint{0.000000in}{0.000000in}}{%
\pgfpathmoveto{\pgfqpoint{0.000000in}{0.000000in}}%
\pgfpathlineto{\pgfqpoint{-0.027778in}{0.000000in}}%
\pgfusepath{stroke,fill}%
}%
\begin{pgfscope}%
\pgfsys@transformshift{0.781944in}{2.049381in}%
\pgfsys@useobject{currentmarker}{}%
\end{pgfscope}%
\end{pgfscope}%
\begin{pgfscope}%
\pgfsetbuttcap%
\pgfsetroundjoin%
\definecolor{currentfill}{rgb}{0.000000,0.000000,0.000000}%
\pgfsetfillcolor{currentfill}%
\pgfsetlinewidth{0.602250pt}%
\definecolor{currentstroke}{rgb}{0.000000,0.000000,0.000000}%
\pgfsetstrokecolor{currentstroke}%
\pgfsetdash{}{0pt}%
\pgfsys@defobject{currentmarker}{\pgfqpoint{0.000000in}{0.000000in}}{\pgfqpoint{0.027778in}{0.000000in}}{%
\pgfpathmoveto{\pgfqpoint{0.000000in}{0.000000in}}%
\pgfpathlineto{\pgfqpoint{0.027778in}{0.000000in}}%
\pgfusepath{stroke,fill}%
}%
\begin{pgfscope}%
\pgfsys@transformshift{2.920660in}{2.049381in}%
\pgfsys@useobject{currentmarker}{}%
\end{pgfscope}%
\end{pgfscope}%
\begin{pgfscope}%
\pgfpathrectangle{\pgfqpoint{0.781944in}{0.552778in}}{\pgfqpoint{2.138715in}{1.650000in}}%
\pgfusepath{clip}%
\pgfsetrectcap%
\pgfsetroundjoin%
\pgfsetlinewidth{0.803000pt}%
\definecolor{currentstroke}{rgb}{0.690196,0.690196,0.690196}%
\pgfsetstrokecolor{currentstroke}%
\pgfsetstrokeopacity{0.300000}%
\pgfsetdash{}{0pt}%
\pgfpathmoveto{\pgfqpoint{0.781944in}{2.091192in}}%
\pgfpathlineto{\pgfqpoint{2.920660in}{2.091192in}}%
\pgfusepath{stroke}%
\end{pgfscope}%
\begin{pgfscope}%
\pgfsetbuttcap%
\pgfsetroundjoin%
\definecolor{currentfill}{rgb}{0.000000,0.000000,0.000000}%
\pgfsetfillcolor{currentfill}%
\pgfsetlinewidth{0.602250pt}%
\definecolor{currentstroke}{rgb}{0.000000,0.000000,0.000000}%
\pgfsetstrokecolor{currentstroke}%
\pgfsetdash{}{0pt}%
\pgfsys@defobject{currentmarker}{\pgfqpoint{-0.027778in}{0.000000in}}{\pgfqpoint{0.000000in}{0.000000in}}{%
\pgfpathmoveto{\pgfqpoint{0.000000in}{0.000000in}}%
\pgfpathlineto{\pgfqpoint{-0.027778in}{0.000000in}}%
\pgfusepath{stroke,fill}%
}%
\begin{pgfscope}%
\pgfsys@transformshift{0.781944in}{2.091192in}%
\pgfsys@useobject{currentmarker}{}%
\end{pgfscope}%
\end{pgfscope}%
\begin{pgfscope}%
\pgfsetbuttcap%
\pgfsetroundjoin%
\definecolor{currentfill}{rgb}{0.000000,0.000000,0.000000}%
\pgfsetfillcolor{currentfill}%
\pgfsetlinewidth{0.602250pt}%
\definecolor{currentstroke}{rgb}{0.000000,0.000000,0.000000}%
\pgfsetstrokecolor{currentstroke}%
\pgfsetdash{}{0pt}%
\pgfsys@defobject{currentmarker}{\pgfqpoint{0.000000in}{0.000000in}}{\pgfqpoint{0.027778in}{0.000000in}}{%
\pgfpathmoveto{\pgfqpoint{0.000000in}{0.000000in}}%
\pgfpathlineto{\pgfqpoint{0.027778in}{0.000000in}}%
\pgfusepath{stroke,fill}%
}%
\begin{pgfscope}%
\pgfsys@transformshift{2.920660in}{2.091192in}%
\pgfsys@useobject{currentmarker}{}%
\end{pgfscope}%
\end{pgfscope}%
\begin{pgfscope}%
\pgfpathrectangle{\pgfqpoint{0.781944in}{0.552778in}}{\pgfqpoint{2.138715in}{1.650000in}}%
\pgfusepath{clip}%
\pgfsetrectcap%
\pgfsetroundjoin%
\pgfsetlinewidth{0.803000pt}%
\definecolor{currentstroke}{rgb}{0.690196,0.690196,0.690196}%
\pgfsetstrokecolor{currentstroke}%
\pgfsetstrokeopacity{0.300000}%
\pgfsetdash{}{0pt}%
\pgfpathmoveto{\pgfqpoint{0.781944in}{2.133004in}}%
\pgfpathlineto{\pgfqpoint{2.920660in}{2.133004in}}%
\pgfusepath{stroke}%
\end{pgfscope}%
\begin{pgfscope}%
\pgfsetbuttcap%
\pgfsetroundjoin%
\definecolor{currentfill}{rgb}{0.000000,0.000000,0.000000}%
\pgfsetfillcolor{currentfill}%
\pgfsetlinewidth{0.602250pt}%
\definecolor{currentstroke}{rgb}{0.000000,0.000000,0.000000}%
\pgfsetstrokecolor{currentstroke}%
\pgfsetdash{}{0pt}%
\pgfsys@defobject{currentmarker}{\pgfqpoint{-0.027778in}{0.000000in}}{\pgfqpoint{0.000000in}{0.000000in}}{%
\pgfpathmoveto{\pgfqpoint{0.000000in}{0.000000in}}%
\pgfpathlineto{\pgfqpoint{-0.027778in}{0.000000in}}%
\pgfusepath{stroke,fill}%
}%
\begin{pgfscope}%
\pgfsys@transformshift{0.781944in}{2.133004in}%
\pgfsys@useobject{currentmarker}{}%
\end{pgfscope}%
\end{pgfscope}%
\begin{pgfscope}%
\pgfsetbuttcap%
\pgfsetroundjoin%
\definecolor{currentfill}{rgb}{0.000000,0.000000,0.000000}%
\pgfsetfillcolor{currentfill}%
\pgfsetlinewidth{0.602250pt}%
\definecolor{currentstroke}{rgb}{0.000000,0.000000,0.000000}%
\pgfsetstrokecolor{currentstroke}%
\pgfsetdash{}{0pt}%
\pgfsys@defobject{currentmarker}{\pgfqpoint{0.000000in}{0.000000in}}{\pgfqpoint{0.027778in}{0.000000in}}{%
\pgfpathmoveto{\pgfqpoint{0.000000in}{0.000000in}}%
\pgfpathlineto{\pgfqpoint{0.027778in}{0.000000in}}%
\pgfusepath{stroke,fill}%
}%
\begin{pgfscope}%
\pgfsys@transformshift{2.920660in}{2.133004in}%
\pgfsys@useobject{currentmarker}{}%
\end{pgfscope}%
\end{pgfscope}%
\begin{pgfscope}%
\pgfpathrectangle{\pgfqpoint{0.781944in}{0.552778in}}{\pgfqpoint{2.138715in}{1.650000in}}%
\pgfusepath{clip}%
\pgfsetrectcap%
\pgfsetroundjoin%
\pgfsetlinewidth{0.803000pt}%
\definecolor{currentstroke}{rgb}{0.690196,0.690196,0.690196}%
\pgfsetstrokecolor{currentstroke}%
\pgfsetstrokeopacity{0.300000}%
\pgfsetdash{}{0pt}%
\pgfpathmoveto{\pgfqpoint{0.781944in}{2.174816in}}%
\pgfpathlineto{\pgfqpoint{2.920660in}{2.174816in}}%
\pgfusepath{stroke}%
\end{pgfscope}%
\begin{pgfscope}%
\pgfsetbuttcap%
\pgfsetroundjoin%
\definecolor{currentfill}{rgb}{0.000000,0.000000,0.000000}%
\pgfsetfillcolor{currentfill}%
\pgfsetlinewidth{0.602250pt}%
\definecolor{currentstroke}{rgb}{0.000000,0.000000,0.000000}%
\pgfsetstrokecolor{currentstroke}%
\pgfsetdash{}{0pt}%
\pgfsys@defobject{currentmarker}{\pgfqpoint{-0.027778in}{0.000000in}}{\pgfqpoint{0.000000in}{0.000000in}}{%
\pgfpathmoveto{\pgfqpoint{0.000000in}{0.000000in}}%
\pgfpathlineto{\pgfqpoint{-0.027778in}{0.000000in}}%
\pgfusepath{stroke,fill}%
}%
\begin{pgfscope}%
\pgfsys@transformshift{0.781944in}{2.174816in}%
\pgfsys@useobject{currentmarker}{}%
\end{pgfscope}%
\end{pgfscope}%
\begin{pgfscope}%
\pgfsetbuttcap%
\pgfsetroundjoin%
\definecolor{currentfill}{rgb}{0.000000,0.000000,0.000000}%
\pgfsetfillcolor{currentfill}%
\pgfsetlinewidth{0.602250pt}%
\definecolor{currentstroke}{rgb}{0.000000,0.000000,0.000000}%
\pgfsetstrokecolor{currentstroke}%
\pgfsetdash{}{0pt}%
\pgfsys@defobject{currentmarker}{\pgfqpoint{0.000000in}{0.000000in}}{\pgfqpoint{0.027778in}{0.000000in}}{%
\pgfpathmoveto{\pgfqpoint{0.000000in}{0.000000in}}%
\pgfpathlineto{\pgfqpoint{0.027778in}{0.000000in}}%
\pgfusepath{stroke,fill}%
}%
\begin{pgfscope}%
\pgfsys@transformshift{2.920660in}{2.174816in}%
\pgfsys@useobject{currentmarker}{}%
\end{pgfscope}%
\end{pgfscope}%
\begin{pgfscope}%
\definecolor{textcolor}{rgb}{0.000000,0.000000,0.000000}%
\pgfsetstrokecolor{textcolor}%
\pgfsetfillcolor{textcolor}%
\pgftext[x=0.420833in,y=1.377778in,,bottom,rotate=90.000000]{\color{textcolor}\rmfamily\fontsize{10.000000}{12.000000}\selectfont Ereignisszahl}%
\end{pgfscope}%
\begin{pgfscope}%
\pgfpathrectangle{\pgfqpoint{0.781944in}{0.552778in}}{\pgfqpoint{2.138715in}{1.650000in}}%
\pgfusepath{clip}%
\pgfsetrectcap%
\pgfsetroundjoin%
\pgfsetlinewidth{1.505625pt}%
\definecolor{currentstroke}{rgb}{0.121569,0.466667,0.705882}%
\pgfsetstrokecolor{currentstroke}%
\pgfsetdash{}{0pt}%
\pgfpathmoveto{\pgfqpoint{0.781889in}{0.627778in}}%
\pgfpathlineto{\pgfqpoint{0.782272in}{0.627778in}}%
\pgfpathlineto{\pgfqpoint{0.782272in}{0.633004in}}%
\pgfpathlineto{\pgfqpoint{0.783803in}{0.633004in}}%
\pgfpathlineto{\pgfqpoint{0.784186in}{0.633004in}}%
\pgfpathlineto{\pgfqpoint{0.784186in}{0.627778in}}%
\pgfpathlineto{\pgfqpoint{0.784569in}{0.638231in}}%
\pgfpathlineto{\pgfqpoint{0.785717in}{0.627778in}}%
\pgfpathlineto{\pgfqpoint{0.786100in}{0.627778in}}%
\pgfpathlineto{\pgfqpoint{0.786866in}{0.638231in}}%
\pgfpathlineto{\pgfqpoint{0.787631in}{0.638231in}}%
\pgfpathlineto{\pgfqpoint{0.788014in}{0.638231in}}%
\pgfpathlineto{\pgfqpoint{0.788014in}{0.648684in}}%
\pgfpathlineto{\pgfqpoint{0.789545in}{0.627778in}}%
\pgfpathlineto{\pgfqpoint{0.789928in}{0.627778in}}%
\pgfpathlineto{\pgfqpoint{0.790694in}{0.643457in}}%
\pgfpathlineto{\pgfqpoint{0.791460in}{0.638231in}}%
\pgfpathlineto{\pgfqpoint{0.791842in}{0.638231in}}%
\pgfpathlineto{\pgfqpoint{0.791842in}{0.633004in}}%
\pgfpathlineto{\pgfqpoint{0.792608in}{0.648684in}}%
\pgfpathlineto{\pgfqpoint{0.793374in}{0.638231in}}%
\pgfpathlineto{\pgfqpoint{0.793757in}{0.638231in}}%
\pgfpathlineto{\pgfqpoint{0.794522in}{0.653910in}}%
\pgfpathlineto{\pgfqpoint{0.795288in}{0.638231in}}%
\pgfpathlineto{\pgfqpoint{0.795671in}{0.638231in}}%
\pgfpathlineto{\pgfqpoint{0.795671in}{0.633004in}}%
\pgfpathlineto{\pgfqpoint{0.796054in}{0.648684in}}%
\pgfpathlineto{\pgfqpoint{0.797202in}{0.633004in}}%
\pgfpathlineto{\pgfqpoint{0.797585in}{0.633004in}}%
\pgfpathlineto{\pgfqpoint{0.797585in}{0.627778in}}%
\pgfpathlineto{\pgfqpoint{0.798733in}{0.648684in}}%
\pgfpathlineto{\pgfqpoint{0.799116in}{0.633004in}}%
\pgfpathlineto{\pgfqpoint{0.799499in}{0.633004in}}%
\pgfpathlineto{\pgfqpoint{0.799499in}{0.648684in}}%
\pgfpathlineto{\pgfqpoint{0.801030in}{0.638231in}}%
\pgfpathlineto{\pgfqpoint{0.801413in}{0.638231in}}%
\pgfpathlineto{\pgfqpoint{0.801413in}{0.633004in}}%
\pgfpathlineto{\pgfqpoint{0.801796in}{0.653910in}}%
\pgfpathlineto{\pgfqpoint{0.802944in}{0.633004in}}%
\pgfpathlineto{\pgfqpoint{0.803327in}{0.633004in}}%
\pgfpathlineto{\pgfqpoint{0.804093in}{0.643457in}}%
\pgfpathlineto{\pgfqpoint{0.804476in}{0.627778in}}%
\pgfpathlineto{\pgfqpoint{0.804859in}{0.638231in}}%
\pgfpathlineto{\pgfqpoint{0.805241in}{0.638231in}}%
\pgfpathlineto{\pgfqpoint{0.805241in}{0.627778in}}%
\pgfpathlineto{\pgfqpoint{0.806390in}{0.648684in}}%
\pgfpathlineto{\pgfqpoint{0.806773in}{0.643457in}}%
\pgfpathlineto{\pgfqpoint{0.807156in}{0.643457in}}%
\pgfpathlineto{\pgfqpoint{0.807156in}{0.627778in}}%
\pgfpathlineto{\pgfqpoint{0.807538in}{0.648684in}}%
\pgfpathlineto{\pgfqpoint{0.808687in}{0.638231in}}%
\pgfpathlineto{\pgfqpoint{0.809070in}{0.638231in}}%
\pgfpathlineto{\pgfqpoint{0.809070in}{0.633004in}}%
\pgfpathlineto{\pgfqpoint{0.809835in}{0.648684in}}%
\pgfpathlineto{\pgfqpoint{0.810601in}{0.638231in}}%
\pgfpathlineto{\pgfqpoint{0.810984in}{0.638231in}}%
\pgfpathlineto{\pgfqpoint{0.811750in}{0.664363in}}%
\pgfpathlineto{\pgfqpoint{0.811367in}{0.627778in}}%
\pgfpathlineto{\pgfqpoint{0.812515in}{0.653910in}}%
\pgfpathlineto{\pgfqpoint{0.812898in}{0.653910in}}%
\pgfpathlineto{\pgfqpoint{0.813281in}{0.638231in}}%
\pgfpathlineto{\pgfqpoint{0.814047in}{0.659137in}}%
\pgfpathlineto{\pgfqpoint{0.814429in}{0.648684in}}%
\pgfpathlineto{\pgfqpoint{0.814812in}{0.648684in}}%
\pgfpathlineto{\pgfqpoint{0.815195in}{0.659137in}}%
\pgfpathlineto{\pgfqpoint{0.816344in}{0.627778in}}%
\pgfpathlineto{\pgfqpoint{0.816726in}{0.627778in}}%
\pgfpathlineto{\pgfqpoint{0.818258in}{0.648684in}}%
\pgfpathlineto{\pgfqpoint{0.818641in}{0.648684in}}%
\pgfpathlineto{\pgfqpoint{0.819406in}{0.653910in}}%
\pgfpathlineto{\pgfqpoint{0.820172in}{0.633004in}}%
\pgfpathlineto{\pgfqpoint{0.820555in}{0.633004in}}%
\pgfpathlineto{\pgfqpoint{0.820555in}{0.659137in}}%
\pgfpathlineto{\pgfqpoint{0.821320in}{0.627778in}}%
\pgfpathlineto{\pgfqpoint{0.822086in}{0.633004in}}%
\pgfpathlineto{\pgfqpoint{0.822469in}{0.633004in}}%
\pgfpathlineto{\pgfqpoint{0.823234in}{0.659137in}}%
\pgfpathlineto{\pgfqpoint{0.824000in}{0.638231in}}%
\pgfpathlineto{\pgfqpoint{0.824383in}{0.638231in}}%
\pgfpathlineto{\pgfqpoint{0.824383in}{0.659137in}}%
\pgfpathlineto{\pgfqpoint{0.825914in}{0.638231in}}%
\pgfpathlineto{\pgfqpoint{0.826297in}{0.638231in}}%
\pgfpathlineto{\pgfqpoint{0.826680in}{0.653910in}}%
\pgfpathlineto{\pgfqpoint{0.827828in}{0.648684in}}%
\pgfpathlineto{\pgfqpoint{0.828211in}{0.648684in}}%
\pgfpathlineto{\pgfqpoint{0.828211in}{0.653910in}}%
\pgfpathlineto{\pgfqpoint{0.829743in}{0.643457in}}%
\pgfpathlineto{\pgfqpoint{0.830125in}{0.643457in}}%
\pgfpathlineto{\pgfqpoint{0.830125in}{0.659137in}}%
\pgfpathlineto{\pgfqpoint{0.830508in}{0.633004in}}%
\pgfpathlineto{\pgfqpoint{0.831657in}{0.648684in}}%
\pgfpathlineto{\pgfqpoint{0.832040in}{0.648684in}}%
\pgfpathlineto{\pgfqpoint{0.832040in}{0.664363in}}%
\pgfpathlineto{\pgfqpoint{0.832805in}{0.633004in}}%
\pgfpathlineto{\pgfqpoint{0.833571in}{0.659137in}}%
\pgfpathlineto{\pgfqpoint{0.833954in}{0.659137in}}%
\pgfpathlineto{\pgfqpoint{0.833954in}{0.638231in}}%
\pgfpathlineto{\pgfqpoint{0.834719in}{0.664363in}}%
\pgfpathlineto{\pgfqpoint{0.835485in}{0.643457in}}%
\pgfpathlineto{\pgfqpoint{0.835868in}{0.643457in}}%
\pgfpathlineto{\pgfqpoint{0.835868in}{0.669590in}}%
\pgfpathlineto{\pgfqpoint{0.836251in}{0.633004in}}%
\pgfpathlineto{\pgfqpoint{0.837399in}{0.633004in}}%
\pgfpathlineto{\pgfqpoint{0.837782in}{0.633004in}}%
\pgfpathlineto{\pgfqpoint{0.839313in}{0.659137in}}%
\pgfpathlineto{\pgfqpoint{0.839696in}{0.659137in}}%
\pgfpathlineto{\pgfqpoint{0.839696in}{0.674816in}}%
\pgfpathlineto{\pgfqpoint{0.841227in}{0.643457in}}%
\pgfpathlineto{\pgfqpoint{0.841610in}{0.643457in}}%
\pgfpathlineto{\pgfqpoint{0.842759in}{0.669590in}}%
\pgfpathlineto{\pgfqpoint{0.843142in}{0.659137in}}%
\pgfpathlineto{\pgfqpoint{0.843524in}{0.659137in}}%
\pgfpathlineto{\pgfqpoint{0.843524in}{0.643457in}}%
\pgfpathlineto{\pgfqpoint{0.844673in}{0.669590in}}%
\pgfpathlineto{\pgfqpoint{0.845056in}{0.648684in}}%
\pgfpathlineto{\pgfqpoint{0.845439in}{0.648684in}}%
\pgfpathlineto{\pgfqpoint{0.846204in}{0.643457in}}%
\pgfpathlineto{\pgfqpoint{0.846970in}{0.674816in}}%
\pgfpathlineto{\pgfqpoint{0.847353in}{0.674816in}}%
\pgfpathlineto{\pgfqpoint{0.848118in}{0.643457in}}%
\pgfpathlineto{\pgfqpoint{0.848884in}{0.653910in}}%
\pgfpathlineto{\pgfqpoint{0.849267in}{0.653910in}}%
\pgfpathlineto{\pgfqpoint{0.849267in}{0.643457in}}%
\pgfpathlineto{\pgfqpoint{0.850415in}{0.690496in}}%
\pgfpathlineto{\pgfqpoint{0.850798in}{0.659137in}}%
\pgfpathlineto{\pgfqpoint{0.851181in}{0.659137in}}%
\pgfpathlineto{\pgfqpoint{0.851181in}{0.633004in}}%
\pgfpathlineto{\pgfqpoint{0.852712in}{0.659137in}}%
\pgfpathlineto{\pgfqpoint{0.853095in}{0.659137in}}%
\pgfpathlineto{\pgfqpoint{0.853861in}{0.643457in}}%
\pgfpathlineto{\pgfqpoint{0.854627in}{0.695722in}}%
\pgfpathlineto{\pgfqpoint{0.855009in}{0.695722in}}%
\pgfpathlineto{\pgfqpoint{0.856158in}{0.643457in}}%
\pgfpathlineto{\pgfqpoint{0.856541in}{0.664363in}}%
\pgfpathlineto{\pgfqpoint{0.856924in}{0.664363in}}%
\pgfpathlineto{\pgfqpoint{0.857306in}{0.653910in}}%
\pgfpathlineto{\pgfqpoint{0.858072in}{0.669590in}}%
\pgfpathlineto{\pgfqpoint{0.858455in}{0.664363in}}%
\pgfpathlineto{\pgfqpoint{0.858838in}{0.664363in}}%
\pgfpathlineto{\pgfqpoint{0.859221in}{0.643457in}}%
\pgfpathlineto{\pgfqpoint{0.860369in}{0.653910in}}%
\pgfpathlineto{\pgfqpoint{0.860752in}{0.653910in}}%
\pgfpathlineto{\pgfqpoint{0.861135in}{0.685269in}}%
\pgfpathlineto{\pgfqpoint{0.862283in}{0.659137in}}%
\pgfpathlineto{\pgfqpoint{0.862666in}{0.659137in}}%
\pgfpathlineto{\pgfqpoint{0.862666in}{0.674816in}}%
\pgfpathlineto{\pgfqpoint{0.864197in}{0.669590in}}%
\pgfpathlineto{\pgfqpoint{0.864963in}{0.669590in}}%
\pgfpathlineto{\pgfqpoint{0.865346in}{0.653910in}}%
\pgfpathlineto{\pgfqpoint{0.865729in}{0.680043in}}%
\pgfpathlineto{\pgfqpoint{0.866494in}{0.669590in}}%
\pgfpathlineto{\pgfqpoint{0.866877in}{0.669590in}}%
\pgfpathlineto{\pgfqpoint{0.866877in}{0.695722in}}%
\pgfpathlineto{\pgfqpoint{0.868026in}{0.659137in}}%
\pgfpathlineto{\pgfqpoint{0.868408in}{0.664363in}}%
\pgfpathlineto{\pgfqpoint{0.868791in}{0.664363in}}%
\pgfpathlineto{\pgfqpoint{0.868791in}{0.659137in}}%
\pgfpathlineto{\pgfqpoint{0.869940in}{0.700949in}}%
\pgfpathlineto{\pgfqpoint{0.870323in}{0.659137in}}%
\pgfpathlineto{\pgfqpoint{0.870705in}{0.659137in}}%
\pgfpathlineto{\pgfqpoint{0.871854in}{0.643457in}}%
\pgfpathlineto{\pgfqpoint{0.872237in}{0.695722in}}%
\pgfpathlineto{\pgfqpoint{0.872620in}{0.695722in}}%
\pgfpathlineto{\pgfqpoint{0.873768in}{0.664363in}}%
\pgfpathlineto{\pgfqpoint{0.874151in}{0.685269in}}%
\pgfpathlineto{\pgfqpoint{0.874534in}{0.685269in}}%
\pgfpathlineto{\pgfqpoint{0.874534in}{0.664363in}}%
\pgfpathlineto{\pgfqpoint{0.874917in}{0.695722in}}%
\pgfpathlineto{\pgfqpoint{0.876065in}{0.669590in}}%
\pgfpathlineto{\pgfqpoint{0.876448in}{0.669590in}}%
\pgfpathlineto{\pgfqpoint{0.877596in}{0.716628in}}%
\pgfpathlineto{\pgfqpoint{0.877979in}{0.653910in}}%
\pgfpathlineto{\pgfqpoint{0.878362in}{0.653910in}}%
\pgfpathlineto{\pgfqpoint{0.879510in}{0.695722in}}%
\pgfpathlineto{\pgfqpoint{0.879893in}{0.664363in}}%
\pgfpathlineto{\pgfqpoint{0.880276in}{0.664363in}}%
\pgfpathlineto{\pgfqpoint{0.880276in}{0.711401in}}%
\pgfpathlineto{\pgfqpoint{0.881807in}{0.680043in}}%
\pgfpathlineto{\pgfqpoint{0.882190in}{0.680043in}}%
\pgfpathlineto{\pgfqpoint{0.882573in}{0.669590in}}%
\pgfpathlineto{\pgfqpoint{0.883339in}{0.690496in}}%
\pgfpathlineto{\pgfqpoint{0.883722in}{0.685269in}}%
\pgfpathlineto{\pgfqpoint{0.884104in}{0.685269in}}%
\pgfpathlineto{\pgfqpoint{0.884104in}{0.690496in}}%
\pgfpathlineto{\pgfqpoint{0.885253in}{0.653910in}}%
\pgfpathlineto{\pgfqpoint{0.885636in}{0.690496in}}%
\pgfpathlineto{\pgfqpoint{0.886019in}{0.690496in}}%
\pgfpathlineto{\pgfqpoint{0.886019in}{0.648684in}}%
\pgfpathlineto{\pgfqpoint{0.887550in}{0.680043in}}%
\pgfpathlineto{\pgfqpoint{0.887933in}{0.680043in}}%
\pgfpathlineto{\pgfqpoint{0.889081in}{0.737534in}}%
\pgfpathlineto{\pgfqpoint{0.889464in}{0.685269in}}%
\pgfpathlineto{\pgfqpoint{0.889847in}{0.685269in}}%
\pgfpathlineto{\pgfqpoint{0.890995in}{0.674816in}}%
\pgfpathlineto{\pgfqpoint{0.891378in}{0.721854in}}%
\pgfpathlineto{\pgfqpoint{0.891761in}{0.721854in}}%
\pgfpathlineto{\pgfqpoint{0.892144in}{0.674816in}}%
\pgfpathlineto{\pgfqpoint{0.893292in}{0.700949in}}%
\pgfpathlineto{\pgfqpoint{0.893675in}{0.700949in}}%
\pgfpathlineto{\pgfqpoint{0.894058in}{0.664363in}}%
\pgfpathlineto{\pgfqpoint{0.895207in}{0.690496in}}%
\pgfpathlineto{\pgfqpoint{0.895589in}{0.690496in}}%
\pgfpathlineto{\pgfqpoint{0.895589in}{0.664363in}}%
\pgfpathlineto{\pgfqpoint{0.895972in}{0.732307in}}%
\pgfpathlineto{\pgfqpoint{0.897121in}{0.695722in}}%
\pgfpathlineto{\pgfqpoint{0.897504in}{0.695722in}}%
\pgfpathlineto{\pgfqpoint{0.897504in}{0.716628in}}%
\pgfpathlineto{\pgfqpoint{0.898269in}{0.648684in}}%
\pgfpathlineto{\pgfqpoint{0.899035in}{0.664363in}}%
\pgfpathlineto{\pgfqpoint{0.899418in}{0.664363in}}%
\pgfpathlineto{\pgfqpoint{0.900183in}{0.716628in}}%
\pgfpathlineto{\pgfqpoint{0.900949in}{0.685269in}}%
\pgfpathlineto{\pgfqpoint{0.901332in}{0.685269in}}%
\pgfpathlineto{\pgfqpoint{0.901332in}{0.732307in}}%
\pgfpathlineto{\pgfqpoint{0.901715in}{0.680043in}}%
\pgfpathlineto{\pgfqpoint{0.902863in}{0.716628in}}%
\pgfpathlineto{\pgfqpoint{0.903246in}{0.716628in}}%
\pgfpathlineto{\pgfqpoint{0.903629in}{0.680043in}}%
\pgfpathlineto{\pgfqpoint{0.904394in}{0.742760in}}%
\pgfpathlineto{\pgfqpoint{0.904777in}{0.706175in}}%
\pgfpathlineto{\pgfqpoint{0.905160in}{0.706175in}}%
\pgfpathlineto{\pgfqpoint{0.905926in}{0.680043in}}%
\pgfpathlineto{\pgfqpoint{0.906691in}{0.685269in}}%
\pgfpathlineto{\pgfqpoint{0.907074in}{0.685269in}}%
\pgfpathlineto{\pgfqpoint{0.908223in}{0.711401in}}%
\pgfpathlineto{\pgfqpoint{0.908606in}{0.674816in}}%
\pgfpathlineto{\pgfqpoint{0.908988in}{0.674816in}}%
\pgfpathlineto{\pgfqpoint{0.910137in}{0.659137in}}%
\pgfpathlineto{\pgfqpoint{0.910520in}{0.727081in}}%
\pgfpathlineto{\pgfqpoint{0.910903in}{0.727081in}}%
\pgfpathlineto{\pgfqpoint{0.910903in}{0.669590in}}%
\pgfpathlineto{\pgfqpoint{0.912434in}{0.685269in}}%
\pgfpathlineto{\pgfqpoint{0.912817in}{0.685269in}}%
\pgfpathlineto{\pgfqpoint{0.913965in}{0.737534in}}%
\pgfpathlineto{\pgfqpoint{0.913582in}{0.674816in}}%
\pgfpathlineto{\pgfqpoint{0.914348in}{0.674816in}}%
\pgfpathlineto{\pgfqpoint{0.914731in}{0.674816in}}%
\pgfpathlineto{\pgfqpoint{0.915114in}{0.727081in}}%
\pgfpathlineto{\pgfqpoint{0.915497in}{0.669590in}}%
\pgfpathlineto{\pgfqpoint{0.916262in}{0.685269in}}%
\pgfpathlineto{\pgfqpoint{0.917028in}{0.685269in}}%
\pgfpathlineto{\pgfqpoint{0.918176in}{0.721854in}}%
\pgfpathlineto{\pgfqpoint{0.918559in}{0.706175in}}%
\pgfpathlineto{\pgfqpoint{0.918942in}{0.706175in}}%
\pgfpathlineto{\pgfqpoint{0.919325in}{0.685269in}}%
\pgfpathlineto{\pgfqpoint{0.920090in}{0.758440in}}%
\pgfpathlineto{\pgfqpoint{0.920473in}{0.737534in}}%
\pgfpathlineto{\pgfqpoint{0.920856in}{0.737534in}}%
\pgfpathlineto{\pgfqpoint{0.922005in}{0.664363in}}%
\pgfpathlineto{\pgfqpoint{0.921239in}{0.747987in}}%
\pgfpathlineto{\pgfqpoint{0.922387in}{0.700949in}}%
\pgfpathlineto{\pgfqpoint{0.922770in}{0.700949in}}%
\pgfpathlineto{\pgfqpoint{0.922770in}{0.727081in}}%
\pgfpathlineto{\pgfqpoint{0.924302in}{0.716628in}}%
\pgfpathlineto{\pgfqpoint{0.924684in}{0.716628in}}%
\pgfpathlineto{\pgfqpoint{0.925833in}{0.674816in}}%
\pgfpathlineto{\pgfqpoint{0.926216in}{0.716628in}}%
\pgfpathlineto{\pgfqpoint{0.926599in}{0.716628in}}%
\pgfpathlineto{\pgfqpoint{0.927747in}{0.758440in}}%
\pgfpathlineto{\pgfqpoint{0.926981in}{0.690496in}}%
\pgfpathlineto{\pgfqpoint{0.928130in}{0.727081in}}%
\pgfpathlineto{\pgfqpoint{0.928513in}{0.727081in}}%
\pgfpathlineto{\pgfqpoint{0.928513in}{0.690496in}}%
\pgfpathlineto{\pgfqpoint{0.930044in}{0.747987in}}%
\pgfpathlineto{\pgfqpoint{0.930427in}{0.747987in}}%
\pgfpathlineto{\pgfqpoint{0.930427in}{0.774119in}}%
\pgfpathlineto{\pgfqpoint{0.930810in}{0.711401in}}%
\pgfpathlineto{\pgfqpoint{0.931958in}{0.711401in}}%
\pgfpathlineto{\pgfqpoint{0.932341in}{0.711401in}}%
\pgfpathlineto{\pgfqpoint{0.933490in}{0.774119in}}%
\pgfpathlineto{\pgfqpoint{0.933872in}{0.711401in}}%
\pgfpathlineto{\pgfqpoint{0.934255in}{0.711401in}}%
\pgfpathlineto{\pgfqpoint{0.935404in}{0.747987in}}%
\pgfpathlineto{\pgfqpoint{0.935787in}{0.742760in}}%
\pgfpathlineto{\pgfqpoint{0.936169in}{0.742760in}}%
\pgfpathlineto{\pgfqpoint{0.936552in}{0.763666in}}%
\pgfpathlineto{\pgfqpoint{0.936935in}{0.700949in}}%
\pgfpathlineto{\pgfqpoint{0.937701in}{0.742760in}}%
\pgfpathlineto{\pgfqpoint{0.938083in}{0.742760in}}%
\pgfpathlineto{\pgfqpoint{0.938083in}{0.706175in}}%
\pgfpathlineto{\pgfqpoint{0.939615in}{0.758440in}}%
\pgfpathlineto{\pgfqpoint{0.939998in}{0.758440in}}%
\pgfpathlineto{\pgfqpoint{0.939998in}{0.716628in}}%
\pgfpathlineto{\pgfqpoint{0.941529in}{0.732307in}}%
\pgfpathlineto{\pgfqpoint{0.941912in}{0.732307in}}%
\pgfpathlineto{\pgfqpoint{0.942677in}{0.706175in}}%
\pgfpathlineto{\pgfqpoint{0.943443in}{0.721854in}}%
\pgfpathlineto{\pgfqpoint{0.944209in}{0.721854in}}%
\pgfpathlineto{\pgfqpoint{0.945357in}{0.706175in}}%
\pgfpathlineto{\pgfqpoint{0.945740in}{0.737534in}}%
\pgfpathlineto{\pgfqpoint{0.946123in}{0.737534in}}%
\pgfpathlineto{\pgfqpoint{0.946506in}{0.758440in}}%
\pgfpathlineto{\pgfqpoint{0.947654in}{0.700949in}}%
\pgfpathlineto{\pgfqpoint{0.948037in}{0.700949in}}%
\pgfpathlineto{\pgfqpoint{0.948420in}{0.779346in}}%
\pgfpathlineto{\pgfqpoint{0.949568in}{0.727081in}}%
\pgfpathlineto{\pgfqpoint{0.949951in}{0.727081in}}%
\pgfpathlineto{\pgfqpoint{0.950334in}{0.747987in}}%
\pgfpathlineto{\pgfqpoint{0.950717in}{0.690496in}}%
\pgfpathlineto{\pgfqpoint{0.951483in}{0.711401in}}%
\pgfpathlineto{\pgfqpoint{0.951865in}{0.711401in}}%
\pgfpathlineto{\pgfqpoint{0.953014in}{0.763666in}}%
\pgfpathlineto{\pgfqpoint{0.953397in}{0.695722in}}%
\pgfpathlineto{\pgfqpoint{0.953780in}{0.695722in}}%
\pgfpathlineto{\pgfqpoint{0.955311in}{0.774119in}}%
\pgfpathlineto{\pgfqpoint{0.955694in}{0.774119in}}%
\pgfpathlineto{\pgfqpoint{0.956077in}{0.695722in}}%
\pgfpathlineto{\pgfqpoint{0.957225in}{0.784572in}}%
\pgfpathlineto{\pgfqpoint{0.957608in}{0.784572in}}%
\pgfpathlineto{\pgfqpoint{0.959139in}{0.727081in}}%
\pgfpathlineto{\pgfqpoint{0.959522in}{0.727081in}}%
\pgfpathlineto{\pgfqpoint{0.960670in}{0.784572in}}%
\pgfpathlineto{\pgfqpoint{0.960288in}{0.680043in}}%
\pgfpathlineto{\pgfqpoint{0.961053in}{0.758440in}}%
\pgfpathlineto{\pgfqpoint{0.961436in}{0.758440in}}%
\pgfpathlineto{\pgfqpoint{0.961819in}{0.784572in}}%
\pgfpathlineto{\pgfqpoint{0.962967in}{0.716628in}}%
\pgfpathlineto{\pgfqpoint{0.963350in}{0.716628in}}%
\pgfpathlineto{\pgfqpoint{0.963733in}{0.789799in}}%
\pgfpathlineto{\pgfqpoint{0.964882in}{0.784572in}}%
\pgfpathlineto{\pgfqpoint{0.965264in}{0.784572in}}%
\pgfpathlineto{\pgfqpoint{0.965647in}{0.695722in}}%
\pgfpathlineto{\pgfqpoint{0.966796in}{0.758440in}}%
\pgfpathlineto{\pgfqpoint{0.967179in}{0.758440in}}%
\pgfpathlineto{\pgfqpoint{0.968327in}{0.800252in}}%
\pgfpathlineto{\pgfqpoint{0.967561in}{0.737534in}}%
\pgfpathlineto{\pgfqpoint{0.968710in}{0.768893in}}%
\pgfpathlineto{\pgfqpoint{0.969093in}{0.768893in}}%
\pgfpathlineto{\pgfqpoint{0.969093in}{0.716628in}}%
\pgfpathlineto{\pgfqpoint{0.970624in}{0.784572in}}%
\pgfpathlineto{\pgfqpoint{0.971007in}{0.784572in}}%
\pgfpathlineto{\pgfqpoint{0.971773in}{0.742760in}}%
\pgfpathlineto{\pgfqpoint{0.972538in}{0.779346in}}%
\pgfpathlineto{\pgfqpoint{0.972921in}{0.779346in}}%
\pgfpathlineto{\pgfqpoint{0.973687in}{0.727081in}}%
\pgfpathlineto{\pgfqpoint{0.974452in}{0.800252in}}%
\pgfpathlineto{\pgfqpoint{0.974835in}{0.800252in}}%
\pgfpathlineto{\pgfqpoint{0.975218in}{0.716628in}}%
\pgfpathlineto{\pgfqpoint{0.976366in}{0.758440in}}%
\pgfpathlineto{\pgfqpoint{0.976749in}{0.758440in}}%
\pgfpathlineto{\pgfqpoint{0.977132in}{0.716628in}}%
\pgfpathlineto{\pgfqpoint{0.977898in}{0.810705in}}%
\pgfpathlineto{\pgfqpoint{0.978281in}{0.747987in}}%
\pgfpathlineto{\pgfqpoint{0.978663in}{0.747987in}}%
\pgfpathlineto{\pgfqpoint{0.979046in}{0.711401in}}%
\pgfpathlineto{\pgfqpoint{0.980195in}{0.805478in}}%
\pgfpathlineto{\pgfqpoint{0.980578in}{0.805478in}}%
\pgfpathlineto{\pgfqpoint{0.981343in}{0.690496in}}%
\pgfpathlineto{\pgfqpoint{0.982109in}{0.768893in}}%
\pgfpathlineto{\pgfqpoint{0.982492in}{0.768893in}}%
\pgfpathlineto{\pgfqpoint{0.982492in}{0.779346in}}%
\pgfpathlineto{\pgfqpoint{0.984023in}{0.716628in}}%
\pgfpathlineto{\pgfqpoint{0.984406in}{0.716628in}}%
\pgfpathlineto{\pgfqpoint{0.985937in}{0.774119in}}%
\pgfpathlineto{\pgfqpoint{0.986320in}{0.774119in}}%
\pgfpathlineto{\pgfqpoint{0.986320in}{0.727081in}}%
\pgfpathlineto{\pgfqpoint{0.987851in}{0.737534in}}%
\pgfpathlineto{\pgfqpoint{0.988234in}{0.737534in}}%
\pgfpathlineto{\pgfqpoint{0.989000in}{0.784572in}}%
\pgfpathlineto{\pgfqpoint{0.989383in}{0.721854in}}%
\pgfpathlineto{\pgfqpoint{0.989766in}{0.747987in}}%
\pgfpathlineto{\pgfqpoint{0.990148in}{0.747987in}}%
\pgfpathlineto{\pgfqpoint{0.990148in}{0.768893in}}%
\pgfpathlineto{\pgfqpoint{0.990914in}{0.732307in}}%
\pgfpathlineto{\pgfqpoint{0.991680in}{0.758440in}}%
\pgfpathlineto{\pgfqpoint{0.992063in}{0.758440in}}%
\pgfpathlineto{\pgfqpoint{0.992063in}{0.727081in}}%
\pgfpathlineto{\pgfqpoint{0.993211in}{0.795025in}}%
\pgfpathlineto{\pgfqpoint{0.993594in}{0.795025in}}%
\pgfpathlineto{\pgfqpoint{0.993977in}{0.795025in}}%
\pgfpathlineto{\pgfqpoint{0.993977in}{0.742760in}}%
\pgfpathlineto{\pgfqpoint{0.995508in}{0.742760in}}%
\pgfpathlineto{\pgfqpoint{0.995891in}{0.742760in}}%
\pgfpathlineto{\pgfqpoint{0.996274in}{0.795025in}}%
\pgfpathlineto{\pgfqpoint{0.997039in}{0.732307in}}%
\pgfpathlineto{\pgfqpoint{0.997422in}{0.747987in}}%
\pgfpathlineto{\pgfqpoint{0.997805in}{0.747987in}}%
\pgfpathlineto{\pgfqpoint{0.997805in}{0.789799in}}%
\pgfpathlineto{\pgfqpoint{0.998571in}{0.742760in}}%
\pgfpathlineto{\pgfqpoint{0.999336in}{0.784572in}}%
\pgfpathlineto{\pgfqpoint{0.999719in}{0.784572in}}%
\pgfpathlineto{\pgfqpoint{1.000868in}{0.716628in}}%
\pgfpathlineto{\pgfqpoint{1.001250in}{0.747987in}}%
\pgfpathlineto{\pgfqpoint{1.001633in}{0.747987in}}%
\pgfpathlineto{\pgfqpoint{1.002399in}{0.789799in}}%
\pgfpathlineto{\pgfqpoint{1.003165in}{0.758440in}}%
\pgfpathlineto{\pgfqpoint{1.003547in}{0.758440in}}%
\pgfpathlineto{\pgfqpoint{1.003547in}{0.747987in}}%
\pgfpathlineto{\pgfqpoint{1.003930in}{0.789799in}}%
\pgfpathlineto{\pgfqpoint{1.005079in}{0.789799in}}%
\pgfpathlineto{\pgfqpoint{1.005462in}{0.789799in}}%
\pgfpathlineto{\pgfqpoint{1.005462in}{0.810705in}}%
\pgfpathlineto{\pgfqpoint{1.006227in}{0.732307in}}%
\pgfpathlineto{\pgfqpoint{1.006993in}{0.800252in}}%
\pgfpathlineto{\pgfqpoint{1.007376in}{0.800252in}}%
\pgfpathlineto{\pgfqpoint{1.008141in}{0.747987in}}%
\pgfpathlineto{\pgfqpoint{1.008907in}{0.784572in}}%
\pgfpathlineto{\pgfqpoint{1.009290in}{0.784572in}}%
\pgfpathlineto{\pgfqpoint{1.009290in}{0.753213in}}%
\pgfpathlineto{\pgfqpoint{1.010821in}{0.800252in}}%
\pgfpathlineto{\pgfqpoint{1.011587in}{0.800252in}}%
\pgfpathlineto{\pgfqpoint{1.012353in}{0.742760in}}%
\pgfpathlineto{\pgfqpoint{1.012735in}{0.831611in}}%
\pgfpathlineto{\pgfqpoint{1.013118in}{0.789799in}}%
\pgfpathlineto{\pgfqpoint{1.013501in}{0.789799in}}%
\pgfpathlineto{\pgfqpoint{1.013884in}{0.810705in}}%
\pgfpathlineto{\pgfqpoint{1.015032in}{0.737534in}}%
\pgfpathlineto{\pgfqpoint{1.015415in}{0.737534in}}%
\pgfpathlineto{\pgfqpoint{1.016946in}{0.831611in}}%
\pgfpathlineto{\pgfqpoint{1.017329in}{0.831611in}}%
\pgfpathlineto{\pgfqpoint{1.017329in}{0.737534in}}%
\pgfpathlineto{\pgfqpoint{1.018861in}{0.753213in}}%
\pgfpathlineto{\pgfqpoint{1.019243in}{0.753213in}}%
\pgfpathlineto{\pgfqpoint{1.020775in}{0.800252in}}%
\pgfpathlineto{\pgfqpoint{1.021158in}{0.800252in}}%
\pgfpathlineto{\pgfqpoint{1.022306in}{0.721854in}}%
\pgfpathlineto{\pgfqpoint{1.022689in}{0.768893in}}%
\pgfpathlineto{\pgfqpoint{1.023072in}{0.768893in}}%
\pgfpathlineto{\pgfqpoint{1.023455in}{0.821158in}}%
\pgfpathlineto{\pgfqpoint{1.024220in}{0.758440in}}%
\pgfpathlineto{\pgfqpoint{1.024603in}{0.800252in}}%
\pgfpathlineto{\pgfqpoint{1.024986in}{0.800252in}}%
\pgfpathlineto{\pgfqpoint{1.025369in}{0.727081in}}%
\pgfpathlineto{\pgfqpoint{1.025369in}{0.815931in}}%
\pgfpathlineto{\pgfqpoint{1.026517in}{0.789799in}}%
\pgfpathlineto{\pgfqpoint{1.026900in}{0.789799in}}%
\pgfpathlineto{\pgfqpoint{1.026900in}{0.747987in}}%
\pgfpathlineto{\pgfqpoint{1.027666in}{0.831611in}}%
\pgfpathlineto{\pgfqpoint{1.028431in}{0.810705in}}%
\pgfpathlineto{\pgfqpoint{1.029197in}{0.810705in}}%
\pgfpathlineto{\pgfqpoint{1.029580in}{0.768893in}}%
\pgfpathlineto{\pgfqpoint{1.030728in}{0.784572in}}%
\pgfpathlineto{\pgfqpoint{1.031111in}{0.784572in}}%
\pgfpathlineto{\pgfqpoint{1.031111in}{0.810705in}}%
\pgfpathlineto{\pgfqpoint{1.032643in}{0.737534in}}%
\pgfpathlineto{\pgfqpoint{1.033025in}{0.737534in}}%
\pgfpathlineto{\pgfqpoint{1.033025in}{0.815931in}}%
\pgfpathlineto{\pgfqpoint{1.034557in}{0.747987in}}%
\pgfpathlineto{\pgfqpoint{1.034939in}{0.747987in}}%
\pgfpathlineto{\pgfqpoint{1.036471in}{0.836837in}}%
\pgfpathlineto{\pgfqpoint{1.036854in}{0.836837in}}%
\pgfpathlineto{\pgfqpoint{1.036854in}{0.862969in}}%
\pgfpathlineto{\pgfqpoint{1.038385in}{0.758440in}}%
\pgfpathlineto{\pgfqpoint{1.038768in}{0.758440in}}%
\pgfpathlineto{\pgfqpoint{1.039916in}{0.810705in}}%
\pgfpathlineto{\pgfqpoint{1.040299in}{0.758440in}}%
\pgfpathlineto{\pgfqpoint{1.040682in}{0.758440in}}%
\pgfpathlineto{\pgfqpoint{1.041448in}{0.810705in}}%
\pgfpathlineto{\pgfqpoint{1.042213in}{0.779346in}}%
\pgfpathlineto{\pgfqpoint{1.042596in}{0.779346in}}%
\pgfpathlineto{\pgfqpoint{1.042596in}{0.805478in}}%
\pgfpathlineto{\pgfqpoint{1.043745in}{0.753213in}}%
\pgfpathlineto{\pgfqpoint{1.044127in}{0.784572in}}%
\pgfpathlineto{\pgfqpoint{1.044893in}{0.784572in}}%
\pgfpathlineto{\pgfqpoint{1.046042in}{0.852516in}}%
\pgfpathlineto{\pgfqpoint{1.046424in}{0.789799in}}%
\pgfpathlineto{\pgfqpoint{1.046807in}{0.789799in}}%
\pgfpathlineto{\pgfqpoint{1.046807in}{0.821158in}}%
\pgfpathlineto{\pgfqpoint{1.047573in}{0.763666in}}%
\pgfpathlineto{\pgfqpoint{1.048339in}{0.821158in}}%
\pgfpathlineto{\pgfqpoint{1.048721in}{0.821158in}}%
\pgfpathlineto{\pgfqpoint{1.048721in}{0.737534in}}%
\pgfpathlineto{\pgfqpoint{1.050253in}{0.805478in}}%
\pgfpathlineto{\pgfqpoint{1.050636in}{0.805478in}}%
\pgfpathlineto{\pgfqpoint{1.050636in}{0.763666in}}%
\pgfpathlineto{\pgfqpoint{1.051784in}{0.862969in}}%
\pgfpathlineto{\pgfqpoint{1.052167in}{0.810705in}}%
\pgfpathlineto{\pgfqpoint{1.052550in}{0.810705in}}%
\pgfpathlineto{\pgfqpoint{1.052933in}{0.737534in}}%
\pgfpathlineto{\pgfqpoint{1.052933in}{0.815931in}}%
\pgfpathlineto{\pgfqpoint{1.054081in}{0.800252in}}%
\pgfpathlineto{\pgfqpoint{1.054464in}{0.800252in}}%
\pgfpathlineto{\pgfqpoint{1.054464in}{0.758440in}}%
\pgfpathlineto{\pgfqpoint{1.055229in}{0.852516in}}%
\pgfpathlineto{\pgfqpoint{1.055995in}{0.789799in}}%
\pgfpathlineto{\pgfqpoint{1.056378in}{0.789799in}}%
\pgfpathlineto{\pgfqpoint{1.057526in}{0.815931in}}%
\pgfpathlineto{\pgfqpoint{1.057909in}{0.789799in}}%
\pgfpathlineto{\pgfqpoint{1.058292in}{0.789799in}}%
\pgfpathlineto{\pgfqpoint{1.059058in}{0.821158in}}%
\pgfpathlineto{\pgfqpoint{1.059823in}{0.789799in}}%
\pgfpathlineto{\pgfqpoint{1.060206in}{0.789799in}}%
\pgfpathlineto{\pgfqpoint{1.061355in}{0.737534in}}%
\pgfpathlineto{\pgfqpoint{1.060589in}{0.831611in}}%
\pgfpathlineto{\pgfqpoint{1.061738in}{0.747987in}}%
\pgfpathlineto{\pgfqpoint{1.062120in}{0.747987in}}%
\pgfpathlineto{\pgfqpoint{1.062503in}{0.711401in}}%
\pgfpathlineto{\pgfqpoint{1.063652in}{0.831611in}}%
\pgfpathlineto{\pgfqpoint{1.064035in}{0.831611in}}%
\pgfpathlineto{\pgfqpoint{1.064417in}{0.763666in}}%
\pgfpathlineto{\pgfqpoint{1.065566in}{0.768893in}}%
\pgfpathlineto{\pgfqpoint{1.065949in}{0.768893in}}%
\pgfpathlineto{\pgfqpoint{1.065949in}{0.810705in}}%
\pgfpathlineto{\pgfqpoint{1.066332in}{0.753213in}}%
\pgfpathlineto{\pgfqpoint{1.067480in}{0.800252in}}%
\pgfpathlineto{\pgfqpoint{1.067863in}{0.800252in}}%
\pgfpathlineto{\pgfqpoint{1.069011in}{0.847290in}}%
\pgfpathlineto{\pgfqpoint{1.068629in}{0.732307in}}%
\pgfpathlineto{\pgfqpoint{1.069394in}{0.774119in}}%
\pgfpathlineto{\pgfqpoint{1.069777in}{0.774119in}}%
\pgfpathlineto{\pgfqpoint{1.070543in}{0.815931in}}%
\pgfpathlineto{\pgfqpoint{1.070926in}{0.758440in}}%
\pgfpathlineto{\pgfqpoint{1.071308in}{0.774119in}}%
\pgfpathlineto{\pgfqpoint{1.071691in}{0.774119in}}%
\pgfpathlineto{\pgfqpoint{1.072457in}{0.847290in}}%
\pgfpathlineto{\pgfqpoint{1.072074in}{0.747987in}}%
\pgfpathlineto{\pgfqpoint{1.073222in}{0.836837in}}%
\pgfpathlineto{\pgfqpoint{1.073605in}{0.836837in}}%
\pgfpathlineto{\pgfqpoint{1.075137in}{0.742760in}}%
\pgfpathlineto{\pgfqpoint{1.075519in}{0.742760in}}%
\pgfpathlineto{\pgfqpoint{1.076285in}{0.862969in}}%
\pgfpathlineto{\pgfqpoint{1.077051in}{0.768893in}}%
\pgfpathlineto{\pgfqpoint{1.077434in}{0.768893in}}%
\pgfpathlineto{\pgfqpoint{1.077434in}{0.852516in}}%
\pgfpathlineto{\pgfqpoint{1.078965in}{0.831611in}}%
\pgfpathlineto{\pgfqpoint{1.079348in}{0.831611in}}%
\pgfpathlineto{\pgfqpoint{1.080496in}{0.742760in}}%
\pgfpathlineto{\pgfqpoint{1.080879in}{0.842063in}}%
\pgfpathlineto{\pgfqpoint{1.081262in}{0.842063in}}%
\pgfpathlineto{\pgfqpoint{1.082793in}{0.758440in}}%
\pgfpathlineto{\pgfqpoint{1.083176in}{0.758440in}}%
\pgfpathlineto{\pgfqpoint{1.084325in}{0.831611in}}%
\pgfpathlineto{\pgfqpoint{1.084707in}{0.789799in}}%
\pgfpathlineto{\pgfqpoint{1.085090in}{0.789799in}}%
\pgfpathlineto{\pgfqpoint{1.085090in}{0.763666in}}%
\pgfpathlineto{\pgfqpoint{1.086622in}{0.847290in}}%
\pgfpathlineto{\pgfqpoint{1.087004in}{0.847290in}}%
\pgfpathlineto{\pgfqpoint{1.087387in}{0.789799in}}%
\pgfpathlineto{\pgfqpoint{1.088536in}{0.810705in}}%
\pgfpathlineto{\pgfqpoint{1.088919in}{0.810705in}}%
\pgfpathlineto{\pgfqpoint{1.088919in}{0.753213in}}%
\pgfpathlineto{\pgfqpoint{1.089301in}{0.821158in}}%
\pgfpathlineto{\pgfqpoint{1.090450in}{0.763666in}}%
\pgfpathlineto{\pgfqpoint{1.090833in}{0.763666in}}%
\pgfpathlineto{\pgfqpoint{1.091598in}{0.826384in}}%
\pgfpathlineto{\pgfqpoint{1.092364in}{0.821158in}}%
\pgfpathlineto{\pgfqpoint{1.092747in}{0.821158in}}%
\pgfpathlineto{\pgfqpoint{1.093130in}{0.774119in}}%
\pgfpathlineto{\pgfqpoint{1.094278in}{0.852516in}}%
\pgfpathlineto{\pgfqpoint{1.094661in}{0.852516in}}%
\pgfpathlineto{\pgfqpoint{1.095809in}{0.763666in}}%
\pgfpathlineto{\pgfqpoint{1.096192in}{0.842063in}}%
\pgfpathlineto{\pgfqpoint{1.096575in}{0.842063in}}%
\pgfpathlineto{\pgfqpoint{1.096958in}{0.753213in}}%
\pgfpathlineto{\pgfqpoint{1.098106in}{0.836837in}}%
\pgfpathlineto{\pgfqpoint{1.098489in}{0.836837in}}%
\pgfpathlineto{\pgfqpoint{1.100021in}{0.758440in}}%
\pgfpathlineto{\pgfqpoint{1.100403in}{0.758440in}}%
\pgfpathlineto{\pgfqpoint{1.100403in}{0.810705in}}%
\pgfpathlineto{\pgfqpoint{1.101935in}{0.768893in}}%
\pgfpathlineto{\pgfqpoint{1.102318in}{0.768893in}}%
\pgfpathlineto{\pgfqpoint{1.103466in}{0.831611in}}%
\pgfpathlineto{\pgfqpoint{1.103849in}{0.763666in}}%
\pgfpathlineto{\pgfqpoint{1.104232in}{0.763666in}}%
\pgfpathlineto{\pgfqpoint{1.104232in}{0.821158in}}%
\pgfpathlineto{\pgfqpoint{1.105763in}{0.774119in}}%
\pgfpathlineto{\pgfqpoint{1.106146in}{0.774119in}}%
\pgfpathlineto{\pgfqpoint{1.106912in}{0.873422in}}%
\pgfpathlineto{\pgfqpoint{1.107677in}{0.784572in}}%
\pgfpathlineto{\pgfqpoint{1.108060in}{0.784572in}}%
\pgfpathlineto{\pgfqpoint{1.108443in}{0.889102in}}%
\pgfpathlineto{\pgfqpoint{1.109209in}{0.763666in}}%
\pgfpathlineto{\pgfqpoint{1.109591in}{0.815931in}}%
\pgfpathlineto{\pgfqpoint{1.109974in}{0.815931in}}%
\pgfpathlineto{\pgfqpoint{1.110740in}{0.768893in}}%
\pgfpathlineto{\pgfqpoint{1.111123in}{0.852516in}}%
\pgfpathlineto{\pgfqpoint{1.111505in}{0.768893in}}%
\pgfpathlineto{\pgfqpoint{1.111888in}{0.768893in}}%
\pgfpathlineto{\pgfqpoint{1.112271in}{0.758440in}}%
\pgfpathlineto{\pgfqpoint{1.113420in}{0.847290in}}%
\pgfpathlineto{\pgfqpoint{1.113802in}{0.847290in}}%
\pgfpathlineto{\pgfqpoint{1.114185in}{0.768893in}}%
\pgfpathlineto{\pgfqpoint{1.115334in}{0.842063in}}%
\pgfpathlineto{\pgfqpoint{1.115717in}{0.842063in}}%
\pgfpathlineto{\pgfqpoint{1.115717in}{0.784572in}}%
\pgfpathlineto{\pgfqpoint{1.116099in}{0.862969in}}%
\pgfpathlineto{\pgfqpoint{1.117248in}{0.784572in}}%
\pgfpathlineto{\pgfqpoint{1.117631in}{0.784572in}}%
\pgfpathlineto{\pgfqpoint{1.118779in}{0.821158in}}%
\pgfpathlineto{\pgfqpoint{1.118396in}{0.742760in}}%
\pgfpathlineto{\pgfqpoint{1.119162in}{0.789799in}}%
\pgfpathlineto{\pgfqpoint{1.119928in}{0.789799in}}%
\pgfpathlineto{\pgfqpoint{1.119928in}{0.821158in}}%
\pgfpathlineto{\pgfqpoint{1.121076in}{0.763666in}}%
\pgfpathlineto{\pgfqpoint{1.121459in}{0.795025in}}%
\pgfpathlineto{\pgfqpoint{1.121842in}{0.795025in}}%
\pgfpathlineto{\pgfqpoint{1.122990in}{0.831611in}}%
\pgfpathlineto{\pgfqpoint{1.123373in}{0.763666in}}%
\pgfpathlineto{\pgfqpoint{1.123756in}{0.763666in}}%
\pgfpathlineto{\pgfqpoint{1.124522in}{0.836837in}}%
\pgfpathlineto{\pgfqpoint{1.125287in}{0.763666in}}%
\pgfpathlineto{\pgfqpoint{1.125670in}{0.763666in}}%
\pgfpathlineto{\pgfqpoint{1.125670in}{0.868196in}}%
\pgfpathlineto{\pgfqpoint{1.126436in}{0.737534in}}%
\pgfpathlineto{\pgfqpoint{1.127202in}{0.842063in}}%
\pgfpathlineto{\pgfqpoint{1.127584in}{0.842063in}}%
\pgfpathlineto{\pgfqpoint{1.128350in}{0.779346in}}%
\pgfpathlineto{\pgfqpoint{1.129116in}{0.842063in}}%
\pgfpathlineto{\pgfqpoint{1.129499in}{0.842063in}}%
\pgfpathlineto{\pgfqpoint{1.129499in}{0.784572in}}%
\pgfpathlineto{\pgfqpoint{1.131030in}{0.826384in}}%
\pgfpathlineto{\pgfqpoint{1.131413in}{0.826384in}}%
\pgfpathlineto{\pgfqpoint{1.132178in}{0.795025in}}%
\pgfpathlineto{\pgfqpoint{1.132944in}{0.795025in}}%
\pgfpathlineto{\pgfqpoint{1.133327in}{0.795025in}}%
\pgfpathlineto{\pgfqpoint{1.133327in}{0.763666in}}%
\pgfpathlineto{\pgfqpoint{1.133710in}{0.826384in}}%
\pgfpathlineto{\pgfqpoint{1.134858in}{0.800252in}}%
\pgfpathlineto{\pgfqpoint{1.135241in}{0.800252in}}%
\pgfpathlineto{\pgfqpoint{1.135241in}{0.842063in}}%
\pgfpathlineto{\pgfqpoint{1.136389in}{0.784572in}}%
\pgfpathlineto{\pgfqpoint{1.136772in}{0.826384in}}%
\pgfpathlineto{\pgfqpoint{1.137155in}{0.826384in}}%
\pgfpathlineto{\pgfqpoint{1.137155in}{0.758440in}}%
\pgfpathlineto{\pgfqpoint{1.137538in}{0.842063in}}%
\pgfpathlineto{\pgfqpoint{1.138686in}{0.800252in}}%
\pgfpathlineto{\pgfqpoint{1.139069in}{0.800252in}}%
\pgfpathlineto{\pgfqpoint{1.139069in}{0.753213in}}%
\pgfpathlineto{\pgfqpoint{1.139835in}{0.831611in}}%
\pgfpathlineto{\pgfqpoint{1.140601in}{0.795025in}}%
\pgfpathlineto{\pgfqpoint{1.140983in}{0.795025in}}%
\pgfpathlineto{\pgfqpoint{1.140983in}{0.758440in}}%
\pgfpathlineto{\pgfqpoint{1.141749in}{0.889102in}}%
\pgfpathlineto{\pgfqpoint{1.142515in}{0.826384in}}%
\pgfpathlineto{\pgfqpoint{1.142898in}{0.826384in}}%
\pgfpathlineto{\pgfqpoint{1.143663in}{0.779346in}}%
\pgfpathlineto{\pgfqpoint{1.144429in}{0.800252in}}%
\pgfpathlineto{\pgfqpoint{1.144812in}{0.800252in}}%
\pgfpathlineto{\pgfqpoint{1.144812in}{0.758440in}}%
\pgfpathlineto{\pgfqpoint{1.145577in}{0.842063in}}%
\pgfpathlineto{\pgfqpoint{1.146343in}{0.821158in}}%
\pgfpathlineto{\pgfqpoint{1.146726in}{0.821158in}}%
\pgfpathlineto{\pgfqpoint{1.147492in}{0.878649in}}%
\pgfpathlineto{\pgfqpoint{1.148257in}{0.737534in}}%
\pgfpathlineto{\pgfqpoint{1.148640in}{0.737534in}}%
\pgfpathlineto{\pgfqpoint{1.149023in}{0.831611in}}%
\pgfpathlineto{\pgfqpoint{1.150171in}{0.795025in}}%
\pgfpathlineto{\pgfqpoint{1.150554in}{0.795025in}}%
\pgfpathlineto{\pgfqpoint{1.150554in}{0.857743in}}%
\pgfpathlineto{\pgfqpoint{1.152085in}{0.815931in}}%
\pgfpathlineto{\pgfqpoint{1.152468in}{0.815931in}}%
\pgfpathlineto{\pgfqpoint{1.152468in}{0.831611in}}%
\pgfpathlineto{\pgfqpoint{1.153617in}{0.753213in}}%
\pgfpathlineto{\pgfqpoint{1.154000in}{0.810705in}}%
\pgfpathlineto{\pgfqpoint{1.154382in}{0.810705in}}%
\pgfpathlineto{\pgfqpoint{1.154382in}{0.821158in}}%
\pgfpathlineto{\pgfqpoint{1.155531in}{0.753213in}}%
\pgfpathlineto{\pgfqpoint{1.155914in}{0.753213in}}%
\pgfpathlineto{\pgfqpoint{1.156297in}{0.753213in}}%
\pgfpathlineto{\pgfqpoint{1.157445in}{0.836837in}}%
\pgfpathlineto{\pgfqpoint{1.157828in}{0.747987in}}%
\pgfpathlineto{\pgfqpoint{1.158211in}{0.747987in}}%
\pgfpathlineto{\pgfqpoint{1.159359in}{0.852516in}}%
\pgfpathlineto{\pgfqpoint{1.159742in}{0.831611in}}%
\pgfpathlineto{\pgfqpoint{1.160125in}{0.831611in}}%
\pgfpathlineto{\pgfqpoint{1.160125in}{0.857743in}}%
\pgfpathlineto{\pgfqpoint{1.160508in}{0.789799in}}%
\pgfpathlineto{\pgfqpoint{1.161656in}{0.795025in}}%
\pgfpathlineto{\pgfqpoint{1.162039in}{0.795025in}}%
\pgfpathlineto{\pgfqpoint{1.162039in}{0.920461in}}%
\pgfpathlineto{\pgfqpoint{1.163188in}{0.789799in}}%
\pgfpathlineto{\pgfqpoint{1.163570in}{0.805478in}}%
\pgfpathlineto{\pgfqpoint{1.163953in}{0.805478in}}%
\pgfpathlineto{\pgfqpoint{1.163953in}{0.779346in}}%
\pgfpathlineto{\pgfqpoint{1.164336in}{0.826384in}}%
\pgfpathlineto{\pgfqpoint{1.165485in}{0.784572in}}%
\pgfpathlineto{\pgfqpoint{1.165867in}{0.784572in}}%
\pgfpathlineto{\pgfqpoint{1.167399in}{0.857743in}}%
\pgfpathlineto{\pgfqpoint{1.167782in}{0.857743in}}%
\pgfpathlineto{\pgfqpoint{1.168164in}{0.789799in}}%
\pgfpathlineto{\pgfqpoint{1.169313in}{0.810705in}}%
\pgfpathlineto{\pgfqpoint{1.169696in}{0.810705in}}%
\pgfpathlineto{\pgfqpoint{1.170844in}{0.862969in}}%
\pgfpathlineto{\pgfqpoint{1.170461in}{0.758440in}}%
\pgfpathlineto{\pgfqpoint{1.171227in}{0.826384in}}%
\pgfpathlineto{\pgfqpoint{1.171610in}{0.826384in}}%
\pgfpathlineto{\pgfqpoint{1.172375in}{0.789799in}}%
\pgfpathlineto{\pgfqpoint{1.172758in}{0.836837in}}%
\pgfpathlineto{\pgfqpoint{1.173141in}{0.815931in}}%
\pgfpathlineto{\pgfqpoint{1.173524in}{0.815931in}}%
\pgfpathlineto{\pgfqpoint{1.174290in}{0.779346in}}%
\pgfpathlineto{\pgfqpoint{1.175055in}{0.857743in}}%
\pgfpathlineto{\pgfqpoint{1.175438in}{0.857743in}}%
\pgfpathlineto{\pgfqpoint{1.176204in}{0.768893in}}%
\pgfpathlineto{\pgfqpoint{1.176969in}{0.805478in}}%
\pgfpathlineto{\pgfqpoint{1.177352in}{0.805478in}}%
\pgfpathlineto{\pgfqpoint{1.177352in}{0.763666in}}%
\pgfpathlineto{\pgfqpoint{1.178501in}{0.836837in}}%
\pgfpathlineto{\pgfqpoint{1.178884in}{0.810705in}}%
\pgfpathlineto{\pgfqpoint{1.179266in}{0.810705in}}%
\pgfpathlineto{\pgfqpoint{1.179266in}{0.795025in}}%
\pgfpathlineto{\pgfqpoint{1.180798in}{0.883875in}}%
\pgfpathlineto{\pgfqpoint{1.181181in}{0.883875in}}%
\pgfpathlineto{\pgfqpoint{1.181181in}{0.753213in}}%
\pgfpathlineto{\pgfqpoint{1.181563in}{0.899555in}}%
\pgfpathlineto{\pgfqpoint{1.182712in}{0.836837in}}%
\pgfpathlineto{\pgfqpoint{1.183095in}{0.836837in}}%
\pgfpathlineto{\pgfqpoint{1.183095in}{0.779346in}}%
\pgfpathlineto{\pgfqpoint{1.183860in}{0.862969in}}%
\pgfpathlineto{\pgfqpoint{1.184626in}{0.862969in}}%
\pgfpathlineto{\pgfqpoint{1.185009in}{0.862969in}}%
\pgfpathlineto{\pgfqpoint{1.185775in}{0.899555in}}%
\pgfpathlineto{\pgfqpoint{1.186540in}{0.795025in}}%
\pgfpathlineto{\pgfqpoint{1.186923in}{0.795025in}}%
\pgfpathlineto{\pgfqpoint{1.186923in}{0.862969in}}%
\pgfpathlineto{\pgfqpoint{1.188454in}{0.795025in}}%
\pgfpathlineto{\pgfqpoint{1.188837in}{0.795025in}}%
\pgfpathlineto{\pgfqpoint{1.188837in}{0.779346in}}%
\pgfpathlineto{\pgfqpoint{1.189220in}{0.847290in}}%
\pgfpathlineto{\pgfqpoint{1.190368in}{0.847290in}}%
\pgfpathlineto{\pgfqpoint{1.190751in}{0.847290in}}%
\pgfpathlineto{\pgfqpoint{1.190751in}{0.789799in}}%
\pgfpathlineto{\pgfqpoint{1.191517in}{0.852516in}}%
\pgfpathlineto{\pgfqpoint{1.192283in}{0.815931in}}%
\pgfpathlineto{\pgfqpoint{1.192665in}{0.815931in}}%
\pgfpathlineto{\pgfqpoint{1.192665in}{0.800252in}}%
\pgfpathlineto{\pgfqpoint{1.193431in}{0.831611in}}%
\pgfpathlineto{\pgfqpoint{1.194197in}{0.826384in}}%
\pgfpathlineto{\pgfqpoint{1.194580in}{0.826384in}}%
\pgfpathlineto{\pgfqpoint{1.195728in}{0.857743in}}%
\pgfpathlineto{\pgfqpoint{1.195345in}{0.758440in}}%
\pgfpathlineto{\pgfqpoint{1.196111in}{0.779346in}}%
\pgfpathlineto{\pgfqpoint{1.196494in}{0.779346in}}%
\pgfpathlineto{\pgfqpoint{1.198025in}{0.862969in}}%
\pgfpathlineto{\pgfqpoint{1.198408in}{0.862969in}}%
\pgfpathlineto{\pgfqpoint{1.198408in}{0.779346in}}%
\pgfpathlineto{\pgfqpoint{1.199939in}{0.805478in}}%
\pgfpathlineto{\pgfqpoint{1.200322in}{0.805478in}}%
\pgfpathlineto{\pgfqpoint{1.200322in}{0.821158in}}%
\pgfpathlineto{\pgfqpoint{1.201853in}{0.774119in}}%
\pgfpathlineto{\pgfqpoint{1.202236in}{0.774119in}}%
\pgfpathlineto{\pgfqpoint{1.203002in}{0.873422in}}%
\pgfpathlineto{\pgfqpoint{1.203768in}{0.873422in}}%
\pgfpathlineto{\pgfqpoint{1.204150in}{0.873422in}}%
\pgfpathlineto{\pgfqpoint{1.205299in}{0.784572in}}%
\pgfpathlineto{\pgfqpoint{1.205682in}{0.831611in}}%
\pgfpathlineto{\pgfqpoint{1.206065in}{0.831611in}}%
\pgfpathlineto{\pgfqpoint{1.206065in}{0.852516in}}%
\pgfpathlineto{\pgfqpoint{1.206830in}{0.784572in}}%
\pgfpathlineto{\pgfqpoint{1.207596in}{0.821158in}}%
\pgfpathlineto{\pgfqpoint{1.207979in}{0.821158in}}%
\pgfpathlineto{\pgfqpoint{1.208361in}{0.857743in}}%
\pgfpathlineto{\pgfqpoint{1.209510in}{0.784572in}}%
\pgfpathlineto{\pgfqpoint{1.210276in}{0.784572in}}%
\pgfpathlineto{\pgfqpoint{1.210658in}{0.873422in}}%
\pgfpathlineto{\pgfqpoint{1.211041in}{0.763666in}}%
\pgfpathlineto{\pgfqpoint{1.211807in}{0.852516in}}%
\pgfpathlineto{\pgfqpoint{1.212190in}{0.852516in}}%
\pgfpathlineto{\pgfqpoint{1.213338in}{0.774119in}}%
\pgfpathlineto{\pgfqpoint{1.212955in}{0.862969in}}%
\pgfpathlineto{\pgfqpoint{1.213721in}{0.831611in}}%
\pgfpathlineto{\pgfqpoint{1.214104in}{0.831611in}}%
\pgfpathlineto{\pgfqpoint{1.214104in}{0.774119in}}%
\pgfpathlineto{\pgfqpoint{1.215252in}{0.847290in}}%
\pgfpathlineto{\pgfqpoint{1.215635in}{0.784572in}}%
\pgfpathlineto{\pgfqpoint{1.216018in}{0.784572in}}%
\pgfpathlineto{\pgfqpoint{1.217167in}{0.774119in}}%
\pgfpathlineto{\pgfqpoint{1.217549in}{0.847290in}}%
\pgfpathlineto{\pgfqpoint{1.217932in}{0.847290in}}%
\pgfpathlineto{\pgfqpoint{1.219464in}{0.768893in}}%
\pgfpathlineto{\pgfqpoint{1.219846in}{0.768893in}}%
\pgfpathlineto{\pgfqpoint{1.221378in}{0.852516in}}%
\pgfpathlineto{\pgfqpoint{1.221761in}{0.852516in}}%
\pgfpathlineto{\pgfqpoint{1.222526in}{0.768893in}}%
\pgfpathlineto{\pgfqpoint{1.223292in}{0.810705in}}%
\pgfpathlineto{\pgfqpoint{1.223675in}{0.810705in}}%
\pgfpathlineto{\pgfqpoint{1.224823in}{0.795025in}}%
\pgfpathlineto{\pgfqpoint{1.224058in}{0.894328in}}%
\pgfpathlineto{\pgfqpoint{1.225206in}{0.815931in}}%
\pgfpathlineto{\pgfqpoint{1.225589in}{0.815931in}}%
\pgfpathlineto{\pgfqpoint{1.225589in}{0.800252in}}%
\pgfpathlineto{\pgfqpoint{1.226737in}{0.852516in}}%
\pgfpathlineto{\pgfqpoint{1.227120in}{0.831611in}}%
\pgfpathlineto{\pgfqpoint{1.227503in}{0.831611in}}%
\pgfpathlineto{\pgfqpoint{1.227503in}{0.800252in}}%
\pgfpathlineto{\pgfqpoint{1.228651in}{0.842063in}}%
\pgfpathlineto{\pgfqpoint{1.229034in}{0.821158in}}%
\pgfpathlineto{\pgfqpoint{1.229417in}{0.821158in}}%
\pgfpathlineto{\pgfqpoint{1.229800in}{0.758440in}}%
\pgfpathlineto{\pgfqpoint{1.230948in}{0.878649in}}%
\pgfpathlineto{\pgfqpoint{1.231331in}{0.878649in}}%
\pgfpathlineto{\pgfqpoint{1.232480in}{0.758440in}}%
\pgfpathlineto{\pgfqpoint{1.232863in}{0.894328in}}%
\pgfpathlineto{\pgfqpoint{1.233245in}{0.894328in}}%
\pgfpathlineto{\pgfqpoint{1.233245in}{0.795025in}}%
\pgfpathlineto{\pgfqpoint{1.234777in}{0.805478in}}%
\pgfpathlineto{\pgfqpoint{1.235160in}{0.805478in}}%
\pgfpathlineto{\pgfqpoint{1.235160in}{0.779346in}}%
\pgfpathlineto{\pgfqpoint{1.235542in}{0.889102in}}%
\pgfpathlineto{\pgfqpoint{1.236691in}{0.795025in}}%
\pgfpathlineto{\pgfqpoint{1.237074in}{0.795025in}}%
\pgfpathlineto{\pgfqpoint{1.237074in}{0.821158in}}%
\pgfpathlineto{\pgfqpoint{1.238605in}{0.779346in}}%
\pgfpathlineto{\pgfqpoint{1.238988in}{0.779346in}}%
\pgfpathlineto{\pgfqpoint{1.238988in}{0.826384in}}%
\pgfpathlineto{\pgfqpoint{1.239371in}{0.763666in}}%
\pgfpathlineto{\pgfqpoint{1.240519in}{0.821158in}}%
\pgfpathlineto{\pgfqpoint{1.240902in}{0.821158in}}%
\pgfpathlineto{\pgfqpoint{1.240902in}{0.758440in}}%
\pgfpathlineto{\pgfqpoint{1.241285in}{0.831611in}}%
\pgfpathlineto{\pgfqpoint{1.242433in}{0.758440in}}%
\pgfpathlineto{\pgfqpoint{1.242816in}{0.758440in}}%
\pgfpathlineto{\pgfqpoint{1.243582in}{0.910008in}}%
\pgfpathlineto{\pgfqpoint{1.244348in}{0.847290in}}%
\pgfpathlineto{\pgfqpoint{1.244730in}{0.847290in}}%
\pgfpathlineto{\pgfqpoint{1.244730in}{0.852516in}}%
\pgfpathlineto{\pgfqpoint{1.245496in}{0.779346in}}%
\pgfpathlineto{\pgfqpoint{1.246262in}{0.847290in}}%
\pgfpathlineto{\pgfqpoint{1.246644in}{0.847290in}}%
\pgfpathlineto{\pgfqpoint{1.246644in}{0.800252in}}%
\pgfpathlineto{\pgfqpoint{1.247027in}{0.883875in}}%
\pgfpathlineto{\pgfqpoint{1.248176in}{0.800252in}}%
\pgfpathlineto{\pgfqpoint{1.248559in}{0.800252in}}%
\pgfpathlineto{\pgfqpoint{1.248559in}{0.753213in}}%
\pgfpathlineto{\pgfqpoint{1.248941in}{0.821158in}}%
\pgfpathlineto{\pgfqpoint{1.250090in}{0.821158in}}%
\pgfpathlineto{\pgfqpoint{1.250473in}{0.821158in}}%
\pgfpathlineto{\pgfqpoint{1.250473in}{0.763666in}}%
\pgfpathlineto{\pgfqpoint{1.252004in}{0.842063in}}%
\pgfpathlineto{\pgfqpoint{1.252387in}{0.842063in}}%
\pgfpathlineto{\pgfqpoint{1.253918in}{0.768893in}}%
\pgfpathlineto{\pgfqpoint{1.254301in}{0.768893in}}%
\pgfpathlineto{\pgfqpoint{1.254301in}{0.763666in}}%
\pgfpathlineto{\pgfqpoint{1.255832in}{0.842063in}}%
\pgfpathlineto{\pgfqpoint{1.256215in}{0.842063in}}%
\pgfpathlineto{\pgfqpoint{1.256981in}{0.789799in}}%
\pgfpathlineto{\pgfqpoint{1.257747in}{0.842063in}}%
\pgfpathlineto{\pgfqpoint{1.258129in}{0.842063in}}%
\pgfpathlineto{\pgfqpoint{1.259278in}{0.800252in}}%
\pgfpathlineto{\pgfqpoint{1.259661in}{0.862969in}}%
\pgfpathlineto{\pgfqpoint{1.260044in}{0.862969in}}%
\pgfpathlineto{\pgfqpoint{1.260809in}{0.805478in}}%
\pgfpathlineto{\pgfqpoint{1.261575in}{0.831611in}}%
\pgfpathlineto{\pgfqpoint{1.262341in}{0.831611in}}%
\pgfpathlineto{\pgfqpoint{1.262723in}{0.774119in}}%
\pgfpathlineto{\pgfqpoint{1.263872in}{0.847290in}}%
\pgfpathlineto{\pgfqpoint{1.264255in}{0.847290in}}%
\pgfpathlineto{\pgfqpoint{1.264638in}{0.758440in}}%
\pgfpathlineto{\pgfqpoint{1.264638in}{0.873422in}}%
\pgfpathlineto{\pgfqpoint{1.265786in}{0.810705in}}%
\pgfpathlineto{\pgfqpoint{1.266169in}{0.810705in}}%
\pgfpathlineto{\pgfqpoint{1.266169in}{0.826384in}}%
\pgfpathlineto{\pgfqpoint{1.266552in}{0.779346in}}%
\pgfpathlineto{\pgfqpoint{1.267700in}{0.810705in}}%
\pgfpathlineto{\pgfqpoint{1.268083in}{0.810705in}}%
\pgfpathlineto{\pgfqpoint{1.269231in}{0.847290in}}%
\pgfpathlineto{\pgfqpoint{1.268849in}{0.747987in}}%
\pgfpathlineto{\pgfqpoint{1.269614in}{0.768893in}}%
\pgfpathlineto{\pgfqpoint{1.269997in}{0.768893in}}%
\pgfpathlineto{\pgfqpoint{1.271146in}{0.758440in}}%
\pgfpathlineto{\pgfqpoint{1.271528in}{0.847290in}}%
\pgfpathlineto{\pgfqpoint{1.271911in}{0.847290in}}%
\pgfpathlineto{\pgfqpoint{1.272294in}{0.815931in}}%
\pgfpathlineto{\pgfqpoint{1.273060in}{0.868196in}}%
\pgfpathlineto{\pgfqpoint{1.273443in}{0.847290in}}%
\pgfpathlineto{\pgfqpoint{1.273825in}{0.847290in}}%
\pgfpathlineto{\pgfqpoint{1.274974in}{0.889102in}}%
\pgfpathlineto{\pgfqpoint{1.275357in}{0.805478in}}%
\pgfpathlineto{\pgfqpoint{1.275740in}{0.805478in}}%
\pgfpathlineto{\pgfqpoint{1.275740in}{0.747987in}}%
\pgfpathlineto{\pgfqpoint{1.276122in}{0.821158in}}%
\pgfpathlineto{\pgfqpoint{1.277271in}{0.815931in}}%
\pgfpathlineto{\pgfqpoint{1.278037in}{0.815931in}}%
\pgfpathlineto{\pgfqpoint{1.278802in}{0.774119in}}%
\pgfpathlineto{\pgfqpoint{1.279568in}{0.815931in}}%
\pgfpathlineto{\pgfqpoint{1.279951in}{0.815931in}}%
\pgfpathlineto{\pgfqpoint{1.279951in}{0.753213in}}%
\pgfpathlineto{\pgfqpoint{1.281099in}{0.862969in}}%
\pgfpathlineto{\pgfqpoint{1.281482in}{0.805478in}}%
\pgfpathlineto{\pgfqpoint{1.281865in}{0.805478in}}%
\pgfpathlineto{\pgfqpoint{1.282631in}{0.742760in}}%
\pgfpathlineto{\pgfqpoint{1.283396in}{0.847290in}}%
\pgfpathlineto{\pgfqpoint{1.283779in}{0.847290in}}%
\pgfpathlineto{\pgfqpoint{1.284162in}{0.815931in}}%
\pgfpathlineto{\pgfqpoint{1.284928in}{0.878649in}}%
\pgfpathlineto{\pgfqpoint{1.285310in}{0.826384in}}%
\pgfpathlineto{\pgfqpoint{1.285693in}{0.826384in}}%
\pgfpathlineto{\pgfqpoint{1.286076in}{0.753213in}}%
\pgfpathlineto{\pgfqpoint{1.286459in}{0.847290in}}%
\pgfpathlineto{\pgfqpoint{1.287224in}{0.815931in}}%
\pgfpathlineto{\pgfqpoint{1.287607in}{0.815931in}}%
\pgfpathlineto{\pgfqpoint{1.287607in}{0.805478in}}%
\pgfpathlineto{\pgfqpoint{1.287990in}{0.852516in}}%
\pgfpathlineto{\pgfqpoint{1.289139in}{0.831611in}}%
\pgfpathlineto{\pgfqpoint{1.289521in}{0.831611in}}%
\pgfpathlineto{\pgfqpoint{1.289521in}{0.857743in}}%
\pgfpathlineto{\pgfqpoint{1.289904in}{0.789799in}}%
\pgfpathlineto{\pgfqpoint{1.291053in}{0.815931in}}%
\pgfpathlineto{\pgfqpoint{1.291436in}{0.815931in}}%
\pgfpathlineto{\pgfqpoint{1.291818in}{0.768893in}}%
\pgfpathlineto{\pgfqpoint{1.292967in}{0.800252in}}%
\pgfpathlineto{\pgfqpoint{1.293350in}{0.800252in}}%
\pgfpathlineto{\pgfqpoint{1.294498in}{0.862969in}}%
\pgfpathlineto{\pgfqpoint{1.293733in}{0.789799in}}%
\pgfpathlineto{\pgfqpoint{1.294881in}{0.847290in}}%
\pgfpathlineto{\pgfqpoint{1.295264in}{0.847290in}}%
\pgfpathlineto{\pgfqpoint{1.296412in}{0.732307in}}%
\pgfpathlineto{\pgfqpoint{1.296795in}{0.842063in}}%
\pgfpathlineto{\pgfqpoint{1.297178in}{0.842063in}}%
\pgfpathlineto{\pgfqpoint{1.297561in}{0.789799in}}%
\pgfpathlineto{\pgfqpoint{1.297944in}{0.873422in}}%
\pgfpathlineto{\pgfqpoint{1.298709in}{0.805478in}}%
\pgfpathlineto{\pgfqpoint{1.299092in}{0.805478in}}%
\pgfpathlineto{\pgfqpoint{1.299092in}{0.747987in}}%
\pgfpathlineto{\pgfqpoint{1.299475in}{0.894328in}}%
\pgfpathlineto{\pgfqpoint{1.300624in}{0.842063in}}%
\pgfpathlineto{\pgfqpoint{1.301006in}{0.842063in}}%
\pgfpathlineto{\pgfqpoint{1.301389in}{0.904781in}}%
\pgfpathlineto{\pgfqpoint{1.302538in}{0.810705in}}%
\pgfpathlineto{\pgfqpoint{1.302921in}{0.810705in}}%
\pgfpathlineto{\pgfqpoint{1.302921in}{0.795025in}}%
\pgfpathlineto{\pgfqpoint{1.304452in}{0.868196in}}%
\pgfpathlineto{\pgfqpoint{1.304835in}{0.868196in}}%
\pgfpathlineto{\pgfqpoint{1.305600in}{0.763666in}}%
\pgfpathlineto{\pgfqpoint{1.306366in}{0.821158in}}%
\pgfpathlineto{\pgfqpoint{1.306749in}{0.821158in}}%
\pgfpathlineto{\pgfqpoint{1.307897in}{0.873422in}}%
\pgfpathlineto{\pgfqpoint{1.308280in}{0.747987in}}%
\pgfpathlineto{\pgfqpoint{1.308663in}{0.747987in}}%
\pgfpathlineto{\pgfqpoint{1.309046in}{0.836837in}}%
\pgfpathlineto{\pgfqpoint{1.310194in}{0.805478in}}%
\pgfpathlineto{\pgfqpoint{1.310577in}{0.805478in}}%
\pgfpathlineto{\pgfqpoint{1.310577in}{0.836837in}}%
\pgfpathlineto{\pgfqpoint{1.312108in}{0.805478in}}%
\pgfpathlineto{\pgfqpoint{1.312491in}{0.805478in}}%
\pgfpathlineto{\pgfqpoint{1.313257in}{0.873422in}}%
\pgfpathlineto{\pgfqpoint{1.313640in}{0.753213in}}%
\pgfpathlineto{\pgfqpoint{1.314023in}{0.847290in}}%
\pgfpathlineto{\pgfqpoint{1.314405in}{0.847290in}}%
\pgfpathlineto{\pgfqpoint{1.314405in}{0.779346in}}%
\pgfpathlineto{\pgfqpoint{1.315937in}{0.868196in}}%
\pgfpathlineto{\pgfqpoint{1.316320in}{0.868196in}}%
\pgfpathlineto{\pgfqpoint{1.317851in}{0.753213in}}%
\pgfpathlineto{\pgfqpoint{1.318234in}{0.753213in}}%
\pgfpathlineto{\pgfqpoint{1.318234in}{0.815931in}}%
\pgfpathlineto{\pgfqpoint{1.319765in}{0.789799in}}%
\pgfpathlineto{\pgfqpoint{1.320148in}{0.789799in}}%
\pgfpathlineto{\pgfqpoint{1.321679in}{0.852516in}}%
\pgfpathlineto{\pgfqpoint{1.322062in}{0.852516in}}%
\pgfpathlineto{\pgfqpoint{1.322445in}{0.763666in}}%
\pgfpathlineto{\pgfqpoint{1.323593in}{0.868196in}}%
\pgfpathlineto{\pgfqpoint{1.323976in}{0.868196in}}%
\pgfpathlineto{\pgfqpoint{1.325125in}{0.815931in}}%
\pgfpathlineto{\pgfqpoint{1.325507in}{0.842063in}}%
\pgfpathlineto{\pgfqpoint{1.326273in}{0.842063in}}%
\pgfpathlineto{\pgfqpoint{1.326656in}{0.763666in}}%
\pgfpathlineto{\pgfqpoint{1.327804in}{0.784572in}}%
\pgfpathlineto{\pgfqpoint{1.328187in}{0.784572in}}%
\pgfpathlineto{\pgfqpoint{1.328187in}{0.862969in}}%
\pgfpathlineto{\pgfqpoint{1.328570in}{0.779346in}}%
\pgfpathlineto{\pgfqpoint{1.329719in}{0.821158in}}%
\pgfpathlineto{\pgfqpoint{1.330101in}{0.821158in}}%
\pgfpathlineto{\pgfqpoint{1.330867in}{0.847290in}}%
\pgfpathlineto{\pgfqpoint{1.331633in}{0.753213in}}%
\pgfpathlineto{\pgfqpoint{1.332016in}{0.753213in}}%
\pgfpathlineto{\pgfqpoint{1.332398in}{0.889102in}}%
\pgfpathlineto{\pgfqpoint{1.333547in}{0.789799in}}%
\pgfpathlineto{\pgfqpoint{1.333930in}{0.789799in}}%
\pgfpathlineto{\pgfqpoint{1.334695in}{0.842063in}}%
\pgfpathlineto{\pgfqpoint{1.335461in}{0.779346in}}%
\pgfpathlineto{\pgfqpoint{1.335844in}{0.779346in}}%
\pgfpathlineto{\pgfqpoint{1.336610in}{0.883875in}}%
\pgfpathlineto{\pgfqpoint{1.337375in}{0.847290in}}%
\pgfpathlineto{\pgfqpoint{1.337758in}{0.847290in}}%
\pgfpathlineto{\pgfqpoint{1.338907in}{0.883875in}}%
\pgfpathlineto{\pgfqpoint{1.338141in}{0.795025in}}%
\pgfpathlineto{\pgfqpoint{1.339289in}{0.847290in}}%
\pgfpathlineto{\pgfqpoint{1.339672in}{0.847290in}}%
\pgfpathlineto{\pgfqpoint{1.340055in}{0.784572in}}%
\pgfpathlineto{\pgfqpoint{1.341204in}{0.795025in}}%
\pgfpathlineto{\pgfqpoint{1.341586in}{0.795025in}}%
\pgfpathlineto{\pgfqpoint{1.343118in}{0.847290in}}%
\pgfpathlineto{\pgfqpoint{1.343500in}{0.847290in}}%
\pgfpathlineto{\pgfqpoint{1.344649in}{0.758440in}}%
\pgfpathlineto{\pgfqpoint{1.343883in}{0.852516in}}%
\pgfpathlineto{\pgfqpoint{1.345032in}{0.826384in}}%
\pgfpathlineto{\pgfqpoint{1.345415in}{0.826384in}}%
\pgfpathlineto{\pgfqpoint{1.346180in}{0.810705in}}%
\pgfpathlineto{\pgfqpoint{1.346946in}{0.936140in}}%
\pgfpathlineto{\pgfqpoint{1.347329in}{0.936140in}}%
\pgfpathlineto{\pgfqpoint{1.347329in}{0.789799in}}%
\pgfpathlineto{\pgfqpoint{1.348860in}{0.836837in}}%
\pgfpathlineto{\pgfqpoint{1.349243in}{0.836837in}}%
\pgfpathlineto{\pgfqpoint{1.349243in}{0.815931in}}%
\pgfpathlineto{\pgfqpoint{1.350009in}{0.878649in}}%
\pgfpathlineto{\pgfqpoint{1.350774in}{0.868196in}}%
\pgfpathlineto{\pgfqpoint{1.351157in}{0.868196in}}%
\pgfpathlineto{\pgfqpoint{1.351157in}{0.894328in}}%
\pgfpathlineto{\pgfqpoint{1.352688in}{0.815931in}}%
\pgfpathlineto{\pgfqpoint{1.353071in}{0.815931in}}%
\pgfpathlineto{\pgfqpoint{1.353071in}{0.784572in}}%
\pgfpathlineto{\pgfqpoint{1.354603in}{0.883875in}}%
\pgfpathlineto{\pgfqpoint{1.354985in}{0.883875in}}%
\pgfpathlineto{\pgfqpoint{1.356134in}{0.815931in}}%
\pgfpathlineto{\pgfqpoint{1.355751in}{0.889102in}}%
\pgfpathlineto{\pgfqpoint{1.356517in}{0.857743in}}%
\pgfpathlineto{\pgfqpoint{1.356900in}{0.857743in}}%
\pgfpathlineto{\pgfqpoint{1.356900in}{0.904781in}}%
\pgfpathlineto{\pgfqpoint{1.357282in}{0.784572in}}%
\pgfpathlineto{\pgfqpoint{1.358431in}{0.878649in}}%
\pgfpathlineto{\pgfqpoint{1.358814in}{0.878649in}}%
\pgfpathlineto{\pgfqpoint{1.359197in}{0.810705in}}%
\pgfpathlineto{\pgfqpoint{1.360345in}{0.889102in}}%
\pgfpathlineto{\pgfqpoint{1.361111in}{0.889102in}}%
\pgfpathlineto{\pgfqpoint{1.361111in}{0.915234in}}%
\pgfpathlineto{\pgfqpoint{1.361494in}{0.836837in}}%
\pgfpathlineto{\pgfqpoint{1.362642in}{0.878649in}}%
\pgfpathlineto{\pgfqpoint{1.363025in}{0.878649in}}%
\pgfpathlineto{\pgfqpoint{1.364173in}{0.758440in}}%
\pgfpathlineto{\pgfqpoint{1.364556in}{0.842063in}}%
\pgfpathlineto{\pgfqpoint{1.364939in}{0.842063in}}%
\pgfpathlineto{\pgfqpoint{1.364939in}{0.889102in}}%
\pgfpathlineto{\pgfqpoint{1.365322in}{0.826384in}}%
\pgfpathlineto{\pgfqpoint{1.366470in}{0.826384in}}%
\pgfpathlineto{\pgfqpoint{1.366853in}{0.826384in}}%
\pgfpathlineto{\pgfqpoint{1.366853in}{0.800252in}}%
\pgfpathlineto{\pgfqpoint{1.368002in}{0.883875in}}%
\pgfpathlineto{\pgfqpoint{1.368384in}{0.862969in}}%
\pgfpathlineto{\pgfqpoint{1.368767in}{0.862969in}}%
\pgfpathlineto{\pgfqpoint{1.368767in}{0.868196in}}%
\pgfpathlineto{\pgfqpoint{1.369916in}{0.842063in}}%
\pgfpathlineto{\pgfqpoint{1.370299in}{0.842063in}}%
\pgfpathlineto{\pgfqpoint{1.370681in}{0.842063in}}%
\pgfpathlineto{\pgfqpoint{1.371064in}{0.810705in}}%
\pgfpathlineto{\pgfqpoint{1.372213in}{0.862969in}}%
\pgfpathlineto{\pgfqpoint{1.372596in}{0.862969in}}%
\pgfpathlineto{\pgfqpoint{1.373744in}{0.889102in}}%
\pgfpathlineto{\pgfqpoint{1.374127in}{0.836837in}}%
\pgfpathlineto{\pgfqpoint{1.374510in}{0.836837in}}%
\pgfpathlineto{\pgfqpoint{1.375658in}{0.805478in}}%
\pgfpathlineto{\pgfqpoint{1.376041in}{0.857743in}}%
\pgfpathlineto{\pgfqpoint{1.376424in}{0.857743in}}%
\pgfpathlineto{\pgfqpoint{1.376424in}{0.920461in}}%
\pgfpathlineto{\pgfqpoint{1.376807in}{0.815931in}}%
\pgfpathlineto{\pgfqpoint{1.377955in}{0.847290in}}%
\pgfpathlineto{\pgfqpoint{1.378338in}{0.847290in}}%
\pgfpathlineto{\pgfqpoint{1.378721in}{0.821158in}}%
\pgfpathlineto{\pgfqpoint{1.379869in}{0.868196in}}%
\pgfpathlineto{\pgfqpoint{1.380635in}{0.868196in}}%
\pgfpathlineto{\pgfqpoint{1.380635in}{0.910008in}}%
\pgfpathlineto{\pgfqpoint{1.381018in}{0.847290in}}%
\pgfpathlineto{\pgfqpoint{1.382166in}{0.868196in}}%
\pgfpathlineto{\pgfqpoint{1.382549in}{0.868196in}}%
\pgfpathlineto{\pgfqpoint{1.382549in}{0.920461in}}%
\pgfpathlineto{\pgfqpoint{1.384080in}{0.826384in}}%
\pgfpathlineto{\pgfqpoint{1.384463in}{0.826384in}}%
\pgfpathlineto{\pgfqpoint{1.385612in}{0.925687in}}%
\pgfpathlineto{\pgfqpoint{1.385995in}{0.925687in}}%
\pgfpathlineto{\pgfqpoint{1.386377in}{0.925687in}}%
\pgfpathlineto{\pgfqpoint{1.387526in}{0.842063in}}%
\pgfpathlineto{\pgfqpoint{1.387909in}{0.857743in}}%
\pgfpathlineto{\pgfqpoint{1.388292in}{0.857743in}}%
\pgfpathlineto{\pgfqpoint{1.388292in}{0.810705in}}%
\pgfpathlineto{\pgfqpoint{1.389823in}{0.821158in}}%
\pgfpathlineto{\pgfqpoint{1.390206in}{0.821158in}}%
\pgfpathlineto{\pgfqpoint{1.391354in}{0.967499in}}%
\pgfpathlineto{\pgfqpoint{1.391737in}{0.883875in}}%
\pgfpathlineto{\pgfqpoint{1.392503in}{0.883875in}}%
\pgfpathlineto{\pgfqpoint{1.392503in}{0.936140in}}%
\pgfpathlineto{\pgfqpoint{1.393651in}{0.831611in}}%
\pgfpathlineto{\pgfqpoint{1.394034in}{0.836837in}}%
\pgfpathlineto{\pgfqpoint{1.394417in}{0.836837in}}%
\pgfpathlineto{\pgfqpoint{1.394800in}{0.920461in}}%
\pgfpathlineto{\pgfqpoint{1.395948in}{0.831611in}}%
\pgfpathlineto{\pgfqpoint{1.396331in}{0.831611in}}%
\pgfpathlineto{\pgfqpoint{1.396714in}{0.810705in}}%
\pgfpathlineto{\pgfqpoint{1.397097in}{0.852516in}}%
\pgfpathlineto{\pgfqpoint{1.397862in}{0.852516in}}%
\pgfpathlineto{\pgfqpoint{1.398245in}{0.852516in}}%
\pgfpathlineto{\pgfqpoint{1.398245in}{0.831611in}}%
\pgfpathlineto{\pgfqpoint{1.398628in}{0.904781in}}%
\pgfpathlineto{\pgfqpoint{1.399777in}{0.852516in}}%
\pgfpathlineto{\pgfqpoint{1.400159in}{0.852516in}}%
\pgfpathlineto{\pgfqpoint{1.401308in}{0.836837in}}%
\pgfpathlineto{\pgfqpoint{1.401691in}{0.951820in}}%
\pgfpathlineto{\pgfqpoint{1.402073in}{0.951820in}}%
\pgfpathlineto{\pgfqpoint{1.403222in}{0.883875in}}%
\pgfpathlineto{\pgfqpoint{1.403605in}{0.889102in}}%
\pgfpathlineto{\pgfqpoint{1.403988in}{0.889102in}}%
\pgfpathlineto{\pgfqpoint{1.403988in}{0.920461in}}%
\pgfpathlineto{\pgfqpoint{1.405519in}{0.883875in}}%
\pgfpathlineto{\pgfqpoint{1.405902in}{0.883875in}}%
\pgfpathlineto{\pgfqpoint{1.406667in}{0.831611in}}%
\pgfpathlineto{\pgfqpoint{1.406285in}{0.894328in}}%
\pgfpathlineto{\pgfqpoint{1.407433in}{0.873422in}}%
\pgfpathlineto{\pgfqpoint{1.407816in}{0.873422in}}%
\pgfpathlineto{\pgfqpoint{1.408582in}{0.946593in}}%
\pgfpathlineto{\pgfqpoint{1.409347in}{0.842063in}}%
\pgfpathlineto{\pgfqpoint{1.409730in}{0.842063in}}%
\pgfpathlineto{\pgfqpoint{1.410113in}{0.920461in}}%
\pgfpathlineto{\pgfqpoint{1.411261in}{0.915234in}}%
\pgfpathlineto{\pgfqpoint{1.411644in}{0.915234in}}%
\pgfpathlineto{\pgfqpoint{1.412027in}{0.826384in}}%
\pgfpathlineto{\pgfqpoint{1.412027in}{0.946593in}}%
\pgfpathlineto{\pgfqpoint{1.413176in}{0.930914in}}%
\pgfpathlineto{\pgfqpoint{1.413558in}{0.930914in}}%
\pgfpathlineto{\pgfqpoint{1.413558in}{0.889102in}}%
\pgfpathlineto{\pgfqpoint{1.415090in}{0.988405in}}%
\pgfpathlineto{\pgfqpoint{1.415473in}{0.988405in}}%
\pgfpathlineto{\pgfqpoint{1.415473in}{0.889102in}}%
\pgfpathlineto{\pgfqpoint{1.417004in}{0.925687in}}%
\pgfpathlineto{\pgfqpoint{1.417387in}{0.925687in}}%
\pgfpathlineto{\pgfqpoint{1.417387in}{0.941367in}}%
\pgfpathlineto{\pgfqpoint{1.417770in}{0.899555in}}%
\pgfpathlineto{\pgfqpoint{1.418918in}{0.930914in}}%
\pgfpathlineto{\pgfqpoint{1.419684in}{0.930914in}}%
\pgfpathlineto{\pgfqpoint{1.420067in}{0.878649in}}%
\pgfpathlineto{\pgfqpoint{1.421215in}{0.977952in}}%
\pgfpathlineto{\pgfqpoint{1.421598in}{0.977952in}}%
\pgfpathlineto{\pgfqpoint{1.421981in}{0.998858in}}%
\pgfpathlineto{\pgfqpoint{1.423129in}{0.878649in}}%
\pgfpathlineto{\pgfqpoint{1.423512in}{0.878649in}}%
\pgfpathlineto{\pgfqpoint{1.424660in}{0.967499in}}%
\pgfpathlineto{\pgfqpoint{1.425043in}{0.967499in}}%
\pgfpathlineto{\pgfqpoint{1.425426in}{0.967499in}}%
\pgfpathlineto{\pgfqpoint{1.425426in}{0.878649in}}%
\pgfpathlineto{\pgfqpoint{1.426957in}{1.024990in}}%
\pgfpathlineto{\pgfqpoint{1.427340in}{1.024990in}}%
\pgfpathlineto{\pgfqpoint{1.428106in}{0.873422in}}%
\pgfpathlineto{\pgfqpoint{1.428872in}{0.946593in}}%
\pgfpathlineto{\pgfqpoint{1.429637in}{0.946593in}}%
\pgfpathlineto{\pgfqpoint{1.430020in}{0.878649in}}%
\pgfpathlineto{\pgfqpoint{1.431169in}{1.014537in}}%
\pgfpathlineto{\pgfqpoint{1.431551in}{1.014537in}}%
\pgfpathlineto{\pgfqpoint{1.431934in}{0.915234in}}%
\pgfpathlineto{\pgfqpoint{1.433083in}{1.019764in}}%
\pgfpathlineto{\pgfqpoint{1.433466in}{1.019764in}}%
\pgfpathlineto{\pgfqpoint{1.433848in}{0.894328in}}%
\pgfpathlineto{\pgfqpoint{1.434231in}{1.024990in}}%
\pgfpathlineto{\pgfqpoint{1.434997in}{0.988405in}}%
\pgfpathlineto{\pgfqpoint{1.435763in}{0.988405in}}%
\pgfpathlineto{\pgfqpoint{1.436145in}{0.925687in}}%
\pgfpathlineto{\pgfqpoint{1.436911in}{0.993631in}}%
\pgfpathlineto{\pgfqpoint{1.437294in}{0.962273in}}%
\pgfpathlineto{\pgfqpoint{1.437677in}{0.962273in}}%
\pgfpathlineto{\pgfqpoint{1.437677in}{1.009311in}}%
\pgfpathlineto{\pgfqpoint{1.439208in}{0.977952in}}%
\pgfpathlineto{\pgfqpoint{1.439591in}{0.977952in}}%
\pgfpathlineto{\pgfqpoint{1.440356in}{0.920461in}}%
\pgfpathlineto{\pgfqpoint{1.441122in}{0.977952in}}%
\pgfpathlineto{\pgfqpoint{1.441505in}{0.977952in}}%
\pgfpathlineto{\pgfqpoint{1.441888in}{0.993631in}}%
\pgfpathlineto{\pgfqpoint{1.443036in}{0.941367in}}%
\pgfpathlineto{\pgfqpoint{1.443802in}{0.941367in}}%
\pgfpathlineto{\pgfqpoint{1.444185in}{1.030217in}}%
\pgfpathlineto{\pgfqpoint{1.444950in}{0.930914in}}%
\pgfpathlineto{\pgfqpoint{1.445333in}{0.967499in}}%
\pgfpathlineto{\pgfqpoint{1.445716in}{0.967499in}}%
\pgfpathlineto{\pgfqpoint{1.446099in}{1.056349in}}%
\pgfpathlineto{\pgfqpoint{1.447247in}{0.883875in}}%
\pgfpathlineto{\pgfqpoint{1.447630in}{0.883875in}}%
\pgfpathlineto{\pgfqpoint{1.449162in}{1.082482in}}%
\pgfpathlineto{\pgfqpoint{1.449544in}{1.082482in}}%
\pgfpathlineto{\pgfqpoint{1.449544in}{0.957046in}}%
\pgfpathlineto{\pgfqpoint{1.449927in}{1.108614in}}%
\pgfpathlineto{\pgfqpoint{1.451076in}{0.988405in}}%
\pgfpathlineto{\pgfqpoint{1.451459in}{0.988405in}}%
\pgfpathlineto{\pgfqpoint{1.452224in}{0.925687in}}%
\pgfpathlineto{\pgfqpoint{1.451841in}{1.040670in}}%
\pgfpathlineto{\pgfqpoint{1.452990in}{0.983178in}}%
\pgfpathlineto{\pgfqpoint{1.453373in}{0.983178in}}%
\pgfpathlineto{\pgfqpoint{1.453756in}{1.019764in}}%
\pgfpathlineto{\pgfqpoint{1.454521in}{0.946593in}}%
\pgfpathlineto{\pgfqpoint{1.454904in}{0.951820in}}%
\pgfpathlineto{\pgfqpoint{1.455287in}{0.951820in}}%
\pgfpathlineto{\pgfqpoint{1.456053in}{1.051123in}}%
\pgfpathlineto{\pgfqpoint{1.456818in}{0.915234in}}%
\pgfpathlineto{\pgfqpoint{1.457201in}{0.915234in}}%
\pgfpathlineto{\pgfqpoint{1.457584in}{1.087708in}}%
\pgfpathlineto{\pgfqpoint{1.458732in}{1.009311in}}%
\pgfpathlineto{\pgfqpoint{1.459115in}{1.009311in}}%
\pgfpathlineto{\pgfqpoint{1.460264in}{0.936140in}}%
\pgfpathlineto{\pgfqpoint{1.459498in}{1.082482in}}%
\pgfpathlineto{\pgfqpoint{1.460646in}{1.045896in}}%
\pgfpathlineto{\pgfqpoint{1.461029in}{1.045896in}}%
\pgfpathlineto{\pgfqpoint{1.461795in}{0.957046in}}%
\pgfpathlineto{\pgfqpoint{1.462178in}{1.098161in}}%
\pgfpathlineto{\pgfqpoint{1.462561in}{1.024990in}}%
\pgfpathlineto{\pgfqpoint{1.462943in}{1.024990in}}%
\pgfpathlineto{\pgfqpoint{1.464092in}{1.040670in}}%
\pgfpathlineto{\pgfqpoint{1.464475in}{0.894328in}}%
\pgfpathlineto{\pgfqpoint{1.464858in}{0.894328in}}%
\pgfpathlineto{\pgfqpoint{1.466006in}{1.113840in}}%
\pgfpathlineto{\pgfqpoint{1.466389in}{1.082482in}}%
\pgfpathlineto{\pgfqpoint{1.466772in}{1.082482in}}%
\pgfpathlineto{\pgfqpoint{1.467920in}{0.915234in}}%
\pgfpathlineto{\pgfqpoint{1.468303in}{1.019764in}}%
\pgfpathlineto{\pgfqpoint{1.468686in}{1.019764in}}%
\pgfpathlineto{\pgfqpoint{1.469452in}{1.103388in}}%
\pgfpathlineto{\pgfqpoint{1.470217in}{1.019764in}}%
\pgfpathlineto{\pgfqpoint{1.470600in}{1.019764in}}%
\pgfpathlineto{\pgfqpoint{1.471366in}{1.124293in}}%
\pgfpathlineto{\pgfqpoint{1.472131in}{1.061576in}}%
\pgfpathlineto{\pgfqpoint{1.472514in}{1.061576in}}%
\pgfpathlineto{\pgfqpoint{1.472514in}{1.072029in}}%
\pgfpathlineto{\pgfqpoint{1.474046in}{1.009311in}}%
\pgfpathlineto{\pgfqpoint{1.474428in}{1.009311in}}%
\pgfpathlineto{\pgfqpoint{1.474428in}{0.983178in}}%
\pgfpathlineto{\pgfqpoint{1.475960in}{1.139973in}}%
\pgfpathlineto{\pgfqpoint{1.476343in}{1.139973in}}%
\pgfpathlineto{\pgfqpoint{1.476343in}{1.004084in}}%
\pgfpathlineto{\pgfqpoint{1.477874in}{1.014537in}}%
\pgfpathlineto{\pgfqpoint{1.478257in}{1.014537in}}%
\pgfpathlineto{\pgfqpoint{1.478640in}{1.113840in}}%
\pgfpathlineto{\pgfqpoint{1.479022in}{0.993631in}}%
\pgfpathlineto{\pgfqpoint{1.479788in}{1.082482in}}%
\pgfpathlineto{\pgfqpoint{1.480171in}{1.082482in}}%
\pgfpathlineto{\pgfqpoint{1.480171in}{1.051123in}}%
\pgfpathlineto{\pgfqpoint{1.480936in}{1.171332in}}%
\pgfpathlineto{\pgfqpoint{1.481702in}{1.171332in}}%
\pgfpathlineto{\pgfqpoint{1.482085in}{1.171332in}}%
\pgfpathlineto{\pgfqpoint{1.483616in}{1.024990in}}%
\pgfpathlineto{\pgfqpoint{1.483999in}{1.024990in}}%
\pgfpathlineto{\pgfqpoint{1.484765in}{1.171332in}}%
\pgfpathlineto{\pgfqpoint{1.485530in}{1.056349in}}%
\pgfpathlineto{\pgfqpoint{1.485913in}{1.056349in}}%
\pgfpathlineto{\pgfqpoint{1.485913in}{1.019764in}}%
\pgfpathlineto{\pgfqpoint{1.487062in}{1.176558in}}%
\pgfpathlineto{\pgfqpoint{1.487445in}{1.092935in}}%
\pgfpathlineto{\pgfqpoint{1.487827in}{1.092935in}}%
\pgfpathlineto{\pgfqpoint{1.487827in}{1.051123in}}%
\pgfpathlineto{\pgfqpoint{1.488210in}{1.160879in}}%
\pgfpathlineto{\pgfqpoint{1.489359in}{1.113840in}}%
\pgfpathlineto{\pgfqpoint{1.489742in}{1.113840in}}%
\pgfpathlineto{\pgfqpoint{1.489742in}{1.035443in}}%
\pgfpathlineto{\pgfqpoint{1.490124in}{1.155652in}}%
\pgfpathlineto{\pgfqpoint{1.491273in}{1.103388in}}%
\pgfpathlineto{\pgfqpoint{1.491656in}{1.103388in}}%
\pgfpathlineto{\pgfqpoint{1.491656in}{1.072029in}}%
\pgfpathlineto{\pgfqpoint{1.493187in}{1.155652in}}%
\pgfpathlineto{\pgfqpoint{1.493570in}{1.155652in}}%
\pgfpathlineto{\pgfqpoint{1.493570in}{1.207917in}}%
\pgfpathlineto{\pgfqpoint{1.495101in}{1.098161in}}%
\pgfpathlineto{\pgfqpoint{1.495484in}{1.098161in}}%
\pgfpathlineto{\pgfqpoint{1.495484in}{1.160879in}}%
\pgfpathlineto{\pgfqpoint{1.496250in}{1.072029in}}%
\pgfpathlineto{\pgfqpoint{1.497015in}{1.139973in}}%
\pgfpathlineto{\pgfqpoint{1.497398in}{1.139973in}}%
\pgfpathlineto{\pgfqpoint{1.497398in}{1.087708in}}%
\pgfpathlineto{\pgfqpoint{1.498929in}{1.197464in}}%
\pgfpathlineto{\pgfqpoint{1.499312in}{1.197464in}}%
\pgfpathlineto{\pgfqpoint{1.500461in}{1.024990in}}%
\pgfpathlineto{\pgfqpoint{1.500078in}{1.218370in}}%
\pgfpathlineto{\pgfqpoint{1.500844in}{1.171332in}}%
\pgfpathlineto{\pgfqpoint{1.501226in}{1.171332in}}%
\pgfpathlineto{\pgfqpoint{1.501992in}{1.061576in}}%
\pgfpathlineto{\pgfqpoint{1.501609in}{1.218370in}}%
\pgfpathlineto{\pgfqpoint{1.502758in}{1.187011in}}%
\pgfpathlineto{\pgfqpoint{1.503141in}{1.187011in}}%
\pgfpathlineto{\pgfqpoint{1.504289in}{1.077255in}}%
\pgfpathlineto{\pgfqpoint{1.504672in}{1.103388in}}%
\pgfpathlineto{\pgfqpoint{1.505055in}{1.103388in}}%
\pgfpathlineto{\pgfqpoint{1.505055in}{1.004084in}}%
\pgfpathlineto{\pgfqpoint{1.506586in}{1.213144in}}%
\pgfpathlineto{\pgfqpoint{1.506969in}{1.213144in}}%
\pgfpathlineto{\pgfqpoint{1.508117in}{1.134746in}}%
\pgfpathlineto{\pgfqpoint{1.508500in}{1.150426in}}%
\pgfpathlineto{\pgfqpoint{1.508883in}{1.150426in}}%
\pgfpathlineto{\pgfqpoint{1.509649in}{1.066802in}}%
\pgfpathlineto{\pgfqpoint{1.510414in}{1.139973in}}%
\pgfpathlineto{\pgfqpoint{1.510797in}{1.139973in}}%
\pgfpathlineto{\pgfqpoint{1.511180in}{1.213144in}}%
\pgfpathlineto{\pgfqpoint{1.511563in}{1.087708in}}%
\pgfpathlineto{\pgfqpoint{1.512329in}{1.150426in}}%
\pgfpathlineto{\pgfqpoint{1.512711in}{1.150426in}}%
\pgfpathlineto{\pgfqpoint{1.513094in}{1.197464in}}%
\pgfpathlineto{\pgfqpoint{1.513860in}{1.124293in}}%
\pgfpathlineto{\pgfqpoint{1.514243in}{1.139973in}}%
\pgfpathlineto{\pgfqpoint{1.514626in}{1.139973in}}%
\pgfpathlineto{\pgfqpoint{1.514626in}{1.119067in}}%
\pgfpathlineto{\pgfqpoint{1.515774in}{1.202691in}}%
\pgfpathlineto{\pgfqpoint{1.516157in}{1.160879in}}%
\pgfpathlineto{\pgfqpoint{1.516923in}{1.160879in}}%
\pgfpathlineto{\pgfqpoint{1.518071in}{1.270635in}}%
\pgfpathlineto{\pgfqpoint{1.518454in}{1.098161in}}%
\pgfpathlineto{\pgfqpoint{1.518837in}{1.098161in}}%
\pgfpathlineto{\pgfqpoint{1.519985in}{1.218370in}}%
\pgfpathlineto{\pgfqpoint{1.520368in}{1.160879in}}%
\pgfpathlineto{\pgfqpoint{1.520751in}{1.160879in}}%
\pgfpathlineto{\pgfqpoint{1.520751in}{1.312447in}}%
\pgfpathlineto{\pgfqpoint{1.521134in}{1.150426in}}%
\pgfpathlineto{\pgfqpoint{1.522282in}{1.270635in}}%
\pgfpathlineto{\pgfqpoint{1.522665in}{1.270635in}}%
\pgfpathlineto{\pgfqpoint{1.523431in}{1.139973in}}%
\pgfpathlineto{\pgfqpoint{1.524196in}{1.307220in}}%
\pgfpathlineto{\pgfqpoint{1.524579in}{1.307220in}}%
\pgfpathlineto{\pgfqpoint{1.526110in}{1.202691in}}%
\pgfpathlineto{\pgfqpoint{1.526493in}{1.202691in}}%
\pgfpathlineto{\pgfqpoint{1.527642in}{1.124293in}}%
\pgfpathlineto{\pgfqpoint{1.528025in}{1.286314in}}%
\pgfpathlineto{\pgfqpoint{1.528407in}{1.286314in}}%
\pgfpathlineto{\pgfqpoint{1.529556in}{1.124293in}}%
\pgfpathlineto{\pgfqpoint{1.529939in}{1.124293in}}%
\pgfpathlineto{\pgfqpoint{1.530322in}{1.124293in}}%
\pgfpathlineto{\pgfqpoint{1.530322in}{1.260182in}}%
\pgfpathlineto{\pgfqpoint{1.531853in}{1.197464in}}%
\pgfpathlineto{\pgfqpoint{1.532236in}{1.197464in}}%
\pgfpathlineto{\pgfqpoint{1.532236in}{1.228823in}}%
\pgfpathlineto{\pgfqpoint{1.533001in}{1.155652in}}%
\pgfpathlineto{\pgfqpoint{1.533767in}{1.176558in}}%
\pgfpathlineto{\pgfqpoint{1.534150in}{1.176558in}}%
\pgfpathlineto{\pgfqpoint{1.534916in}{1.045896in}}%
\pgfpathlineto{\pgfqpoint{1.535681in}{1.223597in}}%
\pgfpathlineto{\pgfqpoint{1.536064in}{1.223597in}}%
\pgfpathlineto{\pgfqpoint{1.536064in}{1.328126in}}%
\pgfpathlineto{\pgfqpoint{1.537212in}{1.207917in}}%
\pgfpathlineto{\pgfqpoint{1.537595in}{1.249729in}}%
\pgfpathlineto{\pgfqpoint{1.537978in}{1.249729in}}%
\pgfpathlineto{\pgfqpoint{1.537978in}{1.139973in}}%
\pgfpathlineto{\pgfqpoint{1.539509in}{1.270635in}}%
\pgfpathlineto{\pgfqpoint{1.539892in}{1.270635in}}%
\pgfpathlineto{\pgfqpoint{1.539892in}{1.166105in}}%
\pgfpathlineto{\pgfqpoint{1.541424in}{1.239276in}}%
\pgfpathlineto{\pgfqpoint{1.541806in}{1.239276in}}%
\pgfpathlineto{\pgfqpoint{1.541806in}{1.134746in}}%
\pgfpathlineto{\pgfqpoint{1.543338in}{1.270635in}}%
\pgfpathlineto{\pgfqpoint{1.543721in}{1.270635in}}%
\pgfpathlineto{\pgfqpoint{1.543721in}{1.171332in}}%
\pgfpathlineto{\pgfqpoint{1.545252in}{1.275861in}}%
\pgfpathlineto{\pgfqpoint{1.545635in}{1.275861in}}%
\pgfpathlineto{\pgfqpoint{1.545635in}{1.296767in}}%
\pgfpathlineto{\pgfqpoint{1.546783in}{1.119067in}}%
\pgfpathlineto{\pgfqpoint{1.547166in}{1.187011in}}%
\pgfpathlineto{\pgfqpoint{1.547549in}{1.187011in}}%
\pgfpathlineto{\pgfqpoint{1.547932in}{1.328126in}}%
\pgfpathlineto{\pgfqpoint{1.548697in}{1.181785in}}%
\pgfpathlineto{\pgfqpoint{1.549080in}{1.291541in}}%
\pgfpathlineto{\pgfqpoint{1.549463in}{1.291541in}}%
\pgfpathlineto{\pgfqpoint{1.550229in}{1.176558in}}%
\pgfpathlineto{\pgfqpoint{1.550994in}{1.349032in}}%
\pgfpathlineto{\pgfqpoint{1.551377in}{1.349032in}}%
\pgfpathlineto{\pgfqpoint{1.552909in}{1.171332in}}%
\pgfpathlineto{\pgfqpoint{1.553291in}{1.171332in}}%
\pgfpathlineto{\pgfqpoint{1.554057in}{1.369938in}}%
\pgfpathlineto{\pgfqpoint{1.554823in}{1.286314in}}%
\pgfpathlineto{\pgfqpoint{1.555206in}{1.286314in}}%
\pgfpathlineto{\pgfqpoint{1.556354in}{1.228823in}}%
\pgfpathlineto{\pgfqpoint{1.555971in}{1.296767in}}%
\pgfpathlineto{\pgfqpoint{1.556737in}{1.265408in}}%
\pgfpathlineto{\pgfqpoint{1.557120in}{1.265408in}}%
\pgfpathlineto{\pgfqpoint{1.558268in}{1.239276in}}%
\pgfpathlineto{\pgfqpoint{1.558651in}{1.364712in}}%
\pgfpathlineto{\pgfqpoint{1.559034in}{1.364712in}}%
\pgfpathlineto{\pgfqpoint{1.560182in}{1.171332in}}%
\pgfpathlineto{\pgfqpoint{1.560565in}{1.270635in}}%
\pgfpathlineto{\pgfqpoint{1.560948in}{1.270635in}}%
\pgfpathlineto{\pgfqpoint{1.562096in}{1.364712in}}%
\pgfpathlineto{\pgfqpoint{1.561331in}{1.218370in}}%
\pgfpathlineto{\pgfqpoint{1.562479in}{1.275861in}}%
\pgfpathlineto{\pgfqpoint{1.562862in}{1.275861in}}%
\pgfpathlineto{\pgfqpoint{1.563245in}{1.349032in}}%
\pgfpathlineto{\pgfqpoint{1.564393in}{1.301994in}}%
\pgfpathlineto{\pgfqpoint{1.564776in}{1.301994in}}%
\pgfpathlineto{\pgfqpoint{1.564776in}{1.265408in}}%
\pgfpathlineto{\pgfqpoint{1.566308in}{1.443109in}}%
\pgfpathlineto{\pgfqpoint{1.566690in}{1.443109in}}%
\pgfpathlineto{\pgfqpoint{1.566690in}{1.197464in}}%
\pgfpathlineto{\pgfqpoint{1.568222in}{1.312447in}}%
\pgfpathlineto{\pgfqpoint{1.568605in}{1.312447in}}%
\pgfpathlineto{\pgfqpoint{1.568605in}{1.207917in}}%
\pgfpathlineto{\pgfqpoint{1.570136in}{1.234050in}}%
\pgfpathlineto{\pgfqpoint{1.570519in}{1.234050in}}%
\pgfpathlineto{\pgfqpoint{1.570519in}{1.218370in}}%
\pgfpathlineto{\pgfqpoint{1.571667in}{1.338579in}}%
\pgfpathlineto{\pgfqpoint{1.572050in}{1.244503in}}%
\pgfpathlineto{\pgfqpoint{1.572433in}{1.244503in}}%
\pgfpathlineto{\pgfqpoint{1.572433in}{1.166105in}}%
\pgfpathlineto{\pgfqpoint{1.572816in}{1.390844in}}%
\pgfpathlineto{\pgfqpoint{1.573964in}{1.333353in}}%
\pgfpathlineto{\pgfqpoint{1.574347in}{1.333353in}}%
\pgfpathlineto{\pgfqpoint{1.575495in}{1.234050in}}%
\pgfpathlineto{\pgfqpoint{1.575113in}{1.349032in}}%
\pgfpathlineto{\pgfqpoint{1.575878in}{1.322900in}}%
\pgfpathlineto{\pgfqpoint{1.576261in}{1.322900in}}%
\pgfpathlineto{\pgfqpoint{1.576644in}{1.218370in}}%
\pgfpathlineto{\pgfqpoint{1.577792in}{1.354259in}}%
\pgfpathlineto{\pgfqpoint{1.578175in}{1.354259in}}%
\pgfpathlineto{\pgfqpoint{1.578941in}{1.187011in}}%
\pgfpathlineto{\pgfqpoint{1.578558in}{1.359485in}}%
\pgfpathlineto{\pgfqpoint{1.579707in}{1.338579in}}%
\pgfpathlineto{\pgfqpoint{1.580089in}{1.338579in}}%
\pgfpathlineto{\pgfqpoint{1.581238in}{1.218370in}}%
\pgfpathlineto{\pgfqpoint{1.581621in}{1.317673in}}%
\pgfpathlineto{\pgfqpoint{1.582004in}{1.317673in}}%
\pgfpathlineto{\pgfqpoint{1.582004in}{1.343806in}}%
\pgfpathlineto{\pgfqpoint{1.582769in}{1.265408in}}%
\pgfpathlineto{\pgfqpoint{1.583535in}{1.296767in}}%
\pgfpathlineto{\pgfqpoint{1.583918in}{1.296767in}}%
\pgfpathlineto{\pgfqpoint{1.584301in}{1.223597in}}%
\pgfpathlineto{\pgfqpoint{1.585066in}{1.375165in}}%
\pgfpathlineto{\pgfqpoint{1.585449in}{1.354259in}}%
\pgfpathlineto{\pgfqpoint{1.585832in}{1.354259in}}%
\pgfpathlineto{\pgfqpoint{1.586980in}{1.197464in}}%
\pgfpathlineto{\pgfqpoint{1.587363in}{1.322900in}}%
\pgfpathlineto{\pgfqpoint{1.587746in}{1.322900in}}%
\pgfpathlineto{\pgfqpoint{1.588895in}{1.328126in}}%
\pgfpathlineto{\pgfqpoint{1.589277in}{1.239276in}}%
\pgfpathlineto{\pgfqpoint{1.589660in}{1.239276in}}%
\pgfpathlineto{\pgfqpoint{1.590043in}{1.364712in}}%
\pgfpathlineto{\pgfqpoint{1.591192in}{1.328126in}}%
\pgfpathlineto{\pgfqpoint{1.591574in}{1.328126in}}%
\pgfpathlineto{\pgfqpoint{1.591574in}{1.239276in}}%
\pgfpathlineto{\pgfqpoint{1.593106in}{1.239276in}}%
\pgfpathlineto{\pgfqpoint{1.593489in}{1.239276in}}%
\pgfpathlineto{\pgfqpoint{1.595020in}{1.349032in}}%
\pgfpathlineto{\pgfqpoint{1.595403in}{1.349032in}}%
\pgfpathlineto{\pgfqpoint{1.595785in}{1.396070in}}%
\pgfpathlineto{\pgfqpoint{1.596551in}{1.249729in}}%
\pgfpathlineto{\pgfqpoint{1.596934in}{1.301994in}}%
\pgfpathlineto{\pgfqpoint{1.597317in}{1.301994in}}%
\pgfpathlineto{\pgfqpoint{1.597317in}{1.333353in}}%
\pgfpathlineto{\pgfqpoint{1.598082in}{1.218370in}}%
\pgfpathlineto{\pgfqpoint{1.598848in}{1.307220in}}%
\pgfpathlineto{\pgfqpoint{1.599231in}{1.307220in}}%
\pgfpathlineto{\pgfqpoint{1.599997in}{1.333353in}}%
\pgfpathlineto{\pgfqpoint{1.599614in}{1.234050in}}%
\pgfpathlineto{\pgfqpoint{1.600762in}{1.275861in}}%
\pgfpathlineto{\pgfqpoint{1.601145in}{1.275861in}}%
\pgfpathlineto{\pgfqpoint{1.601528in}{1.380391in}}%
\pgfpathlineto{\pgfqpoint{1.602676in}{1.249729in}}%
\pgfpathlineto{\pgfqpoint{1.603059in}{1.249729in}}%
\pgfpathlineto{\pgfqpoint{1.603442in}{1.223597in}}%
\pgfpathlineto{\pgfqpoint{1.604208in}{1.411750in}}%
\pgfpathlineto{\pgfqpoint{1.604591in}{1.275861in}}%
\pgfpathlineto{\pgfqpoint{1.604973in}{1.275861in}}%
\pgfpathlineto{\pgfqpoint{1.605356in}{1.202691in}}%
\pgfpathlineto{\pgfqpoint{1.605739in}{1.375165in}}%
\pgfpathlineto{\pgfqpoint{1.606505in}{1.333353in}}%
\pgfpathlineto{\pgfqpoint{1.606888in}{1.333353in}}%
\pgfpathlineto{\pgfqpoint{1.608419in}{1.171332in}}%
\pgfpathlineto{\pgfqpoint{1.608802in}{1.171332in}}%
\pgfpathlineto{\pgfqpoint{1.608802in}{1.359485in}}%
\pgfpathlineto{\pgfqpoint{1.610333in}{1.301994in}}%
\pgfpathlineto{\pgfqpoint{1.610716in}{1.301994in}}%
\pgfpathlineto{\pgfqpoint{1.610716in}{1.187011in}}%
\pgfpathlineto{\pgfqpoint{1.611099in}{1.422203in}}%
\pgfpathlineto{\pgfqpoint{1.612247in}{1.343806in}}%
\pgfpathlineto{\pgfqpoint{1.612630in}{1.343806in}}%
\pgfpathlineto{\pgfqpoint{1.613013in}{1.218370in}}%
\pgfpathlineto{\pgfqpoint{1.614161in}{1.291541in}}%
\pgfpathlineto{\pgfqpoint{1.614544in}{1.291541in}}%
\pgfpathlineto{\pgfqpoint{1.614927in}{1.364712in}}%
\pgfpathlineto{\pgfqpoint{1.616075in}{1.187011in}}%
\pgfpathlineto{\pgfqpoint{1.616458in}{1.187011in}}%
\pgfpathlineto{\pgfqpoint{1.616841in}{1.406523in}}%
\pgfpathlineto{\pgfqpoint{1.617990in}{1.343806in}}%
\pgfpathlineto{\pgfqpoint{1.618372in}{1.343806in}}%
\pgfpathlineto{\pgfqpoint{1.618755in}{1.359485in}}%
\pgfpathlineto{\pgfqpoint{1.619904in}{1.223597in}}%
\pgfpathlineto{\pgfqpoint{1.620287in}{1.223597in}}%
\pgfpathlineto{\pgfqpoint{1.620287in}{1.343806in}}%
\pgfpathlineto{\pgfqpoint{1.621818in}{1.322900in}}%
\pgfpathlineto{\pgfqpoint{1.622201in}{1.322900in}}%
\pgfpathlineto{\pgfqpoint{1.622584in}{1.202691in}}%
\pgfpathlineto{\pgfqpoint{1.623732in}{1.301994in}}%
\pgfpathlineto{\pgfqpoint{1.624115in}{1.301994in}}%
\pgfpathlineto{\pgfqpoint{1.624881in}{1.427429in}}%
\pgfpathlineto{\pgfqpoint{1.625263in}{1.244503in}}%
\pgfpathlineto{\pgfqpoint{1.625646in}{1.275861in}}%
\pgfpathlineto{\pgfqpoint{1.626029in}{1.275861in}}%
\pgfpathlineto{\pgfqpoint{1.626412in}{1.218370in}}%
\pgfpathlineto{\pgfqpoint{1.626795in}{1.343806in}}%
\pgfpathlineto{\pgfqpoint{1.627560in}{1.343806in}}%
\pgfpathlineto{\pgfqpoint{1.628326in}{1.343806in}}%
\pgfpathlineto{\pgfqpoint{1.628326in}{1.443109in}}%
\pgfpathlineto{\pgfqpoint{1.629475in}{1.213144in}}%
\pgfpathlineto{\pgfqpoint{1.629857in}{1.228823in}}%
\pgfpathlineto{\pgfqpoint{1.630240in}{1.228823in}}%
\pgfpathlineto{\pgfqpoint{1.630240in}{1.207917in}}%
\pgfpathlineto{\pgfqpoint{1.631772in}{1.427429in}}%
\pgfpathlineto{\pgfqpoint{1.632154in}{1.427429in}}%
\pgfpathlineto{\pgfqpoint{1.632920in}{1.239276in}}%
\pgfpathlineto{\pgfqpoint{1.633686in}{1.333353in}}%
\pgfpathlineto{\pgfqpoint{1.634068in}{1.333353in}}%
\pgfpathlineto{\pgfqpoint{1.634068in}{1.202691in}}%
\pgfpathlineto{\pgfqpoint{1.634834in}{1.453562in}}%
\pgfpathlineto{\pgfqpoint{1.635600in}{1.213144in}}%
\pgfpathlineto{\pgfqpoint{1.635983in}{1.213144in}}%
\pgfpathlineto{\pgfqpoint{1.637131in}{1.505827in}}%
\pgfpathlineto{\pgfqpoint{1.637514in}{1.437882in}}%
\pgfpathlineto{\pgfqpoint{1.637897in}{1.437882in}}%
\pgfpathlineto{\pgfqpoint{1.639428in}{1.197464in}}%
\pgfpathlineto{\pgfqpoint{1.639811in}{1.197464in}}%
\pgfpathlineto{\pgfqpoint{1.640959in}{1.375165in}}%
\pgfpathlineto{\pgfqpoint{1.641342in}{1.296767in}}%
\pgfpathlineto{\pgfqpoint{1.641725in}{1.296767in}}%
\pgfpathlineto{\pgfqpoint{1.641725in}{1.160879in}}%
\pgfpathlineto{\pgfqpoint{1.642491in}{1.396070in}}%
\pgfpathlineto{\pgfqpoint{1.643256in}{1.333353in}}%
\pgfpathlineto{\pgfqpoint{1.643639in}{1.333353in}}%
\pgfpathlineto{\pgfqpoint{1.644788in}{1.187011in}}%
\pgfpathlineto{\pgfqpoint{1.644022in}{1.385617in}}%
\pgfpathlineto{\pgfqpoint{1.645171in}{1.328126in}}%
\pgfpathlineto{\pgfqpoint{1.645553in}{1.328126in}}%
\pgfpathlineto{\pgfqpoint{1.646319in}{1.244503in}}%
\pgfpathlineto{\pgfqpoint{1.647085in}{1.396070in}}%
\pgfpathlineto{\pgfqpoint{1.647468in}{1.396070in}}%
\pgfpathlineto{\pgfqpoint{1.648233in}{1.406523in}}%
\pgfpathlineto{\pgfqpoint{1.648999in}{1.260182in}}%
\pgfpathlineto{\pgfqpoint{1.649382in}{1.260182in}}%
\pgfpathlineto{\pgfqpoint{1.649765in}{1.406523in}}%
\pgfpathlineto{\pgfqpoint{1.649765in}{1.218370in}}%
\pgfpathlineto{\pgfqpoint{1.650913in}{1.375165in}}%
\pgfpathlineto{\pgfqpoint{1.651296in}{1.375165in}}%
\pgfpathlineto{\pgfqpoint{1.651296in}{1.254955in}}%
\pgfpathlineto{\pgfqpoint{1.652827in}{1.301994in}}%
\pgfpathlineto{\pgfqpoint{1.653210in}{1.301994in}}%
\pgfpathlineto{\pgfqpoint{1.653593in}{1.207917in}}%
\pgfpathlineto{\pgfqpoint{1.654358in}{1.317673in}}%
\pgfpathlineto{\pgfqpoint{1.654741in}{1.249729in}}%
\pgfpathlineto{\pgfqpoint{1.655124in}{1.249729in}}%
\pgfpathlineto{\pgfqpoint{1.655890in}{1.469241in}}%
\pgfpathlineto{\pgfqpoint{1.656655in}{1.333353in}}%
\pgfpathlineto{\pgfqpoint{1.657038in}{1.333353in}}%
\pgfpathlineto{\pgfqpoint{1.657421in}{1.234050in}}%
\pgfpathlineto{\pgfqpoint{1.658570in}{1.411750in}}%
\pgfpathlineto{\pgfqpoint{1.658952in}{1.411750in}}%
\pgfpathlineto{\pgfqpoint{1.660101in}{1.218370in}}%
\pgfpathlineto{\pgfqpoint{1.660484in}{1.354259in}}%
\pgfpathlineto{\pgfqpoint{1.660867in}{1.354259in}}%
\pgfpathlineto{\pgfqpoint{1.660867in}{1.223597in}}%
\pgfpathlineto{\pgfqpoint{1.661249in}{1.411750in}}%
\pgfpathlineto{\pgfqpoint{1.662398in}{1.234050in}}%
\pgfpathlineto{\pgfqpoint{1.662781in}{1.234050in}}%
\pgfpathlineto{\pgfqpoint{1.663164in}{1.416976in}}%
\pgfpathlineto{\pgfqpoint{1.664312in}{1.359485in}}%
\pgfpathlineto{\pgfqpoint{1.665078in}{1.359485in}}%
\pgfpathlineto{\pgfqpoint{1.665843in}{1.265408in}}%
\pgfpathlineto{\pgfqpoint{1.666226in}{1.369938in}}%
\pgfpathlineto{\pgfqpoint{1.666609in}{1.322900in}}%
\pgfpathlineto{\pgfqpoint{1.666992in}{1.322900in}}%
\pgfpathlineto{\pgfqpoint{1.667375in}{1.427429in}}%
\pgfpathlineto{\pgfqpoint{1.668140in}{1.228823in}}%
\pgfpathlineto{\pgfqpoint{1.668523in}{1.349032in}}%
\pgfpathlineto{\pgfqpoint{1.668906in}{1.349032in}}%
\pgfpathlineto{\pgfqpoint{1.669289in}{1.396070in}}%
\pgfpathlineto{\pgfqpoint{1.670437in}{1.228823in}}%
\pgfpathlineto{\pgfqpoint{1.670820in}{1.228823in}}%
\pgfpathlineto{\pgfqpoint{1.670820in}{1.422203in}}%
\pgfpathlineto{\pgfqpoint{1.672351in}{1.411750in}}%
\pgfpathlineto{\pgfqpoint{1.672734in}{1.411750in}}%
\pgfpathlineto{\pgfqpoint{1.673500in}{1.333353in}}%
\pgfpathlineto{\pgfqpoint{1.674266in}{1.343806in}}%
\pgfpathlineto{\pgfqpoint{1.674648in}{1.343806in}}%
\pgfpathlineto{\pgfqpoint{1.675031in}{1.380391in}}%
\pgfpathlineto{\pgfqpoint{1.676180in}{1.275861in}}%
\pgfpathlineto{\pgfqpoint{1.676563in}{1.275861in}}%
\pgfpathlineto{\pgfqpoint{1.677328in}{1.207917in}}%
\pgfpathlineto{\pgfqpoint{1.677711in}{1.380391in}}%
\pgfpathlineto{\pgfqpoint{1.678094in}{1.343806in}}%
\pgfpathlineto{\pgfqpoint{1.678477in}{1.343806in}}%
\pgfpathlineto{\pgfqpoint{1.679625in}{1.234050in}}%
\pgfpathlineto{\pgfqpoint{1.678860in}{1.364712in}}%
\pgfpathlineto{\pgfqpoint{1.680008in}{1.275861in}}%
\pgfpathlineto{\pgfqpoint{1.680391in}{1.275861in}}%
\pgfpathlineto{\pgfqpoint{1.680391in}{1.385617in}}%
\pgfpathlineto{\pgfqpoint{1.681922in}{1.338579in}}%
\pgfpathlineto{\pgfqpoint{1.682305in}{1.338579in}}%
\pgfpathlineto{\pgfqpoint{1.682305in}{1.385617in}}%
\pgfpathlineto{\pgfqpoint{1.683071in}{1.213144in}}%
\pgfpathlineto{\pgfqpoint{1.683836in}{1.260182in}}%
\pgfpathlineto{\pgfqpoint{1.684219in}{1.260182in}}%
\pgfpathlineto{\pgfqpoint{1.684219in}{1.218370in}}%
\pgfpathlineto{\pgfqpoint{1.685751in}{1.458788in}}%
\pgfpathlineto{\pgfqpoint{1.686133in}{1.458788in}}%
\pgfpathlineto{\pgfqpoint{1.686516in}{1.249729in}}%
\pgfpathlineto{\pgfqpoint{1.687665in}{1.364712in}}%
\pgfpathlineto{\pgfqpoint{1.688048in}{1.364712in}}%
\pgfpathlineto{\pgfqpoint{1.688430in}{1.380391in}}%
\pgfpathlineto{\pgfqpoint{1.689579in}{1.249729in}}%
\pgfpathlineto{\pgfqpoint{1.689962in}{1.249729in}}%
\pgfpathlineto{\pgfqpoint{1.691493in}{1.390844in}}%
\pgfpathlineto{\pgfqpoint{1.691876in}{1.390844in}}%
\pgfpathlineto{\pgfqpoint{1.693407in}{1.234050in}}%
\pgfpathlineto{\pgfqpoint{1.693790in}{1.234050in}}%
\pgfpathlineto{\pgfqpoint{1.693790in}{1.223597in}}%
\pgfpathlineto{\pgfqpoint{1.694173in}{1.396070in}}%
\pgfpathlineto{\pgfqpoint{1.695321in}{1.385617in}}%
\pgfpathlineto{\pgfqpoint{1.695704in}{1.385617in}}%
\pgfpathlineto{\pgfqpoint{1.696853in}{1.239276in}}%
\pgfpathlineto{\pgfqpoint{1.697235in}{1.448335in}}%
\pgfpathlineto{\pgfqpoint{1.697618in}{1.448335in}}%
\pgfpathlineto{\pgfqpoint{1.698767in}{1.213144in}}%
\pgfpathlineto{\pgfqpoint{1.699150in}{1.333353in}}%
\pgfpathlineto{\pgfqpoint{1.699532in}{1.333353in}}%
\pgfpathlineto{\pgfqpoint{1.699532in}{1.416976in}}%
\pgfpathlineto{\pgfqpoint{1.700298in}{1.192238in}}%
\pgfpathlineto{\pgfqpoint{1.701064in}{1.275861in}}%
\pgfpathlineto{\pgfqpoint{1.701447in}{1.275861in}}%
\pgfpathlineto{\pgfqpoint{1.701447in}{1.171332in}}%
\pgfpathlineto{\pgfqpoint{1.701829in}{1.416976in}}%
\pgfpathlineto{\pgfqpoint{1.702978in}{1.333353in}}%
\pgfpathlineto{\pgfqpoint{1.703361in}{1.333353in}}%
\pgfpathlineto{\pgfqpoint{1.703361in}{1.249729in}}%
\pgfpathlineto{\pgfqpoint{1.704126in}{1.390844in}}%
\pgfpathlineto{\pgfqpoint{1.704892in}{1.286314in}}%
\pgfpathlineto{\pgfqpoint{1.705275in}{1.286314in}}%
\pgfpathlineto{\pgfqpoint{1.705658in}{1.375165in}}%
\pgfpathlineto{\pgfqpoint{1.706806in}{1.301994in}}%
\pgfpathlineto{\pgfqpoint{1.707189in}{1.301994in}}%
\pgfpathlineto{\pgfqpoint{1.707955in}{1.218370in}}%
\pgfpathlineto{\pgfqpoint{1.707572in}{1.364712in}}%
\pgfpathlineto{\pgfqpoint{1.708720in}{1.364712in}}%
\pgfpathlineto{\pgfqpoint{1.709103in}{1.364712in}}%
\pgfpathlineto{\pgfqpoint{1.709486in}{1.234050in}}%
\pgfpathlineto{\pgfqpoint{1.710635in}{1.396070in}}%
\pgfpathlineto{\pgfqpoint{1.711017in}{1.396070in}}%
\pgfpathlineto{\pgfqpoint{1.711400in}{1.171332in}}%
\pgfpathlineto{\pgfqpoint{1.712549in}{1.322900in}}%
\pgfpathlineto{\pgfqpoint{1.712931in}{1.322900in}}%
\pgfpathlineto{\pgfqpoint{1.713697in}{1.202691in}}%
\pgfpathlineto{\pgfqpoint{1.713314in}{1.422203in}}%
\pgfpathlineto{\pgfqpoint{1.714463in}{1.338579in}}%
\pgfpathlineto{\pgfqpoint{1.714846in}{1.338579in}}%
\pgfpathlineto{\pgfqpoint{1.714846in}{1.223597in}}%
\pgfpathlineto{\pgfqpoint{1.715228in}{1.380391in}}%
\pgfpathlineto{\pgfqpoint{1.716377in}{1.322900in}}%
\pgfpathlineto{\pgfqpoint{1.716760in}{1.322900in}}%
\pgfpathlineto{\pgfqpoint{1.717525in}{1.202691in}}%
\pgfpathlineto{\pgfqpoint{1.717143in}{1.390844in}}%
\pgfpathlineto{\pgfqpoint{1.718291in}{1.207917in}}%
\pgfpathlineto{\pgfqpoint{1.718674in}{1.207917in}}%
\pgfpathlineto{\pgfqpoint{1.719057in}{1.307220in}}%
\pgfpathlineto{\pgfqpoint{1.720205in}{1.223597in}}%
\pgfpathlineto{\pgfqpoint{1.720588in}{1.223597in}}%
\pgfpathlineto{\pgfqpoint{1.721737in}{1.171332in}}%
\pgfpathlineto{\pgfqpoint{1.722119in}{1.375165in}}%
\pgfpathlineto{\pgfqpoint{1.722502in}{1.375165in}}%
\pgfpathlineto{\pgfqpoint{1.722885in}{1.166105in}}%
\pgfpathlineto{\pgfqpoint{1.723268in}{1.385617in}}%
\pgfpathlineto{\pgfqpoint{1.724034in}{1.380391in}}%
\pgfpathlineto{\pgfqpoint{1.724416in}{1.380391in}}%
\pgfpathlineto{\pgfqpoint{1.725182in}{1.286314in}}%
\pgfpathlineto{\pgfqpoint{1.725948in}{1.301994in}}%
\pgfpathlineto{\pgfqpoint{1.726331in}{1.301994in}}%
\pgfpathlineto{\pgfqpoint{1.727096in}{1.333353in}}%
\pgfpathlineto{\pgfqpoint{1.727862in}{1.265408in}}%
\pgfpathlineto{\pgfqpoint{1.728245in}{1.265408in}}%
\pgfpathlineto{\pgfqpoint{1.728628in}{1.369938in}}%
\pgfpathlineto{\pgfqpoint{1.729776in}{1.260182in}}%
\pgfpathlineto{\pgfqpoint{1.730159in}{1.260182in}}%
\pgfpathlineto{\pgfqpoint{1.730542in}{1.328126in}}%
\pgfpathlineto{\pgfqpoint{1.731690in}{1.176558in}}%
\pgfpathlineto{\pgfqpoint{1.732073in}{1.176558in}}%
\pgfpathlineto{\pgfqpoint{1.732456in}{1.354259in}}%
\pgfpathlineto{\pgfqpoint{1.733604in}{1.275861in}}%
\pgfpathlineto{\pgfqpoint{1.733987in}{1.275861in}}%
\pgfpathlineto{\pgfqpoint{1.733987in}{1.354259in}}%
\pgfpathlineto{\pgfqpoint{1.735518in}{1.207917in}}%
\pgfpathlineto{\pgfqpoint{1.735901in}{1.207917in}}%
\pgfpathlineto{\pgfqpoint{1.736284in}{1.322900in}}%
\pgfpathlineto{\pgfqpoint{1.737433in}{1.223597in}}%
\pgfpathlineto{\pgfqpoint{1.737815in}{1.223597in}}%
\pgfpathlineto{\pgfqpoint{1.737815in}{1.333353in}}%
\pgfpathlineto{\pgfqpoint{1.739347in}{1.275861in}}%
\pgfpathlineto{\pgfqpoint{1.739730in}{1.275861in}}%
\pgfpathlineto{\pgfqpoint{1.739730in}{1.265408in}}%
\pgfpathlineto{\pgfqpoint{1.740112in}{1.317673in}}%
\pgfpathlineto{\pgfqpoint{1.741261in}{1.307220in}}%
\pgfpathlineto{\pgfqpoint{1.741644in}{1.307220in}}%
\pgfpathlineto{\pgfqpoint{1.741644in}{1.260182in}}%
\pgfpathlineto{\pgfqpoint{1.742792in}{1.333353in}}%
\pgfpathlineto{\pgfqpoint{1.743175in}{1.275861in}}%
\pgfpathlineto{\pgfqpoint{1.743558in}{1.275861in}}%
\pgfpathlineto{\pgfqpoint{1.744706in}{1.385617in}}%
\pgfpathlineto{\pgfqpoint{1.744324in}{1.228823in}}%
\pgfpathlineto{\pgfqpoint{1.745089in}{1.359485in}}%
\pgfpathlineto{\pgfqpoint{1.745472in}{1.359485in}}%
\pgfpathlineto{\pgfqpoint{1.746621in}{1.213144in}}%
\pgfpathlineto{\pgfqpoint{1.747003in}{1.239276in}}%
\pgfpathlineto{\pgfqpoint{1.747386in}{1.239276in}}%
\pgfpathlineto{\pgfqpoint{1.747386in}{1.343806in}}%
\pgfpathlineto{\pgfqpoint{1.747769in}{1.197464in}}%
\pgfpathlineto{\pgfqpoint{1.748918in}{1.239276in}}%
\pgfpathlineto{\pgfqpoint{1.749300in}{1.239276in}}%
\pgfpathlineto{\pgfqpoint{1.749300in}{1.213144in}}%
\pgfpathlineto{\pgfqpoint{1.749683in}{1.322900in}}%
\pgfpathlineto{\pgfqpoint{1.750832in}{1.213144in}}%
\pgfpathlineto{\pgfqpoint{1.751214in}{1.213144in}}%
\pgfpathlineto{\pgfqpoint{1.752746in}{1.432656in}}%
\pgfpathlineto{\pgfqpoint{1.753129in}{1.432656in}}%
\pgfpathlineto{\pgfqpoint{1.754277in}{1.260182in}}%
\pgfpathlineto{\pgfqpoint{1.754660in}{1.281088in}}%
\pgfpathlineto{\pgfqpoint{1.755043in}{1.281088in}}%
\pgfpathlineto{\pgfqpoint{1.755426in}{1.328126in}}%
\pgfpathlineto{\pgfqpoint{1.756574in}{1.207917in}}%
\pgfpathlineto{\pgfqpoint{1.756957in}{1.207917in}}%
\pgfpathlineto{\pgfqpoint{1.758105in}{1.432656in}}%
\pgfpathlineto{\pgfqpoint{1.758488in}{1.359485in}}%
\pgfpathlineto{\pgfqpoint{1.758871in}{1.359485in}}%
\pgfpathlineto{\pgfqpoint{1.760020in}{1.207917in}}%
\pgfpathlineto{\pgfqpoint{1.760402in}{1.260182in}}%
\pgfpathlineto{\pgfqpoint{1.760785in}{1.260182in}}%
\pgfpathlineto{\pgfqpoint{1.760785in}{1.385617in}}%
\pgfpathlineto{\pgfqpoint{1.761934in}{1.239276in}}%
\pgfpathlineto{\pgfqpoint{1.762317in}{1.249729in}}%
\pgfpathlineto{\pgfqpoint{1.762699in}{1.249729in}}%
\pgfpathlineto{\pgfqpoint{1.763082in}{1.322900in}}%
\pgfpathlineto{\pgfqpoint{1.763465in}{1.239276in}}%
\pgfpathlineto{\pgfqpoint{1.764231in}{1.260182in}}%
\pgfpathlineto{\pgfqpoint{1.764614in}{1.260182in}}%
\pgfpathlineto{\pgfqpoint{1.764996in}{1.286314in}}%
\pgfpathlineto{\pgfqpoint{1.766145in}{1.281088in}}%
\pgfpathlineto{\pgfqpoint{1.766528in}{1.281088in}}%
\pgfpathlineto{\pgfqpoint{1.766528in}{1.338579in}}%
\pgfpathlineto{\pgfqpoint{1.767676in}{1.254955in}}%
\pgfpathlineto{\pgfqpoint{1.768059in}{1.281088in}}%
\pgfpathlineto{\pgfqpoint{1.768442in}{1.281088in}}%
\pgfpathlineto{\pgfqpoint{1.769973in}{1.176558in}}%
\pgfpathlineto{\pgfqpoint{1.770356in}{1.176558in}}%
\pgfpathlineto{\pgfqpoint{1.771122in}{1.380391in}}%
\pgfpathlineto{\pgfqpoint{1.771887in}{1.328126in}}%
\pgfpathlineto{\pgfqpoint{1.772270in}{1.328126in}}%
\pgfpathlineto{\pgfqpoint{1.772270in}{1.343806in}}%
\pgfpathlineto{\pgfqpoint{1.773801in}{1.192238in}}%
\pgfpathlineto{\pgfqpoint{1.774184in}{1.192238in}}%
\pgfpathlineto{\pgfqpoint{1.775333in}{1.401297in}}%
\pgfpathlineto{\pgfqpoint{1.775716in}{1.301994in}}%
\pgfpathlineto{\pgfqpoint{1.776098in}{1.301994in}}%
\pgfpathlineto{\pgfqpoint{1.776098in}{1.249729in}}%
\pgfpathlineto{\pgfqpoint{1.777630in}{1.349032in}}%
\pgfpathlineto{\pgfqpoint{1.778013in}{1.349032in}}%
\pgfpathlineto{\pgfqpoint{1.779161in}{1.207917in}}%
\pgfpathlineto{\pgfqpoint{1.779544in}{1.286314in}}%
\pgfpathlineto{\pgfqpoint{1.779927in}{1.286314in}}%
\pgfpathlineto{\pgfqpoint{1.780310in}{1.385617in}}%
\pgfpathlineto{\pgfqpoint{1.781458in}{1.254955in}}%
\pgfpathlineto{\pgfqpoint{1.781841in}{1.254955in}}%
\pgfpathlineto{\pgfqpoint{1.781841in}{1.380391in}}%
\pgfpathlineto{\pgfqpoint{1.782989in}{1.218370in}}%
\pgfpathlineto{\pgfqpoint{1.783372in}{1.317673in}}%
\pgfpathlineto{\pgfqpoint{1.783755in}{1.317673in}}%
\pgfpathlineto{\pgfqpoint{1.784521in}{1.213144in}}%
\pgfpathlineto{\pgfqpoint{1.785286in}{1.411750in}}%
\pgfpathlineto{\pgfqpoint{1.785669in}{1.411750in}}%
\pgfpathlineto{\pgfqpoint{1.786052in}{1.239276in}}%
\pgfpathlineto{\pgfqpoint{1.787201in}{1.375165in}}%
\pgfpathlineto{\pgfqpoint{1.787966in}{1.375165in}}%
\pgfpathlineto{\pgfqpoint{1.789115in}{1.228823in}}%
\pgfpathlineto{\pgfqpoint{1.789497in}{1.234050in}}%
\pgfpathlineto{\pgfqpoint{1.789880in}{1.234050in}}%
\pgfpathlineto{\pgfqpoint{1.791412in}{1.406523in}}%
\pgfpathlineto{\pgfqpoint{1.791794in}{1.406523in}}%
\pgfpathlineto{\pgfqpoint{1.793326in}{1.228823in}}%
\pgfpathlineto{\pgfqpoint{1.793709in}{1.228823in}}%
\pgfpathlineto{\pgfqpoint{1.794091in}{1.406523in}}%
\pgfpathlineto{\pgfqpoint{1.795240in}{1.176558in}}%
\pgfpathlineto{\pgfqpoint{1.795623in}{1.176558in}}%
\pgfpathlineto{\pgfqpoint{1.796388in}{1.422203in}}%
\pgfpathlineto{\pgfqpoint{1.797154in}{1.354259in}}%
\pgfpathlineto{\pgfqpoint{1.797537in}{1.354259in}}%
\pgfpathlineto{\pgfqpoint{1.798685in}{1.307220in}}%
\pgfpathlineto{\pgfqpoint{1.798303in}{1.427429in}}%
\pgfpathlineto{\pgfqpoint{1.799068in}{1.343806in}}%
\pgfpathlineto{\pgfqpoint{1.799451in}{1.343806in}}%
\pgfpathlineto{\pgfqpoint{1.800217in}{1.396070in}}%
\pgfpathlineto{\pgfqpoint{1.800600in}{1.244503in}}%
\pgfpathlineto{\pgfqpoint{1.800982in}{1.354259in}}%
\pgfpathlineto{\pgfqpoint{1.801748in}{1.354259in}}%
\pgfpathlineto{\pgfqpoint{1.802131in}{1.207917in}}%
\pgfpathlineto{\pgfqpoint{1.802514in}{1.448335in}}%
\pgfpathlineto{\pgfqpoint{1.803279in}{1.432656in}}%
\pgfpathlineto{\pgfqpoint{1.803662in}{1.432656in}}%
\pgfpathlineto{\pgfqpoint{1.804811in}{1.265408in}}%
\pgfpathlineto{\pgfqpoint{1.804045in}{1.448335in}}%
\pgfpathlineto{\pgfqpoint{1.805194in}{1.307220in}}%
\pgfpathlineto{\pgfqpoint{1.805576in}{1.307220in}}%
\pgfpathlineto{\pgfqpoint{1.805576in}{1.254955in}}%
\pgfpathlineto{\pgfqpoint{1.807108in}{1.437882in}}%
\pgfpathlineto{\pgfqpoint{1.807491in}{1.437882in}}%
\pgfpathlineto{\pgfqpoint{1.808639in}{1.307220in}}%
\pgfpathlineto{\pgfqpoint{1.809022in}{1.406523in}}%
\pgfpathlineto{\pgfqpoint{1.809405in}{1.406523in}}%
\pgfpathlineto{\pgfqpoint{1.809405in}{1.281088in}}%
\pgfpathlineto{\pgfqpoint{1.810170in}{1.552865in}}%
\pgfpathlineto{\pgfqpoint{1.810936in}{1.333353in}}%
\pgfpathlineto{\pgfqpoint{1.811319in}{1.333353in}}%
\pgfpathlineto{\pgfqpoint{1.811702in}{1.375165in}}%
\pgfpathlineto{\pgfqpoint{1.812850in}{1.333353in}}%
\pgfpathlineto{\pgfqpoint{1.813233in}{1.333353in}}%
\pgfpathlineto{\pgfqpoint{1.813233in}{1.443109in}}%
\pgfpathlineto{\pgfqpoint{1.813999in}{1.328126in}}%
\pgfpathlineto{\pgfqpoint{1.814764in}{1.338579in}}%
\pgfpathlineto{\pgfqpoint{1.815530in}{1.338579in}}%
\pgfpathlineto{\pgfqpoint{1.815530in}{1.464015in}}%
\pgfpathlineto{\pgfqpoint{1.817061in}{1.296767in}}%
\pgfpathlineto{\pgfqpoint{1.817444in}{1.296767in}}%
\pgfpathlineto{\pgfqpoint{1.818975in}{1.416976in}}%
\pgfpathlineto{\pgfqpoint{1.819358in}{1.416976in}}%
\pgfpathlineto{\pgfqpoint{1.820124in}{1.453562in}}%
\pgfpathlineto{\pgfqpoint{1.820890in}{1.338579in}}%
\pgfpathlineto{\pgfqpoint{1.821272in}{1.338579in}}%
\pgfpathlineto{\pgfqpoint{1.821655in}{1.328126in}}%
\pgfpathlineto{\pgfqpoint{1.822804in}{1.490147in}}%
\pgfpathlineto{\pgfqpoint{1.823187in}{1.490147in}}%
\pgfpathlineto{\pgfqpoint{1.824718in}{1.275861in}}%
\pgfpathlineto{\pgfqpoint{1.825101in}{1.275861in}}%
\pgfpathlineto{\pgfqpoint{1.825484in}{1.453562in}}%
\pgfpathlineto{\pgfqpoint{1.826632in}{1.338579in}}%
\pgfpathlineto{\pgfqpoint{1.827015in}{1.338579in}}%
\pgfpathlineto{\pgfqpoint{1.827015in}{1.286314in}}%
\pgfpathlineto{\pgfqpoint{1.827398in}{1.521506in}}%
\pgfpathlineto{\pgfqpoint{1.828546in}{1.432656in}}%
\pgfpathlineto{\pgfqpoint{1.828929in}{1.432656in}}%
\pgfpathlineto{\pgfqpoint{1.830077in}{1.328126in}}%
\pgfpathlineto{\pgfqpoint{1.830460in}{1.516280in}}%
\pgfpathlineto{\pgfqpoint{1.830843in}{1.516280in}}%
\pgfpathlineto{\pgfqpoint{1.832374in}{1.281088in}}%
\pgfpathlineto{\pgfqpoint{1.832757in}{1.281088in}}%
\pgfpathlineto{\pgfqpoint{1.833523in}{1.573771in}}%
\pgfpathlineto{\pgfqpoint{1.834289in}{1.448335in}}%
\pgfpathlineto{\pgfqpoint{1.834671in}{1.448335in}}%
\pgfpathlineto{\pgfqpoint{1.834671in}{1.312447in}}%
\pgfpathlineto{\pgfqpoint{1.836203in}{1.369938in}}%
\pgfpathlineto{\pgfqpoint{1.836586in}{1.369938in}}%
\pgfpathlineto{\pgfqpoint{1.837734in}{1.542412in}}%
\pgfpathlineto{\pgfqpoint{1.836968in}{1.328126in}}%
\pgfpathlineto{\pgfqpoint{1.838117in}{1.531959in}}%
\pgfpathlineto{\pgfqpoint{1.838500in}{1.531959in}}%
\pgfpathlineto{\pgfqpoint{1.840031in}{1.385617in}}%
\pgfpathlineto{\pgfqpoint{1.840414in}{1.385617in}}%
\pgfpathlineto{\pgfqpoint{1.840797in}{1.317673in}}%
\pgfpathlineto{\pgfqpoint{1.841945in}{1.516280in}}%
\pgfpathlineto{\pgfqpoint{1.842328in}{1.516280in}}%
\pgfpathlineto{\pgfqpoint{1.842328in}{1.542412in}}%
\pgfpathlineto{\pgfqpoint{1.843094in}{1.432656in}}%
\pgfpathlineto{\pgfqpoint{1.843859in}{1.448335in}}%
\pgfpathlineto{\pgfqpoint{1.844242in}{1.448335in}}%
\pgfpathlineto{\pgfqpoint{1.844625in}{1.537185in}}%
\pgfpathlineto{\pgfqpoint{1.845774in}{1.406523in}}%
\pgfpathlineto{\pgfqpoint{1.846156in}{1.406523in}}%
\pgfpathlineto{\pgfqpoint{1.846156in}{1.385617in}}%
\pgfpathlineto{\pgfqpoint{1.847305in}{1.542412in}}%
\pgfpathlineto{\pgfqpoint{1.847688in}{1.500600in}}%
\pgfpathlineto{\pgfqpoint{1.848070in}{1.500600in}}%
\pgfpathlineto{\pgfqpoint{1.848453in}{1.369938in}}%
\pgfpathlineto{\pgfqpoint{1.849602in}{1.458788in}}%
\pgfpathlineto{\pgfqpoint{1.849985in}{1.458788in}}%
\pgfpathlineto{\pgfqpoint{1.851133in}{1.563318in}}%
\pgfpathlineto{\pgfqpoint{1.851516in}{1.411750in}}%
\pgfpathlineto{\pgfqpoint{1.851899in}{1.411750in}}%
\pgfpathlineto{\pgfqpoint{1.852664in}{1.542412in}}%
\pgfpathlineto{\pgfqpoint{1.853430in}{1.490147in}}%
\pgfpathlineto{\pgfqpoint{1.853813in}{1.490147in}}%
\pgfpathlineto{\pgfqpoint{1.853813in}{1.375165in}}%
\pgfpathlineto{\pgfqpoint{1.854579in}{1.537185in}}%
\pgfpathlineto{\pgfqpoint{1.855344in}{1.500600in}}%
\pgfpathlineto{\pgfqpoint{1.855727in}{1.500600in}}%
\pgfpathlineto{\pgfqpoint{1.856493in}{1.589450in}}%
\pgfpathlineto{\pgfqpoint{1.856876in}{1.385617in}}%
\pgfpathlineto{\pgfqpoint{1.857258in}{1.474468in}}%
\pgfpathlineto{\pgfqpoint{1.857641in}{1.474468in}}%
\pgfpathlineto{\pgfqpoint{1.857641in}{1.636489in}}%
\pgfpathlineto{\pgfqpoint{1.859173in}{1.542412in}}%
\pgfpathlineto{\pgfqpoint{1.859555in}{1.542412in}}%
\pgfpathlineto{\pgfqpoint{1.860321in}{1.505827in}}%
\pgfpathlineto{\pgfqpoint{1.861087in}{1.636489in}}%
\pgfpathlineto{\pgfqpoint{1.861470in}{1.636489in}}%
\pgfpathlineto{\pgfqpoint{1.862618in}{1.484921in}}%
\pgfpathlineto{\pgfqpoint{1.863001in}{1.542412in}}%
\pgfpathlineto{\pgfqpoint{1.863384in}{1.542412in}}%
\pgfpathlineto{\pgfqpoint{1.864532in}{1.458788in}}%
\pgfpathlineto{\pgfqpoint{1.863767in}{1.725339in}}%
\pgfpathlineto{\pgfqpoint{1.864915in}{1.516280in}}%
\pgfpathlineto{\pgfqpoint{1.865298in}{1.516280in}}%
\pgfpathlineto{\pgfqpoint{1.866446in}{1.432656in}}%
\pgfpathlineto{\pgfqpoint{1.866829in}{1.646942in}}%
\pgfpathlineto{\pgfqpoint{1.867212in}{1.646942in}}%
\pgfpathlineto{\pgfqpoint{1.867212in}{1.484921in}}%
\pgfpathlineto{\pgfqpoint{1.868360in}{1.657395in}}%
\pgfpathlineto{\pgfqpoint{1.868743in}{1.521506in}}%
\pgfpathlineto{\pgfqpoint{1.869126in}{1.521506in}}%
\pgfpathlineto{\pgfqpoint{1.869126in}{1.469241in}}%
\pgfpathlineto{\pgfqpoint{1.869892in}{1.594677in}}%
\pgfpathlineto{\pgfqpoint{1.870657in}{1.500600in}}%
\pgfpathlineto{\pgfqpoint{1.871040in}{1.500600in}}%
\pgfpathlineto{\pgfqpoint{1.872189in}{1.641715in}}%
\pgfpathlineto{\pgfqpoint{1.872572in}{1.563318in}}%
\pgfpathlineto{\pgfqpoint{1.872954in}{1.563318in}}%
\pgfpathlineto{\pgfqpoint{1.873720in}{1.484921in}}%
\pgfpathlineto{\pgfqpoint{1.874486in}{1.631262in}}%
\pgfpathlineto{\pgfqpoint{1.874869in}{1.631262in}}%
\pgfpathlineto{\pgfqpoint{1.876400in}{1.469241in}}%
\pgfpathlineto{\pgfqpoint{1.876783in}{1.469241in}}%
\pgfpathlineto{\pgfqpoint{1.877931in}{1.735792in}}%
\pgfpathlineto{\pgfqpoint{1.878314in}{1.673074in}}%
\pgfpathlineto{\pgfqpoint{1.878697in}{1.673074in}}%
\pgfpathlineto{\pgfqpoint{1.878697in}{1.547638in}}%
\pgfpathlineto{\pgfqpoint{1.879463in}{1.767151in}}%
\pgfpathlineto{\pgfqpoint{1.880228in}{1.730565in}}%
\pgfpathlineto{\pgfqpoint{1.880611in}{1.730565in}}%
\pgfpathlineto{\pgfqpoint{1.880994in}{1.542412in}}%
\pgfpathlineto{\pgfqpoint{1.882142in}{1.693980in}}%
\pgfpathlineto{\pgfqpoint{1.882908in}{1.693980in}}%
\pgfpathlineto{\pgfqpoint{1.883674in}{1.469241in}}%
\pgfpathlineto{\pgfqpoint{1.884057in}{1.746245in}}%
\pgfpathlineto{\pgfqpoint{1.884439in}{1.495374in}}%
\pgfpathlineto{\pgfqpoint{1.884822in}{1.495374in}}%
\pgfpathlineto{\pgfqpoint{1.885588in}{1.699206in}}%
\pgfpathlineto{\pgfqpoint{1.886353in}{1.584224in}}%
\pgfpathlineto{\pgfqpoint{1.886736in}{1.584224in}}%
\pgfpathlineto{\pgfqpoint{1.886736in}{1.725339in}}%
\pgfpathlineto{\pgfqpoint{1.888268in}{1.683527in}}%
\pgfpathlineto{\pgfqpoint{1.888650in}{1.683527in}}%
\pgfpathlineto{\pgfqpoint{1.889799in}{1.720112in}}%
\pgfpathlineto{\pgfqpoint{1.889416in}{1.631262in}}%
\pgfpathlineto{\pgfqpoint{1.890182in}{1.688753in}}%
\pgfpathlineto{\pgfqpoint{1.890565in}{1.688753in}}%
\pgfpathlineto{\pgfqpoint{1.891330in}{1.610356in}}%
\pgfpathlineto{\pgfqpoint{1.892096in}{1.814189in}}%
\pgfpathlineto{\pgfqpoint{1.892479in}{1.814189in}}%
\pgfpathlineto{\pgfqpoint{1.892862in}{1.605130in}}%
\pgfpathlineto{\pgfqpoint{1.894010in}{1.777604in}}%
\pgfpathlineto{\pgfqpoint{1.894393in}{1.777604in}}%
\pgfpathlineto{\pgfqpoint{1.894393in}{1.829868in}}%
\pgfpathlineto{\pgfqpoint{1.894776in}{1.563318in}}%
\pgfpathlineto{\pgfqpoint{1.895924in}{1.662621in}}%
\pgfpathlineto{\pgfqpoint{1.896307in}{1.662621in}}%
\pgfpathlineto{\pgfqpoint{1.896307in}{1.474468in}}%
\pgfpathlineto{\pgfqpoint{1.897838in}{1.782830in}}%
\pgfpathlineto{\pgfqpoint{1.898221in}{1.782830in}}%
\pgfpathlineto{\pgfqpoint{1.898221in}{1.657395in}}%
\pgfpathlineto{\pgfqpoint{1.899753in}{1.793283in}}%
\pgfpathlineto{\pgfqpoint{1.900135in}{1.793283in}}%
\pgfpathlineto{\pgfqpoint{1.900518in}{1.657395in}}%
\pgfpathlineto{\pgfqpoint{1.901667in}{1.798509in}}%
\pgfpathlineto{\pgfqpoint{1.902050in}{1.798509in}}%
\pgfpathlineto{\pgfqpoint{1.902432in}{1.646942in}}%
\pgfpathlineto{\pgfqpoint{1.903198in}{1.908266in}}%
\pgfpathlineto{\pgfqpoint{1.903581in}{1.730565in}}%
\pgfpathlineto{\pgfqpoint{1.903964in}{1.730565in}}%
\pgfpathlineto{\pgfqpoint{1.903964in}{1.835095in}}%
\pgfpathlineto{\pgfqpoint{1.904729in}{1.620809in}}%
\pgfpathlineto{\pgfqpoint{1.905495in}{1.699206in}}%
\pgfpathlineto{\pgfqpoint{1.905878in}{1.699206in}}%
\pgfpathlineto{\pgfqpoint{1.906261in}{1.840321in}}%
\pgfpathlineto{\pgfqpoint{1.906643in}{1.652168in}}%
\pgfpathlineto{\pgfqpoint{1.907409in}{1.714886in}}%
\pgfpathlineto{\pgfqpoint{1.907792in}{1.714886in}}%
\pgfpathlineto{\pgfqpoint{1.908558in}{1.704433in}}%
\pgfpathlineto{\pgfqpoint{1.909323in}{1.897813in}}%
\pgfpathlineto{\pgfqpoint{1.909706in}{1.897813in}}%
\pgfpathlineto{\pgfqpoint{1.911237in}{1.673074in}}%
\pgfpathlineto{\pgfqpoint{1.911620in}{1.673074in}}%
\pgfpathlineto{\pgfqpoint{1.912769in}{1.814189in}}%
\pgfpathlineto{\pgfqpoint{1.912003in}{1.636489in}}%
\pgfpathlineto{\pgfqpoint{1.913152in}{1.793283in}}%
\pgfpathlineto{\pgfqpoint{1.913534in}{1.793283in}}%
\pgfpathlineto{\pgfqpoint{1.913534in}{1.761924in}}%
\pgfpathlineto{\pgfqpoint{1.914300in}{1.913492in}}%
\pgfpathlineto{\pgfqpoint{1.915066in}{1.866454in}}%
\pgfpathlineto{\pgfqpoint{1.915449in}{1.866454in}}%
\pgfpathlineto{\pgfqpoint{1.915449in}{1.573771in}}%
\pgfpathlineto{\pgfqpoint{1.916980in}{1.782830in}}%
\pgfpathlineto{\pgfqpoint{1.917363in}{1.782830in}}%
\pgfpathlineto{\pgfqpoint{1.918128in}{1.725339in}}%
\pgfpathlineto{\pgfqpoint{1.917746in}{1.876907in}}%
\pgfpathlineto{\pgfqpoint{1.918894in}{1.782830in}}%
\pgfpathlineto{\pgfqpoint{1.919277in}{1.782830in}}%
\pgfpathlineto{\pgfqpoint{1.919660in}{1.944851in}}%
\pgfpathlineto{\pgfqpoint{1.920043in}{1.746245in}}%
\pgfpathlineto{\pgfqpoint{1.920808in}{1.871680in}}%
\pgfpathlineto{\pgfqpoint{1.921191in}{1.871680in}}%
\pgfpathlineto{\pgfqpoint{1.921191in}{1.876907in}}%
\pgfpathlineto{\pgfqpoint{1.921574in}{1.767151in}}%
\pgfpathlineto{\pgfqpoint{1.922722in}{1.782830in}}%
\pgfpathlineto{\pgfqpoint{1.923105in}{1.782830in}}%
\pgfpathlineto{\pgfqpoint{1.923105in}{1.887360in}}%
\pgfpathlineto{\pgfqpoint{1.923871in}{1.720112in}}%
\pgfpathlineto{\pgfqpoint{1.924636in}{1.808962in}}%
\pgfpathlineto{\pgfqpoint{1.925019in}{1.808962in}}%
\pgfpathlineto{\pgfqpoint{1.925785in}{1.714886in}}%
\pgfpathlineto{\pgfqpoint{1.926168in}{1.892586in}}%
\pgfpathlineto{\pgfqpoint{1.926551in}{1.835095in}}%
\pgfpathlineto{\pgfqpoint{1.926933in}{1.835095in}}%
\pgfpathlineto{\pgfqpoint{1.926933in}{1.939624in}}%
\pgfpathlineto{\pgfqpoint{1.928465in}{1.704433in}}%
\pgfpathlineto{\pgfqpoint{1.928848in}{1.704433in}}%
\pgfpathlineto{\pgfqpoint{1.929230in}{1.950077in}}%
\pgfpathlineto{\pgfqpoint{1.930379in}{1.819415in}}%
\pgfpathlineto{\pgfqpoint{1.930762in}{1.819415in}}%
\pgfpathlineto{\pgfqpoint{1.931527in}{1.693980in}}%
\pgfpathlineto{\pgfqpoint{1.932293in}{1.767151in}}%
\pgfpathlineto{\pgfqpoint{1.932676in}{1.767151in}}%
\pgfpathlineto{\pgfqpoint{1.932676in}{1.908266in}}%
\pgfpathlineto{\pgfqpoint{1.934207in}{1.856001in}}%
\pgfpathlineto{\pgfqpoint{1.934590in}{1.856001in}}%
\pgfpathlineto{\pgfqpoint{1.934973in}{1.641715in}}%
\pgfpathlineto{\pgfqpoint{1.935739in}{1.970983in}}%
\pgfpathlineto{\pgfqpoint{1.936121in}{1.730565in}}%
\pgfpathlineto{\pgfqpoint{1.936504in}{1.730565in}}%
\pgfpathlineto{\pgfqpoint{1.937270in}{1.929172in}}%
\pgfpathlineto{\pgfqpoint{1.938036in}{1.856001in}}%
\pgfpathlineto{\pgfqpoint{1.938418in}{1.856001in}}%
\pgfpathlineto{\pgfqpoint{1.939567in}{1.950077in}}%
\pgfpathlineto{\pgfqpoint{1.939950in}{1.793283in}}%
\pgfpathlineto{\pgfqpoint{1.940333in}{1.793283in}}%
\pgfpathlineto{\pgfqpoint{1.940715in}{1.882133in}}%
\pgfpathlineto{\pgfqpoint{1.941098in}{1.610356in}}%
\pgfpathlineto{\pgfqpoint{1.941864in}{1.741018in}}%
\pgfpathlineto{\pgfqpoint{1.942247in}{1.741018in}}%
\pgfpathlineto{\pgfqpoint{1.942630in}{1.882133in}}%
\pgfpathlineto{\pgfqpoint{1.943778in}{1.850774in}}%
\pgfpathlineto{\pgfqpoint{1.944161in}{1.850774in}}%
\pgfpathlineto{\pgfqpoint{1.944926in}{1.735792in}}%
\pgfpathlineto{\pgfqpoint{1.945692in}{1.772377in}}%
\pgfpathlineto{\pgfqpoint{1.946075in}{1.772377in}}%
\pgfpathlineto{\pgfqpoint{1.947223in}{1.944851in}}%
\pgfpathlineto{\pgfqpoint{1.947606in}{1.819415in}}%
\pgfpathlineto{\pgfqpoint{1.947989in}{1.819415in}}%
\pgfpathlineto{\pgfqpoint{1.948755in}{1.923945in}}%
\pgfpathlineto{\pgfqpoint{1.949520in}{1.845548in}}%
\pgfpathlineto{\pgfqpoint{1.949903in}{1.845548in}}%
\pgfpathlineto{\pgfqpoint{1.949903in}{1.788057in}}%
\pgfpathlineto{\pgfqpoint{1.950669in}{1.939624in}}%
\pgfpathlineto{\pgfqpoint{1.951435in}{1.798509in}}%
\pgfpathlineto{\pgfqpoint{1.951817in}{1.798509in}}%
\pgfpathlineto{\pgfqpoint{1.951817in}{2.038928in}}%
\pgfpathlineto{\pgfqpoint{1.953349in}{1.788057in}}%
\pgfpathlineto{\pgfqpoint{1.953732in}{1.788057in}}%
\pgfpathlineto{\pgfqpoint{1.954114in}{1.944851in}}%
\pgfpathlineto{\pgfqpoint{1.955263in}{1.882133in}}%
\pgfpathlineto{\pgfqpoint{1.955646in}{1.882133in}}%
\pgfpathlineto{\pgfqpoint{1.956794in}{1.798509in}}%
\pgfpathlineto{\pgfqpoint{1.956411in}{2.054607in}}%
\pgfpathlineto{\pgfqpoint{1.957177in}{1.923945in}}%
\pgfpathlineto{\pgfqpoint{1.957560in}{1.923945in}}%
\pgfpathlineto{\pgfqpoint{1.958326in}{2.033701in}}%
\pgfpathlineto{\pgfqpoint{1.958708in}{1.741018in}}%
\pgfpathlineto{\pgfqpoint{1.959091in}{1.824642in}}%
\pgfpathlineto{\pgfqpoint{1.959474in}{1.824642in}}%
\pgfpathlineto{\pgfqpoint{1.960623in}{1.897813in}}%
\pgfpathlineto{\pgfqpoint{1.959857in}{1.798509in}}%
\pgfpathlineto{\pgfqpoint{1.961005in}{1.876907in}}%
\pgfpathlineto{\pgfqpoint{1.961388in}{1.876907in}}%
\pgfpathlineto{\pgfqpoint{1.962154in}{1.819415in}}%
\pgfpathlineto{\pgfqpoint{1.962919in}{1.923945in}}%
\pgfpathlineto{\pgfqpoint{1.963302in}{1.923945in}}%
\pgfpathlineto{\pgfqpoint{1.963302in}{1.887360in}}%
\pgfpathlineto{\pgfqpoint{1.964451in}{2.033701in}}%
\pgfpathlineto{\pgfqpoint{1.964834in}{1.986663in}}%
\pgfpathlineto{\pgfqpoint{1.965216in}{1.986663in}}%
\pgfpathlineto{\pgfqpoint{1.965599in}{1.840321in}}%
\pgfpathlineto{\pgfqpoint{1.966748in}{1.892586in}}%
\pgfpathlineto{\pgfqpoint{1.967131in}{1.892586in}}%
\pgfpathlineto{\pgfqpoint{1.967131in}{1.997116in}}%
\pgfpathlineto{\pgfqpoint{1.967896in}{1.819415in}}%
\pgfpathlineto{\pgfqpoint{1.968662in}{1.970983in}}%
\pgfpathlineto{\pgfqpoint{1.969045in}{1.970983in}}%
\pgfpathlineto{\pgfqpoint{1.970193in}{1.876907in}}%
\pgfpathlineto{\pgfqpoint{1.970576in}{1.887360in}}%
\pgfpathlineto{\pgfqpoint{1.970959in}{1.887360in}}%
\pgfpathlineto{\pgfqpoint{1.972107in}{2.112098in}}%
\pgfpathlineto{\pgfqpoint{1.972490in}{1.808962in}}%
\pgfpathlineto{\pgfqpoint{1.972873in}{1.808962in}}%
\pgfpathlineto{\pgfqpoint{1.972873in}{1.997116in}}%
\pgfpathlineto{\pgfqpoint{1.973639in}{1.751471in}}%
\pgfpathlineto{\pgfqpoint{1.974404in}{1.970983in}}%
\pgfpathlineto{\pgfqpoint{1.974787in}{1.970983in}}%
\pgfpathlineto{\pgfqpoint{1.975553in}{1.986663in}}%
\pgfpathlineto{\pgfqpoint{1.976319in}{1.829868in}}%
\pgfpathlineto{\pgfqpoint{1.976701in}{1.829868in}}%
\pgfpathlineto{\pgfqpoint{1.977084in}{1.976210in}}%
\pgfpathlineto{\pgfqpoint{1.978233in}{1.939624in}}%
\pgfpathlineto{\pgfqpoint{1.978616in}{1.939624in}}%
\pgfpathlineto{\pgfqpoint{1.979764in}{1.767151in}}%
\pgfpathlineto{\pgfqpoint{1.979381in}{2.033701in}}%
\pgfpathlineto{\pgfqpoint{1.980147in}{1.918719in}}%
\pgfpathlineto{\pgfqpoint{1.980530in}{1.918719in}}%
\pgfpathlineto{\pgfqpoint{1.980530in}{2.075513in}}%
\pgfpathlineto{\pgfqpoint{1.981678in}{1.819415in}}%
\pgfpathlineto{\pgfqpoint{1.982061in}{1.950077in}}%
\pgfpathlineto{\pgfqpoint{1.982827in}{1.950077in}}%
\pgfpathlineto{\pgfqpoint{1.983592in}{1.965757in}}%
\pgfpathlineto{\pgfqpoint{1.983209in}{1.772377in}}%
\pgfpathlineto{\pgfqpoint{1.984358in}{1.887360in}}%
\pgfpathlineto{\pgfqpoint{1.984741in}{1.887360in}}%
\pgfpathlineto{\pgfqpoint{1.985889in}{2.044154in}}%
\pgfpathlineto{\pgfqpoint{1.986272in}{1.688753in}}%
\pgfpathlineto{\pgfqpoint{1.986655in}{1.688753in}}%
\pgfpathlineto{\pgfqpoint{1.987038in}{2.033701in}}%
\pgfpathlineto{\pgfqpoint{1.988186in}{1.903039in}}%
\pgfpathlineto{\pgfqpoint{1.988569in}{1.903039in}}%
\pgfpathlineto{\pgfqpoint{1.989335in}{1.777604in}}%
\pgfpathlineto{\pgfqpoint{1.988952in}{2.096419in}}%
\pgfpathlineto{\pgfqpoint{1.990100in}{1.814189in}}%
\pgfpathlineto{\pgfqpoint{1.990483in}{1.814189in}}%
\pgfpathlineto{\pgfqpoint{1.990866in}{1.965757in}}%
\pgfpathlineto{\pgfqpoint{1.991632in}{1.688753in}}%
\pgfpathlineto{\pgfqpoint{1.992015in}{1.934398in}}%
\pgfpathlineto{\pgfqpoint{1.992397in}{1.934398in}}%
\pgfpathlineto{\pgfqpoint{1.993546in}{1.939624in}}%
\pgfpathlineto{\pgfqpoint{1.993929in}{1.746245in}}%
\pgfpathlineto{\pgfqpoint{1.994312in}{1.746245in}}%
\pgfpathlineto{\pgfqpoint{1.994312in}{1.965757in}}%
\pgfpathlineto{\pgfqpoint{1.995843in}{1.913492in}}%
\pgfpathlineto{\pgfqpoint{1.996226in}{1.913492in}}%
\pgfpathlineto{\pgfqpoint{1.996226in}{1.808962in}}%
\pgfpathlineto{\pgfqpoint{1.997374in}{2.127778in}}%
\pgfpathlineto{\pgfqpoint{1.997757in}{1.970983in}}%
\pgfpathlineto{\pgfqpoint{1.998140in}{1.970983in}}%
\pgfpathlineto{\pgfqpoint{1.998140in}{1.798509in}}%
\pgfpathlineto{\pgfqpoint{1.999671in}{1.835095in}}%
\pgfpathlineto{\pgfqpoint{2.000054in}{1.835095in}}%
\pgfpathlineto{\pgfqpoint{2.000054in}{2.007569in}}%
\pgfpathlineto{\pgfqpoint{2.001585in}{1.699206in}}%
\pgfpathlineto{\pgfqpoint{2.001968in}{1.699206in}}%
\pgfpathlineto{\pgfqpoint{2.002734in}{1.981436in}}%
\pgfpathlineto{\pgfqpoint{2.003499in}{1.699206in}}%
\pgfpathlineto{\pgfqpoint{2.003882in}{1.699206in}}%
\pgfpathlineto{\pgfqpoint{2.004648in}{1.970983in}}%
\pgfpathlineto{\pgfqpoint{2.005414in}{1.662621in}}%
\pgfpathlineto{\pgfqpoint{2.005796in}{1.662621in}}%
\pgfpathlineto{\pgfqpoint{2.006562in}{1.997116in}}%
\pgfpathlineto{\pgfqpoint{2.007328in}{1.908266in}}%
\pgfpathlineto{\pgfqpoint{2.007711in}{1.908266in}}%
\pgfpathlineto{\pgfqpoint{2.008093in}{1.720112in}}%
\pgfpathlineto{\pgfqpoint{2.009242in}{1.955304in}}%
\pgfpathlineto{\pgfqpoint{2.009625in}{1.955304in}}%
\pgfpathlineto{\pgfqpoint{2.010390in}{1.673074in}}%
\pgfpathlineto{\pgfqpoint{2.010008in}{1.991889in}}%
\pgfpathlineto{\pgfqpoint{2.011156in}{1.965757in}}%
\pgfpathlineto{\pgfqpoint{2.011539in}{1.965757in}}%
\pgfpathlineto{\pgfqpoint{2.011539in}{2.070286in}}%
\pgfpathlineto{\pgfqpoint{2.012687in}{1.819415in}}%
\pgfpathlineto{\pgfqpoint{2.013070in}{1.861227in}}%
\pgfpathlineto{\pgfqpoint{2.013453in}{1.861227in}}%
\pgfpathlineto{\pgfqpoint{2.014984in}{1.955304in}}%
\pgfpathlineto{\pgfqpoint{2.015367in}{1.955304in}}%
\pgfpathlineto{\pgfqpoint{2.016516in}{1.788057in}}%
\pgfpathlineto{\pgfqpoint{2.016899in}{1.866454in}}%
\pgfpathlineto{\pgfqpoint{2.017281in}{1.866454in}}%
\pgfpathlineto{\pgfqpoint{2.017664in}{2.018022in}}%
\pgfpathlineto{\pgfqpoint{2.018813in}{1.782830in}}%
\pgfpathlineto{\pgfqpoint{2.019196in}{1.782830in}}%
\pgfpathlineto{\pgfqpoint{2.019196in}{1.955304in}}%
\pgfpathlineto{\pgfqpoint{2.020727in}{1.741018in}}%
\pgfpathlineto{\pgfqpoint{2.021110in}{1.741018in}}%
\pgfpathlineto{\pgfqpoint{2.022258in}{1.876907in}}%
\pgfpathlineto{\pgfqpoint{2.021875in}{1.725339in}}%
\pgfpathlineto{\pgfqpoint{2.022641in}{1.735792in}}%
\pgfpathlineto{\pgfqpoint{2.023024in}{1.735792in}}%
\pgfpathlineto{\pgfqpoint{2.023789in}{1.944851in}}%
\pgfpathlineto{\pgfqpoint{2.024555in}{1.876907in}}%
\pgfpathlineto{\pgfqpoint{2.024938in}{1.876907in}}%
\pgfpathlineto{\pgfqpoint{2.025321in}{1.704433in}}%
\pgfpathlineto{\pgfqpoint{2.026086in}{1.882133in}}%
\pgfpathlineto{\pgfqpoint{2.026469in}{1.876907in}}%
\pgfpathlineto{\pgfqpoint{2.026852in}{1.876907in}}%
\pgfpathlineto{\pgfqpoint{2.027618in}{1.683527in}}%
\pgfpathlineto{\pgfqpoint{2.028001in}{1.944851in}}%
\pgfpathlineto{\pgfqpoint{2.028383in}{1.876907in}}%
\pgfpathlineto{\pgfqpoint{2.028766in}{1.876907in}}%
\pgfpathlineto{\pgfqpoint{2.028766in}{1.688753in}}%
\pgfpathlineto{\pgfqpoint{2.029532in}{1.986663in}}%
\pgfpathlineto{\pgfqpoint{2.030298in}{1.850774in}}%
\pgfpathlineto{\pgfqpoint{2.030680in}{1.850774in}}%
\pgfpathlineto{\pgfqpoint{2.031446in}{1.876907in}}%
\pgfpathlineto{\pgfqpoint{2.032212in}{1.615583in}}%
\pgfpathlineto{\pgfqpoint{2.032595in}{1.615583in}}%
\pgfpathlineto{\pgfqpoint{2.032595in}{1.965757in}}%
\pgfpathlineto{\pgfqpoint{2.034126in}{1.751471in}}%
\pgfpathlineto{\pgfqpoint{2.034509in}{1.751471in}}%
\pgfpathlineto{\pgfqpoint{2.034509in}{1.835095in}}%
\pgfpathlineto{\pgfqpoint{2.035274in}{1.725339in}}%
\pgfpathlineto{\pgfqpoint{2.036040in}{1.772377in}}%
\pgfpathlineto{\pgfqpoint{2.036423in}{1.772377in}}%
\pgfpathlineto{\pgfqpoint{2.036423in}{1.824642in}}%
\pgfpathlineto{\pgfqpoint{2.037954in}{1.735792in}}%
\pgfpathlineto{\pgfqpoint{2.038337in}{1.735792in}}%
\pgfpathlineto{\pgfqpoint{2.038337in}{1.709659in}}%
\pgfpathlineto{\pgfqpoint{2.039103in}{1.814189in}}%
\pgfpathlineto{\pgfqpoint{2.039868in}{1.761924in}}%
\pgfpathlineto{\pgfqpoint{2.040251in}{1.761924in}}%
\pgfpathlineto{\pgfqpoint{2.040634in}{1.976210in}}%
\pgfpathlineto{\pgfqpoint{2.041017in}{1.735792in}}%
\pgfpathlineto{\pgfqpoint{2.041782in}{1.772377in}}%
\pgfpathlineto{\pgfqpoint{2.042165in}{1.772377in}}%
\pgfpathlineto{\pgfqpoint{2.043314in}{1.803736in}}%
\pgfpathlineto{\pgfqpoint{2.043697in}{1.730565in}}%
\pgfpathlineto{\pgfqpoint{2.044079in}{1.730565in}}%
\pgfpathlineto{\pgfqpoint{2.045228in}{1.850774in}}%
\pgfpathlineto{\pgfqpoint{2.045611in}{1.699206in}}%
\pgfpathlineto{\pgfqpoint{2.045994in}{1.699206in}}%
\pgfpathlineto{\pgfqpoint{2.046759in}{1.840321in}}%
\pgfpathlineto{\pgfqpoint{2.046376in}{1.626036in}}%
\pgfpathlineto{\pgfqpoint{2.047525in}{1.714886in}}%
\pgfpathlineto{\pgfqpoint{2.047908in}{1.714886in}}%
\pgfpathlineto{\pgfqpoint{2.049439in}{1.819415in}}%
\pgfpathlineto{\pgfqpoint{2.049822in}{1.819415in}}%
\pgfpathlineto{\pgfqpoint{2.049822in}{1.537185in}}%
\pgfpathlineto{\pgfqpoint{2.050970in}{1.897813in}}%
\pgfpathlineto{\pgfqpoint{2.051353in}{1.761924in}}%
\pgfpathlineto{\pgfqpoint{2.051736in}{1.761924in}}%
\pgfpathlineto{\pgfqpoint{2.052885in}{1.584224in}}%
\pgfpathlineto{\pgfqpoint{2.052119in}{1.803736in}}%
\pgfpathlineto{\pgfqpoint{2.053267in}{1.693980in}}%
\pgfpathlineto{\pgfqpoint{2.053650in}{1.693980in}}%
\pgfpathlineto{\pgfqpoint{2.054033in}{1.876907in}}%
\pgfpathlineto{\pgfqpoint{2.055182in}{1.725339in}}%
\pgfpathlineto{\pgfqpoint{2.055564in}{1.725339in}}%
\pgfpathlineto{\pgfqpoint{2.055947in}{1.746245in}}%
\pgfpathlineto{\pgfqpoint{2.056713in}{1.589450in}}%
\pgfpathlineto{\pgfqpoint{2.057096in}{1.714886in}}%
\pgfpathlineto{\pgfqpoint{2.057479in}{1.714886in}}%
\pgfpathlineto{\pgfqpoint{2.057479in}{1.824642in}}%
\pgfpathlineto{\pgfqpoint{2.058244in}{1.678300in}}%
\pgfpathlineto{\pgfqpoint{2.059010in}{1.798509in}}%
\pgfpathlineto{\pgfqpoint{2.059393in}{1.798509in}}%
\pgfpathlineto{\pgfqpoint{2.060541in}{1.636489in}}%
\pgfpathlineto{\pgfqpoint{2.060158in}{1.829868in}}%
\pgfpathlineto{\pgfqpoint{2.060924in}{1.808962in}}%
\pgfpathlineto{\pgfqpoint{2.061307in}{1.808962in}}%
\pgfpathlineto{\pgfqpoint{2.062455in}{1.683527in}}%
\pgfpathlineto{\pgfqpoint{2.062838in}{1.866454in}}%
\pgfpathlineto{\pgfqpoint{2.063221in}{1.866454in}}%
\pgfpathlineto{\pgfqpoint{2.063221in}{1.631262in}}%
\pgfpathlineto{\pgfqpoint{2.064752in}{1.808962in}}%
\pgfpathlineto{\pgfqpoint{2.065135in}{1.808962in}}%
\pgfpathlineto{\pgfqpoint{2.066666in}{1.578997in}}%
\pgfpathlineto{\pgfqpoint{2.067049in}{1.578997in}}%
\pgfpathlineto{\pgfqpoint{2.068581in}{1.918719in}}%
\pgfpathlineto{\pgfqpoint{2.068963in}{1.918719in}}%
\pgfpathlineto{\pgfqpoint{2.069346in}{1.620809in}}%
\pgfpathlineto{\pgfqpoint{2.070495in}{1.782830in}}%
\pgfpathlineto{\pgfqpoint{2.071260in}{1.782830in}}%
\pgfpathlineto{\pgfqpoint{2.071260in}{1.589450in}}%
\pgfpathlineto{\pgfqpoint{2.072792in}{1.636489in}}%
\pgfpathlineto{\pgfqpoint{2.073175in}{1.636489in}}%
\pgfpathlineto{\pgfqpoint{2.073557in}{1.547638in}}%
\pgfpathlineto{\pgfqpoint{2.074323in}{1.741018in}}%
\pgfpathlineto{\pgfqpoint{2.074706in}{1.709659in}}%
\pgfpathlineto{\pgfqpoint{2.075089in}{1.709659in}}%
\pgfpathlineto{\pgfqpoint{2.075089in}{1.547638in}}%
\pgfpathlineto{\pgfqpoint{2.075472in}{1.845548in}}%
\pgfpathlineto{\pgfqpoint{2.076620in}{1.605130in}}%
\pgfpathlineto{\pgfqpoint{2.077003in}{1.605130in}}%
\pgfpathlineto{\pgfqpoint{2.077386in}{1.709659in}}%
\pgfpathlineto{\pgfqpoint{2.078534in}{1.673074in}}%
\pgfpathlineto{\pgfqpoint{2.079683in}{1.673074in}}%
\pgfpathlineto{\pgfqpoint{2.079683in}{1.725339in}}%
\pgfpathlineto{\pgfqpoint{2.081214in}{1.652168in}}%
\pgfpathlineto{\pgfqpoint{2.081597in}{1.652168in}}%
\pgfpathlineto{\pgfqpoint{2.082362in}{1.589450in}}%
\pgfpathlineto{\pgfqpoint{2.082745in}{1.819415in}}%
\pgfpathlineto{\pgfqpoint{2.083128in}{1.626036in}}%
\pgfpathlineto{\pgfqpoint{2.083511in}{1.626036in}}%
\pgfpathlineto{\pgfqpoint{2.083511in}{1.631262in}}%
\pgfpathlineto{\pgfqpoint{2.083894in}{1.505827in}}%
\pgfpathlineto{\pgfqpoint{2.085042in}{1.526732in}}%
\pgfpathlineto{\pgfqpoint{2.085425in}{1.526732in}}%
\pgfpathlineto{\pgfqpoint{2.086956in}{1.704433in}}%
\pgfpathlineto{\pgfqpoint{2.087339in}{1.704433in}}%
\pgfpathlineto{\pgfqpoint{2.088871in}{1.453562in}}%
\pgfpathlineto{\pgfqpoint{2.089253in}{1.453562in}}%
\pgfpathlineto{\pgfqpoint{2.089636in}{1.646942in}}%
\pgfpathlineto{\pgfqpoint{2.090785in}{1.578997in}}%
\pgfpathlineto{\pgfqpoint{2.091168in}{1.578997in}}%
\pgfpathlineto{\pgfqpoint{2.091168in}{1.657395in}}%
\pgfpathlineto{\pgfqpoint{2.092699in}{1.537185in}}%
\pgfpathlineto{\pgfqpoint{2.093082in}{1.537185in}}%
\pgfpathlineto{\pgfqpoint{2.093082in}{1.500600in}}%
\pgfpathlineto{\pgfqpoint{2.094613in}{1.741018in}}%
\pgfpathlineto{\pgfqpoint{2.094996in}{1.741018in}}%
\pgfpathlineto{\pgfqpoint{2.096144in}{1.505827in}}%
\pgfpathlineto{\pgfqpoint{2.096527in}{1.610356in}}%
\pgfpathlineto{\pgfqpoint{2.096910in}{1.610356in}}%
\pgfpathlineto{\pgfqpoint{2.097293in}{1.526732in}}%
\pgfpathlineto{\pgfqpoint{2.098058in}{1.615583in}}%
\pgfpathlineto{\pgfqpoint{2.098441in}{1.605130in}}%
\pgfpathlineto{\pgfqpoint{2.098824in}{1.605130in}}%
\pgfpathlineto{\pgfqpoint{2.099207in}{1.662621in}}%
\pgfpathlineto{\pgfqpoint{2.099973in}{1.589450in}}%
\pgfpathlineto{\pgfqpoint{2.100355in}{1.631262in}}%
\pgfpathlineto{\pgfqpoint{2.100738in}{1.631262in}}%
\pgfpathlineto{\pgfqpoint{2.102270in}{1.474468in}}%
\pgfpathlineto{\pgfqpoint{2.102652in}{1.474468in}}%
\pgfpathlineto{\pgfqpoint{2.102652in}{1.594677in}}%
\pgfpathlineto{\pgfqpoint{2.104184in}{1.464015in}}%
\pgfpathlineto{\pgfqpoint{2.104567in}{1.464015in}}%
\pgfpathlineto{\pgfqpoint{2.104949in}{1.563318in}}%
\pgfpathlineto{\pgfqpoint{2.106098in}{1.500600in}}%
\pgfpathlineto{\pgfqpoint{2.106481in}{1.500600in}}%
\pgfpathlineto{\pgfqpoint{2.107246in}{1.427429in}}%
\pgfpathlineto{\pgfqpoint{2.108012in}{1.568544in}}%
\pgfpathlineto{\pgfqpoint{2.108395in}{1.568544in}}%
\pgfpathlineto{\pgfqpoint{2.109543in}{1.615583in}}%
\pgfpathlineto{\pgfqpoint{2.109161in}{1.474468in}}%
\pgfpathlineto{\pgfqpoint{2.109926in}{1.516280in}}%
\pgfpathlineto{\pgfqpoint{2.110309in}{1.516280in}}%
\pgfpathlineto{\pgfqpoint{2.110309in}{1.385617in}}%
\pgfpathlineto{\pgfqpoint{2.111840in}{1.490147in}}%
\pgfpathlineto{\pgfqpoint{2.112223in}{1.490147in}}%
\pgfpathlineto{\pgfqpoint{2.112223in}{1.385617in}}%
\pgfpathlineto{\pgfqpoint{2.112606in}{1.573771in}}%
\pgfpathlineto{\pgfqpoint{2.113755in}{1.411750in}}%
\pgfpathlineto{\pgfqpoint{2.114137in}{1.411750in}}%
\pgfpathlineto{\pgfqpoint{2.115286in}{1.317673in}}%
\pgfpathlineto{\pgfqpoint{2.115669in}{1.626036in}}%
\pgfpathlineto{\pgfqpoint{2.116052in}{1.626036in}}%
\pgfpathlineto{\pgfqpoint{2.116052in}{1.636489in}}%
\pgfpathlineto{\pgfqpoint{2.116817in}{1.453562in}}%
\pgfpathlineto{\pgfqpoint{2.117583in}{1.458788in}}%
\pgfpathlineto{\pgfqpoint{2.117966in}{1.458788in}}%
\pgfpathlineto{\pgfqpoint{2.118348in}{1.578997in}}%
\pgfpathlineto{\pgfqpoint{2.119497in}{1.573771in}}%
\pgfpathlineto{\pgfqpoint{2.119880in}{1.573771in}}%
\pgfpathlineto{\pgfqpoint{2.121028in}{1.464015in}}%
\pgfpathlineto{\pgfqpoint{2.121411in}{1.610356in}}%
\pgfpathlineto{\pgfqpoint{2.121794in}{1.610356in}}%
\pgfpathlineto{\pgfqpoint{2.122177in}{1.469241in}}%
\pgfpathlineto{\pgfqpoint{2.123325in}{1.521506in}}%
\pgfpathlineto{\pgfqpoint{2.123708in}{1.521506in}}%
\pgfpathlineto{\pgfqpoint{2.123708in}{1.526732in}}%
\pgfpathlineto{\pgfqpoint{2.125239in}{1.375165in}}%
\pgfpathlineto{\pgfqpoint{2.125622in}{1.375165in}}%
\pgfpathlineto{\pgfqpoint{2.126005in}{1.537185in}}%
\pgfpathlineto{\pgfqpoint{2.126388in}{1.349032in}}%
\pgfpathlineto{\pgfqpoint{2.127154in}{1.437882in}}%
\pgfpathlineto{\pgfqpoint{2.127536in}{1.437882in}}%
\pgfpathlineto{\pgfqpoint{2.128685in}{1.573771in}}%
\pgfpathlineto{\pgfqpoint{2.128302in}{1.307220in}}%
\pgfpathlineto{\pgfqpoint{2.129068in}{1.317673in}}%
\pgfpathlineto{\pgfqpoint{2.129451in}{1.317673in}}%
\pgfpathlineto{\pgfqpoint{2.129833in}{1.464015in}}%
\pgfpathlineto{\pgfqpoint{2.130982in}{1.390844in}}%
\pgfpathlineto{\pgfqpoint{2.131365in}{1.390844in}}%
\pgfpathlineto{\pgfqpoint{2.132513in}{1.349032in}}%
\pgfpathlineto{\pgfqpoint{2.132896in}{1.542412in}}%
\pgfpathlineto{\pgfqpoint{2.133279in}{1.542412in}}%
\pgfpathlineto{\pgfqpoint{2.134045in}{1.343806in}}%
\pgfpathlineto{\pgfqpoint{2.134810in}{1.432656in}}%
\pgfpathlineto{\pgfqpoint{2.135193in}{1.432656in}}%
\pgfpathlineto{\pgfqpoint{2.136342in}{1.328126in}}%
\pgfpathlineto{\pgfqpoint{2.136724in}{1.453562in}}%
\pgfpathlineto{\pgfqpoint{2.137107in}{1.453562in}}%
\pgfpathlineto{\pgfqpoint{2.137107in}{1.354259in}}%
\pgfpathlineto{\pgfqpoint{2.138638in}{1.396070in}}%
\pgfpathlineto{\pgfqpoint{2.139021in}{1.396070in}}%
\pgfpathlineto{\pgfqpoint{2.139404in}{1.416976in}}%
\pgfpathlineto{\pgfqpoint{2.139787in}{1.312447in}}%
\pgfpathlineto{\pgfqpoint{2.140553in}{1.322900in}}%
\pgfpathlineto{\pgfqpoint{2.140935in}{1.322900in}}%
\pgfpathlineto{\pgfqpoint{2.141318in}{1.307220in}}%
\pgfpathlineto{\pgfqpoint{2.142467in}{1.474468in}}%
\pgfpathlineto{\pgfqpoint{2.142850in}{1.474468in}}%
\pgfpathlineto{\pgfqpoint{2.143232in}{1.375165in}}%
\pgfpathlineto{\pgfqpoint{2.144381in}{1.375165in}}%
\pgfpathlineto{\pgfqpoint{2.144764in}{1.375165in}}%
\pgfpathlineto{\pgfqpoint{2.145529in}{1.354259in}}%
\pgfpathlineto{\pgfqpoint{2.145147in}{1.443109in}}%
\pgfpathlineto{\pgfqpoint{2.146295in}{1.354259in}}%
\pgfpathlineto{\pgfqpoint{2.146678in}{1.354259in}}%
\pgfpathlineto{\pgfqpoint{2.147444in}{1.500600in}}%
\pgfpathlineto{\pgfqpoint{2.148209in}{1.453562in}}%
\pgfpathlineto{\pgfqpoint{2.148592in}{1.453562in}}%
\pgfpathlineto{\pgfqpoint{2.150123in}{1.307220in}}%
\pgfpathlineto{\pgfqpoint{2.150506in}{1.307220in}}%
\pgfpathlineto{\pgfqpoint{2.150889in}{1.427429in}}%
\pgfpathlineto{\pgfqpoint{2.152038in}{1.343806in}}%
\pgfpathlineto{\pgfqpoint{2.152420in}{1.343806in}}%
\pgfpathlineto{\pgfqpoint{2.152420in}{1.281088in}}%
\pgfpathlineto{\pgfqpoint{2.152803in}{1.448335in}}%
\pgfpathlineto{\pgfqpoint{2.153952in}{1.359485in}}%
\pgfpathlineto{\pgfqpoint{2.154335in}{1.359485in}}%
\pgfpathlineto{\pgfqpoint{2.154335in}{1.458788in}}%
\pgfpathlineto{\pgfqpoint{2.155866in}{1.260182in}}%
\pgfpathlineto{\pgfqpoint{2.156249in}{1.260182in}}%
\pgfpathlineto{\pgfqpoint{2.156249in}{1.239276in}}%
\pgfpathlineto{\pgfqpoint{2.156631in}{1.401297in}}%
\pgfpathlineto{\pgfqpoint{2.157780in}{1.317673in}}%
\pgfpathlineto{\pgfqpoint{2.158163in}{1.317673in}}%
\pgfpathlineto{\pgfqpoint{2.158163in}{1.249729in}}%
\pgfpathlineto{\pgfqpoint{2.158928in}{1.406523in}}%
\pgfpathlineto{\pgfqpoint{2.159694in}{1.291541in}}%
\pgfpathlineto{\pgfqpoint{2.160077in}{1.291541in}}%
\pgfpathlineto{\pgfqpoint{2.160460in}{1.375165in}}%
\pgfpathlineto{\pgfqpoint{2.161225in}{1.249729in}}%
\pgfpathlineto{\pgfqpoint{2.161608in}{1.322900in}}%
\pgfpathlineto{\pgfqpoint{2.161991in}{1.322900in}}%
\pgfpathlineto{\pgfqpoint{2.161991in}{1.390844in}}%
\pgfpathlineto{\pgfqpoint{2.163522in}{1.390844in}}%
\pgfpathlineto{\pgfqpoint{2.163905in}{1.390844in}}%
\pgfpathlineto{\pgfqpoint{2.164671in}{1.223597in}}%
\pgfpathlineto{\pgfqpoint{2.165437in}{1.281088in}}%
\pgfpathlineto{\pgfqpoint{2.165819in}{1.281088in}}%
\pgfpathlineto{\pgfqpoint{2.165819in}{1.244503in}}%
\pgfpathlineto{\pgfqpoint{2.166968in}{1.312447in}}%
\pgfpathlineto{\pgfqpoint{2.167351in}{1.296767in}}%
\pgfpathlineto{\pgfqpoint{2.167734in}{1.296767in}}%
\pgfpathlineto{\pgfqpoint{2.168116in}{1.218370in}}%
\pgfpathlineto{\pgfqpoint{2.168499in}{1.359485in}}%
\pgfpathlineto{\pgfqpoint{2.169265in}{1.322900in}}%
\pgfpathlineto{\pgfqpoint{2.169648in}{1.322900in}}%
\pgfpathlineto{\pgfqpoint{2.170796in}{1.286314in}}%
\pgfpathlineto{\pgfqpoint{2.170031in}{1.369938in}}%
\pgfpathlineto{\pgfqpoint{2.171179in}{1.322900in}}%
\pgfpathlineto{\pgfqpoint{2.171562in}{1.322900in}}%
\pgfpathlineto{\pgfqpoint{2.172328in}{1.406523in}}%
\pgfpathlineto{\pgfqpoint{2.172710in}{1.176558in}}%
\pgfpathlineto{\pgfqpoint{2.173093in}{1.301994in}}%
\pgfpathlineto{\pgfqpoint{2.173476in}{1.301994in}}%
\pgfpathlineto{\pgfqpoint{2.174625in}{1.270635in}}%
\pgfpathlineto{\pgfqpoint{2.175007in}{1.328126in}}%
\pgfpathlineto{\pgfqpoint{2.175390in}{1.328126in}}%
\pgfpathlineto{\pgfqpoint{2.176921in}{1.234050in}}%
\pgfpathlineto{\pgfqpoint{2.177304in}{1.234050in}}%
\pgfpathlineto{\pgfqpoint{2.177687in}{1.317673in}}%
\pgfpathlineto{\pgfqpoint{2.178836in}{1.223597in}}%
\pgfpathlineto{\pgfqpoint{2.179218in}{1.223597in}}%
\pgfpathlineto{\pgfqpoint{2.179218in}{1.187011in}}%
\pgfpathlineto{\pgfqpoint{2.179984in}{1.338579in}}%
\pgfpathlineto{\pgfqpoint{2.180750in}{1.275861in}}%
\pgfpathlineto{\pgfqpoint{2.181133in}{1.275861in}}%
\pgfpathlineto{\pgfqpoint{2.181515in}{1.364712in}}%
\pgfpathlineto{\pgfqpoint{2.181515in}{1.197464in}}%
\pgfpathlineto{\pgfqpoint{2.182664in}{1.228823in}}%
\pgfpathlineto{\pgfqpoint{2.183047in}{1.228823in}}%
\pgfpathlineto{\pgfqpoint{2.183047in}{1.181785in}}%
\pgfpathlineto{\pgfqpoint{2.184195in}{1.333353in}}%
\pgfpathlineto{\pgfqpoint{2.184578in}{1.213144in}}%
\pgfpathlineto{\pgfqpoint{2.184961in}{1.213144in}}%
\pgfpathlineto{\pgfqpoint{2.184961in}{1.317673in}}%
\pgfpathlineto{\pgfqpoint{2.185727in}{1.197464in}}%
\pgfpathlineto{\pgfqpoint{2.186492in}{1.260182in}}%
\pgfpathlineto{\pgfqpoint{2.186875in}{1.260182in}}%
\pgfpathlineto{\pgfqpoint{2.186875in}{1.098161in}}%
\pgfpathlineto{\pgfqpoint{2.188406in}{1.202691in}}%
\pgfpathlineto{\pgfqpoint{2.188789in}{1.202691in}}%
\pgfpathlineto{\pgfqpoint{2.188789in}{1.317673in}}%
\pgfpathlineto{\pgfqpoint{2.189938in}{1.129520in}}%
\pgfpathlineto{\pgfqpoint{2.190321in}{1.260182in}}%
\pgfpathlineto{\pgfqpoint{2.190703in}{1.260182in}}%
\pgfpathlineto{\pgfqpoint{2.191086in}{1.145199in}}%
\pgfpathlineto{\pgfqpoint{2.192235in}{1.171332in}}%
\pgfpathlineto{\pgfqpoint{2.192618in}{1.171332in}}%
\pgfpathlineto{\pgfqpoint{2.193000in}{1.286314in}}%
\pgfpathlineto{\pgfqpoint{2.193766in}{1.166105in}}%
\pgfpathlineto{\pgfqpoint{2.194149in}{1.223597in}}%
\pgfpathlineto{\pgfqpoint{2.194532in}{1.223597in}}%
\pgfpathlineto{\pgfqpoint{2.195297in}{1.270635in}}%
\pgfpathlineto{\pgfqpoint{2.196063in}{1.270635in}}%
\pgfpathlineto{\pgfqpoint{2.196446in}{1.270635in}}%
\pgfpathlineto{\pgfqpoint{2.196829in}{1.150426in}}%
\pgfpathlineto{\pgfqpoint{2.197977in}{1.307220in}}%
\pgfpathlineto{\pgfqpoint{2.198360in}{1.307220in}}%
\pgfpathlineto{\pgfqpoint{2.199126in}{1.134746in}}%
\pgfpathlineto{\pgfqpoint{2.199891in}{1.187011in}}%
\pgfpathlineto{\pgfqpoint{2.200274in}{1.187011in}}%
\pgfpathlineto{\pgfqpoint{2.200274in}{1.312447in}}%
\pgfpathlineto{\pgfqpoint{2.201805in}{1.244503in}}%
\pgfpathlineto{\pgfqpoint{2.202188in}{1.244503in}}%
\pgfpathlineto{\pgfqpoint{2.203337in}{1.066802in}}%
\pgfpathlineto{\pgfqpoint{2.203720in}{1.187011in}}%
\pgfpathlineto{\pgfqpoint{2.204102in}{1.187011in}}%
\pgfpathlineto{\pgfqpoint{2.204868in}{1.270635in}}%
\pgfpathlineto{\pgfqpoint{2.205251in}{1.134746in}}%
\pgfpathlineto{\pgfqpoint{2.205634in}{1.176558in}}%
\pgfpathlineto{\pgfqpoint{2.206017in}{1.176558in}}%
\pgfpathlineto{\pgfqpoint{2.207165in}{1.228823in}}%
\pgfpathlineto{\pgfqpoint{2.207548in}{1.129520in}}%
\pgfpathlineto{\pgfqpoint{2.207931in}{1.129520in}}%
\pgfpathlineto{\pgfqpoint{2.209079in}{1.307220in}}%
\pgfpathlineto{\pgfqpoint{2.209462in}{1.145199in}}%
\pgfpathlineto{\pgfqpoint{2.209845in}{1.145199in}}%
\pgfpathlineto{\pgfqpoint{2.209845in}{1.113840in}}%
\pgfpathlineto{\pgfqpoint{2.210611in}{1.207917in}}%
\pgfpathlineto{\pgfqpoint{2.211376in}{1.150426in}}%
\pgfpathlineto{\pgfqpoint{2.211759in}{1.150426in}}%
\pgfpathlineto{\pgfqpoint{2.212525in}{1.072029in}}%
\pgfpathlineto{\pgfqpoint{2.212142in}{1.202691in}}%
\pgfpathlineto{\pgfqpoint{2.213290in}{1.176558in}}%
\pgfpathlineto{\pgfqpoint{2.213673in}{1.176558in}}%
\pgfpathlineto{\pgfqpoint{2.214056in}{1.249729in}}%
\pgfpathlineto{\pgfqpoint{2.214822in}{1.124293in}}%
\pgfpathlineto{\pgfqpoint{2.215204in}{1.187011in}}%
\pgfpathlineto{\pgfqpoint{2.215587in}{1.187011in}}%
\pgfpathlineto{\pgfqpoint{2.215587in}{1.197464in}}%
\pgfpathlineto{\pgfqpoint{2.216736in}{1.035443in}}%
\pgfpathlineto{\pgfqpoint{2.217119in}{1.072029in}}%
\pgfpathlineto{\pgfqpoint{2.217501in}{1.072029in}}%
\pgfpathlineto{\pgfqpoint{2.217501in}{1.166105in}}%
\pgfpathlineto{\pgfqpoint{2.219033in}{1.150426in}}%
\pgfpathlineto{\pgfqpoint{2.219416in}{1.150426in}}%
\pgfpathlineto{\pgfqpoint{2.220181in}{1.082482in}}%
\pgfpathlineto{\pgfqpoint{2.220947in}{1.171332in}}%
\pgfpathlineto{\pgfqpoint{2.221330in}{1.171332in}}%
\pgfpathlineto{\pgfqpoint{2.221713in}{1.019764in}}%
\pgfpathlineto{\pgfqpoint{2.222861in}{1.072029in}}%
\pgfpathlineto{\pgfqpoint{2.223244in}{1.072029in}}%
\pgfpathlineto{\pgfqpoint{2.223627in}{1.176558in}}%
\pgfpathlineto{\pgfqpoint{2.224775in}{1.145199in}}%
\pgfpathlineto{\pgfqpoint{2.225158in}{1.145199in}}%
\pgfpathlineto{\pgfqpoint{2.225924in}{1.197464in}}%
\pgfpathlineto{\pgfqpoint{2.226689in}{1.082482in}}%
\pgfpathlineto{\pgfqpoint{2.227072in}{1.082482in}}%
\pgfpathlineto{\pgfqpoint{2.227072in}{1.166105in}}%
\pgfpathlineto{\pgfqpoint{2.227455in}{1.019764in}}%
\pgfpathlineto{\pgfqpoint{2.228604in}{1.056349in}}%
\pgfpathlineto{\pgfqpoint{2.228986in}{1.056349in}}%
\pgfpathlineto{\pgfqpoint{2.230135in}{1.187011in}}%
\pgfpathlineto{\pgfqpoint{2.230518in}{1.056349in}}%
\pgfpathlineto{\pgfqpoint{2.230901in}{1.056349in}}%
\pgfpathlineto{\pgfqpoint{2.231666in}{1.197464in}}%
\pgfpathlineto{\pgfqpoint{2.232049in}{1.024990in}}%
\pgfpathlineto{\pgfqpoint{2.232432in}{1.056349in}}%
\pgfpathlineto{\pgfqpoint{2.232815in}{1.056349in}}%
\pgfpathlineto{\pgfqpoint{2.233580in}{1.166105in}}%
\pgfpathlineto{\pgfqpoint{2.233963in}{1.014537in}}%
\pgfpathlineto{\pgfqpoint{2.234346in}{1.145199in}}%
\pgfpathlineto{\pgfqpoint{2.234729in}{1.145199in}}%
\pgfpathlineto{\pgfqpoint{2.236260in}{1.066802in}}%
\pgfpathlineto{\pgfqpoint{2.236643in}{1.066802in}}%
\pgfpathlineto{\pgfqpoint{2.236643in}{1.166105in}}%
\pgfpathlineto{\pgfqpoint{2.237026in}{0.988405in}}%
\pgfpathlineto{\pgfqpoint{2.238174in}{1.108614in}}%
\pgfpathlineto{\pgfqpoint{2.238557in}{1.108614in}}%
\pgfpathlineto{\pgfqpoint{2.238940in}{1.061576in}}%
\pgfpathlineto{\pgfqpoint{2.239323in}{1.155652in}}%
\pgfpathlineto{\pgfqpoint{2.240088in}{1.103388in}}%
\pgfpathlineto{\pgfqpoint{2.240471in}{1.103388in}}%
\pgfpathlineto{\pgfqpoint{2.240854in}{1.197464in}}%
\pgfpathlineto{\pgfqpoint{2.241620in}{0.977952in}}%
\pgfpathlineto{\pgfqpoint{2.242003in}{1.134746in}}%
\pgfpathlineto{\pgfqpoint{2.242768in}{1.134746in}}%
\pgfpathlineto{\pgfqpoint{2.243917in}{1.035443in}}%
\pgfpathlineto{\pgfqpoint{2.244300in}{1.124293in}}%
\pgfpathlineto{\pgfqpoint{2.244682in}{1.124293in}}%
\pgfpathlineto{\pgfqpoint{2.246214in}{1.004084in}}%
\pgfpathlineto{\pgfqpoint{2.246597in}{1.004084in}}%
\pgfpathlineto{\pgfqpoint{2.246597in}{1.061576in}}%
\pgfpathlineto{\pgfqpoint{2.248128in}{1.035443in}}%
\pgfpathlineto{\pgfqpoint{2.248511in}{1.035443in}}%
\pgfpathlineto{\pgfqpoint{2.249276in}{1.160879in}}%
\pgfpathlineto{\pgfqpoint{2.249659in}{0.977952in}}%
\pgfpathlineto{\pgfqpoint{2.250042in}{1.051123in}}%
\pgfpathlineto{\pgfqpoint{2.250425in}{1.051123in}}%
\pgfpathlineto{\pgfqpoint{2.250808in}{1.124293in}}%
\pgfpathlineto{\pgfqpoint{2.251956in}{1.087708in}}%
\pgfpathlineto{\pgfqpoint{2.252339in}{1.087708in}}%
\pgfpathlineto{\pgfqpoint{2.252722in}{1.160879in}}%
\pgfpathlineto{\pgfqpoint{2.253487in}{1.056349in}}%
\pgfpathlineto{\pgfqpoint{2.253870in}{1.087708in}}%
\pgfpathlineto{\pgfqpoint{2.254253in}{1.087708in}}%
\pgfpathlineto{\pgfqpoint{2.254253in}{0.993631in}}%
\pgfpathlineto{\pgfqpoint{2.255402in}{1.108614in}}%
\pgfpathlineto{\pgfqpoint{2.255784in}{1.014537in}}%
\pgfpathlineto{\pgfqpoint{2.256167in}{1.014537in}}%
\pgfpathlineto{\pgfqpoint{2.256167in}{1.004084in}}%
\pgfpathlineto{\pgfqpoint{2.256550in}{1.087708in}}%
\pgfpathlineto{\pgfqpoint{2.257699in}{1.066802in}}%
\pgfpathlineto{\pgfqpoint{2.258081in}{1.066802in}}%
\pgfpathlineto{\pgfqpoint{2.258081in}{1.119067in}}%
\pgfpathlineto{\pgfqpoint{2.259613in}{0.951820in}}%
\pgfpathlineto{\pgfqpoint{2.259996in}{0.951820in}}%
\pgfpathlineto{\pgfqpoint{2.260761in}{1.061576in}}%
\pgfpathlineto{\pgfqpoint{2.261527in}{1.056349in}}%
\pgfpathlineto{\pgfqpoint{2.261910in}{1.056349in}}%
\pgfpathlineto{\pgfqpoint{2.263058in}{0.962273in}}%
\pgfpathlineto{\pgfqpoint{2.263441in}{0.967499in}}%
\pgfpathlineto{\pgfqpoint{2.263824in}{0.967499in}}%
\pgfpathlineto{\pgfqpoint{2.264972in}{1.072029in}}%
\pgfpathlineto{\pgfqpoint{2.264207in}{0.936140in}}%
\pgfpathlineto{\pgfqpoint{2.265355in}{1.019764in}}%
\pgfpathlineto{\pgfqpoint{2.265738in}{1.019764in}}%
\pgfpathlineto{\pgfqpoint{2.266504in}{0.962273in}}%
\pgfpathlineto{\pgfqpoint{2.267269in}{1.040670in}}%
\pgfpathlineto{\pgfqpoint{2.267652in}{1.040670in}}%
\pgfpathlineto{\pgfqpoint{2.268035in}{0.951820in}}%
\pgfpathlineto{\pgfqpoint{2.269184in}{0.951820in}}%
\pgfpathlineto{\pgfqpoint{2.269566in}{0.951820in}}%
\pgfpathlineto{\pgfqpoint{2.269566in}{0.904781in}}%
\pgfpathlineto{\pgfqpoint{2.270715in}{1.040670in}}%
\pgfpathlineto{\pgfqpoint{2.271098in}{1.019764in}}%
\pgfpathlineto{\pgfqpoint{2.271481in}{1.019764in}}%
\pgfpathlineto{\pgfqpoint{2.271481in}{1.045896in}}%
\pgfpathlineto{\pgfqpoint{2.273012in}{0.936140in}}%
\pgfpathlineto{\pgfqpoint{2.273395in}{0.936140in}}%
\pgfpathlineto{\pgfqpoint{2.274926in}{1.035443in}}%
\pgfpathlineto{\pgfqpoint{2.275309in}{1.035443in}}%
\pgfpathlineto{\pgfqpoint{2.276457in}{0.941367in}}%
\pgfpathlineto{\pgfqpoint{2.275692in}{1.045896in}}%
\pgfpathlineto{\pgfqpoint{2.276840in}{0.983178in}}%
\pgfpathlineto{\pgfqpoint{2.277223in}{0.983178in}}%
\pgfpathlineto{\pgfqpoint{2.277223in}{1.004084in}}%
\pgfpathlineto{\pgfqpoint{2.278754in}{0.941367in}}%
\pgfpathlineto{\pgfqpoint{2.279137in}{0.941367in}}%
\pgfpathlineto{\pgfqpoint{2.279137in}{0.920461in}}%
\pgfpathlineto{\pgfqpoint{2.280668in}{0.983178in}}%
\pgfpathlineto{\pgfqpoint{2.281051in}{0.983178in}}%
\pgfpathlineto{\pgfqpoint{2.281051in}{0.936140in}}%
\pgfpathlineto{\pgfqpoint{2.281817in}{1.004084in}}%
\pgfpathlineto{\pgfqpoint{2.282583in}{0.998858in}}%
\pgfpathlineto{\pgfqpoint{2.282965in}{0.998858in}}%
\pgfpathlineto{\pgfqpoint{2.282965in}{0.962273in}}%
\pgfpathlineto{\pgfqpoint{2.283348in}{1.009311in}}%
\pgfpathlineto{\pgfqpoint{2.284497in}{0.962273in}}%
\pgfpathlineto{\pgfqpoint{2.284880in}{0.962273in}}%
\pgfpathlineto{\pgfqpoint{2.285645in}{0.930914in}}%
\pgfpathlineto{\pgfqpoint{2.285262in}{1.056349in}}%
\pgfpathlineto{\pgfqpoint{2.286411in}{0.983178in}}%
\pgfpathlineto{\pgfqpoint{2.286794in}{0.983178in}}%
\pgfpathlineto{\pgfqpoint{2.287559in}{1.051123in}}%
\pgfpathlineto{\pgfqpoint{2.288325in}{0.889102in}}%
\pgfpathlineto{\pgfqpoint{2.288708in}{0.889102in}}%
\pgfpathlineto{\pgfqpoint{2.289856in}{0.983178in}}%
\pgfpathlineto{\pgfqpoint{2.290239in}{0.951820in}}%
\pgfpathlineto{\pgfqpoint{2.290622in}{0.951820in}}%
\pgfpathlineto{\pgfqpoint{2.291770in}{0.988405in}}%
\pgfpathlineto{\pgfqpoint{2.292153in}{0.883875in}}%
\pgfpathlineto{\pgfqpoint{2.292536in}{0.883875in}}%
\pgfpathlineto{\pgfqpoint{2.293302in}{0.993631in}}%
\pgfpathlineto{\pgfqpoint{2.294067in}{0.930914in}}%
\pgfpathlineto{\pgfqpoint{2.295216in}{0.930914in}}%
\pgfpathlineto{\pgfqpoint{2.296364in}{1.014537in}}%
\pgfpathlineto{\pgfqpoint{2.295982in}{0.836837in}}%
\pgfpathlineto{\pgfqpoint{2.296747in}{0.930914in}}%
\pgfpathlineto{\pgfqpoint{2.297130in}{0.930914in}}%
\pgfpathlineto{\pgfqpoint{2.297896in}{0.967499in}}%
\pgfpathlineto{\pgfqpoint{2.298661in}{0.915234in}}%
\pgfpathlineto{\pgfqpoint{2.299044in}{0.915234in}}%
\pgfpathlineto{\pgfqpoint{2.300193in}{0.962273in}}%
\pgfpathlineto{\pgfqpoint{2.299427in}{0.862969in}}%
\pgfpathlineto{\pgfqpoint{2.300576in}{0.936140in}}%
\pgfpathlineto{\pgfqpoint{2.300958in}{0.936140in}}%
\pgfpathlineto{\pgfqpoint{2.300958in}{0.993631in}}%
\pgfpathlineto{\pgfqpoint{2.302490in}{0.920461in}}%
\pgfpathlineto{\pgfqpoint{2.302873in}{0.920461in}}%
\pgfpathlineto{\pgfqpoint{2.303255in}{1.019764in}}%
\pgfpathlineto{\pgfqpoint{2.303638in}{0.915234in}}%
\pgfpathlineto{\pgfqpoint{2.304404in}{0.936140in}}%
\pgfpathlineto{\pgfqpoint{2.304787in}{0.936140in}}%
\pgfpathlineto{\pgfqpoint{2.305935in}{0.946593in}}%
\pgfpathlineto{\pgfqpoint{2.306318in}{0.899555in}}%
\pgfpathlineto{\pgfqpoint{2.306701in}{0.899555in}}%
\pgfpathlineto{\pgfqpoint{2.307084in}{0.862969in}}%
\pgfpathlineto{\pgfqpoint{2.307467in}{0.930914in}}%
\pgfpathlineto{\pgfqpoint{2.308232in}{0.894328in}}%
\pgfpathlineto{\pgfqpoint{2.308615in}{0.894328in}}%
\pgfpathlineto{\pgfqpoint{2.309764in}{0.842063in}}%
\pgfpathlineto{\pgfqpoint{2.309381in}{0.915234in}}%
\pgfpathlineto{\pgfqpoint{2.310146in}{0.894328in}}%
\pgfpathlineto{\pgfqpoint{2.310529in}{0.894328in}}%
\pgfpathlineto{\pgfqpoint{2.310529in}{0.883875in}}%
\pgfpathlineto{\pgfqpoint{2.312060in}{0.941367in}}%
\pgfpathlineto{\pgfqpoint{2.312443in}{0.941367in}}%
\pgfpathlineto{\pgfqpoint{2.313592in}{0.993631in}}%
\pgfpathlineto{\pgfqpoint{2.313975in}{0.873422in}}%
\pgfpathlineto{\pgfqpoint{2.314357in}{0.873422in}}%
\pgfpathlineto{\pgfqpoint{2.314357in}{0.862969in}}%
\pgfpathlineto{\pgfqpoint{2.315123in}{0.946593in}}%
\pgfpathlineto{\pgfqpoint{2.315889in}{0.920461in}}%
\pgfpathlineto{\pgfqpoint{2.316272in}{0.920461in}}%
\pgfpathlineto{\pgfqpoint{2.317037in}{0.800252in}}%
\pgfpathlineto{\pgfqpoint{2.317420in}{0.962273in}}%
\pgfpathlineto{\pgfqpoint{2.317803in}{0.930914in}}%
\pgfpathlineto{\pgfqpoint{2.318186in}{0.930914in}}%
\pgfpathlineto{\pgfqpoint{2.319334in}{0.857743in}}%
\pgfpathlineto{\pgfqpoint{2.318951in}{0.967499in}}%
\pgfpathlineto{\pgfqpoint{2.319717in}{0.878649in}}%
\pgfpathlineto{\pgfqpoint{2.320100in}{0.878649in}}%
\pgfpathlineto{\pgfqpoint{2.321248in}{0.946593in}}%
\pgfpathlineto{\pgfqpoint{2.321631in}{0.899555in}}%
\pgfpathlineto{\pgfqpoint{2.322014in}{0.899555in}}%
\pgfpathlineto{\pgfqpoint{2.322014in}{0.857743in}}%
\pgfpathlineto{\pgfqpoint{2.323545in}{0.910008in}}%
\pgfpathlineto{\pgfqpoint{2.323928in}{0.910008in}}%
\pgfpathlineto{\pgfqpoint{2.324694in}{0.852516in}}%
\pgfpathlineto{\pgfqpoint{2.325460in}{0.910008in}}%
\pgfpathlineto{\pgfqpoint{2.325842in}{0.910008in}}%
\pgfpathlineto{\pgfqpoint{2.325842in}{0.821158in}}%
\pgfpathlineto{\pgfqpoint{2.327374in}{0.836837in}}%
\pgfpathlineto{\pgfqpoint{2.327757in}{0.836837in}}%
\pgfpathlineto{\pgfqpoint{2.329288in}{0.930914in}}%
\pgfpathlineto{\pgfqpoint{2.329671in}{0.930914in}}%
\pgfpathlineto{\pgfqpoint{2.331202in}{0.826384in}}%
\pgfpathlineto{\pgfqpoint{2.331585in}{0.826384in}}%
\pgfpathlineto{\pgfqpoint{2.332350in}{0.805478in}}%
\pgfpathlineto{\pgfqpoint{2.332733in}{0.883875in}}%
\pgfpathlineto{\pgfqpoint{2.333116in}{0.842063in}}%
\pgfpathlineto{\pgfqpoint{2.333499in}{0.842063in}}%
\pgfpathlineto{\pgfqpoint{2.334265in}{0.826384in}}%
\pgfpathlineto{\pgfqpoint{2.335030in}{0.862969in}}%
\pgfpathlineto{\pgfqpoint{2.335413in}{0.862969in}}%
\pgfpathlineto{\pgfqpoint{2.335413in}{0.800252in}}%
\pgfpathlineto{\pgfqpoint{2.336562in}{0.894328in}}%
\pgfpathlineto{\pgfqpoint{2.336944in}{0.842063in}}%
\pgfpathlineto{\pgfqpoint{2.337327in}{0.842063in}}%
\pgfpathlineto{\pgfqpoint{2.338476in}{0.805478in}}%
\pgfpathlineto{\pgfqpoint{2.338859in}{0.862969in}}%
\pgfpathlineto{\pgfqpoint{2.339241in}{0.862969in}}%
\pgfpathlineto{\pgfqpoint{2.339624in}{0.857743in}}%
\pgfpathlineto{\pgfqpoint{2.340773in}{0.915234in}}%
\pgfpathlineto{\pgfqpoint{2.341156in}{0.915234in}}%
\pgfpathlineto{\pgfqpoint{2.341156in}{0.930914in}}%
\pgfpathlineto{\pgfqpoint{2.342304in}{0.774119in}}%
\pgfpathlineto{\pgfqpoint{2.342687in}{0.795025in}}%
\pgfpathlineto{\pgfqpoint{2.343070in}{0.795025in}}%
\pgfpathlineto{\pgfqpoint{2.343835in}{0.889102in}}%
\pgfpathlineto{\pgfqpoint{2.344601in}{0.815931in}}%
\pgfpathlineto{\pgfqpoint{2.344984in}{0.815931in}}%
\pgfpathlineto{\pgfqpoint{2.345367in}{0.878649in}}%
\pgfpathlineto{\pgfqpoint{2.346515in}{0.862969in}}%
\pgfpathlineto{\pgfqpoint{2.346898in}{0.862969in}}%
\pgfpathlineto{\pgfqpoint{2.348047in}{0.800252in}}%
\pgfpathlineto{\pgfqpoint{2.348429in}{0.800252in}}%
\pgfpathlineto{\pgfqpoint{2.348812in}{0.800252in}}%
\pgfpathlineto{\pgfqpoint{2.349578in}{0.878649in}}%
\pgfpathlineto{\pgfqpoint{2.350343in}{0.805478in}}%
\pgfpathlineto{\pgfqpoint{2.350726in}{0.805478in}}%
\pgfpathlineto{\pgfqpoint{2.350726in}{0.753213in}}%
\pgfpathlineto{\pgfqpoint{2.351109in}{0.836837in}}%
\pgfpathlineto{\pgfqpoint{2.352258in}{0.836837in}}%
\pgfpathlineto{\pgfqpoint{2.352640in}{0.836837in}}%
\pgfpathlineto{\pgfqpoint{2.353789in}{0.795025in}}%
\pgfpathlineto{\pgfqpoint{2.354172in}{0.883875in}}%
\pgfpathlineto{\pgfqpoint{2.354555in}{0.883875in}}%
\pgfpathlineto{\pgfqpoint{2.355320in}{0.805478in}}%
\pgfpathlineto{\pgfqpoint{2.356086in}{0.831611in}}%
\pgfpathlineto{\pgfqpoint{2.356469in}{0.831611in}}%
\pgfpathlineto{\pgfqpoint{2.356852in}{0.857743in}}%
\pgfpathlineto{\pgfqpoint{2.358000in}{0.821158in}}%
\pgfpathlineto{\pgfqpoint{2.358383in}{0.821158in}}%
\pgfpathlineto{\pgfqpoint{2.358766in}{0.784572in}}%
\pgfpathlineto{\pgfqpoint{2.358766in}{0.868196in}}%
\pgfpathlineto{\pgfqpoint{2.359914in}{0.852516in}}%
\pgfpathlineto{\pgfqpoint{2.360297in}{0.852516in}}%
\pgfpathlineto{\pgfqpoint{2.360680in}{0.889102in}}%
\pgfpathlineto{\pgfqpoint{2.361828in}{0.800252in}}%
\pgfpathlineto{\pgfqpoint{2.362211in}{0.800252in}}%
\pgfpathlineto{\pgfqpoint{2.363743in}{0.873422in}}%
\pgfpathlineto{\pgfqpoint{2.364125in}{0.873422in}}%
\pgfpathlineto{\pgfqpoint{2.364508in}{0.747987in}}%
\pgfpathlineto{\pgfqpoint{2.365657in}{0.810705in}}%
\pgfpathlineto{\pgfqpoint{2.366040in}{0.810705in}}%
\pgfpathlineto{\pgfqpoint{2.367188in}{0.831611in}}%
\pgfpathlineto{\pgfqpoint{2.366805in}{0.779346in}}%
\pgfpathlineto{\pgfqpoint{2.367571in}{0.821158in}}%
\pgfpathlineto{\pgfqpoint{2.367954in}{0.821158in}}%
\pgfpathlineto{\pgfqpoint{2.368719in}{0.795025in}}%
\pgfpathlineto{\pgfqpoint{2.369485in}{0.800252in}}%
\pgfpathlineto{\pgfqpoint{2.370251in}{0.800252in}}%
\pgfpathlineto{\pgfqpoint{2.371016in}{0.727081in}}%
\pgfpathlineto{\pgfqpoint{2.370633in}{0.857743in}}%
\pgfpathlineto{\pgfqpoint{2.371782in}{0.774119in}}%
\pgfpathlineto{\pgfqpoint{2.372165in}{0.774119in}}%
\pgfpathlineto{\pgfqpoint{2.372165in}{0.763666in}}%
\pgfpathlineto{\pgfqpoint{2.373313in}{0.847290in}}%
\pgfpathlineto{\pgfqpoint{2.373696in}{0.831611in}}%
\pgfpathlineto{\pgfqpoint{2.374079in}{0.831611in}}%
\pgfpathlineto{\pgfqpoint{2.375227in}{0.842063in}}%
\pgfpathlineto{\pgfqpoint{2.375610in}{0.795025in}}%
\pgfpathlineto{\pgfqpoint{2.375993in}{0.795025in}}%
\pgfpathlineto{\pgfqpoint{2.375993in}{0.774119in}}%
\pgfpathlineto{\pgfqpoint{2.376376in}{0.836837in}}%
\pgfpathlineto{\pgfqpoint{2.377524in}{0.821158in}}%
\pgfpathlineto{\pgfqpoint{2.377907in}{0.821158in}}%
\pgfpathlineto{\pgfqpoint{2.378673in}{0.847290in}}%
\pgfpathlineto{\pgfqpoint{2.379439in}{0.737534in}}%
\pgfpathlineto{\pgfqpoint{2.379821in}{0.737534in}}%
\pgfpathlineto{\pgfqpoint{2.381353in}{0.805478in}}%
\pgfpathlineto{\pgfqpoint{2.381736in}{0.805478in}}%
\pgfpathlineto{\pgfqpoint{2.382118in}{0.742760in}}%
\pgfpathlineto{\pgfqpoint{2.383267in}{0.763666in}}%
\pgfpathlineto{\pgfqpoint{2.383650in}{0.763666in}}%
\pgfpathlineto{\pgfqpoint{2.384415in}{0.821158in}}%
\pgfpathlineto{\pgfqpoint{2.385181in}{0.768893in}}%
\pgfpathlineto{\pgfqpoint{2.385564in}{0.768893in}}%
\pgfpathlineto{\pgfqpoint{2.385564in}{0.800252in}}%
\pgfpathlineto{\pgfqpoint{2.387095in}{0.789799in}}%
\pgfpathlineto{\pgfqpoint{2.387478in}{0.789799in}}%
\pgfpathlineto{\pgfqpoint{2.387478in}{0.815931in}}%
\pgfpathlineto{\pgfqpoint{2.389009in}{0.737534in}}%
\pgfpathlineto{\pgfqpoint{2.389392in}{0.737534in}}%
\pgfpathlineto{\pgfqpoint{2.389775in}{0.721854in}}%
\pgfpathlineto{\pgfqpoint{2.390923in}{0.795025in}}%
\pgfpathlineto{\pgfqpoint{2.391306in}{0.795025in}}%
\pgfpathlineto{\pgfqpoint{2.392072in}{0.716628in}}%
\pgfpathlineto{\pgfqpoint{2.392838in}{0.821158in}}%
\pgfpathlineto{\pgfqpoint{2.393220in}{0.821158in}}%
\pgfpathlineto{\pgfqpoint{2.393603in}{0.742760in}}%
\pgfpathlineto{\pgfqpoint{2.394752in}{0.784572in}}%
\pgfpathlineto{\pgfqpoint{2.395135in}{0.784572in}}%
\pgfpathlineto{\pgfqpoint{2.396283in}{0.716628in}}%
\pgfpathlineto{\pgfqpoint{2.396666in}{0.737534in}}%
\pgfpathlineto{\pgfqpoint{2.397049in}{0.737534in}}%
\pgfpathlineto{\pgfqpoint{2.397049in}{0.800252in}}%
\pgfpathlineto{\pgfqpoint{2.397432in}{0.716628in}}%
\pgfpathlineto{\pgfqpoint{2.398580in}{0.774119in}}%
\pgfpathlineto{\pgfqpoint{2.398963in}{0.774119in}}%
\pgfpathlineto{\pgfqpoint{2.399729in}{0.737534in}}%
\pgfpathlineto{\pgfqpoint{2.400111in}{0.784572in}}%
\pgfpathlineto{\pgfqpoint{2.400494in}{0.784572in}}%
\pgfpathlineto{\pgfqpoint{2.401260in}{0.784572in}}%
\pgfpathlineto{\pgfqpoint{2.402026in}{0.716628in}}%
\pgfpathlineto{\pgfqpoint{2.401643in}{0.789799in}}%
\pgfpathlineto{\pgfqpoint{2.402791in}{0.753213in}}%
\pgfpathlineto{\pgfqpoint{2.403174in}{0.753213in}}%
\pgfpathlineto{\pgfqpoint{2.403174in}{0.706175in}}%
\pgfpathlineto{\pgfqpoint{2.404705in}{0.805478in}}%
\pgfpathlineto{\pgfqpoint{2.405088in}{0.805478in}}%
\pgfpathlineto{\pgfqpoint{2.405471in}{0.706175in}}%
\pgfpathlineto{\pgfqpoint{2.406620in}{0.815931in}}%
\pgfpathlineto{\pgfqpoint{2.407002in}{0.815931in}}%
\pgfpathlineto{\pgfqpoint{2.407768in}{0.711401in}}%
\pgfpathlineto{\pgfqpoint{2.408534in}{0.732307in}}%
\pgfpathlineto{\pgfqpoint{2.408916in}{0.732307in}}%
\pgfpathlineto{\pgfqpoint{2.409682in}{0.716628in}}%
\pgfpathlineto{\pgfqpoint{2.410448in}{0.774119in}}%
\pgfpathlineto{\pgfqpoint{2.410831in}{0.774119in}}%
\pgfpathlineto{\pgfqpoint{2.411596in}{0.721854in}}%
\pgfpathlineto{\pgfqpoint{2.412362in}{0.742760in}}%
\pgfpathlineto{\pgfqpoint{2.412745in}{0.742760in}}%
\pgfpathlineto{\pgfqpoint{2.413128in}{0.716628in}}%
\pgfpathlineto{\pgfqpoint{2.413893in}{0.774119in}}%
\pgfpathlineto{\pgfqpoint{2.414276in}{0.753213in}}%
\pgfpathlineto{\pgfqpoint{2.415042in}{0.753213in}}%
\pgfpathlineto{\pgfqpoint{2.415042in}{0.810705in}}%
\pgfpathlineto{\pgfqpoint{2.416573in}{0.674816in}}%
\pgfpathlineto{\pgfqpoint{2.416956in}{0.674816in}}%
\pgfpathlineto{\pgfqpoint{2.418487in}{0.753213in}}%
\pgfpathlineto{\pgfqpoint{2.419253in}{0.753213in}}%
\pgfpathlineto{\pgfqpoint{2.419253in}{0.711401in}}%
\pgfpathlineto{\pgfqpoint{2.419636in}{0.789799in}}%
\pgfpathlineto{\pgfqpoint{2.420784in}{0.721854in}}%
\pgfpathlineto{\pgfqpoint{2.421167in}{0.721854in}}%
\pgfpathlineto{\pgfqpoint{2.421550in}{0.695722in}}%
\pgfpathlineto{\pgfqpoint{2.422698in}{0.774119in}}%
\pgfpathlineto{\pgfqpoint{2.423081in}{0.774119in}}%
\pgfpathlineto{\pgfqpoint{2.423847in}{0.721854in}}%
\pgfpathlineto{\pgfqpoint{2.424613in}{0.727081in}}%
\pgfpathlineto{\pgfqpoint{2.424995in}{0.727081in}}%
\pgfpathlineto{\pgfqpoint{2.426144in}{0.680043in}}%
\pgfpathlineto{\pgfqpoint{2.425378in}{0.732307in}}%
\pgfpathlineto{\pgfqpoint{2.426527in}{0.700949in}}%
\pgfpathlineto{\pgfqpoint{2.426909in}{0.700949in}}%
\pgfpathlineto{\pgfqpoint{2.426909in}{0.758440in}}%
\pgfpathlineto{\pgfqpoint{2.427292in}{0.674816in}}%
\pgfpathlineto{\pgfqpoint{2.428441in}{0.711401in}}%
\pgfpathlineto{\pgfqpoint{2.428824in}{0.711401in}}%
\pgfpathlineto{\pgfqpoint{2.428824in}{0.669590in}}%
\pgfpathlineto{\pgfqpoint{2.430355in}{0.727081in}}%
\pgfpathlineto{\pgfqpoint{2.430738in}{0.727081in}}%
\pgfpathlineto{\pgfqpoint{2.430738in}{0.747987in}}%
\pgfpathlineto{\pgfqpoint{2.432269in}{0.664363in}}%
\pgfpathlineto{\pgfqpoint{2.432652in}{0.664363in}}%
\pgfpathlineto{\pgfqpoint{2.433800in}{0.747987in}}%
\pgfpathlineto{\pgfqpoint{2.434183in}{0.721854in}}%
\pgfpathlineto{\pgfqpoint{2.434566in}{0.721854in}}%
\pgfpathlineto{\pgfqpoint{2.434566in}{0.789799in}}%
\pgfpathlineto{\pgfqpoint{2.435332in}{0.700949in}}%
\pgfpathlineto{\pgfqpoint{2.436097in}{0.742760in}}%
\pgfpathlineto{\pgfqpoint{2.436480in}{0.742760in}}%
\pgfpathlineto{\pgfqpoint{2.437629in}{0.700949in}}%
\pgfpathlineto{\pgfqpoint{2.438012in}{0.716628in}}%
\pgfpathlineto{\pgfqpoint{2.438394in}{0.716628in}}%
\pgfpathlineto{\pgfqpoint{2.438394in}{0.737534in}}%
\pgfpathlineto{\pgfqpoint{2.438777in}{0.690496in}}%
\pgfpathlineto{\pgfqpoint{2.439926in}{0.737534in}}%
\pgfpathlineto{\pgfqpoint{2.440309in}{0.737534in}}%
\pgfpathlineto{\pgfqpoint{2.441457in}{0.680043in}}%
\pgfpathlineto{\pgfqpoint{2.441074in}{0.753213in}}%
\pgfpathlineto{\pgfqpoint{2.441840in}{0.706175in}}%
\pgfpathlineto{\pgfqpoint{2.442223in}{0.706175in}}%
\pgfpathlineto{\pgfqpoint{2.442988in}{0.727081in}}%
\pgfpathlineto{\pgfqpoint{2.443371in}{0.669590in}}%
\pgfpathlineto{\pgfqpoint{2.443754in}{0.685269in}}%
\pgfpathlineto{\pgfqpoint{2.444137in}{0.685269in}}%
\pgfpathlineto{\pgfqpoint{2.444520in}{0.758440in}}%
\pgfpathlineto{\pgfqpoint{2.445668in}{0.711401in}}%
\pgfpathlineto{\pgfqpoint{2.446051in}{0.711401in}}%
\pgfpathlineto{\pgfqpoint{2.446051in}{0.664363in}}%
\pgfpathlineto{\pgfqpoint{2.446434in}{0.737534in}}%
\pgfpathlineto{\pgfqpoint{2.447582in}{0.690496in}}%
\pgfpathlineto{\pgfqpoint{2.447965in}{0.690496in}}%
\pgfpathlineto{\pgfqpoint{2.448731in}{0.674816in}}%
\pgfpathlineto{\pgfqpoint{2.449496in}{0.732307in}}%
\pgfpathlineto{\pgfqpoint{2.449879in}{0.732307in}}%
\pgfpathlineto{\pgfqpoint{2.450645in}{0.669590in}}%
\pgfpathlineto{\pgfqpoint{2.451028in}{0.737534in}}%
\pgfpathlineto{\pgfqpoint{2.451411in}{0.695722in}}%
\pgfpathlineto{\pgfqpoint{2.451793in}{0.695722in}}%
\pgfpathlineto{\pgfqpoint{2.452176in}{0.664363in}}%
\pgfpathlineto{\pgfqpoint{2.452559in}{0.716628in}}%
\pgfpathlineto{\pgfqpoint{2.453325in}{0.716628in}}%
\pgfpathlineto{\pgfqpoint{2.453708in}{0.716628in}}%
\pgfpathlineto{\pgfqpoint{2.454856in}{0.774119in}}%
\pgfpathlineto{\pgfqpoint{2.455239in}{0.685269in}}%
\pgfpathlineto{\pgfqpoint{2.455622in}{0.685269in}}%
\pgfpathlineto{\pgfqpoint{2.456005in}{0.721854in}}%
\pgfpathlineto{\pgfqpoint{2.457153in}{0.711401in}}%
\pgfpathlineto{\pgfqpoint{2.457536in}{0.711401in}}%
\pgfpathlineto{\pgfqpoint{2.457536in}{0.721854in}}%
\pgfpathlineto{\pgfqpoint{2.458684in}{0.680043in}}%
\pgfpathlineto{\pgfqpoint{2.459067in}{0.690496in}}%
\pgfpathlineto{\pgfqpoint{2.459450in}{0.690496in}}%
\pgfpathlineto{\pgfqpoint{2.459450in}{0.711401in}}%
\pgfpathlineto{\pgfqpoint{2.460216in}{0.664363in}}%
\pgfpathlineto{\pgfqpoint{2.460981in}{0.680043in}}%
\pgfpathlineto{\pgfqpoint{2.461364in}{0.680043in}}%
\pgfpathlineto{\pgfqpoint{2.461747in}{0.716628in}}%
\pgfpathlineto{\pgfqpoint{2.462896in}{0.700949in}}%
\pgfpathlineto{\pgfqpoint{2.463278in}{0.700949in}}%
\pgfpathlineto{\pgfqpoint{2.464044in}{0.674816in}}%
\pgfpathlineto{\pgfqpoint{2.464810in}{0.721854in}}%
\pgfpathlineto{\pgfqpoint{2.465193in}{0.721854in}}%
\pgfpathlineto{\pgfqpoint{2.466341in}{0.664363in}}%
\pgfpathlineto{\pgfqpoint{2.466724in}{0.690496in}}%
\pgfpathlineto{\pgfqpoint{2.467107in}{0.690496in}}%
\pgfpathlineto{\pgfqpoint{2.467107in}{0.711401in}}%
\pgfpathlineto{\pgfqpoint{2.468638in}{0.685269in}}%
\pgfpathlineto{\pgfqpoint{2.469021in}{0.685269in}}%
\pgfpathlineto{\pgfqpoint{2.469021in}{0.674816in}}%
\pgfpathlineto{\pgfqpoint{2.470552in}{0.711401in}}%
\pgfpathlineto{\pgfqpoint{2.470935in}{0.711401in}}%
\pgfpathlineto{\pgfqpoint{2.470935in}{0.659137in}}%
\pgfpathlineto{\pgfqpoint{2.472466in}{0.680043in}}%
\pgfpathlineto{\pgfqpoint{2.472849in}{0.680043in}}%
\pgfpathlineto{\pgfqpoint{2.472849in}{0.669590in}}%
\pgfpathlineto{\pgfqpoint{2.473232in}{0.700949in}}%
\pgfpathlineto{\pgfqpoint{2.474380in}{0.685269in}}%
\pgfpathlineto{\pgfqpoint{2.474763in}{0.685269in}}%
\pgfpathlineto{\pgfqpoint{2.475146in}{0.653910in}}%
\pgfpathlineto{\pgfqpoint{2.475529in}{0.706175in}}%
\pgfpathlineto{\pgfqpoint{2.476295in}{0.695722in}}%
\pgfpathlineto{\pgfqpoint{2.476677in}{0.695722in}}%
\pgfpathlineto{\pgfqpoint{2.477826in}{0.659137in}}%
\pgfpathlineto{\pgfqpoint{2.478209in}{0.664363in}}%
\pgfpathlineto{\pgfqpoint{2.478592in}{0.664363in}}%
\pgfpathlineto{\pgfqpoint{2.478974in}{0.700949in}}%
\pgfpathlineto{\pgfqpoint{2.480123in}{0.674816in}}%
\pgfpathlineto{\pgfqpoint{2.480506in}{0.674816in}}%
\pgfpathlineto{\pgfqpoint{2.481271in}{0.700949in}}%
\pgfpathlineto{\pgfqpoint{2.480889in}{0.664363in}}%
\pgfpathlineto{\pgfqpoint{2.482037in}{0.674816in}}%
\pgfpathlineto{\pgfqpoint{2.482420in}{0.674816in}}%
\pgfpathlineto{\pgfqpoint{2.483186in}{0.700949in}}%
\pgfpathlineto{\pgfqpoint{2.483951in}{0.674816in}}%
\pgfpathlineto{\pgfqpoint{2.484334in}{0.674816in}}%
\pgfpathlineto{\pgfqpoint{2.484717in}{0.695722in}}%
\pgfpathlineto{\pgfqpoint{2.485865in}{0.680043in}}%
\pgfpathlineto{\pgfqpoint{2.486248in}{0.680043in}}%
\pgfpathlineto{\pgfqpoint{2.487014in}{0.700949in}}%
\pgfpathlineto{\pgfqpoint{2.486631in}{0.669590in}}%
\pgfpathlineto{\pgfqpoint{2.487779in}{0.695722in}}%
\pgfpathlineto{\pgfqpoint{2.488162in}{0.695722in}}%
\pgfpathlineto{\pgfqpoint{2.489311in}{0.669590in}}%
\pgfpathlineto{\pgfqpoint{2.488545in}{0.716628in}}%
\pgfpathlineto{\pgfqpoint{2.489694in}{0.685269in}}%
\pgfpathlineto{\pgfqpoint{2.490076in}{0.685269in}}%
\pgfpathlineto{\pgfqpoint{2.490842in}{0.711401in}}%
\pgfpathlineto{\pgfqpoint{2.491225in}{0.680043in}}%
\pgfpathlineto{\pgfqpoint{2.491608in}{0.690496in}}%
\pgfpathlineto{\pgfqpoint{2.491991in}{0.690496in}}%
\pgfpathlineto{\pgfqpoint{2.492373in}{0.716628in}}%
\pgfpathlineto{\pgfqpoint{2.492756in}{0.669590in}}%
\pgfpathlineto{\pgfqpoint{2.493522in}{0.680043in}}%
\pgfpathlineto{\pgfqpoint{2.493905in}{0.680043in}}%
\pgfpathlineto{\pgfqpoint{2.495053in}{0.706175in}}%
\pgfpathlineto{\pgfqpoint{2.495436in}{0.643457in}}%
\pgfpathlineto{\pgfqpoint{2.495819in}{0.643457in}}%
\pgfpathlineto{\pgfqpoint{2.496585in}{0.706175in}}%
\pgfpathlineto{\pgfqpoint{2.497350in}{0.669590in}}%
\pgfpathlineto{\pgfqpoint{2.497733in}{0.669590in}}%
\pgfpathlineto{\pgfqpoint{2.498116in}{0.706175in}}%
\pgfpathlineto{\pgfqpoint{2.498499in}{0.648684in}}%
\pgfpathlineto{\pgfqpoint{2.499264in}{0.674816in}}%
\pgfpathlineto{\pgfqpoint{2.500030in}{0.674816in}}%
\pgfpathlineto{\pgfqpoint{2.501561in}{0.690496in}}%
\pgfpathlineto{\pgfqpoint{2.501944in}{0.690496in}}%
\pgfpathlineto{\pgfqpoint{2.501944in}{0.664363in}}%
\pgfpathlineto{\pgfqpoint{2.503476in}{0.706175in}}%
\pgfpathlineto{\pgfqpoint{2.503858in}{0.706175in}}%
\pgfpathlineto{\pgfqpoint{2.503858in}{0.659137in}}%
\pgfpathlineto{\pgfqpoint{2.505390in}{0.669590in}}%
\pgfpathlineto{\pgfqpoint{2.506155in}{0.669590in}}%
\pgfpathlineto{\pgfqpoint{2.506155in}{0.659137in}}%
\pgfpathlineto{\pgfqpoint{2.506538in}{0.700949in}}%
\pgfpathlineto{\pgfqpoint{2.507687in}{0.659137in}}%
\pgfpathlineto{\pgfqpoint{2.508069in}{0.659137in}}%
\pgfpathlineto{\pgfqpoint{2.509601in}{0.690496in}}%
\pgfpathlineto{\pgfqpoint{2.509984in}{0.690496in}}%
\pgfpathlineto{\pgfqpoint{2.509984in}{0.653910in}}%
\pgfpathlineto{\pgfqpoint{2.510749in}{0.700949in}}%
\pgfpathlineto{\pgfqpoint{2.511515in}{0.653910in}}%
\pgfpathlineto{\pgfqpoint{2.511898in}{0.653910in}}%
\pgfpathlineto{\pgfqpoint{2.513046in}{0.643457in}}%
\pgfpathlineto{\pgfqpoint{2.513429in}{0.690496in}}%
\pgfpathlineto{\pgfqpoint{2.514195in}{0.690496in}}%
\pgfpathlineto{\pgfqpoint{2.514195in}{0.648684in}}%
\pgfpathlineto{\pgfqpoint{2.515726in}{0.669590in}}%
\pgfpathlineto{\pgfqpoint{2.516109in}{0.669590in}}%
\pgfpathlineto{\pgfqpoint{2.516492in}{0.706175in}}%
\pgfpathlineto{\pgfqpoint{2.517257in}{0.664363in}}%
\pgfpathlineto{\pgfqpoint{2.517640in}{0.669590in}}%
\pgfpathlineto{\pgfqpoint{2.518023in}{0.669590in}}%
\pgfpathlineto{\pgfqpoint{2.518023in}{0.674816in}}%
\pgfpathlineto{\pgfqpoint{2.519172in}{0.664363in}}%
\pgfpathlineto{\pgfqpoint{2.519554in}{0.664363in}}%
\pgfpathlineto{\pgfqpoint{2.519937in}{0.664363in}}%
\pgfpathlineto{\pgfqpoint{2.519937in}{0.638231in}}%
\pgfpathlineto{\pgfqpoint{2.521469in}{0.669590in}}%
\pgfpathlineto{\pgfqpoint{2.521851in}{0.669590in}}%
\pgfpathlineto{\pgfqpoint{2.521851in}{0.674816in}}%
\pgfpathlineto{\pgfqpoint{2.523000in}{0.653910in}}%
\pgfpathlineto{\pgfqpoint{2.523383in}{0.659137in}}%
\pgfpathlineto{\pgfqpoint{2.523765in}{0.659137in}}%
\pgfpathlineto{\pgfqpoint{2.524914in}{0.674816in}}%
\pgfpathlineto{\pgfqpoint{2.525297in}{0.669590in}}%
\pgfpathlineto{\pgfqpoint{2.525680in}{0.669590in}}%
\pgfpathlineto{\pgfqpoint{2.526445in}{0.643457in}}%
\pgfpathlineto{\pgfqpoint{2.527211in}{0.643457in}}%
\pgfpathlineto{\pgfqpoint{2.527594in}{0.643457in}}%
\pgfpathlineto{\pgfqpoint{2.528359in}{0.680043in}}%
\pgfpathlineto{\pgfqpoint{2.529125in}{0.653910in}}%
\pgfpathlineto{\pgfqpoint{2.529508in}{0.653910in}}%
\pgfpathlineto{\pgfqpoint{2.529891in}{0.680043in}}%
\pgfpathlineto{\pgfqpoint{2.531039in}{0.643457in}}%
\pgfpathlineto{\pgfqpoint{2.531422in}{0.643457in}}%
\pgfpathlineto{\pgfqpoint{2.531422in}{0.690496in}}%
\pgfpathlineto{\pgfqpoint{2.532953in}{0.669590in}}%
\pgfpathlineto{\pgfqpoint{2.533336in}{0.669590in}}%
\pgfpathlineto{\pgfqpoint{2.534102in}{0.690496in}}%
\pgfpathlineto{\pgfqpoint{2.534868in}{0.648684in}}%
\pgfpathlineto{\pgfqpoint{2.535250in}{0.648684in}}%
\pgfpathlineto{\pgfqpoint{2.535250in}{0.664363in}}%
\pgfpathlineto{\pgfqpoint{2.536399in}{0.643457in}}%
\pgfpathlineto{\pgfqpoint{2.536782in}{0.664363in}}%
\pgfpathlineto{\pgfqpoint{2.537165in}{0.664363in}}%
\pgfpathlineto{\pgfqpoint{2.537165in}{0.690496in}}%
\pgfpathlineto{\pgfqpoint{2.537547in}{0.653910in}}%
\pgfpathlineto{\pgfqpoint{2.538696in}{0.664363in}}%
\pgfpathlineto{\pgfqpoint{2.539079in}{0.664363in}}%
\pgfpathlineto{\pgfqpoint{2.539079in}{0.690496in}}%
\pgfpathlineto{\pgfqpoint{2.540610in}{0.643457in}}%
\pgfpathlineto{\pgfqpoint{2.540993in}{0.643457in}}%
\pgfpathlineto{\pgfqpoint{2.542524in}{0.680043in}}%
\pgfpathlineto{\pgfqpoint{2.542907in}{0.680043in}}%
\pgfpathlineto{\pgfqpoint{2.543673in}{0.643457in}}%
\pgfpathlineto{\pgfqpoint{2.544438in}{0.685269in}}%
\pgfpathlineto{\pgfqpoint{2.544821in}{0.685269in}}%
\pgfpathlineto{\pgfqpoint{2.546352in}{0.643457in}}%
\pgfpathlineto{\pgfqpoint{2.546735in}{0.643457in}}%
\pgfpathlineto{\pgfqpoint{2.546735in}{0.638231in}}%
\pgfpathlineto{\pgfqpoint{2.547118in}{0.664363in}}%
\pgfpathlineto{\pgfqpoint{2.548267in}{0.664363in}}%
\pgfpathlineto{\pgfqpoint{2.548649in}{0.664363in}}%
\pgfpathlineto{\pgfqpoint{2.549415in}{0.638231in}}%
\pgfpathlineto{\pgfqpoint{2.549032in}{0.669590in}}%
\pgfpathlineto{\pgfqpoint{2.550181in}{0.664363in}}%
\pgfpathlineto{\pgfqpoint{2.550946in}{0.664363in}}%
\pgfpathlineto{\pgfqpoint{2.550946in}{0.680043in}}%
\pgfpathlineto{\pgfqpoint{2.551329in}{0.643457in}}%
\pgfpathlineto{\pgfqpoint{2.552478in}{0.648684in}}%
\pgfpathlineto{\pgfqpoint{2.552861in}{0.648684in}}%
\pgfpathlineto{\pgfqpoint{2.552861in}{0.695722in}}%
\pgfpathlineto{\pgfqpoint{2.553243in}{0.638231in}}%
\pgfpathlineto{\pgfqpoint{2.554392in}{0.664363in}}%
\pgfpathlineto{\pgfqpoint{2.554775in}{0.664363in}}%
\pgfpathlineto{\pgfqpoint{2.554775in}{0.648684in}}%
\pgfpathlineto{\pgfqpoint{2.555540in}{0.669590in}}%
\pgfpathlineto{\pgfqpoint{2.556306in}{0.653910in}}%
\pgfpathlineto{\pgfqpoint{2.556689in}{0.653910in}}%
\pgfpathlineto{\pgfqpoint{2.556689in}{0.659137in}}%
\pgfpathlineto{\pgfqpoint{2.557072in}{0.648684in}}%
\pgfpathlineto{\pgfqpoint{2.558220in}{0.659137in}}%
\pgfpathlineto{\pgfqpoint{2.558986in}{0.659137in}}%
\pgfpathlineto{\pgfqpoint{2.560134in}{0.674816in}}%
\pgfpathlineto{\pgfqpoint{2.560517in}{0.648684in}}%
\pgfpathlineto{\pgfqpoint{2.560900in}{0.648684in}}%
\pgfpathlineto{\pgfqpoint{2.561666in}{0.690496in}}%
\pgfpathlineto{\pgfqpoint{2.562431in}{0.638231in}}%
\pgfpathlineto{\pgfqpoint{2.562814in}{0.638231in}}%
\pgfpathlineto{\pgfqpoint{2.563197in}{0.664363in}}%
\pgfpathlineto{\pgfqpoint{2.564345in}{0.659137in}}%
\pgfpathlineto{\pgfqpoint{2.564728in}{0.659137in}}%
\pgfpathlineto{\pgfqpoint{2.565877in}{0.648684in}}%
\pgfpathlineto{\pgfqpoint{2.565494in}{0.680043in}}%
\pgfpathlineto{\pgfqpoint{2.566260in}{0.648684in}}%
\pgfpathlineto{\pgfqpoint{2.566642in}{0.648684in}}%
\pgfpathlineto{\pgfqpoint{2.566642in}{0.669590in}}%
\pgfpathlineto{\pgfqpoint{2.568174in}{0.669590in}}%
\pgfpathlineto{\pgfqpoint{2.568557in}{0.669590in}}%
\pgfpathlineto{\pgfqpoint{2.568939in}{0.638231in}}%
\pgfpathlineto{\pgfqpoint{2.570088in}{0.638231in}}%
\pgfpathlineto{\pgfqpoint{2.570471in}{0.638231in}}%
\pgfpathlineto{\pgfqpoint{2.572002in}{0.664363in}}%
\pgfpathlineto{\pgfqpoint{2.572385in}{0.664363in}}%
\pgfpathlineto{\pgfqpoint{2.573533in}{0.643457in}}%
\pgfpathlineto{\pgfqpoint{2.573916in}{0.653910in}}%
\pgfpathlineto{\pgfqpoint{2.574299in}{0.653910in}}%
\pgfpathlineto{\pgfqpoint{2.574299in}{0.659137in}}%
\pgfpathlineto{\pgfqpoint{2.575448in}{0.633004in}}%
\pgfpathlineto{\pgfqpoint{2.575830in}{0.648684in}}%
\pgfpathlineto{\pgfqpoint{2.576213in}{0.648684in}}%
\pgfpathlineto{\pgfqpoint{2.576596in}{0.674816in}}%
\pgfpathlineto{\pgfqpoint{2.577745in}{0.643457in}}%
\pgfpathlineto{\pgfqpoint{2.578127in}{0.643457in}}%
\pgfpathlineto{\pgfqpoint{2.578127in}{0.664363in}}%
\pgfpathlineto{\pgfqpoint{2.579659in}{0.659137in}}%
\pgfpathlineto{\pgfqpoint{2.580042in}{0.659137in}}%
\pgfpathlineto{\pgfqpoint{2.580807in}{0.643457in}}%
\pgfpathlineto{\pgfqpoint{2.581573in}{0.669590in}}%
\pgfpathlineto{\pgfqpoint{2.581956in}{0.669590in}}%
\pgfpathlineto{\pgfqpoint{2.581956in}{0.638231in}}%
\pgfpathlineto{\pgfqpoint{2.583487in}{0.659137in}}%
\pgfpathlineto{\pgfqpoint{2.584253in}{0.659137in}}%
\pgfpathlineto{\pgfqpoint{2.584635in}{0.669590in}}%
\pgfpathlineto{\pgfqpoint{2.585018in}{0.633004in}}%
\pgfpathlineto{\pgfqpoint{2.585784in}{0.643457in}}%
\pgfpathlineto{\pgfqpoint{2.586167in}{0.643457in}}%
\pgfpathlineto{\pgfqpoint{2.587698in}{0.674816in}}%
\pgfpathlineto{\pgfqpoint{2.588081in}{0.674816in}}%
\pgfpathlineto{\pgfqpoint{2.588081in}{0.633004in}}%
\pgfpathlineto{\pgfqpoint{2.589612in}{0.680043in}}%
\pgfpathlineto{\pgfqpoint{2.589995in}{0.680043in}}%
\pgfpathlineto{\pgfqpoint{2.591526in}{0.638231in}}%
\pgfpathlineto{\pgfqpoint{2.591909in}{0.638231in}}%
\pgfpathlineto{\pgfqpoint{2.591909in}{0.664363in}}%
\pgfpathlineto{\pgfqpoint{2.593441in}{0.643457in}}%
\pgfpathlineto{\pgfqpoint{2.593823in}{0.643457in}}%
\pgfpathlineto{\pgfqpoint{2.593823in}{0.659137in}}%
\pgfpathlineto{\pgfqpoint{2.594589in}{0.638231in}}%
\pgfpathlineto{\pgfqpoint{2.595355in}{0.648684in}}%
\pgfpathlineto{\pgfqpoint{2.595738in}{0.648684in}}%
\pgfpathlineto{\pgfqpoint{2.595738in}{0.643457in}}%
\pgfpathlineto{\pgfqpoint{2.596120in}{0.653910in}}%
\pgfpathlineto{\pgfqpoint{2.597269in}{0.648684in}}%
\pgfpathlineto{\pgfqpoint{2.597652in}{0.648684in}}%
\pgfpathlineto{\pgfqpoint{2.597652in}{0.664363in}}%
\pgfpathlineto{\pgfqpoint{2.599183in}{0.664363in}}%
\pgfpathlineto{\pgfqpoint{2.599566in}{0.664363in}}%
\pgfpathlineto{\pgfqpoint{2.599566in}{0.638231in}}%
\pgfpathlineto{\pgfqpoint{2.601097in}{0.659137in}}%
\pgfpathlineto{\pgfqpoint{2.601480in}{0.659137in}}%
\pgfpathlineto{\pgfqpoint{2.601480in}{0.638231in}}%
\pgfpathlineto{\pgfqpoint{2.602246in}{0.680043in}}%
\pgfpathlineto{\pgfqpoint{2.603011in}{0.638231in}}%
\pgfpathlineto{\pgfqpoint{2.603394in}{0.638231in}}%
\pgfpathlineto{\pgfqpoint{2.604160in}{0.674816in}}%
\pgfpathlineto{\pgfqpoint{2.604925in}{0.669590in}}%
\pgfpathlineto{\pgfqpoint{2.605308in}{0.669590in}}%
\pgfpathlineto{\pgfqpoint{2.605308in}{0.643457in}}%
\pgfpathlineto{\pgfqpoint{2.606840in}{0.648684in}}%
\pgfpathlineto{\pgfqpoint{2.607222in}{0.648684in}}%
\pgfpathlineto{\pgfqpoint{2.608371in}{0.627778in}}%
\pgfpathlineto{\pgfqpoint{2.607605in}{0.664363in}}%
\pgfpathlineto{\pgfqpoint{2.608754in}{0.659137in}}%
\pgfpathlineto{\pgfqpoint{2.609137in}{0.659137in}}%
\pgfpathlineto{\pgfqpoint{2.609137in}{0.643457in}}%
\pgfpathlineto{\pgfqpoint{2.610668in}{0.653910in}}%
\pgfpathlineto{\pgfqpoint{2.611051in}{0.653910in}}%
\pgfpathlineto{\pgfqpoint{2.612582in}{0.633004in}}%
\pgfpathlineto{\pgfqpoint{2.612965in}{0.633004in}}%
\pgfpathlineto{\pgfqpoint{2.613348in}{0.659137in}}%
\pgfpathlineto{\pgfqpoint{2.614496in}{0.638231in}}%
\pgfpathlineto{\pgfqpoint{2.614879in}{0.638231in}}%
\pgfpathlineto{\pgfqpoint{2.615262in}{0.669590in}}%
\pgfpathlineto{\pgfqpoint{2.616410in}{0.653910in}}%
\pgfpathlineto{\pgfqpoint{2.616793in}{0.653910in}}%
\pgfpathlineto{\pgfqpoint{2.616793in}{0.633004in}}%
\pgfpathlineto{\pgfqpoint{2.617942in}{0.664363in}}%
\pgfpathlineto{\pgfqpoint{2.618325in}{0.648684in}}%
\pgfpathlineto{\pgfqpoint{2.618707in}{0.648684in}}%
\pgfpathlineto{\pgfqpoint{2.619090in}{0.653910in}}%
\pgfpathlineto{\pgfqpoint{2.620239in}{0.633004in}}%
\pgfpathlineto{\pgfqpoint{2.620621in}{0.633004in}}%
\pgfpathlineto{\pgfqpoint{2.621004in}{0.664363in}}%
\pgfpathlineto{\pgfqpoint{2.622153in}{0.659137in}}%
\pgfpathlineto{\pgfqpoint{2.622536in}{0.659137in}}%
\pgfpathlineto{\pgfqpoint{2.623684in}{0.638231in}}%
\pgfpathlineto{\pgfqpoint{2.624067in}{0.664363in}}%
\pgfpathlineto{\pgfqpoint{2.624450in}{0.664363in}}%
\pgfpathlineto{\pgfqpoint{2.624450in}{0.638231in}}%
\pgfpathlineto{\pgfqpoint{2.625981in}{0.648684in}}%
\pgfpathlineto{\pgfqpoint{2.626364in}{0.648684in}}%
\pgfpathlineto{\pgfqpoint{2.626364in}{0.643457in}}%
\pgfpathlineto{\pgfqpoint{2.626747in}{0.659137in}}%
\pgfpathlineto{\pgfqpoint{2.627895in}{0.643457in}}%
\pgfpathlineto{\pgfqpoint{2.628278in}{0.643457in}}%
\pgfpathlineto{\pgfqpoint{2.629427in}{0.638231in}}%
\pgfpathlineto{\pgfqpoint{2.629809in}{0.669590in}}%
\pgfpathlineto{\pgfqpoint{2.630192in}{0.669590in}}%
\pgfpathlineto{\pgfqpoint{2.630958in}{0.633004in}}%
\pgfpathlineto{\pgfqpoint{2.631724in}{0.633004in}}%
\pgfpathlineto{\pgfqpoint{2.632489in}{0.633004in}}%
\pgfpathlineto{\pgfqpoint{2.632872in}{0.659137in}}%
\pgfpathlineto{\pgfqpoint{2.634021in}{0.638231in}}%
\pgfpathlineto{\pgfqpoint{2.634403in}{0.638231in}}%
\pgfpathlineto{\pgfqpoint{2.634403in}{0.653910in}}%
\pgfpathlineto{\pgfqpoint{2.635935in}{0.638231in}}%
\pgfpathlineto{\pgfqpoint{2.636318in}{0.638231in}}%
\pgfpathlineto{\pgfqpoint{2.636318in}{0.664363in}}%
\pgfpathlineto{\pgfqpoint{2.637849in}{0.643457in}}%
\pgfpathlineto{\pgfqpoint{2.638997in}{0.643457in}}%
\pgfpathlineto{\pgfqpoint{2.639763in}{0.664363in}}%
\pgfpathlineto{\pgfqpoint{2.640529in}{0.633004in}}%
\pgfpathlineto{\pgfqpoint{2.640911in}{0.633004in}}%
\pgfpathlineto{\pgfqpoint{2.640911in}{0.653910in}}%
\pgfpathlineto{\pgfqpoint{2.642443in}{0.648684in}}%
\pgfpathlineto{\pgfqpoint{2.643208in}{0.648684in}}%
\pgfpathlineto{\pgfqpoint{2.643974in}{0.633004in}}%
\pgfpathlineto{\pgfqpoint{2.644740in}{0.653910in}}%
\pgfpathlineto{\pgfqpoint{2.645123in}{0.653910in}}%
\pgfpathlineto{\pgfqpoint{2.646271in}{0.627778in}}%
\pgfpathlineto{\pgfqpoint{2.646654in}{0.659137in}}%
\pgfpathlineto{\pgfqpoint{2.647037in}{0.659137in}}%
\pgfpathlineto{\pgfqpoint{2.647420in}{0.627778in}}%
\pgfpathlineto{\pgfqpoint{2.648185in}{0.664363in}}%
\pgfpathlineto{\pgfqpoint{2.648568in}{0.659137in}}%
\pgfpathlineto{\pgfqpoint{2.648951in}{0.659137in}}%
\pgfpathlineto{\pgfqpoint{2.650099in}{0.627778in}}%
\pgfpathlineto{\pgfqpoint{2.650482in}{0.653910in}}%
\pgfpathlineto{\pgfqpoint{2.650865in}{0.653910in}}%
\pgfpathlineto{\pgfqpoint{2.650865in}{0.659137in}}%
\pgfpathlineto{\pgfqpoint{2.651248in}{0.638231in}}%
\pgfpathlineto{\pgfqpoint{2.652396in}{0.648684in}}%
\pgfpathlineto{\pgfqpoint{2.652779in}{0.648684in}}%
\pgfpathlineto{\pgfqpoint{2.652779in}{0.633004in}}%
\pgfpathlineto{\pgfqpoint{2.653162in}{0.653910in}}%
\pgfpathlineto{\pgfqpoint{2.654311in}{0.638231in}}%
\pgfpathlineto{\pgfqpoint{2.654693in}{0.638231in}}%
\pgfpathlineto{\pgfqpoint{2.654693in}{0.627778in}}%
\pgfpathlineto{\pgfqpoint{2.655842in}{0.653910in}}%
\pgfpathlineto{\pgfqpoint{2.656225in}{0.633004in}}%
\pgfpathlineto{\pgfqpoint{2.656608in}{0.633004in}}%
\pgfpathlineto{\pgfqpoint{2.657756in}{0.664363in}}%
\pgfpathlineto{\pgfqpoint{2.658139in}{0.638231in}}%
\pgfpathlineto{\pgfqpoint{2.658522in}{0.638231in}}%
\pgfpathlineto{\pgfqpoint{2.658522in}{0.653910in}}%
\pgfpathlineto{\pgfqpoint{2.660053in}{0.643457in}}%
\pgfpathlineto{\pgfqpoint{2.660436in}{0.643457in}}%
\pgfpathlineto{\pgfqpoint{2.660436in}{0.627778in}}%
\pgfpathlineto{\pgfqpoint{2.661201in}{0.664363in}}%
\pgfpathlineto{\pgfqpoint{2.661967in}{0.633004in}}%
\pgfpathlineto{\pgfqpoint{2.662350in}{0.633004in}}%
\pgfpathlineto{\pgfqpoint{2.662350in}{0.664363in}}%
\pgfpathlineto{\pgfqpoint{2.663881in}{0.643457in}}%
\pgfpathlineto{\pgfqpoint{2.664264in}{0.643457in}}%
\pgfpathlineto{\pgfqpoint{2.664264in}{0.627778in}}%
\pgfpathlineto{\pgfqpoint{2.665413in}{0.653910in}}%
\pgfpathlineto{\pgfqpoint{2.665795in}{0.648684in}}%
\pgfpathlineto{\pgfqpoint{2.666178in}{0.648684in}}%
\pgfpathlineto{\pgfqpoint{2.666178in}{0.633004in}}%
\pgfpathlineto{\pgfqpoint{2.667710in}{0.633004in}}%
\pgfpathlineto{\pgfqpoint{2.668092in}{0.633004in}}%
\pgfpathlineto{\pgfqpoint{2.668092in}{0.653910in}}%
\pgfpathlineto{\pgfqpoint{2.669624in}{0.643457in}}%
\pgfpathlineto{\pgfqpoint{2.670007in}{0.643457in}}%
\pgfpathlineto{\pgfqpoint{2.670007in}{0.653910in}}%
\pgfpathlineto{\pgfqpoint{2.671538in}{0.643457in}}%
\pgfpathlineto{\pgfqpoint{2.671921in}{0.643457in}}%
\pgfpathlineto{\pgfqpoint{2.671921in}{0.627778in}}%
\pgfpathlineto{\pgfqpoint{2.672304in}{0.648684in}}%
\pgfpathlineto{\pgfqpoint{2.673452in}{0.638231in}}%
\pgfpathlineto{\pgfqpoint{2.673835in}{0.638231in}}%
\pgfpathlineto{\pgfqpoint{2.674983in}{0.664363in}}%
\pgfpathlineto{\pgfqpoint{2.674218in}{0.633004in}}%
\pgfpathlineto{\pgfqpoint{2.675366in}{0.653910in}}%
\pgfpathlineto{\pgfqpoint{2.675749in}{0.653910in}}%
\pgfpathlineto{\pgfqpoint{2.675749in}{0.638231in}}%
\pgfpathlineto{\pgfqpoint{2.677280in}{0.643457in}}%
\pgfpathlineto{\pgfqpoint{2.677663in}{0.643457in}}%
\pgfpathlineto{\pgfqpoint{2.677663in}{0.627778in}}%
\pgfpathlineto{\pgfqpoint{2.678046in}{0.648684in}}%
\pgfpathlineto{\pgfqpoint{2.679194in}{0.633004in}}%
\pgfpathlineto{\pgfqpoint{2.679577in}{0.633004in}}%
\pgfpathlineto{\pgfqpoint{2.680343in}{0.643457in}}%
\pgfpathlineto{\pgfqpoint{2.681109in}{0.633004in}}%
\pgfpathlineto{\pgfqpoint{2.681874in}{0.633004in}}%
\pgfpathlineto{\pgfqpoint{2.681874in}{0.653910in}}%
\pgfpathlineto{\pgfqpoint{2.683406in}{0.653910in}}%
\pgfpathlineto{\pgfqpoint{2.683788in}{0.653910in}}%
\pgfpathlineto{\pgfqpoint{2.685320in}{0.633004in}}%
\pgfpathlineto{\pgfqpoint{2.686085in}{0.633004in}}%
\pgfpathlineto{\pgfqpoint{2.686468in}{0.643457in}}%
\pgfpathlineto{\pgfqpoint{2.687617in}{0.643457in}}%
\pgfpathlineto{\pgfqpoint{2.688000in}{0.643457in}}%
\pgfpathlineto{\pgfqpoint{2.688382in}{0.633004in}}%
\pgfpathlineto{\pgfqpoint{2.689531in}{0.653910in}}%
\pgfpathlineto{\pgfqpoint{2.689914in}{0.653910in}}%
\pgfpathlineto{\pgfqpoint{2.691062in}{0.627778in}}%
\pgfpathlineto{\pgfqpoint{2.691445in}{0.653910in}}%
\pgfpathlineto{\pgfqpoint{2.691828in}{0.653910in}}%
\pgfpathlineto{\pgfqpoint{2.692976in}{0.633004in}}%
\pgfpathlineto{\pgfqpoint{2.693359in}{0.638231in}}%
\pgfpathlineto{\pgfqpoint{2.693742in}{0.638231in}}%
\pgfpathlineto{\pgfqpoint{2.693742in}{0.659137in}}%
\pgfpathlineto{\pgfqpoint{2.694125in}{0.633004in}}%
\pgfpathlineto{\pgfqpoint{2.695273in}{0.648684in}}%
\pgfpathlineto{\pgfqpoint{2.695656in}{0.648684in}}%
\pgfpathlineto{\pgfqpoint{2.695656in}{0.659137in}}%
\pgfpathlineto{\pgfqpoint{2.696422in}{0.633004in}}%
\pgfpathlineto{\pgfqpoint{2.697188in}{0.648684in}}%
\pgfpathlineto{\pgfqpoint{2.697570in}{0.648684in}}%
\pgfpathlineto{\pgfqpoint{2.698336in}{0.627778in}}%
\pgfpathlineto{\pgfqpoint{2.698719in}{0.653910in}}%
\pgfpathlineto{\pgfqpoint{2.699102in}{0.633004in}}%
\pgfpathlineto{\pgfqpoint{2.700250in}{0.633004in}}%
\pgfpathlineto{\pgfqpoint{2.701781in}{0.648684in}}%
\pgfpathlineto{\pgfqpoint{2.702164in}{0.648684in}}%
\pgfpathlineto{\pgfqpoint{2.703313in}{0.633004in}}%
\pgfpathlineto{\pgfqpoint{2.702930in}{0.659137in}}%
\pgfpathlineto{\pgfqpoint{2.703696in}{0.633004in}}%
\pgfpathlineto{\pgfqpoint{2.704078in}{0.633004in}}%
\pgfpathlineto{\pgfqpoint{2.704078in}{0.648684in}}%
\pgfpathlineto{\pgfqpoint{2.704844in}{0.627778in}}%
\pgfpathlineto{\pgfqpoint{2.705610in}{0.648684in}}%
\pgfpathlineto{\pgfqpoint{2.705993in}{0.648684in}}%
\pgfpathlineto{\pgfqpoint{2.705993in}{0.633004in}}%
\pgfpathlineto{\pgfqpoint{2.707524in}{0.659137in}}%
\pgfpathlineto{\pgfqpoint{2.707907in}{0.659137in}}%
\pgfpathlineto{\pgfqpoint{2.708672in}{0.627778in}}%
\pgfpathlineto{\pgfqpoint{2.709438in}{0.643457in}}%
\pgfpathlineto{\pgfqpoint{2.710204in}{0.643457in}}%
\pgfpathlineto{\pgfqpoint{2.710587in}{0.633004in}}%
\pgfpathlineto{\pgfqpoint{2.711735in}{0.653910in}}%
\pgfpathlineto{\pgfqpoint{2.712118in}{0.653910in}}%
\pgfpathlineto{\pgfqpoint{2.713649in}{0.633004in}}%
\pgfpathlineto{\pgfqpoint{2.714032in}{0.633004in}}%
\pgfpathlineto{\pgfqpoint{2.714798in}{0.648684in}}%
\pgfpathlineto{\pgfqpoint{2.714415in}{0.627778in}}%
\pgfpathlineto{\pgfqpoint{2.715563in}{0.643457in}}%
\pgfpathlineto{\pgfqpoint{2.715946in}{0.643457in}}%
\pgfpathlineto{\pgfqpoint{2.715946in}{0.633004in}}%
\pgfpathlineto{\pgfqpoint{2.716329in}{0.648684in}}%
\pgfpathlineto{\pgfqpoint{2.717477in}{0.638231in}}%
\pgfpathlineto{\pgfqpoint{2.718243in}{0.638231in}}%
\pgfpathlineto{\pgfqpoint{2.718243in}{0.627778in}}%
\pgfpathlineto{\pgfqpoint{2.718626in}{0.653910in}}%
\pgfpathlineto{\pgfqpoint{2.719774in}{0.633004in}}%
\pgfpathlineto{\pgfqpoint{2.720157in}{0.633004in}}%
\pgfpathlineto{\pgfqpoint{2.720157in}{0.653910in}}%
\pgfpathlineto{\pgfqpoint{2.721689in}{0.633004in}}%
\pgfpathlineto{\pgfqpoint{2.722837in}{0.633004in}}%
\pgfpathlineto{\pgfqpoint{2.724368in}{0.653910in}}%
\pgfpathlineto{\pgfqpoint{2.724751in}{0.653910in}}%
\pgfpathlineto{\pgfqpoint{2.725134in}{0.627778in}}%
\pgfpathlineto{\pgfqpoint{2.726283in}{0.638231in}}%
\pgfpathlineto{\pgfqpoint{2.726665in}{0.638231in}}%
\pgfpathlineto{\pgfqpoint{2.727048in}{0.627778in}}%
\pgfpathlineto{\pgfqpoint{2.727431in}{0.643457in}}%
\pgfpathlineto{\pgfqpoint{2.728197in}{0.638231in}}%
\pgfpathlineto{\pgfqpoint{2.728580in}{0.638231in}}%
\pgfpathlineto{\pgfqpoint{2.728580in}{0.648684in}}%
\pgfpathlineto{\pgfqpoint{2.728962in}{0.627778in}}%
\pgfpathlineto{\pgfqpoint{2.730111in}{0.638231in}}%
\pgfpathlineto{\pgfqpoint{2.730494in}{0.638231in}}%
\pgfpathlineto{\pgfqpoint{2.730494in}{0.633004in}}%
\pgfpathlineto{\pgfqpoint{2.730877in}{0.648684in}}%
\pgfpathlineto{\pgfqpoint{2.732025in}{0.648684in}}%
\pgfpathlineto{\pgfqpoint{2.732408in}{0.648684in}}%
\pgfpathlineto{\pgfqpoint{2.732408in}{0.633004in}}%
\pgfpathlineto{\pgfqpoint{2.732791in}{0.653910in}}%
\pgfpathlineto{\pgfqpoint{2.733939in}{0.653910in}}%
\pgfpathlineto{\pgfqpoint{2.734322in}{0.653910in}}%
\pgfpathlineto{\pgfqpoint{2.734705in}{0.627778in}}%
\pgfpathlineto{\pgfqpoint{2.735853in}{0.633004in}}%
\pgfpathlineto{\pgfqpoint{2.736619in}{0.633004in}}%
\pgfpathlineto{\pgfqpoint{2.736619in}{0.648684in}}%
\pgfpathlineto{\pgfqpoint{2.738150in}{0.643457in}}%
\pgfpathlineto{\pgfqpoint{2.738533in}{0.643457in}}%
\pgfpathlineto{\pgfqpoint{2.739682in}{0.627778in}}%
\pgfpathlineto{\pgfqpoint{2.739299in}{0.653910in}}%
\pgfpathlineto{\pgfqpoint{2.740064in}{0.643457in}}%
\pgfpathlineto{\pgfqpoint{2.740447in}{0.643457in}}%
\pgfpathlineto{\pgfqpoint{2.741213in}{0.633004in}}%
\pgfpathlineto{\pgfqpoint{2.740830in}{0.653910in}}%
\pgfpathlineto{\pgfqpoint{2.741979in}{0.648684in}}%
\pgfpathlineto{\pgfqpoint{2.742361in}{0.648684in}}%
\pgfpathlineto{\pgfqpoint{2.743127in}{0.627778in}}%
\pgfpathlineto{\pgfqpoint{2.743510in}{0.659137in}}%
\pgfpathlineto{\pgfqpoint{2.743893in}{0.638231in}}%
\pgfpathlineto{\pgfqpoint{2.744276in}{0.638231in}}%
\pgfpathlineto{\pgfqpoint{2.745041in}{0.633004in}}%
\pgfpathlineto{\pgfqpoint{2.745807in}{0.653910in}}%
\pgfpathlineto{\pgfqpoint{2.746190in}{0.653910in}}%
\pgfpathlineto{\pgfqpoint{2.746190in}{0.627778in}}%
\pgfpathlineto{\pgfqpoint{2.747721in}{0.643457in}}%
\pgfpathlineto{\pgfqpoint{2.748104in}{0.643457in}}%
\pgfpathlineto{\pgfqpoint{2.748104in}{0.638231in}}%
\pgfpathlineto{\pgfqpoint{2.748870in}{0.648684in}}%
\pgfpathlineto{\pgfqpoint{2.749635in}{0.643457in}}%
\pgfpathlineto{\pgfqpoint{2.750018in}{0.643457in}}%
\pgfpathlineto{\pgfqpoint{2.750018in}{0.648684in}}%
\pgfpathlineto{\pgfqpoint{2.750784in}{0.627778in}}%
\pgfpathlineto{\pgfqpoint{2.751549in}{0.633004in}}%
\pgfpathlineto{\pgfqpoint{2.751932in}{0.633004in}}%
\pgfpathlineto{\pgfqpoint{2.753081in}{0.648684in}}%
\pgfpathlineto{\pgfqpoint{2.753464in}{0.627778in}}%
\pgfpathlineto{\pgfqpoint{2.753846in}{0.627778in}}%
\pgfpathlineto{\pgfqpoint{2.753846in}{0.653910in}}%
\pgfpathlineto{\pgfqpoint{2.755378in}{0.638231in}}%
\pgfpathlineto{\pgfqpoint{2.755760in}{0.638231in}}%
\pgfpathlineto{\pgfqpoint{2.755760in}{0.633004in}}%
\pgfpathlineto{\pgfqpoint{2.757292in}{0.633004in}}%
\pgfpathlineto{\pgfqpoint{2.757675in}{0.633004in}}%
\pgfpathlineto{\pgfqpoint{2.757675in}{0.627778in}}%
\pgfpathlineto{\pgfqpoint{2.758057in}{0.638231in}}%
\pgfpathlineto{\pgfqpoint{2.759206in}{0.638231in}}%
\pgfpathlineto{\pgfqpoint{2.759589in}{0.638231in}}%
\pgfpathlineto{\pgfqpoint{2.759589in}{0.659137in}}%
\pgfpathlineto{\pgfqpoint{2.759972in}{0.633004in}}%
\pgfpathlineto{\pgfqpoint{2.761120in}{0.633004in}}%
\pgfpathlineto{\pgfqpoint{2.761503in}{0.633004in}}%
\pgfpathlineto{\pgfqpoint{2.761503in}{0.643457in}}%
\pgfpathlineto{\pgfqpoint{2.763034in}{0.627778in}}%
\pgfpathlineto{\pgfqpoint{2.763417in}{0.627778in}}%
\pgfpathlineto{\pgfqpoint{2.763417in}{0.638231in}}%
\pgfpathlineto{\pgfqpoint{2.764948in}{0.633004in}}%
\pgfpathlineto{\pgfqpoint{2.766097in}{0.633004in}}%
\pgfpathlineto{\pgfqpoint{2.766863in}{0.653910in}}%
\pgfpathlineto{\pgfqpoint{2.766480in}{0.627778in}}%
\pgfpathlineto{\pgfqpoint{2.767628in}{0.633004in}}%
\pgfpathlineto{\pgfqpoint{2.768394in}{0.633004in}}%
\pgfpathlineto{\pgfqpoint{2.769925in}{0.653910in}}%
\pgfpathlineto{\pgfqpoint{2.770308in}{0.653910in}}%
\pgfpathlineto{\pgfqpoint{2.770691in}{0.627778in}}%
\pgfpathlineto{\pgfqpoint{2.771457in}{0.664363in}}%
\pgfpathlineto{\pgfqpoint{2.771839in}{0.633004in}}%
\pgfpathlineto{\pgfqpoint{2.772222in}{0.633004in}}%
\pgfpathlineto{\pgfqpoint{2.772222in}{0.638231in}}%
\pgfpathlineto{\pgfqpoint{2.773754in}{0.627778in}}%
\pgfpathlineto{\pgfqpoint{2.774136in}{0.627778in}}%
\pgfpathlineto{\pgfqpoint{2.775285in}{0.643457in}}%
\pgfpathlineto{\pgfqpoint{2.775668in}{0.633004in}}%
\pgfpathlineto{\pgfqpoint{2.776050in}{0.633004in}}%
\pgfpathlineto{\pgfqpoint{2.776433in}{0.659137in}}%
\pgfpathlineto{\pgfqpoint{2.777582in}{0.638231in}}%
\pgfpathlineto{\pgfqpoint{2.777965in}{0.638231in}}%
\pgfpathlineto{\pgfqpoint{2.779113in}{0.643457in}}%
\pgfpathlineto{\pgfqpoint{2.779496in}{0.627778in}}%
\pgfpathlineto{\pgfqpoint{2.779879in}{0.627778in}}%
\pgfpathlineto{\pgfqpoint{2.780262in}{0.648684in}}%
\pgfpathlineto{\pgfqpoint{2.781410in}{0.633004in}}%
\pgfpathlineto{\pgfqpoint{2.782559in}{0.633004in}}%
\pgfpathlineto{\pgfqpoint{2.782559in}{0.627778in}}%
\pgfpathlineto{\pgfqpoint{2.783707in}{0.648684in}}%
\pgfpathlineto{\pgfqpoint{2.784090in}{0.633004in}}%
\pgfpathlineto{\pgfqpoint{2.785621in}{0.633004in}}%
\pgfpathlineto{\pgfqpoint{2.786004in}{0.643457in}}%
\pgfpathlineto{\pgfqpoint{2.787153in}{0.638231in}}%
\pgfpathlineto{\pgfqpoint{2.787535in}{0.638231in}}%
\pgfpathlineto{\pgfqpoint{2.787535in}{0.633004in}}%
\pgfpathlineto{\pgfqpoint{2.787918in}{0.643457in}}%
\pgfpathlineto{\pgfqpoint{2.789067in}{0.638231in}}%
\pgfpathlineto{\pgfqpoint{2.789450in}{0.638231in}}%
\pgfpathlineto{\pgfqpoint{2.789450in}{0.633004in}}%
\pgfpathlineto{\pgfqpoint{2.790981in}{0.633004in}}%
\pgfpathlineto{\pgfqpoint{2.791364in}{0.633004in}}%
\pgfpathlineto{\pgfqpoint{2.791364in}{0.638231in}}%
\pgfpathlineto{\pgfqpoint{2.792895in}{0.638231in}}%
\pgfpathlineto{\pgfqpoint{2.793278in}{0.638231in}}%
\pgfpathlineto{\pgfqpoint{2.794044in}{0.627778in}}%
\pgfpathlineto{\pgfqpoint{2.794426in}{0.648684in}}%
\pgfpathlineto{\pgfqpoint{2.794809in}{0.643457in}}%
\pgfpathlineto{\pgfqpoint{2.795192in}{0.643457in}}%
\pgfpathlineto{\pgfqpoint{2.795192in}{0.638231in}}%
\pgfpathlineto{\pgfqpoint{2.796723in}{0.638231in}}%
\pgfpathlineto{\pgfqpoint{2.797106in}{0.638231in}}%
\pgfpathlineto{\pgfqpoint{2.797106in}{0.633004in}}%
\pgfpathlineto{\pgfqpoint{2.798637in}{0.648684in}}%
\pgfpathlineto{\pgfqpoint{2.799020in}{0.648684in}}%
\pgfpathlineto{\pgfqpoint{2.799403in}{0.633004in}}%
\pgfpathlineto{\pgfqpoint{2.800552in}{0.648684in}}%
\pgfpathlineto{\pgfqpoint{2.800934in}{0.648684in}}%
\pgfpathlineto{\pgfqpoint{2.801317in}{0.633004in}}%
\pgfpathlineto{\pgfqpoint{2.802466in}{0.633004in}}%
\pgfpathlineto{\pgfqpoint{2.802849in}{0.633004in}}%
\pgfpathlineto{\pgfqpoint{2.802849in}{0.627778in}}%
\pgfpathlineto{\pgfqpoint{2.803614in}{0.648684in}}%
\pgfpathlineto{\pgfqpoint{2.804380in}{0.638231in}}%
\pgfpathlineto{\pgfqpoint{2.804763in}{0.638231in}}%
\pgfpathlineto{\pgfqpoint{2.804763in}{0.633004in}}%
\pgfpathlineto{\pgfqpoint{2.806294in}{0.633004in}}%
\pgfpathlineto{\pgfqpoint{2.806677in}{0.633004in}}%
\pgfpathlineto{\pgfqpoint{2.806677in}{0.648684in}}%
\pgfpathlineto{\pgfqpoint{2.807825in}{0.627778in}}%
\pgfpathlineto{\pgfqpoint{2.808208in}{0.638231in}}%
\pgfpathlineto{\pgfqpoint{2.808591in}{0.638231in}}%
\pgfpathlineto{\pgfqpoint{2.808974in}{0.648684in}}%
\pgfpathlineto{\pgfqpoint{2.809357in}{0.627778in}}%
\pgfpathlineto{\pgfqpoint{2.810122in}{0.648684in}}%
\pgfpathlineto{\pgfqpoint{2.810505in}{0.648684in}}%
\pgfpathlineto{\pgfqpoint{2.810888in}{0.627778in}}%
\pgfpathlineto{\pgfqpoint{2.812037in}{0.638231in}}%
\pgfpathlineto{\pgfqpoint{2.812419in}{0.638231in}}%
\pgfpathlineto{\pgfqpoint{2.812419in}{0.648684in}}%
\pgfpathlineto{\pgfqpoint{2.813568in}{0.627778in}}%
\pgfpathlineto{\pgfqpoint{2.813951in}{0.643457in}}%
\pgfpathlineto{\pgfqpoint{2.814333in}{0.643457in}}%
\pgfpathlineto{\pgfqpoint{2.814333in}{0.633004in}}%
\pgfpathlineto{\pgfqpoint{2.815482in}{0.648684in}}%
\pgfpathlineto{\pgfqpoint{2.815865in}{0.643457in}}%
\pgfpathlineto{\pgfqpoint{2.816248in}{0.643457in}}%
\pgfpathlineto{\pgfqpoint{2.816248in}{0.627778in}}%
\pgfpathlineto{\pgfqpoint{2.817779in}{0.627778in}}%
\pgfpathlineto{\pgfqpoint{2.818162in}{0.627778in}}%
\pgfpathlineto{\pgfqpoint{2.818545in}{0.653910in}}%
\pgfpathlineto{\pgfqpoint{2.819693in}{0.627778in}}%
\pgfpathlineto{\pgfqpoint{2.820459in}{0.627778in}}%
\pgfpathlineto{\pgfqpoint{2.821607in}{0.648684in}}%
\pgfpathlineto{\pgfqpoint{2.821990in}{0.638231in}}%
\pgfpathlineto{\pgfqpoint{2.822373in}{0.638231in}}%
\pgfpathlineto{\pgfqpoint{2.822373in}{0.633004in}}%
\pgfpathlineto{\pgfqpoint{2.823139in}{0.648684in}}%
\pgfpathlineto{\pgfqpoint{2.823904in}{0.643457in}}%
\pgfpathlineto{\pgfqpoint{2.824287in}{0.643457in}}%
\pgfpathlineto{\pgfqpoint{2.824287in}{0.648684in}}%
\pgfpathlineto{\pgfqpoint{2.824670in}{0.633004in}}%
\pgfpathlineto{\pgfqpoint{2.825818in}{0.633004in}}%
\pgfpathlineto{\pgfqpoint{2.826201in}{0.633004in}}%
\pgfpathlineto{\pgfqpoint{2.826201in}{0.627778in}}%
\pgfpathlineto{\pgfqpoint{2.826967in}{0.648684in}}%
\pgfpathlineto{\pgfqpoint{2.827733in}{0.643457in}}%
\pgfpathlineto{\pgfqpoint{2.828115in}{0.643457in}}%
\pgfpathlineto{\pgfqpoint{2.828115in}{0.633004in}}%
\pgfpathlineto{\pgfqpoint{2.829264in}{0.648684in}}%
\pgfpathlineto{\pgfqpoint{2.829647in}{0.638231in}}%
\pgfpathlineto{\pgfqpoint{2.830412in}{0.638231in}}%
\pgfpathlineto{\pgfqpoint{2.830412in}{0.633004in}}%
\pgfpathlineto{\pgfqpoint{2.830795in}{0.643457in}}%
\pgfpathlineto{\pgfqpoint{2.831944in}{0.643457in}}%
\pgfpathlineto{\pgfqpoint{2.832327in}{0.643457in}}%
\pgfpathlineto{\pgfqpoint{2.832327in}{0.627778in}}%
\pgfpathlineto{\pgfqpoint{2.833858in}{0.638231in}}%
\pgfpathlineto{\pgfqpoint{2.834241in}{0.638231in}}%
\pgfpathlineto{\pgfqpoint{2.834623in}{0.643457in}}%
\pgfpathlineto{\pgfqpoint{2.835772in}{0.627778in}}%
\pgfpathlineto{\pgfqpoint{2.836155in}{0.627778in}}%
\pgfpathlineto{\pgfqpoint{2.836155in}{0.648684in}}%
\pgfpathlineto{\pgfqpoint{2.837686in}{0.633004in}}%
\pgfpathlineto{\pgfqpoint{2.838452in}{0.633004in}}%
\pgfpathlineto{\pgfqpoint{2.838452in}{0.638231in}}%
\pgfpathlineto{\pgfqpoint{2.839600in}{0.627778in}}%
\pgfpathlineto{\pgfqpoint{2.839983in}{0.633004in}}%
\pgfpathlineto{\pgfqpoint{2.840366in}{0.633004in}}%
\pgfpathlineto{\pgfqpoint{2.841514in}{0.643457in}}%
\pgfpathlineto{\pgfqpoint{2.841897in}{0.627778in}}%
\pgfpathlineto{\pgfqpoint{2.842280in}{0.627778in}}%
\pgfpathlineto{\pgfqpoint{2.842280in}{0.638231in}}%
\pgfpathlineto{\pgfqpoint{2.843811in}{0.627778in}}%
\pgfpathlineto{\pgfqpoint{2.844194in}{0.627778in}}%
\pgfpathlineto{\pgfqpoint{2.845726in}{0.648684in}}%
\pgfpathlineto{\pgfqpoint{2.846108in}{0.648684in}}%
\pgfpathlineto{\pgfqpoint{2.846874in}{0.627778in}}%
\pgfpathlineto{\pgfqpoint{2.847640in}{0.643457in}}%
\pgfpathlineto{\pgfqpoint{2.848023in}{0.643457in}}%
\pgfpathlineto{\pgfqpoint{2.848023in}{0.653910in}}%
\pgfpathlineto{\pgfqpoint{2.848405in}{0.627778in}}%
\pgfpathlineto{\pgfqpoint{2.849554in}{0.638231in}}%
\pgfpathlineto{\pgfqpoint{2.849937in}{0.638231in}}%
\pgfpathlineto{\pgfqpoint{2.850320in}{0.627778in}}%
\pgfpathlineto{\pgfqpoint{2.851468in}{0.627778in}}%
\pgfpathlineto{\pgfqpoint{2.851851in}{0.627778in}}%
\pgfpathlineto{\pgfqpoint{2.852234in}{0.648684in}}%
\pgfpathlineto{\pgfqpoint{2.853382in}{0.633004in}}%
\pgfpathlineto{\pgfqpoint{2.853765in}{0.633004in}}%
\pgfpathlineto{\pgfqpoint{2.854531in}{0.653910in}}%
\pgfpathlineto{\pgfqpoint{2.855296in}{0.633004in}}%
\pgfpathlineto{\pgfqpoint{2.855679in}{0.633004in}}%
\pgfpathlineto{\pgfqpoint{2.856445in}{0.643457in}}%
\pgfpathlineto{\pgfqpoint{2.856062in}{0.627778in}}%
\pgfpathlineto{\pgfqpoint{2.857210in}{0.633004in}}%
\pgfpathlineto{\pgfqpoint{2.858359in}{0.633004in}}%
\pgfpathlineto{\pgfqpoint{2.858359in}{0.643457in}}%
\pgfpathlineto{\pgfqpoint{2.859890in}{0.633004in}}%
\pgfpathlineto{\pgfqpoint{2.860273in}{0.633004in}}%
\pgfpathlineto{\pgfqpoint{2.860273in}{0.643457in}}%
\pgfpathlineto{\pgfqpoint{2.861039in}{0.627778in}}%
\pgfpathlineto{\pgfqpoint{2.861804in}{0.627778in}}%
\pgfpathlineto{\pgfqpoint{2.862187in}{0.627778in}}%
\pgfpathlineto{\pgfqpoint{2.862570in}{0.648684in}}%
\pgfpathlineto{\pgfqpoint{2.863719in}{0.638231in}}%
\pgfpathlineto{\pgfqpoint{2.864101in}{0.638231in}}%
\pgfpathlineto{\pgfqpoint{2.864101in}{0.643457in}}%
\pgfpathlineto{\pgfqpoint{2.864484in}{0.627778in}}%
\pgfpathlineto{\pgfqpoint{2.865633in}{0.633004in}}%
\pgfpathlineto{\pgfqpoint{2.866016in}{0.633004in}}%
\pgfpathlineto{\pgfqpoint{2.866781in}{0.643457in}}%
\pgfpathlineto{\pgfqpoint{2.867547in}{0.627778in}}%
\pgfpathlineto{\pgfqpoint{2.868313in}{0.627778in}}%
\pgfpathlineto{\pgfqpoint{2.868313in}{0.653910in}}%
\pgfpathlineto{\pgfqpoint{2.869844in}{0.633004in}}%
\pgfpathlineto{\pgfqpoint{2.870227in}{0.633004in}}%
\pgfpathlineto{\pgfqpoint{2.871375in}{0.643457in}}%
\pgfpathlineto{\pgfqpoint{2.870992in}{0.627778in}}%
\pgfpathlineto{\pgfqpoint{2.871758in}{0.638231in}}%
\pgfpathlineto{\pgfqpoint{2.872524in}{0.638231in}}%
\pgfpathlineto{\pgfqpoint{2.872524in}{0.633004in}}%
\pgfpathlineto{\pgfqpoint{2.872906in}{0.643457in}}%
\pgfpathlineto{\pgfqpoint{2.874055in}{0.638231in}}%
\pgfpathlineto{\pgfqpoint{2.874821in}{0.638231in}}%
\pgfpathlineto{\pgfqpoint{2.874821in}{0.627778in}}%
\pgfpathlineto{\pgfqpoint{2.875969in}{0.643457in}}%
\pgfpathlineto{\pgfqpoint{2.876352in}{0.638231in}}%
\pgfpathlineto{\pgfqpoint{2.876735in}{0.638231in}}%
\pgfpathlineto{\pgfqpoint{2.876735in}{0.633004in}}%
\pgfpathlineto{\pgfqpoint{2.878266in}{0.643457in}}%
\pgfpathlineto{\pgfqpoint{2.878649in}{0.643457in}}%
\pgfpathlineto{\pgfqpoint{2.880180in}{0.627778in}}%
\pgfpathlineto{\pgfqpoint{2.880563in}{0.627778in}}%
\pgfpathlineto{\pgfqpoint{2.881329in}{0.643457in}}%
\pgfpathlineto{\pgfqpoint{2.882094in}{0.638231in}}%
\pgfpathlineto{\pgfqpoint{2.882477in}{0.638231in}}%
\pgfpathlineto{\pgfqpoint{2.883243in}{0.627778in}}%
\pgfpathlineto{\pgfqpoint{2.884009in}{0.627778in}}%
\pgfpathlineto{\pgfqpoint{2.884391in}{0.627778in}}%
\pgfpathlineto{\pgfqpoint{2.884391in}{0.648684in}}%
\pgfpathlineto{\pgfqpoint{2.885923in}{0.638231in}}%
\pgfpathlineto{\pgfqpoint{2.886688in}{0.638231in}}%
\pgfpathlineto{\pgfqpoint{2.886688in}{0.659137in}}%
\pgfpathlineto{\pgfqpoint{2.887071in}{0.627778in}}%
\pgfpathlineto{\pgfqpoint{2.888220in}{0.633004in}}%
\pgfpathlineto{\pgfqpoint{2.888603in}{0.633004in}}%
\pgfpathlineto{\pgfqpoint{2.890134in}{0.643457in}}%
\pgfpathlineto{\pgfqpoint{2.890517in}{0.643457in}}%
\pgfpathlineto{\pgfqpoint{2.892048in}{0.627778in}}%
\pgfpathlineto{\pgfqpoint{2.892431in}{0.627778in}}%
\pgfpathlineto{\pgfqpoint{2.892431in}{0.643457in}}%
\pgfpathlineto{\pgfqpoint{2.893962in}{0.627778in}}%
\pgfpathlineto{\pgfqpoint{2.894345in}{0.627778in}}%
\pgfpathlineto{\pgfqpoint{2.894345in}{0.638231in}}%
\pgfpathlineto{\pgfqpoint{2.895876in}{0.633004in}}%
\pgfpathlineto{\pgfqpoint{2.896642in}{0.633004in}}%
\pgfpathlineto{\pgfqpoint{2.896642in}{0.627778in}}%
\pgfpathlineto{\pgfqpoint{2.897408in}{0.648684in}}%
\pgfpathlineto{\pgfqpoint{2.898173in}{0.643457in}}%
\pgfpathlineto{\pgfqpoint{2.898556in}{0.643457in}}%
\pgfpathlineto{\pgfqpoint{2.899322in}{0.633004in}}%
\pgfpathlineto{\pgfqpoint{2.899705in}{0.653910in}}%
\pgfpathlineto{\pgfqpoint{2.900087in}{0.638231in}}%
\pgfpathlineto{\pgfqpoint{2.900470in}{0.638231in}}%
\pgfpathlineto{\pgfqpoint{2.900470in}{0.633004in}}%
\pgfpathlineto{\pgfqpoint{2.902002in}{0.643457in}}%
\pgfpathlineto{\pgfqpoint{2.902384in}{0.643457in}}%
\pgfpathlineto{\pgfqpoint{2.902384in}{0.633004in}}%
\pgfpathlineto{\pgfqpoint{2.903916in}{0.653910in}}%
\pgfpathlineto{\pgfqpoint{2.904299in}{0.653910in}}%
\pgfpathlineto{\pgfqpoint{2.905830in}{0.627778in}}%
\pgfpathlineto{\pgfqpoint{2.906213in}{0.627778in}}%
\pgfpathlineto{\pgfqpoint{2.906978in}{0.643457in}}%
\pgfpathlineto{\pgfqpoint{2.907744in}{0.638231in}}%
\pgfpathlineto{\pgfqpoint{2.908127in}{0.638231in}}%
\pgfpathlineto{\pgfqpoint{2.908127in}{0.627778in}}%
\pgfpathlineto{\pgfqpoint{2.908510in}{0.648684in}}%
\pgfpathlineto{\pgfqpoint{2.909658in}{0.627778in}}%
\pgfpathlineto{\pgfqpoint{2.910041in}{0.627778in}}%
\pgfpathlineto{\pgfqpoint{2.911189in}{0.643457in}}%
\pgfpathlineto{\pgfqpoint{2.911572in}{0.627778in}}%
\pgfpathlineto{\pgfqpoint{2.911955in}{0.627778in}}%
\pgfpathlineto{\pgfqpoint{2.911955in}{0.643457in}}%
\pgfpathlineto{\pgfqpoint{2.913486in}{0.633004in}}%
\pgfpathlineto{\pgfqpoint{2.913869in}{0.633004in}}%
\pgfpathlineto{\pgfqpoint{2.913869in}{0.627778in}}%
\pgfpathlineto{\pgfqpoint{2.915401in}{0.643457in}}%
\pgfpathlineto{\pgfqpoint{2.915783in}{0.643457in}}%
\pgfpathlineto{\pgfqpoint{2.916166in}{0.627778in}}%
\pgfpathlineto{\pgfqpoint{2.917315in}{0.638231in}}%
\pgfpathlineto{\pgfqpoint{2.918080in}{0.638231in}}%
\pgfpathlineto{\pgfqpoint{2.918080in}{0.627778in}}%
\pgfpathlineto{\pgfqpoint{2.919612in}{0.638231in}}%
\pgfpathlineto{\pgfqpoint{2.919995in}{0.638231in}}%
\pgfpathlineto{\pgfqpoint{2.919995in}{0.633004in}}%
\pgfpathlineto{\pgfqpoint{2.920760in}{0.638231in}}%
\pgfusepath{stroke}%
\end{pgfscope}%
\begin{pgfscope}%
\pgfpathrectangle{\pgfqpoint{0.781944in}{0.552778in}}{\pgfqpoint{2.138715in}{1.650000in}}%
\pgfusepath{clip}%
\pgfsetrectcap%
\pgfsetroundjoin%
\pgfsetlinewidth{1.505625pt}%
\definecolor{currentstroke}{rgb}{1.000000,0.498039,0.054902}%
\pgfsetstrokecolor{currentstroke}%
\pgfsetstrokeopacity{0.800000}%
\pgfsetdash{}{0pt}%
\pgfpathmoveto{\pgfqpoint{0.781889in}{0.669590in}}%
\pgfpathlineto{\pgfqpoint{0.781889in}{0.653910in}}%
\pgfpathlineto{\pgfqpoint{0.783420in}{0.653910in}}%
\pgfpathlineto{\pgfqpoint{0.783803in}{0.653910in}}%
\pgfpathlineto{\pgfqpoint{0.783803in}{0.648684in}}%
\pgfpathlineto{\pgfqpoint{0.785334in}{0.669590in}}%
\pgfpathlineto{\pgfqpoint{0.785717in}{0.669590in}}%
\pgfpathlineto{\pgfqpoint{0.786866in}{0.648684in}}%
\pgfpathlineto{\pgfqpoint{0.787248in}{0.664363in}}%
\pgfpathlineto{\pgfqpoint{0.787631in}{0.664363in}}%
\pgfpathlineto{\pgfqpoint{0.788780in}{0.643457in}}%
\pgfpathlineto{\pgfqpoint{0.788014in}{0.685269in}}%
\pgfpathlineto{\pgfqpoint{0.789163in}{0.664363in}}%
\pgfpathlineto{\pgfqpoint{0.789928in}{0.664363in}}%
\pgfpathlineto{\pgfqpoint{0.789928in}{0.685269in}}%
\pgfpathlineto{\pgfqpoint{0.791077in}{0.638231in}}%
\pgfpathlineto{\pgfqpoint{0.791460in}{0.653910in}}%
\pgfpathlineto{\pgfqpoint{0.792225in}{0.653910in}}%
\pgfpathlineto{\pgfqpoint{0.792225in}{0.680043in}}%
\pgfpathlineto{\pgfqpoint{0.792608in}{0.638231in}}%
\pgfpathlineto{\pgfqpoint{0.793757in}{0.674816in}}%
\pgfpathlineto{\pgfqpoint{0.794139in}{0.674816in}}%
\pgfpathlineto{\pgfqpoint{0.794905in}{0.648684in}}%
\pgfpathlineto{\pgfqpoint{0.795671in}{0.653910in}}%
\pgfpathlineto{\pgfqpoint{0.796054in}{0.653910in}}%
\pgfpathlineto{\pgfqpoint{0.796054in}{0.659137in}}%
\pgfpathlineto{\pgfqpoint{0.796819in}{0.643457in}}%
\pgfpathlineto{\pgfqpoint{0.797585in}{0.643457in}}%
\pgfpathlineto{\pgfqpoint{0.797968in}{0.643457in}}%
\pgfpathlineto{\pgfqpoint{0.797968in}{0.674816in}}%
\pgfpathlineto{\pgfqpoint{0.799499in}{0.653910in}}%
\pgfpathlineto{\pgfqpoint{0.800648in}{0.653910in}}%
\pgfpathlineto{\pgfqpoint{0.801030in}{0.674816in}}%
\pgfpathlineto{\pgfqpoint{0.802179in}{0.653910in}}%
\pgfpathlineto{\pgfqpoint{0.802562in}{0.653910in}}%
\pgfpathlineto{\pgfqpoint{0.804093in}{0.680043in}}%
\pgfpathlineto{\pgfqpoint{0.804859in}{0.680043in}}%
\pgfpathlineto{\pgfqpoint{0.806390in}{0.659137in}}%
\pgfpathlineto{\pgfqpoint{0.806773in}{0.659137in}}%
\pgfpathlineto{\pgfqpoint{0.806773in}{0.680043in}}%
\pgfpathlineto{\pgfqpoint{0.808304in}{0.643457in}}%
\pgfpathlineto{\pgfqpoint{0.809070in}{0.643457in}}%
\pgfpathlineto{\pgfqpoint{0.810218in}{0.685269in}}%
\pgfpathlineto{\pgfqpoint{0.810601in}{0.680043in}}%
\pgfpathlineto{\pgfqpoint{0.810984in}{0.680043in}}%
\pgfpathlineto{\pgfqpoint{0.811367in}{0.664363in}}%
\pgfpathlineto{\pgfqpoint{0.811750in}{0.690496in}}%
\pgfpathlineto{\pgfqpoint{0.812515in}{0.674816in}}%
\pgfpathlineto{\pgfqpoint{0.812898in}{0.674816in}}%
\pgfpathlineto{\pgfqpoint{0.813664in}{0.659137in}}%
\pgfpathlineto{\pgfqpoint{0.813281in}{0.690496in}}%
\pgfpathlineto{\pgfqpoint{0.814429in}{0.659137in}}%
\pgfpathlineto{\pgfqpoint{0.814812in}{0.659137in}}%
\pgfpathlineto{\pgfqpoint{0.815961in}{0.685269in}}%
\pgfpathlineto{\pgfqpoint{0.815578in}{0.643457in}}%
\pgfpathlineto{\pgfqpoint{0.816344in}{0.680043in}}%
\pgfpathlineto{\pgfqpoint{0.816726in}{0.680043in}}%
\pgfpathlineto{\pgfqpoint{0.816726in}{0.659137in}}%
\pgfpathlineto{\pgfqpoint{0.818258in}{0.700949in}}%
\pgfpathlineto{\pgfqpoint{0.818641in}{0.700949in}}%
\pgfpathlineto{\pgfqpoint{0.819789in}{0.648684in}}%
\pgfpathlineto{\pgfqpoint{0.820172in}{0.690496in}}%
\pgfpathlineto{\pgfqpoint{0.820555in}{0.690496in}}%
\pgfpathlineto{\pgfqpoint{0.820938in}{0.706175in}}%
\pgfpathlineto{\pgfqpoint{0.822086in}{0.643457in}}%
\pgfpathlineto{\pgfqpoint{0.822469in}{0.643457in}}%
\pgfpathlineto{\pgfqpoint{0.822469in}{0.706175in}}%
\pgfpathlineto{\pgfqpoint{0.824000in}{0.669590in}}%
\pgfpathlineto{\pgfqpoint{0.824383in}{0.669590in}}%
\pgfpathlineto{\pgfqpoint{0.825149in}{0.664363in}}%
\pgfpathlineto{\pgfqpoint{0.825914in}{0.690496in}}%
\pgfpathlineto{\pgfqpoint{0.826297in}{0.690496in}}%
\pgfpathlineto{\pgfqpoint{0.827063in}{0.695722in}}%
\pgfpathlineto{\pgfqpoint{0.827828in}{0.659137in}}%
\pgfpathlineto{\pgfqpoint{0.828211in}{0.659137in}}%
\pgfpathlineto{\pgfqpoint{0.828594in}{0.695722in}}%
\pgfpathlineto{\pgfqpoint{0.829743in}{0.685269in}}%
\pgfpathlineto{\pgfqpoint{0.830125in}{0.685269in}}%
\pgfpathlineto{\pgfqpoint{0.830891in}{0.721854in}}%
\pgfpathlineto{\pgfqpoint{0.830508in}{0.680043in}}%
\pgfpathlineto{\pgfqpoint{0.831657in}{0.685269in}}%
\pgfpathlineto{\pgfqpoint{0.832040in}{0.685269in}}%
\pgfpathlineto{\pgfqpoint{0.832805in}{0.716628in}}%
\pgfpathlineto{\pgfqpoint{0.832422in}{0.674816in}}%
\pgfpathlineto{\pgfqpoint{0.833571in}{0.706175in}}%
\pgfpathlineto{\pgfqpoint{0.833954in}{0.706175in}}%
\pgfpathlineto{\pgfqpoint{0.833954in}{0.680043in}}%
\pgfpathlineto{\pgfqpoint{0.835102in}{0.753213in}}%
\pgfpathlineto{\pgfqpoint{0.835485in}{0.695722in}}%
\pgfpathlineto{\pgfqpoint{0.835868in}{0.695722in}}%
\pgfpathlineto{\pgfqpoint{0.837016in}{0.732307in}}%
\pgfpathlineto{\pgfqpoint{0.836251in}{0.685269in}}%
\pgfpathlineto{\pgfqpoint{0.837399in}{0.685269in}}%
\pgfpathlineto{\pgfqpoint{0.837782in}{0.685269in}}%
\pgfpathlineto{\pgfqpoint{0.839313in}{0.727081in}}%
\pgfpathlineto{\pgfqpoint{0.839696in}{0.727081in}}%
\pgfpathlineto{\pgfqpoint{0.839696in}{0.685269in}}%
\pgfpathlineto{\pgfqpoint{0.840079in}{0.732307in}}%
\pgfpathlineto{\pgfqpoint{0.841227in}{0.700949in}}%
\pgfpathlineto{\pgfqpoint{0.841993in}{0.700949in}}%
\pgfpathlineto{\pgfqpoint{0.842376in}{0.732307in}}%
\pgfpathlineto{\pgfqpoint{0.843142in}{0.680043in}}%
\pgfpathlineto{\pgfqpoint{0.843524in}{0.685269in}}%
\pgfpathlineto{\pgfqpoint{0.843907in}{0.685269in}}%
\pgfpathlineto{\pgfqpoint{0.843907in}{0.747987in}}%
\pgfpathlineto{\pgfqpoint{0.845439in}{0.721854in}}%
\pgfpathlineto{\pgfqpoint{0.845821in}{0.721854in}}%
\pgfpathlineto{\pgfqpoint{0.845821in}{0.716628in}}%
\pgfpathlineto{\pgfqpoint{0.846970in}{0.747987in}}%
\pgfpathlineto{\pgfqpoint{0.847353in}{0.727081in}}%
\pgfpathlineto{\pgfqpoint{0.847736in}{0.727081in}}%
\pgfpathlineto{\pgfqpoint{0.848884in}{0.706175in}}%
\pgfpathlineto{\pgfqpoint{0.848501in}{0.779346in}}%
\pgfpathlineto{\pgfqpoint{0.849267in}{0.716628in}}%
\pgfpathlineto{\pgfqpoint{0.849650in}{0.716628in}}%
\pgfpathlineto{\pgfqpoint{0.850798in}{0.742760in}}%
\pgfpathlineto{\pgfqpoint{0.850033in}{0.685269in}}%
\pgfpathlineto{\pgfqpoint{0.851181in}{0.716628in}}%
\pgfpathlineto{\pgfqpoint{0.851564in}{0.716628in}}%
\pgfpathlineto{\pgfqpoint{0.851564in}{0.742760in}}%
\pgfpathlineto{\pgfqpoint{0.852712in}{0.685269in}}%
\pgfpathlineto{\pgfqpoint{0.853095in}{0.716628in}}%
\pgfpathlineto{\pgfqpoint{0.853478in}{0.716628in}}%
\pgfpathlineto{\pgfqpoint{0.853478in}{0.711401in}}%
\pgfpathlineto{\pgfqpoint{0.854244in}{0.774119in}}%
\pgfpathlineto{\pgfqpoint{0.855009in}{0.732307in}}%
\pgfpathlineto{\pgfqpoint{0.855392in}{0.732307in}}%
\pgfpathlineto{\pgfqpoint{0.855392in}{0.789799in}}%
\pgfpathlineto{\pgfqpoint{0.856158in}{0.721854in}}%
\pgfpathlineto{\pgfqpoint{0.856924in}{0.737534in}}%
\pgfpathlineto{\pgfqpoint{0.857306in}{0.737534in}}%
\pgfpathlineto{\pgfqpoint{0.858072in}{0.774119in}}%
\pgfpathlineto{\pgfqpoint{0.858838in}{0.732307in}}%
\pgfpathlineto{\pgfqpoint{0.859221in}{0.732307in}}%
\pgfpathlineto{\pgfqpoint{0.859221in}{0.800252in}}%
\pgfpathlineto{\pgfqpoint{0.860752in}{0.742760in}}%
\pgfpathlineto{\pgfqpoint{0.861135in}{0.742760in}}%
\pgfpathlineto{\pgfqpoint{0.861135in}{0.690496in}}%
\pgfpathlineto{\pgfqpoint{0.862666in}{0.763666in}}%
\pgfpathlineto{\pgfqpoint{0.863049in}{0.763666in}}%
\pgfpathlineto{\pgfqpoint{0.863049in}{0.732307in}}%
\pgfpathlineto{\pgfqpoint{0.864197in}{0.810705in}}%
\pgfpathlineto{\pgfqpoint{0.864580in}{0.758440in}}%
\pgfpathlineto{\pgfqpoint{0.865346in}{0.758440in}}%
\pgfpathlineto{\pgfqpoint{0.866111in}{0.831611in}}%
\pgfpathlineto{\pgfqpoint{0.865729in}{0.727081in}}%
\pgfpathlineto{\pgfqpoint{0.866877in}{0.779346in}}%
\pgfpathlineto{\pgfqpoint{0.867260in}{0.779346in}}%
\pgfpathlineto{\pgfqpoint{0.867260in}{0.815931in}}%
\pgfpathlineto{\pgfqpoint{0.868791in}{0.758440in}}%
\pgfpathlineto{\pgfqpoint{0.869174in}{0.758440in}}%
\pgfpathlineto{\pgfqpoint{0.869174in}{0.747987in}}%
\pgfpathlineto{\pgfqpoint{0.869940in}{0.826384in}}%
\pgfpathlineto{\pgfqpoint{0.870705in}{0.789799in}}%
\pgfpathlineto{\pgfqpoint{0.871471in}{0.789799in}}%
\pgfpathlineto{\pgfqpoint{0.871471in}{0.821158in}}%
\pgfpathlineto{\pgfqpoint{0.871854in}{0.768893in}}%
\pgfpathlineto{\pgfqpoint{0.873002in}{0.768893in}}%
\pgfpathlineto{\pgfqpoint{0.873385in}{0.768893in}}%
\pgfpathlineto{\pgfqpoint{0.874151in}{0.810705in}}%
\pgfpathlineto{\pgfqpoint{0.874917in}{0.784572in}}%
\pgfpathlineto{\pgfqpoint{0.875299in}{0.784572in}}%
\pgfpathlineto{\pgfqpoint{0.875299in}{0.742760in}}%
\pgfpathlineto{\pgfqpoint{0.876831in}{0.847290in}}%
\pgfpathlineto{\pgfqpoint{0.877214in}{0.847290in}}%
\pgfpathlineto{\pgfqpoint{0.877214in}{0.784572in}}%
\pgfpathlineto{\pgfqpoint{0.878745in}{0.810705in}}%
\pgfpathlineto{\pgfqpoint{0.879128in}{0.810705in}}%
\pgfpathlineto{\pgfqpoint{0.879893in}{0.842063in}}%
\pgfpathlineto{\pgfqpoint{0.879510in}{0.774119in}}%
\pgfpathlineto{\pgfqpoint{0.880659in}{0.815931in}}%
\pgfpathlineto{\pgfqpoint{0.881042in}{0.815931in}}%
\pgfpathlineto{\pgfqpoint{0.882190in}{0.784572in}}%
\pgfpathlineto{\pgfqpoint{0.882573in}{0.862969in}}%
\pgfpathlineto{\pgfqpoint{0.882956in}{0.862969in}}%
\pgfpathlineto{\pgfqpoint{0.883339in}{0.789799in}}%
\pgfpathlineto{\pgfqpoint{0.884487in}{0.831611in}}%
\pgfpathlineto{\pgfqpoint{0.884870in}{0.831611in}}%
\pgfpathlineto{\pgfqpoint{0.884870in}{0.857743in}}%
\pgfpathlineto{\pgfqpoint{0.886019in}{0.774119in}}%
\pgfpathlineto{\pgfqpoint{0.886401in}{0.815931in}}%
\pgfpathlineto{\pgfqpoint{0.886784in}{0.815931in}}%
\pgfpathlineto{\pgfqpoint{0.886784in}{0.789799in}}%
\pgfpathlineto{\pgfqpoint{0.888316in}{0.889102in}}%
\pgfpathlineto{\pgfqpoint{0.888698in}{0.889102in}}%
\pgfpathlineto{\pgfqpoint{0.889081in}{0.831611in}}%
\pgfpathlineto{\pgfqpoint{0.890230in}{0.852516in}}%
\pgfpathlineto{\pgfqpoint{0.890613in}{0.852516in}}%
\pgfpathlineto{\pgfqpoint{0.890613in}{0.826384in}}%
\pgfpathlineto{\pgfqpoint{0.890995in}{0.862969in}}%
\pgfpathlineto{\pgfqpoint{0.892144in}{0.857743in}}%
\pgfpathlineto{\pgfqpoint{0.892527in}{0.857743in}}%
\pgfpathlineto{\pgfqpoint{0.894058in}{0.800252in}}%
\pgfpathlineto{\pgfqpoint{0.894441in}{0.800252in}}%
\pgfpathlineto{\pgfqpoint{0.895589in}{0.889102in}}%
\pgfpathlineto{\pgfqpoint{0.895972in}{0.805478in}}%
\pgfpathlineto{\pgfqpoint{0.896355in}{0.805478in}}%
\pgfpathlineto{\pgfqpoint{0.897504in}{0.878649in}}%
\pgfpathlineto{\pgfqpoint{0.897886in}{0.836837in}}%
\pgfpathlineto{\pgfqpoint{0.898269in}{0.836837in}}%
\pgfpathlineto{\pgfqpoint{0.899418in}{0.936140in}}%
\pgfpathlineto{\pgfqpoint{0.899800in}{0.899555in}}%
\pgfpathlineto{\pgfqpoint{0.900183in}{0.899555in}}%
\pgfpathlineto{\pgfqpoint{0.901332in}{0.805478in}}%
\pgfpathlineto{\pgfqpoint{0.900566in}{0.904781in}}%
\pgfpathlineto{\pgfqpoint{0.901715in}{0.862969in}}%
\pgfpathlineto{\pgfqpoint{0.902097in}{0.862969in}}%
\pgfpathlineto{\pgfqpoint{0.902863in}{0.842063in}}%
\pgfpathlineto{\pgfqpoint{0.903629in}{0.910008in}}%
\pgfpathlineto{\pgfqpoint{0.904012in}{0.910008in}}%
\pgfpathlineto{\pgfqpoint{0.904012in}{0.873422in}}%
\pgfpathlineto{\pgfqpoint{0.905543in}{0.936140in}}%
\pgfpathlineto{\pgfqpoint{0.905926in}{0.936140in}}%
\pgfpathlineto{\pgfqpoint{0.907457in}{0.868196in}}%
\pgfpathlineto{\pgfqpoint{0.907840in}{0.868196in}}%
\pgfpathlineto{\pgfqpoint{0.908988in}{0.910008in}}%
\pgfpathlineto{\pgfqpoint{0.908223in}{0.862969in}}%
\pgfpathlineto{\pgfqpoint{0.909371in}{0.889102in}}%
\pgfpathlineto{\pgfqpoint{0.909754in}{0.889102in}}%
\pgfpathlineto{\pgfqpoint{0.910903in}{0.972726in}}%
\pgfpathlineto{\pgfqpoint{0.910520in}{0.857743in}}%
\pgfpathlineto{\pgfqpoint{0.911285in}{0.868196in}}%
\pgfpathlineto{\pgfqpoint{0.911668in}{0.868196in}}%
\pgfpathlineto{\pgfqpoint{0.911668in}{0.972726in}}%
\pgfpathlineto{\pgfqpoint{0.913200in}{0.910008in}}%
\pgfpathlineto{\pgfqpoint{0.913582in}{0.910008in}}%
\pgfpathlineto{\pgfqpoint{0.913965in}{0.852516in}}%
\pgfpathlineto{\pgfqpoint{0.914348in}{0.925687in}}%
\pgfpathlineto{\pgfqpoint{0.915114in}{0.894328in}}%
\pgfpathlineto{\pgfqpoint{0.915497in}{0.894328in}}%
\pgfpathlineto{\pgfqpoint{0.915497in}{0.925687in}}%
\pgfpathlineto{\pgfqpoint{0.915879in}{0.836837in}}%
\pgfpathlineto{\pgfqpoint{0.917028in}{0.857743in}}%
\pgfpathlineto{\pgfqpoint{0.917411in}{0.857743in}}%
\pgfpathlineto{\pgfqpoint{0.917411in}{0.852516in}}%
\pgfpathlineto{\pgfqpoint{0.918559in}{0.941367in}}%
\pgfpathlineto{\pgfqpoint{0.918942in}{0.915234in}}%
\pgfpathlineto{\pgfqpoint{0.919325in}{0.915234in}}%
\pgfpathlineto{\pgfqpoint{0.920473in}{0.946593in}}%
\pgfpathlineto{\pgfqpoint{0.919708in}{0.821158in}}%
\pgfpathlineto{\pgfqpoint{0.920856in}{0.883875in}}%
\pgfpathlineto{\pgfqpoint{0.921239in}{0.883875in}}%
\pgfpathlineto{\pgfqpoint{0.921239in}{1.014537in}}%
\pgfpathlineto{\pgfqpoint{0.922770in}{0.962273in}}%
\pgfpathlineto{\pgfqpoint{0.923153in}{0.962273in}}%
\pgfpathlineto{\pgfqpoint{0.923153in}{0.967499in}}%
\pgfpathlineto{\pgfqpoint{0.923919in}{0.899555in}}%
\pgfpathlineto{\pgfqpoint{0.924684in}{0.925687in}}%
\pgfpathlineto{\pgfqpoint{0.925067in}{0.925687in}}%
\pgfpathlineto{\pgfqpoint{0.925450in}{1.014537in}}%
\pgfpathlineto{\pgfqpoint{0.926599in}{0.977952in}}%
\pgfpathlineto{\pgfqpoint{0.926981in}{0.977952in}}%
\pgfpathlineto{\pgfqpoint{0.927364in}{0.936140in}}%
\pgfpathlineto{\pgfqpoint{0.927747in}{0.993631in}}%
\pgfpathlineto{\pgfqpoint{0.928513in}{0.988405in}}%
\pgfpathlineto{\pgfqpoint{0.928896in}{0.988405in}}%
\pgfpathlineto{\pgfqpoint{0.928896in}{0.930914in}}%
\pgfpathlineto{\pgfqpoint{0.930427in}{1.009311in}}%
\pgfpathlineto{\pgfqpoint{0.930810in}{1.009311in}}%
\pgfpathlineto{\pgfqpoint{0.930810in}{0.894328in}}%
\pgfpathlineto{\pgfqpoint{0.931958in}{1.024990in}}%
\pgfpathlineto{\pgfqpoint{0.932341in}{1.014537in}}%
\pgfpathlineto{\pgfqpoint{0.932724in}{1.014537in}}%
\pgfpathlineto{\pgfqpoint{0.932724in}{0.936140in}}%
\pgfpathlineto{\pgfqpoint{0.934255in}{0.936140in}}%
\pgfpathlineto{\pgfqpoint{0.934638in}{0.936140in}}%
\pgfpathlineto{\pgfqpoint{0.935787in}{1.092935in}}%
\pgfpathlineto{\pgfqpoint{0.935404in}{0.889102in}}%
\pgfpathlineto{\pgfqpoint{0.936169in}{0.967499in}}%
\pgfpathlineto{\pgfqpoint{0.936935in}{0.967499in}}%
\pgfpathlineto{\pgfqpoint{0.937318in}{1.072029in}}%
\pgfpathlineto{\pgfqpoint{0.937701in}{0.910008in}}%
\pgfpathlineto{\pgfqpoint{0.938466in}{1.009311in}}%
\pgfpathlineto{\pgfqpoint{0.938849in}{1.009311in}}%
\pgfpathlineto{\pgfqpoint{0.939232in}{0.936140in}}%
\pgfpathlineto{\pgfqpoint{0.939998in}{1.072029in}}%
\pgfpathlineto{\pgfqpoint{0.940380in}{1.035443in}}%
\pgfpathlineto{\pgfqpoint{0.940763in}{1.035443in}}%
\pgfpathlineto{\pgfqpoint{0.941529in}{0.930914in}}%
\pgfpathlineto{\pgfqpoint{0.942295in}{0.993631in}}%
\pgfpathlineto{\pgfqpoint{0.942677in}{0.993631in}}%
\pgfpathlineto{\pgfqpoint{0.943060in}{0.936140in}}%
\pgfpathlineto{\pgfqpoint{0.944209in}{1.030217in}}%
\pgfpathlineto{\pgfqpoint{0.944592in}{1.030217in}}%
\pgfpathlineto{\pgfqpoint{0.944974in}{0.904781in}}%
\pgfpathlineto{\pgfqpoint{0.946123in}{0.988405in}}%
\pgfpathlineto{\pgfqpoint{0.946506in}{0.988405in}}%
\pgfpathlineto{\pgfqpoint{0.946889in}{1.061576in}}%
\pgfpathlineto{\pgfqpoint{0.948037in}{1.024990in}}%
\pgfpathlineto{\pgfqpoint{0.948420in}{1.024990in}}%
\pgfpathlineto{\pgfqpoint{0.949186in}{1.014537in}}%
\pgfpathlineto{\pgfqpoint{0.949951in}{1.061576in}}%
\pgfpathlineto{\pgfqpoint{0.950334in}{1.061576in}}%
\pgfpathlineto{\pgfqpoint{0.951100in}{0.998858in}}%
\pgfpathlineto{\pgfqpoint{0.951483in}{1.124293in}}%
\pgfpathlineto{\pgfqpoint{0.951865in}{1.030217in}}%
\pgfpathlineto{\pgfqpoint{0.952248in}{1.030217in}}%
\pgfpathlineto{\pgfqpoint{0.952248in}{0.977952in}}%
\pgfpathlineto{\pgfqpoint{0.953780in}{1.077255in}}%
\pgfpathlineto{\pgfqpoint{0.954162in}{1.077255in}}%
\pgfpathlineto{\pgfqpoint{0.954545in}{0.972726in}}%
\pgfpathlineto{\pgfqpoint{0.955311in}{1.103388in}}%
\pgfpathlineto{\pgfqpoint{0.955694in}{1.061576in}}%
\pgfpathlineto{\pgfqpoint{0.956077in}{1.061576in}}%
\pgfpathlineto{\pgfqpoint{0.957225in}{1.004084in}}%
\pgfpathlineto{\pgfqpoint{0.956842in}{1.077255in}}%
\pgfpathlineto{\pgfqpoint{0.957608in}{1.040670in}}%
\pgfpathlineto{\pgfqpoint{0.957991in}{1.040670in}}%
\pgfpathlineto{\pgfqpoint{0.959522in}{1.150426in}}%
\pgfpathlineto{\pgfqpoint{0.959905in}{1.150426in}}%
\pgfpathlineto{\pgfqpoint{0.959905in}{0.998858in}}%
\pgfpathlineto{\pgfqpoint{0.961436in}{1.051123in}}%
\pgfpathlineto{\pgfqpoint{0.961819in}{1.051123in}}%
\pgfpathlineto{\pgfqpoint{0.962585in}{0.983178in}}%
\pgfpathlineto{\pgfqpoint{0.962202in}{1.077255in}}%
\pgfpathlineto{\pgfqpoint{0.963350in}{1.035443in}}%
\pgfpathlineto{\pgfqpoint{0.963733in}{1.035443in}}%
\pgfpathlineto{\pgfqpoint{0.963733in}{1.014537in}}%
\pgfpathlineto{\pgfqpoint{0.965264in}{1.098161in}}%
\pgfpathlineto{\pgfqpoint{0.965647in}{1.098161in}}%
\pgfpathlineto{\pgfqpoint{0.966030in}{0.988405in}}%
\pgfpathlineto{\pgfqpoint{0.967179in}{1.019764in}}%
\pgfpathlineto{\pgfqpoint{0.967561in}{1.019764in}}%
\pgfpathlineto{\pgfqpoint{0.967561in}{0.998858in}}%
\pgfpathlineto{\pgfqpoint{0.968710in}{1.171332in}}%
\pgfpathlineto{\pgfqpoint{0.969093in}{1.077255in}}%
\pgfpathlineto{\pgfqpoint{0.969476in}{1.077255in}}%
\pgfpathlineto{\pgfqpoint{0.970624in}{1.024990in}}%
\pgfpathlineto{\pgfqpoint{0.970241in}{1.181785in}}%
\pgfpathlineto{\pgfqpoint{0.971007in}{1.061576in}}%
\pgfpathlineto{\pgfqpoint{0.971390in}{1.061576in}}%
\pgfpathlineto{\pgfqpoint{0.971390in}{1.040670in}}%
\pgfpathlineto{\pgfqpoint{0.971773in}{1.119067in}}%
\pgfpathlineto{\pgfqpoint{0.972921in}{1.087708in}}%
\pgfpathlineto{\pgfqpoint{0.973304in}{1.087708in}}%
\pgfpathlineto{\pgfqpoint{0.974452in}{1.176558in}}%
\pgfpathlineto{\pgfqpoint{0.974835in}{1.030217in}}%
\pgfpathlineto{\pgfqpoint{0.975218in}{1.030217in}}%
\pgfpathlineto{\pgfqpoint{0.975218in}{0.993631in}}%
\pgfpathlineto{\pgfqpoint{0.975601in}{1.092935in}}%
\pgfpathlineto{\pgfqpoint{0.976749in}{1.066802in}}%
\pgfpathlineto{\pgfqpoint{0.977132in}{1.066802in}}%
\pgfpathlineto{\pgfqpoint{0.977515in}{1.134746in}}%
\pgfpathlineto{\pgfqpoint{0.978663in}{1.098161in}}%
\pgfpathlineto{\pgfqpoint{0.979046in}{1.098161in}}%
\pgfpathlineto{\pgfqpoint{0.980195in}{1.160879in}}%
\pgfpathlineto{\pgfqpoint{0.979429in}{1.045896in}}%
\pgfpathlineto{\pgfqpoint{0.980578in}{1.150426in}}%
\pgfpathlineto{\pgfqpoint{0.980960in}{1.150426in}}%
\pgfpathlineto{\pgfqpoint{0.982109in}{1.030217in}}%
\pgfpathlineto{\pgfqpoint{0.981343in}{1.181785in}}%
\pgfpathlineto{\pgfqpoint{0.982492in}{1.160879in}}%
\pgfpathlineto{\pgfqpoint{0.982875in}{1.160879in}}%
\pgfpathlineto{\pgfqpoint{0.984023in}{1.213144in}}%
\pgfpathlineto{\pgfqpoint{0.984406in}{1.066802in}}%
\pgfpathlineto{\pgfqpoint{0.984789in}{1.066802in}}%
\pgfpathlineto{\pgfqpoint{0.985554in}{1.171332in}}%
\pgfpathlineto{\pgfqpoint{0.986320in}{1.098161in}}%
\pgfpathlineto{\pgfqpoint{0.986703in}{1.098161in}}%
\pgfpathlineto{\pgfqpoint{0.986703in}{1.150426in}}%
\pgfpathlineto{\pgfqpoint{0.987469in}{1.077255in}}%
\pgfpathlineto{\pgfqpoint{0.988234in}{1.108614in}}%
\pgfpathlineto{\pgfqpoint{0.988617in}{1.108614in}}%
\pgfpathlineto{\pgfqpoint{0.988617in}{1.139973in}}%
\pgfpathlineto{\pgfqpoint{0.990148in}{1.072029in}}%
\pgfpathlineto{\pgfqpoint{0.990531in}{1.072029in}}%
\pgfpathlineto{\pgfqpoint{0.990914in}{1.166105in}}%
\pgfpathlineto{\pgfqpoint{0.992063in}{1.108614in}}%
\pgfpathlineto{\pgfqpoint{0.992445in}{1.108614in}}%
\pgfpathlineto{\pgfqpoint{0.993211in}{1.087708in}}%
\pgfpathlineto{\pgfqpoint{0.993977in}{1.197464in}}%
\pgfpathlineto{\pgfqpoint{0.994360in}{1.197464in}}%
\pgfpathlineto{\pgfqpoint{0.995125in}{1.119067in}}%
\pgfpathlineto{\pgfqpoint{0.995891in}{1.129520in}}%
\pgfpathlineto{\pgfqpoint{0.996274in}{1.129520in}}%
\pgfpathlineto{\pgfqpoint{0.996274in}{1.004084in}}%
\pgfpathlineto{\pgfqpoint{0.997422in}{1.218370in}}%
\pgfpathlineto{\pgfqpoint{0.997805in}{1.187011in}}%
\pgfpathlineto{\pgfqpoint{0.998188in}{1.187011in}}%
\pgfpathlineto{\pgfqpoint{0.998953in}{1.113840in}}%
\pgfpathlineto{\pgfqpoint{0.998571in}{1.228823in}}%
\pgfpathlineto{\pgfqpoint{0.999719in}{1.129520in}}%
\pgfpathlineto{\pgfqpoint{1.000102in}{1.129520in}}%
\pgfpathlineto{\pgfqpoint{1.000868in}{1.103388in}}%
\pgfpathlineto{\pgfqpoint{1.001633in}{1.113840in}}%
\pgfpathlineto{\pgfqpoint{1.002016in}{1.113840in}}%
\pgfpathlineto{\pgfqpoint{1.002016in}{1.213144in}}%
\pgfpathlineto{\pgfqpoint{1.002782in}{1.014537in}}%
\pgfpathlineto{\pgfqpoint{1.003547in}{1.139973in}}%
\pgfpathlineto{\pgfqpoint{1.003930in}{1.139973in}}%
\pgfpathlineto{\pgfqpoint{1.003930in}{1.103388in}}%
\pgfpathlineto{\pgfqpoint{1.005462in}{1.275861in}}%
\pgfpathlineto{\pgfqpoint{1.005844in}{1.275861in}}%
\pgfpathlineto{\pgfqpoint{1.005844in}{1.301994in}}%
\pgfpathlineto{\pgfqpoint{1.006610in}{1.082482in}}%
\pgfpathlineto{\pgfqpoint{1.007376in}{1.181785in}}%
\pgfpathlineto{\pgfqpoint{1.007759in}{1.181785in}}%
\pgfpathlineto{\pgfqpoint{1.007759in}{1.103388in}}%
\pgfpathlineto{\pgfqpoint{1.009290in}{1.270635in}}%
\pgfpathlineto{\pgfqpoint{1.009673in}{1.270635in}}%
\pgfpathlineto{\pgfqpoint{1.011204in}{1.087708in}}%
\pgfpathlineto{\pgfqpoint{1.011587in}{1.087708in}}%
\pgfpathlineto{\pgfqpoint{1.013118in}{1.218370in}}%
\pgfpathlineto{\pgfqpoint{1.013884in}{1.218370in}}%
\pgfpathlineto{\pgfqpoint{1.013884in}{1.234050in}}%
\pgfpathlineto{\pgfqpoint{1.014267in}{1.150426in}}%
\pgfpathlineto{\pgfqpoint{1.015415in}{1.160879in}}%
\pgfpathlineto{\pgfqpoint{1.015798in}{1.160879in}}%
\pgfpathlineto{\pgfqpoint{1.016946in}{1.322900in}}%
\pgfpathlineto{\pgfqpoint{1.017329in}{1.166105in}}%
\pgfpathlineto{\pgfqpoint{1.017712in}{1.166105in}}%
\pgfpathlineto{\pgfqpoint{1.018861in}{1.213144in}}%
\pgfpathlineto{\pgfqpoint{1.018095in}{1.145199in}}%
\pgfpathlineto{\pgfqpoint{1.019243in}{1.166105in}}%
\pgfpathlineto{\pgfqpoint{1.019626in}{1.166105in}}%
\pgfpathlineto{\pgfqpoint{1.019626in}{1.129520in}}%
\pgfpathlineto{\pgfqpoint{1.021158in}{1.223597in}}%
\pgfpathlineto{\pgfqpoint{1.021540in}{1.223597in}}%
\pgfpathlineto{\pgfqpoint{1.021923in}{1.301994in}}%
\pgfpathlineto{\pgfqpoint{1.023072in}{1.124293in}}%
\pgfpathlineto{\pgfqpoint{1.023455in}{1.124293in}}%
\pgfpathlineto{\pgfqpoint{1.023455in}{1.087708in}}%
\pgfpathlineto{\pgfqpoint{1.023837in}{1.301994in}}%
\pgfpathlineto{\pgfqpoint{1.024986in}{1.202691in}}%
\pgfpathlineto{\pgfqpoint{1.025369in}{1.202691in}}%
\pgfpathlineto{\pgfqpoint{1.025369in}{1.254955in}}%
\pgfpathlineto{\pgfqpoint{1.026517in}{1.019764in}}%
\pgfpathlineto{\pgfqpoint{1.026900in}{1.192238in}}%
\pgfpathlineto{\pgfqpoint{1.027283in}{1.192238in}}%
\pgfpathlineto{\pgfqpoint{1.027666in}{1.234050in}}%
\pgfpathlineto{\pgfqpoint{1.028814in}{1.108614in}}%
\pgfpathlineto{\pgfqpoint{1.029197in}{1.108614in}}%
\pgfpathlineto{\pgfqpoint{1.029963in}{1.213144in}}%
\pgfpathlineto{\pgfqpoint{1.030728in}{1.139973in}}%
\pgfpathlineto{\pgfqpoint{1.031111in}{1.139973in}}%
\pgfpathlineto{\pgfqpoint{1.031111in}{1.207917in}}%
\pgfpathlineto{\pgfqpoint{1.032643in}{1.160879in}}%
\pgfpathlineto{\pgfqpoint{1.033025in}{1.160879in}}%
\pgfpathlineto{\pgfqpoint{1.033408in}{1.249729in}}%
\pgfpathlineto{\pgfqpoint{1.034557in}{1.223597in}}%
\pgfpathlineto{\pgfqpoint{1.034939in}{1.223597in}}%
\pgfpathlineto{\pgfqpoint{1.034939in}{1.160879in}}%
\pgfpathlineto{\pgfqpoint{1.035322in}{1.254955in}}%
\pgfpathlineto{\pgfqpoint{1.036471in}{1.223597in}}%
\pgfpathlineto{\pgfqpoint{1.036854in}{1.223597in}}%
\pgfpathlineto{\pgfqpoint{1.036854in}{1.244503in}}%
\pgfpathlineto{\pgfqpoint{1.038002in}{1.145199in}}%
\pgfpathlineto{\pgfqpoint{1.038385in}{1.223597in}}%
\pgfpathlineto{\pgfqpoint{1.039151in}{1.223597in}}%
\pgfpathlineto{\pgfqpoint{1.039533in}{1.124293in}}%
\pgfpathlineto{\pgfqpoint{1.039916in}{1.286314in}}%
\pgfpathlineto{\pgfqpoint{1.040682in}{1.228823in}}%
\pgfpathlineto{\pgfqpoint{1.041065in}{1.228823in}}%
\pgfpathlineto{\pgfqpoint{1.041065in}{1.129520in}}%
\pgfpathlineto{\pgfqpoint{1.041448in}{1.369938in}}%
\pgfpathlineto{\pgfqpoint{1.042596in}{1.213144in}}%
\pgfpathlineto{\pgfqpoint{1.042979in}{1.213144in}}%
\pgfpathlineto{\pgfqpoint{1.044127in}{1.134746in}}%
\pgfpathlineto{\pgfqpoint{1.043362in}{1.291541in}}%
\pgfpathlineto{\pgfqpoint{1.044510in}{1.197464in}}%
\pgfpathlineto{\pgfqpoint{1.044893in}{1.197464in}}%
\pgfpathlineto{\pgfqpoint{1.044893in}{1.181785in}}%
\pgfpathlineto{\pgfqpoint{1.046424in}{1.270635in}}%
\pgfpathlineto{\pgfqpoint{1.046807in}{1.270635in}}%
\pgfpathlineto{\pgfqpoint{1.047190in}{1.155652in}}%
\pgfpathlineto{\pgfqpoint{1.048339in}{1.228823in}}%
\pgfpathlineto{\pgfqpoint{1.048721in}{1.228823in}}%
\pgfpathlineto{\pgfqpoint{1.049104in}{1.307220in}}%
\pgfpathlineto{\pgfqpoint{1.050253in}{1.166105in}}%
\pgfpathlineto{\pgfqpoint{1.050636in}{1.166105in}}%
\pgfpathlineto{\pgfqpoint{1.051784in}{1.307220in}}%
\pgfpathlineto{\pgfqpoint{1.052167in}{1.213144in}}%
\pgfpathlineto{\pgfqpoint{1.052550in}{1.213144in}}%
\pgfpathlineto{\pgfqpoint{1.052933in}{1.270635in}}%
\pgfpathlineto{\pgfqpoint{1.054081in}{1.066802in}}%
\pgfpathlineto{\pgfqpoint{1.054464in}{1.066802in}}%
\pgfpathlineto{\pgfqpoint{1.055995in}{1.301994in}}%
\pgfpathlineto{\pgfqpoint{1.056378in}{1.301994in}}%
\pgfpathlineto{\pgfqpoint{1.057144in}{1.171332in}}%
\pgfpathlineto{\pgfqpoint{1.056761in}{1.317673in}}%
\pgfpathlineto{\pgfqpoint{1.057909in}{1.291541in}}%
\pgfpathlineto{\pgfqpoint{1.058292in}{1.291541in}}%
\pgfpathlineto{\pgfqpoint{1.059058in}{1.192238in}}%
\pgfpathlineto{\pgfqpoint{1.059823in}{1.281088in}}%
\pgfpathlineto{\pgfqpoint{1.060589in}{1.281088in}}%
\pgfpathlineto{\pgfqpoint{1.062120in}{1.166105in}}%
\pgfpathlineto{\pgfqpoint{1.062503in}{1.166105in}}%
\pgfpathlineto{\pgfqpoint{1.062886in}{1.307220in}}%
\pgfpathlineto{\pgfqpoint{1.064035in}{1.249729in}}%
\pgfpathlineto{\pgfqpoint{1.064417in}{1.249729in}}%
\pgfpathlineto{\pgfqpoint{1.065183in}{1.181785in}}%
\pgfpathlineto{\pgfqpoint{1.064800in}{1.260182in}}%
\pgfpathlineto{\pgfqpoint{1.065949in}{1.213144in}}%
\pgfpathlineto{\pgfqpoint{1.066332in}{1.213144in}}%
\pgfpathlineto{\pgfqpoint{1.067480in}{1.176558in}}%
\pgfpathlineto{\pgfqpoint{1.067097in}{1.333353in}}%
\pgfpathlineto{\pgfqpoint{1.067863in}{1.228823in}}%
\pgfpathlineto{\pgfqpoint{1.068246in}{1.228823in}}%
\pgfpathlineto{\pgfqpoint{1.069394in}{1.291541in}}%
\pgfpathlineto{\pgfqpoint{1.069777in}{1.160879in}}%
\pgfpathlineto{\pgfqpoint{1.070160in}{1.160879in}}%
\pgfpathlineto{\pgfqpoint{1.070926in}{1.301994in}}%
\pgfpathlineto{\pgfqpoint{1.071691in}{1.207917in}}%
\pgfpathlineto{\pgfqpoint{1.072074in}{1.207917in}}%
\pgfpathlineto{\pgfqpoint{1.072457in}{1.338579in}}%
\pgfpathlineto{\pgfqpoint{1.073605in}{1.166105in}}%
\pgfpathlineto{\pgfqpoint{1.073988in}{1.166105in}}%
\pgfpathlineto{\pgfqpoint{1.075137in}{1.396070in}}%
\pgfpathlineto{\pgfqpoint{1.075519in}{1.249729in}}%
\pgfpathlineto{\pgfqpoint{1.075902in}{1.249729in}}%
\pgfpathlineto{\pgfqpoint{1.075902in}{1.265408in}}%
\pgfpathlineto{\pgfqpoint{1.077051in}{1.113840in}}%
\pgfpathlineto{\pgfqpoint{1.077434in}{1.197464in}}%
\pgfpathlineto{\pgfqpoint{1.077816in}{1.197464in}}%
\pgfpathlineto{\pgfqpoint{1.078965in}{1.166105in}}%
\pgfpathlineto{\pgfqpoint{1.079348in}{1.322900in}}%
\pgfpathlineto{\pgfqpoint{1.079731in}{1.322900in}}%
\pgfpathlineto{\pgfqpoint{1.080879in}{1.139973in}}%
\pgfpathlineto{\pgfqpoint{1.081262in}{1.296767in}}%
\pgfpathlineto{\pgfqpoint{1.081645in}{1.296767in}}%
\pgfpathlineto{\pgfqpoint{1.082028in}{1.150426in}}%
\pgfpathlineto{\pgfqpoint{1.082793in}{1.333353in}}%
\pgfpathlineto{\pgfqpoint{1.083176in}{1.249729in}}%
\pgfpathlineto{\pgfqpoint{1.083559in}{1.249729in}}%
\pgfpathlineto{\pgfqpoint{1.083942in}{1.296767in}}%
\pgfpathlineto{\pgfqpoint{1.084707in}{1.145199in}}%
\pgfpathlineto{\pgfqpoint{1.085090in}{1.296767in}}%
\pgfpathlineto{\pgfqpoint{1.085473in}{1.296767in}}%
\pgfpathlineto{\pgfqpoint{1.085473in}{1.181785in}}%
\pgfpathlineto{\pgfqpoint{1.086622in}{1.301994in}}%
\pgfpathlineto{\pgfqpoint{1.087004in}{1.213144in}}%
\pgfpathlineto{\pgfqpoint{1.087387in}{1.213144in}}%
\pgfpathlineto{\pgfqpoint{1.088536in}{1.301994in}}%
\pgfpathlineto{\pgfqpoint{1.088919in}{1.254955in}}%
\pgfpathlineto{\pgfqpoint{1.089301in}{1.254955in}}%
\pgfpathlineto{\pgfqpoint{1.090067in}{1.328126in}}%
\pgfpathlineto{\pgfqpoint{1.090833in}{1.202691in}}%
\pgfpathlineto{\pgfqpoint{1.091216in}{1.202691in}}%
\pgfpathlineto{\pgfqpoint{1.091598in}{1.317673in}}%
\pgfpathlineto{\pgfqpoint{1.092747in}{1.218370in}}%
\pgfpathlineto{\pgfqpoint{1.093130in}{1.218370in}}%
\pgfpathlineto{\pgfqpoint{1.093512in}{1.312447in}}%
\pgfpathlineto{\pgfqpoint{1.093895in}{1.113840in}}%
\pgfpathlineto{\pgfqpoint{1.094661in}{1.265408in}}%
\pgfpathlineto{\pgfqpoint{1.095427in}{1.265408in}}%
\pgfpathlineto{\pgfqpoint{1.095809in}{1.333353in}}%
\pgfpathlineto{\pgfqpoint{1.096575in}{1.239276in}}%
\pgfpathlineto{\pgfqpoint{1.096958in}{1.275861in}}%
\pgfpathlineto{\pgfqpoint{1.097341in}{1.275861in}}%
\pgfpathlineto{\pgfqpoint{1.098489in}{1.317673in}}%
\pgfpathlineto{\pgfqpoint{1.098106in}{1.171332in}}%
\pgfpathlineto{\pgfqpoint{1.098872in}{1.213144in}}%
\pgfpathlineto{\pgfqpoint{1.099255in}{1.213144in}}%
\pgfpathlineto{\pgfqpoint{1.100403in}{1.301994in}}%
\pgfpathlineto{\pgfqpoint{1.100786in}{1.239276in}}%
\pgfpathlineto{\pgfqpoint{1.101169in}{1.239276in}}%
\pgfpathlineto{\pgfqpoint{1.101935in}{1.171332in}}%
\pgfpathlineto{\pgfqpoint{1.101552in}{1.265408in}}%
\pgfpathlineto{\pgfqpoint{1.102700in}{1.228823in}}%
\pgfpathlineto{\pgfqpoint{1.103083in}{1.228823in}}%
\pgfpathlineto{\pgfqpoint{1.103849in}{1.286314in}}%
\pgfpathlineto{\pgfqpoint{1.104615in}{1.082482in}}%
\pgfpathlineto{\pgfqpoint{1.104997in}{1.082482in}}%
\pgfpathlineto{\pgfqpoint{1.105380in}{1.270635in}}%
\pgfpathlineto{\pgfqpoint{1.106529in}{1.265408in}}%
\pgfpathlineto{\pgfqpoint{1.106912in}{1.265408in}}%
\pgfpathlineto{\pgfqpoint{1.108443in}{1.160879in}}%
\pgfpathlineto{\pgfqpoint{1.108826in}{1.160879in}}%
\pgfpathlineto{\pgfqpoint{1.108826in}{1.422203in}}%
\pgfpathlineto{\pgfqpoint{1.110357in}{1.301994in}}%
\pgfpathlineto{\pgfqpoint{1.110740in}{1.301994in}}%
\pgfpathlineto{\pgfqpoint{1.111888in}{1.181785in}}%
\pgfpathlineto{\pgfqpoint{1.111505in}{1.343806in}}%
\pgfpathlineto{\pgfqpoint{1.112271in}{1.202691in}}%
\pgfpathlineto{\pgfqpoint{1.112654in}{1.202691in}}%
\pgfpathlineto{\pgfqpoint{1.113037in}{1.291541in}}%
\pgfpathlineto{\pgfqpoint{1.113802in}{1.134746in}}%
\pgfpathlineto{\pgfqpoint{1.114185in}{1.223597in}}%
\pgfpathlineto{\pgfqpoint{1.114568in}{1.223597in}}%
\pgfpathlineto{\pgfqpoint{1.115717in}{1.181785in}}%
\pgfpathlineto{\pgfqpoint{1.116099in}{1.338579in}}%
\pgfpathlineto{\pgfqpoint{1.116482in}{1.338579in}}%
\pgfpathlineto{\pgfqpoint{1.118014in}{1.187011in}}%
\pgfpathlineto{\pgfqpoint{1.118396in}{1.187011in}}%
\pgfpathlineto{\pgfqpoint{1.119162in}{1.338579in}}%
\pgfpathlineto{\pgfqpoint{1.119928in}{1.192238in}}%
\pgfpathlineto{\pgfqpoint{1.120311in}{1.192238in}}%
\pgfpathlineto{\pgfqpoint{1.121459in}{1.176558in}}%
\pgfpathlineto{\pgfqpoint{1.121842in}{1.333353in}}%
\pgfpathlineto{\pgfqpoint{1.122225in}{1.333353in}}%
\pgfpathlineto{\pgfqpoint{1.122990in}{1.176558in}}%
\pgfpathlineto{\pgfqpoint{1.122608in}{1.375165in}}%
\pgfpathlineto{\pgfqpoint{1.123756in}{1.223597in}}%
\pgfpathlineto{\pgfqpoint{1.124139in}{1.223597in}}%
\pgfpathlineto{\pgfqpoint{1.124139in}{1.197464in}}%
\pgfpathlineto{\pgfqpoint{1.124522in}{1.265408in}}%
\pgfpathlineto{\pgfqpoint{1.125670in}{1.234050in}}%
\pgfpathlineto{\pgfqpoint{1.126053in}{1.234050in}}%
\pgfpathlineto{\pgfqpoint{1.126436in}{1.066802in}}%
\pgfpathlineto{\pgfqpoint{1.127584in}{1.312447in}}%
\pgfpathlineto{\pgfqpoint{1.127967in}{1.312447in}}%
\pgfpathlineto{\pgfqpoint{1.129499in}{1.207917in}}%
\pgfpathlineto{\pgfqpoint{1.129881in}{1.207917in}}%
\pgfpathlineto{\pgfqpoint{1.131030in}{1.343806in}}%
\pgfpathlineto{\pgfqpoint{1.131413in}{1.244503in}}%
\pgfpathlineto{\pgfqpoint{1.131795in}{1.244503in}}%
\pgfpathlineto{\pgfqpoint{1.132178in}{1.270635in}}%
\pgfpathlineto{\pgfqpoint{1.132944in}{1.145199in}}%
\pgfpathlineto{\pgfqpoint{1.133327in}{1.260182in}}%
\pgfpathlineto{\pgfqpoint{1.133710in}{1.260182in}}%
\pgfpathlineto{\pgfqpoint{1.133710in}{1.317673in}}%
\pgfpathlineto{\pgfqpoint{1.134858in}{1.181785in}}%
\pgfpathlineto{\pgfqpoint{1.135241in}{1.254955in}}%
\pgfpathlineto{\pgfqpoint{1.135624in}{1.254955in}}%
\pgfpathlineto{\pgfqpoint{1.136772in}{1.286314in}}%
\pgfpathlineto{\pgfqpoint{1.137155in}{1.134746in}}%
\pgfpathlineto{\pgfqpoint{1.137538in}{1.134746in}}%
\pgfpathlineto{\pgfqpoint{1.137538in}{1.328126in}}%
\pgfpathlineto{\pgfqpoint{1.139069in}{1.166105in}}%
\pgfpathlineto{\pgfqpoint{1.139452in}{1.166105in}}%
\pgfpathlineto{\pgfqpoint{1.140983in}{1.296767in}}%
\pgfpathlineto{\pgfqpoint{1.141366in}{1.296767in}}%
\pgfpathlineto{\pgfqpoint{1.142515in}{1.187011in}}%
\pgfpathlineto{\pgfqpoint{1.142132in}{1.343806in}}%
\pgfpathlineto{\pgfqpoint{1.142898in}{1.234050in}}%
\pgfpathlineto{\pgfqpoint{1.143280in}{1.234050in}}%
\pgfpathlineto{\pgfqpoint{1.143280in}{1.296767in}}%
\pgfpathlineto{\pgfqpoint{1.143663in}{1.150426in}}%
\pgfpathlineto{\pgfqpoint{1.144812in}{1.265408in}}%
\pgfpathlineto{\pgfqpoint{1.145195in}{1.265408in}}%
\pgfpathlineto{\pgfqpoint{1.145195in}{1.166105in}}%
\pgfpathlineto{\pgfqpoint{1.146726in}{1.213144in}}%
\pgfpathlineto{\pgfqpoint{1.147109in}{1.213144in}}%
\pgfpathlineto{\pgfqpoint{1.147109in}{1.087708in}}%
\pgfpathlineto{\pgfqpoint{1.147874in}{1.286314in}}%
\pgfpathlineto{\pgfqpoint{1.148640in}{1.234050in}}%
\pgfpathlineto{\pgfqpoint{1.149023in}{1.234050in}}%
\pgfpathlineto{\pgfqpoint{1.149023in}{1.145199in}}%
\pgfpathlineto{\pgfqpoint{1.150554in}{1.312447in}}%
\pgfpathlineto{\pgfqpoint{1.150937in}{1.312447in}}%
\pgfpathlineto{\pgfqpoint{1.150937in}{1.338579in}}%
\pgfpathlineto{\pgfqpoint{1.151320in}{1.139973in}}%
\pgfpathlineto{\pgfqpoint{1.152468in}{1.296767in}}%
\pgfpathlineto{\pgfqpoint{1.152851in}{1.296767in}}%
\pgfpathlineto{\pgfqpoint{1.152851in}{1.218370in}}%
\pgfpathlineto{\pgfqpoint{1.154382in}{1.322900in}}%
\pgfpathlineto{\pgfqpoint{1.154765in}{1.322900in}}%
\pgfpathlineto{\pgfqpoint{1.156297in}{1.103388in}}%
\pgfpathlineto{\pgfqpoint{1.156679in}{1.103388in}}%
\pgfpathlineto{\pgfqpoint{1.156679in}{1.270635in}}%
\pgfpathlineto{\pgfqpoint{1.158211in}{1.160879in}}%
\pgfpathlineto{\pgfqpoint{1.158594in}{1.160879in}}%
\pgfpathlineto{\pgfqpoint{1.158976in}{1.296767in}}%
\pgfpathlineto{\pgfqpoint{1.160125in}{1.150426in}}%
\pgfpathlineto{\pgfqpoint{1.160508in}{1.150426in}}%
\pgfpathlineto{\pgfqpoint{1.160508in}{1.119067in}}%
\pgfpathlineto{\pgfqpoint{1.161656in}{1.296767in}}%
\pgfpathlineto{\pgfqpoint{1.162039in}{1.218370in}}%
\pgfpathlineto{\pgfqpoint{1.162422in}{1.218370in}}%
\pgfpathlineto{\pgfqpoint{1.163188in}{1.291541in}}%
\pgfpathlineto{\pgfqpoint{1.163570in}{1.113840in}}%
\pgfpathlineto{\pgfqpoint{1.163953in}{1.234050in}}%
\pgfpathlineto{\pgfqpoint{1.164336in}{1.234050in}}%
\pgfpathlineto{\pgfqpoint{1.164336in}{1.385617in}}%
\pgfpathlineto{\pgfqpoint{1.164719in}{1.145199in}}%
\pgfpathlineto{\pgfqpoint{1.165867in}{1.187011in}}%
\pgfpathlineto{\pgfqpoint{1.166250in}{1.187011in}}%
\pgfpathlineto{\pgfqpoint{1.167016in}{1.338579in}}%
\pgfpathlineto{\pgfqpoint{1.167782in}{1.223597in}}%
\pgfpathlineto{\pgfqpoint{1.168164in}{1.223597in}}%
\pgfpathlineto{\pgfqpoint{1.169313in}{1.301994in}}%
\pgfpathlineto{\pgfqpoint{1.169696in}{1.119067in}}%
\pgfpathlineto{\pgfqpoint{1.170078in}{1.119067in}}%
\pgfpathlineto{\pgfqpoint{1.171227in}{1.265408in}}%
\pgfpathlineto{\pgfqpoint{1.171610in}{1.139973in}}%
\pgfpathlineto{\pgfqpoint{1.171993in}{1.139973in}}%
\pgfpathlineto{\pgfqpoint{1.171993in}{1.234050in}}%
\pgfpathlineto{\pgfqpoint{1.173141in}{1.087708in}}%
\pgfpathlineto{\pgfqpoint{1.173524in}{1.213144in}}%
\pgfpathlineto{\pgfqpoint{1.173907in}{1.213144in}}%
\pgfpathlineto{\pgfqpoint{1.174290in}{1.218370in}}%
\pgfpathlineto{\pgfqpoint{1.175438in}{1.129520in}}%
\pgfpathlineto{\pgfqpoint{1.175821in}{1.129520in}}%
\pgfpathlineto{\pgfqpoint{1.176587in}{1.296767in}}%
\pgfpathlineto{\pgfqpoint{1.177352in}{1.228823in}}%
\pgfpathlineto{\pgfqpoint{1.177735in}{1.228823in}}%
\pgfpathlineto{\pgfqpoint{1.178118in}{1.155652in}}%
\pgfpathlineto{\pgfqpoint{1.178501in}{1.312447in}}%
\pgfpathlineto{\pgfqpoint{1.179266in}{1.213144in}}%
\pgfpathlineto{\pgfqpoint{1.179649in}{1.213144in}}%
\pgfpathlineto{\pgfqpoint{1.180032in}{1.322900in}}%
\pgfpathlineto{\pgfqpoint{1.181181in}{1.139973in}}%
\pgfpathlineto{\pgfqpoint{1.181563in}{1.139973in}}%
\pgfpathlineto{\pgfqpoint{1.181563in}{1.124293in}}%
\pgfpathlineto{\pgfqpoint{1.183095in}{1.244503in}}%
\pgfpathlineto{\pgfqpoint{1.183478in}{1.244503in}}%
\pgfpathlineto{\pgfqpoint{1.184243in}{1.171332in}}%
\pgfpathlineto{\pgfqpoint{1.185009in}{1.265408in}}%
\pgfpathlineto{\pgfqpoint{1.185392in}{1.265408in}}%
\pgfpathlineto{\pgfqpoint{1.186540in}{1.192238in}}%
\pgfpathlineto{\pgfqpoint{1.186923in}{1.207917in}}%
\pgfpathlineto{\pgfqpoint{1.187306in}{1.207917in}}%
\pgfpathlineto{\pgfqpoint{1.187306in}{1.254955in}}%
\pgfpathlineto{\pgfqpoint{1.188072in}{1.155652in}}%
\pgfpathlineto{\pgfqpoint{1.188837in}{1.223597in}}%
\pgfpathlineto{\pgfqpoint{1.189220in}{1.223597in}}%
\pgfpathlineto{\pgfqpoint{1.190751in}{1.134746in}}%
\pgfpathlineto{\pgfqpoint{1.191134in}{1.134746in}}%
\pgfpathlineto{\pgfqpoint{1.191517in}{1.239276in}}%
\pgfpathlineto{\pgfqpoint{1.192665in}{1.207917in}}%
\pgfpathlineto{\pgfqpoint{1.193048in}{1.207917in}}%
\pgfpathlineto{\pgfqpoint{1.193048in}{1.181785in}}%
\pgfpathlineto{\pgfqpoint{1.193431in}{1.228823in}}%
\pgfpathlineto{\pgfqpoint{1.194580in}{1.207917in}}%
\pgfpathlineto{\pgfqpoint{1.194962in}{1.207917in}}%
\pgfpathlineto{\pgfqpoint{1.195345in}{1.061576in}}%
\pgfpathlineto{\pgfqpoint{1.196494in}{1.249729in}}%
\pgfpathlineto{\pgfqpoint{1.196877in}{1.249729in}}%
\pgfpathlineto{\pgfqpoint{1.197259in}{1.108614in}}%
\pgfpathlineto{\pgfqpoint{1.198408in}{1.223597in}}%
\pgfpathlineto{\pgfqpoint{1.198791in}{1.223597in}}%
\pgfpathlineto{\pgfqpoint{1.199939in}{1.134746in}}%
\pgfpathlineto{\pgfqpoint{1.199556in}{1.301994in}}%
\pgfpathlineto{\pgfqpoint{1.200322in}{1.181785in}}%
\pgfpathlineto{\pgfqpoint{1.200705in}{1.181785in}}%
\pgfpathlineto{\pgfqpoint{1.201088in}{1.145199in}}%
\pgfpathlineto{\pgfqpoint{1.202236in}{1.286314in}}%
\pgfpathlineto{\pgfqpoint{1.202619in}{1.286314in}}%
\pgfpathlineto{\pgfqpoint{1.204150in}{1.014537in}}%
\pgfpathlineto{\pgfqpoint{1.204533in}{1.014537in}}%
\pgfpathlineto{\pgfqpoint{1.206065in}{1.317673in}}%
\pgfpathlineto{\pgfqpoint{1.206447in}{1.317673in}}%
\pgfpathlineto{\pgfqpoint{1.207596in}{1.160879in}}%
\pgfpathlineto{\pgfqpoint{1.207979in}{1.197464in}}%
\pgfpathlineto{\pgfqpoint{1.208361in}{1.197464in}}%
\pgfpathlineto{\pgfqpoint{1.209510in}{1.228823in}}%
\pgfpathlineto{\pgfqpoint{1.208744in}{1.150426in}}%
\pgfpathlineto{\pgfqpoint{1.209893in}{1.223597in}}%
\pgfpathlineto{\pgfqpoint{1.210276in}{1.223597in}}%
\pgfpathlineto{\pgfqpoint{1.211424in}{1.077255in}}%
\pgfpathlineto{\pgfqpoint{1.211041in}{1.244503in}}%
\pgfpathlineto{\pgfqpoint{1.211807in}{1.223597in}}%
\pgfpathlineto{\pgfqpoint{1.212190in}{1.223597in}}%
\pgfpathlineto{\pgfqpoint{1.212573in}{1.134746in}}%
\pgfpathlineto{\pgfqpoint{1.213721in}{1.275861in}}%
\pgfpathlineto{\pgfqpoint{1.214104in}{1.275861in}}%
\pgfpathlineto{\pgfqpoint{1.215635in}{1.134746in}}%
\pgfpathlineto{\pgfqpoint{1.216018in}{1.134746in}}%
\pgfpathlineto{\pgfqpoint{1.216784in}{1.202691in}}%
\pgfpathlineto{\pgfqpoint{1.217167in}{1.098161in}}%
\pgfpathlineto{\pgfqpoint{1.217549in}{1.129520in}}%
\pgfpathlineto{\pgfqpoint{1.217932in}{1.129520in}}%
\pgfpathlineto{\pgfqpoint{1.217932in}{1.066802in}}%
\pgfpathlineto{\pgfqpoint{1.218315in}{1.270635in}}%
\pgfpathlineto{\pgfqpoint{1.219464in}{1.160879in}}%
\pgfpathlineto{\pgfqpoint{1.219846in}{1.160879in}}%
\pgfpathlineto{\pgfqpoint{1.219846in}{1.134746in}}%
\pgfpathlineto{\pgfqpoint{1.220995in}{1.181785in}}%
\pgfpathlineto{\pgfqpoint{1.221378in}{1.176558in}}%
\pgfpathlineto{\pgfqpoint{1.221761in}{1.176558in}}%
\pgfpathlineto{\pgfqpoint{1.221761in}{1.119067in}}%
\pgfpathlineto{\pgfqpoint{1.222909in}{1.328126in}}%
\pgfpathlineto{\pgfqpoint{1.223292in}{1.260182in}}%
\pgfpathlineto{\pgfqpoint{1.223675in}{1.260182in}}%
\pgfpathlineto{\pgfqpoint{1.223675in}{1.077255in}}%
\pgfpathlineto{\pgfqpoint{1.225206in}{1.244503in}}%
\pgfpathlineto{\pgfqpoint{1.225589in}{1.244503in}}%
\pgfpathlineto{\pgfqpoint{1.225972in}{1.066802in}}%
\pgfpathlineto{\pgfqpoint{1.226737in}{1.249729in}}%
\pgfpathlineto{\pgfqpoint{1.227120in}{1.155652in}}%
\pgfpathlineto{\pgfqpoint{1.227503in}{1.155652in}}%
\pgfpathlineto{\pgfqpoint{1.227503in}{1.192238in}}%
\pgfpathlineto{\pgfqpoint{1.227886in}{1.072029in}}%
\pgfpathlineto{\pgfqpoint{1.229034in}{1.134746in}}%
\pgfpathlineto{\pgfqpoint{1.229417in}{1.134746in}}%
\pgfpathlineto{\pgfqpoint{1.229417in}{1.108614in}}%
\pgfpathlineto{\pgfqpoint{1.230183in}{1.228823in}}%
\pgfpathlineto{\pgfqpoint{1.230948in}{1.192238in}}%
\pgfpathlineto{\pgfqpoint{1.231331in}{1.192238in}}%
\pgfpathlineto{\pgfqpoint{1.231331in}{1.145199in}}%
\pgfpathlineto{\pgfqpoint{1.232863in}{1.197464in}}%
\pgfpathlineto{\pgfqpoint{1.233245in}{1.197464in}}%
\pgfpathlineto{\pgfqpoint{1.233245in}{1.082482in}}%
\pgfpathlineto{\pgfqpoint{1.233628in}{1.207917in}}%
\pgfpathlineto{\pgfqpoint{1.234777in}{1.129520in}}%
\pgfpathlineto{\pgfqpoint{1.235160in}{1.129520in}}%
\pgfpathlineto{\pgfqpoint{1.235160in}{1.197464in}}%
\pgfpathlineto{\pgfqpoint{1.236691in}{1.155652in}}%
\pgfpathlineto{\pgfqpoint{1.237074in}{1.155652in}}%
\pgfpathlineto{\pgfqpoint{1.238222in}{1.045896in}}%
\pgfpathlineto{\pgfqpoint{1.237457in}{1.249729in}}%
\pgfpathlineto{\pgfqpoint{1.238605in}{1.139973in}}%
\pgfpathlineto{\pgfqpoint{1.238988in}{1.139973in}}%
\pgfpathlineto{\pgfqpoint{1.239754in}{1.181785in}}%
\pgfpathlineto{\pgfqpoint{1.240136in}{1.129520in}}%
\pgfpathlineto{\pgfqpoint{1.240519in}{1.139973in}}%
\pgfpathlineto{\pgfqpoint{1.240902in}{1.139973in}}%
\pgfpathlineto{\pgfqpoint{1.240902in}{1.218370in}}%
\pgfpathlineto{\pgfqpoint{1.241668in}{1.119067in}}%
\pgfpathlineto{\pgfqpoint{1.242433in}{1.192238in}}%
\pgfpathlineto{\pgfqpoint{1.242816in}{1.192238in}}%
\pgfpathlineto{\pgfqpoint{1.242816in}{1.056349in}}%
\pgfpathlineto{\pgfqpoint{1.244348in}{1.192238in}}%
\pgfpathlineto{\pgfqpoint{1.244730in}{1.192238in}}%
\pgfpathlineto{\pgfqpoint{1.246262in}{1.119067in}}%
\pgfpathlineto{\pgfqpoint{1.246644in}{1.119067in}}%
\pgfpathlineto{\pgfqpoint{1.246644in}{1.192238in}}%
\pgfpathlineto{\pgfqpoint{1.247027in}{1.108614in}}%
\pgfpathlineto{\pgfqpoint{1.248176in}{1.171332in}}%
\pgfpathlineto{\pgfqpoint{1.248559in}{1.171332in}}%
\pgfpathlineto{\pgfqpoint{1.249707in}{1.213144in}}%
\pgfpathlineto{\pgfqpoint{1.250090in}{1.145199in}}%
\pgfpathlineto{\pgfqpoint{1.250473in}{1.145199in}}%
\pgfpathlineto{\pgfqpoint{1.251621in}{1.207917in}}%
\pgfpathlineto{\pgfqpoint{1.252004in}{1.092935in}}%
\pgfpathlineto{\pgfqpoint{1.252387in}{1.092935in}}%
\pgfpathlineto{\pgfqpoint{1.252387in}{1.176558in}}%
\pgfpathlineto{\pgfqpoint{1.253535in}{0.972726in}}%
\pgfpathlineto{\pgfqpoint{1.253918in}{1.171332in}}%
\pgfpathlineto{\pgfqpoint{1.254301in}{1.171332in}}%
\pgfpathlineto{\pgfqpoint{1.255450in}{1.040670in}}%
\pgfpathlineto{\pgfqpoint{1.255832in}{1.134746in}}%
\pgfpathlineto{\pgfqpoint{1.256215in}{1.134746in}}%
\pgfpathlineto{\pgfqpoint{1.256981in}{1.192238in}}%
\pgfpathlineto{\pgfqpoint{1.257747in}{1.066802in}}%
\pgfpathlineto{\pgfqpoint{1.258129in}{1.066802in}}%
\pgfpathlineto{\pgfqpoint{1.258895in}{1.160879in}}%
\pgfpathlineto{\pgfqpoint{1.259661in}{1.072029in}}%
\pgfpathlineto{\pgfqpoint{1.260044in}{1.072029in}}%
\pgfpathlineto{\pgfqpoint{1.261192in}{1.234050in}}%
\pgfpathlineto{\pgfqpoint{1.260426in}{1.024990in}}%
\pgfpathlineto{\pgfqpoint{1.261575in}{1.176558in}}%
\pgfpathlineto{\pgfqpoint{1.261958in}{1.176558in}}%
\pgfpathlineto{\pgfqpoint{1.262341in}{1.004084in}}%
\pgfpathlineto{\pgfqpoint{1.263489in}{1.072029in}}%
\pgfpathlineto{\pgfqpoint{1.263872in}{1.072029in}}%
\pgfpathlineto{\pgfqpoint{1.265403in}{1.234050in}}%
\pgfpathlineto{\pgfqpoint{1.265786in}{1.234050in}}%
\pgfpathlineto{\pgfqpoint{1.267317in}{1.035443in}}%
\pgfpathlineto{\pgfqpoint{1.267700in}{1.035443in}}%
\pgfpathlineto{\pgfqpoint{1.268849in}{1.171332in}}%
\pgfpathlineto{\pgfqpoint{1.269231in}{1.145199in}}%
\pgfpathlineto{\pgfqpoint{1.269614in}{1.145199in}}%
\pgfpathlineto{\pgfqpoint{1.269614in}{1.207917in}}%
\pgfpathlineto{\pgfqpoint{1.269997in}{1.019764in}}%
\pgfpathlineto{\pgfqpoint{1.271146in}{1.197464in}}%
\pgfpathlineto{\pgfqpoint{1.271528in}{1.197464in}}%
\pgfpathlineto{\pgfqpoint{1.272294in}{1.098161in}}%
\pgfpathlineto{\pgfqpoint{1.273060in}{1.145199in}}%
\pgfpathlineto{\pgfqpoint{1.273443in}{1.145199in}}%
\pgfpathlineto{\pgfqpoint{1.273825in}{1.072029in}}%
\pgfpathlineto{\pgfqpoint{1.274208in}{1.155652in}}%
\pgfpathlineto{\pgfqpoint{1.274974in}{1.139973in}}%
\pgfpathlineto{\pgfqpoint{1.275357in}{1.139973in}}%
\pgfpathlineto{\pgfqpoint{1.275740in}{1.207917in}}%
\pgfpathlineto{\pgfqpoint{1.276122in}{1.035443in}}%
\pgfpathlineto{\pgfqpoint{1.276888in}{1.108614in}}%
\pgfpathlineto{\pgfqpoint{1.277271in}{1.108614in}}%
\pgfpathlineto{\pgfqpoint{1.278037in}{1.187011in}}%
\pgfpathlineto{\pgfqpoint{1.278802in}{1.072029in}}%
\pgfpathlineto{\pgfqpoint{1.279185in}{1.072029in}}%
\pgfpathlineto{\pgfqpoint{1.279568in}{1.249729in}}%
\pgfpathlineto{\pgfqpoint{1.280716in}{1.187011in}}%
\pgfpathlineto{\pgfqpoint{1.281099in}{1.187011in}}%
\pgfpathlineto{\pgfqpoint{1.281482in}{1.066802in}}%
\pgfpathlineto{\pgfqpoint{1.281865in}{1.265408in}}%
\pgfpathlineto{\pgfqpoint{1.282631in}{1.113840in}}%
\pgfpathlineto{\pgfqpoint{1.283013in}{1.113840in}}%
\pgfpathlineto{\pgfqpoint{1.283779in}{1.035443in}}%
\pgfpathlineto{\pgfqpoint{1.284545in}{1.187011in}}%
\pgfpathlineto{\pgfqpoint{1.284928in}{1.187011in}}%
\pgfpathlineto{\pgfqpoint{1.285310in}{1.045896in}}%
\pgfpathlineto{\pgfqpoint{1.285310in}{1.265408in}}%
\pgfpathlineto{\pgfqpoint{1.286459in}{1.155652in}}%
\pgfpathlineto{\pgfqpoint{1.286842in}{1.155652in}}%
\pgfpathlineto{\pgfqpoint{1.287990in}{1.176558in}}%
\pgfpathlineto{\pgfqpoint{1.288373in}{1.035443in}}%
\pgfpathlineto{\pgfqpoint{1.288756in}{1.035443in}}%
\pgfpathlineto{\pgfqpoint{1.289521in}{1.108614in}}%
\pgfpathlineto{\pgfqpoint{1.290287in}{1.056349in}}%
\pgfpathlineto{\pgfqpoint{1.290670in}{1.056349in}}%
\pgfpathlineto{\pgfqpoint{1.291053in}{1.145199in}}%
\pgfpathlineto{\pgfqpoint{1.292201in}{1.092935in}}%
\pgfpathlineto{\pgfqpoint{1.292584in}{1.092935in}}%
\pgfpathlineto{\pgfqpoint{1.292584in}{1.066802in}}%
\pgfpathlineto{\pgfqpoint{1.293733in}{1.145199in}}%
\pgfpathlineto{\pgfqpoint{1.294115in}{1.139973in}}%
\pgfpathlineto{\pgfqpoint{1.294498in}{1.139973in}}%
\pgfpathlineto{\pgfqpoint{1.294881in}{1.176558in}}%
\pgfpathlineto{\pgfqpoint{1.295647in}{1.056349in}}%
\pgfpathlineto{\pgfqpoint{1.296030in}{1.077255in}}%
\pgfpathlineto{\pgfqpoint{1.296412in}{1.077255in}}%
\pgfpathlineto{\pgfqpoint{1.297561in}{1.056349in}}%
\pgfpathlineto{\pgfqpoint{1.296795in}{1.202691in}}%
\pgfpathlineto{\pgfqpoint{1.297944in}{1.103388in}}%
\pgfpathlineto{\pgfqpoint{1.298327in}{1.103388in}}%
\pgfpathlineto{\pgfqpoint{1.298327in}{1.077255in}}%
\pgfpathlineto{\pgfqpoint{1.299858in}{1.166105in}}%
\pgfpathlineto{\pgfqpoint{1.300241in}{1.166105in}}%
\pgfpathlineto{\pgfqpoint{1.300241in}{1.082482in}}%
\pgfpathlineto{\pgfqpoint{1.301772in}{1.166105in}}%
\pgfpathlineto{\pgfqpoint{1.302155in}{1.166105in}}%
\pgfpathlineto{\pgfqpoint{1.302538in}{0.993631in}}%
\pgfpathlineto{\pgfqpoint{1.302921in}{1.171332in}}%
\pgfpathlineto{\pgfqpoint{1.303686in}{1.077255in}}%
\pgfpathlineto{\pgfqpoint{1.304069in}{1.077255in}}%
\pgfpathlineto{\pgfqpoint{1.304069in}{1.181785in}}%
\pgfpathlineto{\pgfqpoint{1.305217in}{1.045896in}}%
\pgfpathlineto{\pgfqpoint{1.305600in}{1.119067in}}%
\pgfpathlineto{\pgfqpoint{1.305983in}{1.119067in}}%
\pgfpathlineto{\pgfqpoint{1.306366in}{1.223597in}}%
\pgfpathlineto{\pgfqpoint{1.307514in}{1.134746in}}%
\pgfpathlineto{\pgfqpoint{1.307897in}{1.134746in}}%
\pgfpathlineto{\pgfqpoint{1.308280in}{0.983178in}}%
\pgfpathlineto{\pgfqpoint{1.309046in}{1.139973in}}%
\pgfpathlineto{\pgfqpoint{1.309429in}{1.129520in}}%
\pgfpathlineto{\pgfqpoint{1.309811in}{1.129520in}}%
\pgfpathlineto{\pgfqpoint{1.310194in}{1.045896in}}%
\pgfpathlineto{\pgfqpoint{1.310960in}{1.265408in}}%
\pgfpathlineto{\pgfqpoint{1.311343in}{1.228823in}}%
\pgfpathlineto{\pgfqpoint{1.311726in}{1.228823in}}%
\pgfpathlineto{\pgfqpoint{1.311726in}{1.019764in}}%
\pgfpathlineto{\pgfqpoint{1.313257in}{1.045896in}}%
\pgfpathlineto{\pgfqpoint{1.313640in}{1.045896in}}%
\pgfpathlineto{\pgfqpoint{1.314405in}{1.239276in}}%
\pgfpathlineto{\pgfqpoint{1.315171in}{1.181785in}}%
\pgfpathlineto{\pgfqpoint{1.315554in}{1.181785in}}%
\pgfpathlineto{\pgfqpoint{1.315554in}{1.014537in}}%
\pgfpathlineto{\pgfqpoint{1.317085in}{1.077255in}}%
\pgfpathlineto{\pgfqpoint{1.317468in}{1.077255in}}%
\pgfpathlineto{\pgfqpoint{1.317851in}{1.040670in}}%
\pgfpathlineto{\pgfqpoint{1.318999in}{1.155652in}}%
\pgfpathlineto{\pgfqpoint{1.319382in}{1.155652in}}%
\pgfpathlineto{\pgfqpoint{1.319765in}{1.056349in}}%
\pgfpathlineto{\pgfqpoint{1.320914in}{1.087708in}}%
\pgfpathlineto{\pgfqpoint{1.321296in}{1.087708in}}%
\pgfpathlineto{\pgfqpoint{1.321679in}{1.171332in}}%
\pgfpathlineto{\pgfqpoint{1.322445in}{1.072029in}}%
\pgfpathlineto{\pgfqpoint{1.322828in}{1.160879in}}%
\pgfpathlineto{\pgfqpoint{1.323211in}{1.160879in}}%
\pgfpathlineto{\pgfqpoint{1.324359in}{1.082482in}}%
\pgfpathlineto{\pgfqpoint{1.323976in}{1.197464in}}%
\pgfpathlineto{\pgfqpoint{1.324742in}{1.108614in}}%
\pgfpathlineto{\pgfqpoint{1.325125in}{1.108614in}}%
\pgfpathlineto{\pgfqpoint{1.326273in}{1.155652in}}%
\pgfpathlineto{\pgfqpoint{1.325890in}{0.993631in}}%
\pgfpathlineto{\pgfqpoint{1.326656in}{1.040670in}}%
\pgfpathlineto{\pgfqpoint{1.327039in}{1.040670in}}%
\pgfpathlineto{\pgfqpoint{1.327422in}{1.171332in}}%
\pgfpathlineto{\pgfqpoint{1.328570in}{1.108614in}}%
\pgfpathlineto{\pgfqpoint{1.328953in}{1.108614in}}%
\pgfpathlineto{\pgfqpoint{1.329336in}{1.181785in}}%
\pgfpathlineto{\pgfqpoint{1.329719in}{1.066802in}}%
\pgfpathlineto{\pgfqpoint{1.330484in}{1.092935in}}%
\pgfpathlineto{\pgfqpoint{1.330867in}{1.092935in}}%
\pgfpathlineto{\pgfqpoint{1.331250in}{1.166105in}}%
\pgfpathlineto{\pgfqpoint{1.331633in}{1.024990in}}%
\pgfpathlineto{\pgfqpoint{1.332398in}{1.124293in}}%
\pgfpathlineto{\pgfqpoint{1.332781in}{1.124293in}}%
\pgfpathlineto{\pgfqpoint{1.333930in}{1.030217in}}%
\pgfpathlineto{\pgfqpoint{1.333164in}{1.134746in}}%
\pgfpathlineto{\pgfqpoint{1.334313in}{1.134746in}}%
\pgfpathlineto{\pgfqpoint{1.334695in}{1.134746in}}%
\pgfpathlineto{\pgfqpoint{1.334695in}{1.166105in}}%
\pgfpathlineto{\pgfqpoint{1.336227in}{1.045896in}}%
\pgfpathlineto{\pgfqpoint{1.336610in}{1.045896in}}%
\pgfpathlineto{\pgfqpoint{1.338141in}{1.181785in}}%
\pgfpathlineto{\pgfqpoint{1.338524in}{1.181785in}}%
\pgfpathlineto{\pgfqpoint{1.338907in}{1.051123in}}%
\pgfpathlineto{\pgfqpoint{1.340055in}{1.061576in}}%
\pgfpathlineto{\pgfqpoint{1.340438in}{1.061576in}}%
\pgfpathlineto{\pgfqpoint{1.341204in}{1.197464in}}%
\pgfpathlineto{\pgfqpoint{1.341969in}{1.009311in}}%
\pgfpathlineto{\pgfqpoint{1.342352in}{1.009311in}}%
\pgfpathlineto{\pgfqpoint{1.342735in}{1.192238in}}%
\pgfpathlineto{\pgfqpoint{1.343883in}{1.103388in}}%
\pgfpathlineto{\pgfqpoint{1.344266in}{1.103388in}}%
\pgfpathlineto{\pgfqpoint{1.344266in}{1.207917in}}%
\pgfpathlineto{\pgfqpoint{1.344649in}{1.066802in}}%
\pgfpathlineto{\pgfqpoint{1.345797in}{1.187011in}}%
\pgfpathlineto{\pgfqpoint{1.346180in}{1.187011in}}%
\pgfpathlineto{\pgfqpoint{1.347712in}{0.967499in}}%
\pgfpathlineto{\pgfqpoint{1.348094in}{0.967499in}}%
\pgfpathlineto{\pgfqpoint{1.349243in}{1.218370in}}%
\pgfpathlineto{\pgfqpoint{1.349626in}{1.082482in}}%
\pgfpathlineto{\pgfqpoint{1.350009in}{1.082482in}}%
\pgfpathlineto{\pgfqpoint{1.350009in}{1.192238in}}%
\pgfpathlineto{\pgfqpoint{1.350391in}{1.066802in}}%
\pgfpathlineto{\pgfqpoint{1.351540in}{1.129520in}}%
\pgfpathlineto{\pgfqpoint{1.351923in}{1.129520in}}%
\pgfpathlineto{\pgfqpoint{1.353071in}{1.045896in}}%
\pgfpathlineto{\pgfqpoint{1.352688in}{1.155652in}}%
\pgfpathlineto{\pgfqpoint{1.353454in}{1.045896in}}%
\pgfpathlineto{\pgfqpoint{1.353837in}{1.045896in}}%
\pgfpathlineto{\pgfqpoint{1.353837in}{0.993631in}}%
\pgfpathlineto{\pgfqpoint{1.354220in}{1.119067in}}%
\pgfpathlineto{\pgfqpoint{1.355368in}{1.082482in}}%
\pgfpathlineto{\pgfqpoint{1.355751in}{1.082482in}}%
\pgfpathlineto{\pgfqpoint{1.355751in}{1.072029in}}%
\pgfpathlineto{\pgfqpoint{1.356517in}{1.239276in}}%
\pgfpathlineto{\pgfqpoint{1.357282in}{1.113840in}}%
\pgfpathlineto{\pgfqpoint{1.357665in}{1.113840in}}%
\pgfpathlineto{\pgfqpoint{1.358048in}{1.139973in}}%
\pgfpathlineto{\pgfqpoint{1.359197in}{1.056349in}}%
\pgfpathlineto{\pgfqpoint{1.359579in}{1.056349in}}%
\pgfpathlineto{\pgfqpoint{1.359962in}{1.197464in}}%
\pgfpathlineto{\pgfqpoint{1.361111in}{1.113840in}}%
\pgfpathlineto{\pgfqpoint{1.361494in}{1.113840in}}%
\pgfpathlineto{\pgfqpoint{1.361876in}{1.040670in}}%
\pgfpathlineto{\pgfqpoint{1.362642in}{1.166105in}}%
\pgfpathlineto{\pgfqpoint{1.363025in}{1.072029in}}%
\pgfpathlineto{\pgfqpoint{1.363408in}{1.072029in}}%
\pgfpathlineto{\pgfqpoint{1.364173in}{1.192238in}}%
\pgfpathlineto{\pgfqpoint{1.364556in}{1.066802in}}%
\pgfpathlineto{\pgfqpoint{1.364939in}{1.077255in}}%
\pgfpathlineto{\pgfqpoint{1.365322in}{1.077255in}}%
\pgfpathlineto{\pgfqpoint{1.365322in}{1.134746in}}%
\pgfpathlineto{\pgfqpoint{1.366087in}{1.030217in}}%
\pgfpathlineto{\pgfqpoint{1.366853in}{1.077255in}}%
\pgfpathlineto{\pgfqpoint{1.367236in}{1.077255in}}%
\pgfpathlineto{\pgfqpoint{1.367619in}{1.009311in}}%
\pgfpathlineto{\pgfqpoint{1.368767in}{1.139973in}}%
\pgfpathlineto{\pgfqpoint{1.369150in}{1.139973in}}%
\pgfpathlineto{\pgfqpoint{1.369916in}{1.024990in}}%
\pgfpathlineto{\pgfqpoint{1.369533in}{1.150426in}}%
\pgfpathlineto{\pgfqpoint{1.370681in}{1.087708in}}%
\pgfpathlineto{\pgfqpoint{1.371064in}{1.087708in}}%
\pgfpathlineto{\pgfqpoint{1.372596in}{1.134746in}}%
\pgfpathlineto{\pgfqpoint{1.372978in}{1.134746in}}%
\pgfpathlineto{\pgfqpoint{1.373744in}{1.192238in}}%
\pgfpathlineto{\pgfqpoint{1.374510in}{1.056349in}}%
\pgfpathlineto{\pgfqpoint{1.374893in}{1.056349in}}%
\pgfpathlineto{\pgfqpoint{1.376424in}{1.139973in}}%
\pgfpathlineto{\pgfqpoint{1.376807in}{1.139973in}}%
\pgfpathlineto{\pgfqpoint{1.376807in}{1.129520in}}%
\pgfpathlineto{\pgfqpoint{1.377955in}{1.239276in}}%
\pgfpathlineto{\pgfqpoint{1.378338in}{1.218370in}}%
\pgfpathlineto{\pgfqpoint{1.378721in}{1.218370in}}%
\pgfpathlineto{\pgfqpoint{1.378721in}{1.103388in}}%
\pgfpathlineto{\pgfqpoint{1.380252in}{1.176558in}}%
\pgfpathlineto{\pgfqpoint{1.380635in}{1.176558in}}%
\pgfpathlineto{\pgfqpoint{1.381018in}{1.056349in}}%
\pgfpathlineto{\pgfqpoint{1.381784in}{1.202691in}}%
\pgfpathlineto{\pgfqpoint{1.382166in}{1.098161in}}%
\pgfpathlineto{\pgfqpoint{1.382549in}{1.098161in}}%
\pgfpathlineto{\pgfqpoint{1.382549in}{1.051123in}}%
\pgfpathlineto{\pgfqpoint{1.382932in}{1.129520in}}%
\pgfpathlineto{\pgfqpoint{1.384080in}{1.129520in}}%
\pgfpathlineto{\pgfqpoint{1.384463in}{1.129520in}}%
\pgfpathlineto{\pgfqpoint{1.385229in}{1.030217in}}%
\pgfpathlineto{\pgfqpoint{1.385995in}{1.166105in}}%
\pgfpathlineto{\pgfqpoint{1.386377in}{1.166105in}}%
\pgfpathlineto{\pgfqpoint{1.386760in}{1.035443in}}%
\pgfpathlineto{\pgfqpoint{1.387909in}{1.213144in}}%
\pgfpathlineto{\pgfqpoint{1.388292in}{1.213144in}}%
\pgfpathlineto{\pgfqpoint{1.389440in}{1.103388in}}%
\pgfpathlineto{\pgfqpoint{1.389823in}{1.223597in}}%
\pgfpathlineto{\pgfqpoint{1.390206in}{1.223597in}}%
\pgfpathlineto{\pgfqpoint{1.391354in}{1.098161in}}%
\pgfpathlineto{\pgfqpoint{1.391737in}{1.150426in}}%
\pgfpathlineto{\pgfqpoint{1.392120in}{1.150426in}}%
\pgfpathlineto{\pgfqpoint{1.392503in}{1.213144in}}%
\pgfpathlineto{\pgfqpoint{1.393268in}{1.030217in}}%
\pgfpathlineto{\pgfqpoint{1.393651in}{1.202691in}}%
\pgfpathlineto{\pgfqpoint{1.394034in}{1.202691in}}%
\pgfpathlineto{\pgfqpoint{1.394034in}{1.260182in}}%
\pgfpathlineto{\pgfqpoint{1.394800in}{1.134746in}}%
\pgfpathlineto{\pgfqpoint{1.395565in}{1.228823in}}%
\pgfpathlineto{\pgfqpoint{1.395948in}{1.228823in}}%
\pgfpathlineto{\pgfqpoint{1.397480in}{1.082482in}}%
\pgfpathlineto{\pgfqpoint{1.397862in}{1.082482in}}%
\pgfpathlineto{\pgfqpoint{1.398245in}{1.270635in}}%
\pgfpathlineto{\pgfqpoint{1.399394in}{1.160879in}}%
\pgfpathlineto{\pgfqpoint{1.399777in}{1.160879in}}%
\pgfpathlineto{\pgfqpoint{1.400925in}{1.092935in}}%
\pgfpathlineto{\pgfqpoint{1.401308in}{1.270635in}}%
\pgfpathlineto{\pgfqpoint{1.401691in}{1.270635in}}%
\pgfpathlineto{\pgfqpoint{1.402073in}{1.098161in}}%
\pgfpathlineto{\pgfqpoint{1.403222in}{1.228823in}}%
\pgfpathlineto{\pgfqpoint{1.403605in}{1.228823in}}%
\pgfpathlineto{\pgfqpoint{1.403988in}{1.301994in}}%
\pgfpathlineto{\pgfqpoint{1.404753in}{1.124293in}}%
\pgfpathlineto{\pgfqpoint{1.405136in}{1.155652in}}%
\pgfpathlineto{\pgfqpoint{1.405519in}{1.155652in}}%
\pgfpathlineto{\pgfqpoint{1.405519in}{1.139973in}}%
\pgfpathlineto{\pgfqpoint{1.406285in}{1.192238in}}%
\pgfpathlineto{\pgfqpoint{1.407050in}{1.176558in}}%
\pgfpathlineto{\pgfqpoint{1.407433in}{1.176558in}}%
\pgfpathlineto{\pgfqpoint{1.408582in}{1.260182in}}%
\pgfpathlineto{\pgfqpoint{1.408199in}{1.155652in}}%
\pgfpathlineto{\pgfqpoint{1.408964in}{1.213144in}}%
\pgfpathlineto{\pgfqpoint{1.409347in}{1.213144in}}%
\pgfpathlineto{\pgfqpoint{1.410496in}{1.092935in}}%
\pgfpathlineto{\pgfqpoint{1.410879in}{1.254955in}}%
\pgfpathlineto{\pgfqpoint{1.411261in}{1.254955in}}%
\pgfpathlineto{\pgfqpoint{1.412410in}{1.082482in}}%
\pgfpathlineto{\pgfqpoint{1.412793in}{1.296767in}}%
\pgfpathlineto{\pgfqpoint{1.413176in}{1.296767in}}%
\pgfpathlineto{\pgfqpoint{1.413941in}{1.312447in}}%
\pgfpathlineto{\pgfqpoint{1.414707in}{1.098161in}}%
\pgfpathlineto{\pgfqpoint{1.415090in}{1.098161in}}%
\pgfpathlineto{\pgfqpoint{1.416621in}{1.249729in}}%
\pgfpathlineto{\pgfqpoint{1.417004in}{1.249729in}}%
\pgfpathlineto{\pgfqpoint{1.417004in}{1.343806in}}%
\pgfpathlineto{\pgfqpoint{1.418152in}{1.166105in}}%
\pgfpathlineto{\pgfqpoint{1.418535in}{1.181785in}}%
\pgfpathlineto{\pgfqpoint{1.418918in}{1.181785in}}%
\pgfpathlineto{\pgfqpoint{1.419301in}{1.270635in}}%
\pgfpathlineto{\pgfqpoint{1.419684in}{1.150426in}}%
\pgfpathlineto{\pgfqpoint{1.420449in}{1.176558in}}%
\pgfpathlineto{\pgfqpoint{1.420832in}{1.176558in}}%
\pgfpathlineto{\pgfqpoint{1.421981in}{1.291541in}}%
\pgfpathlineto{\pgfqpoint{1.422363in}{1.202691in}}%
\pgfpathlineto{\pgfqpoint{1.422746in}{1.202691in}}%
\pgfpathlineto{\pgfqpoint{1.422746in}{1.155652in}}%
\pgfpathlineto{\pgfqpoint{1.423895in}{1.281088in}}%
\pgfpathlineto{\pgfqpoint{1.424278in}{1.187011in}}%
\pgfpathlineto{\pgfqpoint{1.424660in}{1.187011in}}%
\pgfpathlineto{\pgfqpoint{1.425426in}{1.307220in}}%
\pgfpathlineto{\pgfqpoint{1.426192in}{1.223597in}}%
\pgfpathlineto{\pgfqpoint{1.426575in}{1.223597in}}%
\pgfpathlineto{\pgfqpoint{1.426957in}{1.312447in}}%
\pgfpathlineto{\pgfqpoint{1.427340in}{1.160879in}}%
\pgfpathlineto{\pgfqpoint{1.428106in}{1.307220in}}%
\pgfpathlineto{\pgfqpoint{1.428489in}{1.307220in}}%
\pgfpathlineto{\pgfqpoint{1.429254in}{1.380391in}}%
\pgfpathlineto{\pgfqpoint{1.430020in}{1.202691in}}%
\pgfpathlineto{\pgfqpoint{1.430403in}{1.202691in}}%
\pgfpathlineto{\pgfqpoint{1.430403in}{1.265408in}}%
\pgfpathlineto{\pgfqpoint{1.431934in}{1.239276in}}%
\pgfpathlineto{\pgfqpoint{1.432317in}{1.239276in}}%
\pgfpathlineto{\pgfqpoint{1.433083in}{1.176558in}}%
\pgfpathlineto{\pgfqpoint{1.433848in}{1.301994in}}%
\pgfpathlineto{\pgfqpoint{1.434231in}{1.301994in}}%
\pgfpathlineto{\pgfqpoint{1.434614in}{1.202691in}}%
\pgfpathlineto{\pgfqpoint{1.434997in}{1.343806in}}%
\pgfpathlineto{\pgfqpoint{1.435763in}{1.249729in}}%
\pgfpathlineto{\pgfqpoint{1.436145in}{1.249729in}}%
\pgfpathlineto{\pgfqpoint{1.437294in}{1.312447in}}%
\pgfpathlineto{\pgfqpoint{1.437677in}{1.160879in}}%
\pgfpathlineto{\pgfqpoint{1.438060in}{1.160879in}}%
\pgfpathlineto{\pgfqpoint{1.438825in}{1.312447in}}%
\pgfpathlineto{\pgfqpoint{1.439591in}{1.281088in}}%
\pgfpathlineto{\pgfqpoint{1.439974in}{1.281088in}}%
\pgfpathlineto{\pgfqpoint{1.439974in}{1.223597in}}%
\pgfpathlineto{\pgfqpoint{1.441122in}{1.416976in}}%
\pgfpathlineto{\pgfqpoint{1.441505in}{1.286314in}}%
\pgfpathlineto{\pgfqpoint{1.441888in}{1.286314in}}%
\pgfpathlineto{\pgfqpoint{1.441888in}{1.322900in}}%
\pgfpathlineto{\pgfqpoint{1.443419in}{1.265408in}}%
\pgfpathlineto{\pgfqpoint{1.443802in}{1.265408in}}%
\pgfpathlineto{\pgfqpoint{1.443802in}{1.166105in}}%
\pgfpathlineto{\pgfqpoint{1.444185in}{1.380391in}}%
\pgfpathlineto{\pgfqpoint{1.445333in}{1.213144in}}%
\pgfpathlineto{\pgfqpoint{1.445716in}{1.213144in}}%
\pgfpathlineto{\pgfqpoint{1.446482in}{1.359485in}}%
\pgfpathlineto{\pgfqpoint{1.446865in}{1.139973in}}%
\pgfpathlineto{\pgfqpoint{1.447247in}{1.333353in}}%
\pgfpathlineto{\pgfqpoint{1.447630in}{1.333353in}}%
\pgfpathlineto{\pgfqpoint{1.448013in}{1.192238in}}%
\pgfpathlineto{\pgfqpoint{1.448013in}{1.385617in}}%
\pgfpathlineto{\pgfqpoint{1.449162in}{1.281088in}}%
\pgfpathlineto{\pgfqpoint{1.449544in}{1.281088in}}%
\pgfpathlineto{\pgfqpoint{1.449927in}{1.333353in}}%
\pgfpathlineto{\pgfqpoint{1.450310in}{1.254955in}}%
\pgfpathlineto{\pgfqpoint{1.451076in}{1.301994in}}%
\pgfpathlineto{\pgfqpoint{1.451459in}{1.301994in}}%
\pgfpathlineto{\pgfqpoint{1.452224in}{1.249729in}}%
\pgfpathlineto{\pgfqpoint{1.452990in}{1.443109in}}%
\pgfpathlineto{\pgfqpoint{1.453373in}{1.443109in}}%
\pgfpathlineto{\pgfqpoint{1.454904in}{1.171332in}}%
\pgfpathlineto{\pgfqpoint{1.455287in}{1.171332in}}%
\pgfpathlineto{\pgfqpoint{1.455287in}{1.333353in}}%
\pgfpathlineto{\pgfqpoint{1.456818in}{1.286314in}}%
\pgfpathlineto{\pgfqpoint{1.457201in}{1.286314in}}%
\pgfpathlineto{\pgfqpoint{1.457584in}{1.411750in}}%
\pgfpathlineto{\pgfqpoint{1.458350in}{1.213144in}}%
\pgfpathlineto{\pgfqpoint{1.458732in}{1.322900in}}%
\pgfpathlineto{\pgfqpoint{1.459498in}{1.322900in}}%
\pgfpathlineto{\pgfqpoint{1.459881in}{1.249729in}}%
\pgfpathlineto{\pgfqpoint{1.460646in}{1.369938in}}%
\pgfpathlineto{\pgfqpoint{1.461029in}{1.260182in}}%
\pgfpathlineto{\pgfqpoint{1.461412in}{1.260182in}}%
\pgfpathlineto{\pgfqpoint{1.462561in}{1.448335in}}%
\pgfpathlineto{\pgfqpoint{1.462178in}{1.254955in}}%
\pgfpathlineto{\pgfqpoint{1.462943in}{1.270635in}}%
\pgfpathlineto{\pgfqpoint{1.463326in}{1.270635in}}%
\pgfpathlineto{\pgfqpoint{1.463709in}{1.448335in}}%
\pgfpathlineto{\pgfqpoint{1.464858in}{1.437882in}}%
\pgfpathlineto{\pgfqpoint{1.465240in}{1.437882in}}%
\pgfpathlineto{\pgfqpoint{1.466006in}{1.202691in}}%
\pgfpathlineto{\pgfqpoint{1.466772in}{1.385617in}}%
\pgfpathlineto{\pgfqpoint{1.467155in}{1.385617in}}%
\pgfpathlineto{\pgfqpoint{1.467537in}{1.416976in}}%
\pgfpathlineto{\pgfqpoint{1.468686in}{1.239276in}}%
\pgfpathlineto{\pgfqpoint{1.469069in}{1.239276in}}%
\pgfpathlineto{\pgfqpoint{1.469834in}{1.197464in}}%
\pgfpathlineto{\pgfqpoint{1.470600in}{1.390844in}}%
\pgfpathlineto{\pgfqpoint{1.470983in}{1.390844in}}%
\pgfpathlineto{\pgfqpoint{1.471749in}{1.281088in}}%
\pgfpathlineto{\pgfqpoint{1.471366in}{1.437882in}}%
\pgfpathlineto{\pgfqpoint{1.472514in}{1.437882in}}%
\pgfpathlineto{\pgfqpoint{1.473280in}{1.437882in}}%
\pgfpathlineto{\pgfqpoint{1.474428in}{1.260182in}}%
\pgfpathlineto{\pgfqpoint{1.474811in}{1.343806in}}%
\pgfpathlineto{\pgfqpoint{1.475194in}{1.343806in}}%
\pgfpathlineto{\pgfqpoint{1.476725in}{1.270635in}}%
\pgfpathlineto{\pgfqpoint{1.477108in}{1.270635in}}%
\pgfpathlineto{\pgfqpoint{1.477874in}{1.385617in}}%
\pgfpathlineto{\pgfqpoint{1.478640in}{1.307220in}}%
\pgfpathlineto{\pgfqpoint{1.479022in}{1.307220in}}%
\pgfpathlineto{\pgfqpoint{1.480171in}{1.411750in}}%
\pgfpathlineto{\pgfqpoint{1.480554in}{1.270635in}}%
\pgfpathlineto{\pgfqpoint{1.480936in}{1.270635in}}%
\pgfpathlineto{\pgfqpoint{1.482085in}{1.453562in}}%
\pgfpathlineto{\pgfqpoint{1.482468in}{1.234050in}}%
\pgfpathlineto{\pgfqpoint{1.482851in}{1.234050in}}%
\pgfpathlineto{\pgfqpoint{1.483999in}{1.401297in}}%
\pgfpathlineto{\pgfqpoint{1.484382in}{1.234050in}}%
\pgfpathlineto{\pgfqpoint{1.484765in}{1.234050in}}%
\pgfpathlineto{\pgfqpoint{1.485913in}{1.464015in}}%
\pgfpathlineto{\pgfqpoint{1.486296in}{1.416976in}}%
\pgfpathlineto{\pgfqpoint{1.486679in}{1.416976in}}%
\pgfpathlineto{\pgfqpoint{1.486679in}{1.448335in}}%
\pgfpathlineto{\pgfqpoint{1.488210in}{1.265408in}}%
\pgfpathlineto{\pgfqpoint{1.488593in}{1.265408in}}%
\pgfpathlineto{\pgfqpoint{1.489742in}{1.380391in}}%
\pgfpathlineto{\pgfqpoint{1.490124in}{1.354259in}}%
\pgfpathlineto{\pgfqpoint{1.490507in}{1.354259in}}%
\pgfpathlineto{\pgfqpoint{1.490890in}{1.286314in}}%
\pgfpathlineto{\pgfqpoint{1.491273in}{1.432656in}}%
\pgfpathlineto{\pgfqpoint{1.492039in}{1.380391in}}%
\pgfpathlineto{\pgfqpoint{1.492421in}{1.380391in}}%
\pgfpathlineto{\pgfqpoint{1.492421in}{1.401297in}}%
\pgfpathlineto{\pgfqpoint{1.492804in}{1.275861in}}%
\pgfpathlineto{\pgfqpoint{1.493953in}{1.354259in}}%
\pgfpathlineto{\pgfqpoint{1.494336in}{1.354259in}}%
\pgfpathlineto{\pgfqpoint{1.495101in}{1.469241in}}%
\pgfpathlineto{\pgfqpoint{1.494718in}{1.254955in}}%
\pgfpathlineto{\pgfqpoint{1.495867in}{1.270635in}}%
\pgfpathlineto{\pgfqpoint{1.496250in}{1.270635in}}%
\pgfpathlineto{\pgfqpoint{1.497398in}{1.448335in}}%
\pgfpathlineto{\pgfqpoint{1.497781in}{1.432656in}}%
\pgfpathlineto{\pgfqpoint{1.498164in}{1.432656in}}%
\pgfpathlineto{\pgfqpoint{1.498547in}{1.260182in}}%
\pgfpathlineto{\pgfqpoint{1.499312in}{1.464015in}}%
\pgfpathlineto{\pgfqpoint{1.499695in}{1.317673in}}%
\pgfpathlineto{\pgfqpoint{1.500078in}{1.317673in}}%
\pgfpathlineto{\pgfqpoint{1.500461in}{1.427429in}}%
\pgfpathlineto{\pgfqpoint{1.501609in}{1.328126in}}%
\pgfpathlineto{\pgfqpoint{1.501992in}{1.328126in}}%
\pgfpathlineto{\pgfqpoint{1.501992in}{1.422203in}}%
\pgfpathlineto{\pgfqpoint{1.503523in}{1.369938in}}%
\pgfpathlineto{\pgfqpoint{1.503906in}{1.369938in}}%
\pgfpathlineto{\pgfqpoint{1.504672in}{1.401297in}}%
\pgfpathlineto{\pgfqpoint{1.505438in}{1.312447in}}%
\pgfpathlineto{\pgfqpoint{1.505820in}{1.312447in}}%
\pgfpathlineto{\pgfqpoint{1.506586in}{1.448335in}}%
\pgfpathlineto{\pgfqpoint{1.507352in}{1.375165in}}%
\pgfpathlineto{\pgfqpoint{1.507735in}{1.375165in}}%
\pgfpathlineto{\pgfqpoint{1.507735in}{1.484921in}}%
\pgfpathlineto{\pgfqpoint{1.508500in}{1.317673in}}%
\pgfpathlineto{\pgfqpoint{1.509266in}{1.401297in}}%
\pgfpathlineto{\pgfqpoint{1.509649in}{1.401297in}}%
\pgfpathlineto{\pgfqpoint{1.509649in}{1.333353in}}%
\pgfpathlineto{\pgfqpoint{1.510414in}{1.448335in}}%
\pgfpathlineto{\pgfqpoint{1.511180in}{1.338579in}}%
\pgfpathlineto{\pgfqpoint{1.511563in}{1.338579in}}%
\pgfpathlineto{\pgfqpoint{1.512711in}{1.422203in}}%
\pgfpathlineto{\pgfqpoint{1.512329in}{1.275861in}}%
\pgfpathlineto{\pgfqpoint{1.513094in}{1.406523in}}%
\pgfpathlineto{\pgfqpoint{1.513477in}{1.406523in}}%
\pgfpathlineto{\pgfqpoint{1.514243in}{1.213144in}}%
\pgfpathlineto{\pgfqpoint{1.513860in}{1.458788in}}%
\pgfpathlineto{\pgfqpoint{1.515008in}{1.422203in}}%
\pgfpathlineto{\pgfqpoint{1.515391in}{1.422203in}}%
\pgfpathlineto{\pgfqpoint{1.515391in}{1.338579in}}%
\pgfpathlineto{\pgfqpoint{1.516157in}{1.511053in}}%
\pgfpathlineto{\pgfqpoint{1.516923in}{1.437882in}}%
\pgfpathlineto{\pgfqpoint{1.517305in}{1.437882in}}%
\pgfpathlineto{\pgfqpoint{1.518837in}{1.317673in}}%
\pgfpathlineto{\pgfqpoint{1.519219in}{1.317673in}}%
\pgfpathlineto{\pgfqpoint{1.519985in}{1.396070in}}%
\pgfpathlineto{\pgfqpoint{1.520751in}{1.349032in}}%
\pgfpathlineto{\pgfqpoint{1.521134in}{1.349032in}}%
\pgfpathlineto{\pgfqpoint{1.521516in}{1.458788in}}%
\pgfpathlineto{\pgfqpoint{1.522665in}{1.338579in}}%
\pgfpathlineto{\pgfqpoint{1.523048in}{1.338579in}}%
\pgfpathlineto{\pgfqpoint{1.523048in}{1.286314in}}%
\pgfpathlineto{\pgfqpoint{1.523813in}{1.416976in}}%
\pgfpathlineto{\pgfqpoint{1.524579in}{1.349032in}}%
\pgfpathlineto{\pgfqpoint{1.524962in}{1.349032in}}%
\pgfpathlineto{\pgfqpoint{1.526110in}{1.401297in}}%
\pgfpathlineto{\pgfqpoint{1.526493in}{1.275861in}}%
\pgfpathlineto{\pgfqpoint{1.526876in}{1.275861in}}%
\pgfpathlineto{\pgfqpoint{1.528407in}{1.432656in}}%
\pgfpathlineto{\pgfqpoint{1.528790in}{1.432656in}}%
\pgfpathlineto{\pgfqpoint{1.529173in}{1.322900in}}%
\pgfpathlineto{\pgfqpoint{1.530322in}{1.328126in}}%
\pgfpathlineto{\pgfqpoint{1.530704in}{1.328126in}}%
\pgfpathlineto{\pgfqpoint{1.530704in}{1.322900in}}%
\pgfpathlineto{\pgfqpoint{1.531470in}{1.448335in}}%
\pgfpathlineto{\pgfqpoint{1.532236in}{1.354259in}}%
\pgfpathlineto{\pgfqpoint{1.532619in}{1.354259in}}%
\pgfpathlineto{\pgfqpoint{1.533384in}{1.260182in}}%
\pgfpathlineto{\pgfqpoint{1.533001in}{1.416976in}}%
\pgfpathlineto{\pgfqpoint{1.534150in}{1.401297in}}%
\pgfpathlineto{\pgfqpoint{1.534533in}{1.401297in}}%
\pgfpathlineto{\pgfqpoint{1.535681in}{1.234050in}}%
\pgfpathlineto{\pgfqpoint{1.536064in}{1.317673in}}%
\pgfpathlineto{\pgfqpoint{1.536447in}{1.317673in}}%
\pgfpathlineto{\pgfqpoint{1.537595in}{1.511053in}}%
\pgfpathlineto{\pgfqpoint{1.536830in}{1.270635in}}%
\pgfpathlineto{\pgfqpoint{1.537978in}{1.396070in}}%
\pgfpathlineto{\pgfqpoint{1.538361in}{1.396070in}}%
\pgfpathlineto{\pgfqpoint{1.538744in}{1.422203in}}%
\pgfpathlineto{\pgfqpoint{1.539892in}{1.270635in}}%
\pgfpathlineto{\pgfqpoint{1.540275in}{1.270635in}}%
\pgfpathlineto{\pgfqpoint{1.541424in}{1.437882in}}%
\pgfpathlineto{\pgfqpoint{1.541806in}{1.390844in}}%
\pgfpathlineto{\pgfqpoint{1.542572in}{1.390844in}}%
\pgfpathlineto{\pgfqpoint{1.542955in}{1.427429in}}%
\pgfpathlineto{\pgfqpoint{1.544103in}{1.228823in}}%
\pgfpathlineto{\pgfqpoint{1.544486in}{1.228823in}}%
\pgfpathlineto{\pgfqpoint{1.544486in}{1.427429in}}%
\pgfpathlineto{\pgfqpoint{1.546018in}{1.301994in}}%
\pgfpathlineto{\pgfqpoint{1.546400in}{1.301994in}}%
\pgfpathlineto{\pgfqpoint{1.547166in}{1.432656in}}%
\pgfpathlineto{\pgfqpoint{1.547932in}{1.390844in}}%
\pgfpathlineto{\pgfqpoint{1.548315in}{1.390844in}}%
\pgfpathlineto{\pgfqpoint{1.549080in}{1.448335in}}%
\pgfpathlineto{\pgfqpoint{1.549846in}{1.291541in}}%
\pgfpathlineto{\pgfqpoint{1.550229in}{1.291541in}}%
\pgfpathlineto{\pgfqpoint{1.551760in}{1.369938in}}%
\pgfpathlineto{\pgfqpoint{1.552143in}{1.369938in}}%
\pgfpathlineto{\pgfqpoint{1.553291in}{1.396070in}}%
\pgfpathlineto{\pgfqpoint{1.553674in}{1.254955in}}%
\pgfpathlineto{\pgfqpoint{1.554057in}{1.254955in}}%
\pgfpathlineto{\pgfqpoint{1.555588in}{1.422203in}}%
\pgfpathlineto{\pgfqpoint{1.555971in}{1.422203in}}%
\pgfpathlineto{\pgfqpoint{1.556354in}{1.296767in}}%
\pgfpathlineto{\pgfqpoint{1.557502in}{1.322900in}}%
\pgfpathlineto{\pgfqpoint{1.557885in}{1.322900in}}%
\pgfpathlineto{\pgfqpoint{1.558268in}{1.469241in}}%
\pgfpathlineto{\pgfqpoint{1.558268in}{1.296767in}}%
\pgfpathlineto{\pgfqpoint{1.559417in}{1.390844in}}%
\pgfpathlineto{\pgfqpoint{1.559799in}{1.390844in}}%
\pgfpathlineto{\pgfqpoint{1.559799in}{1.416976in}}%
\pgfpathlineto{\pgfqpoint{1.560565in}{1.301994in}}%
\pgfpathlineto{\pgfqpoint{1.561331in}{1.312447in}}%
\pgfpathlineto{\pgfqpoint{1.561714in}{1.312447in}}%
\pgfpathlineto{\pgfqpoint{1.562479in}{1.364712in}}%
\pgfpathlineto{\pgfqpoint{1.563245in}{1.307220in}}%
\pgfpathlineto{\pgfqpoint{1.563628in}{1.307220in}}%
\pgfpathlineto{\pgfqpoint{1.563628in}{1.249729in}}%
\pgfpathlineto{\pgfqpoint{1.564393in}{1.416976in}}%
\pgfpathlineto{\pgfqpoint{1.565159in}{1.401297in}}%
\pgfpathlineto{\pgfqpoint{1.565542in}{1.401297in}}%
\pgfpathlineto{\pgfqpoint{1.565542in}{1.260182in}}%
\pgfpathlineto{\pgfqpoint{1.566690in}{1.484921in}}%
\pgfpathlineto{\pgfqpoint{1.567073in}{1.307220in}}%
\pgfpathlineto{\pgfqpoint{1.567456in}{1.307220in}}%
\pgfpathlineto{\pgfqpoint{1.567456in}{1.427429in}}%
\pgfpathlineto{\pgfqpoint{1.568987in}{1.401297in}}%
\pgfpathlineto{\pgfqpoint{1.569370in}{1.401297in}}%
\pgfpathlineto{\pgfqpoint{1.569753in}{1.448335in}}%
\pgfpathlineto{\pgfqpoint{1.570902in}{1.406523in}}%
\pgfpathlineto{\pgfqpoint{1.571284in}{1.406523in}}%
\pgfpathlineto{\pgfqpoint{1.571667in}{1.275861in}}%
\pgfpathlineto{\pgfqpoint{1.572816in}{1.354259in}}%
\pgfpathlineto{\pgfqpoint{1.573199in}{1.354259in}}%
\pgfpathlineto{\pgfqpoint{1.573964in}{1.490147in}}%
\pgfpathlineto{\pgfqpoint{1.574347in}{1.322900in}}%
\pgfpathlineto{\pgfqpoint{1.574730in}{1.369938in}}%
\pgfpathlineto{\pgfqpoint{1.575113in}{1.369938in}}%
\pgfpathlineto{\pgfqpoint{1.575495in}{1.218370in}}%
\pgfpathlineto{\pgfqpoint{1.576261in}{1.406523in}}%
\pgfpathlineto{\pgfqpoint{1.576644in}{1.281088in}}%
\pgfpathlineto{\pgfqpoint{1.577027in}{1.281088in}}%
\pgfpathlineto{\pgfqpoint{1.577027in}{1.254955in}}%
\pgfpathlineto{\pgfqpoint{1.577410in}{1.401297in}}%
\pgfpathlineto{\pgfqpoint{1.578558in}{1.281088in}}%
\pgfpathlineto{\pgfqpoint{1.578941in}{1.281088in}}%
\pgfpathlineto{\pgfqpoint{1.578941in}{1.422203in}}%
\pgfpathlineto{\pgfqpoint{1.580472in}{1.296767in}}%
\pgfpathlineto{\pgfqpoint{1.580855in}{1.296767in}}%
\pgfpathlineto{\pgfqpoint{1.582386in}{1.416976in}}%
\pgfpathlineto{\pgfqpoint{1.582769in}{1.416976in}}%
\pgfpathlineto{\pgfqpoint{1.583152in}{1.286314in}}%
\pgfpathlineto{\pgfqpoint{1.584301in}{1.422203in}}%
\pgfpathlineto{\pgfqpoint{1.584683in}{1.422203in}}%
\pgfpathlineto{\pgfqpoint{1.586215in}{1.254955in}}%
\pgfpathlineto{\pgfqpoint{1.586598in}{1.254955in}}%
\pgfpathlineto{\pgfqpoint{1.586980in}{1.432656in}}%
\pgfpathlineto{\pgfqpoint{1.588129in}{1.375165in}}%
\pgfpathlineto{\pgfqpoint{1.588512in}{1.375165in}}%
\pgfpathlineto{\pgfqpoint{1.588512in}{1.390844in}}%
\pgfpathlineto{\pgfqpoint{1.589277in}{1.281088in}}%
\pgfpathlineto{\pgfqpoint{1.590043in}{1.286314in}}%
\pgfpathlineto{\pgfqpoint{1.590426in}{1.286314in}}%
\pgfpathlineto{\pgfqpoint{1.590809in}{1.401297in}}%
\pgfpathlineto{\pgfqpoint{1.591957in}{1.312447in}}%
\pgfpathlineto{\pgfqpoint{1.592340in}{1.312447in}}%
\pgfpathlineto{\pgfqpoint{1.593871in}{1.474468in}}%
\pgfpathlineto{\pgfqpoint{1.594254in}{1.474468in}}%
\pgfpathlineto{\pgfqpoint{1.595403in}{1.281088in}}%
\pgfpathlineto{\pgfqpoint{1.595785in}{1.406523in}}%
\pgfpathlineto{\pgfqpoint{1.596168in}{1.406523in}}%
\pgfpathlineto{\pgfqpoint{1.597700in}{1.291541in}}%
\pgfpathlineto{\pgfqpoint{1.598082in}{1.291541in}}%
\pgfpathlineto{\pgfqpoint{1.599231in}{1.422203in}}%
\pgfpathlineto{\pgfqpoint{1.598848in}{1.275861in}}%
\pgfpathlineto{\pgfqpoint{1.599614in}{1.369938in}}%
\pgfpathlineto{\pgfqpoint{1.599997in}{1.369938in}}%
\pgfpathlineto{\pgfqpoint{1.601145in}{1.286314in}}%
\pgfpathlineto{\pgfqpoint{1.601528in}{1.448335in}}%
\pgfpathlineto{\pgfqpoint{1.601911in}{1.448335in}}%
\pgfpathlineto{\pgfqpoint{1.601911in}{1.254955in}}%
\pgfpathlineto{\pgfqpoint{1.603442in}{1.333353in}}%
\pgfpathlineto{\pgfqpoint{1.603825in}{1.333353in}}%
\pgfpathlineto{\pgfqpoint{1.604208in}{1.448335in}}%
\pgfpathlineto{\pgfqpoint{1.605356in}{1.396070in}}%
\pgfpathlineto{\pgfqpoint{1.605739in}{1.396070in}}%
\pgfpathlineto{\pgfqpoint{1.605739in}{1.228823in}}%
\pgfpathlineto{\pgfqpoint{1.606888in}{1.416976in}}%
\pgfpathlineto{\pgfqpoint{1.607270in}{1.359485in}}%
\pgfpathlineto{\pgfqpoint{1.607653in}{1.359485in}}%
\pgfpathlineto{\pgfqpoint{1.608036in}{1.270635in}}%
\pgfpathlineto{\pgfqpoint{1.608802in}{1.385617in}}%
\pgfpathlineto{\pgfqpoint{1.609185in}{1.385617in}}%
\pgfpathlineto{\pgfqpoint{1.609567in}{1.385617in}}%
\pgfpathlineto{\pgfqpoint{1.609950in}{1.254955in}}%
\pgfpathlineto{\pgfqpoint{1.611099in}{1.317673in}}%
\pgfpathlineto{\pgfqpoint{1.611482in}{1.317673in}}%
\pgfpathlineto{\pgfqpoint{1.612630in}{1.422203in}}%
\pgfpathlineto{\pgfqpoint{1.611864in}{1.254955in}}%
\pgfpathlineto{\pgfqpoint{1.613013in}{1.364712in}}%
\pgfpathlineto{\pgfqpoint{1.613396in}{1.364712in}}%
\pgfpathlineto{\pgfqpoint{1.613779in}{1.375165in}}%
\pgfpathlineto{\pgfqpoint{1.614927in}{1.234050in}}%
\pgfpathlineto{\pgfqpoint{1.615310in}{1.234050in}}%
\pgfpathlineto{\pgfqpoint{1.616841in}{1.385617in}}%
\pgfpathlineto{\pgfqpoint{1.617224in}{1.385617in}}%
\pgfpathlineto{\pgfqpoint{1.618372in}{1.333353in}}%
\pgfpathlineto{\pgfqpoint{1.618755in}{1.469241in}}%
\pgfpathlineto{\pgfqpoint{1.619138in}{1.469241in}}%
\pgfpathlineto{\pgfqpoint{1.620669in}{1.307220in}}%
\pgfpathlineto{\pgfqpoint{1.621052in}{1.307220in}}%
\pgfpathlineto{\pgfqpoint{1.621818in}{1.385617in}}%
\pgfpathlineto{\pgfqpoint{1.622201in}{1.254955in}}%
\pgfpathlineto{\pgfqpoint{1.622584in}{1.286314in}}%
\pgfpathlineto{\pgfqpoint{1.622966in}{1.286314in}}%
\pgfpathlineto{\pgfqpoint{1.622966in}{1.301994in}}%
\pgfpathlineto{\pgfqpoint{1.624115in}{1.239276in}}%
\pgfpathlineto{\pgfqpoint{1.624498in}{1.291541in}}%
\pgfpathlineto{\pgfqpoint{1.624881in}{1.291541in}}%
\pgfpathlineto{\pgfqpoint{1.625646in}{1.390844in}}%
\pgfpathlineto{\pgfqpoint{1.625263in}{1.286314in}}%
\pgfpathlineto{\pgfqpoint{1.626412in}{1.301994in}}%
\pgfpathlineto{\pgfqpoint{1.626795in}{1.301994in}}%
\pgfpathlineto{\pgfqpoint{1.627178in}{1.228823in}}%
\pgfpathlineto{\pgfqpoint{1.627178in}{1.375165in}}%
\pgfpathlineto{\pgfqpoint{1.628326in}{1.239276in}}%
\pgfpathlineto{\pgfqpoint{1.628709in}{1.239276in}}%
\pgfpathlineto{\pgfqpoint{1.628709in}{1.411750in}}%
\pgfpathlineto{\pgfqpoint{1.630240in}{1.244503in}}%
\pgfpathlineto{\pgfqpoint{1.630623in}{1.244503in}}%
\pgfpathlineto{\pgfqpoint{1.631772in}{1.390844in}}%
\pgfpathlineto{\pgfqpoint{1.632154in}{1.359485in}}%
\pgfpathlineto{\pgfqpoint{1.632537in}{1.359485in}}%
\pgfpathlineto{\pgfqpoint{1.632537in}{1.296767in}}%
\pgfpathlineto{\pgfqpoint{1.632920in}{1.396070in}}%
\pgfpathlineto{\pgfqpoint{1.634068in}{1.385617in}}%
\pgfpathlineto{\pgfqpoint{1.634451in}{1.385617in}}%
\pgfpathlineto{\pgfqpoint{1.634834in}{1.291541in}}%
\pgfpathlineto{\pgfqpoint{1.635983in}{1.380391in}}%
\pgfpathlineto{\pgfqpoint{1.636365in}{1.380391in}}%
\pgfpathlineto{\pgfqpoint{1.637897in}{1.275861in}}%
\pgfpathlineto{\pgfqpoint{1.638280in}{1.275861in}}%
\pgfpathlineto{\pgfqpoint{1.638280in}{1.458788in}}%
\pgfpathlineto{\pgfqpoint{1.639811in}{1.234050in}}%
\pgfpathlineto{\pgfqpoint{1.640194in}{1.234050in}}%
\pgfpathlineto{\pgfqpoint{1.640959in}{1.338579in}}%
\pgfpathlineto{\pgfqpoint{1.641725in}{1.301994in}}%
\pgfpathlineto{\pgfqpoint{1.642108in}{1.301994in}}%
\pgfpathlineto{\pgfqpoint{1.642491in}{1.202691in}}%
\pgfpathlineto{\pgfqpoint{1.643639in}{1.469241in}}%
\pgfpathlineto{\pgfqpoint{1.644022in}{1.469241in}}%
\pgfpathlineto{\pgfqpoint{1.644022in}{1.223597in}}%
\pgfpathlineto{\pgfqpoint{1.645553in}{1.375165in}}%
\pgfpathlineto{\pgfqpoint{1.645936in}{1.375165in}}%
\pgfpathlineto{\pgfqpoint{1.646319in}{1.192238in}}%
\pgfpathlineto{\pgfqpoint{1.646702in}{1.411750in}}%
\pgfpathlineto{\pgfqpoint{1.647468in}{1.281088in}}%
\pgfpathlineto{\pgfqpoint{1.647850in}{1.281088in}}%
\pgfpathlineto{\pgfqpoint{1.648999in}{1.160879in}}%
\pgfpathlineto{\pgfqpoint{1.648233in}{1.291541in}}%
\pgfpathlineto{\pgfqpoint{1.649382in}{1.254955in}}%
\pgfpathlineto{\pgfqpoint{1.649765in}{1.254955in}}%
\pgfpathlineto{\pgfqpoint{1.651296in}{1.406523in}}%
\pgfpathlineto{\pgfqpoint{1.651679in}{1.406523in}}%
\pgfpathlineto{\pgfqpoint{1.652827in}{1.239276in}}%
\pgfpathlineto{\pgfqpoint{1.653210in}{1.281088in}}%
\pgfpathlineto{\pgfqpoint{1.653593in}{1.281088in}}%
\pgfpathlineto{\pgfqpoint{1.653593in}{1.333353in}}%
\pgfpathlineto{\pgfqpoint{1.655124in}{1.234050in}}%
\pgfpathlineto{\pgfqpoint{1.655507in}{1.234050in}}%
\pgfpathlineto{\pgfqpoint{1.656655in}{1.296767in}}%
\pgfpathlineto{\pgfqpoint{1.655890in}{1.202691in}}%
\pgfpathlineto{\pgfqpoint{1.657038in}{1.296767in}}%
\pgfpathlineto{\pgfqpoint{1.657421in}{1.296767in}}%
\pgfpathlineto{\pgfqpoint{1.657804in}{1.385617in}}%
\pgfpathlineto{\pgfqpoint{1.658187in}{1.228823in}}%
\pgfpathlineto{\pgfqpoint{1.658952in}{1.307220in}}%
\pgfpathlineto{\pgfqpoint{1.659335in}{1.307220in}}%
\pgfpathlineto{\pgfqpoint{1.659335in}{1.155652in}}%
\pgfpathlineto{\pgfqpoint{1.660867in}{1.218370in}}%
\pgfpathlineto{\pgfqpoint{1.661249in}{1.218370in}}%
\pgfpathlineto{\pgfqpoint{1.661249in}{1.349032in}}%
\pgfpathlineto{\pgfqpoint{1.662781in}{1.260182in}}%
\pgfpathlineto{\pgfqpoint{1.663164in}{1.260182in}}%
\pgfpathlineto{\pgfqpoint{1.664312in}{1.129520in}}%
\pgfpathlineto{\pgfqpoint{1.664695in}{1.375165in}}%
\pgfpathlineto{\pgfqpoint{1.665078in}{1.375165in}}%
\pgfpathlineto{\pgfqpoint{1.665461in}{1.145199in}}%
\pgfpathlineto{\pgfqpoint{1.666609in}{1.270635in}}%
\pgfpathlineto{\pgfqpoint{1.666992in}{1.270635in}}%
\pgfpathlineto{\pgfqpoint{1.668140in}{1.218370in}}%
\pgfpathlineto{\pgfqpoint{1.667758in}{1.275861in}}%
\pgfpathlineto{\pgfqpoint{1.668523in}{1.265408in}}%
\pgfpathlineto{\pgfqpoint{1.668906in}{1.265408in}}%
\pgfpathlineto{\pgfqpoint{1.669289in}{1.234050in}}%
\pgfpathlineto{\pgfqpoint{1.670055in}{1.275861in}}%
\pgfpathlineto{\pgfqpoint{1.670437in}{1.260182in}}%
\pgfpathlineto{\pgfqpoint{1.670820in}{1.260182in}}%
\pgfpathlineto{\pgfqpoint{1.670820in}{1.213144in}}%
\pgfpathlineto{\pgfqpoint{1.671203in}{1.281088in}}%
\pgfpathlineto{\pgfqpoint{1.672351in}{1.270635in}}%
\pgfpathlineto{\pgfqpoint{1.672734in}{1.270635in}}%
\pgfpathlineto{\pgfqpoint{1.674266in}{1.103388in}}%
\pgfpathlineto{\pgfqpoint{1.674648in}{1.103388in}}%
\pgfpathlineto{\pgfqpoint{1.675414in}{1.265408in}}%
\pgfpathlineto{\pgfqpoint{1.676180in}{1.197464in}}%
\pgfpathlineto{\pgfqpoint{1.676563in}{1.197464in}}%
\pgfpathlineto{\pgfqpoint{1.676945in}{1.354259in}}%
\pgfpathlineto{\pgfqpoint{1.677328in}{1.187011in}}%
\pgfpathlineto{\pgfqpoint{1.678094in}{1.265408in}}%
\pgfpathlineto{\pgfqpoint{1.678477in}{1.265408in}}%
\pgfpathlineto{\pgfqpoint{1.678477in}{1.291541in}}%
\pgfpathlineto{\pgfqpoint{1.679242in}{1.150426in}}%
\pgfpathlineto{\pgfqpoint{1.680008in}{1.265408in}}%
\pgfpathlineto{\pgfqpoint{1.680391in}{1.265408in}}%
\pgfpathlineto{\pgfqpoint{1.680391in}{1.343806in}}%
\pgfpathlineto{\pgfqpoint{1.681922in}{1.244503in}}%
\pgfpathlineto{\pgfqpoint{1.682305in}{1.244503in}}%
\pgfpathlineto{\pgfqpoint{1.682305in}{1.187011in}}%
\pgfpathlineto{\pgfqpoint{1.683836in}{1.234050in}}%
\pgfpathlineto{\pgfqpoint{1.684219in}{1.234050in}}%
\pgfpathlineto{\pgfqpoint{1.684219in}{1.260182in}}%
\pgfpathlineto{\pgfqpoint{1.685368in}{1.155652in}}%
\pgfpathlineto{\pgfqpoint{1.685751in}{1.207917in}}%
\pgfpathlineto{\pgfqpoint{1.686133in}{1.207917in}}%
\pgfpathlineto{\pgfqpoint{1.687282in}{1.139973in}}%
\pgfpathlineto{\pgfqpoint{1.686516in}{1.354259in}}%
\pgfpathlineto{\pgfqpoint{1.687665in}{1.207917in}}%
\pgfpathlineto{\pgfqpoint{1.688048in}{1.207917in}}%
\pgfpathlineto{\pgfqpoint{1.688430in}{1.187011in}}%
\pgfpathlineto{\pgfqpoint{1.688813in}{1.281088in}}%
\pgfpathlineto{\pgfqpoint{1.689579in}{1.207917in}}%
\pgfpathlineto{\pgfqpoint{1.689962in}{1.207917in}}%
\pgfpathlineto{\pgfqpoint{1.689962in}{1.307220in}}%
\pgfpathlineto{\pgfqpoint{1.691110in}{1.124293in}}%
\pgfpathlineto{\pgfqpoint{1.691493in}{1.187011in}}%
\pgfpathlineto{\pgfqpoint{1.691876in}{1.187011in}}%
\pgfpathlineto{\pgfqpoint{1.691876in}{1.155652in}}%
\pgfpathlineto{\pgfqpoint{1.693024in}{1.301994in}}%
\pgfpathlineto{\pgfqpoint{1.693407in}{1.265408in}}%
\pgfpathlineto{\pgfqpoint{1.693790in}{1.265408in}}%
\pgfpathlineto{\pgfqpoint{1.693790in}{1.286314in}}%
\pgfpathlineto{\pgfqpoint{1.694556in}{1.129520in}}%
\pgfpathlineto{\pgfqpoint{1.695321in}{1.129520in}}%
\pgfpathlineto{\pgfqpoint{1.695704in}{1.129520in}}%
\pgfpathlineto{\pgfqpoint{1.695704in}{1.254955in}}%
\pgfpathlineto{\pgfqpoint{1.696853in}{1.072029in}}%
\pgfpathlineto{\pgfqpoint{1.697235in}{1.160879in}}%
\pgfpathlineto{\pgfqpoint{1.697618in}{1.160879in}}%
\pgfpathlineto{\pgfqpoint{1.697618in}{1.260182in}}%
\pgfpathlineto{\pgfqpoint{1.699150in}{1.223597in}}%
\pgfpathlineto{\pgfqpoint{1.699532in}{1.223597in}}%
\pgfpathlineto{\pgfqpoint{1.699532in}{1.072029in}}%
\pgfpathlineto{\pgfqpoint{1.701064in}{1.103388in}}%
\pgfpathlineto{\pgfqpoint{1.701447in}{1.103388in}}%
\pgfpathlineto{\pgfqpoint{1.702595in}{1.092935in}}%
\pgfpathlineto{\pgfqpoint{1.702978in}{1.187011in}}%
\pgfpathlineto{\pgfqpoint{1.703361in}{1.187011in}}%
\pgfpathlineto{\pgfqpoint{1.703744in}{1.139973in}}%
\pgfpathlineto{\pgfqpoint{1.704892in}{1.265408in}}%
\pgfpathlineto{\pgfqpoint{1.705275in}{1.265408in}}%
\pgfpathlineto{\pgfqpoint{1.705275in}{1.270635in}}%
\pgfpathlineto{\pgfqpoint{1.706806in}{1.155652in}}%
\pgfpathlineto{\pgfqpoint{1.707189in}{1.155652in}}%
\pgfpathlineto{\pgfqpoint{1.707955in}{1.265408in}}%
\pgfpathlineto{\pgfqpoint{1.708720in}{1.207917in}}%
\pgfpathlineto{\pgfqpoint{1.709103in}{1.207917in}}%
\pgfpathlineto{\pgfqpoint{1.709103in}{1.113840in}}%
\pgfpathlineto{\pgfqpoint{1.709486in}{1.322900in}}%
\pgfpathlineto{\pgfqpoint{1.710635in}{1.181785in}}%
\pgfpathlineto{\pgfqpoint{1.711017in}{1.181785in}}%
\pgfpathlineto{\pgfqpoint{1.711400in}{1.113840in}}%
\pgfpathlineto{\pgfqpoint{1.711783in}{1.213144in}}%
\pgfpathlineto{\pgfqpoint{1.712549in}{1.145199in}}%
\pgfpathlineto{\pgfqpoint{1.712931in}{1.145199in}}%
\pgfpathlineto{\pgfqpoint{1.712931in}{1.119067in}}%
\pgfpathlineto{\pgfqpoint{1.713314in}{1.213144in}}%
\pgfpathlineto{\pgfqpoint{1.714463in}{1.202691in}}%
\pgfpathlineto{\pgfqpoint{1.714846in}{1.202691in}}%
\pgfpathlineto{\pgfqpoint{1.714846in}{1.040670in}}%
\pgfpathlineto{\pgfqpoint{1.715611in}{1.260182in}}%
\pgfpathlineto{\pgfqpoint{1.716377in}{1.239276in}}%
\pgfpathlineto{\pgfqpoint{1.716760in}{1.239276in}}%
\pgfpathlineto{\pgfqpoint{1.716760in}{1.098161in}}%
\pgfpathlineto{\pgfqpoint{1.718291in}{1.202691in}}%
\pgfpathlineto{\pgfqpoint{1.718674in}{1.202691in}}%
\pgfpathlineto{\pgfqpoint{1.719822in}{1.108614in}}%
\pgfpathlineto{\pgfqpoint{1.720205in}{1.207917in}}%
\pgfpathlineto{\pgfqpoint{1.720588in}{1.207917in}}%
\pgfpathlineto{\pgfqpoint{1.720971in}{1.113840in}}%
\pgfpathlineto{\pgfqpoint{1.722119in}{1.181785in}}%
\pgfpathlineto{\pgfqpoint{1.722502in}{1.181785in}}%
\pgfpathlineto{\pgfqpoint{1.722885in}{1.139973in}}%
\pgfpathlineto{\pgfqpoint{1.723268in}{1.228823in}}%
\pgfpathlineto{\pgfqpoint{1.724034in}{1.155652in}}%
\pgfpathlineto{\pgfqpoint{1.724416in}{1.155652in}}%
\pgfpathlineto{\pgfqpoint{1.724416in}{1.197464in}}%
\pgfpathlineto{\pgfqpoint{1.725565in}{1.139973in}}%
\pgfpathlineto{\pgfqpoint{1.725948in}{1.139973in}}%
\pgfpathlineto{\pgfqpoint{1.726331in}{1.139973in}}%
\pgfpathlineto{\pgfqpoint{1.727096in}{1.202691in}}%
\pgfpathlineto{\pgfqpoint{1.727862in}{1.113840in}}%
\pgfpathlineto{\pgfqpoint{1.728245in}{1.113840in}}%
\pgfpathlineto{\pgfqpoint{1.729010in}{1.176558in}}%
\pgfpathlineto{\pgfqpoint{1.729776in}{1.082482in}}%
\pgfpathlineto{\pgfqpoint{1.730159in}{1.082482in}}%
\pgfpathlineto{\pgfqpoint{1.730542in}{1.187011in}}%
\pgfpathlineto{\pgfqpoint{1.731690in}{1.150426in}}%
\pgfpathlineto{\pgfqpoint{1.732073in}{1.150426in}}%
\pgfpathlineto{\pgfqpoint{1.733221in}{1.098161in}}%
\pgfpathlineto{\pgfqpoint{1.733604in}{1.270635in}}%
\pgfpathlineto{\pgfqpoint{1.733987in}{1.270635in}}%
\pgfpathlineto{\pgfqpoint{1.735136in}{1.108614in}}%
\pgfpathlineto{\pgfqpoint{1.735518in}{1.181785in}}%
\pgfpathlineto{\pgfqpoint{1.735901in}{1.181785in}}%
\pgfpathlineto{\pgfqpoint{1.736284in}{1.108614in}}%
\pgfpathlineto{\pgfqpoint{1.737433in}{1.301994in}}%
\pgfpathlineto{\pgfqpoint{1.737815in}{1.301994in}}%
\pgfpathlineto{\pgfqpoint{1.738581in}{1.051123in}}%
\pgfpathlineto{\pgfqpoint{1.739347in}{1.124293in}}%
\pgfpathlineto{\pgfqpoint{1.739730in}{1.124293in}}%
\pgfpathlineto{\pgfqpoint{1.739730in}{1.166105in}}%
\pgfpathlineto{\pgfqpoint{1.741261in}{1.056349in}}%
\pgfpathlineto{\pgfqpoint{1.741644in}{1.056349in}}%
\pgfpathlineto{\pgfqpoint{1.742409in}{1.176558in}}%
\pgfpathlineto{\pgfqpoint{1.743175in}{1.056349in}}%
\pgfpathlineto{\pgfqpoint{1.743558in}{1.056349in}}%
\pgfpathlineto{\pgfqpoint{1.743941in}{1.187011in}}%
\pgfpathlineto{\pgfqpoint{1.744324in}{1.051123in}}%
\pgfpathlineto{\pgfqpoint{1.745089in}{1.119067in}}%
\pgfpathlineto{\pgfqpoint{1.745472in}{1.119067in}}%
\pgfpathlineto{\pgfqpoint{1.745472in}{1.019764in}}%
\pgfpathlineto{\pgfqpoint{1.747003in}{1.166105in}}%
\pgfpathlineto{\pgfqpoint{1.747386in}{1.166105in}}%
\pgfpathlineto{\pgfqpoint{1.747386in}{1.202691in}}%
\pgfpathlineto{\pgfqpoint{1.748535in}{1.119067in}}%
\pgfpathlineto{\pgfqpoint{1.748918in}{1.124293in}}%
\pgfpathlineto{\pgfqpoint{1.749300in}{1.124293in}}%
\pgfpathlineto{\pgfqpoint{1.749300in}{1.066802in}}%
\pgfpathlineto{\pgfqpoint{1.750066in}{1.134746in}}%
\pgfpathlineto{\pgfqpoint{1.750832in}{1.134746in}}%
\pgfpathlineto{\pgfqpoint{1.751214in}{1.134746in}}%
\pgfpathlineto{\pgfqpoint{1.752746in}{1.040670in}}%
\pgfpathlineto{\pgfqpoint{1.753129in}{1.040670in}}%
\pgfpathlineto{\pgfqpoint{1.754660in}{1.160879in}}%
\pgfpathlineto{\pgfqpoint{1.755043in}{1.160879in}}%
\pgfpathlineto{\pgfqpoint{1.755808in}{1.024990in}}%
\pgfpathlineto{\pgfqpoint{1.756574in}{1.072029in}}%
\pgfpathlineto{\pgfqpoint{1.756957in}{1.072029in}}%
\pgfpathlineto{\pgfqpoint{1.757723in}{1.213144in}}%
\pgfpathlineto{\pgfqpoint{1.758488in}{1.045896in}}%
\pgfpathlineto{\pgfqpoint{1.758871in}{1.045896in}}%
\pgfpathlineto{\pgfqpoint{1.759637in}{1.171332in}}%
\pgfpathlineto{\pgfqpoint{1.760402in}{1.103388in}}%
\pgfpathlineto{\pgfqpoint{1.760785in}{1.103388in}}%
\pgfpathlineto{\pgfqpoint{1.761934in}{1.040670in}}%
\pgfpathlineto{\pgfqpoint{1.762317in}{1.124293in}}%
\pgfpathlineto{\pgfqpoint{1.762699in}{1.124293in}}%
\pgfpathlineto{\pgfqpoint{1.762699in}{1.024990in}}%
\pgfpathlineto{\pgfqpoint{1.764231in}{1.051123in}}%
\pgfpathlineto{\pgfqpoint{1.764614in}{1.051123in}}%
\pgfpathlineto{\pgfqpoint{1.764614in}{1.213144in}}%
\pgfpathlineto{\pgfqpoint{1.766145in}{1.087708in}}%
\pgfpathlineto{\pgfqpoint{1.766528in}{1.087708in}}%
\pgfpathlineto{\pgfqpoint{1.767293in}{1.171332in}}%
\pgfpathlineto{\pgfqpoint{1.767676in}{1.077255in}}%
\pgfpathlineto{\pgfqpoint{1.768059in}{1.098161in}}%
\pgfpathlineto{\pgfqpoint{1.768442in}{1.098161in}}%
\pgfpathlineto{\pgfqpoint{1.768825in}{1.202691in}}%
\pgfpathlineto{\pgfqpoint{1.769973in}{1.082482in}}%
\pgfpathlineto{\pgfqpoint{1.770356in}{1.082482in}}%
\pgfpathlineto{\pgfqpoint{1.770356in}{1.040670in}}%
\pgfpathlineto{\pgfqpoint{1.771887in}{1.176558in}}%
\pgfpathlineto{\pgfqpoint{1.772270in}{1.176558in}}%
\pgfpathlineto{\pgfqpoint{1.772653in}{1.045896in}}%
\pgfpathlineto{\pgfqpoint{1.773801in}{1.056349in}}%
\pgfpathlineto{\pgfqpoint{1.774184in}{1.056349in}}%
\pgfpathlineto{\pgfqpoint{1.774184in}{1.103388in}}%
\pgfpathlineto{\pgfqpoint{1.775716in}{1.072029in}}%
\pgfpathlineto{\pgfqpoint{1.776098in}{1.072029in}}%
\pgfpathlineto{\pgfqpoint{1.776098in}{1.024990in}}%
\pgfpathlineto{\pgfqpoint{1.776481in}{1.129520in}}%
\pgfpathlineto{\pgfqpoint{1.777630in}{1.108614in}}%
\pgfpathlineto{\pgfqpoint{1.778013in}{1.108614in}}%
\pgfpathlineto{\pgfqpoint{1.778395in}{1.176558in}}%
\pgfpathlineto{\pgfqpoint{1.779161in}{1.014537in}}%
\pgfpathlineto{\pgfqpoint{1.779544in}{1.139973in}}%
\pgfpathlineto{\pgfqpoint{1.779927in}{1.139973in}}%
\pgfpathlineto{\pgfqpoint{1.781075in}{1.145199in}}%
\pgfpathlineto{\pgfqpoint{1.781458in}{1.082482in}}%
\pgfpathlineto{\pgfqpoint{1.781841in}{1.082482in}}%
\pgfpathlineto{\pgfqpoint{1.781841in}{1.234050in}}%
\pgfpathlineto{\pgfqpoint{1.782607in}{1.056349in}}%
\pgfpathlineto{\pgfqpoint{1.783372in}{1.171332in}}%
\pgfpathlineto{\pgfqpoint{1.783755in}{1.171332in}}%
\pgfpathlineto{\pgfqpoint{1.784138in}{1.030217in}}%
\pgfpathlineto{\pgfqpoint{1.785286in}{1.134746in}}%
\pgfpathlineto{\pgfqpoint{1.785669in}{1.134746in}}%
\pgfpathlineto{\pgfqpoint{1.787201in}{1.056349in}}%
\pgfpathlineto{\pgfqpoint{1.787583in}{1.056349in}}%
\pgfpathlineto{\pgfqpoint{1.788349in}{1.098161in}}%
\pgfpathlineto{\pgfqpoint{1.789115in}{1.024990in}}%
\pgfpathlineto{\pgfqpoint{1.789497in}{1.024990in}}%
\pgfpathlineto{\pgfqpoint{1.789497in}{1.129520in}}%
\pgfpathlineto{\pgfqpoint{1.791029in}{1.082482in}}%
\pgfpathlineto{\pgfqpoint{1.791412in}{1.082482in}}%
\pgfpathlineto{\pgfqpoint{1.792177in}{1.187011in}}%
\pgfpathlineto{\pgfqpoint{1.791794in}{1.072029in}}%
\pgfpathlineto{\pgfqpoint{1.792943in}{1.119067in}}%
\pgfpathlineto{\pgfqpoint{1.793326in}{1.119067in}}%
\pgfpathlineto{\pgfqpoint{1.794091in}{1.030217in}}%
\pgfpathlineto{\pgfqpoint{1.794857in}{1.207917in}}%
\pgfpathlineto{\pgfqpoint{1.795240in}{1.207917in}}%
\pgfpathlineto{\pgfqpoint{1.796388in}{0.993631in}}%
\pgfpathlineto{\pgfqpoint{1.796771in}{1.145199in}}%
\pgfpathlineto{\pgfqpoint{1.797154in}{1.145199in}}%
\pgfpathlineto{\pgfqpoint{1.797154in}{1.077255in}}%
\pgfpathlineto{\pgfqpoint{1.797920in}{1.218370in}}%
\pgfpathlineto{\pgfqpoint{1.798685in}{1.129520in}}%
\pgfpathlineto{\pgfqpoint{1.799068in}{1.129520in}}%
\pgfpathlineto{\pgfqpoint{1.799834in}{1.014537in}}%
\pgfpathlineto{\pgfqpoint{1.800217in}{1.244503in}}%
\pgfpathlineto{\pgfqpoint{1.800600in}{1.139973in}}%
\pgfpathlineto{\pgfqpoint{1.800982in}{1.139973in}}%
\pgfpathlineto{\pgfqpoint{1.801748in}{1.072029in}}%
\pgfpathlineto{\pgfqpoint{1.801365in}{1.223597in}}%
\pgfpathlineto{\pgfqpoint{1.802514in}{1.124293in}}%
\pgfpathlineto{\pgfqpoint{1.802897in}{1.124293in}}%
\pgfpathlineto{\pgfqpoint{1.802897in}{1.056349in}}%
\pgfpathlineto{\pgfqpoint{1.803662in}{1.181785in}}%
\pgfpathlineto{\pgfqpoint{1.804428in}{1.181785in}}%
\pgfpathlineto{\pgfqpoint{1.804811in}{1.181785in}}%
\pgfpathlineto{\pgfqpoint{1.805194in}{1.009311in}}%
\pgfpathlineto{\pgfqpoint{1.806342in}{1.218370in}}%
\pgfpathlineto{\pgfqpoint{1.806725in}{1.218370in}}%
\pgfpathlineto{\pgfqpoint{1.807873in}{1.045896in}}%
\pgfpathlineto{\pgfqpoint{1.808256in}{1.082482in}}%
\pgfpathlineto{\pgfqpoint{1.808639in}{1.082482in}}%
\pgfpathlineto{\pgfqpoint{1.809405in}{1.160879in}}%
\pgfpathlineto{\pgfqpoint{1.810170in}{1.056349in}}%
\pgfpathlineto{\pgfqpoint{1.810553in}{1.056349in}}%
\pgfpathlineto{\pgfqpoint{1.810936in}{1.176558in}}%
\pgfpathlineto{\pgfqpoint{1.812084in}{1.176558in}}%
\pgfpathlineto{\pgfqpoint{1.812467in}{1.176558in}}%
\pgfpathlineto{\pgfqpoint{1.812467in}{1.014537in}}%
\pgfpathlineto{\pgfqpoint{1.813999in}{1.082482in}}%
\pgfpathlineto{\pgfqpoint{1.814381in}{1.082482in}}%
\pgfpathlineto{\pgfqpoint{1.814381in}{1.176558in}}%
\pgfpathlineto{\pgfqpoint{1.815913in}{1.139973in}}%
\pgfpathlineto{\pgfqpoint{1.816296in}{1.139973in}}%
\pgfpathlineto{\pgfqpoint{1.816296in}{1.061576in}}%
\pgfpathlineto{\pgfqpoint{1.817444in}{1.197464in}}%
\pgfpathlineto{\pgfqpoint{1.817827in}{1.160879in}}%
\pgfpathlineto{\pgfqpoint{1.818210in}{1.160879in}}%
\pgfpathlineto{\pgfqpoint{1.818975in}{1.176558in}}%
\pgfpathlineto{\pgfqpoint{1.818593in}{1.124293in}}%
\pgfpathlineto{\pgfqpoint{1.819741in}{1.155652in}}%
\pgfpathlineto{\pgfqpoint{1.820124in}{1.155652in}}%
\pgfpathlineto{\pgfqpoint{1.821272in}{1.223597in}}%
\pgfpathlineto{\pgfqpoint{1.821655in}{1.082482in}}%
\pgfpathlineto{\pgfqpoint{1.822038in}{1.082482in}}%
\pgfpathlineto{\pgfqpoint{1.822804in}{1.207917in}}%
\pgfpathlineto{\pgfqpoint{1.823569in}{1.150426in}}%
\pgfpathlineto{\pgfqpoint{1.823952in}{1.150426in}}%
\pgfpathlineto{\pgfqpoint{1.823952in}{1.218370in}}%
\pgfpathlineto{\pgfqpoint{1.824335in}{1.119067in}}%
\pgfpathlineto{\pgfqpoint{1.825484in}{1.129520in}}%
\pgfpathlineto{\pgfqpoint{1.825866in}{1.129520in}}%
\pgfpathlineto{\pgfqpoint{1.826632in}{1.197464in}}%
\pgfpathlineto{\pgfqpoint{1.826249in}{1.077255in}}%
\pgfpathlineto{\pgfqpoint{1.827398in}{1.150426in}}%
\pgfpathlineto{\pgfqpoint{1.827780in}{1.150426in}}%
\pgfpathlineto{\pgfqpoint{1.827780in}{1.260182in}}%
\pgfpathlineto{\pgfqpoint{1.828546in}{1.134746in}}%
\pgfpathlineto{\pgfqpoint{1.829312in}{1.192238in}}%
\pgfpathlineto{\pgfqpoint{1.829695in}{1.192238in}}%
\pgfpathlineto{\pgfqpoint{1.830077in}{1.098161in}}%
\pgfpathlineto{\pgfqpoint{1.831226in}{1.134746in}}%
\pgfpathlineto{\pgfqpoint{1.831609in}{1.134746in}}%
\pgfpathlineto{\pgfqpoint{1.832374in}{1.082482in}}%
\pgfpathlineto{\pgfqpoint{1.832757in}{1.228823in}}%
\pgfpathlineto{\pgfqpoint{1.833140in}{1.155652in}}%
\pgfpathlineto{\pgfqpoint{1.833523in}{1.155652in}}%
\pgfpathlineto{\pgfqpoint{1.833523in}{1.239276in}}%
\pgfpathlineto{\pgfqpoint{1.833906in}{1.139973in}}%
\pgfpathlineto{\pgfqpoint{1.835054in}{1.192238in}}%
\pgfpathlineto{\pgfqpoint{1.835437in}{1.192238in}}%
\pgfpathlineto{\pgfqpoint{1.835437in}{1.275861in}}%
\pgfpathlineto{\pgfqpoint{1.836586in}{1.134746in}}%
\pgfpathlineto{\pgfqpoint{1.836968in}{1.249729in}}%
\pgfpathlineto{\pgfqpoint{1.837351in}{1.249729in}}%
\pgfpathlineto{\pgfqpoint{1.838500in}{1.124293in}}%
\pgfpathlineto{\pgfqpoint{1.838883in}{1.223597in}}%
\pgfpathlineto{\pgfqpoint{1.839265in}{1.223597in}}%
\pgfpathlineto{\pgfqpoint{1.840414in}{1.077255in}}%
\pgfpathlineto{\pgfqpoint{1.840031in}{1.286314in}}%
\pgfpathlineto{\pgfqpoint{1.840797in}{1.129520in}}%
\pgfpathlineto{\pgfqpoint{1.841180in}{1.129520in}}%
\pgfpathlineto{\pgfqpoint{1.841945in}{1.218370in}}%
\pgfpathlineto{\pgfqpoint{1.842711in}{1.139973in}}%
\pgfpathlineto{\pgfqpoint{1.843094in}{1.139973in}}%
\pgfpathlineto{\pgfqpoint{1.844625in}{1.260182in}}%
\pgfpathlineto{\pgfqpoint{1.845008in}{1.260182in}}%
\pgfpathlineto{\pgfqpoint{1.845391in}{1.181785in}}%
\pgfpathlineto{\pgfqpoint{1.845774in}{1.296767in}}%
\pgfpathlineto{\pgfqpoint{1.846539in}{1.234050in}}%
\pgfpathlineto{\pgfqpoint{1.846922in}{1.234050in}}%
\pgfpathlineto{\pgfqpoint{1.846922in}{1.254955in}}%
\pgfpathlineto{\pgfqpoint{1.847688in}{1.181785in}}%
\pgfpathlineto{\pgfqpoint{1.848453in}{1.202691in}}%
\pgfpathlineto{\pgfqpoint{1.848836in}{1.202691in}}%
\pgfpathlineto{\pgfqpoint{1.848836in}{1.150426in}}%
\pgfpathlineto{\pgfqpoint{1.850367in}{1.312447in}}%
\pgfpathlineto{\pgfqpoint{1.850750in}{1.312447in}}%
\pgfpathlineto{\pgfqpoint{1.851516in}{1.124293in}}%
\pgfpathlineto{\pgfqpoint{1.852282in}{1.249729in}}%
\pgfpathlineto{\pgfqpoint{1.852664in}{1.249729in}}%
\pgfpathlineto{\pgfqpoint{1.853047in}{1.286314in}}%
\pgfpathlineto{\pgfqpoint{1.854196in}{1.087708in}}%
\pgfpathlineto{\pgfqpoint{1.854579in}{1.087708in}}%
\pgfpathlineto{\pgfqpoint{1.854579in}{1.239276in}}%
\pgfpathlineto{\pgfqpoint{1.856110in}{1.218370in}}%
\pgfpathlineto{\pgfqpoint{1.856876in}{1.218370in}}%
\pgfpathlineto{\pgfqpoint{1.857258in}{1.207917in}}%
\pgfpathlineto{\pgfqpoint{1.858407in}{1.291541in}}%
\pgfpathlineto{\pgfqpoint{1.858790in}{1.291541in}}%
\pgfpathlineto{\pgfqpoint{1.859173in}{1.092935in}}%
\pgfpathlineto{\pgfqpoint{1.860321in}{1.260182in}}%
\pgfpathlineto{\pgfqpoint{1.860704in}{1.260182in}}%
\pgfpathlineto{\pgfqpoint{1.860704in}{1.265408in}}%
\pgfpathlineto{\pgfqpoint{1.861470in}{1.166105in}}%
\pgfpathlineto{\pgfqpoint{1.862235in}{1.176558in}}%
\pgfpathlineto{\pgfqpoint{1.862618in}{1.176558in}}%
\pgfpathlineto{\pgfqpoint{1.863001in}{1.312447in}}%
\pgfpathlineto{\pgfqpoint{1.864149in}{1.239276in}}%
\pgfpathlineto{\pgfqpoint{1.864532in}{1.239276in}}%
\pgfpathlineto{\pgfqpoint{1.865298in}{1.349032in}}%
\pgfpathlineto{\pgfqpoint{1.866063in}{1.234050in}}%
\pgfpathlineto{\pgfqpoint{1.866446in}{1.234050in}}%
\pgfpathlineto{\pgfqpoint{1.867595in}{1.270635in}}%
\pgfpathlineto{\pgfqpoint{1.866829in}{1.160879in}}%
\pgfpathlineto{\pgfqpoint{1.867978in}{1.197464in}}%
\pgfpathlineto{\pgfqpoint{1.868360in}{1.197464in}}%
\pgfpathlineto{\pgfqpoint{1.868360in}{1.291541in}}%
\pgfpathlineto{\pgfqpoint{1.869892in}{1.213144in}}%
\pgfpathlineto{\pgfqpoint{1.870275in}{1.213144in}}%
\pgfpathlineto{\pgfqpoint{1.870275in}{1.275861in}}%
\pgfpathlineto{\pgfqpoint{1.871806in}{1.249729in}}%
\pgfpathlineto{\pgfqpoint{1.872189in}{1.249729in}}%
\pgfpathlineto{\pgfqpoint{1.872189in}{1.139973in}}%
\pgfpathlineto{\pgfqpoint{1.873720in}{1.411750in}}%
\pgfpathlineto{\pgfqpoint{1.874103in}{1.411750in}}%
\pgfpathlineto{\pgfqpoint{1.875634in}{1.207917in}}%
\pgfpathlineto{\pgfqpoint{1.876017in}{1.207917in}}%
\pgfpathlineto{\pgfqpoint{1.876400in}{1.354259in}}%
\pgfpathlineto{\pgfqpoint{1.877548in}{1.307220in}}%
\pgfpathlineto{\pgfqpoint{1.878314in}{1.307220in}}%
\pgfpathlineto{\pgfqpoint{1.878697in}{1.197464in}}%
\pgfpathlineto{\pgfqpoint{1.879080in}{1.349032in}}%
\pgfpathlineto{\pgfqpoint{1.879845in}{1.223597in}}%
\pgfpathlineto{\pgfqpoint{1.880228in}{1.223597in}}%
\pgfpathlineto{\pgfqpoint{1.881377in}{1.359485in}}%
\pgfpathlineto{\pgfqpoint{1.880994in}{1.197464in}}%
\pgfpathlineto{\pgfqpoint{1.881760in}{1.307220in}}%
\pgfpathlineto{\pgfqpoint{1.882142in}{1.307220in}}%
\pgfpathlineto{\pgfqpoint{1.882525in}{1.197464in}}%
\pgfpathlineto{\pgfqpoint{1.883674in}{1.228823in}}%
\pgfpathlineto{\pgfqpoint{1.884057in}{1.228823in}}%
\pgfpathlineto{\pgfqpoint{1.885205in}{1.375165in}}%
\pgfpathlineto{\pgfqpoint{1.884439in}{1.218370in}}%
\pgfpathlineto{\pgfqpoint{1.885588in}{1.234050in}}%
\pgfpathlineto{\pgfqpoint{1.885971in}{1.234050in}}%
\pgfpathlineto{\pgfqpoint{1.887502in}{1.479694in}}%
\pgfpathlineto{\pgfqpoint{1.887885in}{1.479694in}}%
\pgfpathlineto{\pgfqpoint{1.889416in}{1.249729in}}%
\pgfpathlineto{\pgfqpoint{1.889799in}{1.249729in}}%
\pgfpathlineto{\pgfqpoint{1.889799in}{1.119067in}}%
\pgfpathlineto{\pgfqpoint{1.890947in}{1.359485in}}%
\pgfpathlineto{\pgfqpoint{1.891330in}{1.218370in}}%
\pgfpathlineto{\pgfqpoint{1.891713in}{1.218370in}}%
\pgfpathlineto{\pgfqpoint{1.892096in}{1.416976in}}%
\pgfpathlineto{\pgfqpoint{1.893244in}{1.333353in}}%
\pgfpathlineto{\pgfqpoint{1.893627in}{1.333353in}}%
\pgfpathlineto{\pgfqpoint{1.894010in}{1.223597in}}%
\pgfpathlineto{\pgfqpoint{1.895159in}{1.223597in}}%
\pgfpathlineto{\pgfqpoint{1.895541in}{1.223597in}}%
\pgfpathlineto{\pgfqpoint{1.896307in}{1.401297in}}%
\pgfpathlineto{\pgfqpoint{1.897073in}{1.369938in}}%
\pgfpathlineto{\pgfqpoint{1.897456in}{1.369938in}}%
\pgfpathlineto{\pgfqpoint{1.897456in}{1.286314in}}%
\pgfpathlineto{\pgfqpoint{1.897838in}{1.390844in}}%
\pgfpathlineto{\pgfqpoint{1.898987in}{1.375165in}}%
\pgfpathlineto{\pgfqpoint{1.899370in}{1.375165in}}%
\pgfpathlineto{\pgfqpoint{1.900901in}{1.223597in}}%
\pgfpathlineto{\pgfqpoint{1.901284in}{1.223597in}}%
\pgfpathlineto{\pgfqpoint{1.902050in}{1.291541in}}%
\pgfpathlineto{\pgfqpoint{1.901667in}{1.207917in}}%
\pgfpathlineto{\pgfqpoint{1.902815in}{1.228823in}}%
\pgfpathlineto{\pgfqpoint{1.903198in}{1.228823in}}%
\pgfpathlineto{\pgfqpoint{1.903198in}{1.380391in}}%
\pgfpathlineto{\pgfqpoint{1.904729in}{1.187011in}}%
\pgfpathlineto{\pgfqpoint{1.905112in}{1.187011in}}%
\pgfpathlineto{\pgfqpoint{1.906261in}{1.406523in}}%
\pgfpathlineto{\pgfqpoint{1.906643in}{1.296767in}}%
\pgfpathlineto{\pgfqpoint{1.907026in}{1.296767in}}%
\pgfpathlineto{\pgfqpoint{1.907409in}{1.448335in}}%
\pgfpathlineto{\pgfqpoint{1.908558in}{1.333353in}}%
\pgfpathlineto{\pgfqpoint{1.908940in}{1.333353in}}%
\pgfpathlineto{\pgfqpoint{1.910089in}{1.291541in}}%
\pgfpathlineto{\pgfqpoint{1.910472in}{1.307220in}}%
\pgfpathlineto{\pgfqpoint{1.910855in}{1.307220in}}%
\pgfpathlineto{\pgfqpoint{1.910855in}{1.244503in}}%
\pgfpathlineto{\pgfqpoint{1.912386in}{1.411750in}}%
\pgfpathlineto{\pgfqpoint{1.912769in}{1.411750in}}%
\pgfpathlineto{\pgfqpoint{1.913534in}{1.234050in}}%
\pgfpathlineto{\pgfqpoint{1.914300in}{1.333353in}}%
\pgfpathlineto{\pgfqpoint{1.914683in}{1.333353in}}%
\pgfpathlineto{\pgfqpoint{1.915831in}{1.275861in}}%
\pgfpathlineto{\pgfqpoint{1.915066in}{1.338579in}}%
\pgfpathlineto{\pgfqpoint{1.916214in}{1.281088in}}%
\pgfpathlineto{\pgfqpoint{1.916597in}{1.281088in}}%
\pgfpathlineto{\pgfqpoint{1.918128in}{1.448335in}}%
\pgfpathlineto{\pgfqpoint{1.918511in}{1.448335in}}%
\pgfpathlineto{\pgfqpoint{1.919277in}{1.526732in}}%
\pgfpathlineto{\pgfqpoint{1.920043in}{1.307220in}}%
\pgfpathlineto{\pgfqpoint{1.920425in}{1.307220in}}%
\pgfpathlineto{\pgfqpoint{1.920808in}{1.234050in}}%
\pgfpathlineto{\pgfqpoint{1.921957in}{1.380391in}}%
\pgfpathlineto{\pgfqpoint{1.922340in}{1.380391in}}%
\pgfpathlineto{\pgfqpoint{1.922722in}{1.249729in}}%
\pgfpathlineto{\pgfqpoint{1.923871in}{1.406523in}}%
\pgfpathlineto{\pgfqpoint{1.924254in}{1.406523in}}%
\pgfpathlineto{\pgfqpoint{1.925019in}{1.307220in}}%
\pgfpathlineto{\pgfqpoint{1.925785in}{1.359485in}}%
\pgfpathlineto{\pgfqpoint{1.926168in}{1.359485in}}%
\pgfpathlineto{\pgfqpoint{1.927316in}{1.254955in}}%
\pgfpathlineto{\pgfqpoint{1.926551in}{1.380391in}}%
\pgfpathlineto{\pgfqpoint{1.927699in}{1.343806in}}%
\pgfpathlineto{\pgfqpoint{1.928082in}{1.343806in}}%
\pgfpathlineto{\pgfqpoint{1.928082in}{1.406523in}}%
\pgfpathlineto{\pgfqpoint{1.929230in}{1.124293in}}%
\pgfpathlineto{\pgfqpoint{1.929613in}{1.291541in}}%
\pgfpathlineto{\pgfqpoint{1.929996in}{1.291541in}}%
\pgfpathlineto{\pgfqpoint{1.931145in}{1.213144in}}%
\pgfpathlineto{\pgfqpoint{1.931527in}{1.427429in}}%
\pgfpathlineto{\pgfqpoint{1.931910in}{1.427429in}}%
\pgfpathlineto{\pgfqpoint{1.933059in}{1.474468in}}%
\pgfpathlineto{\pgfqpoint{1.933442in}{1.286314in}}%
\pgfpathlineto{\pgfqpoint{1.933824in}{1.286314in}}%
\pgfpathlineto{\pgfqpoint{1.934207in}{1.411750in}}%
\pgfpathlineto{\pgfqpoint{1.935356in}{1.244503in}}%
\pgfpathlineto{\pgfqpoint{1.935739in}{1.244503in}}%
\pgfpathlineto{\pgfqpoint{1.936121in}{1.516280in}}%
\pgfpathlineto{\pgfqpoint{1.937270in}{1.338579in}}%
\pgfpathlineto{\pgfqpoint{1.937653in}{1.338579in}}%
\pgfpathlineto{\pgfqpoint{1.938036in}{1.286314in}}%
\pgfpathlineto{\pgfqpoint{1.939184in}{1.427429in}}%
\pgfpathlineto{\pgfqpoint{1.939567in}{1.427429in}}%
\pgfpathlineto{\pgfqpoint{1.940333in}{1.275861in}}%
\pgfpathlineto{\pgfqpoint{1.941098in}{1.427429in}}%
\pgfpathlineto{\pgfqpoint{1.941481in}{1.427429in}}%
\pgfpathlineto{\pgfqpoint{1.943012in}{1.249729in}}%
\pgfpathlineto{\pgfqpoint{1.943395in}{1.249729in}}%
\pgfpathlineto{\pgfqpoint{1.944161in}{1.432656in}}%
\pgfpathlineto{\pgfqpoint{1.944926in}{1.328126in}}%
\pgfpathlineto{\pgfqpoint{1.945309in}{1.328126in}}%
\pgfpathlineto{\pgfqpoint{1.945309in}{1.427429in}}%
\pgfpathlineto{\pgfqpoint{1.946075in}{1.322900in}}%
\pgfpathlineto{\pgfqpoint{1.946841in}{1.401297in}}%
\pgfpathlineto{\pgfqpoint{1.947223in}{1.401297in}}%
\pgfpathlineto{\pgfqpoint{1.947989in}{1.265408in}}%
\pgfpathlineto{\pgfqpoint{1.948372in}{1.416976in}}%
\pgfpathlineto{\pgfqpoint{1.948755in}{1.369938in}}%
\pgfpathlineto{\pgfqpoint{1.949138in}{1.369938in}}%
\pgfpathlineto{\pgfqpoint{1.949520in}{1.443109in}}%
\pgfpathlineto{\pgfqpoint{1.949903in}{1.275861in}}%
\pgfpathlineto{\pgfqpoint{1.950669in}{1.396070in}}%
\pgfpathlineto{\pgfqpoint{1.951052in}{1.396070in}}%
\pgfpathlineto{\pgfqpoint{1.952200in}{1.213144in}}%
\pgfpathlineto{\pgfqpoint{1.951435in}{1.443109in}}%
\pgfpathlineto{\pgfqpoint{1.952583in}{1.364712in}}%
\pgfpathlineto{\pgfqpoint{1.952966in}{1.364712in}}%
\pgfpathlineto{\pgfqpoint{1.952966in}{1.411750in}}%
\pgfpathlineto{\pgfqpoint{1.954114in}{1.249729in}}%
\pgfpathlineto{\pgfqpoint{1.954497in}{1.390844in}}%
\pgfpathlineto{\pgfqpoint{1.954880in}{1.390844in}}%
\pgfpathlineto{\pgfqpoint{1.954880in}{1.296767in}}%
\pgfpathlineto{\pgfqpoint{1.956411in}{1.333353in}}%
\pgfpathlineto{\pgfqpoint{1.956794in}{1.333353in}}%
\pgfpathlineto{\pgfqpoint{1.956794in}{1.291541in}}%
\pgfpathlineto{\pgfqpoint{1.957177in}{1.396070in}}%
\pgfpathlineto{\pgfqpoint{1.958326in}{1.354259in}}%
\pgfpathlineto{\pgfqpoint{1.958708in}{1.354259in}}%
\pgfpathlineto{\pgfqpoint{1.959474in}{1.385617in}}%
\pgfpathlineto{\pgfqpoint{1.959857in}{1.244503in}}%
\pgfpathlineto{\pgfqpoint{1.960240in}{1.322900in}}%
\pgfpathlineto{\pgfqpoint{1.960623in}{1.322900in}}%
\pgfpathlineto{\pgfqpoint{1.961771in}{1.260182in}}%
\pgfpathlineto{\pgfqpoint{1.961388in}{1.385617in}}%
\pgfpathlineto{\pgfqpoint{1.962154in}{1.317673in}}%
\pgfpathlineto{\pgfqpoint{1.962537in}{1.317673in}}%
\pgfpathlineto{\pgfqpoint{1.963685in}{1.239276in}}%
\pgfpathlineto{\pgfqpoint{1.964068in}{1.338579in}}%
\pgfpathlineto{\pgfqpoint{1.964451in}{1.338579in}}%
\pgfpathlineto{\pgfqpoint{1.965599in}{1.254955in}}%
\pgfpathlineto{\pgfqpoint{1.965982in}{1.416976in}}%
\pgfpathlineto{\pgfqpoint{1.966365in}{1.416976in}}%
\pgfpathlineto{\pgfqpoint{1.967896in}{1.197464in}}%
\pgfpathlineto{\pgfqpoint{1.968279in}{1.197464in}}%
\pgfpathlineto{\pgfqpoint{1.969428in}{1.390844in}}%
\pgfpathlineto{\pgfqpoint{1.968662in}{1.181785in}}%
\pgfpathlineto{\pgfqpoint{1.969810in}{1.354259in}}%
\pgfpathlineto{\pgfqpoint{1.970193in}{1.354259in}}%
\pgfpathlineto{\pgfqpoint{1.970576in}{1.495374in}}%
\pgfpathlineto{\pgfqpoint{1.970576in}{1.228823in}}%
\pgfpathlineto{\pgfqpoint{1.971725in}{1.307220in}}%
\pgfpathlineto{\pgfqpoint{1.972107in}{1.307220in}}%
\pgfpathlineto{\pgfqpoint{1.972490in}{1.333353in}}%
\pgfpathlineto{\pgfqpoint{1.973639in}{1.244503in}}%
\pgfpathlineto{\pgfqpoint{1.974022in}{1.244503in}}%
\pgfpathlineto{\pgfqpoint{1.974022in}{1.406523in}}%
\pgfpathlineto{\pgfqpoint{1.974787in}{1.202691in}}%
\pgfpathlineto{\pgfqpoint{1.975553in}{1.234050in}}%
\pgfpathlineto{\pgfqpoint{1.975936in}{1.234050in}}%
\pgfpathlineto{\pgfqpoint{1.977084in}{1.213144in}}%
\pgfpathlineto{\pgfqpoint{1.977467in}{1.396070in}}%
\pgfpathlineto{\pgfqpoint{1.977850in}{1.396070in}}%
\pgfpathlineto{\pgfqpoint{1.979381in}{1.218370in}}%
\pgfpathlineto{\pgfqpoint{1.979764in}{1.218370in}}%
\pgfpathlineto{\pgfqpoint{1.981295in}{1.380391in}}%
\pgfpathlineto{\pgfqpoint{1.981678in}{1.380391in}}%
\pgfpathlineto{\pgfqpoint{1.981678in}{1.390844in}}%
\pgfpathlineto{\pgfqpoint{1.983209in}{1.213144in}}%
\pgfpathlineto{\pgfqpoint{1.983592in}{1.213144in}}%
\pgfpathlineto{\pgfqpoint{1.983592in}{1.354259in}}%
\pgfpathlineto{\pgfqpoint{1.985124in}{1.312447in}}%
\pgfpathlineto{\pgfqpoint{1.985506in}{1.312447in}}%
\pgfpathlineto{\pgfqpoint{1.985506in}{1.354259in}}%
\pgfpathlineto{\pgfqpoint{1.985889in}{1.187011in}}%
\pgfpathlineto{\pgfqpoint{1.987038in}{1.281088in}}%
\pgfpathlineto{\pgfqpoint{1.987421in}{1.281088in}}%
\pgfpathlineto{\pgfqpoint{1.987421in}{1.213144in}}%
\pgfpathlineto{\pgfqpoint{1.987803in}{1.380391in}}%
\pgfpathlineto{\pgfqpoint{1.988952in}{1.223597in}}%
\pgfpathlineto{\pgfqpoint{1.989335in}{1.223597in}}%
\pgfpathlineto{\pgfqpoint{1.990866in}{1.328126in}}%
\pgfpathlineto{\pgfqpoint{1.991249in}{1.328126in}}%
\pgfpathlineto{\pgfqpoint{1.991249in}{1.364712in}}%
\pgfpathlineto{\pgfqpoint{1.992397in}{1.187011in}}%
\pgfpathlineto{\pgfqpoint{1.992780in}{1.281088in}}%
\pgfpathlineto{\pgfqpoint{1.993163in}{1.281088in}}%
\pgfpathlineto{\pgfqpoint{1.993929in}{1.192238in}}%
\pgfpathlineto{\pgfqpoint{1.994312in}{1.401297in}}%
\pgfpathlineto{\pgfqpoint{1.994694in}{1.249729in}}%
\pgfpathlineto{\pgfqpoint{1.995077in}{1.249729in}}%
\pgfpathlineto{\pgfqpoint{1.995460in}{1.228823in}}%
\pgfpathlineto{\pgfqpoint{1.996609in}{1.369938in}}%
\pgfpathlineto{\pgfqpoint{1.996991in}{1.369938in}}%
\pgfpathlineto{\pgfqpoint{1.997374in}{1.228823in}}%
\pgfpathlineto{\pgfqpoint{1.998523in}{1.354259in}}%
\pgfpathlineto{\pgfqpoint{1.998906in}{1.354259in}}%
\pgfpathlineto{\pgfqpoint{1.999288in}{1.234050in}}%
\pgfpathlineto{\pgfqpoint{2.000437in}{1.281088in}}%
\pgfpathlineto{\pgfqpoint{2.000820in}{1.281088in}}%
\pgfpathlineto{\pgfqpoint{2.000820in}{1.218370in}}%
\pgfpathlineto{\pgfqpoint{2.002351in}{1.312447in}}%
\pgfpathlineto{\pgfqpoint{2.002734in}{1.312447in}}%
\pgfpathlineto{\pgfqpoint{2.002734in}{1.234050in}}%
\pgfpathlineto{\pgfqpoint{2.004265in}{1.296767in}}%
\pgfpathlineto{\pgfqpoint{2.004648in}{1.296767in}}%
\pgfpathlineto{\pgfqpoint{2.005414in}{1.139973in}}%
\pgfpathlineto{\pgfqpoint{2.006179in}{1.270635in}}%
\pgfpathlineto{\pgfqpoint{2.006562in}{1.270635in}}%
\pgfpathlineto{\pgfqpoint{2.006562in}{1.359485in}}%
\pgfpathlineto{\pgfqpoint{2.006945in}{1.207917in}}%
\pgfpathlineto{\pgfqpoint{2.008093in}{1.275861in}}%
\pgfpathlineto{\pgfqpoint{2.008476in}{1.275861in}}%
\pgfpathlineto{\pgfqpoint{2.009625in}{1.171332in}}%
\pgfpathlineto{\pgfqpoint{2.010008in}{1.249729in}}%
\pgfpathlineto{\pgfqpoint{2.010390in}{1.249729in}}%
\pgfpathlineto{\pgfqpoint{2.010390in}{1.197464in}}%
\pgfpathlineto{\pgfqpoint{2.010773in}{1.265408in}}%
\pgfpathlineto{\pgfqpoint{2.011922in}{1.218370in}}%
\pgfpathlineto{\pgfqpoint{2.012305in}{1.218370in}}%
\pgfpathlineto{\pgfqpoint{2.013453in}{1.150426in}}%
\pgfpathlineto{\pgfqpoint{2.013836in}{1.364712in}}%
\pgfpathlineto{\pgfqpoint{2.014219in}{1.364712in}}%
\pgfpathlineto{\pgfqpoint{2.015367in}{1.145199in}}%
\pgfpathlineto{\pgfqpoint{2.015750in}{1.171332in}}%
\pgfpathlineto{\pgfqpoint{2.016133in}{1.171332in}}%
\pgfpathlineto{\pgfqpoint{2.016133in}{1.228823in}}%
\pgfpathlineto{\pgfqpoint{2.017664in}{1.150426in}}%
\pgfpathlineto{\pgfqpoint{2.018047in}{1.150426in}}%
\pgfpathlineto{\pgfqpoint{2.018430in}{1.265408in}}%
\pgfpathlineto{\pgfqpoint{2.018813in}{1.098161in}}%
\pgfpathlineto{\pgfqpoint{2.019578in}{1.218370in}}%
\pgfpathlineto{\pgfqpoint{2.019961in}{1.218370in}}%
\pgfpathlineto{\pgfqpoint{2.019961in}{1.312447in}}%
\pgfpathlineto{\pgfqpoint{2.021492in}{1.092935in}}%
\pgfpathlineto{\pgfqpoint{2.021875in}{1.092935in}}%
\pgfpathlineto{\pgfqpoint{2.023024in}{1.317673in}}%
\pgfpathlineto{\pgfqpoint{2.023407in}{1.249729in}}%
\pgfpathlineto{\pgfqpoint{2.023789in}{1.249729in}}%
\pgfpathlineto{\pgfqpoint{2.023789in}{1.197464in}}%
\pgfpathlineto{\pgfqpoint{2.025321in}{1.343806in}}%
\pgfpathlineto{\pgfqpoint{2.025704in}{1.343806in}}%
\pgfpathlineto{\pgfqpoint{2.026469in}{1.160879in}}%
\pgfpathlineto{\pgfqpoint{2.027235in}{1.234050in}}%
\pgfpathlineto{\pgfqpoint{2.027618in}{1.234050in}}%
\pgfpathlineto{\pgfqpoint{2.028766in}{1.239276in}}%
\pgfpathlineto{\pgfqpoint{2.029149in}{1.124293in}}%
\pgfpathlineto{\pgfqpoint{2.029532in}{1.124293in}}%
\pgfpathlineto{\pgfqpoint{2.030298in}{1.270635in}}%
\pgfpathlineto{\pgfqpoint{2.031063in}{1.239276in}}%
\pgfpathlineto{\pgfqpoint{2.031446in}{1.239276in}}%
\pgfpathlineto{\pgfqpoint{2.031829in}{1.187011in}}%
\pgfpathlineto{\pgfqpoint{2.032595in}{1.291541in}}%
\pgfpathlineto{\pgfqpoint{2.032977in}{1.187011in}}%
\pgfpathlineto{\pgfqpoint{2.033360in}{1.187011in}}%
\pgfpathlineto{\pgfqpoint{2.033743in}{1.312447in}}%
\pgfpathlineto{\pgfqpoint{2.034892in}{1.197464in}}%
\pgfpathlineto{\pgfqpoint{2.035274in}{1.197464in}}%
\pgfpathlineto{\pgfqpoint{2.036423in}{1.045896in}}%
\pgfpathlineto{\pgfqpoint{2.036806in}{1.301994in}}%
\pgfpathlineto{\pgfqpoint{2.037189in}{1.301994in}}%
\pgfpathlineto{\pgfqpoint{2.038720in}{1.108614in}}%
\pgfpathlineto{\pgfqpoint{2.039103in}{1.108614in}}%
\pgfpathlineto{\pgfqpoint{2.040251in}{1.228823in}}%
\pgfpathlineto{\pgfqpoint{2.040634in}{1.119067in}}%
\pgfpathlineto{\pgfqpoint{2.041017in}{1.119067in}}%
\pgfpathlineto{\pgfqpoint{2.041782in}{1.260182in}}%
\pgfpathlineto{\pgfqpoint{2.042548in}{1.207917in}}%
\pgfpathlineto{\pgfqpoint{2.042931in}{1.207917in}}%
\pgfpathlineto{\pgfqpoint{2.042931in}{1.087708in}}%
\pgfpathlineto{\pgfqpoint{2.043697in}{1.291541in}}%
\pgfpathlineto{\pgfqpoint{2.044462in}{1.113840in}}%
\pgfpathlineto{\pgfqpoint{2.044845in}{1.113840in}}%
\pgfpathlineto{\pgfqpoint{2.044845in}{1.322900in}}%
\pgfpathlineto{\pgfqpoint{2.046376in}{1.166105in}}%
\pgfpathlineto{\pgfqpoint{2.046759in}{1.166105in}}%
\pgfpathlineto{\pgfqpoint{2.046759in}{1.192238in}}%
\pgfpathlineto{\pgfqpoint{2.048291in}{1.014537in}}%
\pgfpathlineto{\pgfqpoint{2.048673in}{1.014537in}}%
\pgfpathlineto{\pgfqpoint{2.049822in}{1.197464in}}%
\pgfpathlineto{\pgfqpoint{2.050205in}{1.197464in}}%
\pgfpathlineto{\pgfqpoint{2.050588in}{1.197464in}}%
\pgfpathlineto{\pgfqpoint{2.050970in}{1.061576in}}%
\pgfpathlineto{\pgfqpoint{2.051736in}{1.260182in}}%
\pgfpathlineto{\pgfqpoint{2.052119in}{1.072029in}}%
\pgfpathlineto{\pgfqpoint{2.052502in}{1.072029in}}%
\pgfpathlineto{\pgfqpoint{2.054033in}{1.223597in}}%
\pgfpathlineto{\pgfqpoint{2.054416in}{1.223597in}}%
\pgfpathlineto{\pgfqpoint{2.055564in}{1.124293in}}%
\pgfpathlineto{\pgfqpoint{2.055947in}{1.145199in}}%
\pgfpathlineto{\pgfqpoint{2.056330in}{1.145199in}}%
\pgfpathlineto{\pgfqpoint{2.056713in}{1.103388in}}%
\pgfpathlineto{\pgfqpoint{2.057479in}{1.202691in}}%
\pgfpathlineto{\pgfqpoint{2.057861in}{1.187011in}}%
\pgfpathlineto{\pgfqpoint{2.058244in}{1.187011in}}%
\pgfpathlineto{\pgfqpoint{2.058244in}{1.286314in}}%
\pgfpathlineto{\pgfqpoint{2.059775in}{1.103388in}}%
\pgfpathlineto{\pgfqpoint{2.060158in}{1.103388in}}%
\pgfpathlineto{\pgfqpoint{2.060924in}{1.270635in}}%
\pgfpathlineto{\pgfqpoint{2.061690in}{1.072029in}}%
\pgfpathlineto{\pgfqpoint{2.062072in}{1.072029in}}%
\pgfpathlineto{\pgfqpoint{2.062838in}{0.988405in}}%
\pgfpathlineto{\pgfqpoint{2.063604in}{1.234050in}}%
\pgfpathlineto{\pgfqpoint{2.063987in}{1.234050in}}%
\pgfpathlineto{\pgfqpoint{2.065518in}{1.087708in}}%
\pgfpathlineto{\pgfqpoint{2.065901in}{1.087708in}}%
\pgfpathlineto{\pgfqpoint{2.065901in}{1.244503in}}%
\pgfpathlineto{\pgfqpoint{2.067432in}{1.139973in}}%
\pgfpathlineto{\pgfqpoint{2.067815in}{1.139973in}}%
\pgfpathlineto{\pgfqpoint{2.067815in}{1.160879in}}%
\pgfpathlineto{\pgfqpoint{2.068963in}{1.030217in}}%
\pgfpathlineto{\pgfqpoint{2.069346in}{1.108614in}}%
\pgfpathlineto{\pgfqpoint{2.069729in}{1.108614in}}%
\pgfpathlineto{\pgfqpoint{2.070112in}{1.030217in}}%
\pgfpathlineto{\pgfqpoint{2.070495in}{1.176558in}}%
\pgfpathlineto{\pgfqpoint{2.071260in}{1.176558in}}%
\pgfpathlineto{\pgfqpoint{2.071643in}{1.176558in}}%
\pgfpathlineto{\pgfqpoint{2.072026in}{1.213144in}}%
\pgfpathlineto{\pgfqpoint{2.072792in}{1.056349in}}%
\pgfpathlineto{\pgfqpoint{2.073175in}{1.129520in}}%
\pgfpathlineto{\pgfqpoint{2.073557in}{1.129520in}}%
\pgfpathlineto{\pgfqpoint{2.074323in}{1.176558in}}%
\pgfpathlineto{\pgfqpoint{2.075089in}{1.056349in}}%
\pgfpathlineto{\pgfqpoint{2.075472in}{1.056349in}}%
\pgfpathlineto{\pgfqpoint{2.075854in}{1.176558in}}%
\pgfpathlineto{\pgfqpoint{2.077003in}{1.129520in}}%
\pgfpathlineto{\pgfqpoint{2.077386in}{1.129520in}}%
\pgfpathlineto{\pgfqpoint{2.078534in}{1.077255in}}%
\pgfpathlineto{\pgfqpoint{2.078917in}{1.092935in}}%
\pgfpathlineto{\pgfqpoint{2.079300in}{1.092935in}}%
\pgfpathlineto{\pgfqpoint{2.080448in}{1.244503in}}%
\pgfpathlineto{\pgfqpoint{2.080831in}{1.024990in}}%
\pgfpathlineto{\pgfqpoint{2.081214in}{1.024990in}}%
\pgfpathlineto{\pgfqpoint{2.081214in}{1.181785in}}%
\pgfpathlineto{\pgfqpoint{2.082745in}{1.103388in}}%
\pgfpathlineto{\pgfqpoint{2.083128in}{1.103388in}}%
\pgfpathlineto{\pgfqpoint{2.084277in}{1.009311in}}%
\pgfpathlineto{\pgfqpoint{2.084659in}{1.228823in}}%
\pgfpathlineto{\pgfqpoint{2.085042in}{1.228823in}}%
\pgfpathlineto{\pgfqpoint{2.085808in}{1.024990in}}%
\pgfpathlineto{\pgfqpoint{2.086574in}{1.113840in}}%
\pgfpathlineto{\pgfqpoint{2.086956in}{1.113840in}}%
\pgfpathlineto{\pgfqpoint{2.088105in}{1.019764in}}%
\pgfpathlineto{\pgfqpoint{2.088488in}{1.019764in}}%
\pgfpathlineto{\pgfqpoint{2.089253in}{1.019764in}}%
\pgfpathlineto{\pgfqpoint{2.089253in}{1.009311in}}%
\pgfpathlineto{\pgfqpoint{2.090019in}{1.103388in}}%
\pgfpathlineto{\pgfqpoint{2.090785in}{1.030217in}}%
\pgfpathlineto{\pgfqpoint{2.091168in}{1.030217in}}%
\pgfpathlineto{\pgfqpoint{2.091168in}{0.915234in}}%
\pgfpathlineto{\pgfqpoint{2.091550in}{1.134746in}}%
\pgfpathlineto{\pgfqpoint{2.092699in}{1.087708in}}%
\pgfpathlineto{\pgfqpoint{2.093082in}{1.087708in}}%
\pgfpathlineto{\pgfqpoint{2.093082in}{1.145199in}}%
\pgfpathlineto{\pgfqpoint{2.094613in}{1.014537in}}%
\pgfpathlineto{\pgfqpoint{2.094996in}{1.014537in}}%
\pgfpathlineto{\pgfqpoint{2.094996in}{1.124293in}}%
\pgfpathlineto{\pgfqpoint{2.095379in}{0.988405in}}%
\pgfpathlineto{\pgfqpoint{2.096527in}{1.019764in}}%
\pgfpathlineto{\pgfqpoint{2.096910in}{1.019764in}}%
\pgfpathlineto{\pgfqpoint{2.096910in}{1.113840in}}%
\pgfpathlineto{\pgfqpoint{2.097293in}{0.988405in}}%
\pgfpathlineto{\pgfqpoint{2.098441in}{0.988405in}}%
\pgfpathlineto{\pgfqpoint{2.098824in}{0.988405in}}%
\pgfpathlineto{\pgfqpoint{2.099207in}{1.150426in}}%
\pgfpathlineto{\pgfqpoint{2.100355in}{1.113840in}}%
\pgfpathlineto{\pgfqpoint{2.100738in}{1.113840in}}%
\pgfpathlineto{\pgfqpoint{2.101887in}{0.983178in}}%
\pgfpathlineto{\pgfqpoint{2.102270in}{1.030217in}}%
\pgfpathlineto{\pgfqpoint{2.102652in}{1.030217in}}%
\pgfpathlineto{\pgfqpoint{2.102652in}{0.983178in}}%
\pgfpathlineto{\pgfqpoint{2.103801in}{1.077255in}}%
\pgfpathlineto{\pgfqpoint{2.104184in}{1.051123in}}%
\pgfpathlineto{\pgfqpoint{2.104949in}{1.051123in}}%
\pgfpathlineto{\pgfqpoint{2.105332in}{0.967499in}}%
\pgfpathlineto{\pgfqpoint{2.105715in}{1.077255in}}%
\pgfpathlineto{\pgfqpoint{2.106481in}{1.004084in}}%
\pgfpathlineto{\pgfqpoint{2.106864in}{1.004084in}}%
\pgfpathlineto{\pgfqpoint{2.107629in}{1.155652in}}%
\pgfpathlineto{\pgfqpoint{2.108395in}{1.004084in}}%
\pgfpathlineto{\pgfqpoint{2.108778in}{1.004084in}}%
\pgfpathlineto{\pgfqpoint{2.109161in}{0.946593in}}%
\pgfpathlineto{\pgfqpoint{2.109543in}{1.092935in}}%
\pgfpathlineto{\pgfqpoint{2.110309in}{0.967499in}}%
\pgfpathlineto{\pgfqpoint{2.110692in}{0.967499in}}%
\pgfpathlineto{\pgfqpoint{2.110692in}{0.962273in}}%
\pgfpathlineto{\pgfqpoint{2.111458in}{1.061576in}}%
\pgfpathlineto{\pgfqpoint{2.112223in}{1.035443in}}%
\pgfpathlineto{\pgfqpoint{2.112606in}{1.035443in}}%
\pgfpathlineto{\pgfqpoint{2.112606in}{1.056349in}}%
\pgfpathlineto{\pgfqpoint{2.112989in}{0.925687in}}%
\pgfpathlineto{\pgfqpoint{2.114137in}{1.045896in}}%
\pgfpathlineto{\pgfqpoint{2.114520in}{1.045896in}}%
\pgfpathlineto{\pgfqpoint{2.115286in}{0.946593in}}%
\pgfpathlineto{\pgfqpoint{2.115669in}{1.066802in}}%
\pgfpathlineto{\pgfqpoint{2.116052in}{1.061576in}}%
\pgfpathlineto{\pgfqpoint{2.116434in}{1.061576in}}%
\pgfpathlineto{\pgfqpoint{2.116817in}{0.988405in}}%
\pgfpathlineto{\pgfqpoint{2.117200in}{1.066802in}}%
\pgfpathlineto{\pgfqpoint{2.117966in}{1.035443in}}%
\pgfpathlineto{\pgfqpoint{2.118348in}{1.035443in}}%
\pgfpathlineto{\pgfqpoint{2.119114in}{1.056349in}}%
\pgfpathlineto{\pgfqpoint{2.119497in}{0.941367in}}%
\pgfpathlineto{\pgfqpoint{2.119880in}{1.056349in}}%
\pgfpathlineto{\pgfqpoint{2.120263in}{1.056349in}}%
\pgfpathlineto{\pgfqpoint{2.120263in}{1.061576in}}%
\pgfpathlineto{\pgfqpoint{2.121411in}{0.972726in}}%
\pgfpathlineto{\pgfqpoint{2.121794in}{0.983178in}}%
\pgfpathlineto{\pgfqpoint{2.122177in}{0.983178in}}%
\pgfpathlineto{\pgfqpoint{2.122942in}{0.967499in}}%
\pgfpathlineto{\pgfqpoint{2.122560in}{1.072029in}}%
\pgfpathlineto{\pgfqpoint{2.123708in}{0.993631in}}%
\pgfpathlineto{\pgfqpoint{2.124091in}{0.993631in}}%
\pgfpathlineto{\pgfqpoint{2.124091in}{1.051123in}}%
\pgfpathlineto{\pgfqpoint{2.125622in}{0.878649in}}%
\pgfpathlineto{\pgfqpoint{2.126005in}{0.878649in}}%
\pgfpathlineto{\pgfqpoint{2.126388in}{1.077255in}}%
\pgfpathlineto{\pgfqpoint{2.127536in}{0.920461in}}%
\pgfpathlineto{\pgfqpoint{2.127919in}{0.920461in}}%
\pgfpathlineto{\pgfqpoint{2.128302in}{1.014537in}}%
\pgfpathlineto{\pgfqpoint{2.129451in}{0.899555in}}%
\pgfpathlineto{\pgfqpoint{2.129833in}{0.899555in}}%
\pgfpathlineto{\pgfqpoint{2.130216in}{1.040670in}}%
\pgfpathlineto{\pgfqpoint{2.131365in}{0.915234in}}%
\pgfpathlineto{\pgfqpoint{2.131748in}{0.915234in}}%
\pgfpathlineto{\pgfqpoint{2.131748in}{0.993631in}}%
\pgfpathlineto{\pgfqpoint{2.133279in}{0.951820in}}%
\pgfpathlineto{\pgfqpoint{2.133662in}{0.951820in}}%
\pgfpathlineto{\pgfqpoint{2.133662in}{0.941367in}}%
\pgfpathlineto{\pgfqpoint{2.134427in}{1.035443in}}%
\pgfpathlineto{\pgfqpoint{2.135193in}{0.946593in}}%
\pgfpathlineto{\pgfqpoint{2.135576in}{0.946593in}}%
\pgfpathlineto{\pgfqpoint{2.135576in}{0.915234in}}%
\pgfpathlineto{\pgfqpoint{2.136724in}{1.098161in}}%
\pgfpathlineto{\pgfqpoint{2.137107in}{0.998858in}}%
\pgfpathlineto{\pgfqpoint{2.137490in}{0.998858in}}%
\pgfpathlineto{\pgfqpoint{2.137873in}{0.920461in}}%
\pgfpathlineto{\pgfqpoint{2.139021in}{0.988405in}}%
\pgfpathlineto{\pgfqpoint{2.139404in}{0.988405in}}%
\pgfpathlineto{\pgfqpoint{2.139404in}{1.040670in}}%
\pgfpathlineto{\pgfqpoint{2.139787in}{0.941367in}}%
\pgfpathlineto{\pgfqpoint{2.140935in}{0.941367in}}%
\pgfpathlineto{\pgfqpoint{2.141318in}{0.941367in}}%
\pgfpathlineto{\pgfqpoint{2.141701in}{1.014537in}}%
\pgfpathlineto{\pgfqpoint{2.142850in}{1.014537in}}%
\pgfpathlineto{\pgfqpoint{2.143232in}{1.014537in}}%
\pgfpathlineto{\pgfqpoint{2.143998in}{0.889102in}}%
\pgfpathlineto{\pgfqpoint{2.144764in}{1.051123in}}%
\pgfpathlineto{\pgfqpoint{2.145147in}{1.051123in}}%
\pgfpathlineto{\pgfqpoint{2.145912in}{0.883875in}}%
\pgfpathlineto{\pgfqpoint{2.146678in}{0.883875in}}%
\pgfpathlineto{\pgfqpoint{2.147061in}{0.883875in}}%
\pgfpathlineto{\pgfqpoint{2.147826in}{1.024990in}}%
\pgfpathlineto{\pgfqpoint{2.148592in}{0.972726in}}%
\pgfpathlineto{\pgfqpoint{2.149358in}{0.972726in}}%
\pgfpathlineto{\pgfqpoint{2.150506in}{0.826384in}}%
\pgfpathlineto{\pgfqpoint{2.150123in}{1.056349in}}%
\pgfpathlineto{\pgfqpoint{2.150889in}{0.967499in}}%
\pgfpathlineto{\pgfqpoint{2.151272in}{0.967499in}}%
\pgfpathlineto{\pgfqpoint{2.151655in}{0.962273in}}%
\pgfpathlineto{\pgfqpoint{2.152803in}{1.004084in}}%
\pgfpathlineto{\pgfqpoint{2.153186in}{1.004084in}}%
\pgfpathlineto{\pgfqpoint{2.153952in}{0.842063in}}%
\pgfpathlineto{\pgfqpoint{2.154717in}{1.009311in}}%
\pgfpathlineto{\pgfqpoint{2.155100in}{1.009311in}}%
\pgfpathlineto{\pgfqpoint{2.155100in}{1.024990in}}%
\pgfpathlineto{\pgfqpoint{2.155866in}{0.899555in}}%
\pgfpathlineto{\pgfqpoint{2.156631in}{0.910008in}}%
\pgfpathlineto{\pgfqpoint{2.157014in}{0.910008in}}%
\pgfpathlineto{\pgfqpoint{2.158163in}{0.983178in}}%
\pgfpathlineto{\pgfqpoint{2.158546in}{0.957046in}}%
\pgfpathlineto{\pgfqpoint{2.158928in}{0.957046in}}%
\pgfpathlineto{\pgfqpoint{2.159311in}{1.009311in}}%
\pgfpathlineto{\pgfqpoint{2.160077in}{0.904781in}}%
\pgfpathlineto{\pgfqpoint{2.160460in}{0.930914in}}%
\pgfpathlineto{\pgfqpoint{2.160843in}{0.930914in}}%
\pgfpathlineto{\pgfqpoint{2.160843in}{0.962273in}}%
\pgfpathlineto{\pgfqpoint{2.161608in}{0.904781in}}%
\pgfpathlineto{\pgfqpoint{2.162374in}{0.941367in}}%
\pgfpathlineto{\pgfqpoint{2.162757in}{0.941367in}}%
\pgfpathlineto{\pgfqpoint{2.163140in}{0.894328in}}%
\pgfpathlineto{\pgfqpoint{2.163905in}{1.024990in}}%
\pgfpathlineto{\pgfqpoint{2.164288in}{0.936140in}}%
\pgfpathlineto{\pgfqpoint{2.165054in}{0.936140in}}%
\pgfpathlineto{\pgfqpoint{2.165437in}{0.889102in}}%
\pgfpathlineto{\pgfqpoint{2.166202in}{0.983178in}}%
\pgfpathlineto{\pgfqpoint{2.166585in}{0.936140in}}%
\pgfpathlineto{\pgfqpoint{2.166968in}{0.936140in}}%
\pgfpathlineto{\pgfqpoint{2.166968in}{0.998858in}}%
\pgfpathlineto{\pgfqpoint{2.168499in}{0.894328in}}%
\pgfpathlineto{\pgfqpoint{2.168882in}{0.894328in}}%
\pgfpathlineto{\pgfqpoint{2.169265in}{0.962273in}}%
\pgfpathlineto{\pgfqpoint{2.170031in}{0.847290in}}%
\pgfpathlineto{\pgfqpoint{2.170413in}{0.920461in}}%
\pgfpathlineto{\pgfqpoint{2.170796in}{0.920461in}}%
\pgfpathlineto{\pgfqpoint{2.170796in}{0.842063in}}%
\pgfpathlineto{\pgfqpoint{2.171945in}{0.983178in}}%
\pgfpathlineto{\pgfqpoint{2.172328in}{0.852516in}}%
\pgfpathlineto{\pgfqpoint{2.172710in}{0.852516in}}%
\pgfpathlineto{\pgfqpoint{2.173093in}{0.930914in}}%
\pgfpathlineto{\pgfqpoint{2.174242in}{0.899555in}}%
\pgfpathlineto{\pgfqpoint{2.174625in}{0.899555in}}%
\pgfpathlineto{\pgfqpoint{2.174625in}{0.946593in}}%
\pgfpathlineto{\pgfqpoint{2.175007in}{0.878649in}}%
\pgfpathlineto{\pgfqpoint{2.176156in}{0.920461in}}%
\pgfpathlineto{\pgfqpoint{2.176539in}{0.920461in}}%
\pgfpathlineto{\pgfqpoint{2.177304in}{0.847290in}}%
\pgfpathlineto{\pgfqpoint{2.177687in}{0.957046in}}%
\pgfpathlineto{\pgfqpoint{2.178070in}{0.946593in}}%
\pgfpathlineto{\pgfqpoint{2.178453in}{0.946593in}}%
\pgfpathlineto{\pgfqpoint{2.178453in}{0.889102in}}%
\pgfpathlineto{\pgfqpoint{2.179601in}{0.983178in}}%
\pgfpathlineto{\pgfqpoint{2.179984in}{0.889102in}}%
\pgfpathlineto{\pgfqpoint{2.180367in}{0.889102in}}%
\pgfpathlineto{\pgfqpoint{2.180367in}{0.857743in}}%
\pgfpathlineto{\pgfqpoint{2.181515in}{0.915234in}}%
\pgfpathlineto{\pgfqpoint{2.181898in}{0.857743in}}%
\pgfpathlineto{\pgfqpoint{2.182664in}{0.857743in}}%
\pgfpathlineto{\pgfqpoint{2.182664in}{0.925687in}}%
\pgfpathlineto{\pgfqpoint{2.184195in}{0.831611in}}%
\pgfpathlineto{\pgfqpoint{2.184578in}{0.831611in}}%
\pgfpathlineto{\pgfqpoint{2.184961in}{0.946593in}}%
\pgfpathlineto{\pgfqpoint{2.186109in}{0.862969in}}%
\pgfpathlineto{\pgfqpoint{2.186492in}{0.862969in}}%
\pgfpathlineto{\pgfqpoint{2.186875in}{0.930914in}}%
\pgfpathlineto{\pgfqpoint{2.188024in}{0.878649in}}%
\pgfpathlineto{\pgfqpoint{2.188406in}{0.878649in}}%
\pgfpathlineto{\pgfqpoint{2.188789in}{0.925687in}}%
\pgfpathlineto{\pgfqpoint{2.189938in}{0.857743in}}%
\pgfpathlineto{\pgfqpoint{2.190321in}{0.857743in}}%
\pgfpathlineto{\pgfqpoint{2.190321in}{0.957046in}}%
\pgfpathlineto{\pgfqpoint{2.191852in}{0.862969in}}%
\pgfpathlineto{\pgfqpoint{2.192235in}{0.862969in}}%
\pgfpathlineto{\pgfqpoint{2.192618in}{0.852516in}}%
\pgfpathlineto{\pgfqpoint{2.193766in}{0.930914in}}%
\pgfpathlineto{\pgfqpoint{2.194149in}{0.930914in}}%
\pgfpathlineto{\pgfqpoint{2.194532in}{0.857743in}}%
\pgfpathlineto{\pgfqpoint{2.195680in}{0.957046in}}%
\pgfpathlineto{\pgfqpoint{2.196063in}{0.957046in}}%
\pgfpathlineto{\pgfqpoint{2.197594in}{0.815931in}}%
\pgfpathlineto{\pgfqpoint{2.197977in}{0.815931in}}%
\pgfpathlineto{\pgfqpoint{2.199126in}{0.930914in}}%
\pgfpathlineto{\pgfqpoint{2.199508in}{0.878649in}}%
\pgfpathlineto{\pgfqpoint{2.199891in}{0.878649in}}%
\pgfpathlineto{\pgfqpoint{2.200274in}{0.967499in}}%
\pgfpathlineto{\pgfqpoint{2.201423in}{0.805478in}}%
\pgfpathlineto{\pgfqpoint{2.201805in}{0.805478in}}%
\pgfpathlineto{\pgfqpoint{2.201805in}{0.910008in}}%
\pgfpathlineto{\pgfqpoint{2.203337in}{0.810705in}}%
\pgfpathlineto{\pgfqpoint{2.203720in}{0.810705in}}%
\pgfpathlineto{\pgfqpoint{2.204485in}{0.899555in}}%
\pgfpathlineto{\pgfqpoint{2.205251in}{0.873422in}}%
\pgfpathlineto{\pgfqpoint{2.205634in}{0.873422in}}%
\pgfpathlineto{\pgfqpoint{2.206399in}{0.930914in}}%
\pgfpathlineto{\pgfqpoint{2.207165in}{0.847290in}}%
\pgfpathlineto{\pgfqpoint{2.207548in}{0.847290in}}%
\pgfpathlineto{\pgfqpoint{2.207931in}{0.920461in}}%
\pgfpathlineto{\pgfqpoint{2.208314in}{0.815931in}}%
\pgfpathlineto{\pgfqpoint{2.209079in}{0.873422in}}%
\pgfpathlineto{\pgfqpoint{2.209462in}{0.873422in}}%
\pgfpathlineto{\pgfqpoint{2.210611in}{0.925687in}}%
\pgfpathlineto{\pgfqpoint{2.210993in}{0.810705in}}%
\pgfpathlineto{\pgfqpoint{2.211376in}{0.810705in}}%
\pgfpathlineto{\pgfqpoint{2.211376in}{0.800252in}}%
\pgfpathlineto{\pgfqpoint{2.212142in}{0.883875in}}%
\pgfpathlineto{\pgfqpoint{2.212908in}{0.831611in}}%
\pgfpathlineto{\pgfqpoint{2.213290in}{0.831611in}}%
\pgfpathlineto{\pgfqpoint{2.213673in}{0.910008in}}%
\pgfpathlineto{\pgfqpoint{2.214056in}{0.826384in}}%
\pgfpathlineto{\pgfqpoint{2.214822in}{0.883875in}}%
\pgfpathlineto{\pgfqpoint{2.215204in}{0.883875in}}%
\pgfpathlineto{\pgfqpoint{2.216353in}{0.836837in}}%
\pgfpathlineto{\pgfqpoint{2.215970in}{0.925687in}}%
\pgfpathlineto{\pgfqpoint{2.216736in}{0.842063in}}%
\pgfpathlineto{\pgfqpoint{2.217119in}{0.842063in}}%
\pgfpathlineto{\pgfqpoint{2.218267in}{0.831611in}}%
\pgfpathlineto{\pgfqpoint{2.218650in}{0.894328in}}%
\pgfpathlineto{\pgfqpoint{2.219033in}{0.894328in}}%
\pgfpathlineto{\pgfqpoint{2.220564in}{0.789799in}}%
\pgfpathlineto{\pgfqpoint{2.220947in}{0.789799in}}%
\pgfpathlineto{\pgfqpoint{2.222478in}{0.873422in}}%
\pgfpathlineto{\pgfqpoint{2.222861in}{0.873422in}}%
\pgfpathlineto{\pgfqpoint{2.222861in}{0.789799in}}%
\pgfpathlineto{\pgfqpoint{2.224392in}{0.894328in}}%
\pgfpathlineto{\pgfqpoint{2.224775in}{0.894328in}}%
\pgfpathlineto{\pgfqpoint{2.225541in}{0.810705in}}%
\pgfpathlineto{\pgfqpoint{2.226307in}{0.831611in}}%
\pgfpathlineto{\pgfqpoint{2.227072in}{0.831611in}}%
\pgfpathlineto{\pgfqpoint{2.227838in}{0.899555in}}%
\pgfpathlineto{\pgfqpoint{2.228604in}{0.795025in}}%
\pgfpathlineto{\pgfqpoint{2.228986in}{0.795025in}}%
\pgfpathlineto{\pgfqpoint{2.230518in}{0.894328in}}%
\pgfpathlineto{\pgfqpoint{2.230901in}{0.894328in}}%
\pgfpathlineto{\pgfqpoint{2.231283in}{0.774119in}}%
\pgfpathlineto{\pgfqpoint{2.232432in}{0.889102in}}%
\pgfpathlineto{\pgfqpoint{2.232815in}{0.889102in}}%
\pgfpathlineto{\pgfqpoint{2.233963in}{0.805478in}}%
\pgfpathlineto{\pgfqpoint{2.233580in}{0.899555in}}%
\pgfpathlineto{\pgfqpoint{2.234346in}{0.852516in}}%
\pgfpathlineto{\pgfqpoint{2.234729in}{0.852516in}}%
\pgfpathlineto{\pgfqpoint{2.235112in}{0.836837in}}%
\pgfpathlineto{\pgfqpoint{2.236260in}{0.904781in}}%
\pgfpathlineto{\pgfqpoint{2.236643in}{0.904781in}}%
\pgfpathlineto{\pgfqpoint{2.238174in}{0.789799in}}%
\pgfpathlineto{\pgfqpoint{2.238557in}{0.789799in}}%
\pgfpathlineto{\pgfqpoint{2.239323in}{0.878649in}}%
\pgfpathlineto{\pgfqpoint{2.240088in}{0.831611in}}%
\pgfpathlineto{\pgfqpoint{2.240471in}{0.831611in}}%
\pgfpathlineto{\pgfqpoint{2.240471in}{0.779346in}}%
\pgfpathlineto{\pgfqpoint{2.242003in}{0.810705in}}%
\pgfpathlineto{\pgfqpoint{2.242385in}{0.810705in}}%
\pgfpathlineto{\pgfqpoint{2.243917in}{0.847290in}}%
\pgfpathlineto{\pgfqpoint{2.244300in}{0.847290in}}%
\pgfpathlineto{\pgfqpoint{2.244682in}{0.826384in}}%
\pgfpathlineto{\pgfqpoint{2.245065in}{0.910008in}}%
\pgfpathlineto{\pgfqpoint{2.245831in}{0.842063in}}%
\pgfpathlineto{\pgfqpoint{2.246214in}{0.842063in}}%
\pgfpathlineto{\pgfqpoint{2.246214in}{0.852516in}}%
\pgfpathlineto{\pgfqpoint{2.246979in}{0.753213in}}%
\pgfpathlineto{\pgfqpoint{2.247745in}{0.753213in}}%
\pgfpathlineto{\pgfqpoint{2.248128in}{0.753213in}}%
\pgfpathlineto{\pgfqpoint{2.249276in}{0.842063in}}%
\pgfpathlineto{\pgfqpoint{2.249659in}{0.831611in}}%
\pgfpathlineto{\pgfqpoint{2.250042in}{0.831611in}}%
\pgfpathlineto{\pgfqpoint{2.250042in}{0.883875in}}%
\pgfpathlineto{\pgfqpoint{2.251191in}{0.784572in}}%
\pgfpathlineto{\pgfqpoint{2.251573in}{0.836837in}}%
\pgfpathlineto{\pgfqpoint{2.251956in}{0.836837in}}%
\pgfpathlineto{\pgfqpoint{2.251956in}{0.815931in}}%
\pgfpathlineto{\pgfqpoint{2.253487in}{0.873422in}}%
\pgfpathlineto{\pgfqpoint{2.253870in}{0.873422in}}%
\pgfpathlineto{\pgfqpoint{2.255402in}{0.789799in}}%
\pgfpathlineto{\pgfqpoint{2.255784in}{0.789799in}}%
\pgfpathlineto{\pgfqpoint{2.256933in}{0.847290in}}%
\pgfpathlineto{\pgfqpoint{2.256167in}{0.774119in}}%
\pgfpathlineto{\pgfqpoint{2.257316in}{0.842063in}}%
\pgfpathlineto{\pgfqpoint{2.257699in}{0.842063in}}%
\pgfpathlineto{\pgfqpoint{2.257699in}{0.779346in}}%
\pgfpathlineto{\pgfqpoint{2.259230in}{0.784572in}}%
\pgfpathlineto{\pgfqpoint{2.259613in}{0.784572in}}%
\pgfpathlineto{\pgfqpoint{2.261144in}{0.826384in}}%
\pgfpathlineto{\pgfqpoint{2.261910in}{0.826384in}}%
\pgfpathlineto{\pgfqpoint{2.262293in}{0.779346in}}%
\pgfpathlineto{\pgfqpoint{2.262675in}{0.889102in}}%
\pgfpathlineto{\pgfqpoint{2.263441in}{0.810705in}}%
\pgfpathlineto{\pgfqpoint{2.264207in}{0.810705in}}%
\pgfpathlineto{\pgfqpoint{2.264207in}{0.868196in}}%
\pgfpathlineto{\pgfqpoint{2.264972in}{0.747987in}}%
\pgfpathlineto{\pgfqpoint{2.265738in}{0.836837in}}%
\pgfpathlineto{\pgfqpoint{2.266121in}{0.836837in}}%
\pgfpathlineto{\pgfqpoint{2.266504in}{0.774119in}}%
\pgfpathlineto{\pgfqpoint{2.267652in}{0.842063in}}%
\pgfpathlineto{\pgfqpoint{2.268035in}{0.842063in}}%
\pgfpathlineto{\pgfqpoint{2.268801in}{0.753213in}}%
\pgfpathlineto{\pgfqpoint{2.269566in}{0.821158in}}%
\pgfpathlineto{\pgfqpoint{2.269949in}{0.821158in}}%
\pgfpathlineto{\pgfqpoint{2.270332in}{0.758440in}}%
\pgfpathlineto{\pgfqpoint{2.271481in}{0.758440in}}%
\pgfpathlineto{\pgfqpoint{2.271863in}{0.758440in}}%
\pgfpathlineto{\pgfqpoint{2.272246in}{0.732307in}}%
\pgfpathlineto{\pgfqpoint{2.272629in}{0.831611in}}%
\pgfpathlineto{\pgfqpoint{2.273395in}{0.789799in}}%
\pgfpathlineto{\pgfqpoint{2.273777in}{0.789799in}}%
\pgfpathlineto{\pgfqpoint{2.273777in}{0.857743in}}%
\pgfpathlineto{\pgfqpoint{2.275309in}{0.831611in}}%
\pgfpathlineto{\pgfqpoint{2.275692in}{0.831611in}}%
\pgfpathlineto{\pgfqpoint{2.276457in}{0.784572in}}%
\pgfpathlineto{\pgfqpoint{2.276840in}{0.857743in}}%
\pgfpathlineto{\pgfqpoint{2.277223in}{0.857743in}}%
\pgfpathlineto{\pgfqpoint{2.277606in}{0.857743in}}%
\pgfpathlineto{\pgfqpoint{2.277989in}{0.763666in}}%
\pgfpathlineto{\pgfqpoint{2.279137in}{0.831611in}}%
\pgfpathlineto{\pgfqpoint{2.279520in}{0.831611in}}%
\pgfpathlineto{\pgfqpoint{2.279903in}{0.742760in}}%
\pgfpathlineto{\pgfqpoint{2.281051in}{0.789799in}}%
\pgfpathlineto{\pgfqpoint{2.281434in}{0.789799in}}%
\pgfpathlineto{\pgfqpoint{2.281434in}{0.810705in}}%
\pgfpathlineto{\pgfqpoint{2.282200in}{0.747987in}}%
\pgfpathlineto{\pgfqpoint{2.282965in}{0.747987in}}%
\pgfpathlineto{\pgfqpoint{2.283348in}{0.747987in}}%
\pgfpathlineto{\pgfqpoint{2.284497in}{0.768893in}}%
\pgfpathlineto{\pgfqpoint{2.284114in}{0.706175in}}%
\pgfpathlineto{\pgfqpoint{2.284880in}{0.758440in}}%
\pgfpathlineto{\pgfqpoint{2.285262in}{0.758440in}}%
\pgfpathlineto{\pgfqpoint{2.286028in}{0.800252in}}%
\pgfpathlineto{\pgfqpoint{2.286411in}{0.737534in}}%
\pgfpathlineto{\pgfqpoint{2.286794in}{0.758440in}}%
\pgfpathlineto{\pgfqpoint{2.287177in}{0.758440in}}%
\pgfpathlineto{\pgfqpoint{2.287559in}{0.821158in}}%
\pgfpathlineto{\pgfqpoint{2.288708in}{0.742760in}}%
\pgfpathlineto{\pgfqpoint{2.289091in}{0.742760in}}%
\pgfpathlineto{\pgfqpoint{2.289856in}{0.826384in}}%
\pgfpathlineto{\pgfqpoint{2.290622in}{0.795025in}}%
\pgfpathlineto{\pgfqpoint{2.291388in}{0.795025in}}%
\pgfpathlineto{\pgfqpoint{2.291770in}{0.768893in}}%
\pgfpathlineto{\pgfqpoint{2.292919in}{0.826384in}}%
\pgfpathlineto{\pgfqpoint{2.293302in}{0.826384in}}%
\pgfpathlineto{\pgfqpoint{2.293302in}{0.763666in}}%
\pgfpathlineto{\pgfqpoint{2.294833in}{0.774119in}}%
\pgfpathlineto{\pgfqpoint{2.295216in}{0.774119in}}%
\pgfpathlineto{\pgfqpoint{2.295599in}{0.821158in}}%
\pgfpathlineto{\pgfqpoint{2.296747in}{0.732307in}}%
\pgfpathlineto{\pgfqpoint{2.297130in}{0.732307in}}%
\pgfpathlineto{\pgfqpoint{2.298279in}{0.800252in}}%
\pgfpathlineto{\pgfqpoint{2.298661in}{0.763666in}}%
\pgfpathlineto{\pgfqpoint{2.299044in}{0.763666in}}%
\pgfpathlineto{\pgfqpoint{2.299810in}{0.800252in}}%
\pgfpathlineto{\pgfqpoint{2.299427in}{0.737534in}}%
\pgfpathlineto{\pgfqpoint{2.300576in}{0.742760in}}%
\pgfpathlineto{\pgfqpoint{2.300958in}{0.742760in}}%
\pgfpathlineto{\pgfqpoint{2.301724in}{0.805478in}}%
\pgfpathlineto{\pgfqpoint{2.302490in}{0.742760in}}%
\pgfpathlineto{\pgfqpoint{2.302873in}{0.742760in}}%
\pgfpathlineto{\pgfqpoint{2.304404in}{0.795025in}}%
\pgfpathlineto{\pgfqpoint{2.304787in}{0.795025in}}%
\pgfpathlineto{\pgfqpoint{2.304787in}{0.674816in}}%
\pgfpathlineto{\pgfqpoint{2.306318in}{0.747987in}}%
\pgfpathlineto{\pgfqpoint{2.306701in}{0.747987in}}%
\pgfpathlineto{\pgfqpoint{2.306701in}{0.795025in}}%
\pgfpathlineto{\pgfqpoint{2.307084in}{0.721854in}}%
\pgfpathlineto{\pgfqpoint{2.308232in}{0.753213in}}%
\pgfpathlineto{\pgfqpoint{2.308615in}{0.753213in}}%
\pgfpathlineto{\pgfqpoint{2.309764in}{0.732307in}}%
\pgfpathlineto{\pgfqpoint{2.308998in}{0.774119in}}%
\pgfpathlineto{\pgfqpoint{2.310146in}{0.768893in}}%
\pgfpathlineto{\pgfqpoint{2.310529in}{0.768893in}}%
\pgfpathlineto{\pgfqpoint{2.310529in}{0.774119in}}%
\pgfpathlineto{\pgfqpoint{2.311295in}{0.721854in}}%
\pgfpathlineto{\pgfqpoint{2.312060in}{0.753213in}}%
\pgfpathlineto{\pgfqpoint{2.312826in}{0.753213in}}%
\pgfpathlineto{\pgfqpoint{2.312826in}{0.758440in}}%
\pgfpathlineto{\pgfqpoint{2.313209in}{0.732307in}}%
\pgfpathlineto{\pgfqpoint{2.314357in}{0.758440in}}%
\pgfpathlineto{\pgfqpoint{2.314740in}{0.758440in}}%
\pgfpathlineto{\pgfqpoint{2.315123in}{0.727081in}}%
\pgfpathlineto{\pgfqpoint{2.315889in}{0.774119in}}%
\pgfpathlineto{\pgfqpoint{2.316272in}{0.737534in}}%
\pgfpathlineto{\pgfqpoint{2.316654in}{0.737534in}}%
\pgfpathlineto{\pgfqpoint{2.318186in}{0.779346in}}%
\pgfpathlineto{\pgfqpoint{2.318569in}{0.779346in}}%
\pgfpathlineto{\pgfqpoint{2.320100in}{0.721854in}}%
\pgfpathlineto{\pgfqpoint{2.320866in}{0.721854in}}%
\pgfpathlineto{\pgfqpoint{2.322014in}{0.706175in}}%
\pgfpathlineto{\pgfqpoint{2.322397in}{0.779346in}}%
\pgfpathlineto{\pgfqpoint{2.323163in}{0.779346in}}%
\pgfpathlineto{\pgfqpoint{2.323545in}{0.706175in}}%
\pgfpathlineto{\pgfqpoint{2.324694in}{0.732307in}}%
\pgfpathlineto{\pgfqpoint{2.325077in}{0.732307in}}%
\pgfpathlineto{\pgfqpoint{2.326225in}{0.706175in}}%
\pgfpathlineto{\pgfqpoint{2.326608in}{0.795025in}}%
\pgfpathlineto{\pgfqpoint{2.326991in}{0.795025in}}%
\pgfpathlineto{\pgfqpoint{2.328522in}{0.716628in}}%
\pgfpathlineto{\pgfqpoint{2.328905in}{0.716628in}}%
\pgfpathlineto{\pgfqpoint{2.329288in}{0.747987in}}%
\pgfpathlineto{\pgfqpoint{2.329671in}{0.711401in}}%
\pgfpathlineto{\pgfqpoint{2.330436in}{0.737534in}}%
\pgfpathlineto{\pgfqpoint{2.330819in}{0.737534in}}%
\pgfpathlineto{\pgfqpoint{2.330819in}{0.685269in}}%
\pgfpathlineto{\pgfqpoint{2.332350in}{0.732307in}}%
\pgfpathlineto{\pgfqpoint{2.332733in}{0.732307in}}%
\pgfpathlineto{\pgfqpoint{2.332733in}{0.763666in}}%
\pgfpathlineto{\pgfqpoint{2.334265in}{0.737534in}}%
\pgfpathlineto{\pgfqpoint{2.334647in}{0.737534in}}%
\pgfpathlineto{\pgfqpoint{2.334647in}{0.711401in}}%
\pgfpathlineto{\pgfqpoint{2.336179in}{0.753213in}}%
\pgfpathlineto{\pgfqpoint{2.336944in}{0.753213in}}%
\pgfpathlineto{\pgfqpoint{2.336944in}{0.711401in}}%
\pgfpathlineto{\pgfqpoint{2.338093in}{0.763666in}}%
\pgfpathlineto{\pgfqpoint{2.338476in}{0.763666in}}%
\pgfpathlineto{\pgfqpoint{2.338859in}{0.763666in}}%
\pgfpathlineto{\pgfqpoint{2.339624in}{0.706175in}}%
\pgfpathlineto{\pgfqpoint{2.340007in}{0.774119in}}%
\pgfpathlineto{\pgfqpoint{2.340390in}{0.742760in}}%
\pgfpathlineto{\pgfqpoint{2.341156in}{0.742760in}}%
\pgfpathlineto{\pgfqpoint{2.341156in}{0.695722in}}%
\pgfpathlineto{\pgfqpoint{2.341538in}{0.747987in}}%
\pgfpathlineto{\pgfqpoint{2.342687in}{0.732307in}}%
\pgfpathlineto{\pgfqpoint{2.343070in}{0.732307in}}%
\pgfpathlineto{\pgfqpoint{2.343835in}{0.674816in}}%
\pgfpathlineto{\pgfqpoint{2.344601in}{0.732307in}}%
\pgfpathlineto{\pgfqpoint{2.344984in}{0.732307in}}%
\pgfpathlineto{\pgfqpoint{2.344984in}{0.737534in}}%
\pgfpathlineto{\pgfqpoint{2.346132in}{0.706175in}}%
\pgfpathlineto{\pgfqpoint{2.346515in}{0.737534in}}%
\pgfpathlineto{\pgfqpoint{2.346898in}{0.737534in}}%
\pgfpathlineto{\pgfqpoint{2.348429in}{0.695722in}}%
\pgfpathlineto{\pgfqpoint{2.348812in}{0.695722in}}%
\pgfpathlineto{\pgfqpoint{2.349195in}{0.732307in}}%
\pgfpathlineto{\pgfqpoint{2.349961in}{0.690496in}}%
\pgfpathlineto{\pgfqpoint{2.350343in}{0.690496in}}%
\pgfpathlineto{\pgfqpoint{2.350726in}{0.690496in}}%
\pgfpathlineto{\pgfqpoint{2.351875in}{0.742760in}}%
\pgfpathlineto{\pgfqpoint{2.352258in}{0.706175in}}%
\pgfpathlineto{\pgfqpoint{2.353023in}{0.706175in}}%
\pgfpathlineto{\pgfqpoint{2.353023in}{0.747987in}}%
\pgfpathlineto{\pgfqpoint{2.353789in}{0.700949in}}%
\pgfpathlineto{\pgfqpoint{2.354555in}{0.732307in}}%
\pgfpathlineto{\pgfqpoint{2.354937in}{0.732307in}}%
\pgfpathlineto{\pgfqpoint{2.355703in}{0.700949in}}%
\pgfpathlineto{\pgfqpoint{2.356469in}{0.742760in}}%
\pgfpathlineto{\pgfqpoint{2.356852in}{0.742760in}}%
\pgfpathlineto{\pgfqpoint{2.356852in}{0.674816in}}%
\pgfpathlineto{\pgfqpoint{2.358383in}{0.700949in}}%
\pgfpathlineto{\pgfqpoint{2.358766in}{0.700949in}}%
\pgfpathlineto{\pgfqpoint{2.358766in}{0.727081in}}%
\pgfpathlineto{\pgfqpoint{2.360297in}{0.685269in}}%
\pgfpathlineto{\pgfqpoint{2.360680in}{0.685269in}}%
\pgfpathlineto{\pgfqpoint{2.360680in}{0.747987in}}%
\pgfpathlineto{\pgfqpoint{2.362211in}{0.721854in}}%
\pgfpathlineto{\pgfqpoint{2.362594in}{0.721854in}}%
\pgfpathlineto{\pgfqpoint{2.362977in}{0.695722in}}%
\pgfpathlineto{\pgfqpoint{2.363743in}{0.747987in}}%
\pgfpathlineto{\pgfqpoint{2.364125in}{0.747987in}}%
\pgfpathlineto{\pgfqpoint{2.364508in}{0.747987in}}%
\pgfpathlineto{\pgfqpoint{2.365657in}{0.695722in}}%
\pgfpathlineto{\pgfqpoint{2.366040in}{0.727081in}}%
\pgfpathlineto{\pgfqpoint{2.366422in}{0.727081in}}%
\pgfpathlineto{\pgfqpoint{2.367571in}{0.690496in}}%
\pgfpathlineto{\pgfqpoint{2.367954in}{0.700949in}}%
\pgfpathlineto{\pgfqpoint{2.368337in}{0.700949in}}%
\pgfpathlineto{\pgfqpoint{2.369485in}{0.664363in}}%
\pgfpathlineto{\pgfqpoint{2.369868in}{0.758440in}}%
\pgfpathlineto{\pgfqpoint{2.370251in}{0.758440in}}%
\pgfpathlineto{\pgfqpoint{2.370633in}{0.659137in}}%
\pgfpathlineto{\pgfqpoint{2.371782in}{0.753213in}}%
\pgfpathlineto{\pgfqpoint{2.372165in}{0.753213in}}%
\pgfpathlineto{\pgfqpoint{2.372548in}{0.695722in}}%
\pgfpathlineto{\pgfqpoint{2.373696in}{0.695722in}}%
\pgfpathlineto{\pgfqpoint{2.374462in}{0.695722in}}%
\pgfpathlineto{\pgfqpoint{2.374845in}{0.706175in}}%
\pgfpathlineto{\pgfqpoint{2.375993in}{0.706175in}}%
\pgfpathlineto{\pgfqpoint{2.376376in}{0.706175in}}%
\pgfpathlineto{\pgfqpoint{2.376759in}{0.721854in}}%
\pgfpathlineto{\pgfqpoint{2.377907in}{0.680043in}}%
\pgfpathlineto{\pgfqpoint{2.378290in}{0.680043in}}%
\pgfpathlineto{\pgfqpoint{2.378290in}{0.664363in}}%
\pgfpathlineto{\pgfqpoint{2.378673in}{0.727081in}}%
\pgfpathlineto{\pgfqpoint{2.379821in}{0.674816in}}%
\pgfpathlineto{\pgfqpoint{2.380204in}{0.674816in}}%
\pgfpathlineto{\pgfqpoint{2.380204in}{0.706175in}}%
\pgfpathlineto{\pgfqpoint{2.381736in}{0.685269in}}%
\pgfpathlineto{\pgfqpoint{2.382118in}{0.685269in}}%
\pgfpathlineto{\pgfqpoint{2.382501in}{0.753213in}}%
\pgfpathlineto{\pgfqpoint{2.382884in}{0.653910in}}%
\pgfpathlineto{\pgfqpoint{2.383650in}{0.695722in}}%
\pgfpathlineto{\pgfqpoint{2.384033in}{0.695722in}}%
\pgfpathlineto{\pgfqpoint{2.384798in}{0.721854in}}%
\pgfpathlineto{\pgfqpoint{2.385564in}{0.669590in}}%
\pgfpathlineto{\pgfqpoint{2.385947in}{0.669590in}}%
\pgfpathlineto{\pgfqpoint{2.385947in}{0.716628in}}%
\pgfpathlineto{\pgfqpoint{2.387478in}{0.680043in}}%
\pgfpathlineto{\pgfqpoint{2.387861in}{0.680043in}}%
\pgfpathlineto{\pgfqpoint{2.387861in}{0.664363in}}%
\pgfpathlineto{\pgfqpoint{2.388244in}{0.690496in}}%
\pgfpathlineto{\pgfqpoint{2.389392in}{0.674816in}}%
\pgfpathlineto{\pgfqpoint{2.389775in}{0.674816in}}%
\pgfpathlineto{\pgfqpoint{2.390923in}{0.716628in}}%
\pgfpathlineto{\pgfqpoint{2.391306in}{0.700949in}}%
\pgfpathlineto{\pgfqpoint{2.391689in}{0.700949in}}%
\pgfpathlineto{\pgfqpoint{2.391689in}{0.685269in}}%
\pgfpathlineto{\pgfqpoint{2.392838in}{0.706175in}}%
\pgfpathlineto{\pgfqpoint{2.393220in}{0.690496in}}%
\pgfpathlineto{\pgfqpoint{2.393603in}{0.690496in}}%
\pgfpathlineto{\pgfqpoint{2.393986in}{0.674816in}}%
\pgfpathlineto{\pgfqpoint{2.395135in}{0.732307in}}%
\pgfpathlineto{\pgfqpoint{2.395517in}{0.732307in}}%
\pgfpathlineto{\pgfqpoint{2.396666in}{0.653910in}}%
\pgfpathlineto{\pgfqpoint{2.397049in}{0.742760in}}%
\pgfpathlineto{\pgfqpoint{2.397432in}{0.742760in}}%
\pgfpathlineto{\pgfqpoint{2.398963in}{0.680043in}}%
\pgfpathlineto{\pgfqpoint{2.399346in}{0.680043in}}%
\pgfpathlineto{\pgfqpoint{2.400877in}{0.732307in}}%
\pgfpathlineto{\pgfqpoint{2.401260in}{0.732307in}}%
\pgfpathlineto{\pgfqpoint{2.402408in}{0.664363in}}%
\pgfpathlineto{\pgfqpoint{2.402791in}{0.711401in}}%
\pgfpathlineto{\pgfqpoint{2.403174in}{0.711401in}}%
\pgfpathlineto{\pgfqpoint{2.403557in}{0.674816in}}%
\pgfpathlineto{\pgfqpoint{2.404323in}{0.721854in}}%
\pgfpathlineto{\pgfqpoint{2.404705in}{0.685269in}}%
\pgfpathlineto{\pgfqpoint{2.405088in}{0.685269in}}%
\pgfpathlineto{\pgfqpoint{2.405854in}{0.742760in}}%
\pgfpathlineto{\pgfqpoint{2.405471in}{0.669590in}}%
\pgfpathlineto{\pgfqpoint{2.406620in}{0.695722in}}%
\pgfpathlineto{\pgfqpoint{2.407002in}{0.695722in}}%
\pgfpathlineto{\pgfqpoint{2.407002in}{0.669590in}}%
\pgfpathlineto{\pgfqpoint{2.407768in}{0.737534in}}%
\pgfpathlineto{\pgfqpoint{2.408534in}{0.711401in}}%
\pgfpathlineto{\pgfqpoint{2.408916in}{0.711401in}}%
\pgfpathlineto{\pgfqpoint{2.409299in}{0.669590in}}%
\pgfpathlineto{\pgfqpoint{2.409682in}{0.716628in}}%
\pgfpathlineto{\pgfqpoint{2.410448in}{0.706175in}}%
\pgfpathlineto{\pgfqpoint{2.410831in}{0.706175in}}%
\pgfpathlineto{\pgfqpoint{2.410831in}{0.721854in}}%
\pgfpathlineto{\pgfqpoint{2.411979in}{0.680043in}}%
\pgfpathlineto{\pgfqpoint{2.412362in}{0.690496in}}%
\pgfpathlineto{\pgfqpoint{2.412745in}{0.690496in}}%
\pgfpathlineto{\pgfqpoint{2.413128in}{0.727081in}}%
\pgfpathlineto{\pgfqpoint{2.414276in}{0.727081in}}%
\pgfpathlineto{\pgfqpoint{2.414659in}{0.727081in}}%
\pgfpathlineto{\pgfqpoint{2.415425in}{0.669590in}}%
\pgfpathlineto{\pgfqpoint{2.416190in}{0.711401in}}%
\pgfpathlineto{\pgfqpoint{2.416573in}{0.711401in}}%
\pgfpathlineto{\pgfqpoint{2.417722in}{0.659137in}}%
\pgfpathlineto{\pgfqpoint{2.418104in}{0.700949in}}%
\pgfpathlineto{\pgfqpoint{2.419253in}{0.700949in}}%
\pgfpathlineto{\pgfqpoint{2.419253in}{0.674816in}}%
\pgfpathlineto{\pgfqpoint{2.420784in}{0.700949in}}%
\pgfpathlineto{\pgfqpoint{2.421167in}{0.700949in}}%
\pgfpathlineto{\pgfqpoint{2.422316in}{0.664363in}}%
\pgfpathlineto{\pgfqpoint{2.422698in}{0.690496in}}%
\pgfpathlineto{\pgfqpoint{2.423081in}{0.690496in}}%
\pgfpathlineto{\pgfqpoint{2.424230in}{0.716628in}}%
\pgfpathlineto{\pgfqpoint{2.423847in}{0.685269in}}%
\pgfpathlineto{\pgfqpoint{2.424613in}{0.706175in}}%
\pgfpathlineto{\pgfqpoint{2.425378in}{0.706175in}}%
\pgfpathlineto{\pgfqpoint{2.426909in}{0.690496in}}%
\pgfpathlineto{\pgfqpoint{2.427292in}{0.690496in}}%
\pgfpathlineto{\pgfqpoint{2.428441in}{0.648684in}}%
\pgfpathlineto{\pgfqpoint{2.427675in}{0.711401in}}%
\pgfpathlineto{\pgfqpoint{2.428824in}{0.706175in}}%
\pgfpathlineto{\pgfqpoint{2.429206in}{0.706175in}}%
\pgfpathlineto{\pgfqpoint{2.429589in}{0.664363in}}%
\pgfpathlineto{\pgfqpoint{2.430738in}{0.674816in}}%
\pgfpathlineto{\pgfqpoint{2.431121in}{0.674816in}}%
\pgfpathlineto{\pgfqpoint{2.431886in}{0.706175in}}%
\pgfpathlineto{\pgfqpoint{2.432652in}{0.695722in}}%
\pgfpathlineto{\pgfqpoint{2.433035in}{0.695722in}}%
\pgfpathlineto{\pgfqpoint{2.433418in}{0.659137in}}%
\pgfpathlineto{\pgfqpoint{2.434183in}{0.700949in}}%
\pgfpathlineto{\pgfqpoint{2.434566in}{0.690496in}}%
\pgfpathlineto{\pgfqpoint{2.434949in}{0.690496in}}%
\pgfpathlineto{\pgfqpoint{2.436097in}{0.659137in}}%
\pgfpathlineto{\pgfqpoint{2.435715in}{0.700949in}}%
\pgfpathlineto{\pgfqpoint{2.436480in}{0.690496in}}%
\pgfpathlineto{\pgfqpoint{2.437246in}{0.690496in}}%
\pgfpathlineto{\pgfqpoint{2.437246in}{0.685269in}}%
\pgfpathlineto{\pgfqpoint{2.438012in}{0.711401in}}%
\pgfpathlineto{\pgfqpoint{2.438777in}{0.695722in}}%
\pgfpathlineto{\pgfqpoint{2.439160in}{0.695722in}}%
\pgfpathlineto{\pgfqpoint{2.439543in}{0.659137in}}%
\pgfpathlineto{\pgfqpoint{2.440691in}{0.674816in}}%
\pgfpathlineto{\pgfqpoint{2.441074in}{0.674816in}}%
\pgfpathlineto{\pgfqpoint{2.441074in}{0.706175in}}%
\pgfpathlineto{\pgfqpoint{2.441457in}{0.664363in}}%
\pgfpathlineto{\pgfqpoint{2.442606in}{0.669590in}}%
\pgfpathlineto{\pgfqpoint{2.442988in}{0.669590in}}%
\pgfpathlineto{\pgfqpoint{2.442988in}{0.659137in}}%
\pgfpathlineto{\pgfqpoint{2.444520in}{0.690496in}}%
\pgfpathlineto{\pgfqpoint{2.444903in}{0.690496in}}%
\pgfpathlineto{\pgfqpoint{2.446051in}{0.659137in}}%
\pgfpathlineto{\pgfqpoint{2.446434in}{0.706175in}}%
\pgfpathlineto{\pgfqpoint{2.446817in}{0.706175in}}%
\pgfpathlineto{\pgfqpoint{2.447199in}{0.669590in}}%
\pgfpathlineto{\pgfqpoint{2.448348in}{0.685269in}}%
\pgfpathlineto{\pgfqpoint{2.448731in}{0.685269in}}%
\pgfpathlineto{\pgfqpoint{2.448731in}{0.700949in}}%
\pgfpathlineto{\pgfqpoint{2.449496in}{0.648684in}}%
\pgfpathlineto{\pgfqpoint{2.450262in}{0.674816in}}%
\pgfpathlineto{\pgfqpoint{2.450645in}{0.674816in}}%
\pgfpathlineto{\pgfqpoint{2.452176in}{0.695722in}}%
\pgfpathlineto{\pgfqpoint{2.452559in}{0.695722in}}%
\pgfpathlineto{\pgfqpoint{2.452942in}{0.659137in}}%
\pgfpathlineto{\pgfqpoint{2.453708in}{0.711401in}}%
\pgfpathlineto{\pgfqpoint{2.454090in}{0.700949in}}%
\pgfpathlineto{\pgfqpoint{2.454473in}{0.700949in}}%
\pgfpathlineto{\pgfqpoint{2.454856in}{0.664363in}}%
\pgfpathlineto{\pgfqpoint{2.456005in}{0.685269in}}%
\pgfpathlineto{\pgfqpoint{2.456387in}{0.685269in}}%
\pgfpathlineto{\pgfqpoint{2.456387in}{0.732307in}}%
\pgfpathlineto{\pgfqpoint{2.456770in}{0.669590in}}%
\pgfpathlineto{\pgfqpoint{2.457919in}{0.669590in}}%
\pgfpathlineto{\pgfqpoint{2.458302in}{0.669590in}}%
\pgfpathlineto{\pgfqpoint{2.458302in}{0.700949in}}%
\pgfpathlineto{\pgfqpoint{2.459833in}{0.680043in}}%
\pgfpathlineto{\pgfqpoint{2.460216in}{0.680043in}}%
\pgfpathlineto{\pgfqpoint{2.460216in}{0.664363in}}%
\pgfpathlineto{\pgfqpoint{2.461747in}{0.674816in}}%
\pgfpathlineto{\pgfqpoint{2.462513in}{0.674816in}}%
\pgfpathlineto{\pgfqpoint{2.462513in}{0.716628in}}%
\pgfpathlineto{\pgfqpoint{2.463661in}{0.659137in}}%
\pgfpathlineto{\pgfqpoint{2.464044in}{0.685269in}}%
\pgfpathlineto{\pgfqpoint{2.464427in}{0.685269in}}%
\pgfpathlineto{\pgfqpoint{2.464427in}{0.669590in}}%
\pgfpathlineto{\pgfqpoint{2.465193in}{0.711401in}}%
\pgfpathlineto{\pgfqpoint{2.465958in}{0.685269in}}%
\pgfpathlineto{\pgfqpoint{2.466341in}{0.685269in}}%
\pgfpathlineto{\pgfqpoint{2.467489in}{0.669590in}}%
\pgfpathlineto{\pgfqpoint{2.466724in}{0.695722in}}%
\pgfpathlineto{\pgfqpoint{2.467872in}{0.685269in}}%
\pgfpathlineto{\pgfqpoint{2.468255in}{0.685269in}}%
\pgfpathlineto{\pgfqpoint{2.468255in}{0.706175in}}%
\pgfpathlineto{\pgfqpoint{2.469021in}{0.674816in}}%
\pgfpathlineto{\pgfqpoint{2.469786in}{0.680043in}}%
\pgfpathlineto{\pgfqpoint{2.470169in}{0.680043in}}%
\pgfpathlineto{\pgfqpoint{2.471318in}{0.716628in}}%
\pgfpathlineto{\pgfqpoint{2.470552in}{0.674816in}}%
\pgfpathlineto{\pgfqpoint{2.471701in}{0.680043in}}%
\pgfpathlineto{\pgfqpoint{2.472083in}{0.680043in}}%
\pgfpathlineto{\pgfqpoint{2.472849in}{0.695722in}}%
\pgfpathlineto{\pgfqpoint{2.472466in}{0.643457in}}%
\pgfpathlineto{\pgfqpoint{2.473615in}{0.669590in}}%
\pgfpathlineto{\pgfqpoint{2.473998in}{0.669590in}}%
\pgfpathlineto{\pgfqpoint{2.473998in}{0.674816in}}%
\pgfpathlineto{\pgfqpoint{2.474380in}{0.653910in}}%
\pgfpathlineto{\pgfqpoint{2.475529in}{0.659137in}}%
\pgfpathlineto{\pgfqpoint{2.475912in}{0.659137in}}%
\pgfpathlineto{\pgfqpoint{2.475912in}{0.680043in}}%
\pgfpathlineto{\pgfqpoint{2.477443in}{0.669590in}}%
\pgfpathlineto{\pgfqpoint{2.477826in}{0.669590in}}%
\pgfpathlineto{\pgfqpoint{2.477826in}{0.659137in}}%
\pgfpathlineto{\pgfqpoint{2.478209in}{0.711401in}}%
\pgfpathlineto{\pgfqpoint{2.479357in}{0.695722in}}%
\pgfpathlineto{\pgfqpoint{2.479740in}{0.695722in}}%
\pgfpathlineto{\pgfqpoint{2.480123in}{0.706175in}}%
\pgfpathlineto{\pgfqpoint{2.481271in}{0.664363in}}%
\pgfpathlineto{\pgfqpoint{2.482037in}{0.664363in}}%
\pgfpathlineto{\pgfqpoint{2.482420in}{0.700949in}}%
\pgfpathlineto{\pgfqpoint{2.483568in}{0.674816in}}%
\pgfpathlineto{\pgfqpoint{2.483951in}{0.674816in}}%
\pgfpathlineto{\pgfqpoint{2.483951in}{0.669590in}}%
\pgfpathlineto{\pgfqpoint{2.484334in}{0.690496in}}%
\pgfpathlineto{\pgfqpoint{2.485482in}{0.685269in}}%
\pgfpathlineto{\pgfqpoint{2.485865in}{0.685269in}}%
\pgfpathlineto{\pgfqpoint{2.485865in}{0.659137in}}%
\pgfpathlineto{\pgfqpoint{2.487397in}{0.685269in}}%
\pgfpathlineto{\pgfqpoint{2.487779in}{0.685269in}}%
\pgfpathlineto{\pgfqpoint{2.488928in}{0.653910in}}%
\pgfpathlineto{\pgfqpoint{2.488162in}{0.700949in}}%
\pgfpathlineto{\pgfqpoint{2.489311in}{0.659137in}}%
\pgfpathlineto{\pgfqpoint{2.489694in}{0.659137in}}%
\pgfpathlineto{\pgfqpoint{2.490459in}{0.685269in}}%
\pgfpathlineto{\pgfqpoint{2.490076in}{0.653910in}}%
\pgfpathlineto{\pgfqpoint{2.491225in}{0.685269in}}%
\pgfpathlineto{\pgfqpoint{2.491991in}{0.685269in}}%
\pgfpathlineto{\pgfqpoint{2.491991in}{0.690496in}}%
\pgfpathlineto{\pgfqpoint{2.493139in}{0.674816in}}%
\pgfpathlineto{\pgfqpoint{2.493522in}{0.685269in}}%
\pgfpathlineto{\pgfqpoint{2.493905in}{0.685269in}}%
\pgfpathlineto{\pgfqpoint{2.495053in}{0.664363in}}%
\pgfpathlineto{\pgfqpoint{2.495436in}{0.680043in}}%
\pgfpathlineto{\pgfqpoint{2.495819in}{0.680043in}}%
\pgfpathlineto{\pgfqpoint{2.496202in}{0.669590in}}%
\pgfpathlineto{\pgfqpoint{2.496585in}{0.716628in}}%
\pgfpathlineto{\pgfqpoint{2.497350in}{0.674816in}}%
\pgfpathlineto{\pgfqpoint{2.497733in}{0.674816in}}%
\pgfpathlineto{\pgfqpoint{2.497733in}{0.643457in}}%
\pgfpathlineto{\pgfqpoint{2.498116in}{0.711401in}}%
\pgfpathlineto{\pgfqpoint{2.499264in}{0.669590in}}%
\pgfpathlineto{\pgfqpoint{2.499647in}{0.669590in}}%
\pgfpathlineto{\pgfqpoint{2.500796in}{0.690496in}}%
\pgfpathlineto{\pgfqpoint{2.500413in}{0.638231in}}%
\pgfpathlineto{\pgfqpoint{2.501179in}{0.674816in}}%
\pgfpathlineto{\pgfqpoint{2.501561in}{0.674816in}}%
\pgfpathlineto{\pgfqpoint{2.502710in}{0.700949in}}%
\pgfpathlineto{\pgfqpoint{2.501944in}{0.664363in}}%
\pgfpathlineto{\pgfqpoint{2.503093in}{0.674816in}}%
\pgfpathlineto{\pgfqpoint{2.503858in}{0.674816in}}%
\pgfpathlineto{\pgfqpoint{2.503858in}{0.700949in}}%
\pgfpathlineto{\pgfqpoint{2.505007in}{0.653910in}}%
\pgfpathlineto{\pgfqpoint{2.505390in}{0.659137in}}%
\pgfpathlineto{\pgfqpoint{2.505772in}{0.659137in}}%
\pgfpathlineto{\pgfqpoint{2.506538in}{0.690496in}}%
\pgfpathlineto{\pgfqpoint{2.507304in}{0.674816in}}%
\pgfpathlineto{\pgfqpoint{2.507687in}{0.674816in}}%
\pgfpathlineto{\pgfqpoint{2.507687in}{0.664363in}}%
\pgfpathlineto{\pgfqpoint{2.509218in}{0.680043in}}%
\pgfpathlineto{\pgfqpoint{2.509984in}{0.680043in}}%
\pgfpathlineto{\pgfqpoint{2.509984in}{0.690496in}}%
\pgfpathlineto{\pgfqpoint{2.511515in}{0.669590in}}%
\pgfpathlineto{\pgfqpoint{2.511898in}{0.669590in}}%
\pgfpathlineto{\pgfqpoint{2.511898in}{0.695722in}}%
\pgfpathlineto{\pgfqpoint{2.512281in}{0.653910in}}%
\pgfpathlineto{\pgfqpoint{2.513429in}{0.685269in}}%
\pgfpathlineto{\pgfqpoint{2.514195in}{0.685269in}}%
\pgfpathlineto{\pgfqpoint{2.515343in}{0.659137in}}%
\pgfpathlineto{\pgfqpoint{2.515726in}{0.706175in}}%
\pgfpathlineto{\pgfqpoint{2.516109in}{0.706175in}}%
\pgfpathlineto{\pgfqpoint{2.516109in}{0.653910in}}%
\pgfpathlineto{\pgfqpoint{2.517640in}{0.690496in}}%
\pgfpathlineto{\pgfqpoint{2.518023in}{0.690496in}}%
\pgfpathlineto{\pgfqpoint{2.518023in}{0.659137in}}%
\pgfpathlineto{\pgfqpoint{2.519554in}{0.680043in}}%
\pgfpathlineto{\pgfqpoint{2.520320in}{0.680043in}}%
\pgfpathlineto{\pgfqpoint{2.520320in}{0.664363in}}%
\pgfpathlineto{\pgfqpoint{2.520703in}{0.690496in}}%
\pgfpathlineto{\pgfqpoint{2.521851in}{0.674816in}}%
\pgfpathlineto{\pgfqpoint{2.522234in}{0.674816in}}%
\pgfpathlineto{\pgfqpoint{2.522234in}{0.685269in}}%
\pgfpathlineto{\pgfqpoint{2.523383in}{0.653910in}}%
\pgfpathlineto{\pgfqpoint{2.523765in}{0.674816in}}%
\pgfpathlineto{\pgfqpoint{2.524148in}{0.674816in}}%
\pgfpathlineto{\pgfqpoint{2.524914in}{0.653910in}}%
\pgfpathlineto{\pgfqpoint{2.524531in}{0.680043in}}%
\pgfpathlineto{\pgfqpoint{2.525680in}{0.664363in}}%
\pgfpathlineto{\pgfqpoint{2.526062in}{0.664363in}}%
\pgfpathlineto{\pgfqpoint{2.527211in}{0.695722in}}%
\pgfpathlineto{\pgfqpoint{2.526445in}{0.659137in}}%
\pgfpathlineto{\pgfqpoint{2.527594in}{0.685269in}}%
\pgfpathlineto{\pgfqpoint{2.528742in}{0.685269in}}%
\pgfpathlineto{\pgfqpoint{2.529125in}{0.659137in}}%
\pgfpathlineto{\pgfqpoint{2.529508in}{0.690496in}}%
\pgfpathlineto{\pgfqpoint{2.530274in}{0.664363in}}%
\pgfpathlineto{\pgfqpoint{2.530656in}{0.664363in}}%
\pgfpathlineto{\pgfqpoint{2.531039in}{0.653910in}}%
\pgfpathlineto{\pgfqpoint{2.532188in}{0.659137in}}%
\pgfpathlineto{\pgfqpoint{2.532571in}{0.659137in}}%
\pgfpathlineto{\pgfqpoint{2.532571in}{0.700949in}}%
\pgfpathlineto{\pgfqpoint{2.532953in}{0.648684in}}%
\pgfpathlineto{\pgfqpoint{2.534102in}{0.669590in}}%
\pgfpathlineto{\pgfqpoint{2.534485in}{0.669590in}}%
\pgfpathlineto{\pgfqpoint{2.535633in}{0.680043in}}%
\pgfpathlineto{\pgfqpoint{2.536016in}{0.648684in}}%
\pgfpathlineto{\pgfqpoint{2.536399in}{0.648684in}}%
\pgfpathlineto{\pgfqpoint{2.537930in}{0.685269in}}%
\pgfpathlineto{\pgfqpoint{2.538313in}{0.685269in}}%
\pgfpathlineto{\pgfqpoint{2.538313in}{0.690496in}}%
\pgfpathlineto{\pgfqpoint{2.538696in}{0.648684in}}%
\pgfpathlineto{\pgfqpoint{2.539844in}{0.685269in}}%
\pgfpathlineto{\pgfqpoint{2.540227in}{0.685269in}}%
\pgfpathlineto{\pgfqpoint{2.540227in}{0.638231in}}%
\pgfpathlineto{\pgfqpoint{2.541759in}{0.669590in}}%
\pgfpathlineto{\pgfqpoint{2.542524in}{0.669590in}}%
\pgfpathlineto{\pgfqpoint{2.543290in}{0.648684in}}%
\pgfpathlineto{\pgfqpoint{2.543673in}{0.690496in}}%
\pgfpathlineto{\pgfqpoint{2.544055in}{0.674816in}}%
\pgfpathlineto{\pgfqpoint{2.544438in}{0.674816in}}%
\pgfpathlineto{\pgfqpoint{2.544438in}{0.659137in}}%
\pgfpathlineto{\pgfqpoint{2.545970in}{0.669590in}}%
\pgfpathlineto{\pgfqpoint{2.546735in}{0.669590in}}%
\pgfpathlineto{\pgfqpoint{2.546735in}{0.653910in}}%
\pgfpathlineto{\pgfqpoint{2.547501in}{0.680043in}}%
\pgfpathlineto{\pgfqpoint{2.548267in}{0.674816in}}%
\pgfpathlineto{\pgfqpoint{2.548649in}{0.674816in}}%
\pgfpathlineto{\pgfqpoint{2.549032in}{0.664363in}}%
\pgfpathlineto{\pgfqpoint{2.550181in}{0.685269in}}%
\pgfpathlineto{\pgfqpoint{2.550564in}{0.685269in}}%
\pgfpathlineto{\pgfqpoint{2.550946in}{0.659137in}}%
\pgfpathlineto{\pgfqpoint{2.552095in}{0.680043in}}%
\pgfpathlineto{\pgfqpoint{2.552478in}{0.680043in}}%
\pgfpathlineto{\pgfqpoint{2.552478in}{0.659137in}}%
\pgfpathlineto{\pgfqpoint{2.552861in}{0.690496in}}%
\pgfpathlineto{\pgfqpoint{2.554009in}{0.674816in}}%
\pgfpathlineto{\pgfqpoint{2.554392in}{0.674816in}}%
\pgfpathlineto{\pgfqpoint{2.555158in}{0.648684in}}%
\pgfpathlineto{\pgfqpoint{2.555923in}{0.695722in}}%
\pgfpathlineto{\pgfqpoint{2.556306in}{0.695722in}}%
\pgfpathlineto{\pgfqpoint{2.557455in}{0.643457in}}%
\pgfpathlineto{\pgfqpoint{2.557837in}{0.669590in}}%
\pgfpathlineto{\pgfqpoint{2.558220in}{0.669590in}}%
\pgfpathlineto{\pgfqpoint{2.558603in}{0.706175in}}%
\pgfpathlineto{\pgfqpoint{2.559369in}{0.653910in}}%
\pgfpathlineto{\pgfqpoint{2.559752in}{0.680043in}}%
\pgfpathlineto{\pgfqpoint{2.560134in}{0.680043in}}%
\pgfpathlineto{\pgfqpoint{2.560134in}{0.659137in}}%
\pgfpathlineto{\pgfqpoint{2.561666in}{0.685269in}}%
\pgfpathlineto{\pgfqpoint{2.562431in}{0.685269in}}%
\pgfpathlineto{\pgfqpoint{2.563963in}{0.653910in}}%
\pgfpathlineto{\pgfqpoint{2.564345in}{0.653910in}}%
\pgfpathlineto{\pgfqpoint{2.565494in}{0.711401in}}%
\pgfpathlineto{\pgfqpoint{2.565877in}{0.685269in}}%
\pgfpathlineto{\pgfqpoint{2.566260in}{0.685269in}}%
\pgfpathlineto{\pgfqpoint{2.567408in}{0.648684in}}%
\pgfpathlineto{\pgfqpoint{2.567791in}{0.706175in}}%
\pgfpathlineto{\pgfqpoint{2.568174in}{0.706175in}}%
\pgfpathlineto{\pgfqpoint{2.569322in}{0.648684in}}%
\pgfpathlineto{\pgfqpoint{2.569705in}{0.653910in}}%
\pgfpathlineto{\pgfqpoint{2.570088in}{0.653910in}}%
\pgfpathlineto{\pgfqpoint{2.570471in}{0.674816in}}%
\pgfpathlineto{\pgfqpoint{2.571619in}{0.653910in}}%
\pgfpathlineto{\pgfqpoint{2.572002in}{0.653910in}}%
\pgfpathlineto{\pgfqpoint{2.572385in}{0.700949in}}%
\pgfpathlineto{\pgfqpoint{2.573533in}{0.643457in}}%
\pgfpathlineto{\pgfqpoint{2.573916in}{0.643457in}}%
\pgfpathlineto{\pgfqpoint{2.574299in}{0.685269in}}%
\pgfpathlineto{\pgfqpoint{2.575448in}{0.664363in}}%
\pgfpathlineto{\pgfqpoint{2.575830in}{0.664363in}}%
\pgfpathlineto{\pgfqpoint{2.576213in}{0.700949in}}%
\pgfpathlineto{\pgfqpoint{2.576979in}{0.643457in}}%
\pgfpathlineto{\pgfqpoint{2.577362in}{0.648684in}}%
\pgfpathlineto{\pgfqpoint{2.577745in}{0.648684in}}%
\pgfpathlineto{\pgfqpoint{2.578127in}{0.669590in}}%
\pgfpathlineto{\pgfqpoint{2.579276in}{0.669590in}}%
\pgfpathlineto{\pgfqpoint{2.579659in}{0.669590in}}%
\pgfpathlineto{\pgfqpoint{2.579659in}{0.685269in}}%
\pgfpathlineto{\pgfqpoint{2.580807in}{0.653910in}}%
\pgfpathlineto{\pgfqpoint{2.581190in}{0.659137in}}%
\pgfpathlineto{\pgfqpoint{2.581573in}{0.659137in}}%
\pgfpathlineto{\pgfqpoint{2.581956in}{0.695722in}}%
\pgfpathlineto{\pgfqpoint{2.581956in}{0.643457in}}%
\pgfpathlineto{\pgfqpoint{2.583104in}{0.653910in}}%
\pgfpathlineto{\pgfqpoint{2.583487in}{0.653910in}}%
\pgfpathlineto{\pgfqpoint{2.585018in}{0.700949in}}%
\pgfpathlineto{\pgfqpoint{2.585401in}{0.700949in}}%
\pgfpathlineto{\pgfqpoint{2.585784in}{0.664363in}}%
\pgfpathlineto{\pgfqpoint{2.586932in}{0.674816in}}%
\pgfpathlineto{\pgfqpoint{2.587315in}{0.674816in}}%
\pgfpathlineto{\pgfqpoint{2.587315in}{0.648684in}}%
\pgfpathlineto{\pgfqpoint{2.588847in}{0.659137in}}%
\pgfpathlineto{\pgfqpoint{2.589229in}{0.659137in}}%
\pgfpathlineto{\pgfqpoint{2.589229in}{0.669590in}}%
\pgfpathlineto{\pgfqpoint{2.589995in}{0.638231in}}%
\pgfpathlineto{\pgfqpoint{2.590761in}{0.669590in}}%
\pgfpathlineto{\pgfqpoint{2.591144in}{0.669590in}}%
\pgfpathlineto{\pgfqpoint{2.591909in}{0.643457in}}%
\pgfpathlineto{\pgfqpoint{2.592675in}{0.685269in}}%
\pgfpathlineto{\pgfqpoint{2.593058in}{0.685269in}}%
\pgfpathlineto{\pgfqpoint{2.593058in}{0.638231in}}%
\pgfpathlineto{\pgfqpoint{2.594589in}{0.680043in}}%
\pgfpathlineto{\pgfqpoint{2.594972in}{0.680043in}}%
\pgfpathlineto{\pgfqpoint{2.595738in}{0.648684in}}%
\pgfpathlineto{\pgfqpoint{2.596503in}{0.648684in}}%
\pgfpathlineto{\pgfqpoint{2.597269in}{0.648684in}}%
\pgfpathlineto{\pgfqpoint{2.598035in}{0.669590in}}%
\pgfpathlineto{\pgfqpoint{2.598800in}{0.638231in}}%
\pgfpathlineto{\pgfqpoint{2.599183in}{0.638231in}}%
\pgfpathlineto{\pgfqpoint{2.599566in}{0.680043in}}%
\pgfpathlineto{\pgfqpoint{2.600714in}{0.680043in}}%
\pgfpathlineto{\pgfqpoint{2.601097in}{0.680043in}}%
\pgfpathlineto{\pgfqpoint{2.601863in}{0.648684in}}%
\pgfpathlineto{\pgfqpoint{2.602628in}{0.648684in}}%
\pgfpathlineto{\pgfqpoint{2.603011in}{0.648684in}}%
\pgfpathlineto{\pgfqpoint{2.604160in}{0.680043in}}%
\pgfpathlineto{\pgfqpoint{2.604543in}{0.659137in}}%
\pgfpathlineto{\pgfqpoint{2.604925in}{0.659137in}}%
\pgfpathlineto{\pgfqpoint{2.605691in}{0.690496in}}%
\pgfpathlineto{\pgfqpoint{2.606074in}{0.638231in}}%
\pgfpathlineto{\pgfqpoint{2.606457in}{0.659137in}}%
\pgfpathlineto{\pgfqpoint{2.606840in}{0.659137in}}%
\pgfpathlineto{\pgfqpoint{2.607222in}{0.695722in}}%
\pgfpathlineto{\pgfqpoint{2.607988in}{0.648684in}}%
\pgfpathlineto{\pgfqpoint{2.608371in}{0.653910in}}%
\pgfpathlineto{\pgfqpoint{2.608754in}{0.653910in}}%
\pgfpathlineto{\pgfqpoint{2.608754in}{0.659137in}}%
\pgfpathlineto{\pgfqpoint{2.609902in}{0.643457in}}%
\pgfpathlineto{\pgfqpoint{2.610285in}{0.653910in}}%
\pgfpathlineto{\pgfqpoint{2.610668in}{0.653910in}}%
\pgfpathlineto{\pgfqpoint{2.610668in}{0.680043in}}%
\pgfpathlineto{\pgfqpoint{2.611816in}{0.648684in}}%
\pgfpathlineto{\pgfqpoint{2.612199in}{0.680043in}}%
\pgfpathlineto{\pgfqpoint{2.612582in}{0.680043in}}%
\pgfpathlineto{\pgfqpoint{2.613348in}{0.638231in}}%
\pgfpathlineto{\pgfqpoint{2.614113in}{0.648684in}}%
\pgfpathlineto{\pgfqpoint{2.614496in}{0.648684in}}%
\pgfpathlineto{\pgfqpoint{2.614496in}{0.690496in}}%
\pgfpathlineto{\pgfqpoint{2.616028in}{0.674816in}}%
\pgfpathlineto{\pgfqpoint{2.616410in}{0.674816in}}%
\pgfpathlineto{\pgfqpoint{2.617176in}{0.638231in}}%
\pgfpathlineto{\pgfqpoint{2.617942in}{0.648684in}}%
\pgfpathlineto{\pgfqpoint{2.618325in}{0.648684in}}%
\pgfpathlineto{\pgfqpoint{2.619473in}{0.685269in}}%
\pgfpathlineto{\pgfqpoint{2.619856in}{0.674816in}}%
\pgfpathlineto{\pgfqpoint{2.620239in}{0.674816in}}%
\pgfpathlineto{\pgfqpoint{2.621004in}{0.685269in}}%
\pgfpathlineto{\pgfqpoint{2.621770in}{0.643457in}}%
\pgfpathlineto{\pgfqpoint{2.622153in}{0.643457in}}%
\pgfpathlineto{\pgfqpoint{2.623301in}{0.680043in}}%
\pgfpathlineto{\pgfqpoint{2.623684in}{0.659137in}}%
\pgfpathlineto{\pgfqpoint{2.624067in}{0.659137in}}%
\pgfpathlineto{\pgfqpoint{2.624450in}{0.638231in}}%
\pgfpathlineto{\pgfqpoint{2.625598in}{0.664363in}}%
\pgfpathlineto{\pgfqpoint{2.625981in}{0.664363in}}%
\pgfpathlineto{\pgfqpoint{2.626364in}{0.633004in}}%
\pgfpathlineto{\pgfqpoint{2.627512in}{0.633004in}}%
\pgfpathlineto{\pgfqpoint{2.627895in}{0.633004in}}%
\pgfpathlineto{\pgfqpoint{2.628278in}{0.669590in}}%
\pgfpathlineto{\pgfqpoint{2.629427in}{0.653910in}}%
\pgfpathlineto{\pgfqpoint{2.629809in}{0.653910in}}%
\pgfpathlineto{\pgfqpoint{2.629809in}{0.680043in}}%
\pgfpathlineto{\pgfqpoint{2.631341in}{0.648684in}}%
\pgfpathlineto{\pgfqpoint{2.631724in}{0.648684in}}%
\pgfpathlineto{\pgfqpoint{2.633255in}{0.674816in}}%
\pgfpathlineto{\pgfqpoint{2.634021in}{0.674816in}}%
\pgfpathlineto{\pgfqpoint{2.635169in}{0.648684in}}%
\pgfpathlineto{\pgfqpoint{2.634403in}{0.690496in}}%
\pgfpathlineto{\pgfqpoint{2.635552in}{0.659137in}}%
\pgfpathlineto{\pgfqpoint{2.636318in}{0.659137in}}%
\pgfpathlineto{\pgfqpoint{2.636318in}{0.648684in}}%
\pgfpathlineto{\pgfqpoint{2.637849in}{0.680043in}}%
\pgfpathlineto{\pgfqpoint{2.638232in}{0.680043in}}%
\pgfpathlineto{\pgfqpoint{2.639380in}{0.648684in}}%
\pgfpathlineto{\pgfqpoint{2.639763in}{0.680043in}}%
\pgfpathlineto{\pgfqpoint{2.640146in}{0.680043in}}%
\pgfpathlineto{\pgfqpoint{2.640529in}{0.653910in}}%
\pgfpathlineto{\pgfqpoint{2.641677in}{0.669590in}}%
\pgfpathlineto{\pgfqpoint{2.642060in}{0.669590in}}%
\pgfpathlineto{\pgfqpoint{2.642443in}{0.643457in}}%
\pgfpathlineto{\pgfqpoint{2.643591in}{0.659137in}}%
\pgfpathlineto{\pgfqpoint{2.643974in}{0.659137in}}%
\pgfpathlineto{\pgfqpoint{2.643974in}{0.680043in}}%
\pgfpathlineto{\pgfqpoint{2.644740in}{0.643457in}}%
\pgfpathlineto{\pgfqpoint{2.645505in}{0.659137in}}%
\pgfpathlineto{\pgfqpoint{2.646654in}{0.659137in}}%
\pgfpathlineto{\pgfqpoint{2.646654in}{0.633004in}}%
\pgfpathlineto{\pgfqpoint{2.647420in}{0.680043in}}%
\pgfpathlineto{\pgfqpoint{2.648185in}{0.653910in}}%
\pgfpathlineto{\pgfqpoint{2.648568in}{0.653910in}}%
\pgfpathlineto{\pgfqpoint{2.648568in}{0.648684in}}%
\pgfpathlineto{\pgfqpoint{2.649717in}{0.690496in}}%
\pgfpathlineto{\pgfqpoint{2.650099in}{0.664363in}}%
\pgfpathlineto{\pgfqpoint{2.650482in}{0.664363in}}%
\pgfpathlineto{\pgfqpoint{2.651631in}{0.648684in}}%
\pgfpathlineto{\pgfqpoint{2.652014in}{0.680043in}}%
\pgfpathlineto{\pgfqpoint{2.652396in}{0.680043in}}%
\pgfpathlineto{\pgfqpoint{2.653928in}{0.648684in}}%
\pgfpathlineto{\pgfqpoint{2.654311in}{0.648684in}}%
\pgfpathlineto{\pgfqpoint{2.654311in}{0.680043in}}%
\pgfpathlineto{\pgfqpoint{2.655842in}{0.633004in}}%
\pgfpathlineto{\pgfqpoint{2.656225in}{0.633004in}}%
\pgfpathlineto{\pgfqpoint{2.657756in}{0.664363in}}%
\pgfpathlineto{\pgfqpoint{2.658139in}{0.664363in}}%
\pgfpathlineto{\pgfqpoint{2.659287in}{0.638231in}}%
\pgfpathlineto{\pgfqpoint{2.658904in}{0.669590in}}%
\pgfpathlineto{\pgfqpoint{2.659670in}{0.643457in}}%
\pgfpathlineto{\pgfqpoint{2.660053in}{0.643457in}}%
\pgfpathlineto{\pgfqpoint{2.660436in}{0.685269in}}%
\pgfpathlineto{\pgfqpoint{2.661584in}{0.680043in}}%
\pgfpathlineto{\pgfqpoint{2.661967in}{0.680043in}}%
\pgfpathlineto{\pgfqpoint{2.662733in}{0.643457in}}%
\pgfpathlineto{\pgfqpoint{2.663498in}{0.664363in}}%
\pgfpathlineto{\pgfqpoint{2.664264in}{0.664363in}}%
\pgfpathlineto{\pgfqpoint{2.664264in}{0.643457in}}%
\pgfpathlineto{\pgfqpoint{2.664647in}{0.680043in}}%
\pgfpathlineto{\pgfqpoint{2.665795in}{0.659137in}}%
\pgfpathlineto{\pgfqpoint{2.666178in}{0.659137in}}%
\pgfpathlineto{\pgfqpoint{2.666178in}{0.653910in}}%
\pgfpathlineto{\pgfqpoint{2.667710in}{0.674816in}}%
\pgfpathlineto{\pgfqpoint{2.668092in}{0.674816in}}%
\pgfpathlineto{\pgfqpoint{2.668092in}{0.648684in}}%
\pgfpathlineto{\pgfqpoint{2.669624in}{0.653910in}}%
\pgfpathlineto{\pgfqpoint{2.670389in}{0.653910in}}%
\pgfpathlineto{\pgfqpoint{2.671155in}{0.685269in}}%
\pgfpathlineto{\pgfqpoint{2.671538in}{0.633004in}}%
\pgfpathlineto{\pgfqpoint{2.671921in}{0.659137in}}%
\pgfpathlineto{\pgfqpoint{2.672304in}{0.659137in}}%
\pgfpathlineto{\pgfqpoint{2.672304in}{0.638231in}}%
\pgfpathlineto{\pgfqpoint{2.673835in}{0.695722in}}%
\pgfpathlineto{\pgfqpoint{2.674218in}{0.695722in}}%
\pgfpathlineto{\pgfqpoint{2.674601in}{0.633004in}}%
\pgfpathlineto{\pgfqpoint{2.675749in}{0.664363in}}%
\pgfpathlineto{\pgfqpoint{2.676515in}{0.664363in}}%
\pgfpathlineto{\pgfqpoint{2.677663in}{0.685269in}}%
\pgfpathlineto{\pgfqpoint{2.678046in}{0.643457in}}%
\pgfpathlineto{\pgfqpoint{2.678429in}{0.643457in}}%
\pgfpathlineto{\pgfqpoint{2.679577in}{0.695722in}}%
\pgfpathlineto{\pgfqpoint{2.679960in}{0.648684in}}%
\pgfpathlineto{\pgfqpoint{2.680343in}{0.648684in}}%
\pgfpathlineto{\pgfqpoint{2.680726in}{0.669590in}}%
\pgfpathlineto{\pgfqpoint{2.681874in}{0.659137in}}%
\pgfpathlineto{\pgfqpoint{2.682257in}{0.659137in}}%
\pgfpathlineto{\pgfqpoint{2.682257in}{0.648684in}}%
\pgfpathlineto{\pgfqpoint{2.682640in}{0.669590in}}%
\pgfpathlineto{\pgfqpoint{2.683788in}{0.653910in}}%
\pgfpathlineto{\pgfqpoint{2.684554in}{0.653910in}}%
\pgfpathlineto{\pgfqpoint{2.684554in}{0.685269in}}%
\pgfpathlineto{\pgfqpoint{2.685703in}{0.638231in}}%
\pgfpathlineto{\pgfqpoint{2.686085in}{0.669590in}}%
\pgfpathlineto{\pgfqpoint{2.686468in}{0.669590in}}%
\pgfpathlineto{\pgfqpoint{2.686851in}{0.648684in}}%
\pgfpathlineto{\pgfqpoint{2.687234in}{0.674816in}}%
\pgfpathlineto{\pgfqpoint{2.688000in}{0.653910in}}%
\pgfpathlineto{\pgfqpoint{2.688765in}{0.653910in}}%
\pgfpathlineto{\pgfqpoint{2.688765in}{0.638231in}}%
\pgfpathlineto{\pgfqpoint{2.689914in}{0.685269in}}%
\pgfpathlineto{\pgfqpoint{2.690297in}{0.653910in}}%
\pgfpathlineto{\pgfqpoint{2.691062in}{0.653910in}}%
\pgfpathlineto{\pgfqpoint{2.691062in}{0.685269in}}%
\pgfpathlineto{\pgfqpoint{2.691445in}{0.643457in}}%
\pgfpathlineto{\pgfqpoint{2.692594in}{0.653910in}}%
\pgfpathlineto{\pgfqpoint{2.692976in}{0.653910in}}%
\pgfpathlineto{\pgfqpoint{2.692976in}{0.659137in}}%
\pgfpathlineto{\pgfqpoint{2.694508in}{0.653910in}}%
\pgfpathlineto{\pgfqpoint{2.694891in}{0.653910in}}%
\pgfpathlineto{\pgfqpoint{2.695656in}{0.638231in}}%
\pgfpathlineto{\pgfqpoint{2.696422in}{0.664363in}}%
\pgfpathlineto{\pgfqpoint{2.696805in}{0.664363in}}%
\pgfpathlineto{\pgfqpoint{2.697570in}{0.638231in}}%
\pgfpathlineto{\pgfqpoint{2.698336in}{0.648684in}}%
\pgfpathlineto{\pgfqpoint{2.698719in}{0.648684in}}%
\pgfpathlineto{\pgfqpoint{2.699102in}{0.643457in}}%
\pgfpathlineto{\pgfqpoint{2.700250in}{0.669590in}}%
\pgfpathlineto{\pgfqpoint{2.700633in}{0.669590in}}%
\pgfpathlineto{\pgfqpoint{2.701016in}{0.638231in}}%
\pgfpathlineto{\pgfqpoint{2.702164in}{0.653910in}}%
\pgfpathlineto{\pgfqpoint{2.702547in}{0.653910in}}%
\pgfpathlineto{\pgfqpoint{2.703313in}{0.659137in}}%
\pgfpathlineto{\pgfqpoint{2.704078in}{0.643457in}}%
\pgfpathlineto{\pgfqpoint{2.704461in}{0.643457in}}%
\pgfpathlineto{\pgfqpoint{2.705227in}{0.669590in}}%
\pgfpathlineto{\pgfqpoint{2.705993in}{0.653910in}}%
\pgfpathlineto{\pgfqpoint{2.706375in}{0.653910in}}%
\pgfpathlineto{\pgfqpoint{2.706758in}{0.664363in}}%
\pgfpathlineto{\pgfqpoint{2.707141in}{0.648684in}}%
\pgfpathlineto{\pgfqpoint{2.707907in}{0.659137in}}%
\pgfpathlineto{\pgfqpoint{2.708290in}{0.659137in}}%
\pgfpathlineto{\pgfqpoint{2.708290in}{0.643457in}}%
\pgfpathlineto{\pgfqpoint{2.708672in}{0.674816in}}%
\pgfpathlineto{\pgfqpoint{2.709821in}{0.643457in}}%
\pgfpathlineto{\pgfqpoint{2.710204in}{0.643457in}}%
\pgfpathlineto{\pgfqpoint{2.710204in}{0.674816in}}%
\pgfpathlineto{\pgfqpoint{2.711735in}{0.648684in}}%
\pgfpathlineto{\pgfqpoint{2.712501in}{0.648684in}}%
\pgfpathlineto{\pgfqpoint{2.712501in}{0.643457in}}%
\pgfpathlineto{\pgfqpoint{2.713649in}{0.674816in}}%
\pgfpathlineto{\pgfqpoint{2.714032in}{0.653910in}}%
\pgfpathlineto{\pgfqpoint{2.714415in}{0.653910in}}%
\pgfpathlineto{\pgfqpoint{2.714415in}{0.643457in}}%
\pgfpathlineto{\pgfqpoint{2.715181in}{0.690496in}}%
\pgfpathlineto{\pgfqpoint{2.715946in}{0.659137in}}%
\pgfpathlineto{\pgfqpoint{2.716329in}{0.659137in}}%
\pgfpathlineto{\pgfqpoint{2.716329in}{0.680043in}}%
\pgfpathlineto{\pgfqpoint{2.717860in}{0.638231in}}%
\pgfpathlineto{\pgfqpoint{2.718243in}{0.638231in}}%
\pgfpathlineto{\pgfqpoint{2.719009in}{0.669590in}}%
\pgfpathlineto{\pgfqpoint{2.719774in}{0.653910in}}%
\pgfpathlineto{\pgfqpoint{2.720157in}{0.653910in}}%
\pgfpathlineto{\pgfqpoint{2.720157in}{0.669590in}}%
\pgfpathlineto{\pgfqpoint{2.720540in}{0.643457in}}%
\pgfpathlineto{\pgfqpoint{2.721689in}{0.664363in}}%
\pgfpathlineto{\pgfqpoint{2.722071in}{0.664363in}}%
\pgfpathlineto{\pgfqpoint{2.722454in}{0.648684in}}%
\pgfpathlineto{\pgfqpoint{2.723603in}{0.669590in}}%
\pgfpathlineto{\pgfqpoint{2.723986in}{0.669590in}}%
\pgfpathlineto{\pgfqpoint{2.723986in}{0.685269in}}%
\pgfpathlineto{\pgfqpoint{2.724751in}{0.638231in}}%
\pgfpathlineto{\pgfqpoint{2.725517in}{0.643457in}}%
\pgfpathlineto{\pgfqpoint{2.725900in}{0.643457in}}%
\pgfpathlineto{\pgfqpoint{2.726283in}{0.680043in}}%
\pgfpathlineto{\pgfqpoint{2.727431in}{0.664363in}}%
\pgfpathlineto{\pgfqpoint{2.727814in}{0.664363in}}%
\pgfpathlineto{\pgfqpoint{2.728580in}{0.638231in}}%
\pgfpathlineto{\pgfqpoint{2.729345in}{0.659137in}}%
\pgfpathlineto{\pgfqpoint{2.729728in}{0.659137in}}%
\pgfpathlineto{\pgfqpoint{2.729728in}{0.638231in}}%
\pgfpathlineto{\pgfqpoint{2.730494in}{0.669590in}}%
\pgfpathlineto{\pgfqpoint{2.731259in}{0.669590in}}%
\pgfpathlineto{\pgfqpoint{2.731642in}{0.669590in}}%
\pgfpathlineto{\pgfqpoint{2.732025in}{0.633004in}}%
\pgfpathlineto{\pgfqpoint{2.732791in}{0.680043in}}%
\pgfpathlineto{\pgfqpoint{2.733174in}{0.638231in}}%
\pgfpathlineto{\pgfqpoint{2.733556in}{0.638231in}}%
\pgfpathlineto{\pgfqpoint{2.734322in}{0.627778in}}%
\pgfpathlineto{\pgfqpoint{2.734705in}{0.680043in}}%
\pgfpathlineto{\pgfqpoint{2.735088in}{0.653910in}}%
\pgfpathlineto{\pgfqpoint{2.735471in}{0.653910in}}%
\pgfpathlineto{\pgfqpoint{2.735853in}{0.695722in}}%
\pgfpathlineto{\pgfqpoint{2.736236in}{0.648684in}}%
\pgfpathlineto{\pgfqpoint{2.737002in}{0.653910in}}%
\pgfpathlineto{\pgfqpoint{2.737385in}{0.653910in}}%
\pgfpathlineto{\pgfqpoint{2.738533in}{0.664363in}}%
\pgfpathlineto{\pgfqpoint{2.738916in}{0.653910in}}%
\pgfpathlineto{\pgfqpoint{2.739299in}{0.653910in}}%
\pgfpathlineto{\pgfqpoint{2.740064in}{0.680043in}}%
\pgfpathlineto{\pgfqpoint{2.740830in}{0.653910in}}%
\pgfpathlineto{\pgfqpoint{2.741213in}{0.653910in}}%
\pgfpathlineto{\pgfqpoint{2.741979in}{0.664363in}}%
\pgfpathlineto{\pgfqpoint{2.742744in}{0.638231in}}%
\pgfpathlineto{\pgfqpoint{2.743127in}{0.638231in}}%
\pgfpathlineto{\pgfqpoint{2.743510in}{0.674816in}}%
\pgfpathlineto{\pgfqpoint{2.744658in}{0.643457in}}%
\pgfpathlineto{\pgfqpoint{2.745041in}{0.643457in}}%
\pgfpathlineto{\pgfqpoint{2.746573in}{0.680043in}}%
\pgfpathlineto{\pgfqpoint{2.746955in}{0.680043in}}%
\pgfpathlineto{\pgfqpoint{2.747338in}{0.643457in}}%
\pgfpathlineto{\pgfqpoint{2.748487in}{0.648684in}}%
\pgfpathlineto{\pgfqpoint{2.748870in}{0.648684in}}%
\pgfpathlineto{\pgfqpoint{2.750018in}{0.669590in}}%
\pgfpathlineto{\pgfqpoint{2.750401in}{0.653910in}}%
\pgfpathlineto{\pgfqpoint{2.750784in}{0.653910in}}%
\pgfpathlineto{\pgfqpoint{2.751167in}{0.674816in}}%
\pgfpathlineto{\pgfqpoint{2.751549in}{0.633004in}}%
\pgfpathlineto{\pgfqpoint{2.752315in}{0.659137in}}%
\pgfpathlineto{\pgfqpoint{2.752698in}{0.659137in}}%
\pgfpathlineto{\pgfqpoint{2.752698in}{0.643457in}}%
\pgfpathlineto{\pgfqpoint{2.753081in}{0.669590in}}%
\pgfpathlineto{\pgfqpoint{2.754229in}{0.648684in}}%
\pgfpathlineto{\pgfqpoint{2.754612in}{0.648684in}}%
\pgfpathlineto{\pgfqpoint{2.754612in}{0.643457in}}%
\pgfpathlineto{\pgfqpoint{2.755378in}{0.664363in}}%
\pgfpathlineto{\pgfqpoint{2.756143in}{0.653910in}}%
\pgfpathlineto{\pgfqpoint{2.756526in}{0.653910in}}%
\pgfpathlineto{\pgfqpoint{2.757292in}{0.633004in}}%
\pgfpathlineto{\pgfqpoint{2.758057in}{0.669590in}}%
\pgfpathlineto{\pgfqpoint{2.758440in}{0.669590in}}%
\pgfpathlineto{\pgfqpoint{2.758440in}{0.685269in}}%
\pgfpathlineto{\pgfqpoint{2.759972in}{0.653910in}}%
\pgfpathlineto{\pgfqpoint{2.760354in}{0.653910in}}%
\pgfpathlineto{\pgfqpoint{2.760354in}{0.643457in}}%
\pgfpathlineto{\pgfqpoint{2.761886in}{0.653910in}}%
\pgfpathlineto{\pgfqpoint{2.762651in}{0.653910in}}%
\pgfpathlineto{\pgfqpoint{2.762651in}{0.664363in}}%
\pgfpathlineto{\pgfqpoint{2.763800in}{0.633004in}}%
\pgfpathlineto{\pgfqpoint{2.764183in}{0.648684in}}%
\pgfpathlineto{\pgfqpoint{2.764566in}{0.648684in}}%
\pgfpathlineto{\pgfqpoint{2.764948in}{0.685269in}}%
\pgfpathlineto{\pgfqpoint{2.766097in}{0.638231in}}%
\pgfpathlineto{\pgfqpoint{2.766480in}{0.638231in}}%
\pgfpathlineto{\pgfqpoint{2.766480in}{0.664363in}}%
\pgfpathlineto{\pgfqpoint{2.768011in}{0.638231in}}%
\pgfpathlineto{\pgfqpoint{2.768394in}{0.638231in}}%
\pgfpathlineto{\pgfqpoint{2.768777in}{0.664363in}}%
\pgfpathlineto{\pgfqpoint{2.769925in}{0.643457in}}%
\pgfpathlineto{\pgfqpoint{2.770308in}{0.643457in}}%
\pgfpathlineto{\pgfqpoint{2.770308in}{0.653910in}}%
\pgfpathlineto{\pgfqpoint{2.771074in}{0.638231in}}%
\pgfpathlineto{\pgfqpoint{2.771839in}{0.648684in}}%
\pgfpathlineto{\pgfqpoint{2.772605in}{0.648684in}}%
\pgfpathlineto{\pgfqpoint{2.773754in}{0.633004in}}%
\pgfpathlineto{\pgfqpoint{2.772988in}{0.664363in}}%
\pgfpathlineto{\pgfqpoint{2.774136in}{0.664363in}}%
\pgfpathlineto{\pgfqpoint{2.774519in}{0.664363in}}%
\pgfpathlineto{\pgfqpoint{2.774519in}{0.680043in}}%
\pgfpathlineto{\pgfqpoint{2.775285in}{0.643457in}}%
\pgfpathlineto{\pgfqpoint{2.776050in}{0.648684in}}%
\pgfpathlineto{\pgfqpoint{2.776816in}{0.648684in}}%
\pgfpathlineto{\pgfqpoint{2.776816in}{0.653910in}}%
\pgfpathlineto{\pgfqpoint{2.777582in}{0.633004in}}%
\pgfpathlineto{\pgfqpoint{2.778347in}{0.653910in}}%
\pgfpathlineto{\pgfqpoint{2.778730in}{0.653910in}}%
\pgfpathlineto{\pgfqpoint{2.778730in}{0.643457in}}%
\pgfpathlineto{\pgfqpoint{2.779879in}{0.659137in}}%
\pgfpathlineto{\pgfqpoint{2.780262in}{0.653910in}}%
\pgfpathlineto{\pgfqpoint{2.780644in}{0.653910in}}%
\pgfpathlineto{\pgfqpoint{2.781027in}{0.643457in}}%
\pgfpathlineto{\pgfqpoint{2.781793in}{0.659137in}}%
\pgfpathlineto{\pgfqpoint{2.782176in}{0.648684in}}%
\pgfpathlineto{\pgfqpoint{2.782559in}{0.648684in}}%
\pgfpathlineto{\pgfqpoint{2.782559in}{0.633004in}}%
\pgfpathlineto{\pgfqpoint{2.783707in}{0.659137in}}%
\pgfpathlineto{\pgfqpoint{2.784090in}{0.648684in}}%
\pgfpathlineto{\pgfqpoint{2.784473in}{0.648684in}}%
\pgfpathlineto{\pgfqpoint{2.784473in}{0.674816in}}%
\pgfpathlineto{\pgfqpoint{2.784856in}{0.638231in}}%
\pgfpathlineto{\pgfqpoint{2.786004in}{0.653910in}}%
\pgfpathlineto{\pgfqpoint{2.786387in}{0.653910in}}%
\pgfpathlineto{\pgfqpoint{2.786387in}{0.643457in}}%
\pgfpathlineto{\pgfqpoint{2.786770in}{0.664363in}}%
\pgfpathlineto{\pgfqpoint{2.787918in}{0.643457in}}%
\pgfpathlineto{\pgfqpoint{2.788301in}{0.643457in}}%
\pgfpathlineto{\pgfqpoint{2.788301in}{0.669590in}}%
\pgfpathlineto{\pgfqpoint{2.789067in}{0.633004in}}%
\pgfpathlineto{\pgfqpoint{2.789832in}{0.664363in}}%
\pgfpathlineto{\pgfqpoint{2.790215in}{0.664363in}}%
\pgfpathlineto{\pgfqpoint{2.790981in}{0.685269in}}%
\pgfpathlineto{\pgfqpoint{2.790598in}{0.648684in}}%
\pgfpathlineto{\pgfqpoint{2.791747in}{0.669590in}}%
\pgfpathlineto{\pgfqpoint{2.792129in}{0.669590in}}%
\pgfpathlineto{\pgfqpoint{2.793278in}{0.680043in}}%
\pgfpathlineto{\pgfqpoint{2.793661in}{0.633004in}}%
\pgfpathlineto{\pgfqpoint{2.794044in}{0.633004in}}%
\pgfpathlineto{\pgfqpoint{2.795575in}{0.680043in}}%
\pgfpathlineto{\pgfqpoint{2.795958in}{0.680043in}}%
\pgfpathlineto{\pgfqpoint{2.796723in}{0.643457in}}%
\pgfpathlineto{\pgfqpoint{2.797489in}{0.664363in}}%
\pgfpathlineto{\pgfqpoint{2.797872in}{0.664363in}}%
\pgfpathlineto{\pgfqpoint{2.797872in}{0.648684in}}%
\pgfpathlineto{\pgfqpoint{2.798637in}{0.685269in}}%
\pgfpathlineto{\pgfqpoint{2.799403in}{0.653910in}}%
\pgfpathlineto{\pgfqpoint{2.800169in}{0.653910in}}%
\pgfpathlineto{\pgfqpoint{2.801317in}{0.674816in}}%
\pgfpathlineto{\pgfqpoint{2.800552in}{0.648684in}}%
\pgfpathlineto{\pgfqpoint{2.801700in}{0.648684in}}%
\pgfpathlineto{\pgfqpoint{2.802466in}{0.648684in}}%
\pgfpathlineto{\pgfqpoint{2.802466in}{0.643457in}}%
\pgfpathlineto{\pgfqpoint{2.803997in}{0.669590in}}%
\pgfpathlineto{\pgfqpoint{2.804380in}{0.669590in}}%
\pgfpathlineto{\pgfqpoint{2.804380in}{0.643457in}}%
\pgfpathlineto{\pgfqpoint{2.805146in}{0.680043in}}%
\pgfpathlineto{\pgfqpoint{2.805911in}{0.664363in}}%
\pgfpathlineto{\pgfqpoint{2.806294in}{0.664363in}}%
\pgfpathlineto{\pgfqpoint{2.807060in}{0.648684in}}%
\pgfpathlineto{\pgfqpoint{2.807825in}{0.680043in}}%
\pgfpathlineto{\pgfqpoint{2.808208in}{0.680043in}}%
\pgfpathlineto{\pgfqpoint{2.808591in}{0.648684in}}%
\pgfpathlineto{\pgfqpoint{2.809740in}{0.669590in}}%
\pgfpathlineto{\pgfqpoint{2.810122in}{0.669590in}}%
\pgfpathlineto{\pgfqpoint{2.810122in}{0.638231in}}%
\pgfpathlineto{\pgfqpoint{2.811654in}{0.648684in}}%
\pgfpathlineto{\pgfqpoint{2.812037in}{0.648684in}}%
\pgfpathlineto{\pgfqpoint{2.813185in}{0.638231in}}%
\pgfpathlineto{\pgfqpoint{2.813568in}{0.664363in}}%
\pgfpathlineto{\pgfqpoint{2.813951in}{0.664363in}}%
\pgfpathlineto{\pgfqpoint{2.814716in}{0.633004in}}%
\pgfpathlineto{\pgfqpoint{2.815482in}{0.669590in}}%
\pgfpathlineto{\pgfqpoint{2.815865in}{0.669590in}}%
\pgfpathlineto{\pgfqpoint{2.816630in}{0.680043in}}%
\pgfpathlineto{\pgfqpoint{2.817396in}{0.638231in}}%
\pgfpathlineto{\pgfqpoint{2.817779in}{0.638231in}}%
\pgfpathlineto{\pgfqpoint{2.818162in}{0.669590in}}%
\pgfpathlineto{\pgfqpoint{2.819310in}{0.653910in}}%
\pgfpathlineto{\pgfqpoint{2.819693in}{0.653910in}}%
\pgfpathlineto{\pgfqpoint{2.819693in}{0.664363in}}%
\pgfpathlineto{\pgfqpoint{2.820459in}{0.648684in}}%
\pgfpathlineto{\pgfqpoint{2.821224in}{0.653910in}}%
\pgfpathlineto{\pgfqpoint{2.821607in}{0.653910in}}%
\pgfpathlineto{\pgfqpoint{2.821607in}{0.643457in}}%
\pgfpathlineto{\pgfqpoint{2.822756in}{0.669590in}}%
\pgfpathlineto{\pgfqpoint{2.823139in}{0.659137in}}%
\pgfpathlineto{\pgfqpoint{2.823521in}{0.659137in}}%
\pgfpathlineto{\pgfqpoint{2.824287in}{0.638231in}}%
\pgfpathlineto{\pgfqpoint{2.825053in}{0.659137in}}%
\pgfpathlineto{\pgfqpoint{2.825436in}{0.659137in}}%
\pgfpathlineto{\pgfqpoint{2.825818in}{0.633004in}}%
\pgfpathlineto{\pgfqpoint{2.826967in}{0.674816in}}%
\pgfpathlineto{\pgfqpoint{2.827350in}{0.674816in}}%
\pgfpathlineto{\pgfqpoint{2.827350in}{0.643457in}}%
\pgfpathlineto{\pgfqpoint{2.828881in}{0.653910in}}%
\pgfpathlineto{\pgfqpoint{2.829647in}{0.653910in}}%
\pgfpathlineto{\pgfqpoint{2.830030in}{0.627778in}}%
\pgfpathlineto{\pgfqpoint{2.830795in}{0.664363in}}%
\pgfpathlineto{\pgfqpoint{2.831178in}{0.659137in}}%
\pgfpathlineto{\pgfqpoint{2.831561in}{0.659137in}}%
\pgfpathlineto{\pgfqpoint{2.832327in}{0.643457in}}%
\pgfpathlineto{\pgfqpoint{2.831944in}{0.669590in}}%
\pgfpathlineto{\pgfqpoint{2.833092in}{0.653910in}}%
\pgfpathlineto{\pgfqpoint{2.833475in}{0.653910in}}%
\pgfpathlineto{\pgfqpoint{2.833475in}{0.643457in}}%
\pgfpathlineto{\pgfqpoint{2.835006in}{0.674816in}}%
\pgfpathlineto{\pgfqpoint{2.835389in}{0.674816in}}%
\pgfpathlineto{\pgfqpoint{2.835772in}{0.643457in}}%
\pgfpathlineto{\pgfqpoint{2.836920in}{0.664363in}}%
\pgfpathlineto{\pgfqpoint{2.837686in}{0.664363in}}%
\pgfpathlineto{\pgfqpoint{2.837686in}{0.638231in}}%
\pgfpathlineto{\pgfqpoint{2.839217in}{0.664363in}}%
\pgfpathlineto{\pgfqpoint{2.839600in}{0.664363in}}%
\pgfpathlineto{\pgfqpoint{2.840366in}{0.638231in}}%
\pgfpathlineto{\pgfqpoint{2.839983in}{0.674816in}}%
\pgfpathlineto{\pgfqpoint{2.841132in}{0.653910in}}%
\pgfpathlineto{\pgfqpoint{2.841514in}{0.653910in}}%
\pgfpathlineto{\pgfqpoint{2.841514in}{0.633004in}}%
\pgfpathlineto{\pgfqpoint{2.843046in}{0.674816in}}%
\pgfpathlineto{\pgfqpoint{2.843429in}{0.674816in}}%
\pgfpathlineto{\pgfqpoint{2.844960in}{0.633004in}}%
\pgfpathlineto{\pgfqpoint{2.845343in}{0.633004in}}%
\pgfpathlineto{\pgfqpoint{2.845726in}{0.664363in}}%
\pgfpathlineto{\pgfqpoint{2.846874in}{0.653910in}}%
\pgfpathlineto{\pgfqpoint{2.847257in}{0.653910in}}%
\pgfpathlineto{\pgfqpoint{2.847257in}{0.643457in}}%
\pgfpathlineto{\pgfqpoint{2.848405in}{0.685269in}}%
\pgfpathlineto{\pgfqpoint{2.848788in}{0.643457in}}%
\pgfpathlineto{\pgfqpoint{2.849171in}{0.643457in}}%
\pgfpathlineto{\pgfqpoint{2.849937in}{0.669590in}}%
\pgfpathlineto{\pgfqpoint{2.850702in}{0.659137in}}%
\pgfpathlineto{\pgfqpoint{2.851085in}{0.659137in}}%
\pgfpathlineto{\pgfqpoint{2.852234in}{0.627778in}}%
\pgfpathlineto{\pgfqpoint{2.852616in}{0.648684in}}%
\pgfpathlineto{\pgfqpoint{2.852999in}{0.648684in}}%
\pgfpathlineto{\pgfqpoint{2.854148in}{0.669590in}}%
\pgfpathlineto{\pgfqpoint{2.853382in}{0.643457in}}%
\pgfpathlineto{\pgfqpoint{2.854531in}{0.659137in}}%
\pgfpathlineto{\pgfqpoint{2.854913in}{0.659137in}}%
\pgfpathlineto{\pgfqpoint{2.856062in}{0.633004in}}%
\pgfpathlineto{\pgfqpoint{2.856445in}{0.669590in}}%
\pgfpathlineto{\pgfqpoint{2.856828in}{0.669590in}}%
\pgfpathlineto{\pgfqpoint{2.858359in}{0.633004in}}%
\pgfpathlineto{\pgfqpoint{2.858742in}{0.633004in}}%
\pgfpathlineto{\pgfqpoint{2.859890in}{0.685269in}}%
\pgfpathlineto{\pgfqpoint{2.860273in}{0.664363in}}%
\pgfpathlineto{\pgfqpoint{2.860656in}{0.664363in}}%
\pgfpathlineto{\pgfqpoint{2.860656in}{0.643457in}}%
\pgfpathlineto{\pgfqpoint{2.862187in}{0.648684in}}%
\pgfpathlineto{\pgfqpoint{2.862570in}{0.648684in}}%
\pgfpathlineto{\pgfqpoint{2.862570in}{0.643457in}}%
\pgfpathlineto{\pgfqpoint{2.864101in}{0.669590in}}%
\pgfpathlineto{\pgfqpoint{2.864484in}{0.669590in}}%
\pgfpathlineto{\pgfqpoint{2.864867in}{0.633004in}}%
\pgfpathlineto{\pgfqpoint{2.866016in}{0.659137in}}%
\pgfpathlineto{\pgfqpoint{2.866398in}{0.659137in}}%
\pgfpathlineto{\pgfqpoint{2.867164in}{0.638231in}}%
\pgfpathlineto{\pgfqpoint{2.867930in}{0.669590in}}%
\pgfpathlineto{\pgfqpoint{2.869078in}{0.669590in}}%
\pgfpathlineto{\pgfqpoint{2.869461in}{0.643457in}}%
\pgfpathlineto{\pgfqpoint{2.870610in}{0.674816in}}%
\pgfpathlineto{\pgfqpoint{2.870992in}{0.674816in}}%
\pgfpathlineto{\pgfqpoint{2.871375in}{0.643457in}}%
\pgfpathlineto{\pgfqpoint{2.871758in}{0.680043in}}%
\pgfpathlineto{\pgfqpoint{2.872524in}{0.653910in}}%
\pgfpathlineto{\pgfqpoint{2.872906in}{0.653910in}}%
\pgfpathlineto{\pgfqpoint{2.872906in}{0.669590in}}%
\pgfpathlineto{\pgfqpoint{2.873289in}{0.648684in}}%
\pgfpathlineto{\pgfqpoint{2.874438in}{0.659137in}}%
\pgfpathlineto{\pgfqpoint{2.875203in}{0.659137in}}%
\pgfpathlineto{\pgfqpoint{2.875969in}{0.674816in}}%
\pgfpathlineto{\pgfqpoint{2.876352in}{0.648684in}}%
\pgfpathlineto{\pgfqpoint{2.876735in}{0.653910in}}%
\pgfpathlineto{\pgfqpoint{2.877118in}{0.653910in}}%
\pgfpathlineto{\pgfqpoint{2.877118in}{0.659137in}}%
\pgfpathlineto{\pgfqpoint{2.877500in}{0.638231in}}%
\pgfpathlineto{\pgfqpoint{2.878649in}{0.648684in}}%
\pgfpathlineto{\pgfqpoint{2.879032in}{0.648684in}}%
\pgfpathlineto{\pgfqpoint{2.879032in}{0.669590in}}%
\pgfpathlineto{\pgfqpoint{2.880563in}{0.664363in}}%
\pgfpathlineto{\pgfqpoint{2.880946in}{0.664363in}}%
\pgfpathlineto{\pgfqpoint{2.880946in}{0.653910in}}%
\pgfpathlineto{\pgfqpoint{2.881329in}{0.669590in}}%
\pgfpathlineto{\pgfqpoint{2.882477in}{0.664363in}}%
\pgfpathlineto{\pgfqpoint{2.882860in}{0.664363in}}%
\pgfpathlineto{\pgfqpoint{2.884009in}{0.643457in}}%
\pgfpathlineto{\pgfqpoint{2.884391in}{0.648684in}}%
\pgfpathlineto{\pgfqpoint{2.885157in}{0.648684in}}%
\pgfpathlineto{\pgfqpoint{2.886306in}{0.669590in}}%
\pgfpathlineto{\pgfqpoint{2.885540in}{0.643457in}}%
\pgfpathlineto{\pgfqpoint{2.886688in}{0.643457in}}%
\pgfpathlineto{\pgfqpoint{2.887071in}{0.643457in}}%
\pgfpathlineto{\pgfqpoint{2.887071in}{0.653910in}}%
\pgfpathlineto{\pgfqpoint{2.887454in}{0.638231in}}%
\pgfpathlineto{\pgfqpoint{2.888603in}{0.648684in}}%
\pgfpathlineto{\pgfqpoint{2.888985in}{0.648684in}}%
\pgfpathlineto{\pgfqpoint{2.888985in}{0.633004in}}%
\pgfpathlineto{\pgfqpoint{2.889368in}{0.669590in}}%
\pgfpathlineto{\pgfqpoint{2.890517in}{0.659137in}}%
\pgfpathlineto{\pgfqpoint{2.891665in}{0.659137in}}%
\pgfpathlineto{\pgfqpoint{2.891665in}{0.648684in}}%
\pgfpathlineto{\pgfqpoint{2.893196in}{0.659137in}}%
\pgfpathlineto{\pgfqpoint{2.893579in}{0.659137in}}%
\pgfpathlineto{\pgfqpoint{2.893962in}{0.674816in}}%
\pgfpathlineto{\pgfqpoint{2.894345in}{0.648684in}}%
\pgfpathlineto{\pgfqpoint{2.895111in}{0.674816in}}%
\pgfpathlineto{\pgfqpoint{2.895493in}{0.674816in}}%
\pgfpathlineto{\pgfqpoint{2.897025in}{0.648684in}}%
\pgfpathlineto{\pgfqpoint{2.897790in}{0.648684in}}%
\pgfpathlineto{\pgfqpoint{2.897790in}{0.643457in}}%
\pgfpathlineto{\pgfqpoint{2.899322in}{0.690496in}}%
\pgfpathlineto{\pgfqpoint{2.899705in}{0.690496in}}%
\pgfpathlineto{\pgfqpoint{2.900470in}{0.643457in}}%
\pgfpathlineto{\pgfqpoint{2.901236in}{0.653910in}}%
\pgfpathlineto{\pgfqpoint{2.901619in}{0.653910in}}%
\pgfpathlineto{\pgfqpoint{2.901619in}{0.633004in}}%
\pgfpathlineto{\pgfqpoint{2.902767in}{0.674816in}}%
\pgfpathlineto{\pgfqpoint{2.903150in}{0.643457in}}%
\pgfpathlineto{\pgfqpoint{2.903533in}{0.643457in}}%
\pgfpathlineto{\pgfqpoint{2.904299in}{0.685269in}}%
\pgfpathlineto{\pgfqpoint{2.905064in}{0.648684in}}%
\pgfpathlineto{\pgfqpoint{2.905447in}{0.648684in}}%
\pgfpathlineto{\pgfqpoint{2.905447in}{0.638231in}}%
\pgfpathlineto{\pgfqpoint{2.905830in}{0.669590in}}%
\pgfpathlineto{\pgfqpoint{2.906978in}{0.638231in}}%
\pgfpathlineto{\pgfqpoint{2.907361in}{0.638231in}}%
\pgfpathlineto{\pgfqpoint{2.907361in}{0.653910in}}%
\pgfpathlineto{\pgfqpoint{2.907744in}{0.633004in}}%
\pgfpathlineto{\pgfqpoint{2.908893in}{0.648684in}}%
\pgfpathlineto{\pgfqpoint{2.910041in}{0.648684in}}%
\pgfpathlineto{\pgfqpoint{2.910807in}{0.680043in}}%
\pgfpathlineto{\pgfqpoint{2.911572in}{0.659137in}}%
\pgfpathlineto{\pgfqpoint{2.911955in}{0.659137in}}%
\pgfpathlineto{\pgfqpoint{2.912338in}{0.653910in}}%
\pgfpathlineto{\pgfqpoint{2.913486in}{0.695722in}}%
\pgfpathlineto{\pgfqpoint{2.913869in}{0.695722in}}%
\pgfpathlineto{\pgfqpoint{2.915401in}{0.643457in}}%
\pgfpathlineto{\pgfqpoint{2.915783in}{0.643457in}}%
\pgfpathlineto{\pgfqpoint{2.915783in}{0.659137in}}%
\pgfpathlineto{\pgfqpoint{2.916166in}{0.638231in}}%
\pgfpathlineto{\pgfqpoint{2.917315in}{0.648684in}}%
\pgfpathlineto{\pgfqpoint{2.917698in}{0.648684in}}%
\pgfpathlineto{\pgfqpoint{2.918080in}{0.680043in}}%
\pgfpathlineto{\pgfqpoint{2.919229in}{0.638231in}}%
\pgfpathlineto{\pgfqpoint{2.919612in}{0.638231in}}%
\pgfpathlineto{\pgfqpoint{2.919995in}{0.669590in}}%
\pgfpathlineto{\pgfqpoint{2.920760in}{0.648684in}}%
\pgfusepath{stroke}%
\end{pgfscope}%
\begin{pgfscope}%
\pgfpathrectangle{\pgfqpoint{0.781944in}{0.552778in}}{\pgfqpoint{2.138715in}{1.650000in}}%
\pgfusepath{clip}%
\pgfsetrectcap%
\pgfsetroundjoin%
\pgfsetlinewidth{1.505625pt}%
\definecolor{currentstroke}{rgb}{0.172549,0.627451,0.172549}%
\pgfsetstrokecolor{currentstroke}%
\pgfsetstrokeopacity{0.800000}%
\pgfsetdash{}{0pt}%
\pgfpathmoveto{\pgfqpoint{0.781889in}{0.659137in}}%
\pgfpathlineto{\pgfqpoint{0.782272in}{0.659137in}}%
\pgfpathlineto{\pgfqpoint{0.782272in}{0.648684in}}%
\pgfpathlineto{\pgfqpoint{0.783037in}{0.680043in}}%
\pgfpathlineto{\pgfqpoint{0.783803in}{0.680043in}}%
\pgfpathlineto{\pgfqpoint{0.784186in}{0.680043in}}%
\pgfpathlineto{\pgfqpoint{0.784186in}{0.659137in}}%
\pgfpathlineto{\pgfqpoint{0.785717in}{0.695722in}}%
\pgfpathlineto{\pgfqpoint{0.786100in}{0.695722in}}%
\pgfpathlineto{\pgfqpoint{0.786483in}{0.659137in}}%
\pgfpathlineto{\pgfqpoint{0.787631in}{0.669590in}}%
\pgfpathlineto{\pgfqpoint{0.788014in}{0.669590in}}%
\pgfpathlineto{\pgfqpoint{0.788014in}{0.643457in}}%
\pgfpathlineto{\pgfqpoint{0.788397in}{0.674816in}}%
\pgfpathlineto{\pgfqpoint{0.789545in}{0.653910in}}%
\pgfpathlineto{\pgfqpoint{0.789928in}{0.653910in}}%
\pgfpathlineto{\pgfqpoint{0.789928in}{0.648684in}}%
\pgfpathlineto{\pgfqpoint{0.791077in}{0.685269in}}%
\pgfpathlineto{\pgfqpoint{0.791460in}{0.664363in}}%
\pgfpathlineto{\pgfqpoint{0.791842in}{0.664363in}}%
\pgfpathlineto{\pgfqpoint{0.792225in}{0.653910in}}%
\pgfpathlineto{\pgfqpoint{0.793374in}{0.680043in}}%
\pgfpathlineto{\pgfqpoint{0.793757in}{0.680043in}}%
\pgfpathlineto{\pgfqpoint{0.793757in}{0.685269in}}%
\pgfpathlineto{\pgfqpoint{0.795288in}{0.664363in}}%
\pgfpathlineto{\pgfqpoint{0.795671in}{0.664363in}}%
\pgfpathlineto{\pgfqpoint{0.795671in}{0.659137in}}%
\pgfpathlineto{\pgfqpoint{0.797202in}{0.685269in}}%
\pgfpathlineto{\pgfqpoint{0.797585in}{0.685269in}}%
\pgfpathlineto{\pgfqpoint{0.798733in}{0.659137in}}%
\pgfpathlineto{\pgfqpoint{0.799116in}{0.690496in}}%
\pgfpathlineto{\pgfqpoint{0.799499in}{0.690496in}}%
\pgfpathlineto{\pgfqpoint{0.800648in}{0.706175in}}%
\pgfpathlineto{\pgfqpoint{0.799882in}{0.648684in}}%
\pgfpathlineto{\pgfqpoint{0.801030in}{0.706175in}}%
\pgfpathlineto{\pgfqpoint{0.801413in}{0.706175in}}%
\pgfpathlineto{\pgfqpoint{0.801796in}{0.669590in}}%
\pgfpathlineto{\pgfqpoint{0.802944in}{0.685269in}}%
\pgfpathlineto{\pgfqpoint{0.803327in}{0.685269in}}%
\pgfpathlineto{\pgfqpoint{0.803710in}{0.653910in}}%
\pgfpathlineto{\pgfqpoint{0.804859in}{0.669590in}}%
\pgfpathlineto{\pgfqpoint{0.805241in}{0.669590in}}%
\pgfpathlineto{\pgfqpoint{0.806007in}{0.695722in}}%
\pgfpathlineto{\pgfqpoint{0.805624in}{0.653910in}}%
\pgfpathlineto{\pgfqpoint{0.806773in}{0.659137in}}%
\pgfpathlineto{\pgfqpoint{0.807156in}{0.659137in}}%
\pgfpathlineto{\pgfqpoint{0.807156in}{0.653910in}}%
\pgfpathlineto{\pgfqpoint{0.808304in}{0.706175in}}%
\pgfpathlineto{\pgfqpoint{0.808687in}{0.659137in}}%
\pgfpathlineto{\pgfqpoint{0.809070in}{0.659137in}}%
\pgfpathlineto{\pgfqpoint{0.809453in}{0.706175in}}%
\pgfpathlineto{\pgfqpoint{0.810601in}{0.664363in}}%
\pgfpathlineto{\pgfqpoint{0.810984in}{0.664363in}}%
\pgfpathlineto{\pgfqpoint{0.812132in}{0.680043in}}%
\pgfpathlineto{\pgfqpoint{0.812515in}{0.680043in}}%
\pgfpathlineto{\pgfqpoint{0.812898in}{0.680043in}}%
\pgfpathlineto{\pgfqpoint{0.812898in}{0.700949in}}%
\pgfpathlineto{\pgfqpoint{0.813281in}{0.653910in}}%
\pgfpathlineto{\pgfqpoint{0.814429in}{0.674816in}}%
\pgfpathlineto{\pgfqpoint{0.814812in}{0.674816in}}%
\pgfpathlineto{\pgfqpoint{0.815961in}{0.700949in}}%
\pgfpathlineto{\pgfqpoint{0.816344in}{0.700949in}}%
\pgfpathlineto{\pgfqpoint{0.816726in}{0.700949in}}%
\pgfpathlineto{\pgfqpoint{0.817492in}{0.680043in}}%
\pgfpathlineto{\pgfqpoint{0.818258in}{0.680043in}}%
\pgfpathlineto{\pgfqpoint{0.818641in}{0.680043in}}%
\pgfpathlineto{\pgfqpoint{0.818641in}{0.711401in}}%
\pgfpathlineto{\pgfqpoint{0.820172in}{0.695722in}}%
\pgfpathlineto{\pgfqpoint{0.820938in}{0.695722in}}%
\pgfpathlineto{\pgfqpoint{0.822086in}{0.669590in}}%
\pgfpathlineto{\pgfqpoint{0.822469in}{0.695722in}}%
\pgfpathlineto{\pgfqpoint{0.822852in}{0.695722in}}%
\pgfpathlineto{\pgfqpoint{0.823234in}{0.659137in}}%
\pgfpathlineto{\pgfqpoint{0.824383in}{0.706175in}}%
\pgfpathlineto{\pgfqpoint{0.824766in}{0.706175in}}%
\pgfpathlineto{\pgfqpoint{0.824766in}{0.653910in}}%
\pgfpathlineto{\pgfqpoint{0.825531in}{0.711401in}}%
\pgfpathlineto{\pgfqpoint{0.826297in}{0.685269in}}%
\pgfpathlineto{\pgfqpoint{0.826680in}{0.685269in}}%
\pgfpathlineto{\pgfqpoint{0.826680in}{0.674816in}}%
\pgfpathlineto{\pgfqpoint{0.828211in}{0.711401in}}%
\pgfpathlineto{\pgfqpoint{0.828594in}{0.711401in}}%
\pgfpathlineto{\pgfqpoint{0.829360in}{0.659137in}}%
\pgfpathlineto{\pgfqpoint{0.830125in}{0.721854in}}%
\pgfpathlineto{\pgfqpoint{0.830508in}{0.721854in}}%
\pgfpathlineto{\pgfqpoint{0.832040in}{0.659137in}}%
\pgfpathlineto{\pgfqpoint{0.832422in}{0.659137in}}%
\pgfpathlineto{\pgfqpoint{0.832422in}{0.721854in}}%
\pgfpathlineto{\pgfqpoint{0.833954in}{0.664363in}}%
\pgfpathlineto{\pgfqpoint{0.834337in}{0.664363in}}%
\pgfpathlineto{\pgfqpoint{0.834337in}{0.706175in}}%
\pgfpathlineto{\pgfqpoint{0.835868in}{0.680043in}}%
\pgfpathlineto{\pgfqpoint{0.836251in}{0.680043in}}%
\pgfpathlineto{\pgfqpoint{0.837399in}{0.716628in}}%
\pgfpathlineto{\pgfqpoint{0.837782in}{0.669590in}}%
\pgfpathlineto{\pgfqpoint{0.838165in}{0.669590in}}%
\pgfpathlineto{\pgfqpoint{0.838165in}{0.664363in}}%
\pgfpathlineto{\pgfqpoint{0.839313in}{0.711401in}}%
\pgfpathlineto{\pgfqpoint{0.839696in}{0.685269in}}%
\pgfpathlineto{\pgfqpoint{0.840079in}{0.685269in}}%
\pgfpathlineto{\pgfqpoint{0.840845in}{0.711401in}}%
\pgfpathlineto{\pgfqpoint{0.840462in}{0.674816in}}%
\pgfpathlineto{\pgfqpoint{0.841610in}{0.706175in}}%
\pgfpathlineto{\pgfqpoint{0.841993in}{0.706175in}}%
\pgfpathlineto{\pgfqpoint{0.843524in}{0.669590in}}%
\pgfpathlineto{\pgfqpoint{0.843907in}{0.669590in}}%
\pgfpathlineto{\pgfqpoint{0.843907in}{0.727081in}}%
\pgfpathlineto{\pgfqpoint{0.845439in}{0.695722in}}%
\pgfpathlineto{\pgfqpoint{0.845821in}{0.695722in}}%
\pgfpathlineto{\pgfqpoint{0.847353in}{0.669590in}}%
\pgfpathlineto{\pgfqpoint{0.847736in}{0.669590in}}%
\pgfpathlineto{\pgfqpoint{0.848118in}{0.716628in}}%
\pgfpathlineto{\pgfqpoint{0.849267in}{0.695722in}}%
\pgfpathlineto{\pgfqpoint{0.849650in}{0.695722in}}%
\pgfpathlineto{\pgfqpoint{0.850798in}{0.727081in}}%
\pgfpathlineto{\pgfqpoint{0.851181in}{0.685269in}}%
\pgfpathlineto{\pgfqpoint{0.851564in}{0.685269in}}%
\pgfpathlineto{\pgfqpoint{0.853095in}{0.732307in}}%
\pgfpathlineto{\pgfqpoint{0.853478in}{0.732307in}}%
\pgfpathlineto{\pgfqpoint{0.854627in}{0.685269in}}%
\pgfpathlineto{\pgfqpoint{0.855009in}{0.685269in}}%
\pgfpathlineto{\pgfqpoint{0.855392in}{0.685269in}}%
\pgfpathlineto{\pgfqpoint{0.855392in}{0.690496in}}%
\pgfpathlineto{\pgfqpoint{0.856158in}{0.653910in}}%
\pgfpathlineto{\pgfqpoint{0.856924in}{0.674816in}}%
\pgfpathlineto{\pgfqpoint{0.857306in}{0.674816in}}%
\pgfpathlineto{\pgfqpoint{0.857689in}{0.737534in}}%
\pgfpathlineto{\pgfqpoint{0.858838in}{0.721854in}}%
\pgfpathlineto{\pgfqpoint{0.859221in}{0.721854in}}%
\pgfpathlineto{\pgfqpoint{0.859603in}{0.674816in}}%
\pgfpathlineto{\pgfqpoint{0.860752in}{0.674816in}}%
\pgfpathlineto{\pgfqpoint{0.861135in}{0.674816in}}%
\pgfpathlineto{\pgfqpoint{0.862666in}{0.721854in}}%
\pgfpathlineto{\pgfqpoint{0.863049in}{0.721854in}}%
\pgfpathlineto{\pgfqpoint{0.864197in}{0.669590in}}%
\pgfpathlineto{\pgfqpoint{0.864580in}{0.674816in}}%
\pgfpathlineto{\pgfqpoint{0.864963in}{0.674816in}}%
\pgfpathlineto{\pgfqpoint{0.864963in}{0.669590in}}%
\pgfpathlineto{\pgfqpoint{0.865729in}{0.727081in}}%
\pgfpathlineto{\pgfqpoint{0.866494in}{0.685269in}}%
\pgfpathlineto{\pgfqpoint{0.866877in}{0.685269in}}%
\pgfpathlineto{\pgfqpoint{0.867260in}{0.664363in}}%
\pgfpathlineto{\pgfqpoint{0.868408in}{0.732307in}}%
\pgfpathlineto{\pgfqpoint{0.868791in}{0.732307in}}%
\pgfpathlineto{\pgfqpoint{0.869557in}{0.690496in}}%
\pgfpathlineto{\pgfqpoint{0.870323in}{0.700949in}}%
\pgfpathlineto{\pgfqpoint{0.870705in}{0.700949in}}%
\pgfpathlineto{\pgfqpoint{0.870705in}{0.680043in}}%
\pgfpathlineto{\pgfqpoint{0.871088in}{0.753213in}}%
\pgfpathlineto{\pgfqpoint{0.872237in}{0.727081in}}%
\pgfpathlineto{\pgfqpoint{0.872620in}{0.727081in}}%
\pgfpathlineto{\pgfqpoint{0.873385in}{0.680043in}}%
\pgfpathlineto{\pgfqpoint{0.873002in}{0.737534in}}%
\pgfpathlineto{\pgfqpoint{0.874151in}{0.700949in}}%
\pgfpathlineto{\pgfqpoint{0.874917in}{0.700949in}}%
\pgfpathlineto{\pgfqpoint{0.875299in}{0.747987in}}%
\pgfpathlineto{\pgfqpoint{0.876065in}{0.680043in}}%
\pgfpathlineto{\pgfqpoint{0.876448in}{0.700949in}}%
\pgfpathlineto{\pgfqpoint{0.876831in}{0.700949in}}%
\pgfpathlineto{\pgfqpoint{0.877214in}{0.737534in}}%
\pgfpathlineto{\pgfqpoint{0.878362in}{0.695722in}}%
\pgfpathlineto{\pgfqpoint{0.878745in}{0.695722in}}%
\pgfpathlineto{\pgfqpoint{0.878745in}{0.674816in}}%
\pgfpathlineto{\pgfqpoint{0.880276in}{0.711401in}}%
\pgfpathlineto{\pgfqpoint{0.880659in}{0.711401in}}%
\pgfpathlineto{\pgfqpoint{0.880659in}{0.680043in}}%
\pgfpathlineto{\pgfqpoint{0.881425in}{0.721854in}}%
\pgfpathlineto{\pgfqpoint{0.882190in}{0.700949in}}%
\pgfpathlineto{\pgfqpoint{0.882573in}{0.700949in}}%
\pgfpathlineto{\pgfqpoint{0.883339in}{0.685269in}}%
\pgfpathlineto{\pgfqpoint{0.882956in}{0.716628in}}%
\pgfpathlineto{\pgfqpoint{0.884104in}{0.706175in}}%
\pgfpathlineto{\pgfqpoint{0.884487in}{0.706175in}}%
\pgfpathlineto{\pgfqpoint{0.884487in}{0.690496in}}%
\pgfpathlineto{\pgfqpoint{0.885253in}{0.737534in}}%
\pgfpathlineto{\pgfqpoint{0.886019in}{0.732307in}}%
\pgfpathlineto{\pgfqpoint{0.886401in}{0.732307in}}%
\pgfpathlineto{\pgfqpoint{0.887933in}{0.690496in}}%
\pgfpathlineto{\pgfqpoint{0.888316in}{0.690496in}}%
\pgfpathlineto{\pgfqpoint{0.889464in}{0.732307in}}%
\pgfpathlineto{\pgfqpoint{0.889847in}{0.695722in}}%
\pgfpathlineto{\pgfqpoint{0.890230in}{0.695722in}}%
\pgfpathlineto{\pgfqpoint{0.891761in}{0.737534in}}%
\pgfpathlineto{\pgfqpoint{0.892144in}{0.737534in}}%
\pgfpathlineto{\pgfqpoint{0.892910in}{0.685269in}}%
\pgfpathlineto{\pgfqpoint{0.893675in}{0.700949in}}%
\pgfpathlineto{\pgfqpoint{0.894058in}{0.700949in}}%
\pgfpathlineto{\pgfqpoint{0.894824in}{0.747987in}}%
\pgfpathlineto{\pgfqpoint{0.895207in}{0.690496in}}%
\pgfpathlineto{\pgfqpoint{0.895589in}{0.706175in}}%
\pgfpathlineto{\pgfqpoint{0.895972in}{0.706175in}}%
\pgfpathlineto{\pgfqpoint{0.896738in}{0.690496in}}%
\pgfpathlineto{\pgfqpoint{0.897504in}{0.700949in}}%
\pgfpathlineto{\pgfqpoint{0.897886in}{0.700949in}}%
\pgfpathlineto{\pgfqpoint{0.898652in}{0.758440in}}%
\pgfpathlineto{\pgfqpoint{0.898269in}{0.685269in}}%
\pgfpathlineto{\pgfqpoint{0.899418in}{0.721854in}}%
\pgfpathlineto{\pgfqpoint{0.899800in}{0.721854in}}%
\pgfpathlineto{\pgfqpoint{0.900949in}{0.695722in}}%
\pgfpathlineto{\pgfqpoint{0.901332in}{0.747987in}}%
\pgfpathlineto{\pgfqpoint{0.901715in}{0.747987in}}%
\pgfpathlineto{\pgfqpoint{0.903246in}{0.690496in}}%
\pgfpathlineto{\pgfqpoint{0.903629in}{0.690496in}}%
\pgfpathlineto{\pgfqpoint{0.903629in}{0.664363in}}%
\pgfpathlineto{\pgfqpoint{0.904012in}{0.711401in}}%
\pgfpathlineto{\pgfqpoint{0.905160in}{0.711401in}}%
\pgfpathlineto{\pgfqpoint{0.905543in}{0.711401in}}%
\pgfpathlineto{\pgfqpoint{0.906309in}{0.737534in}}%
\pgfpathlineto{\pgfqpoint{0.905926in}{0.690496in}}%
\pgfpathlineto{\pgfqpoint{0.907074in}{0.727081in}}%
\pgfpathlineto{\pgfqpoint{0.907457in}{0.727081in}}%
\pgfpathlineto{\pgfqpoint{0.907457in}{0.674816in}}%
\pgfpathlineto{\pgfqpoint{0.907840in}{0.732307in}}%
\pgfpathlineto{\pgfqpoint{0.908988in}{0.680043in}}%
\pgfpathlineto{\pgfqpoint{0.909371in}{0.680043in}}%
\pgfpathlineto{\pgfqpoint{0.910903in}{0.732307in}}%
\pgfpathlineto{\pgfqpoint{0.911285in}{0.732307in}}%
\pgfpathlineto{\pgfqpoint{0.911285in}{0.758440in}}%
\pgfpathlineto{\pgfqpoint{0.911668in}{0.711401in}}%
\pgfpathlineto{\pgfqpoint{0.912817in}{0.711401in}}%
\pgfpathlineto{\pgfqpoint{0.913582in}{0.711401in}}%
\pgfpathlineto{\pgfqpoint{0.913965in}{0.747987in}}%
\pgfpathlineto{\pgfqpoint{0.915114in}{0.721854in}}%
\pgfpathlineto{\pgfqpoint{0.915497in}{0.721854in}}%
\pgfpathlineto{\pgfqpoint{0.915879in}{0.685269in}}%
\pgfpathlineto{\pgfqpoint{0.916645in}{0.742760in}}%
\pgfpathlineto{\pgfqpoint{0.917028in}{0.685269in}}%
\pgfpathlineto{\pgfqpoint{0.917411in}{0.685269in}}%
\pgfpathlineto{\pgfqpoint{0.918559in}{0.737534in}}%
\pgfpathlineto{\pgfqpoint{0.918942in}{0.674816in}}%
\pgfpathlineto{\pgfqpoint{0.919325in}{0.674816in}}%
\pgfpathlineto{\pgfqpoint{0.919708in}{0.747987in}}%
\pgfpathlineto{\pgfqpoint{0.920090in}{0.664363in}}%
\pgfpathlineto{\pgfqpoint{0.920856in}{0.727081in}}%
\pgfpathlineto{\pgfqpoint{0.921239in}{0.727081in}}%
\pgfpathlineto{\pgfqpoint{0.921239in}{0.685269in}}%
\pgfpathlineto{\pgfqpoint{0.922770in}{0.758440in}}%
\pgfpathlineto{\pgfqpoint{0.923153in}{0.758440in}}%
\pgfpathlineto{\pgfqpoint{0.923919in}{0.674816in}}%
\pgfpathlineto{\pgfqpoint{0.924684in}{0.716628in}}%
\pgfpathlineto{\pgfqpoint{0.925067in}{0.716628in}}%
\pgfpathlineto{\pgfqpoint{0.925067in}{0.774119in}}%
\pgfpathlineto{\pgfqpoint{0.926216in}{0.706175in}}%
\pgfpathlineto{\pgfqpoint{0.926599in}{0.742760in}}%
\pgfpathlineto{\pgfqpoint{0.926981in}{0.742760in}}%
\pgfpathlineto{\pgfqpoint{0.926981in}{0.695722in}}%
\pgfpathlineto{\pgfqpoint{0.928130in}{0.768893in}}%
\pgfpathlineto{\pgfqpoint{0.928513in}{0.763666in}}%
\pgfpathlineto{\pgfqpoint{0.928896in}{0.763666in}}%
\pgfpathlineto{\pgfqpoint{0.929278in}{0.674816in}}%
\pgfpathlineto{\pgfqpoint{0.930427in}{0.706175in}}%
\pgfpathlineto{\pgfqpoint{0.930810in}{0.706175in}}%
\pgfpathlineto{\pgfqpoint{0.930810in}{0.742760in}}%
\pgfpathlineto{\pgfqpoint{0.932341in}{0.742760in}}%
\pgfpathlineto{\pgfqpoint{0.932724in}{0.742760in}}%
\pgfpathlineto{\pgfqpoint{0.932724in}{0.690496in}}%
\pgfpathlineto{\pgfqpoint{0.934255in}{0.716628in}}%
\pgfpathlineto{\pgfqpoint{0.934638in}{0.716628in}}%
\pgfpathlineto{\pgfqpoint{0.935404in}{0.784572in}}%
\pgfpathlineto{\pgfqpoint{0.936169in}{0.706175in}}%
\pgfpathlineto{\pgfqpoint{0.936552in}{0.706175in}}%
\pgfpathlineto{\pgfqpoint{0.936935in}{0.779346in}}%
\pgfpathlineto{\pgfqpoint{0.937318in}{0.674816in}}%
\pgfpathlineto{\pgfqpoint{0.938083in}{0.737534in}}%
\pgfpathlineto{\pgfqpoint{0.938466in}{0.737534in}}%
\pgfpathlineto{\pgfqpoint{0.938466in}{0.716628in}}%
\pgfpathlineto{\pgfqpoint{0.938849in}{0.774119in}}%
\pgfpathlineto{\pgfqpoint{0.939998in}{0.716628in}}%
\pgfpathlineto{\pgfqpoint{0.940380in}{0.716628in}}%
\pgfpathlineto{\pgfqpoint{0.941146in}{0.747987in}}%
\pgfpathlineto{\pgfqpoint{0.940763in}{0.711401in}}%
\pgfpathlineto{\pgfqpoint{0.941912in}{0.732307in}}%
\pgfpathlineto{\pgfqpoint{0.942295in}{0.732307in}}%
\pgfpathlineto{\pgfqpoint{0.943443in}{0.685269in}}%
\pgfpathlineto{\pgfqpoint{0.943826in}{0.706175in}}%
\pgfpathlineto{\pgfqpoint{0.944209in}{0.706175in}}%
\pgfpathlineto{\pgfqpoint{0.944209in}{0.774119in}}%
\pgfpathlineto{\pgfqpoint{0.945740in}{0.664363in}}%
\pgfpathlineto{\pgfqpoint{0.946123in}{0.664363in}}%
\pgfpathlineto{\pgfqpoint{0.947654in}{0.747987in}}%
\pgfpathlineto{\pgfqpoint{0.948037in}{0.747987in}}%
\pgfpathlineto{\pgfqpoint{0.948803in}{0.690496in}}%
\pgfpathlineto{\pgfqpoint{0.949568in}{0.774119in}}%
\pgfpathlineto{\pgfqpoint{0.949951in}{0.774119in}}%
\pgfpathlineto{\pgfqpoint{0.950334in}{0.685269in}}%
\pgfpathlineto{\pgfqpoint{0.950717in}{0.800252in}}%
\pgfpathlineto{\pgfqpoint{0.951483in}{0.716628in}}%
\pgfpathlineto{\pgfqpoint{0.951865in}{0.716628in}}%
\pgfpathlineto{\pgfqpoint{0.951865in}{0.695722in}}%
\pgfpathlineto{\pgfqpoint{0.952248in}{0.742760in}}%
\pgfpathlineto{\pgfqpoint{0.953397in}{0.711401in}}%
\pgfpathlineto{\pgfqpoint{0.953780in}{0.711401in}}%
\pgfpathlineto{\pgfqpoint{0.953780in}{0.690496in}}%
\pgfpathlineto{\pgfqpoint{0.954162in}{0.768893in}}%
\pgfpathlineto{\pgfqpoint{0.955311in}{0.711401in}}%
\pgfpathlineto{\pgfqpoint{0.956077in}{0.711401in}}%
\pgfpathlineto{\pgfqpoint{0.956459in}{0.784572in}}%
\pgfpathlineto{\pgfqpoint{0.957608in}{0.700949in}}%
\pgfpathlineto{\pgfqpoint{0.957991in}{0.700949in}}%
\pgfpathlineto{\pgfqpoint{0.957991in}{0.747987in}}%
\pgfpathlineto{\pgfqpoint{0.958373in}{0.690496in}}%
\pgfpathlineto{\pgfqpoint{0.959522in}{0.706175in}}%
\pgfpathlineto{\pgfqpoint{0.959905in}{0.706175in}}%
\pgfpathlineto{\pgfqpoint{0.960288in}{0.758440in}}%
\pgfpathlineto{\pgfqpoint{0.961436in}{0.727081in}}%
\pgfpathlineto{\pgfqpoint{0.961819in}{0.727081in}}%
\pgfpathlineto{\pgfqpoint{0.962967in}{0.747987in}}%
\pgfpathlineto{\pgfqpoint{0.963350in}{0.727081in}}%
\pgfpathlineto{\pgfqpoint{0.963733in}{0.727081in}}%
\pgfpathlineto{\pgfqpoint{0.964116in}{0.774119in}}%
\pgfpathlineto{\pgfqpoint{0.965264in}{0.711401in}}%
\pgfpathlineto{\pgfqpoint{0.965647in}{0.711401in}}%
\pgfpathlineto{\pgfqpoint{0.967179in}{0.742760in}}%
\pgfpathlineto{\pgfqpoint{0.967561in}{0.742760in}}%
\pgfpathlineto{\pgfqpoint{0.967561in}{0.706175in}}%
\pgfpathlineto{\pgfqpoint{0.969093in}{0.753213in}}%
\pgfpathlineto{\pgfqpoint{0.969476in}{0.753213in}}%
\pgfpathlineto{\pgfqpoint{0.969476in}{0.747987in}}%
\pgfpathlineto{\pgfqpoint{0.969858in}{0.795025in}}%
\pgfpathlineto{\pgfqpoint{0.971007in}{0.753213in}}%
\pgfpathlineto{\pgfqpoint{0.971390in}{0.753213in}}%
\pgfpathlineto{\pgfqpoint{0.971390in}{0.763666in}}%
\pgfpathlineto{\pgfqpoint{0.972921in}{0.700949in}}%
\pgfpathlineto{\pgfqpoint{0.973304in}{0.700949in}}%
\pgfpathlineto{\pgfqpoint{0.973687in}{0.768893in}}%
\pgfpathlineto{\pgfqpoint{0.974835in}{0.721854in}}%
\pgfpathlineto{\pgfqpoint{0.975218in}{0.721854in}}%
\pgfpathlineto{\pgfqpoint{0.976366in}{0.789799in}}%
\pgfpathlineto{\pgfqpoint{0.976749in}{0.742760in}}%
\pgfpathlineto{\pgfqpoint{0.977132in}{0.742760in}}%
\pgfpathlineto{\pgfqpoint{0.977515in}{0.716628in}}%
\pgfpathlineto{\pgfqpoint{0.978281in}{0.789799in}}%
\pgfpathlineto{\pgfqpoint{0.978663in}{0.721854in}}%
\pgfpathlineto{\pgfqpoint{0.979046in}{0.721854in}}%
\pgfpathlineto{\pgfqpoint{0.979812in}{0.758440in}}%
\pgfpathlineto{\pgfqpoint{0.980578in}{0.737534in}}%
\pgfpathlineto{\pgfqpoint{0.980960in}{0.737534in}}%
\pgfpathlineto{\pgfqpoint{0.982109in}{0.721854in}}%
\pgfpathlineto{\pgfqpoint{0.982492in}{0.737534in}}%
\pgfpathlineto{\pgfqpoint{0.982875in}{0.737534in}}%
\pgfpathlineto{\pgfqpoint{0.982875in}{0.727081in}}%
\pgfpathlineto{\pgfqpoint{0.984406in}{0.774119in}}%
\pgfpathlineto{\pgfqpoint{0.984789in}{0.774119in}}%
\pgfpathlineto{\pgfqpoint{0.984789in}{0.753213in}}%
\pgfpathlineto{\pgfqpoint{0.986320in}{0.763666in}}%
\pgfpathlineto{\pgfqpoint{0.986703in}{0.763666in}}%
\pgfpathlineto{\pgfqpoint{0.986703in}{0.774119in}}%
\pgfpathlineto{\pgfqpoint{0.987851in}{0.721854in}}%
\pgfpathlineto{\pgfqpoint{0.988234in}{0.763666in}}%
\pgfpathlineto{\pgfqpoint{0.988617in}{0.763666in}}%
\pgfpathlineto{\pgfqpoint{0.989383in}{0.721854in}}%
\pgfpathlineto{\pgfqpoint{0.990148in}{0.789799in}}%
\pgfpathlineto{\pgfqpoint{0.990531in}{0.789799in}}%
\pgfpathlineto{\pgfqpoint{0.990914in}{0.737534in}}%
\pgfpathlineto{\pgfqpoint{0.992063in}{0.737534in}}%
\pgfpathlineto{\pgfqpoint{0.992445in}{0.737534in}}%
\pgfpathlineto{\pgfqpoint{0.992445in}{0.795025in}}%
\pgfpathlineto{\pgfqpoint{0.993977in}{0.779346in}}%
\pgfpathlineto{\pgfqpoint{0.994360in}{0.779346in}}%
\pgfpathlineto{\pgfqpoint{0.995125in}{0.727081in}}%
\pgfpathlineto{\pgfqpoint{0.995891in}{0.763666in}}%
\pgfpathlineto{\pgfqpoint{0.996274in}{0.763666in}}%
\pgfpathlineto{\pgfqpoint{0.996274in}{0.774119in}}%
\pgfpathlineto{\pgfqpoint{0.997422in}{0.747987in}}%
\pgfpathlineto{\pgfqpoint{0.997805in}{0.758440in}}%
\pgfpathlineto{\pgfqpoint{0.998188in}{0.758440in}}%
\pgfpathlineto{\pgfqpoint{0.999336in}{0.721854in}}%
\pgfpathlineto{\pgfqpoint{0.999719in}{0.758440in}}%
\pgfpathlineto{\pgfqpoint{1.000102in}{0.758440in}}%
\pgfpathlineto{\pgfqpoint{1.000868in}{0.763666in}}%
\pgfpathlineto{\pgfqpoint{1.001633in}{0.685269in}}%
\pgfpathlineto{\pgfqpoint{1.002016in}{0.685269in}}%
\pgfpathlineto{\pgfqpoint{1.003547in}{0.779346in}}%
\pgfpathlineto{\pgfqpoint{1.003930in}{0.779346in}}%
\pgfpathlineto{\pgfqpoint{1.005079in}{0.732307in}}%
\pgfpathlineto{\pgfqpoint{1.004696in}{0.795025in}}%
\pgfpathlineto{\pgfqpoint{1.005462in}{0.737534in}}%
\pgfpathlineto{\pgfqpoint{1.005844in}{0.737534in}}%
\pgfpathlineto{\pgfqpoint{1.006993in}{0.706175in}}%
\pgfpathlineto{\pgfqpoint{1.007376in}{0.784572in}}%
\pgfpathlineto{\pgfqpoint{1.007759in}{0.784572in}}%
\pgfpathlineto{\pgfqpoint{1.008524in}{0.727081in}}%
\pgfpathlineto{\pgfqpoint{1.008907in}{0.789799in}}%
\pgfpathlineto{\pgfqpoint{1.009290in}{0.763666in}}%
\pgfpathlineto{\pgfqpoint{1.009673in}{0.763666in}}%
\pgfpathlineto{\pgfqpoint{1.010438in}{0.711401in}}%
\pgfpathlineto{\pgfqpoint{1.011204in}{0.768893in}}%
\pgfpathlineto{\pgfqpoint{1.011587in}{0.768893in}}%
\pgfpathlineto{\pgfqpoint{1.013118in}{0.706175in}}%
\pgfpathlineto{\pgfqpoint{1.013501in}{0.706175in}}%
\pgfpathlineto{\pgfqpoint{1.013884in}{0.779346in}}%
\pgfpathlineto{\pgfqpoint{1.015032in}{0.742760in}}%
\pgfpathlineto{\pgfqpoint{1.015415in}{0.742760in}}%
\pgfpathlineto{\pgfqpoint{1.015415in}{0.727081in}}%
\pgfpathlineto{\pgfqpoint{1.016946in}{0.747987in}}%
\pgfpathlineto{\pgfqpoint{1.017329in}{0.747987in}}%
\pgfpathlineto{\pgfqpoint{1.018478in}{0.789799in}}%
\pgfpathlineto{\pgfqpoint{1.018861in}{0.716628in}}%
\pgfpathlineto{\pgfqpoint{1.019243in}{0.716628in}}%
\pgfpathlineto{\pgfqpoint{1.020009in}{0.758440in}}%
\pgfpathlineto{\pgfqpoint{1.020775in}{0.758440in}}%
\pgfpathlineto{\pgfqpoint{1.021158in}{0.758440in}}%
\pgfpathlineto{\pgfqpoint{1.021158in}{0.711401in}}%
\pgfpathlineto{\pgfqpoint{1.021540in}{0.779346in}}%
\pgfpathlineto{\pgfqpoint{1.022689in}{0.763666in}}%
\pgfpathlineto{\pgfqpoint{1.023072in}{0.763666in}}%
\pgfpathlineto{\pgfqpoint{1.023072in}{0.711401in}}%
\pgfpathlineto{\pgfqpoint{1.024603in}{0.737534in}}%
\pgfpathlineto{\pgfqpoint{1.024986in}{0.737534in}}%
\pgfpathlineto{\pgfqpoint{1.024986in}{0.695722in}}%
\pgfpathlineto{\pgfqpoint{1.026134in}{0.747987in}}%
\pgfpathlineto{\pgfqpoint{1.026517in}{0.737534in}}%
\pgfpathlineto{\pgfqpoint{1.026900in}{0.737534in}}%
\pgfpathlineto{\pgfqpoint{1.026900in}{0.716628in}}%
\pgfpathlineto{\pgfqpoint{1.028431in}{0.789799in}}%
\pgfpathlineto{\pgfqpoint{1.028814in}{0.789799in}}%
\pgfpathlineto{\pgfqpoint{1.029197in}{0.711401in}}%
\pgfpathlineto{\pgfqpoint{1.030346in}{0.789799in}}%
\pgfpathlineto{\pgfqpoint{1.030728in}{0.789799in}}%
\pgfpathlineto{\pgfqpoint{1.032260in}{0.711401in}}%
\pgfpathlineto{\pgfqpoint{1.032643in}{0.711401in}}%
\pgfpathlineto{\pgfqpoint{1.032643in}{0.706175in}}%
\pgfpathlineto{\pgfqpoint{1.034174in}{0.784572in}}%
\pgfpathlineto{\pgfqpoint{1.034557in}{0.784572in}}%
\pgfpathlineto{\pgfqpoint{1.035322in}{0.721854in}}%
\pgfpathlineto{\pgfqpoint{1.036088in}{0.747987in}}%
\pgfpathlineto{\pgfqpoint{1.036471in}{0.747987in}}%
\pgfpathlineto{\pgfqpoint{1.036471in}{0.716628in}}%
\pgfpathlineto{\pgfqpoint{1.036854in}{0.815931in}}%
\pgfpathlineto{\pgfqpoint{1.038002in}{0.727081in}}%
\pgfpathlineto{\pgfqpoint{1.038385in}{0.727081in}}%
\pgfpathlineto{\pgfqpoint{1.038768in}{0.768893in}}%
\pgfpathlineto{\pgfqpoint{1.039151in}{0.721854in}}%
\pgfpathlineto{\pgfqpoint{1.039916in}{0.763666in}}%
\pgfpathlineto{\pgfqpoint{1.040299in}{0.763666in}}%
\pgfpathlineto{\pgfqpoint{1.041065in}{0.711401in}}%
\pgfpathlineto{\pgfqpoint{1.041830in}{0.800252in}}%
\pgfpathlineto{\pgfqpoint{1.042213in}{0.800252in}}%
\pgfpathlineto{\pgfqpoint{1.042213in}{0.721854in}}%
\pgfpathlineto{\pgfqpoint{1.043745in}{0.742760in}}%
\pgfpathlineto{\pgfqpoint{1.044510in}{0.742760in}}%
\pgfpathlineto{\pgfqpoint{1.044893in}{0.784572in}}%
\pgfpathlineto{\pgfqpoint{1.046042in}{0.742760in}}%
\pgfpathlineto{\pgfqpoint{1.046424in}{0.742760in}}%
\pgfpathlineto{\pgfqpoint{1.046424in}{0.779346in}}%
\pgfpathlineto{\pgfqpoint{1.047573in}{0.727081in}}%
\pgfpathlineto{\pgfqpoint{1.047956in}{0.732307in}}%
\pgfpathlineto{\pgfqpoint{1.048339in}{0.732307in}}%
\pgfpathlineto{\pgfqpoint{1.048721in}{0.784572in}}%
\pgfpathlineto{\pgfqpoint{1.049870in}{0.784572in}}%
\pgfpathlineto{\pgfqpoint{1.050253in}{0.784572in}}%
\pgfpathlineto{\pgfqpoint{1.051401in}{0.732307in}}%
\pgfpathlineto{\pgfqpoint{1.051784in}{0.742760in}}%
\pgfpathlineto{\pgfqpoint{1.052167in}{0.742760in}}%
\pgfpathlineto{\pgfqpoint{1.052167in}{0.774119in}}%
\pgfpathlineto{\pgfqpoint{1.053698in}{0.727081in}}%
\pgfpathlineto{\pgfqpoint{1.054081in}{0.727081in}}%
\pgfpathlineto{\pgfqpoint{1.054847in}{0.763666in}}%
\pgfpathlineto{\pgfqpoint{1.055612in}{0.747987in}}%
\pgfpathlineto{\pgfqpoint{1.055995in}{0.747987in}}%
\pgfpathlineto{\pgfqpoint{1.056761in}{0.779346in}}%
\pgfpathlineto{\pgfqpoint{1.057144in}{0.721854in}}%
\pgfpathlineto{\pgfqpoint{1.057526in}{0.737534in}}%
\pgfpathlineto{\pgfqpoint{1.057909in}{0.737534in}}%
\pgfpathlineto{\pgfqpoint{1.057909in}{0.789799in}}%
\pgfpathlineto{\pgfqpoint{1.059441in}{0.747987in}}%
\pgfpathlineto{\pgfqpoint{1.059823in}{0.747987in}}%
\pgfpathlineto{\pgfqpoint{1.060206in}{0.737534in}}%
\pgfpathlineto{\pgfqpoint{1.060589in}{0.805478in}}%
\pgfpathlineto{\pgfqpoint{1.061355in}{0.768893in}}%
\pgfpathlineto{\pgfqpoint{1.061738in}{0.768893in}}%
\pgfpathlineto{\pgfqpoint{1.062503in}{0.784572in}}%
\pgfpathlineto{\pgfqpoint{1.062886in}{0.742760in}}%
\pgfpathlineto{\pgfqpoint{1.063269in}{0.768893in}}%
\pgfpathlineto{\pgfqpoint{1.063652in}{0.768893in}}%
\pgfpathlineto{\pgfqpoint{1.065183in}{0.695722in}}%
\pgfpathlineto{\pgfqpoint{1.065566in}{0.695722in}}%
\pgfpathlineto{\pgfqpoint{1.065566in}{0.826384in}}%
\pgfpathlineto{\pgfqpoint{1.067097in}{0.747987in}}%
\pgfpathlineto{\pgfqpoint{1.067480in}{0.747987in}}%
\pgfpathlineto{\pgfqpoint{1.068629in}{0.805478in}}%
\pgfpathlineto{\pgfqpoint{1.068246in}{0.732307in}}%
\pgfpathlineto{\pgfqpoint{1.069011in}{0.742760in}}%
\pgfpathlineto{\pgfqpoint{1.069394in}{0.742760in}}%
\pgfpathlineto{\pgfqpoint{1.070543in}{0.721854in}}%
\pgfpathlineto{\pgfqpoint{1.070160in}{0.763666in}}%
\pgfpathlineto{\pgfqpoint{1.070926in}{0.727081in}}%
\pgfpathlineto{\pgfqpoint{1.071308in}{0.727081in}}%
\pgfpathlineto{\pgfqpoint{1.072074in}{0.784572in}}%
\pgfpathlineto{\pgfqpoint{1.072840in}{0.753213in}}%
\pgfpathlineto{\pgfqpoint{1.073222in}{0.753213in}}%
\pgfpathlineto{\pgfqpoint{1.073222in}{0.742760in}}%
\pgfpathlineto{\pgfqpoint{1.074754in}{0.779346in}}%
\pgfpathlineto{\pgfqpoint{1.075519in}{0.779346in}}%
\pgfpathlineto{\pgfqpoint{1.076285in}{0.742760in}}%
\pgfpathlineto{\pgfqpoint{1.075902in}{0.805478in}}%
\pgfpathlineto{\pgfqpoint{1.077051in}{0.763666in}}%
\pgfpathlineto{\pgfqpoint{1.077434in}{0.763666in}}%
\pgfpathlineto{\pgfqpoint{1.077816in}{0.706175in}}%
\pgfpathlineto{\pgfqpoint{1.078965in}{0.774119in}}%
\pgfpathlineto{\pgfqpoint{1.079348in}{0.774119in}}%
\pgfpathlineto{\pgfqpoint{1.079348in}{0.716628in}}%
\pgfpathlineto{\pgfqpoint{1.080879in}{0.789799in}}%
\pgfpathlineto{\pgfqpoint{1.081262in}{0.789799in}}%
\pgfpathlineto{\pgfqpoint{1.082793in}{0.727081in}}%
\pgfpathlineto{\pgfqpoint{1.083176in}{0.727081in}}%
\pgfpathlineto{\pgfqpoint{1.084325in}{0.795025in}}%
\pgfpathlineto{\pgfqpoint{1.084707in}{0.742760in}}%
\pgfpathlineto{\pgfqpoint{1.085090in}{0.742760in}}%
\pgfpathlineto{\pgfqpoint{1.085856in}{0.716628in}}%
\pgfpathlineto{\pgfqpoint{1.086239in}{0.779346in}}%
\pgfpathlineto{\pgfqpoint{1.086622in}{0.753213in}}%
\pgfpathlineto{\pgfqpoint{1.087004in}{0.753213in}}%
\pgfpathlineto{\pgfqpoint{1.087387in}{0.779346in}}%
\pgfpathlineto{\pgfqpoint{1.088536in}{0.737534in}}%
\pgfpathlineto{\pgfqpoint{1.088919in}{0.737534in}}%
\pgfpathlineto{\pgfqpoint{1.088919in}{0.800252in}}%
\pgfpathlineto{\pgfqpoint{1.090450in}{0.763666in}}%
\pgfpathlineto{\pgfqpoint{1.090833in}{0.763666in}}%
\pgfpathlineto{\pgfqpoint{1.091216in}{0.716628in}}%
\pgfpathlineto{\pgfqpoint{1.092364in}{0.737534in}}%
\pgfpathlineto{\pgfqpoint{1.092747in}{0.737534in}}%
\pgfpathlineto{\pgfqpoint{1.093130in}{0.810705in}}%
\pgfpathlineto{\pgfqpoint{1.093512in}{0.721854in}}%
\pgfpathlineto{\pgfqpoint{1.094278in}{0.768893in}}%
\pgfpathlineto{\pgfqpoint{1.094661in}{0.768893in}}%
\pgfpathlineto{\pgfqpoint{1.094661in}{0.737534in}}%
\pgfpathlineto{\pgfqpoint{1.095809in}{0.800252in}}%
\pgfpathlineto{\pgfqpoint{1.096192in}{0.784572in}}%
\pgfpathlineto{\pgfqpoint{1.096575in}{0.784572in}}%
\pgfpathlineto{\pgfqpoint{1.097724in}{0.747987in}}%
\pgfpathlineto{\pgfqpoint{1.097341in}{0.795025in}}%
\pgfpathlineto{\pgfqpoint{1.098106in}{0.763666in}}%
\pgfpathlineto{\pgfqpoint{1.098872in}{0.763666in}}%
\pgfpathlineto{\pgfqpoint{1.099255in}{0.727081in}}%
\pgfpathlineto{\pgfqpoint{1.100403in}{0.800252in}}%
\pgfpathlineto{\pgfqpoint{1.100786in}{0.800252in}}%
\pgfpathlineto{\pgfqpoint{1.102318in}{0.742760in}}%
\pgfpathlineto{\pgfqpoint{1.102700in}{0.742760in}}%
\pgfpathlineto{\pgfqpoint{1.103083in}{0.810705in}}%
\pgfpathlineto{\pgfqpoint{1.104232in}{0.737534in}}%
\pgfpathlineto{\pgfqpoint{1.104615in}{0.737534in}}%
\pgfpathlineto{\pgfqpoint{1.104615in}{0.800252in}}%
\pgfpathlineto{\pgfqpoint{1.106146in}{0.768893in}}%
\pgfpathlineto{\pgfqpoint{1.106529in}{0.768893in}}%
\pgfpathlineto{\pgfqpoint{1.107677in}{0.831611in}}%
\pgfpathlineto{\pgfqpoint{1.106912in}{0.742760in}}%
\pgfpathlineto{\pgfqpoint{1.108060in}{0.774119in}}%
\pgfpathlineto{\pgfqpoint{1.108443in}{0.774119in}}%
\pgfpathlineto{\pgfqpoint{1.109209in}{0.795025in}}%
\pgfpathlineto{\pgfqpoint{1.109974in}{0.737534in}}%
\pgfpathlineto{\pgfqpoint{1.110357in}{0.737534in}}%
\pgfpathlineto{\pgfqpoint{1.110740in}{0.795025in}}%
\pgfpathlineto{\pgfqpoint{1.111888in}{0.774119in}}%
\pgfpathlineto{\pgfqpoint{1.112271in}{0.774119in}}%
\pgfpathlineto{\pgfqpoint{1.113037in}{0.753213in}}%
\pgfpathlineto{\pgfqpoint{1.112654in}{0.821158in}}%
\pgfpathlineto{\pgfqpoint{1.113802in}{0.753213in}}%
\pgfpathlineto{\pgfqpoint{1.114185in}{0.753213in}}%
\pgfpathlineto{\pgfqpoint{1.114568in}{0.805478in}}%
\pgfpathlineto{\pgfqpoint{1.115717in}{0.706175in}}%
\pgfpathlineto{\pgfqpoint{1.116099in}{0.706175in}}%
\pgfpathlineto{\pgfqpoint{1.116865in}{0.826384in}}%
\pgfpathlineto{\pgfqpoint{1.117631in}{0.742760in}}%
\pgfpathlineto{\pgfqpoint{1.118014in}{0.742760in}}%
\pgfpathlineto{\pgfqpoint{1.118396in}{0.758440in}}%
\pgfpathlineto{\pgfqpoint{1.119545in}{0.737534in}}%
\pgfpathlineto{\pgfqpoint{1.119928in}{0.737534in}}%
\pgfpathlineto{\pgfqpoint{1.120311in}{0.732307in}}%
\pgfpathlineto{\pgfqpoint{1.121459in}{0.789799in}}%
\pgfpathlineto{\pgfqpoint{1.121842in}{0.789799in}}%
\pgfpathlineto{\pgfqpoint{1.123373in}{0.737534in}}%
\pgfpathlineto{\pgfqpoint{1.123756in}{0.737534in}}%
\pgfpathlineto{\pgfqpoint{1.124522in}{0.800252in}}%
\pgfpathlineto{\pgfqpoint{1.125287in}{0.758440in}}%
\pgfpathlineto{\pgfqpoint{1.125670in}{0.758440in}}%
\pgfpathlineto{\pgfqpoint{1.125670in}{0.753213in}}%
\pgfpathlineto{\pgfqpoint{1.127202in}{0.800252in}}%
\pgfpathlineto{\pgfqpoint{1.127584in}{0.800252in}}%
\pgfpathlineto{\pgfqpoint{1.128733in}{0.732307in}}%
\pgfpathlineto{\pgfqpoint{1.129116in}{0.758440in}}%
\pgfpathlineto{\pgfqpoint{1.129499in}{0.758440in}}%
\pgfpathlineto{\pgfqpoint{1.129881in}{0.727081in}}%
\pgfpathlineto{\pgfqpoint{1.131030in}{0.784572in}}%
\pgfpathlineto{\pgfqpoint{1.131413in}{0.784572in}}%
\pgfpathlineto{\pgfqpoint{1.131795in}{0.810705in}}%
\pgfpathlineto{\pgfqpoint{1.132561in}{0.763666in}}%
\pgfpathlineto{\pgfqpoint{1.132944in}{0.774119in}}%
\pgfpathlineto{\pgfqpoint{1.133327in}{0.774119in}}%
\pgfpathlineto{\pgfqpoint{1.133327in}{0.842063in}}%
\pgfpathlineto{\pgfqpoint{1.133710in}{0.742760in}}%
\pgfpathlineto{\pgfqpoint{1.134858in}{0.753213in}}%
\pgfpathlineto{\pgfqpoint{1.135241in}{0.753213in}}%
\pgfpathlineto{\pgfqpoint{1.135624in}{0.826384in}}%
\pgfpathlineto{\pgfqpoint{1.135624in}{0.732307in}}%
\pgfpathlineto{\pgfqpoint{1.136772in}{0.737534in}}%
\pgfpathlineto{\pgfqpoint{1.137155in}{0.737534in}}%
\pgfpathlineto{\pgfqpoint{1.137921in}{0.784572in}}%
\pgfpathlineto{\pgfqpoint{1.137538in}{0.727081in}}%
\pgfpathlineto{\pgfqpoint{1.138686in}{0.742760in}}%
\pgfpathlineto{\pgfqpoint{1.139069in}{0.742760in}}%
\pgfpathlineto{\pgfqpoint{1.139835in}{0.805478in}}%
\pgfpathlineto{\pgfqpoint{1.140601in}{0.800252in}}%
\pgfpathlineto{\pgfqpoint{1.140983in}{0.800252in}}%
\pgfpathlineto{\pgfqpoint{1.140983in}{0.768893in}}%
\pgfpathlineto{\pgfqpoint{1.142515in}{0.768893in}}%
\pgfpathlineto{\pgfqpoint{1.142898in}{0.768893in}}%
\pgfpathlineto{\pgfqpoint{1.143663in}{0.753213in}}%
\pgfpathlineto{\pgfqpoint{1.144046in}{0.810705in}}%
\pgfpathlineto{\pgfqpoint{1.144429in}{0.795025in}}%
\pgfpathlineto{\pgfqpoint{1.144812in}{0.795025in}}%
\pgfpathlineto{\pgfqpoint{1.146343in}{0.758440in}}%
\pgfpathlineto{\pgfqpoint{1.146726in}{0.758440in}}%
\pgfpathlineto{\pgfqpoint{1.147109in}{0.789799in}}%
\pgfpathlineto{\pgfqpoint{1.147874in}{0.727081in}}%
\pgfpathlineto{\pgfqpoint{1.148257in}{0.753213in}}%
\pgfpathlineto{\pgfqpoint{1.148640in}{0.753213in}}%
\pgfpathlineto{\pgfqpoint{1.149789in}{0.842063in}}%
\pgfpathlineto{\pgfqpoint{1.149023in}{0.747987in}}%
\pgfpathlineto{\pgfqpoint{1.150171in}{0.768893in}}%
\pgfpathlineto{\pgfqpoint{1.150554in}{0.768893in}}%
\pgfpathlineto{\pgfqpoint{1.150554in}{0.795025in}}%
\pgfpathlineto{\pgfqpoint{1.151320in}{0.758440in}}%
\pgfpathlineto{\pgfqpoint{1.152085in}{0.768893in}}%
\pgfpathlineto{\pgfqpoint{1.152468in}{0.768893in}}%
\pgfpathlineto{\pgfqpoint{1.153617in}{0.727081in}}%
\pgfpathlineto{\pgfqpoint{1.154000in}{0.805478in}}%
\pgfpathlineto{\pgfqpoint{1.154382in}{0.805478in}}%
\pgfpathlineto{\pgfqpoint{1.155148in}{0.774119in}}%
\pgfpathlineto{\pgfqpoint{1.155914in}{0.810705in}}%
\pgfpathlineto{\pgfqpoint{1.156297in}{0.810705in}}%
\pgfpathlineto{\pgfqpoint{1.156679in}{0.747987in}}%
\pgfpathlineto{\pgfqpoint{1.157828in}{0.758440in}}%
\pgfpathlineto{\pgfqpoint{1.158211in}{0.758440in}}%
\pgfpathlineto{\pgfqpoint{1.159742in}{0.805478in}}%
\pgfpathlineto{\pgfqpoint{1.160125in}{0.805478in}}%
\pgfpathlineto{\pgfqpoint{1.161273in}{0.732307in}}%
\pgfpathlineto{\pgfqpoint{1.161656in}{0.779346in}}%
\pgfpathlineto{\pgfqpoint{1.162039in}{0.779346in}}%
\pgfpathlineto{\pgfqpoint{1.162039in}{0.826384in}}%
\pgfpathlineto{\pgfqpoint{1.163570in}{0.747987in}}%
\pgfpathlineto{\pgfqpoint{1.163953in}{0.747987in}}%
\pgfpathlineto{\pgfqpoint{1.163953in}{0.815931in}}%
\pgfpathlineto{\pgfqpoint{1.165485in}{0.758440in}}%
\pgfpathlineto{\pgfqpoint{1.165867in}{0.758440in}}%
\pgfpathlineto{\pgfqpoint{1.165867in}{0.842063in}}%
\pgfpathlineto{\pgfqpoint{1.167016in}{0.753213in}}%
\pgfpathlineto{\pgfqpoint{1.167399in}{0.758440in}}%
\pgfpathlineto{\pgfqpoint{1.167782in}{0.758440in}}%
\pgfpathlineto{\pgfqpoint{1.168164in}{0.847290in}}%
\pgfpathlineto{\pgfqpoint{1.169313in}{0.836837in}}%
\pgfpathlineto{\pgfqpoint{1.169696in}{0.836837in}}%
\pgfpathlineto{\pgfqpoint{1.169696in}{0.758440in}}%
\pgfpathlineto{\pgfqpoint{1.171227in}{0.795025in}}%
\pgfpathlineto{\pgfqpoint{1.171610in}{0.795025in}}%
\pgfpathlineto{\pgfqpoint{1.171610in}{0.815931in}}%
\pgfpathlineto{\pgfqpoint{1.172758in}{0.774119in}}%
\pgfpathlineto{\pgfqpoint{1.173141in}{0.774119in}}%
\pgfpathlineto{\pgfqpoint{1.173524in}{0.774119in}}%
\pgfpathlineto{\pgfqpoint{1.174672in}{0.815931in}}%
\pgfpathlineto{\pgfqpoint{1.175055in}{0.700949in}}%
\pgfpathlineto{\pgfqpoint{1.175438in}{0.700949in}}%
\pgfpathlineto{\pgfqpoint{1.175821in}{0.789799in}}%
\pgfpathlineto{\pgfqpoint{1.176969in}{0.737534in}}%
\pgfpathlineto{\pgfqpoint{1.177352in}{0.737534in}}%
\pgfpathlineto{\pgfqpoint{1.177352in}{0.795025in}}%
\pgfpathlineto{\pgfqpoint{1.178884in}{0.774119in}}%
\pgfpathlineto{\pgfqpoint{1.179266in}{0.774119in}}%
\pgfpathlineto{\pgfqpoint{1.179649in}{0.810705in}}%
\pgfpathlineto{\pgfqpoint{1.180798in}{0.732307in}}%
\pgfpathlineto{\pgfqpoint{1.181181in}{0.732307in}}%
\pgfpathlineto{\pgfqpoint{1.182712in}{0.842063in}}%
\pgfpathlineto{\pgfqpoint{1.183095in}{0.842063in}}%
\pgfpathlineto{\pgfqpoint{1.184243in}{0.779346in}}%
\pgfpathlineto{\pgfqpoint{1.184626in}{0.821158in}}%
\pgfpathlineto{\pgfqpoint{1.185009in}{0.821158in}}%
\pgfpathlineto{\pgfqpoint{1.185392in}{0.747987in}}%
\pgfpathlineto{\pgfqpoint{1.185392in}{0.852516in}}%
\pgfpathlineto{\pgfqpoint{1.186540in}{0.831611in}}%
\pgfpathlineto{\pgfqpoint{1.186923in}{0.831611in}}%
\pgfpathlineto{\pgfqpoint{1.187306in}{0.758440in}}%
\pgfpathlineto{\pgfqpoint{1.188454in}{0.768893in}}%
\pgfpathlineto{\pgfqpoint{1.188837in}{0.768893in}}%
\pgfpathlineto{\pgfqpoint{1.189986in}{0.805478in}}%
\pgfpathlineto{\pgfqpoint{1.190368in}{0.795025in}}%
\pgfpathlineto{\pgfqpoint{1.190751in}{0.795025in}}%
\pgfpathlineto{\pgfqpoint{1.191900in}{0.821158in}}%
\pgfpathlineto{\pgfqpoint{1.191517in}{0.727081in}}%
\pgfpathlineto{\pgfqpoint{1.192283in}{0.795025in}}%
\pgfpathlineto{\pgfqpoint{1.192665in}{0.795025in}}%
\pgfpathlineto{\pgfqpoint{1.192665in}{0.753213in}}%
\pgfpathlineto{\pgfqpoint{1.194197in}{0.768893in}}%
\pgfpathlineto{\pgfqpoint{1.194580in}{0.768893in}}%
\pgfpathlineto{\pgfqpoint{1.194962in}{0.868196in}}%
\pgfpathlineto{\pgfqpoint{1.196111in}{0.768893in}}%
\pgfpathlineto{\pgfqpoint{1.196494in}{0.768893in}}%
\pgfpathlineto{\pgfqpoint{1.196877in}{0.747987in}}%
\pgfpathlineto{\pgfqpoint{1.197642in}{0.826384in}}%
\pgfpathlineto{\pgfqpoint{1.198025in}{0.821158in}}%
\pgfpathlineto{\pgfqpoint{1.198408in}{0.821158in}}%
\pgfpathlineto{\pgfqpoint{1.199556in}{0.800252in}}%
\pgfpathlineto{\pgfqpoint{1.199939in}{0.899555in}}%
\pgfpathlineto{\pgfqpoint{1.200322in}{0.899555in}}%
\pgfpathlineto{\pgfqpoint{1.201471in}{0.795025in}}%
\pgfpathlineto{\pgfqpoint{1.201853in}{0.821158in}}%
\pgfpathlineto{\pgfqpoint{1.202236in}{0.821158in}}%
\pgfpathlineto{\pgfqpoint{1.203002in}{0.763666in}}%
\pgfpathlineto{\pgfqpoint{1.202619in}{0.831611in}}%
\pgfpathlineto{\pgfqpoint{1.203768in}{0.779346in}}%
\pgfpathlineto{\pgfqpoint{1.204150in}{0.779346in}}%
\pgfpathlineto{\pgfqpoint{1.204150in}{0.883875in}}%
\pgfpathlineto{\pgfqpoint{1.204533in}{0.742760in}}%
\pgfpathlineto{\pgfqpoint{1.205682in}{0.852516in}}%
\pgfpathlineto{\pgfqpoint{1.206065in}{0.852516in}}%
\pgfpathlineto{\pgfqpoint{1.207213in}{0.795025in}}%
\pgfpathlineto{\pgfqpoint{1.206447in}{0.873422in}}%
\pgfpathlineto{\pgfqpoint{1.207596in}{0.805478in}}%
\pgfpathlineto{\pgfqpoint{1.207979in}{0.805478in}}%
\pgfpathlineto{\pgfqpoint{1.207979in}{0.836837in}}%
\pgfpathlineto{\pgfqpoint{1.209510in}{0.763666in}}%
\pgfpathlineto{\pgfqpoint{1.209893in}{0.763666in}}%
\pgfpathlineto{\pgfqpoint{1.210658in}{0.842063in}}%
\pgfpathlineto{\pgfqpoint{1.211424in}{0.789799in}}%
\pgfpathlineto{\pgfqpoint{1.211807in}{0.789799in}}%
\pgfpathlineto{\pgfqpoint{1.212190in}{0.815931in}}%
\pgfpathlineto{\pgfqpoint{1.213338in}{0.768893in}}%
\pgfpathlineto{\pgfqpoint{1.213721in}{0.768893in}}%
\pgfpathlineto{\pgfqpoint{1.213721in}{0.826384in}}%
\pgfpathlineto{\pgfqpoint{1.215252in}{0.779346in}}%
\pgfpathlineto{\pgfqpoint{1.215635in}{0.779346in}}%
\pgfpathlineto{\pgfqpoint{1.216018in}{0.862969in}}%
\pgfpathlineto{\pgfqpoint{1.217167in}{0.800252in}}%
\pgfpathlineto{\pgfqpoint{1.217549in}{0.800252in}}%
\pgfpathlineto{\pgfqpoint{1.217549in}{0.842063in}}%
\pgfpathlineto{\pgfqpoint{1.219081in}{0.826384in}}%
\pgfpathlineto{\pgfqpoint{1.219464in}{0.826384in}}%
\pgfpathlineto{\pgfqpoint{1.220229in}{0.784572in}}%
\pgfpathlineto{\pgfqpoint{1.220612in}{0.852516in}}%
\pgfpathlineto{\pgfqpoint{1.220995in}{0.821158in}}%
\pgfpathlineto{\pgfqpoint{1.221378in}{0.821158in}}%
\pgfpathlineto{\pgfqpoint{1.221378in}{0.852516in}}%
\pgfpathlineto{\pgfqpoint{1.222909in}{0.779346in}}%
\pgfpathlineto{\pgfqpoint{1.223292in}{0.779346in}}%
\pgfpathlineto{\pgfqpoint{1.224440in}{0.821158in}}%
\pgfpathlineto{\pgfqpoint{1.223675in}{0.742760in}}%
\pgfpathlineto{\pgfqpoint{1.224823in}{0.805478in}}%
\pgfpathlineto{\pgfqpoint{1.225206in}{0.805478in}}%
\pgfpathlineto{\pgfqpoint{1.225589in}{0.862969in}}%
\pgfpathlineto{\pgfqpoint{1.226737in}{0.774119in}}%
\pgfpathlineto{\pgfqpoint{1.227120in}{0.774119in}}%
\pgfpathlineto{\pgfqpoint{1.227503in}{0.852516in}}%
\pgfpathlineto{\pgfqpoint{1.228651in}{0.821158in}}%
\pgfpathlineto{\pgfqpoint{1.229034in}{0.821158in}}%
\pgfpathlineto{\pgfqpoint{1.229034in}{0.862969in}}%
\pgfpathlineto{\pgfqpoint{1.230566in}{0.789799in}}%
\pgfpathlineto{\pgfqpoint{1.230948in}{0.789799in}}%
\pgfpathlineto{\pgfqpoint{1.232097in}{0.878649in}}%
\pgfpathlineto{\pgfqpoint{1.232480in}{0.852516in}}%
\pgfpathlineto{\pgfqpoint{1.232863in}{0.852516in}}%
\pgfpathlineto{\pgfqpoint{1.233245in}{0.821158in}}%
\pgfpathlineto{\pgfqpoint{1.234394in}{0.910008in}}%
\pgfpathlineto{\pgfqpoint{1.234777in}{0.910008in}}%
\pgfpathlineto{\pgfqpoint{1.235160in}{0.800252in}}%
\pgfpathlineto{\pgfqpoint{1.236308in}{0.800252in}}%
\pgfpathlineto{\pgfqpoint{1.236691in}{0.800252in}}%
\pgfpathlineto{\pgfqpoint{1.237457in}{0.789799in}}%
\pgfpathlineto{\pgfqpoint{1.237839in}{0.836837in}}%
\pgfpathlineto{\pgfqpoint{1.238222in}{0.831611in}}%
\pgfpathlineto{\pgfqpoint{1.238605in}{0.831611in}}%
\pgfpathlineto{\pgfqpoint{1.238988in}{0.915234in}}%
\pgfpathlineto{\pgfqpoint{1.239371in}{0.810705in}}%
\pgfpathlineto{\pgfqpoint{1.240136in}{0.889102in}}%
\pgfpathlineto{\pgfqpoint{1.240519in}{0.889102in}}%
\pgfpathlineto{\pgfqpoint{1.240519in}{0.779346in}}%
\pgfpathlineto{\pgfqpoint{1.242051in}{0.857743in}}%
\pgfpathlineto{\pgfqpoint{1.242433in}{0.857743in}}%
\pgfpathlineto{\pgfqpoint{1.243582in}{0.826384in}}%
\pgfpathlineto{\pgfqpoint{1.243199in}{0.862969in}}%
\pgfpathlineto{\pgfqpoint{1.243965in}{0.842063in}}%
\pgfpathlineto{\pgfqpoint{1.244348in}{0.842063in}}%
\pgfpathlineto{\pgfqpoint{1.245496in}{0.910008in}}%
\pgfpathlineto{\pgfqpoint{1.245879in}{0.810705in}}%
\pgfpathlineto{\pgfqpoint{1.246262in}{0.810705in}}%
\pgfpathlineto{\pgfqpoint{1.247793in}{0.899555in}}%
\pgfpathlineto{\pgfqpoint{1.248176in}{0.899555in}}%
\pgfpathlineto{\pgfqpoint{1.248941in}{0.857743in}}%
\pgfpathlineto{\pgfqpoint{1.249707in}{0.899555in}}%
\pgfpathlineto{\pgfqpoint{1.250090in}{0.899555in}}%
\pgfpathlineto{\pgfqpoint{1.250090in}{0.936140in}}%
\pgfpathlineto{\pgfqpoint{1.250473in}{0.810705in}}%
\pgfpathlineto{\pgfqpoint{1.251621in}{0.852516in}}%
\pgfpathlineto{\pgfqpoint{1.252004in}{0.852516in}}%
\pgfpathlineto{\pgfqpoint{1.252387in}{0.842063in}}%
\pgfpathlineto{\pgfqpoint{1.253535in}{0.889102in}}%
\pgfpathlineto{\pgfqpoint{1.253918in}{0.889102in}}%
\pgfpathlineto{\pgfqpoint{1.253918in}{0.831611in}}%
\pgfpathlineto{\pgfqpoint{1.255450in}{0.920461in}}%
\pgfpathlineto{\pgfqpoint{1.255832in}{0.920461in}}%
\pgfpathlineto{\pgfqpoint{1.255832in}{0.836837in}}%
\pgfpathlineto{\pgfqpoint{1.257364in}{0.962273in}}%
\pgfpathlineto{\pgfqpoint{1.257747in}{0.962273in}}%
\pgfpathlineto{\pgfqpoint{1.259278in}{0.826384in}}%
\pgfpathlineto{\pgfqpoint{1.259661in}{0.826384in}}%
\pgfpathlineto{\pgfqpoint{1.259661in}{0.894328in}}%
\pgfpathlineto{\pgfqpoint{1.261192in}{0.868196in}}%
\pgfpathlineto{\pgfqpoint{1.261575in}{0.868196in}}%
\pgfpathlineto{\pgfqpoint{1.261575in}{0.852516in}}%
\pgfpathlineto{\pgfqpoint{1.262723in}{0.915234in}}%
\pgfpathlineto{\pgfqpoint{1.263106in}{0.894328in}}%
\pgfpathlineto{\pgfqpoint{1.263489in}{0.894328in}}%
\pgfpathlineto{\pgfqpoint{1.264255in}{0.826384in}}%
\pgfpathlineto{\pgfqpoint{1.265020in}{0.862969in}}%
\pgfpathlineto{\pgfqpoint{1.265403in}{0.862969in}}%
\pgfpathlineto{\pgfqpoint{1.266934in}{0.910008in}}%
\pgfpathlineto{\pgfqpoint{1.267317in}{0.910008in}}%
\pgfpathlineto{\pgfqpoint{1.268466in}{0.805478in}}%
\pgfpathlineto{\pgfqpoint{1.267700in}{0.930914in}}%
\pgfpathlineto{\pgfqpoint{1.268849in}{0.847290in}}%
\pgfpathlineto{\pgfqpoint{1.269231in}{0.847290in}}%
\pgfpathlineto{\pgfqpoint{1.269614in}{0.826384in}}%
\pgfpathlineto{\pgfqpoint{1.270763in}{0.930914in}}%
\pgfpathlineto{\pgfqpoint{1.271146in}{0.930914in}}%
\pgfpathlineto{\pgfqpoint{1.271911in}{0.842063in}}%
\pgfpathlineto{\pgfqpoint{1.272677in}{0.873422in}}%
\pgfpathlineto{\pgfqpoint{1.273060in}{0.873422in}}%
\pgfpathlineto{\pgfqpoint{1.273060in}{0.894328in}}%
\pgfpathlineto{\pgfqpoint{1.273825in}{0.836837in}}%
\pgfpathlineto{\pgfqpoint{1.274591in}{0.852516in}}%
\pgfpathlineto{\pgfqpoint{1.274974in}{0.852516in}}%
\pgfpathlineto{\pgfqpoint{1.276122in}{0.915234in}}%
\pgfpathlineto{\pgfqpoint{1.276505in}{0.889102in}}%
\pgfpathlineto{\pgfqpoint{1.276888in}{0.889102in}}%
\pgfpathlineto{\pgfqpoint{1.276888in}{0.941367in}}%
\pgfpathlineto{\pgfqpoint{1.277271in}{0.800252in}}%
\pgfpathlineto{\pgfqpoint{1.278419in}{0.868196in}}%
\pgfpathlineto{\pgfqpoint{1.278802in}{0.868196in}}%
\pgfpathlineto{\pgfqpoint{1.278802in}{0.962273in}}%
\pgfpathlineto{\pgfqpoint{1.280334in}{0.821158in}}%
\pgfpathlineto{\pgfqpoint{1.280716in}{0.821158in}}%
\pgfpathlineto{\pgfqpoint{1.281482in}{0.946593in}}%
\pgfpathlineto{\pgfqpoint{1.282248in}{0.920461in}}%
\pgfpathlineto{\pgfqpoint{1.282631in}{0.920461in}}%
\pgfpathlineto{\pgfqpoint{1.282631in}{0.925687in}}%
\pgfpathlineto{\pgfqpoint{1.283779in}{0.826384in}}%
\pgfpathlineto{\pgfqpoint{1.284162in}{0.925687in}}%
\pgfpathlineto{\pgfqpoint{1.284545in}{0.925687in}}%
\pgfpathlineto{\pgfqpoint{1.284545in}{1.024990in}}%
\pgfpathlineto{\pgfqpoint{1.285693in}{0.920461in}}%
\pgfpathlineto{\pgfqpoint{1.286076in}{0.936140in}}%
\pgfpathlineto{\pgfqpoint{1.286459in}{0.936140in}}%
\pgfpathlineto{\pgfqpoint{1.286842in}{0.889102in}}%
\pgfpathlineto{\pgfqpoint{1.287607in}{0.967499in}}%
\pgfpathlineto{\pgfqpoint{1.287990in}{0.915234in}}%
\pgfpathlineto{\pgfqpoint{1.288373in}{0.915234in}}%
\pgfpathlineto{\pgfqpoint{1.289904in}{0.836837in}}%
\pgfpathlineto{\pgfqpoint{1.290287in}{0.836837in}}%
\pgfpathlineto{\pgfqpoint{1.290670in}{0.941367in}}%
\pgfpathlineto{\pgfqpoint{1.291818in}{0.883875in}}%
\pgfpathlineto{\pgfqpoint{1.292201in}{0.883875in}}%
\pgfpathlineto{\pgfqpoint{1.292201in}{0.857743in}}%
\pgfpathlineto{\pgfqpoint{1.292967in}{0.977952in}}%
\pgfpathlineto{\pgfqpoint{1.293733in}{0.889102in}}%
\pgfpathlineto{\pgfqpoint{1.294115in}{0.889102in}}%
\pgfpathlineto{\pgfqpoint{1.294115in}{0.857743in}}%
\pgfpathlineto{\pgfqpoint{1.294881in}{0.930914in}}%
\pgfpathlineto{\pgfqpoint{1.295647in}{0.868196in}}%
\pgfpathlineto{\pgfqpoint{1.296030in}{0.868196in}}%
\pgfpathlineto{\pgfqpoint{1.296412in}{0.972726in}}%
\pgfpathlineto{\pgfqpoint{1.297561in}{0.868196in}}%
\pgfpathlineto{\pgfqpoint{1.297944in}{0.868196in}}%
\pgfpathlineto{\pgfqpoint{1.299475in}{0.977952in}}%
\pgfpathlineto{\pgfqpoint{1.300241in}{0.977952in}}%
\pgfpathlineto{\pgfqpoint{1.301006in}{0.993631in}}%
\pgfpathlineto{\pgfqpoint{1.301772in}{0.889102in}}%
\pgfpathlineto{\pgfqpoint{1.302155in}{0.889102in}}%
\pgfpathlineto{\pgfqpoint{1.302538in}{1.014537in}}%
\pgfpathlineto{\pgfqpoint{1.303686in}{0.889102in}}%
\pgfpathlineto{\pgfqpoint{1.304069in}{0.889102in}}%
\pgfpathlineto{\pgfqpoint{1.305217in}{0.993631in}}%
\pgfpathlineto{\pgfqpoint{1.305600in}{0.925687in}}%
\pgfpathlineto{\pgfqpoint{1.305983in}{0.925687in}}%
\pgfpathlineto{\pgfqpoint{1.307132in}{0.904781in}}%
\pgfpathlineto{\pgfqpoint{1.307514in}{1.024990in}}%
\pgfpathlineto{\pgfqpoint{1.307897in}{1.024990in}}%
\pgfpathlineto{\pgfqpoint{1.309429in}{0.925687in}}%
\pgfpathlineto{\pgfqpoint{1.309811in}{0.925687in}}%
\pgfpathlineto{\pgfqpoint{1.309811in}{1.040670in}}%
\pgfpathlineto{\pgfqpoint{1.311343in}{1.040670in}}%
\pgfpathlineto{\pgfqpoint{1.311726in}{1.040670in}}%
\pgfpathlineto{\pgfqpoint{1.312491in}{0.873422in}}%
\pgfpathlineto{\pgfqpoint{1.313257in}{0.988405in}}%
\pgfpathlineto{\pgfqpoint{1.313640in}{0.988405in}}%
\pgfpathlineto{\pgfqpoint{1.314405in}{0.899555in}}%
\pgfpathlineto{\pgfqpoint{1.314023in}{1.024990in}}%
\pgfpathlineto{\pgfqpoint{1.315171in}{0.957046in}}%
\pgfpathlineto{\pgfqpoint{1.315554in}{0.957046in}}%
\pgfpathlineto{\pgfqpoint{1.315554in}{1.024990in}}%
\pgfpathlineto{\pgfqpoint{1.316320in}{0.915234in}}%
\pgfpathlineto{\pgfqpoint{1.317085in}{0.941367in}}%
\pgfpathlineto{\pgfqpoint{1.317468in}{0.941367in}}%
\pgfpathlineto{\pgfqpoint{1.317468in}{1.024990in}}%
\pgfpathlineto{\pgfqpoint{1.318999in}{1.024990in}}%
\pgfpathlineto{\pgfqpoint{1.319382in}{1.024990in}}%
\pgfpathlineto{\pgfqpoint{1.319382in}{0.883875in}}%
\pgfpathlineto{\pgfqpoint{1.319765in}{1.087708in}}%
\pgfpathlineto{\pgfqpoint{1.320914in}{0.946593in}}%
\pgfpathlineto{\pgfqpoint{1.321296in}{0.946593in}}%
\pgfpathlineto{\pgfqpoint{1.321296in}{0.878649in}}%
\pgfpathlineto{\pgfqpoint{1.322062in}{1.014537in}}%
\pgfpathlineto{\pgfqpoint{1.322828in}{1.014537in}}%
\pgfpathlineto{\pgfqpoint{1.323211in}{1.014537in}}%
\pgfpathlineto{\pgfqpoint{1.324742in}{0.941367in}}%
\pgfpathlineto{\pgfqpoint{1.325125in}{0.941367in}}%
\pgfpathlineto{\pgfqpoint{1.325125in}{1.045896in}}%
\pgfpathlineto{\pgfqpoint{1.325890in}{0.915234in}}%
\pgfpathlineto{\pgfqpoint{1.326656in}{0.993631in}}%
\pgfpathlineto{\pgfqpoint{1.327039in}{0.993631in}}%
\pgfpathlineto{\pgfqpoint{1.327039in}{1.014537in}}%
\pgfpathlineto{\pgfqpoint{1.327422in}{0.967499in}}%
\pgfpathlineto{\pgfqpoint{1.328570in}{0.993631in}}%
\pgfpathlineto{\pgfqpoint{1.328953in}{0.993631in}}%
\pgfpathlineto{\pgfqpoint{1.329719in}{1.082482in}}%
\pgfpathlineto{\pgfqpoint{1.330484in}{0.993631in}}%
\pgfpathlineto{\pgfqpoint{1.330867in}{0.993631in}}%
\pgfpathlineto{\pgfqpoint{1.331250in}{0.904781in}}%
\pgfpathlineto{\pgfqpoint{1.332016in}{1.024990in}}%
\pgfpathlineto{\pgfqpoint{1.332398in}{1.019764in}}%
\pgfpathlineto{\pgfqpoint{1.333164in}{1.019764in}}%
\pgfpathlineto{\pgfqpoint{1.334313in}{0.951820in}}%
\pgfpathlineto{\pgfqpoint{1.333930in}{1.066802in}}%
\pgfpathlineto{\pgfqpoint{1.334695in}{1.051123in}}%
\pgfpathlineto{\pgfqpoint{1.335078in}{1.051123in}}%
\pgfpathlineto{\pgfqpoint{1.335844in}{0.930914in}}%
\pgfpathlineto{\pgfqpoint{1.336227in}{1.098161in}}%
\pgfpathlineto{\pgfqpoint{1.336610in}{0.983178in}}%
\pgfpathlineto{\pgfqpoint{1.336992in}{0.983178in}}%
\pgfpathlineto{\pgfqpoint{1.336992in}{0.972726in}}%
\pgfpathlineto{\pgfqpoint{1.337758in}{1.061576in}}%
\pgfpathlineto{\pgfqpoint{1.338524in}{1.056349in}}%
\pgfpathlineto{\pgfqpoint{1.338907in}{1.056349in}}%
\pgfpathlineto{\pgfqpoint{1.338907in}{1.087708in}}%
\pgfpathlineto{\pgfqpoint{1.339672in}{0.967499in}}%
\pgfpathlineto{\pgfqpoint{1.340438in}{1.004084in}}%
\pgfpathlineto{\pgfqpoint{1.340821in}{1.004084in}}%
\pgfpathlineto{\pgfqpoint{1.341204in}{1.056349in}}%
\pgfpathlineto{\pgfqpoint{1.342352in}{1.040670in}}%
\pgfpathlineto{\pgfqpoint{1.342735in}{1.040670in}}%
\pgfpathlineto{\pgfqpoint{1.343118in}{0.967499in}}%
\pgfpathlineto{\pgfqpoint{1.343500in}{1.103388in}}%
\pgfpathlineto{\pgfqpoint{1.344266in}{1.040670in}}%
\pgfpathlineto{\pgfqpoint{1.344649in}{1.040670in}}%
\pgfpathlineto{\pgfqpoint{1.345032in}{1.113840in}}%
\pgfpathlineto{\pgfqpoint{1.346180in}{0.977952in}}%
\pgfpathlineto{\pgfqpoint{1.346563in}{0.977952in}}%
\pgfpathlineto{\pgfqpoint{1.347712in}{1.082482in}}%
\pgfpathlineto{\pgfqpoint{1.348094in}{1.082482in}}%
\pgfpathlineto{\pgfqpoint{1.348477in}{1.082482in}}%
\pgfpathlineto{\pgfqpoint{1.348477in}{0.962273in}}%
\pgfpathlineto{\pgfqpoint{1.349243in}{1.098161in}}%
\pgfpathlineto{\pgfqpoint{1.350009in}{1.066802in}}%
\pgfpathlineto{\pgfqpoint{1.350391in}{1.066802in}}%
\pgfpathlineto{\pgfqpoint{1.350774in}{1.139973in}}%
\pgfpathlineto{\pgfqpoint{1.351540in}{1.040670in}}%
\pgfpathlineto{\pgfqpoint{1.351923in}{1.119067in}}%
\pgfpathlineto{\pgfqpoint{1.352306in}{1.119067in}}%
\pgfpathlineto{\pgfqpoint{1.353837in}{0.998858in}}%
\pgfpathlineto{\pgfqpoint{1.354220in}{0.998858in}}%
\pgfpathlineto{\pgfqpoint{1.354220in}{0.983178in}}%
\pgfpathlineto{\pgfqpoint{1.354603in}{1.087708in}}%
\pgfpathlineto{\pgfqpoint{1.355751in}{1.061576in}}%
\pgfpathlineto{\pgfqpoint{1.356134in}{1.061576in}}%
\pgfpathlineto{\pgfqpoint{1.356134in}{1.066802in}}%
\pgfpathlineto{\pgfqpoint{1.356900in}{0.993631in}}%
\pgfpathlineto{\pgfqpoint{1.357665in}{1.045896in}}%
\pgfpathlineto{\pgfqpoint{1.358048in}{1.045896in}}%
\pgfpathlineto{\pgfqpoint{1.358814in}{1.040670in}}%
\pgfpathlineto{\pgfqpoint{1.359579in}{1.124293in}}%
\pgfpathlineto{\pgfqpoint{1.359962in}{1.124293in}}%
\pgfpathlineto{\pgfqpoint{1.359962in}{1.019764in}}%
\pgfpathlineto{\pgfqpoint{1.361494in}{1.098161in}}%
\pgfpathlineto{\pgfqpoint{1.361876in}{1.098161in}}%
\pgfpathlineto{\pgfqpoint{1.363025in}{1.019764in}}%
\pgfpathlineto{\pgfqpoint{1.363408in}{1.145199in}}%
\pgfpathlineto{\pgfqpoint{1.363790in}{1.145199in}}%
\pgfpathlineto{\pgfqpoint{1.364939in}{1.030217in}}%
\pgfpathlineto{\pgfqpoint{1.365322in}{1.119067in}}%
\pgfpathlineto{\pgfqpoint{1.365705in}{1.119067in}}%
\pgfpathlineto{\pgfqpoint{1.365705in}{1.009311in}}%
\pgfpathlineto{\pgfqpoint{1.367236in}{1.040670in}}%
\pgfpathlineto{\pgfqpoint{1.367619in}{1.040670in}}%
\pgfpathlineto{\pgfqpoint{1.368767in}{1.145199in}}%
\pgfpathlineto{\pgfqpoint{1.368002in}{1.030217in}}%
\pgfpathlineto{\pgfqpoint{1.369150in}{1.092935in}}%
\pgfpathlineto{\pgfqpoint{1.369533in}{1.092935in}}%
\pgfpathlineto{\pgfqpoint{1.370299in}{0.983178in}}%
\pgfpathlineto{\pgfqpoint{1.371064in}{1.061576in}}%
\pgfpathlineto{\pgfqpoint{1.371447in}{1.061576in}}%
\pgfpathlineto{\pgfqpoint{1.372596in}{1.145199in}}%
\pgfpathlineto{\pgfqpoint{1.371830in}{1.024990in}}%
\pgfpathlineto{\pgfqpoint{1.372978in}{1.119067in}}%
\pgfpathlineto{\pgfqpoint{1.373361in}{1.119067in}}%
\pgfpathlineto{\pgfqpoint{1.374893in}{1.024990in}}%
\pgfpathlineto{\pgfqpoint{1.375275in}{1.024990in}}%
\pgfpathlineto{\pgfqpoint{1.376424in}{1.134746in}}%
\pgfpathlineto{\pgfqpoint{1.375658in}{0.977952in}}%
\pgfpathlineto{\pgfqpoint{1.376807in}{1.035443in}}%
\pgfpathlineto{\pgfqpoint{1.377190in}{1.035443in}}%
\pgfpathlineto{\pgfqpoint{1.377190in}{1.129520in}}%
\pgfpathlineto{\pgfqpoint{1.378721in}{1.035443in}}%
\pgfpathlineto{\pgfqpoint{1.379104in}{1.035443in}}%
\pgfpathlineto{\pgfqpoint{1.380252in}{1.207917in}}%
\pgfpathlineto{\pgfqpoint{1.380635in}{1.035443in}}%
\pgfpathlineto{\pgfqpoint{1.381018in}{1.035443in}}%
\pgfpathlineto{\pgfqpoint{1.381784in}{1.155652in}}%
\pgfpathlineto{\pgfqpoint{1.382549in}{1.155652in}}%
\pgfpathlineto{\pgfqpoint{1.382932in}{1.155652in}}%
\pgfpathlineto{\pgfqpoint{1.382932in}{1.187011in}}%
\pgfpathlineto{\pgfqpoint{1.383315in}{0.998858in}}%
\pgfpathlineto{\pgfqpoint{1.384463in}{1.098161in}}%
\pgfpathlineto{\pgfqpoint{1.384846in}{1.098161in}}%
\pgfpathlineto{\pgfqpoint{1.385612in}{1.030217in}}%
\pgfpathlineto{\pgfqpoint{1.386377in}{1.202691in}}%
\pgfpathlineto{\pgfqpoint{1.386760in}{1.202691in}}%
\pgfpathlineto{\pgfqpoint{1.387909in}{1.056349in}}%
\pgfpathlineto{\pgfqpoint{1.388292in}{1.166105in}}%
\pgfpathlineto{\pgfqpoint{1.388674in}{1.166105in}}%
\pgfpathlineto{\pgfqpoint{1.389823in}{1.072029in}}%
\pgfpathlineto{\pgfqpoint{1.390206in}{1.077255in}}%
\pgfpathlineto{\pgfqpoint{1.390589in}{1.077255in}}%
\pgfpathlineto{\pgfqpoint{1.391354in}{1.181785in}}%
\pgfpathlineto{\pgfqpoint{1.390971in}{1.061576in}}%
\pgfpathlineto{\pgfqpoint{1.392120in}{1.098161in}}%
\pgfpathlineto{\pgfqpoint{1.392503in}{1.098161in}}%
\pgfpathlineto{\pgfqpoint{1.392886in}{1.119067in}}%
\pgfpathlineto{\pgfqpoint{1.393268in}{1.066802in}}%
\pgfpathlineto{\pgfqpoint{1.394034in}{1.098161in}}%
\pgfpathlineto{\pgfqpoint{1.394417in}{1.098161in}}%
\pgfpathlineto{\pgfqpoint{1.394417in}{1.155652in}}%
\pgfpathlineto{\pgfqpoint{1.395565in}{1.051123in}}%
\pgfpathlineto{\pgfqpoint{1.395948in}{1.092935in}}%
\pgfpathlineto{\pgfqpoint{1.396331in}{1.092935in}}%
\pgfpathlineto{\pgfqpoint{1.396331in}{1.145199in}}%
\pgfpathlineto{\pgfqpoint{1.397097in}{1.009311in}}%
\pgfpathlineto{\pgfqpoint{1.397862in}{1.124293in}}%
\pgfpathlineto{\pgfqpoint{1.398245in}{1.124293in}}%
\pgfpathlineto{\pgfqpoint{1.399394in}{1.040670in}}%
\pgfpathlineto{\pgfqpoint{1.398628in}{1.171332in}}%
\pgfpathlineto{\pgfqpoint{1.399777in}{1.051123in}}%
\pgfpathlineto{\pgfqpoint{1.400159in}{1.051123in}}%
\pgfpathlineto{\pgfqpoint{1.401308in}{1.234050in}}%
\pgfpathlineto{\pgfqpoint{1.401691in}{1.103388in}}%
\pgfpathlineto{\pgfqpoint{1.402073in}{1.103388in}}%
\pgfpathlineto{\pgfqpoint{1.402456in}{1.061576in}}%
\pgfpathlineto{\pgfqpoint{1.403222in}{1.134746in}}%
\pgfpathlineto{\pgfqpoint{1.403605in}{1.113840in}}%
\pgfpathlineto{\pgfqpoint{1.403988in}{1.113840in}}%
\pgfpathlineto{\pgfqpoint{1.403988in}{1.192238in}}%
\pgfpathlineto{\pgfqpoint{1.405136in}{1.061576in}}%
\pgfpathlineto{\pgfqpoint{1.405519in}{1.187011in}}%
\pgfpathlineto{\pgfqpoint{1.405902in}{1.187011in}}%
\pgfpathlineto{\pgfqpoint{1.407050in}{1.056349in}}%
\pgfpathlineto{\pgfqpoint{1.407433in}{1.155652in}}%
\pgfpathlineto{\pgfqpoint{1.407816in}{1.155652in}}%
\pgfpathlineto{\pgfqpoint{1.408582in}{1.066802in}}%
\pgfpathlineto{\pgfqpoint{1.408964in}{1.187011in}}%
\pgfpathlineto{\pgfqpoint{1.409347in}{1.119067in}}%
\pgfpathlineto{\pgfqpoint{1.409730in}{1.119067in}}%
\pgfpathlineto{\pgfqpoint{1.409730in}{1.192238in}}%
\pgfpathlineto{\pgfqpoint{1.410496in}{1.040670in}}%
\pgfpathlineto{\pgfqpoint{1.411261in}{1.040670in}}%
\pgfpathlineto{\pgfqpoint{1.411644in}{1.040670in}}%
\pgfpathlineto{\pgfqpoint{1.412793in}{1.187011in}}%
\pgfpathlineto{\pgfqpoint{1.413176in}{1.134746in}}%
\pgfpathlineto{\pgfqpoint{1.413558in}{1.134746in}}%
\pgfpathlineto{\pgfqpoint{1.414707in}{1.223597in}}%
\pgfpathlineto{\pgfqpoint{1.415090in}{1.176558in}}%
\pgfpathlineto{\pgfqpoint{1.415473in}{1.176558in}}%
\pgfpathlineto{\pgfqpoint{1.415855in}{1.045896in}}%
\pgfpathlineto{\pgfqpoint{1.417004in}{1.108614in}}%
\pgfpathlineto{\pgfqpoint{1.417387in}{1.108614in}}%
\pgfpathlineto{\pgfqpoint{1.417387in}{1.051123in}}%
\pgfpathlineto{\pgfqpoint{1.417770in}{1.155652in}}%
\pgfpathlineto{\pgfqpoint{1.418918in}{1.087708in}}%
\pgfpathlineto{\pgfqpoint{1.419301in}{1.087708in}}%
\pgfpathlineto{\pgfqpoint{1.419684in}{1.176558in}}%
\pgfpathlineto{\pgfqpoint{1.420449in}{1.077255in}}%
\pgfpathlineto{\pgfqpoint{1.420832in}{1.098161in}}%
\pgfpathlineto{\pgfqpoint{1.421215in}{1.098161in}}%
\pgfpathlineto{\pgfqpoint{1.421598in}{1.213144in}}%
\pgfpathlineto{\pgfqpoint{1.422746in}{1.061576in}}%
\pgfpathlineto{\pgfqpoint{1.423129in}{1.061576in}}%
\pgfpathlineto{\pgfqpoint{1.423129in}{1.145199in}}%
\pgfpathlineto{\pgfqpoint{1.424660in}{1.145199in}}%
\pgfpathlineto{\pgfqpoint{1.425043in}{1.145199in}}%
\pgfpathlineto{\pgfqpoint{1.426192in}{1.092935in}}%
\pgfpathlineto{\pgfqpoint{1.425809in}{1.197464in}}%
\pgfpathlineto{\pgfqpoint{1.426575in}{1.103388in}}%
\pgfpathlineto{\pgfqpoint{1.426957in}{1.103388in}}%
\pgfpathlineto{\pgfqpoint{1.428106in}{1.213144in}}%
\pgfpathlineto{\pgfqpoint{1.428489in}{1.176558in}}%
\pgfpathlineto{\pgfqpoint{1.428872in}{1.176558in}}%
\pgfpathlineto{\pgfqpoint{1.428872in}{1.098161in}}%
\pgfpathlineto{\pgfqpoint{1.430403in}{1.113840in}}%
\pgfpathlineto{\pgfqpoint{1.430786in}{1.113840in}}%
\pgfpathlineto{\pgfqpoint{1.431934in}{1.244503in}}%
\pgfpathlineto{\pgfqpoint{1.432317in}{1.139973in}}%
\pgfpathlineto{\pgfqpoint{1.432700in}{1.139973in}}%
\pgfpathlineto{\pgfqpoint{1.432700in}{1.119067in}}%
\pgfpathlineto{\pgfqpoint{1.433848in}{1.270635in}}%
\pgfpathlineto{\pgfqpoint{1.434231in}{1.176558in}}%
\pgfpathlineto{\pgfqpoint{1.434614in}{1.176558in}}%
\pgfpathlineto{\pgfqpoint{1.435763in}{1.270635in}}%
\pgfpathlineto{\pgfqpoint{1.436145in}{1.072029in}}%
\pgfpathlineto{\pgfqpoint{1.436528in}{1.072029in}}%
\pgfpathlineto{\pgfqpoint{1.437677in}{1.176558in}}%
\pgfpathlineto{\pgfqpoint{1.438060in}{1.108614in}}%
\pgfpathlineto{\pgfqpoint{1.438825in}{1.108614in}}%
\pgfpathlineto{\pgfqpoint{1.439208in}{1.249729in}}%
\pgfpathlineto{\pgfqpoint{1.440356in}{1.150426in}}%
\pgfpathlineto{\pgfqpoint{1.440739in}{1.150426in}}%
\pgfpathlineto{\pgfqpoint{1.441122in}{1.056349in}}%
\pgfpathlineto{\pgfqpoint{1.442271in}{1.192238in}}%
\pgfpathlineto{\pgfqpoint{1.442653in}{1.192238in}}%
\pgfpathlineto{\pgfqpoint{1.442653in}{1.234050in}}%
\pgfpathlineto{\pgfqpoint{1.443036in}{1.155652in}}%
\pgfpathlineto{\pgfqpoint{1.444185in}{1.187011in}}%
\pgfpathlineto{\pgfqpoint{1.444568in}{1.187011in}}%
\pgfpathlineto{\pgfqpoint{1.444568in}{1.092935in}}%
\pgfpathlineto{\pgfqpoint{1.445333in}{1.307220in}}%
\pgfpathlineto{\pgfqpoint{1.446099in}{1.103388in}}%
\pgfpathlineto{\pgfqpoint{1.446482in}{1.103388in}}%
\pgfpathlineto{\pgfqpoint{1.446865in}{1.260182in}}%
\pgfpathlineto{\pgfqpoint{1.447247in}{1.066802in}}%
\pgfpathlineto{\pgfqpoint{1.448013in}{1.150426in}}%
\pgfpathlineto{\pgfqpoint{1.448396in}{1.150426in}}%
\pgfpathlineto{\pgfqpoint{1.449544in}{1.129520in}}%
\pgfpathlineto{\pgfqpoint{1.449927in}{1.207917in}}%
\pgfpathlineto{\pgfqpoint{1.450310in}{1.207917in}}%
\pgfpathlineto{\pgfqpoint{1.450310in}{1.124293in}}%
\pgfpathlineto{\pgfqpoint{1.450693in}{1.265408in}}%
\pgfpathlineto{\pgfqpoint{1.451841in}{1.171332in}}%
\pgfpathlineto{\pgfqpoint{1.452224in}{1.171332in}}%
\pgfpathlineto{\pgfqpoint{1.453373in}{1.181785in}}%
\pgfpathlineto{\pgfqpoint{1.453756in}{1.139973in}}%
\pgfpathlineto{\pgfqpoint{1.454138in}{1.139973in}}%
\pgfpathlineto{\pgfqpoint{1.454904in}{1.244503in}}%
\pgfpathlineto{\pgfqpoint{1.455287in}{1.092935in}}%
\pgfpathlineto{\pgfqpoint{1.455670in}{1.098161in}}%
\pgfpathlineto{\pgfqpoint{1.456053in}{1.098161in}}%
\pgfpathlineto{\pgfqpoint{1.456053in}{1.082482in}}%
\pgfpathlineto{\pgfqpoint{1.457584in}{1.218370in}}%
\pgfpathlineto{\pgfqpoint{1.457967in}{1.218370in}}%
\pgfpathlineto{\pgfqpoint{1.458350in}{1.051123in}}%
\pgfpathlineto{\pgfqpoint{1.459498in}{1.119067in}}%
\pgfpathlineto{\pgfqpoint{1.459881in}{1.119067in}}%
\pgfpathlineto{\pgfqpoint{1.460646in}{1.035443in}}%
\pgfpathlineto{\pgfqpoint{1.460264in}{1.213144in}}%
\pgfpathlineto{\pgfqpoint{1.461412in}{1.176558in}}%
\pgfpathlineto{\pgfqpoint{1.461795in}{1.176558in}}%
\pgfpathlineto{\pgfqpoint{1.462178in}{1.082482in}}%
\pgfpathlineto{\pgfqpoint{1.462943in}{1.207917in}}%
\pgfpathlineto{\pgfqpoint{1.463326in}{1.119067in}}%
\pgfpathlineto{\pgfqpoint{1.463709in}{1.119067in}}%
\pgfpathlineto{\pgfqpoint{1.464092in}{1.291541in}}%
\pgfpathlineto{\pgfqpoint{1.465240in}{1.150426in}}%
\pgfpathlineto{\pgfqpoint{1.465623in}{1.150426in}}%
\pgfpathlineto{\pgfqpoint{1.465623in}{1.218370in}}%
\pgfpathlineto{\pgfqpoint{1.466389in}{1.145199in}}%
\pgfpathlineto{\pgfqpoint{1.467155in}{1.160879in}}%
\pgfpathlineto{\pgfqpoint{1.467537in}{1.160879in}}%
\pgfpathlineto{\pgfqpoint{1.468686in}{1.082482in}}%
\pgfpathlineto{\pgfqpoint{1.469069in}{1.234050in}}%
\pgfpathlineto{\pgfqpoint{1.469452in}{1.234050in}}%
\pgfpathlineto{\pgfqpoint{1.470600in}{1.087708in}}%
\pgfpathlineto{\pgfqpoint{1.470983in}{1.155652in}}%
\pgfpathlineto{\pgfqpoint{1.471366in}{1.155652in}}%
\pgfpathlineto{\pgfqpoint{1.471366in}{1.098161in}}%
\pgfpathlineto{\pgfqpoint{1.472131in}{1.234050in}}%
\pgfpathlineto{\pgfqpoint{1.472897in}{1.166105in}}%
\pgfpathlineto{\pgfqpoint{1.473280in}{1.166105in}}%
\pgfpathlineto{\pgfqpoint{1.473280in}{1.145199in}}%
\pgfpathlineto{\pgfqpoint{1.474428in}{1.228823in}}%
\pgfpathlineto{\pgfqpoint{1.474811in}{1.187011in}}%
\pgfpathlineto{\pgfqpoint{1.475194in}{1.187011in}}%
\pgfpathlineto{\pgfqpoint{1.476343in}{1.139973in}}%
\pgfpathlineto{\pgfqpoint{1.475577in}{1.213144in}}%
\pgfpathlineto{\pgfqpoint{1.476725in}{1.181785in}}%
\pgfpathlineto{\pgfqpoint{1.477108in}{1.181785in}}%
\pgfpathlineto{\pgfqpoint{1.477108in}{1.192238in}}%
\pgfpathlineto{\pgfqpoint{1.477874in}{1.145199in}}%
\pgfpathlineto{\pgfqpoint{1.478640in}{1.155652in}}%
\pgfpathlineto{\pgfqpoint{1.479022in}{1.155652in}}%
\pgfpathlineto{\pgfqpoint{1.479405in}{1.244503in}}%
\pgfpathlineto{\pgfqpoint{1.479405in}{1.145199in}}%
\pgfpathlineto{\pgfqpoint{1.480554in}{1.171332in}}%
\pgfpathlineto{\pgfqpoint{1.480936in}{1.171332in}}%
\pgfpathlineto{\pgfqpoint{1.480936in}{1.124293in}}%
\pgfpathlineto{\pgfqpoint{1.482468in}{1.239276in}}%
\pgfpathlineto{\pgfqpoint{1.482851in}{1.239276in}}%
\pgfpathlineto{\pgfqpoint{1.482851in}{1.113840in}}%
\pgfpathlineto{\pgfqpoint{1.483233in}{1.286314in}}%
\pgfpathlineto{\pgfqpoint{1.484382in}{1.166105in}}%
\pgfpathlineto{\pgfqpoint{1.484765in}{1.166105in}}%
\pgfpathlineto{\pgfqpoint{1.484765in}{1.218370in}}%
\pgfpathlineto{\pgfqpoint{1.485913in}{1.134746in}}%
\pgfpathlineto{\pgfqpoint{1.486296in}{1.192238in}}%
\pgfpathlineto{\pgfqpoint{1.486679in}{1.192238in}}%
\pgfpathlineto{\pgfqpoint{1.486679in}{1.082482in}}%
\pgfpathlineto{\pgfqpoint{1.488210in}{1.249729in}}%
\pgfpathlineto{\pgfqpoint{1.488593in}{1.249729in}}%
\pgfpathlineto{\pgfqpoint{1.489359in}{1.166105in}}%
\pgfpathlineto{\pgfqpoint{1.490124in}{1.166105in}}%
\pgfpathlineto{\pgfqpoint{1.490507in}{1.166105in}}%
\pgfpathlineto{\pgfqpoint{1.491273in}{1.239276in}}%
\pgfpathlineto{\pgfqpoint{1.491656in}{1.119067in}}%
\pgfpathlineto{\pgfqpoint{1.492039in}{1.124293in}}%
\pgfpathlineto{\pgfqpoint{1.492421in}{1.124293in}}%
\pgfpathlineto{\pgfqpoint{1.492421in}{1.045896in}}%
\pgfpathlineto{\pgfqpoint{1.493953in}{1.228823in}}%
\pgfpathlineto{\pgfqpoint{1.494336in}{1.228823in}}%
\pgfpathlineto{\pgfqpoint{1.495484in}{1.113840in}}%
\pgfpathlineto{\pgfqpoint{1.495867in}{1.275861in}}%
\pgfpathlineto{\pgfqpoint{1.496250in}{1.275861in}}%
\pgfpathlineto{\pgfqpoint{1.497015in}{1.082482in}}%
\pgfpathlineto{\pgfqpoint{1.497781in}{1.223597in}}%
\pgfpathlineto{\pgfqpoint{1.498164in}{1.223597in}}%
\pgfpathlineto{\pgfqpoint{1.499312in}{1.066802in}}%
\pgfpathlineto{\pgfqpoint{1.499695in}{1.171332in}}%
\pgfpathlineto{\pgfqpoint{1.500078in}{1.171332in}}%
\pgfpathlineto{\pgfqpoint{1.500844in}{1.213144in}}%
\pgfpathlineto{\pgfqpoint{1.500461in}{1.045896in}}%
\pgfpathlineto{\pgfqpoint{1.501609in}{1.187011in}}%
\pgfpathlineto{\pgfqpoint{1.501992in}{1.187011in}}%
\pgfpathlineto{\pgfqpoint{1.502375in}{1.040670in}}%
\pgfpathlineto{\pgfqpoint{1.503523in}{1.234050in}}%
\pgfpathlineto{\pgfqpoint{1.503906in}{1.234050in}}%
\pgfpathlineto{\pgfqpoint{1.505055in}{1.082482in}}%
\pgfpathlineto{\pgfqpoint{1.505438in}{1.160879in}}%
\pgfpathlineto{\pgfqpoint{1.505820in}{1.160879in}}%
\pgfpathlineto{\pgfqpoint{1.505820in}{1.051123in}}%
\pgfpathlineto{\pgfqpoint{1.506203in}{1.218370in}}%
\pgfpathlineto{\pgfqpoint{1.507352in}{1.056349in}}%
\pgfpathlineto{\pgfqpoint{1.507735in}{1.056349in}}%
\pgfpathlineto{\pgfqpoint{1.508117in}{1.228823in}}%
\pgfpathlineto{\pgfqpoint{1.509266in}{1.134746in}}%
\pgfpathlineto{\pgfqpoint{1.509649in}{1.134746in}}%
\pgfpathlineto{\pgfqpoint{1.510797in}{1.166105in}}%
\pgfpathlineto{\pgfqpoint{1.511180in}{1.087708in}}%
\pgfpathlineto{\pgfqpoint{1.511563in}{1.087708in}}%
\pgfpathlineto{\pgfqpoint{1.511946in}{1.234050in}}%
\pgfpathlineto{\pgfqpoint{1.513094in}{1.192238in}}%
\pgfpathlineto{\pgfqpoint{1.513477in}{1.192238in}}%
\pgfpathlineto{\pgfqpoint{1.513477in}{1.139973in}}%
\pgfpathlineto{\pgfqpoint{1.514626in}{1.317673in}}%
\pgfpathlineto{\pgfqpoint{1.515008in}{1.223597in}}%
\pgfpathlineto{\pgfqpoint{1.515391in}{1.223597in}}%
\pgfpathlineto{\pgfqpoint{1.516540in}{1.333353in}}%
\pgfpathlineto{\pgfqpoint{1.515774in}{1.098161in}}%
\pgfpathlineto{\pgfqpoint{1.516923in}{1.234050in}}%
\pgfpathlineto{\pgfqpoint{1.517305in}{1.234050in}}%
\pgfpathlineto{\pgfqpoint{1.518071in}{1.103388in}}%
\pgfpathlineto{\pgfqpoint{1.518837in}{1.124293in}}%
\pgfpathlineto{\pgfqpoint{1.519219in}{1.124293in}}%
\pgfpathlineto{\pgfqpoint{1.520368in}{1.207917in}}%
\pgfpathlineto{\pgfqpoint{1.519985in}{1.113840in}}%
\pgfpathlineto{\pgfqpoint{1.520751in}{1.207917in}}%
\pgfpathlineto{\pgfqpoint{1.521516in}{1.207917in}}%
\pgfpathlineto{\pgfqpoint{1.521516in}{1.129520in}}%
\pgfpathlineto{\pgfqpoint{1.522282in}{1.249729in}}%
\pgfpathlineto{\pgfqpoint{1.523048in}{1.239276in}}%
\pgfpathlineto{\pgfqpoint{1.523431in}{1.239276in}}%
\pgfpathlineto{\pgfqpoint{1.524196in}{1.045896in}}%
\pgfpathlineto{\pgfqpoint{1.524962in}{1.181785in}}%
\pgfpathlineto{\pgfqpoint{1.525345in}{1.181785in}}%
\pgfpathlineto{\pgfqpoint{1.526110in}{1.124293in}}%
\pgfpathlineto{\pgfqpoint{1.525728in}{1.270635in}}%
\pgfpathlineto{\pgfqpoint{1.526876in}{1.129520in}}%
\pgfpathlineto{\pgfqpoint{1.527259in}{1.129520in}}%
\pgfpathlineto{\pgfqpoint{1.527642in}{1.108614in}}%
\pgfpathlineto{\pgfqpoint{1.528790in}{1.207917in}}%
\pgfpathlineto{\pgfqpoint{1.529173in}{1.207917in}}%
\pgfpathlineto{\pgfqpoint{1.529173in}{1.281088in}}%
\pgfpathlineto{\pgfqpoint{1.529556in}{1.087708in}}%
\pgfpathlineto{\pgfqpoint{1.530704in}{1.213144in}}%
\pgfpathlineto{\pgfqpoint{1.531087in}{1.213144in}}%
\pgfpathlineto{\pgfqpoint{1.531470in}{1.072029in}}%
\pgfpathlineto{\pgfqpoint{1.531853in}{1.254955in}}%
\pgfpathlineto{\pgfqpoint{1.532619in}{1.207917in}}%
\pgfpathlineto{\pgfqpoint{1.533001in}{1.207917in}}%
\pgfpathlineto{\pgfqpoint{1.534533in}{1.077255in}}%
\pgfpathlineto{\pgfqpoint{1.534916in}{1.077255in}}%
\pgfpathlineto{\pgfqpoint{1.534916in}{1.239276in}}%
\pgfpathlineto{\pgfqpoint{1.536447in}{1.024990in}}%
\pgfpathlineto{\pgfqpoint{1.536830in}{1.024990in}}%
\pgfpathlineto{\pgfqpoint{1.538361in}{1.223597in}}%
\pgfpathlineto{\pgfqpoint{1.538744in}{1.223597in}}%
\pgfpathlineto{\pgfqpoint{1.539509in}{1.103388in}}%
\pgfpathlineto{\pgfqpoint{1.539127in}{1.234050in}}%
\pgfpathlineto{\pgfqpoint{1.540275in}{1.181785in}}%
\pgfpathlineto{\pgfqpoint{1.540658in}{1.181785in}}%
\pgfpathlineto{\pgfqpoint{1.540658in}{1.139973in}}%
\pgfpathlineto{\pgfqpoint{1.541424in}{1.322900in}}%
\pgfpathlineto{\pgfqpoint{1.542189in}{1.192238in}}%
\pgfpathlineto{\pgfqpoint{1.542572in}{1.192238in}}%
\pgfpathlineto{\pgfqpoint{1.542572in}{1.035443in}}%
\pgfpathlineto{\pgfqpoint{1.544103in}{1.103388in}}%
\pgfpathlineto{\pgfqpoint{1.544486in}{1.103388in}}%
\pgfpathlineto{\pgfqpoint{1.544486in}{1.092935in}}%
\pgfpathlineto{\pgfqpoint{1.545635in}{1.171332in}}%
\pgfpathlineto{\pgfqpoint{1.546018in}{1.108614in}}%
\pgfpathlineto{\pgfqpoint{1.546400in}{1.108614in}}%
\pgfpathlineto{\pgfqpoint{1.547549in}{1.202691in}}%
\pgfpathlineto{\pgfqpoint{1.547166in}{1.082482in}}%
\pgfpathlineto{\pgfqpoint{1.547932in}{1.166105in}}%
\pgfpathlineto{\pgfqpoint{1.548697in}{1.166105in}}%
\pgfpathlineto{\pgfqpoint{1.549463in}{1.239276in}}%
\pgfpathlineto{\pgfqpoint{1.550229in}{1.145199in}}%
\pgfpathlineto{\pgfqpoint{1.550612in}{1.145199in}}%
\pgfpathlineto{\pgfqpoint{1.551760in}{1.254955in}}%
\pgfpathlineto{\pgfqpoint{1.552143in}{1.082482in}}%
\pgfpathlineto{\pgfqpoint{1.552526in}{1.082482in}}%
\pgfpathlineto{\pgfqpoint{1.552526in}{1.056349in}}%
\pgfpathlineto{\pgfqpoint{1.554057in}{1.223597in}}%
\pgfpathlineto{\pgfqpoint{1.554440in}{1.223597in}}%
\pgfpathlineto{\pgfqpoint{1.554440in}{1.082482in}}%
\pgfpathlineto{\pgfqpoint{1.555588in}{1.286314in}}%
\pgfpathlineto{\pgfqpoint{1.555971in}{1.139973in}}%
\pgfpathlineto{\pgfqpoint{1.556354in}{1.139973in}}%
\pgfpathlineto{\pgfqpoint{1.557885in}{1.202691in}}%
\pgfpathlineto{\pgfqpoint{1.558268in}{1.202691in}}%
\pgfpathlineto{\pgfqpoint{1.559034in}{1.098161in}}%
\pgfpathlineto{\pgfqpoint{1.559799in}{1.129520in}}%
\pgfpathlineto{\pgfqpoint{1.560182in}{1.129520in}}%
\pgfpathlineto{\pgfqpoint{1.560565in}{1.244503in}}%
\pgfpathlineto{\pgfqpoint{1.561331in}{1.098161in}}%
\pgfpathlineto{\pgfqpoint{1.561714in}{1.223597in}}%
\pgfpathlineto{\pgfqpoint{1.562096in}{1.223597in}}%
\pgfpathlineto{\pgfqpoint{1.563245in}{1.124293in}}%
\pgfpathlineto{\pgfqpoint{1.562479in}{1.260182in}}%
\pgfpathlineto{\pgfqpoint{1.563628in}{1.166105in}}%
\pgfpathlineto{\pgfqpoint{1.564011in}{1.166105in}}%
\pgfpathlineto{\pgfqpoint{1.564776in}{1.061576in}}%
\pgfpathlineto{\pgfqpoint{1.565159in}{1.176558in}}%
\pgfpathlineto{\pgfqpoint{1.565542in}{1.119067in}}%
\pgfpathlineto{\pgfqpoint{1.565925in}{1.119067in}}%
\pgfpathlineto{\pgfqpoint{1.567073in}{1.192238in}}%
\pgfpathlineto{\pgfqpoint{1.567456in}{1.187011in}}%
\pgfpathlineto{\pgfqpoint{1.567839in}{1.187011in}}%
\pgfpathlineto{\pgfqpoint{1.567839in}{1.134746in}}%
\pgfpathlineto{\pgfqpoint{1.568222in}{1.265408in}}%
\pgfpathlineto{\pgfqpoint{1.569370in}{1.202691in}}%
\pgfpathlineto{\pgfqpoint{1.569753in}{1.202691in}}%
\pgfpathlineto{\pgfqpoint{1.569753in}{1.066802in}}%
\pgfpathlineto{\pgfqpoint{1.571284in}{1.119067in}}%
\pgfpathlineto{\pgfqpoint{1.571667in}{1.119067in}}%
\pgfpathlineto{\pgfqpoint{1.572050in}{1.218370in}}%
\pgfpathlineto{\pgfqpoint{1.572433in}{1.103388in}}%
\pgfpathlineto{\pgfqpoint{1.573199in}{1.166105in}}%
\pgfpathlineto{\pgfqpoint{1.573581in}{1.166105in}}%
\pgfpathlineto{\pgfqpoint{1.573964in}{1.082482in}}%
\pgfpathlineto{\pgfqpoint{1.574347in}{1.244503in}}%
\pgfpathlineto{\pgfqpoint{1.575113in}{1.119067in}}%
\pgfpathlineto{\pgfqpoint{1.575495in}{1.119067in}}%
\pgfpathlineto{\pgfqpoint{1.576644in}{1.092935in}}%
\pgfpathlineto{\pgfqpoint{1.577027in}{1.228823in}}%
\pgfpathlineto{\pgfqpoint{1.577410in}{1.228823in}}%
\pgfpathlineto{\pgfqpoint{1.578941in}{1.108614in}}%
\pgfpathlineto{\pgfqpoint{1.579324in}{1.108614in}}%
\pgfpathlineto{\pgfqpoint{1.580472in}{1.176558in}}%
\pgfpathlineto{\pgfqpoint{1.580855in}{1.024990in}}%
\pgfpathlineto{\pgfqpoint{1.581238in}{1.024990in}}%
\pgfpathlineto{\pgfqpoint{1.582769in}{1.254955in}}%
\pgfpathlineto{\pgfqpoint{1.583152in}{1.254955in}}%
\pgfpathlineto{\pgfqpoint{1.583918in}{1.061576in}}%
\pgfpathlineto{\pgfqpoint{1.584683in}{1.197464in}}%
\pgfpathlineto{\pgfqpoint{1.585066in}{1.197464in}}%
\pgfpathlineto{\pgfqpoint{1.585449in}{1.066802in}}%
\pgfpathlineto{\pgfqpoint{1.586598in}{1.207917in}}%
\pgfpathlineto{\pgfqpoint{1.586980in}{1.207917in}}%
\pgfpathlineto{\pgfqpoint{1.588129in}{1.281088in}}%
\pgfpathlineto{\pgfqpoint{1.588512in}{1.092935in}}%
\pgfpathlineto{\pgfqpoint{1.588895in}{1.092935in}}%
\pgfpathlineto{\pgfqpoint{1.588895in}{1.087708in}}%
\pgfpathlineto{\pgfqpoint{1.589660in}{1.187011in}}%
\pgfpathlineto{\pgfqpoint{1.590426in}{1.181785in}}%
\pgfpathlineto{\pgfqpoint{1.590809in}{1.181785in}}%
\pgfpathlineto{\pgfqpoint{1.591192in}{1.056349in}}%
\pgfpathlineto{\pgfqpoint{1.592340in}{1.087708in}}%
\pgfpathlineto{\pgfqpoint{1.592723in}{1.087708in}}%
\pgfpathlineto{\pgfqpoint{1.593106in}{1.223597in}}%
\pgfpathlineto{\pgfqpoint{1.594254in}{1.145199in}}%
\pgfpathlineto{\pgfqpoint{1.594637in}{1.145199in}}%
\pgfpathlineto{\pgfqpoint{1.595785in}{1.197464in}}%
\pgfpathlineto{\pgfqpoint{1.595403in}{1.004084in}}%
\pgfpathlineto{\pgfqpoint{1.596168in}{1.160879in}}%
\pgfpathlineto{\pgfqpoint{1.596551in}{1.160879in}}%
\pgfpathlineto{\pgfqpoint{1.597700in}{1.077255in}}%
\pgfpathlineto{\pgfqpoint{1.598082in}{1.134746in}}%
\pgfpathlineto{\pgfqpoint{1.598465in}{1.134746in}}%
\pgfpathlineto{\pgfqpoint{1.598465in}{1.155652in}}%
\pgfpathlineto{\pgfqpoint{1.599231in}{1.119067in}}%
\pgfpathlineto{\pgfqpoint{1.599997in}{1.134746in}}%
\pgfpathlineto{\pgfqpoint{1.600379in}{1.134746in}}%
\pgfpathlineto{\pgfqpoint{1.600379in}{1.160879in}}%
\pgfpathlineto{\pgfqpoint{1.601145in}{1.108614in}}%
\pgfpathlineto{\pgfqpoint{1.601911in}{1.160879in}}%
\pgfpathlineto{\pgfqpoint{1.602294in}{1.160879in}}%
\pgfpathlineto{\pgfqpoint{1.603825in}{1.239276in}}%
\pgfpathlineto{\pgfqpoint{1.604208in}{1.239276in}}%
\pgfpathlineto{\pgfqpoint{1.604973in}{1.124293in}}%
\pgfpathlineto{\pgfqpoint{1.605739in}{1.160879in}}%
\pgfpathlineto{\pgfqpoint{1.606122in}{1.160879in}}%
\pgfpathlineto{\pgfqpoint{1.606888in}{1.192238in}}%
\pgfpathlineto{\pgfqpoint{1.607653in}{1.061576in}}%
\pgfpathlineto{\pgfqpoint{1.608036in}{1.061576in}}%
\pgfpathlineto{\pgfqpoint{1.608036in}{1.223597in}}%
\pgfpathlineto{\pgfqpoint{1.609567in}{1.134746in}}%
\pgfpathlineto{\pgfqpoint{1.609950in}{1.134746in}}%
\pgfpathlineto{\pgfqpoint{1.611482in}{1.244503in}}%
\pgfpathlineto{\pgfqpoint{1.611864in}{1.244503in}}%
\pgfpathlineto{\pgfqpoint{1.611864in}{1.092935in}}%
\pgfpathlineto{\pgfqpoint{1.613396in}{1.129520in}}%
\pgfpathlineto{\pgfqpoint{1.613779in}{1.129520in}}%
\pgfpathlineto{\pgfqpoint{1.614927in}{1.103388in}}%
\pgfpathlineto{\pgfqpoint{1.614161in}{1.213144in}}%
\pgfpathlineto{\pgfqpoint{1.615310in}{1.160879in}}%
\pgfpathlineto{\pgfqpoint{1.615693in}{1.160879in}}%
\pgfpathlineto{\pgfqpoint{1.615693in}{1.281088in}}%
\pgfpathlineto{\pgfqpoint{1.616841in}{1.098161in}}%
\pgfpathlineto{\pgfqpoint{1.617224in}{1.207917in}}%
\pgfpathlineto{\pgfqpoint{1.617607in}{1.207917in}}%
\pgfpathlineto{\pgfqpoint{1.617607in}{1.249729in}}%
\pgfpathlineto{\pgfqpoint{1.618755in}{1.004084in}}%
\pgfpathlineto{\pgfqpoint{1.619138in}{1.077255in}}%
\pgfpathlineto{\pgfqpoint{1.619521in}{1.077255in}}%
\pgfpathlineto{\pgfqpoint{1.621052in}{1.218370in}}%
\pgfpathlineto{\pgfqpoint{1.621435in}{1.218370in}}%
\pgfpathlineto{\pgfqpoint{1.622201in}{1.108614in}}%
\pgfpathlineto{\pgfqpoint{1.622966in}{1.228823in}}%
\pgfpathlineto{\pgfqpoint{1.623349in}{1.228823in}}%
\pgfpathlineto{\pgfqpoint{1.624498in}{1.009311in}}%
\pgfpathlineto{\pgfqpoint{1.624881in}{1.181785in}}%
\pgfpathlineto{\pgfqpoint{1.625263in}{1.181785in}}%
\pgfpathlineto{\pgfqpoint{1.626029in}{1.072029in}}%
\pgfpathlineto{\pgfqpoint{1.626795in}{1.087708in}}%
\pgfpathlineto{\pgfqpoint{1.627178in}{1.087708in}}%
\pgfpathlineto{\pgfqpoint{1.628709in}{1.207917in}}%
\pgfpathlineto{\pgfqpoint{1.629092in}{1.207917in}}%
\pgfpathlineto{\pgfqpoint{1.629475in}{1.239276in}}%
\pgfpathlineto{\pgfqpoint{1.630623in}{1.072029in}}%
\pgfpathlineto{\pgfqpoint{1.631006in}{1.072029in}}%
\pgfpathlineto{\pgfqpoint{1.632154in}{1.197464in}}%
\pgfpathlineto{\pgfqpoint{1.632537in}{1.087708in}}%
\pgfpathlineto{\pgfqpoint{1.632920in}{1.087708in}}%
\pgfpathlineto{\pgfqpoint{1.633303in}{1.171332in}}%
\pgfpathlineto{\pgfqpoint{1.634068in}{1.056349in}}%
\pgfpathlineto{\pgfqpoint{1.634451in}{1.113840in}}%
\pgfpathlineto{\pgfqpoint{1.634834in}{1.113840in}}%
\pgfpathlineto{\pgfqpoint{1.634834in}{1.171332in}}%
\pgfpathlineto{\pgfqpoint{1.635217in}{1.061576in}}%
\pgfpathlineto{\pgfqpoint{1.636365in}{1.108614in}}%
\pgfpathlineto{\pgfqpoint{1.636748in}{1.108614in}}%
\pgfpathlineto{\pgfqpoint{1.637514in}{1.197464in}}%
\pgfpathlineto{\pgfqpoint{1.638280in}{1.087708in}}%
\pgfpathlineto{\pgfqpoint{1.639045in}{1.087708in}}%
\pgfpathlineto{\pgfqpoint{1.639811in}{1.160879in}}%
\pgfpathlineto{\pgfqpoint{1.639428in}{1.056349in}}%
\pgfpathlineto{\pgfqpoint{1.640577in}{1.160879in}}%
\pgfpathlineto{\pgfqpoint{1.640959in}{1.160879in}}%
\pgfpathlineto{\pgfqpoint{1.640959in}{1.139973in}}%
\pgfpathlineto{\pgfqpoint{1.641342in}{1.228823in}}%
\pgfpathlineto{\pgfqpoint{1.642491in}{1.160879in}}%
\pgfpathlineto{\pgfqpoint{1.642874in}{1.160879in}}%
\pgfpathlineto{\pgfqpoint{1.643639in}{1.249729in}}%
\pgfpathlineto{\pgfqpoint{1.644405in}{1.119067in}}%
\pgfpathlineto{\pgfqpoint{1.644788in}{1.119067in}}%
\pgfpathlineto{\pgfqpoint{1.645936in}{1.187011in}}%
\pgfpathlineto{\pgfqpoint{1.645553in}{1.103388in}}%
\pgfpathlineto{\pgfqpoint{1.646319in}{1.103388in}}%
\pgfpathlineto{\pgfqpoint{1.646702in}{1.103388in}}%
\pgfpathlineto{\pgfqpoint{1.647850in}{1.197464in}}%
\pgfpathlineto{\pgfqpoint{1.648233in}{1.160879in}}%
\pgfpathlineto{\pgfqpoint{1.648616in}{1.160879in}}%
\pgfpathlineto{\pgfqpoint{1.648616in}{1.213144in}}%
\pgfpathlineto{\pgfqpoint{1.649382in}{1.108614in}}%
\pgfpathlineto{\pgfqpoint{1.650147in}{1.145199in}}%
\pgfpathlineto{\pgfqpoint{1.650530in}{1.145199in}}%
\pgfpathlineto{\pgfqpoint{1.650530in}{1.265408in}}%
\pgfpathlineto{\pgfqpoint{1.652062in}{1.072029in}}%
\pgfpathlineto{\pgfqpoint{1.652444in}{1.072029in}}%
\pgfpathlineto{\pgfqpoint{1.652444in}{1.181785in}}%
\pgfpathlineto{\pgfqpoint{1.652827in}{1.045896in}}%
\pgfpathlineto{\pgfqpoint{1.653976in}{1.051123in}}%
\pgfpathlineto{\pgfqpoint{1.654358in}{1.051123in}}%
\pgfpathlineto{\pgfqpoint{1.654741in}{1.228823in}}%
\pgfpathlineto{\pgfqpoint{1.655890in}{1.150426in}}%
\pgfpathlineto{\pgfqpoint{1.656273in}{1.150426in}}%
\pgfpathlineto{\pgfqpoint{1.657038in}{1.223597in}}%
\pgfpathlineto{\pgfqpoint{1.656655in}{1.119067in}}%
\pgfpathlineto{\pgfqpoint{1.657804in}{1.150426in}}%
\pgfpathlineto{\pgfqpoint{1.658187in}{1.150426in}}%
\pgfpathlineto{\pgfqpoint{1.658187in}{1.098161in}}%
\pgfpathlineto{\pgfqpoint{1.658952in}{1.218370in}}%
\pgfpathlineto{\pgfqpoint{1.659718in}{1.124293in}}%
\pgfpathlineto{\pgfqpoint{1.660101in}{1.124293in}}%
\pgfpathlineto{\pgfqpoint{1.660101in}{1.092935in}}%
\pgfpathlineto{\pgfqpoint{1.661249in}{1.192238in}}%
\pgfpathlineto{\pgfqpoint{1.661632in}{1.129520in}}%
\pgfpathlineto{\pgfqpoint{1.662015in}{1.129520in}}%
\pgfpathlineto{\pgfqpoint{1.662015in}{1.166105in}}%
\pgfpathlineto{\pgfqpoint{1.663164in}{1.113840in}}%
\pgfpathlineto{\pgfqpoint{1.663546in}{1.129520in}}%
\pgfpathlineto{\pgfqpoint{1.663929in}{1.129520in}}%
\pgfpathlineto{\pgfqpoint{1.665078in}{1.234050in}}%
\pgfpathlineto{\pgfqpoint{1.664695in}{1.077255in}}%
\pgfpathlineto{\pgfqpoint{1.665461in}{1.145199in}}%
\pgfpathlineto{\pgfqpoint{1.665843in}{1.145199in}}%
\pgfpathlineto{\pgfqpoint{1.666609in}{1.077255in}}%
\pgfpathlineto{\pgfqpoint{1.666226in}{1.176558in}}%
\pgfpathlineto{\pgfqpoint{1.667375in}{1.103388in}}%
\pgfpathlineto{\pgfqpoint{1.667758in}{1.103388in}}%
\pgfpathlineto{\pgfqpoint{1.668906in}{1.218370in}}%
\pgfpathlineto{\pgfqpoint{1.669289in}{1.192238in}}%
\pgfpathlineto{\pgfqpoint{1.669672in}{1.192238in}}%
\pgfpathlineto{\pgfqpoint{1.671203in}{1.087708in}}%
\pgfpathlineto{\pgfqpoint{1.671586in}{1.087708in}}%
\pgfpathlineto{\pgfqpoint{1.671969in}{1.187011in}}%
\pgfpathlineto{\pgfqpoint{1.673117in}{1.139973in}}%
\pgfpathlineto{\pgfqpoint{1.673500in}{1.139973in}}%
\pgfpathlineto{\pgfqpoint{1.673883in}{1.223597in}}%
\pgfpathlineto{\pgfqpoint{1.674648in}{1.066802in}}%
\pgfpathlineto{\pgfqpoint{1.675031in}{1.207917in}}%
\pgfpathlineto{\pgfqpoint{1.675414in}{1.207917in}}%
\pgfpathlineto{\pgfqpoint{1.676563in}{1.281088in}}%
\pgfpathlineto{\pgfqpoint{1.676945in}{1.092935in}}%
\pgfpathlineto{\pgfqpoint{1.677328in}{1.092935in}}%
\pgfpathlineto{\pgfqpoint{1.678094in}{1.192238in}}%
\pgfpathlineto{\pgfqpoint{1.678860in}{1.124293in}}%
\pgfpathlineto{\pgfqpoint{1.679242in}{1.124293in}}%
\pgfpathlineto{\pgfqpoint{1.680391in}{1.171332in}}%
\pgfpathlineto{\pgfqpoint{1.680008in}{1.103388in}}%
\pgfpathlineto{\pgfqpoint{1.680774in}{1.166105in}}%
\pgfpathlineto{\pgfqpoint{1.681539in}{1.166105in}}%
\pgfpathlineto{\pgfqpoint{1.681539in}{1.176558in}}%
\pgfpathlineto{\pgfqpoint{1.683071in}{1.098161in}}%
\pgfpathlineto{\pgfqpoint{1.683454in}{1.098161in}}%
\pgfpathlineto{\pgfqpoint{1.684219in}{1.077255in}}%
\pgfpathlineto{\pgfqpoint{1.684985in}{1.197464in}}%
\pgfpathlineto{\pgfqpoint{1.685751in}{1.197464in}}%
\pgfpathlineto{\pgfqpoint{1.685751in}{1.061576in}}%
\pgfpathlineto{\pgfqpoint{1.687282in}{1.092935in}}%
\pgfpathlineto{\pgfqpoint{1.687665in}{1.092935in}}%
\pgfpathlineto{\pgfqpoint{1.688048in}{1.176558in}}%
\pgfpathlineto{\pgfqpoint{1.689196in}{1.171332in}}%
\pgfpathlineto{\pgfqpoint{1.689579in}{1.171332in}}%
\pgfpathlineto{\pgfqpoint{1.690727in}{1.270635in}}%
\pgfpathlineto{\pgfqpoint{1.691110in}{1.035443in}}%
\pgfpathlineto{\pgfqpoint{1.691493in}{1.035443in}}%
\pgfpathlineto{\pgfqpoint{1.691493in}{1.207917in}}%
\pgfpathlineto{\pgfqpoint{1.693024in}{1.077255in}}%
\pgfpathlineto{\pgfqpoint{1.693407in}{1.077255in}}%
\pgfpathlineto{\pgfqpoint{1.693407in}{1.228823in}}%
\pgfpathlineto{\pgfqpoint{1.694938in}{1.202691in}}%
\pgfpathlineto{\pgfqpoint{1.695321in}{1.202691in}}%
\pgfpathlineto{\pgfqpoint{1.695321in}{1.239276in}}%
\pgfpathlineto{\pgfqpoint{1.696470in}{1.098161in}}%
\pgfpathlineto{\pgfqpoint{1.696853in}{1.139973in}}%
\pgfpathlineto{\pgfqpoint{1.697235in}{1.139973in}}%
\pgfpathlineto{\pgfqpoint{1.697618in}{1.077255in}}%
\pgfpathlineto{\pgfqpoint{1.698001in}{1.281088in}}%
\pgfpathlineto{\pgfqpoint{1.698767in}{1.160879in}}%
\pgfpathlineto{\pgfqpoint{1.699150in}{1.160879in}}%
\pgfpathlineto{\pgfqpoint{1.700298in}{1.192238in}}%
\pgfpathlineto{\pgfqpoint{1.699532in}{1.113840in}}%
\pgfpathlineto{\pgfqpoint{1.700681in}{1.176558in}}%
\pgfpathlineto{\pgfqpoint{1.701064in}{1.176558in}}%
\pgfpathlineto{\pgfqpoint{1.701447in}{1.072029in}}%
\pgfpathlineto{\pgfqpoint{1.702595in}{1.192238in}}%
\pgfpathlineto{\pgfqpoint{1.702978in}{1.192238in}}%
\pgfpathlineto{\pgfqpoint{1.702978in}{1.270635in}}%
\pgfpathlineto{\pgfqpoint{1.704126in}{1.113840in}}%
\pgfpathlineto{\pgfqpoint{1.704509in}{1.197464in}}%
\pgfpathlineto{\pgfqpoint{1.704892in}{1.197464in}}%
\pgfpathlineto{\pgfqpoint{1.704892in}{1.249729in}}%
\pgfpathlineto{\pgfqpoint{1.706041in}{1.092935in}}%
\pgfpathlineto{\pgfqpoint{1.706423in}{1.228823in}}%
\pgfpathlineto{\pgfqpoint{1.706806in}{1.228823in}}%
\pgfpathlineto{\pgfqpoint{1.708338in}{1.087708in}}%
\pgfpathlineto{\pgfqpoint{1.708720in}{1.087708in}}%
\pgfpathlineto{\pgfqpoint{1.710252in}{1.260182in}}%
\pgfpathlineto{\pgfqpoint{1.710635in}{1.260182in}}%
\pgfpathlineto{\pgfqpoint{1.711783in}{1.129520in}}%
\pgfpathlineto{\pgfqpoint{1.712166in}{1.129520in}}%
\pgfpathlineto{\pgfqpoint{1.712549in}{1.129520in}}%
\pgfpathlineto{\pgfqpoint{1.712549in}{1.082482in}}%
\pgfpathlineto{\pgfqpoint{1.713697in}{1.197464in}}%
\pgfpathlineto{\pgfqpoint{1.714080in}{1.187011in}}%
\pgfpathlineto{\pgfqpoint{1.714463in}{1.187011in}}%
\pgfpathlineto{\pgfqpoint{1.714846in}{1.082482in}}%
\pgfpathlineto{\pgfqpoint{1.715228in}{1.192238in}}%
\pgfpathlineto{\pgfqpoint{1.715994in}{1.129520in}}%
\pgfpathlineto{\pgfqpoint{1.716377in}{1.129520in}}%
\pgfpathlineto{\pgfqpoint{1.716377in}{1.213144in}}%
\pgfpathlineto{\pgfqpoint{1.716760in}{1.087708in}}%
\pgfpathlineto{\pgfqpoint{1.717908in}{1.113840in}}%
\pgfpathlineto{\pgfqpoint{1.718291in}{1.113840in}}%
\pgfpathlineto{\pgfqpoint{1.718291in}{1.254955in}}%
\pgfpathlineto{\pgfqpoint{1.719822in}{1.160879in}}%
\pgfpathlineto{\pgfqpoint{1.720205in}{1.160879in}}%
\pgfpathlineto{\pgfqpoint{1.720588in}{1.228823in}}%
\pgfpathlineto{\pgfqpoint{1.721737in}{1.160879in}}%
\pgfpathlineto{\pgfqpoint{1.722119in}{1.160879in}}%
\pgfpathlineto{\pgfqpoint{1.722119in}{1.249729in}}%
\pgfpathlineto{\pgfqpoint{1.723651in}{1.082482in}}%
\pgfpathlineto{\pgfqpoint{1.724034in}{1.082482in}}%
\pgfpathlineto{\pgfqpoint{1.724034in}{1.244503in}}%
\pgfpathlineto{\pgfqpoint{1.725565in}{1.181785in}}%
\pgfpathlineto{\pgfqpoint{1.725948in}{1.181785in}}%
\pgfpathlineto{\pgfqpoint{1.727096in}{1.129520in}}%
\pgfpathlineto{\pgfqpoint{1.727479in}{1.234050in}}%
\pgfpathlineto{\pgfqpoint{1.727862in}{1.234050in}}%
\pgfpathlineto{\pgfqpoint{1.727862in}{1.239276in}}%
\pgfpathlineto{\pgfqpoint{1.728628in}{1.145199in}}%
\pgfpathlineto{\pgfqpoint{1.729393in}{1.202691in}}%
\pgfpathlineto{\pgfqpoint{1.729776in}{1.202691in}}%
\pgfpathlineto{\pgfqpoint{1.729776in}{1.108614in}}%
\pgfpathlineto{\pgfqpoint{1.730542in}{1.249729in}}%
\pgfpathlineto{\pgfqpoint{1.731307in}{1.171332in}}%
\pgfpathlineto{\pgfqpoint{1.731690in}{1.171332in}}%
\pgfpathlineto{\pgfqpoint{1.732073in}{1.124293in}}%
\pgfpathlineto{\pgfqpoint{1.733221in}{1.207917in}}%
\pgfpathlineto{\pgfqpoint{1.733604in}{1.207917in}}%
\pgfpathlineto{\pgfqpoint{1.733987in}{1.275861in}}%
\pgfpathlineto{\pgfqpoint{1.735136in}{1.134746in}}%
\pgfpathlineto{\pgfqpoint{1.735518in}{1.134746in}}%
\pgfpathlineto{\pgfqpoint{1.735901in}{1.103388in}}%
\pgfpathlineto{\pgfqpoint{1.737050in}{1.187011in}}%
\pgfpathlineto{\pgfqpoint{1.737433in}{1.187011in}}%
\pgfpathlineto{\pgfqpoint{1.737815in}{1.244503in}}%
\pgfpathlineto{\pgfqpoint{1.738964in}{1.098161in}}%
\pgfpathlineto{\pgfqpoint{1.739347in}{1.098161in}}%
\pgfpathlineto{\pgfqpoint{1.739347in}{1.260182in}}%
\pgfpathlineto{\pgfqpoint{1.740878in}{1.082482in}}%
\pgfpathlineto{\pgfqpoint{1.741261in}{1.082482in}}%
\pgfpathlineto{\pgfqpoint{1.742409in}{1.343806in}}%
\pgfpathlineto{\pgfqpoint{1.742792in}{1.249729in}}%
\pgfpathlineto{\pgfqpoint{1.743175in}{1.249729in}}%
\pgfpathlineto{\pgfqpoint{1.744324in}{1.145199in}}%
\pgfpathlineto{\pgfqpoint{1.744706in}{1.150426in}}%
\pgfpathlineto{\pgfqpoint{1.745089in}{1.150426in}}%
\pgfpathlineto{\pgfqpoint{1.745472in}{1.113840in}}%
\pgfpathlineto{\pgfqpoint{1.746621in}{1.223597in}}%
\pgfpathlineto{\pgfqpoint{1.747003in}{1.223597in}}%
\pgfpathlineto{\pgfqpoint{1.747386in}{1.092935in}}%
\pgfpathlineto{\pgfqpoint{1.748535in}{1.166105in}}%
\pgfpathlineto{\pgfqpoint{1.748918in}{1.166105in}}%
\pgfpathlineto{\pgfqpoint{1.749300in}{1.092935in}}%
\pgfpathlineto{\pgfqpoint{1.749300in}{1.218370in}}%
\pgfpathlineto{\pgfqpoint{1.750449in}{1.197464in}}%
\pgfpathlineto{\pgfqpoint{1.750832in}{1.197464in}}%
\pgfpathlineto{\pgfqpoint{1.750832in}{1.265408in}}%
\pgfpathlineto{\pgfqpoint{1.751597in}{1.098161in}}%
\pgfpathlineto{\pgfqpoint{1.752363in}{1.160879in}}%
\pgfpathlineto{\pgfqpoint{1.752746in}{1.160879in}}%
\pgfpathlineto{\pgfqpoint{1.752746in}{1.249729in}}%
\pgfpathlineto{\pgfqpoint{1.753129in}{1.150426in}}%
\pgfpathlineto{\pgfqpoint{1.754277in}{1.218370in}}%
\pgfpathlineto{\pgfqpoint{1.754660in}{1.218370in}}%
\pgfpathlineto{\pgfqpoint{1.755043in}{1.129520in}}%
\pgfpathlineto{\pgfqpoint{1.755043in}{1.254955in}}%
\pgfpathlineto{\pgfqpoint{1.756191in}{1.145199in}}%
\pgfpathlineto{\pgfqpoint{1.756574in}{1.145199in}}%
\pgfpathlineto{\pgfqpoint{1.756574in}{1.129520in}}%
\pgfpathlineto{\pgfqpoint{1.757340in}{1.286314in}}%
\pgfpathlineto{\pgfqpoint{1.758105in}{1.166105in}}%
\pgfpathlineto{\pgfqpoint{1.758488in}{1.166105in}}%
\pgfpathlineto{\pgfqpoint{1.758871in}{1.286314in}}%
\pgfpathlineto{\pgfqpoint{1.760020in}{1.166105in}}%
\pgfpathlineto{\pgfqpoint{1.760402in}{1.166105in}}%
\pgfpathlineto{\pgfqpoint{1.760402in}{1.317673in}}%
\pgfpathlineto{\pgfqpoint{1.761934in}{1.134746in}}%
\pgfpathlineto{\pgfqpoint{1.762317in}{1.134746in}}%
\pgfpathlineto{\pgfqpoint{1.763082in}{1.254955in}}%
\pgfpathlineto{\pgfqpoint{1.763848in}{1.254955in}}%
\pgfpathlineto{\pgfqpoint{1.764231in}{1.254955in}}%
\pgfpathlineto{\pgfqpoint{1.764231in}{1.045896in}}%
\pgfpathlineto{\pgfqpoint{1.765762in}{1.166105in}}%
\pgfpathlineto{\pgfqpoint{1.766145in}{1.166105in}}%
\pgfpathlineto{\pgfqpoint{1.766528in}{1.218370in}}%
\pgfpathlineto{\pgfqpoint{1.767676in}{1.113840in}}%
\pgfpathlineto{\pgfqpoint{1.768059in}{1.113840in}}%
\pgfpathlineto{\pgfqpoint{1.768442in}{1.239276in}}%
\pgfpathlineto{\pgfqpoint{1.769590in}{1.192238in}}%
\pgfpathlineto{\pgfqpoint{1.769973in}{1.192238in}}%
\pgfpathlineto{\pgfqpoint{1.769973in}{1.171332in}}%
\pgfpathlineto{\pgfqpoint{1.770356in}{1.296767in}}%
\pgfpathlineto{\pgfqpoint{1.771504in}{1.187011in}}%
\pgfpathlineto{\pgfqpoint{1.771887in}{1.187011in}}%
\pgfpathlineto{\pgfqpoint{1.771887in}{1.124293in}}%
\pgfpathlineto{\pgfqpoint{1.772653in}{1.249729in}}%
\pgfpathlineto{\pgfqpoint{1.773419in}{1.124293in}}%
\pgfpathlineto{\pgfqpoint{1.773801in}{1.124293in}}%
\pgfpathlineto{\pgfqpoint{1.774950in}{1.254955in}}%
\pgfpathlineto{\pgfqpoint{1.775333in}{1.134746in}}%
\pgfpathlineto{\pgfqpoint{1.775716in}{1.134746in}}%
\pgfpathlineto{\pgfqpoint{1.776481in}{1.270635in}}%
\pgfpathlineto{\pgfqpoint{1.777247in}{1.223597in}}%
\pgfpathlineto{\pgfqpoint{1.777630in}{1.223597in}}%
\pgfpathlineto{\pgfqpoint{1.778395in}{1.139973in}}%
\pgfpathlineto{\pgfqpoint{1.779161in}{1.176558in}}%
\pgfpathlineto{\pgfqpoint{1.779544in}{1.176558in}}%
\pgfpathlineto{\pgfqpoint{1.780692in}{1.275861in}}%
\pgfpathlineto{\pgfqpoint{1.779927in}{1.166105in}}%
\pgfpathlineto{\pgfqpoint{1.781075in}{1.176558in}}%
\pgfpathlineto{\pgfqpoint{1.781458in}{1.176558in}}%
\pgfpathlineto{\pgfqpoint{1.781458in}{1.281088in}}%
\pgfpathlineto{\pgfqpoint{1.782224in}{1.119067in}}%
\pgfpathlineto{\pgfqpoint{1.782989in}{1.155652in}}%
\pgfpathlineto{\pgfqpoint{1.783372in}{1.155652in}}%
\pgfpathlineto{\pgfqpoint{1.783755in}{1.239276in}}%
\pgfpathlineto{\pgfqpoint{1.784904in}{1.223597in}}%
\pgfpathlineto{\pgfqpoint{1.785286in}{1.223597in}}%
\pgfpathlineto{\pgfqpoint{1.785286in}{1.254955in}}%
\pgfpathlineto{\pgfqpoint{1.786435in}{1.150426in}}%
\pgfpathlineto{\pgfqpoint{1.786818in}{1.228823in}}%
\pgfpathlineto{\pgfqpoint{1.787201in}{1.228823in}}%
\pgfpathlineto{\pgfqpoint{1.787583in}{1.119067in}}%
\pgfpathlineto{\pgfqpoint{1.788732in}{1.333353in}}%
\pgfpathlineto{\pgfqpoint{1.789115in}{1.333353in}}%
\pgfpathlineto{\pgfqpoint{1.789497in}{1.066802in}}%
\pgfpathlineto{\pgfqpoint{1.790646in}{1.249729in}}%
\pgfpathlineto{\pgfqpoint{1.791029in}{1.249729in}}%
\pgfpathlineto{\pgfqpoint{1.791412in}{1.181785in}}%
\pgfpathlineto{\pgfqpoint{1.792560in}{1.338579in}}%
\pgfpathlineto{\pgfqpoint{1.792943in}{1.338579in}}%
\pgfpathlineto{\pgfqpoint{1.793709in}{1.150426in}}%
\pgfpathlineto{\pgfqpoint{1.794474in}{1.176558in}}%
\pgfpathlineto{\pgfqpoint{1.794857in}{1.176558in}}%
\pgfpathlineto{\pgfqpoint{1.795623in}{1.328126in}}%
\pgfpathlineto{\pgfqpoint{1.796006in}{1.082482in}}%
\pgfpathlineto{\pgfqpoint{1.796388in}{1.249729in}}%
\pgfpathlineto{\pgfqpoint{1.796771in}{1.249729in}}%
\pgfpathlineto{\pgfqpoint{1.796771in}{1.207917in}}%
\pgfpathlineto{\pgfqpoint{1.798303in}{1.207917in}}%
\pgfpathlineto{\pgfqpoint{1.798685in}{1.207917in}}%
\pgfpathlineto{\pgfqpoint{1.799068in}{1.239276in}}%
\pgfpathlineto{\pgfqpoint{1.800217in}{1.160879in}}%
\pgfpathlineto{\pgfqpoint{1.800600in}{1.160879in}}%
\pgfpathlineto{\pgfqpoint{1.800600in}{1.113840in}}%
\pgfpathlineto{\pgfqpoint{1.801748in}{1.333353in}}%
\pgfpathlineto{\pgfqpoint{1.802131in}{1.260182in}}%
\pgfpathlineto{\pgfqpoint{1.802514in}{1.260182in}}%
\pgfpathlineto{\pgfqpoint{1.803279in}{1.187011in}}%
\pgfpathlineto{\pgfqpoint{1.804045in}{1.338579in}}%
\pgfpathlineto{\pgfqpoint{1.804428in}{1.338579in}}%
\pgfpathlineto{\pgfqpoint{1.805194in}{1.139973in}}%
\pgfpathlineto{\pgfqpoint{1.805959in}{1.265408in}}%
\pgfpathlineto{\pgfqpoint{1.806342in}{1.265408in}}%
\pgfpathlineto{\pgfqpoint{1.806725in}{1.119067in}}%
\pgfpathlineto{\pgfqpoint{1.807873in}{1.171332in}}%
\pgfpathlineto{\pgfqpoint{1.808256in}{1.171332in}}%
\pgfpathlineto{\pgfqpoint{1.808256in}{1.061576in}}%
\pgfpathlineto{\pgfqpoint{1.809405in}{1.333353in}}%
\pgfpathlineto{\pgfqpoint{1.809787in}{1.213144in}}%
\pgfpathlineto{\pgfqpoint{1.810170in}{1.213144in}}%
\pgfpathlineto{\pgfqpoint{1.810170in}{1.202691in}}%
\pgfpathlineto{\pgfqpoint{1.810553in}{1.281088in}}%
\pgfpathlineto{\pgfqpoint{1.811702in}{1.223597in}}%
\pgfpathlineto{\pgfqpoint{1.812084in}{1.223597in}}%
\pgfpathlineto{\pgfqpoint{1.812084in}{1.249729in}}%
\pgfpathlineto{\pgfqpoint{1.812850in}{1.192238in}}%
\pgfpathlineto{\pgfqpoint{1.813616in}{1.244503in}}%
\pgfpathlineto{\pgfqpoint{1.813999in}{1.244503in}}%
\pgfpathlineto{\pgfqpoint{1.815147in}{1.155652in}}%
\pgfpathlineto{\pgfqpoint{1.815530in}{1.275861in}}%
\pgfpathlineto{\pgfqpoint{1.815913in}{1.275861in}}%
\pgfpathlineto{\pgfqpoint{1.815913in}{1.181785in}}%
\pgfpathlineto{\pgfqpoint{1.817061in}{1.301994in}}%
\pgfpathlineto{\pgfqpoint{1.817444in}{1.275861in}}%
\pgfpathlineto{\pgfqpoint{1.817827in}{1.275861in}}%
\pgfpathlineto{\pgfqpoint{1.817827in}{1.296767in}}%
\pgfpathlineto{\pgfqpoint{1.818975in}{1.103388in}}%
\pgfpathlineto{\pgfqpoint{1.819358in}{1.197464in}}%
\pgfpathlineto{\pgfqpoint{1.819741in}{1.197464in}}%
\pgfpathlineto{\pgfqpoint{1.820124in}{1.124293in}}%
\pgfpathlineto{\pgfqpoint{1.821272in}{1.171332in}}%
\pgfpathlineto{\pgfqpoint{1.821655in}{1.171332in}}%
\pgfpathlineto{\pgfqpoint{1.821655in}{1.145199in}}%
\pgfpathlineto{\pgfqpoint{1.823187in}{1.244503in}}%
\pgfpathlineto{\pgfqpoint{1.823569in}{1.244503in}}%
\pgfpathlineto{\pgfqpoint{1.824335in}{1.427429in}}%
\pgfpathlineto{\pgfqpoint{1.825101in}{1.155652in}}%
\pgfpathlineto{\pgfqpoint{1.825484in}{1.155652in}}%
\pgfpathlineto{\pgfqpoint{1.826632in}{1.286314in}}%
\pgfpathlineto{\pgfqpoint{1.826249in}{1.098161in}}%
\pgfpathlineto{\pgfqpoint{1.827015in}{1.213144in}}%
\pgfpathlineto{\pgfqpoint{1.827398in}{1.213144in}}%
\pgfpathlineto{\pgfqpoint{1.828546in}{1.281088in}}%
\pgfpathlineto{\pgfqpoint{1.828929in}{1.150426in}}%
\pgfpathlineto{\pgfqpoint{1.829312in}{1.150426in}}%
\pgfpathlineto{\pgfqpoint{1.829312in}{1.254955in}}%
\pgfpathlineto{\pgfqpoint{1.830843in}{1.150426in}}%
\pgfpathlineto{\pgfqpoint{1.831226in}{1.150426in}}%
\pgfpathlineto{\pgfqpoint{1.831226in}{1.119067in}}%
\pgfpathlineto{\pgfqpoint{1.831992in}{1.307220in}}%
\pgfpathlineto{\pgfqpoint{1.832757in}{1.218370in}}%
\pgfpathlineto{\pgfqpoint{1.833140in}{1.218370in}}%
\pgfpathlineto{\pgfqpoint{1.834289in}{1.171332in}}%
\pgfpathlineto{\pgfqpoint{1.834671in}{1.249729in}}%
\pgfpathlineto{\pgfqpoint{1.835054in}{1.249729in}}%
\pgfpathlineto{\pgfqpoint{1.835820in}{1.171332in}}%
\pgfpathlineto{\pgfqpoint{1.835437in}{1.286314in}}%
\pgfpathlineto{\pgfqpoint{1.836586in}{1.207917in}}%
\pgfpathlineto{\pgfqpoint{1.837734in}{1.207917in}}%
\pgfpathlineto{\pgfqpoint{1.838883in}{1.166105in}}%
\pgfpathlineto{\pgfqpoint{1.838500in}{1.213144in}}%
\pgfpathlineto{\pgfqpoint{1.839265in}{1.181785in}}%
\pgfpathlineto{\pgfqpoint{1.839648in}{1.181785in}}%
\pgfpathlineto{\pgfqpoint{1.840797in}{1.317673in}}%
\pgfpathlineto{\pgfqpoint{1.841180in}{1.166105in}}%
\pgfpathlineto{\pgfqpoint{1.841562in}{1.166105in}}%
\pgfpathlineto{\pgfqpoint{1.841945in}{1.312447in}}%
\pgfpathlineto{\pgfqpoint{1.843094in}{1.171332in}}%
\pgfpathlineto{\pgfqpoint{1.843477in}{1.171332in}}%
\pgfpathlineto{\pgfqpoint{1.843859in}{1.244503in}}%
\pgfpathlineto{\pgfqpoint{1.843859in}{1.150426in}}%
\pgfpathlineto{\pgfqpoint{1.845008in}{1.176558in}}%
\pgfpathlineto{\pgfqpoint{1.845391in}{1.176558in}}%
\pgfpathlineto{\pgfqpoint{1.846539in}{1.281088in}}%
\pgfpathlineto{\pgfqpoint{1.846922in}{1.155652in}}%
\pgfpathlineto{\pgfqpoint{1.847305in}{1.155652in}}%
\pgfpathlineto{\pgfqpoint{1.847305in}{1.260182in}}%
\pgfpathlineto{\pgfqpoint{1.848836in}{1.187011in}}%
\pgfpathlineto{\pgfqpoint{1.849219in}{1.187011in}}%
\pgfpathlineto{\pgfqpoint{1.850367in}{1.312447in}}%
\pgfpathlineto{\pgfqpoint{1.850750in}{1.150426in}}%
\pgfpathlineto{\pgfqpoint{1.851133in}{1.150426in}}%
\pgfpathlineto{\pgfqpoint{1.851133in}{1.281088in}}%
\pgfpathlineto{\pgfqpoint{1.851899in}{1.092935in}}%
\pgfpathlineto{\pgfqpoint{1.852664in}{1.192238in}}%
\pgfpathlineto{\pgfqpoint{1.853047in}{1.192238in}}%
\pgfpathlineto{\pgfqpoint{1.853813in}{1.322900in}}%
\pgfpathlineto{\pgfqpoint{1.854579in}{1.197464in}}%
\pgfpathlineto{\pgfqpoint{1.854961in}{1.197464in}}%
\pgfpathlineto{\pgfqpoint{1.854961in}{1.129520in}}%
\pgfpathlineto{\pgfqpoint{1.856493in}{1.260182in}}%
\pgfpathlineto{\pgfqpoint{1.856876in}{1.260182in}}%
\pgfpathlineto{\pgfqpoint{1.858407in}{1.103388in}}%
\pgfpathlineto{\pgfqpoint{1.858790in}{1.103388in}}%
\pgfpathlineto{\pgfqpoint{1.858790in}{1.223597in}}%
\pgfpathlineto{\pgfqpoint{1.859555in}{1.092935in}}%
\pgfpathlineto{\pgfqpoint{1.860321in}{1.197464in}}%
\pgfpathlineto{\pgfqpoint{1.860704in}{1.197464in}}%
\pgfpathlineto{\pgfqpoint{1.861852in}{1.281088in}}%
\pgfpathlineto{\pgfqpoint{1.861087in}{1.176558in}}%
\pgfpathlineto{\pgfqpoint{1.862235in}{1.234050in}}%
\pgfpathlineto{\pgfqpoint{1.863001in}{1.234050in}}%
\pgfpathlineto{\pgfqpoint{1.864149in}{1.275861in}}%
\pgfpathlineto{\pgfqpoint{1.864532in}{1.202691in}}%
\pgfpathlineto{\pgfqpoint{1.864915in}{1.202691in}}%
\pgfpathlineto{\pgfqpoint{1.866063in}{1.286314in}}%
\pgfpathlineto{\pgfqpoint{1.866446in}{1.098161in}}%
\pgfpathlineto{\pgfqpoint{1.866829in}{1.098161in}}%
\pgfpathlineto{\pgfqpoint{1.867212in}{1.354259in}}%
\pgfpathlineto{\pgfqpoint{1.868360in}{1.286314in}}%
\pgfpathlineto{\pgfqpoint{1.868743in}{1.286314in}}%
\pgfpathlineto{\pgfqpoint{1.869509in}{1.150426in}}%
\pgfpathlineto{\pgfqpoint{1.870275in}{1.322900in}}%
\pgfpathlineto{\pgfqpoint{1.870657in}{1.322900in}}%
\pgfpathlineto{\pgfqpoint{1.872189in}{1.155652in}}%
\pgfpathlineto{\pgfqpoint{1.872572in}{1.155652in}}%
\pgfpathlineto{\pgfqpoint{1.872572in}{1.291541in}}%
\pgfpathlineto{\pgfqpoint{1.874103in}{1.234050in}}%
\pgfpathlineto{\pgfqpoint{1.874486in}{1.234050in}}%
\pgfpathlineto{\pgfqpoint{1.874486in}{1.124293in}}%
\pgfpathlineto{\pgfqpoint{1.875634in}{1.296767in}}%
\pgfpathlineto{\pgfqpoint{1.876017in}{1.213144in}}%
\pgfpathlineto{\pgfqpoint{1.876400in}{1.213144in}}%
\pgfpathlineto{\pgfqpoint{1.877548in}{1.322900in}}%
\pgfpathlineto{\pgfqpoint{1.876783in}{1.134746in}}%
\pgfpathlineto{\pgfqpoint{1.877931in}{1.187011in}}%
\pgfpathlineto{\pgfqpoint{1.878314in}{1.187011in}}%
\pgfpathlineto{\pgfqpoint{1.878697in}{1.275861in}}%
\pgfpathlineto{\pgfqpoint{1.879845in}{1.213144in}}%
\pgfpathlineto{\pgfqpoint{1.880228in}{1.213144in}}%
\pgfpathlineto{\pgfqpoint{1.880228in}{1.181785in}}%
\pgfpathlineto{\pgfqpoint{1.881377in}{1.296767in}}%
\pgfpathlineto{\pgfqpoint{1.881760in}{1.192238in}}%
\pgfpathlineto{\pgfqpoint{1.882142in}{1.192238in}}%
\pgfpathlineto{\pgfqpoint{1.882142in}{1.270635in}}%
\pgfpathlineto{\pgfqpoint{1.882908in}{1.145199in}}%
\pgfpathlineto{\pgfqpoint{1.883674in}{1.228823in}}%
\pgfpathlineto{\pgfqpoint{1.884057in}{1.228823in}}%
\pgfpathlineto{\pgfqpoint{1.885205in}{1.265408in}}%
\pgfpathlineto{\pgfqpoint{1.885588in}{1.113840in}}%
\pgfpathlineto{\pgfqpoint{1.885971in}{1.113840in}}%
\pgfpathlineto{\pgfqpoint{1.886353in}{1.254955in}}%
\pgfpathlineto{\pgfqpoint{1.887502in}{1.108614in}}%
\pgfpathlineto{\pgfqpoint{1.887885in}{1.108614in}}%
\pgfpathlineto{\pgfqpoint{1.887885in}{1.265408in}}%
\pgfpathlineto{\pgfqpoint{1.889416in}{1.207917in}}%
\pgfpathlineto{\pgfqpoint{1.889799in}{1.207917in}}%
\pgfpathlineto{\pgfqpoint{1.890947in}{1.312447in}}%
\pgfpathlineto{\pgfqpoint{1.890565in}{1.160879in}}%
\pgfpathlineto{\pgfqpoint{1.891330in}{1.301994in}}%
\pgfpathlineto{\pgfqpoint{1.891713in}{1.301994in}}%
\pgfpathlineto{\pgfqpoint{1.891713in}{1.197464in}}%
\pgfpathlineto{\pgfqpoint{1.893244in}{1.218370in}}%
\pgfpathlineto{\pgfqpoint{1.893627in}{1.218370in}}%
\pgfpathlineto{\pgfqpoint{1.894393in}{1.160879in}}%
\pgfpathlineto{\pgfqpoint{1.894010in}{1.244503in}}%
\pgfpathlineto{\pgfqpoint{1.895159in}{1.218370in}}%
\pgfpathlineto{\pgfqpoint{1.895541in}{1.218370in}}%
\pgfpathlineto{\pgfqpoint{1.896307in}{1.155652in}}%
\pgfpathlineto{\pgfqpoint{1.897073in}{1.322900in}}%
\pgfpathlineto{\pgfqpoint{1.897456in}{1.322900in}}%
\pgfpathlineto{\pgfqpoint{1.898604in}{1.124293in}}%
\pgfpathlineto{\pgfqpoint{1.898987in}{1.234050in}}%
\pgfpathlineto{\pgfqpoint{1.899370in}{1.234050in}}%
\pgfpathlineto{\pgfqpoint{1.899370in}{1.213144in}}%
\pgfpathlineto{\pgfqpoint{1.900135in}{1.312447in}}%
\pgfpathlineto{\pgfqpoint{1.900901in}{1.213144in}}%
\pgfpathlineto{\pgfqpoint{1.901284in}{1.213144in}}%
\pgfpathlineto{\pgfqpoint{1.902050in}{1.166105in}}%
\pgfpathlineto{\pgfqpoint{1.901667in}{1.254955in}}%
\pgfpathlineto{\pgfqpoint{1.902815in}{1.244503in}}%
\pgfpathlineto{\pgfqpoint{1.903198in}{1.244503in}}%
\pgfpathlineto{\pgfqpoint{1.903581in}{1.213144in}}%
\pgfpathlineto{\pgfqpoint{1.903964in}{1.307220in}}%
\pgfpathlineto{\pgfqpoint{1.904729in}{1.281088in}}%
\pgfpathlineto{\pgfqpoint{1.905112in}{1.281088in}}%
\pgfpathlineto{\pgfqpoint{1.906643in}{1.145199in}}%
\pgfpathlineto{\pgfqpoint{1.907026in}{1.145199in}}%
\pgfpathlineto{\pgfqpoint{1.907409in}{1.307220in}}%
\pgfpathlineto{\pgfqpoint{1.907792in}{1.129520in}}%
\pgfpathlineto{\pgfqpoint{1.908558in}{1.228823in}}%
\pgfpathlineto{\pgfqpoint{1.908940in}{1.228823in}}%
\pgfpathlineto{\pgfqpoint{1.909323in}{1.333353in}}%
\pgfpathlineto{\pgfqpoint{1.910089in}{1.176558in}}%
\pgfpathlineto{\pgfqpoint{1.910472in}{1.213144in}}%
\pgfpathlineto{\pgfqpoint{1.910855in}{1.213144in}}%
\pgfpathlineto{\pgfqpoint{1.911237in}{1.139973in}}%
\pgfpathlineto{\pgfqpoint{1.911237in}{1.281088in}}%
\pgfpathlineto{\pgfqpoint{1.912386in}{1.207917in}}%
\pgfpathlineto{\pgfqpoint{1.912769in}{1.207917in}}%
\pgfpathlineto{\pgfqpoint{1.912769in}{1.260182in}}%
\pgfpathlineto{\pgfqpoint{1.914300in}{1.155652in}}%
\pgfpathlineto{\pgfqpoint{1.914683in}{1.155652in}}%
\pgfpathlineto{\pgfqpoint{1.915449in}{1.244503in}}%
\pgfpathlineto{\pgfqpoint{1.916214in}{1.108614in}}%
\pgfpathlineto{\pgfqpoint{1.916597in}{1.108614in}}%
\pgfpathlineto{\pgfqpoint{1.917746in}{1.396070in}}%
\pgfpathlineto{\pgfqpoint{1.918128in}{1.213144in}}%
\pgfpathlineto{\pgfqpoint{1.918511in}{1.213144in}}%
\pgfpathlineto{\pgfqpoint{1.918894in}{1.228823in}}%
\pgfpathlineto{\pgfqpoint{1.920043in}{1.108614in}}%
\pgfpathlineto{\pgfqpoint{1.920425in}{1.108614in}}%
\pgfpathlineto{\pgfqpoint{1.920808in}{1.301994in}}%
\pgfpathlineto{\pgfqpoint{1.921957in}{1.218370in}}%
\pgfpathlineto{\pgfqpoint{1.922340in}{1.218370in}}%
\pgfpathlineto{\pgfqpoint{1.923488in}{1.380391in}}%
\pgfpathlineto{\pgfqpoint{1.923871in}{1.301994in}}%
\pgfpathlineto{\pgfqpoint{1.924254in}{1.301994in}}%
\pgfpathlineto{\pgfqpoint{1.925402in}{1.155652in}}%
\pgfpathlineto{\pgfqpoint{1.925785in}{1.228823in}}%
\pgfpathlineto{\pgfqpoint{1.926168in}{1.228823in}}%
\pgfpathlineto{\pgfqpoint{1.926933in}{1.281088in}}%
\pgfpathlineto{\pgfqpoint{1.926551in}{1.197464in}}%
\pgfpathlineto{\pgfqpoint{1.927699in}{1.281088in}}%
\pgfpathlineto{\pgfqpoint{1.928082in}{1.281088in}}%
\pgfpathlineto{\pgfqpoint{1.928082in}{1.202691in}}%
\pgfpathlineto{\pgfqpoint{1.929613in}{1.249729in}}%
\pgfpathlineto{\pgfqpoint{1.929996in}{1.249729in}}%
\pgfpathlineto{\pgfqpoint{1.930379in}{1.275861in}}%
\pgfpathlineto{\pgfqpoint{1.931145in}{1.155652in}}%
\pgfpathlineto{\pgfqpoint{1.931527in}{1.192238in}}%
\pgfpathlineto{\pgfqpoint{1.931910in}{1.192238in}}%
\pgfpathlineto{\pgfqpoint{1.932676in}{1.119067in}}%
\pgfpathlineto{\pgfqpoint{1.933059in}{1.234050in}}%
\pgfpathlineto{\pgfqpoint{1.933442in}{1.129520in}}%
\pgfpathlineto{\pgfqpoint{1.933824in}{1.129520in}}%
\pgfpathlineto{\pgfqpoint{1.934973in}{1.254955in}}%
\pgfpathlineto{\pgfqpoint{1.935356in}{1.092935in}}%
\pgfpathlineto{\pgfqpoint{1.935739in}{1.092935in}}%
\pgfpathlineto{\pgfqpoint{1.937270in}{1.281088in}}%
\pgfpathlineto{\pgfqpoint{1.937653in}{1.281088in}}%
\pgfpathlineto{\pgfqpoint{1.938801in}{1.228823in}}%
\pgfpathlineto{\pgfqpoint{1.938036in}{1.354259in}}%
\pgfpathlineto{\pgfqpoint{1.939184in}{1.265408in}}%
\pgfpathlineto{\pgfqpoint{1.939567in}{1.265408in}}%
\pgfpathlineto{\pgfqpoint{1.939950in}{1.166105in}}%
\pgfpathlineto{\pgfqpoint{1.941098in}{1.197464in}}%
\pgfpathlineto{\pgfqpoint{1.941481in}{1.197464in}}%
\pgfpathlineto{\pgfqpoint{1.941481in}{1.181785in}}%
\pgfpathlineto{\pgfqpoint{1.942630in}{1.234050in}}%
\pgfpathlineto{\pgfqpoint{1.943012in}{1.202691in}}%
\pgfpathlineto{\pgfqpoint{1.943395in}{1.202691in}}%
\pgfpathlineto{\pgfqpoint{1.943395in}{1.286314in}}%
\pgfpathlineto{\pgfqpoint{1.944926in}{1.155652in}}%
\pgfpathlineto{\pgfqpoint{1.945309in}{1.155652in}}%
\pgfpathlineto{\pgfqpoint{1.946458in}{1.301994in}}%
\pgfpathlineto{\pgfqpoint{1.946841in}{1.239276in}}%
\pgfpathlineto{\pgfqpoint{1.947223in}{1.239276in}}%
\pgfpathlineto{\pgfqpoint{1.947223in}{1.103388in}}%
\pgfpathlineto{\pgfqpoint{1.948755in}{1.166105in}}%
\pgfpathlineto{\pgfqpoint{1.949138in}{1.166105in}}%
\pgfpathlineto{\pgfqpoint{1.949903in}{1.312447in}}%
\pgfpathlineto{\pgfqpoint{1.950669in}{1.155652in}}%
\pgfpathlineto{\pgfqpoint{1.951052in}{1.155652in}}%
\pgfpathlineto{\pgfqpoint{1.952200in}{1.317673in}}%
\pgfpathlineto{\pgfqpoint{1.951435in}{1.145199in}}%
\pgfpathlineto{\pgfqpoint{1.952583in}{1.260182in}}%
\pgfpathlineto{\pgfqpoint{1.952966in}{1.260182in}}%
\pgfpathlineto{\pgfqpoint{1.953349in}{1.087708in}}%
\pgfpathlineto{\pgfqpoint{1.954114in}{1.286314in}}%
\pgfpathlineto{\pgfqpoint{1.954497in}{1.254955in}}%
\pgfpathlineto{\pgfqpoint{1.954880in}{1.254955in}}%
\pgfpathlineto{\pgfqpoint{1.955263in}{1.166105in}}%
\pgfpathlineto{\pgfqpoint{1.956411in}{1.181785in}}%
\pgfpathlineto{\pgfqpoint{1.956794in}{1.181785in}}%
\pgfpathlineto{\pgfqpoint{1.956794in}{1.254955in}}%
\pgfpathlineto{\pgfqpoint{1.957943in}{1.124293in}}%
\pgfpathlineto{\pgfqpoint{1.958326in}{1.166105in}}%
\pgfpathlineto{\pgfqpoint{1.958708in}{1.166105in}}%
\pgfpathlineto{\pgfqpoint{1.959474in}{1.155652in}}%
\pgfpathlineto{\pgfqpoint{1.960240in}{1.312447in}}%
\pgfpathlineto{\pgfqpoint{1.960623in}{1.312447in}}%
\pgfpathlineto{\pgfqpoint{1.961005in}{1.124293in}}%
\pgfpathlineto{\pgfqpoint{1.961771in}{1.317673in}}%
\pgfpathlineto{\pgfqpoint{1.962154in}{1.160879in}}%
\pgfpathlineto{\pgfqpoint{1.962537in}{1.160879in}}%
\pgfpathlineto{\pgfqpoint{1.962919in}{1.265408in}}%
\pgfpathlineto{\pgfqpoint{1.964068in}{1.223597in}}%
\pgfpathlineto{\pgfqpoint{1.964451in}{1.223597in}}%
\pgfpathlineto{\pgfqpoint{1.965599in}{1.265408in}}%
\pgfpathlineto{\pgfqpoint{1.965982in}{1.150426in}}%
\pgfpathlineto{\pgfqpoint{1.966365in}{1.150426in}}%
\pgfpathlineto{\pgfqpoint{1.967131in}{1.338579in}}%
\pgfpathlineto{\pgfqpoint{1.967896in}{1.228823in}}%
\pgfpathlineto{\pgfqpoint{1.968279in}{1.228823in}}%
\pgfpathlineto{\pgfqpoint{1.968279in}{1.265408in}}%
\pgfpathlineto{\pgfqpoint{1.969428in}{1.098161in}}%
\pgfpathlineto{\pgfqpoint{1.969810in}{1.213144in}}%
\pgfpathlineto{\pgfqpoint{1.970193in}{1.213144in}}%
\pgfpathlineto{\pgfqpoint{1.970576in}{1.270635in}}%
\pgfpathlineto{\pgfqpoint{1.970959in}{1.134746in}}%
\pgfpathlineto{\pgfqpoint{1.971725in}{1.202691in}}%
\pgfpathlineto{\pgfqpoint{1.972107in}{1.202691in}}%
\pgfpathlineto{\pgfqpoint{1.972490in}{1.291541in}}%
\pgfpathlineto{\pgfqpoint{1.972873in}{1.129520in}}%
\pgfpathlineto{\pgfqpoint{1.973639in}{1.145199in}}%
\pgfpathlineto{\pgfqpoint{1.974022in}{1.145199in}}%
\pgfpathlineto{\pgfqpoint{1.975170in}{1.301994in}}%
\pgfpathlineto{\pgfqpoint{1.975553in}{1.139973in}}%
\pgfpathlineto{\pgfqpoint{1.975936in}{1.139973in}}%
\pgfpathlineto{\pgfqpoint{1.975936in}{1.134746in}}%
\pgfpathlineto{\pgfqpoint{1.977467in}{1.260182in}}%
\pgfpathlineto{\pgfqpoint{1.977850in}{1.260182in}}%
\pgfpathlineto{\pgfqpoint{1.978233in}{1.113840in}}%
\pgfpathlineto{\pgfqpoint{1.979381in}{1.239276in}}%
\pgfpathlineto{\pgfqpoint{1.979764in}{1.239276in}}%
\pgfpathlineto{\pgfqpoint{1.979764in}{1.098161in}}%
\pgfpathlineto{\pgfqpoint{1.981295in}{1.270635in}}%
\pgfpathlineto{\pgfqpoint{1.981678in}{1.270635in}}%
\pgfpathlineto{\pgfqpoint{1.982444in}{1.077255in}}%
\pgfpathlineto{\pgfqpoint{1.983209in}{1.192238in}}%
\pgfpathlineto{\pgfqpoint{1.983592in}{1.192238in}}%
\pgfpathlineto{\pgfqpoint{1.984741in}{1.155652in}}%
\pgfpathlineto{\pgfqpoint{1.985124in}{1.218370in}}%
\pgfpathlineto{\pgfqpoint{1.985506in}{1.218370in}}%
\pgfpathlineto{\pgfqpoint{1.985506in}{1.113840in}}%
\pgfpathlineto{\pgfqpoint{1.985889in}{1.234050in}}%
\pgfpathlineto{\pgfqpoint{1.987038in}{1.213144in}}%
\pgfpathlineto{\pgfqpoint{1.987421in}{1.213144in}}%
\pgfpathlineto{\pgfqpoint{1.988186in}{1.166105in}}%
\pgfpathlineto{\pgfqpoint{1.988569in}{1.218370in}}%
\pgfpathlineto{\pgfqpoint{1.988952in}{1.197464in}}%
\pgfpathlineto{\pgfqpoint{1.989335in}{1.197464in}}%
\pgfpathlineto{\pgfqpoint{1.990100in}{1.301994in}}%
\pgfpathlineto{\pgfqpoint{1.989718in}{1.187011in}}%
\pgfpathlineto{\pgfqpoint{1.990866in}{1.213144in}}%
\pgfpathlineto{\pgfqpoint{1.991249in}{1.213144in}}%
\pgfpathlineto{\pgfqpoint{1.992397in}{1.061576in}}%
\pgfpathlineto{\pgfqpoint{1.992780in}{1.197464in}}%
\pgfpathlineto{\pgfqpoint{1.993546in}{1.197464in}}%
\pgfpathlineto{\pgfqpoint{1.993929in}{1.218370in}}%
\pgfpathlineto{\pgfqpoint{1.994312in}{1.061576in}}%
\pgfpathlineto{\pgfqpoint{1.995077in}{1.166105in}}%
\pgfpathlineto{\pgfqpoint{1.995460in}{1.166105in}}%
\pgfpathlineto{\pgfqpoint{1.996609in}{1.286314in}}%
\pgfpathlineto{\pgfqpoint{1.995843in}{1.087708in}}%
\pgfpathlineto{\pgfqpoint{1.996991in}{1.192238in}}%
\pgfpathlineto{\pgfqpoint{1.997374in}{1.192238in}}%
\pgfpathlineto{\pgfqpoint{1.997374in}{1.145199in}}%
\pgfpathlineto{\pgfqpoint{1.998140in}{1.265408in}}%
\pgfpathlineto{\pgfqpoint{1.998906in}{1.171332in}}%
\pgfpathlineto{\pgfqpoint{1.999288in}{1.171332in}}%
\pgfpathlineto{\pgfqpoint{1.999671in}{1.108614in}}%
\pgfpathlineto{\pgfqpoint{2.000820in}{1.260182in}}%
\pgfpathlineto{\pgfqpoint{2.001202in}{1.260182in}}%
\pgfpathlineto{\pgfqpoint{2.002351in}{1.061576in}}%
\pgfpathlineto{\pgfqpoint{2.002734in}{1.124293in}}%
\pgfpathlineto{\pgfqpoint{2.003117in}{1.124293in}}%
\pgfpathlineto{\pgfqpoint{2.003117in}{1.066802in}}%
\pgfpathlineto{\pgfqpoint{2.003499in}{1.192238in}}%
\pgfpathlineto{\pgfqpoint{2.004648in}{1.103388in}}%
\pgfpathlineto{\pgfqpoint{2.005031in}{1.103388in}}%
\pgfpathlineto{\pgfqpoint{2.005414in}{1.301994in}}%
\pgfpathlineto{\pgfqpoint{2.006562in}{1.192238in}}%
\pgfpathlineto{\pgfqpoint{2.006945in}{1.192238in}}%
\pgfpathlineto{\pgfqpoint{2.007711in}{1.139973in}}%
\pgfpathlineto{\pgfqpoint{2.008476in}{1.322900in}}%
\pgfpathlineto{\pgfqpoint{2.008859in}{1.322900in}}%
\pgfpathlineto{\pgfqpoint{2.009625in}{1.139973in}}%
\pgfpathlineto{\pgfqpoint{2.010390in}{1.166105in}}%
\pgfpathlineto{\pgfqpoint{2.010773in}{1.166105in}}%
\pgfpathlineto{\pgfqpoint{2.010773in}{1.098161in}}%
\pgfpathlineto{\pgfqpoint{2.011922in}{1.187011in}}%
\pgfpathlineto{\pgfqpoint{2.012305in}{1.134746in}}%
\pgfpathlineto{\pgfqpoint{2.012687in}{1.134746in}}%
\pgfpathlineto{\pgfqpoint{2.013836in}{1.270635in}}%
\pgfpathlineto{\pgfqpoint{2.014219in}{1.129520in}}%
\pgfpathlineto{\pgfqpoint{2.014602in}{1.129520in}}%
\pgfpathlineto{\pgfqpoint{2.014984in}{1.275861in}}%
\pgfpathlineto{\pgfqpoint{2.015367in}{1.087708in}}%
\pgfpathlineto{\pgfqpoint{2.016133in}{1.187011in}}%
\pgfpathlineto{\pgfqpoint{2.016516in}{1.187011in}}%
\pgfpathlineto{\pgfqpoint{2.016899in}{1.234050in}}%
\pgfpathlineto{\pgfqpoint{2.018047in}{1.119067in}}%
\pgfpathlineto{\pgfqpoint{2.018430in}{1.119067in}}%
\pgfpathlineto{\pgfqpoint{2.019578in}{1.239276in}}%
\pgfpathlineto{\pgfqpoint{2.019961in}{1.181785in}}%
\pgfpathlineto{\pgfqpoint{2.020344in}{1.181785in}}%
\pgfpathlineto{\pgfqpoint{2.021110in}{1.213144in}}%
\pgfpathlineto{\pgfqpoint{2.020727in}{1.171332in}}%
\pgfpathlineto{\pgfqpoint{2.021875in}{1.171332in}}%
\pgfpathlineto{\pgfqpoint{2.022258in}{1.171332in}}%
\pgfpathlineto{\pgfqpoint{2.022641in}{1.103388in}}%
\pgfpathlineto{\pgfqpoint{2.023407in}{1.213144in}}%
\pgfpathlineto{\pgfqpoint{2.023789in}{1.145199in}}%
\pgfpathlineto{\pgfqpoint{2.024172in}{1.145199in}}%
\pgfpathlineto{\pgfqpoint{2.024938in}{1.260182in}}%
\pgfpathlineto{\pgfqpoint{2.025321in}{1.113840in}}%
\pgfpathlineto{\pgfqpoint{2.025704in}{1.213144in}}%
\pgfpathlineto{\pgfqpoint{2.026086in}{1.213144in}}%
\pgfpathlineto{\pgfqpoint{2.026852in}{1.087708in}}%
\pgfpathlineto{\pgfqpoint{2.027618in}{1.239276in}}%
\pgfpathlineto{\pgfqpoint{2.028001in}{1.239276in}}%
\pgfpathlineto{\pgfqpoint{2.029149in}{1.072029in}}%
\pgfpathlineto{\pgfqpoint{2.029532in}{1.108614in}}%
\pgfpathlineto{\pgfqpoint{2.029915in}{1.108614in}}%
\pgfpathlineto{\pgfqpoint{2.029915in}{1.239276in}}%
\pgfpathlineto{\pgfqpoint{2.031446in}{1.176558in}}%
\pgfpathlineto{\pgfqpoint{2.031829in}{1.176558in}}%
\pgfpathlineto{\pgfqpoint{2.032977in}{1.119067in}}%
\pgfpathlineto{\pgfqpoint{2.032595in}{1.202691in}}%
\pgfpathlineto{\pgfqpoint{2.033360in}{1.134746in}}%
\pgfpathlineto{\pgfqpoint{2.033743in}{1.134746in}}%
\pgfpathlineto{\pgfqpoint{2.034126in}{1.213144in}}%
\pgfpathlineto{\pgfqpoint{2.034892in}{1.113840in}}%
\pgfpathlineto{\pgfqpoint{2.035274in}{1.129520in}}%
\pgfpathlineto{\pgfqpoint{2.035657in}{1.129520in}}%
\pgfpathlineto{\pgfqpoint{2.037189in}{1.239276in}}%
\pgfpathlineto{\pgfqpoint{2.037571in}{1.239276in}}%
\pgfpathlineto{\pgfqpoint{2.038720in}{1.092935in}}%
\pgfpathlineto{\pgfqpoint{2.039103in}{1.239276in}}%
\pgfpathlineto{\pgfqpoint{2.039486in}{1.239276in}}%
\pgfpathlineto{\pgfqpoint{2.040251in}{1.082482in}}%
\pgfpathlineto{\pgfqpoint{2.041017in}{1.234050in}}%
\pgfpathlineto{\pgfqpoint{2.041400in}{1.234050in}}%
\pgfpathlineto{\pgfqpoint{2.041400in}{1.077255in}}%
\pgfpathlineto{\pgfqpoint{2.042931in}{1.082482in}}%
\pgfpathlineto{\pgfqpoint{2.043314in}{1.082482in}}%
\pgfpathlineto{\pgfqpoint{2.043697in}{1.249729in}}%
\pgfpathlineto{\pgfqpoint{2.044079in}{1.061576in}}%
\pgfpathlineto{\pgfqpoint{2.044845in}{1.181785in}}%
\pgfpathlineto{\pgfqpoint{2.045228in}{1.181785in}}%
\pgfpathlineto{\pgfqpoint{2.045228in}{1.087708in}}%
\pgfpathlineto{\pgfqpoint{2.046759in}{1.166105in}}%
\pgfpathlineto{\pgfqpoint{2.047142in}{1.166105in}}%
\pgfpathlineto{\pgfqpoint{2.047908in}{1.119067in}}%
\pgfpathlineto{\pgfqpoint{2.047525in}{1.239276in}}%
\pgfpathlineto{\pgfqpoint{2.048673in}{1.155652in}}%
\pgfpathlineto{\pgfqpoint{2.049056in}{1.155652in}}%
\pgfpathlineto{\pgfqpoint{2.049439in}{1.228823in}}%
\pgfpathlineto{\pgfqpoint{2.049439in}{1.030217in}}%
\pgfpathlineto{\pgfqpoint{2.050588in}{1.072029in}}%
\pgfpathlineto{\pgfqpoint{2.050970in}{1.072029in}}%
\pgfpathlineto{\pgfqpoint{2.052502in}{1.176558in}}%
\pgfpathlineto{\pgfqpoint{2.052885in}{1.176558in}}%
\pgfpathlineto{\pgfqpoint{2.052885in}{1.134746in}}%
\pgfpathlineto{\pgfqpoint{2.053267in}{1.244503in}}%
\pgfpathlineto{\pgfqpoint{2.054416in}{1.166105in}}%
\pgfpathlineto{\pgfqpoint{2.054799in}{1.166105in}}%
\pgfpathlineto{\pgfqpoint{2.055182in}{1.103388in}}%
\pgfpathlineto{\pgfqpoint{2.056330in}{1.260182in}}%
\pgfpathlineto{\pgfqpoint{2.056713in}{1.260182in}}%
\pgfpathlineto{\pgfqpoint{2.056713in}{1.103388in}}%
\pgfpathlineto{\pgfqpoint{2.058244in}{1.134746in}}%
\pgfpathlineto{\pgfqpoint{2.058627in}{1.134746in}}%
\pgfpathlineto{\pgfqpoint{2.059393in}{1.087708in}}%
\pgfpathlineto{\pgfqpoint{2.060158in}{1.176558in}}%
\pgfpathlineto{\pgfqpoint{2.060541in}{1.176558in}}%
\pgfpathlineto{\pgfqpoint{2.060541in}{1.056349in}}%
\pgfpathlineto{\pgfqpoint{2.060924in}{1.197464in}}%
\pgfpathlineto{\pgfqpoint{2.062072in}{1.066802in}}%
\pgfpathlineto{\pgfqpoint{2.062455in}{1.066802in}}%
\pgfpathlineto{\pgfqpoint{2.063604in}{1.218370in}}%
\pgfpathlineto{\pgfqpoint{2.063987in}{1.139973in}}%
\pgfpathlineto{\pgfqpoint{2.064369in}{1.139973in}}%
\pgfpathlineto{\pgfqpoint{2.065518in}{1.181785in}}%
\pgfpathlineto{\pgfqpoint{2.065901in}{1.124293in}}%
\pgfpathlineto{\pgfqpoint{2.066284in}{1.124293in}}%
\pgfpathlineto{\pgfqpoint{2.067049in}{1.228823in}}%
\pgfpathlineto{\pgfqpoint{2.067432in}{1.051123in}}%
\pgfpathlineto{\pgfqpoint{2.067815in}{1.119067in}}%
\pgfpathlineto{\pgfqpoint{2.068198in}{1.119067in}}%
\pgfpathlineto{\pgfqpoint{2.069346in}{1.207917in}}%
\pgfpathlineto{\pgfqpoint{2.069729in}{1.087708in}}%
\pgfpathlineto{\pgfqpoint{2.070112in}{1.087708in}}%
\pgfpathlineto{\pgfqpoint{2.070112in}{1.082482in}}%
\pgfpathlineto{\pgfqpoint{2.071643in}{1.150426in}}%
\pgfpathlineto{\pgfqpoint{2.072026in}{1.150426in}}%
\pgfpathlineto{\pgfqpoint{2.072026in}{1.098161in}}%
\pgfpathlineto{\pgfqpoint{2.073557in}{1.124293in}}%
\pgfpathlineto{\pgfqpoint{2.073940in}{1.124293in}}%
\pgfpathlineto{\pgfqpoint{2.074323in}{1.192238in}}%
\pgfpathlineto{\pgfqpoint{2.075472in}{1.045896in}}%
\pgfpathlineto{\pgfqpoint{2.075854in}{1.045896in}}%
\pgfpathlineto{\pgfqpoint{2.076620in}{1.187011in}}%
\pgfpathlineto{\pgfqpoint{2.077386in}{1.150426in}}%
\pgfpathlineto{\pgfqpoint{2.077769in}{1.150426in}}%
\pgfpathlineto{\pgfqpoint{2.077769in}{1.056349in}}%
\pgfpathlineto{\pgfqpoint{2.078917in}{1.218370in}}%
\pgfpathlineto{\pgfqpoint{2.079300in}{1.108614in}}%
\pgfpathlineto{\pgfqpoint{2.079683in}{1.108614in}}%
\pgfpathlineto{\pgfqpoint{2.080831in}{1.176558in}}%
\pgfpathlineto{\pgfqpoint{2.080065in}{1.092935in}}%
\pgfpathlineto{\pgfqpoint{2.081214in}{1.092935in}}%
\pgfpathlineto{\pgfqpoint{2.081597in}{1.092935in}}%
\pgfpathlineto{\pgfqpoint{2.082745in}{1.072029in}}%
\pgfpathlineto{\pgfqpoint{2.083128in}{1.202691in}}%
\pgfpathlineto{\pgfqpoint{2.083511in}{1.202691in}}%
\pgfpathlineto{\pgfqpoint{2.083511in}{1.035443in}}%
\pgfpathlineto{\pgfqpoint{2.085042in}{1.145199in}}%
\pgfpathlineto{\pgfqpoint{2.085425in}{1.145199in}}%
\pgfpathlineto{\pgfqpoint{2.086574in}{1.087708in}}%
\pgfpathlineto{\pgfqpoint{2.085808in}{1.171332in}}%
\pgfpathlineto{\pgfqpoint{2.086956in}{1.098161in}}%
\pgfpathlineto{\pgfqpoint{2.087339in}{1.098161in}}%
\pgfpathlineto{\pgfqpoint{2.087339in}{1.040670in}}%
\pgfpathlineto{\pgfqpoint{2.087722in}{1.139973in}}%
\pgfpathlineto{\pgfqpoint{2.088871in}{1.113840in}}%
\pgfpathlineto{\pgfqpoint{2.089253in}{1.113840in}}%
\pgfpathlineto{\pgfqpoint{2.090402in}{1.077255in}}%
\pgfpathlineto{\pgfqpoint{2.090019in}{1.181785in}}%
\pgfpathlineto{\pgfqpoint{2.090785in}{1.108614in}}%
\pgfpathlineto{\pgfqpoint{2.091168in}{1.108614in}}%
\pgfpathlineto{\pgfqpoint{2.091933in}{1.051123in}}%
\pgfpathlineto{\pgfqpoint{2.092699in}{1.166105in}}%
\pgfpathlineto{\pgfqpoint{2.093082in}{1.166105in}}%
\pgfpathlineto{\pgfqpoint{2.093847in}{1.072029in}}%
\pgfpathlineto{\pgfqpoint{2.093465in}{1.223597in}}%
\pgfpathlineto{\pgfqpoint{2.094613in}{1.124293in}}%
\pgfpathlineto{\pgfqpoint{2.094996in}{1.124293in}}%
\pgfpathlineto{\pgfqpoint{2.095379in}{1.150426in}}%
\pgfpathlineto{\pgfqpoint{2.096527in}{0.993631in}}%
\pgfpathlineto{\pgfqpoint{2.096910in}{0.993631in}}%
\pgfpathlineto{\pgfqpoint{2.098058in}{1.187011in}}%
\pgfpathlineto{\pgfqpoint{2.098441in}{1.024990in}}%
\pgfpathlineto{\pgfqpoint{2.098824in}{1.024990in}}%
\pgfpathlineto{\pgfqpoint{2.099590in}{1.223597in}}%
\pgfpathlineto{\pgfqpoint{2.100355in}{1.119067in}}%
\pgfpathlineto{\pgfqpoint{2.100738in}{1.119067in}}%
\pgfpathlineto{\pgfqpoint{2.100738in}{1.160879in}}%
\pgfpathlineto{\pgfqpoint{2.101887in}{1.082482in}}%
\pgfpathlineto{\pgfqpoint{2.102270in}{1.155652in}}%
\pgfpathlineto{\pgfqpoint{2.102652in}{1.155652in}}%
\pgfpathlineto{\pgfqpoint{2.104184in}{1.019764in}}%
\pgfpathlineto{\pgfqpoint{2.104567in}{1.019764in}}%
\pgfpathlineto{\pgfqpoint{2.105715in}{1.197464in}}%
\pgfpathlineto{\pgfqpoint{2.106098in}{1.139973in}}%
\pgfpathlineto{\pgfqpoint{2.106481in}{1.139973in}}%
\pgfpathlineto{\pgfqpoint{2.107246in}{1.192238in}}%
\pgfpathlineto{\pgfqpoint{2.107629in}{1.045896in}}%
\pgfpathlineto{\pgfqpoint{2.108012in}{1.187011in}}%
\pgfpathlineto{\pgfqpoint{2.108395in}{1.187011in}}%
\pgfpathlineto{\pgfqpoint{2.109161in}{1.056349in}}%
\pgfpathlineto{\pgfqpoint{2.109926in}{1.166105in}}%
\pgfpathlineto{\pgfqpoint{2.110309in}{1.166105in}}%
\pgfpathlineto{\pgfqpoint{2.110692in}{1.045896in}}%
\pgfpathlineto{\pgfqpoint{2.111840in}{1.181785in}}%
\pgfpathlineto{\pgfqpoint{2.112223in}{1.181785in}}%
\pgfpathlineto{\pgfqpoint{2.112606in}{1.045896in}}%
\pgfpathlineto{\pgfqpoint{2.113755in}{1.061576in}}%
\pgfpathlineto{\pgfqpoint{2.114137in}{1.061576in}}%
\pgfpathlineto{\pgfqpoint{2.114520in}{1.187011in}}%
\pgfpathlineto{\pgfqpoint{2.115286in}{1.024990in}}%
\pgfpathlineto{\pgfqpoint{2.115669in}{1.155652in}}%
\pgfpathlineto{\pgfqpoint{2.116052in}{1.155652in}}%
\pgfpathlineto{\pgfqpoint{2.116052in}{1.019764in}}%
\pgfpathlineto{\pgfqpoint{2.117583in}{1.077255in}}%
\pgfpathlineto{\pgfqpoint{2.117966in}{1.077255in}}%
\pgfpathlineto{\pgfqpoint{2.118348in}{1.160879in}}%
\pgfpathlineto{\pgfqpoint{2.119497in}{1.030217in}}%
\pgfpathlineto{\pgfqpoint{2.119880in}{1.030217in}}%
\pgfpathlineto{\pgfqpoint{2.120263in}{1.155652in}}%
\pgfpathlineto{\pgfqpoint{2.121411in}{1.004084in}}%
\pgfpathlineto{\pgfqpoint{2.121794in}{1.004084in}}%
\pgfpathlineto{\pgfqpoint{2.122942in}{1.192238in}}%
\pgfpathlineto{\pgfqpoint{2.123325in}{1.119067in}}%
\pgfpathlineto{\pgfqpoint{2.123708in}{1.119067in}}%
\pgfpathlineto{\pgfqpoint{2.123708in}{1.171332in}}%
\pgfpathlineto{\pgfqpoint{2.125239in}{1.061576in}}%
\pgfpathlineto{\pgfqpoint{2.125622in}{1.061576in}}%
\pgfpathlineto{\pgfqpoint{2.126388in}{1.139973in}}%
\pgfpathlineto{\pgfqpoint{2.127154in}{1.009311in}}%
\pgfpathlineto{\pgfqpoint{2.127536in}{1.009311in}}%
\pgfpathlineto{\pgfqpoint{2.127536in}{1.134746in}}%
\pgfpathlineto{\pgfqpoint{2.129068in}{1.092935in}}%
\pgfpathlineto{\pgfqpoint{2.129451in}{1.092935in}}%
\pgfpathlineto{\pgfqpoint{2.130216in}{1.207917in}}%
\pgfpathlineto{\pgfqpoint{2.130599in}{1.035443in}}%
\pgfpathlineto{\pgfqpoint{2.130982in}{1.092935in}}%
\pgfpathlineto{\pgfqpoint{2.131365in}{1.092935in}}%
\pgfpathlineto{\pgfqpoint{2.131748in}{1.056349in}}%
\pgfpathlineto{\pgfqpoint{2.132896in}{1.124293in}}%
\pgfpathlineto{\pgfqpoint{2.133279in}{1.124293in}}%
\pgfpathlineto{\pgfqpoint{2.133279in}{1.145199in}}%
\pgfpathlineto{\pgfqpoint{2.134045in}{0.983178in}}%
\pgfpathlineto{\pgfqpoint{2.134810in}{1.124293in}}%
\pgfpathlineto{\pgfqpoint{2.135193in}{1.124293in}}%
\pgfpathlineto{\pgfqpoint{2.135959in}{1.129520in}}%
\pgfpathlineto{\pgfqpoint{2.136724in}{1.024990in}}%
\pgfpathlineto{\pgfqpoint{2.137107in}{1.024990in}}%
\pgfpathlineto{\pgfqpoint{2.137490in}{1.113840in}}%
\pgfpathlineto{\pgfqpoint{2.138256in}{0.998858in}}%
\pgfpathlineto{\pgfqpoint{2.138638in}{1.040670in}}%
\pgfpathlineto{\pgfqpoint{2.139021in}{1.040670in}}%
\pgfpathlineto{\pgfqpoint{2.139021in}{1.171332in}}%
\pgfpathlineto{\pgfqpoint{2.140170in}{1.030217in}}%
\pgfpathlineto{\pgfqpoint{2.140553in}{1.051123in}}%
\pgfpathlineto{\pgfqpoint{2.140935in}{1.051123in}}%
\pgfpathlineto{\pgfqpoint{2.141701in}{1.119067in}}%
\pgfpathlineto{\pgfqpoint{2.142084in}{0.993631in}}%
\pgfpathlineto{\pgfqpoint{2.142467in}{1.072029in}}%
\pgfpathlineto{\pgfqpoint{2.142850in}{1.072029in}}%
\pgfpathlineto{\pgfqpoint{2.143615in}{1.045896in}}%
\pgfpathlineto{\pgfqpoint{2.143232in}{1.202691in}}%
\pgfpathlineto{\pgfqpoint{2.144381in}{1.072029in}}%
\pgfpathlineto{\pgfqpoint{2.144764in}{1.072029in}}%
\pgfpathlineto{\pgfqpoint{2.145147in}{0.962273in}}%
\pgfpathlineto{\pgfqpoint{2.146295in}{1.150426in}}%
\pgfpathlineto{\pgfqpoint{2.146678in}{1.150426in}}%
\pgfpathlineto{\pgfqpoint{2.147061in}{1.014537in}}%
\pgfpathlineto{\pgfqpoint{2.148209in}{1.051123in}}%
\pgfpathlineto{\pgfqpoint{2.148592in}{1.051123in}}%
\pgfpathlineto{\pgfqpoint{2.148592in}{0.972726in}}%
\pgfpathlineto{\pgfqpoint{2.149358in}{1.119067in}}%
\pgfpathlineto{\pgfqpoint{2.150123in}{1.092935in}}%
\pgfpathlineto{\pgfqpoint{2.150506in}{1.092935in}}%
\pgfpathlineto{\pgfqpoint{2.150889in}{1.040670in}}%
\pgfpathlineto{\pgfqpoint{2.151655in}{1.108614in}}%
\pgfpathlineto{\pgfqpoint{2.152038in}{1.082482in}}%
\pgfpathlineto{\pgfqpoint{2.152420in}{1.082482in}}%
\pgfpathlineto{\pgfqpoint{2.153569in}{1.024990in}}%
\pgfpathlineto{\pgfqpoint{2.153186in}{1.160879in}}%
\pgfpathlineto{\pgfqpoint{2.153952in}{1.045896in}}%
\pgfpathlineto{\pgfqpoint{2.154335in}{1.045896in}}%
\pgfpathlineto{\pgfqpoint{2.155866in}{1.139973in}}%
\pgfpathlineto{\pgfqpoint{2.156249in}{1.139973in}}%
\pgfpathlineto{\pgfqpoint{2.156631in}{1.019764in}}%
\pgfpathlineto{\pgfqpoint{2.157780in}{1.019764in}}%
\pgfpathlineto{\pgfqpoint{2.158163in}{1.019764in}}%
\pgfpathlineto{\pgfqpoint{2.158546in}{1.145199in}}%
\pgfpathlineto{\pgfqpoint{2.159311in}{0.972726in}}%
\pgfpathlineto{\pgfqpoint{2.159694in}{1.077255in}}%
\pgfpathlineto{\pgfqpoint{2.160077in}{1.077255in}}%
\pgfpathlineto{\pgfqpoint{2.160460in}{1.024990in}}%
\pgfpathlineto{\pgfqpoint{2.160843in}{1.103388in}}%
\pgfpathlineto{\pgfqpoint{2.161608in}{1.082482in}}%
\pgfpathlineto{\pgfqpoint{2.161991in}{1.082482in}}%
\pgfpathlineto{\pgfqpoint{2.162757in}{0.936140in}}%
\pgfpathlineto{\pgfqpoint{2.163522in}{1.061576in}}%
\pgfpathlineto{\pgfqpoint{2.163905in}{1.061576in}}%
\pgfpathlineto{\pgfqpoint{2.163905in}{0.967499in}}%
\pgfpathlineto{\pgfqpoint{2.164288in}{1.134746in}}%
\pgfpathlineto{\pgfqpoint{2.165437in}{1.077255in}}%
\pgfpathlineto{\pgfqpoint{2.165819in}{1.077255in}}%
\pgfpathlineto{\pgfqpoint{2.165819in}{0.972726in}}%
\pgfpathlineto{\pgfqpoint{2.167351in}{1.124293in}}%
\pgfpathlineto{\pgfqpoint{2.167734in}{1.124293in}}%
\pgfpathlineto{\pgfqpoint{2.168116in}{1.145199in}}%
\pgfpathlineto{\pgfqpoint{2.169265in}{0.998858in}}%
\pgfpathlineto{\pgfqpoint{2.169648in}{0.998858in}}%
\pgfpathlineto{\pgfqpoint{2.169648in}{1.056349in}}%
\pgfpathlineto{\pgfqpoint{2.170031in}{0.983178in}}%
\pgfpathlineto{\pgfqpoint{2.171179in}{1.030217in}}%
\pgfpathlineto{\pgfqpoint{2.171562in}{1.030217in}}%
\pgfpathlineto{\pgfqpoint{2.171562in}{0.998858in}}%
\pgfpathlineto{\pgfqpoint{2.171945in}{1.134746in}}%
\pgfpathlineto{\pgfqpoint{2.173093in}{1.119067in}}%
\pgfpathlineto{\pgfqpoint{2.173476in}{1.119067in}}%
\pgfpathlineto{\pgfqpoint{2.173476in}{1.009311in}}%
\pgfpathlineto{\pgfqpoint{2.175007in}{1.019764in}}%
\pgfpathlineto{\pgfqpoint{2.175390in}{1.019764in}}%
\pgfpathlineto{\pgfqpoint{2.175390in}{0.920461in}}%
\pgfpathlineto{\pgfqpoint{2.176539in}{1.098161in}}%
\pgfpathlineto{\pgfqpoint{2.176921in}{0.930914in}}%
\pgfpathlineto{\pgfqpoint{2.177304in}{0.930914in}}%
\pgfpathlineto{\pgfqpoint{2.177687in}{1.087708in}}%
\pgfpathlineto{\pgfqpoint{2.178836in}{0.998858in}}%
\pgfpathlineto{\pgfqpoint{2.179218in}{0.998858in}}%
\pgfpathlineto{\pgfqpoint{2.179218in}{0.977952in}}%
\pgfpathlineto{\pgfqpoint{2.179601in}{1.124293in}}%
\pgfpathlineto{\pgfqpoint{2.180750in}{1.061576in}}%
\pgfpathlineto{\pgfqpoint{2.181133in}{1.061576in}}%
\pgfpathlineto{\pgfqpoint{2.181515in}{0.936140in}}%
\pgfpathlineto{\pgfqpoint{2.181898in}{1.098161in}}%
\pgfpathlineto{\pgfqpoint{2.182664in}{1.024990in}}%
\pgfpathlineto{\pgfqpoint{2.183047in}{1.024990in}}%
\pgfpathlineto{\pgfqpoint{2.183047in}{1.009311in}}%
\pgfpathlineto{\pgfqpoint{2.184578in}{1.066802in}}%
\pgfpathlineto{\pgfqpoint{2.185344in}{1.066802in}}%
\pgfpathlineto{\pgfqpoint{2.186109in}{1.004084in}}%
\pgfpathlineto{\pgfqpoint{2.186492in}{1.098161in}}%
\pgfpathlineto{\pgfqpoint{2.186875in}{1.024990in}}%
\pgfpathlineto{\pgfqpoint{2.187258in}{1.024990in}}%
\pgfpathlineto{\pgfqpoint{2.187641in}{1.066802in}}%
\pgfpathlineto{\pgfqpoint{2.188789in}{0.967499in}}%
\pgfpathlineto{\pgfqpoint{2.189172in}{0.967499in}}%
\pgfpathlineto{\pgfqpoint{2.189172in}{0.951820in}}%
\pgfpathlineto{\pgfqpoint{2.190321in}{1.061576in}}%
\pgfpathlineto{\pgfqpoint{2.190703in}{1.009311in}}%
\pgfpathlineto{\pgfqpoint{2.191086in}{1.009311in}}%
\pgfpathlineto{\pgfqpoint{2.191086in}{0.915234in}}%
\pgfpathlineto{\pgfqpoint{2.191469in}{1.066802in}}%
\pgfpathlineto{\pgfqpoint{2.192618in}{0.998858in}}%
\pgfpathlineto{\pgfqpoint{2.193000in}{0.998858in}}%
\pgfpathlineto{\pgfqpoint{2.193000in}{1.051123in}}%
\pgfpathlineto{\pgfqpoint{2.194149in}{0.988405in}}%
\pgfpathlineto{\pgfqpoint{2.194532in}{0.993631in}}%
\pgfpathlineto{\pgfqpoint{2.195680in}{0.993631in}}%
\pgfpathlineto{\pgfqpoint{2.196446in}{1.139973in}}%
\pgfpathlineto{\pgfqpoint{2.197211in}{1.066802in}}%
\pgfpathlineto{\pgfqpoint{2.197594in}{1.066802in}}%
\pgfpathlineto{\pgfqpoint{2.198743in}{1.098161in}}%
\pgfpathlineto{\pgfqpoint{2.197977in}{1.019764in}}%
\pgfpathlineto{\pgfqpoint{2.199126in}{1.077255in}}%
\pgfpathlineto{\pgfqpoint{2.199508in}{1.077255in}}%
\pgfpathlineto{\pgfqpoint{2.200274in}{0.910008in}}%
\pgfpathlineto{\pgfqpoint{2.200657in}{1.119067in}}%
\pgfpathlineto{\pgfqpoint{2.201040in}{1.061576in}}%
\pgfpathlineto{\pgfqpoint{2.201423in}{1.061576in}}%
\pgfpathlineto{\pgfqpoint{2.202188in}{0.983178in}}%
\pgfpathlineto{\pgfqpoint{2.202954in}{1.024990in}}%
\pgfpathlineto{\pgfqpoint{2.203337in}{1.024990in}}%
\pgfpathlineto{\pgfqpoint{2.204485in}{0.951820in}}%
\pgfpathlineto{\pgfqpoint{2.204868in}{0.977952in}}%
\pgfpathlineto{\pgfqpoint{2.205251in}{0.977952in}}%
\pgfpathlineto{\pgfqpoint{2.206399in}{1.113840in}}%
\pgfpathlineto{\pgfqpoint{2.206782in}{1.045896in}}%
\pgfpathlineto{\pgfqpoint{2.207548in}{1.045896in}}%
\pgfpathlineto{\pgfqpoint{2.207548in}{0.967499in}}%
\pgfpathlineto{\pgfqpoint{2.209079in}{1.145199in}}%
\pgfpathlineto{\pgfqpoint{2.209462in}{1.145199in}}%
\pgfpathlineto{\pgfqpoint{2.209462in}{0.962273in}}%
\pgfpathlineto{\pgfqpoint{2.210993in}{0.998858in}}%
\pgfpathlineto{\pgfqpoint{2.211376in}{0.998858in}}%
\pgfpathlineto{\pgfqpoint{2.211376in}{1.056349in}}%
\pgfpathlineto{\pgfqpoint{2.212142in}{0.957046in}}%
\pgfpathlineto{\pgfqpoint{2.212908in}{1.004084in}}%
\pgfpathlineto{\pgfqpoint{2.213290in}{1.004084in}}%
\pgfpathlineto{\pgfqpoint{2.214056in}{0.977952in}}%
\pgfpathlineto{\pgfqpoint{2.213673in}{1.051123in}}%
\pgfpathlineto{\pgfqpoint{2.214822in}{1.014537in}}%
\pgfpathlineto{\pgfqpoint{2.215204in}{1.014537in}}%
\pgfpathlineto{\pgfqpoint{2.215970in}{1.092935in}}%
\pgfpathlineto{\pgfqpoint{2.215587in}{1.004084in}}%
\pgfpathlineto{\pgfqpoint{2.216736in}{1.040670in}}%
\pgfpathlineto{\pgfqpoint{2.217119in}{1.040670in}}%
\pgfpathlineto{\pgfqpoint{2.217884in}{0.988405in}}%
\pgfpathlineto{\pgfqpoint{2.218267in}{1.124293in}}%
\pgfpathlineto{\pgfqpoint{2.218650in}{0.993631in}}%
\pgfpathlineto{\pgfqpoint{2.219033in}{0.993631in}}%
\pgfpathlineto{\pgfqpoint{2.219416in}{1.009311in}}%
\pgfpathlineto{\pgfqpoint{2.220564in}{0.941367in}}%
\pgfpathlineto{\pgfqpoint{2.220947in}{0.941367in}}%
\pgfpathlineto{\pgfqpoint{2.221330in}{0.930914in}}%
\pgfpathlineto{\pgfqpoint{2.222478in}{1.051123in}}%
\pgfpathlineto{\pgfqpoint{2.222861in}{1.051123in}}%
\pgfpathlineto{\pgfqpoint{2.224392in}{0.967499in}}%
\pgfpathlineto{\pgfqpoint{2.224775in}{0.967499in}}%
\pgfpathlineto{\pgfqpoint{2.225541in}{1.045896in}}%
\pgfpathlineto{\pgfqpoint{2.226307in}{0.972726in}}%
\pgfpathlineto{\pgfqpoint{2.226689in}{0.972726in}}%
\pgfpathlineto{\pgfqpoint{2.226689in}{1.051123in}}%
\pgfpathlineto{\pgfqpoint{2.227838in}{0.925687in}}%
\pgfpathlineto{\pgfqpoint{2.228221in}{0.941367in}}%
\pgfpathlineto{\pgfqpoint{2.228604in}{0.941367in}}%
\pgfpathlineto{\pgfqpoint{2.228986in}{1.077255in}}%
\pgfpathlineto{\pgfqpoint{2.229752in}{0.920461in}}%
\pgfpathlineto{\pgfqpoint{2.230135in}{0.951820in}}%
\pgfpathlineto{\pgfqpoint{2.230518in}{0.951820in}}%
\pgfpathlineto{\pgfqpoint{2.230901in}{1.077255in}}%
\pgfpathlineto{\pgfqpoint{2.232049in}{1.009311in}}%
\pgfpathlineto{\pgfqpoint{2.232432in}{1.009311in}}%
\pgfpathlineto{\pgfqpoint{2.232432in}{1.066802in}}%
\pgfpathlineto{\pgfqpoint{2.233580in}{0.977952in}}%
\pgfpathlineto{\pgfqpoint{2.233963in}{0.983178in}}%
\pgfpathlineto{\pgfqpoint{2.234346in}{0.983178in}}%
\pgfpathlineto{\pgfqpoint{2.234729in}{0.993631in}}%
\pgfpathlineto{\pgfqpoint{2.235877in}{0.936140in}}%
\pgfpathlineto{\pgfqpoint{2.236260in}{0.936140in}}%
\pgfpathlineto{\pgfqpoint{2.236260in}{1.035443in}}%
\pgfpathlineto{\pgfqpoint{2.237791in}{0.910008in}}%
\pgfpathlineto{\pgfqpoint{2.238174in}{0.910008in}}%
\pgfpathlineto{\pgfqpoint{2.239323in}{1.019764in}}%
\pgfpathlineto{\pgfqpoint{2.239706in}{0.983178in}}%
\pgfpathlineto{\pgfqpoint{2.240088in}{0.983178in}}%
\pgfpathlineto{\pgfqpoint{2.240471in}{0.915234in}}%
\pgfpathlineto{\pgfqpoint{2.240854in}{0.988405in}}%
\pgfpathlineto{\pgfqpoint{2.241620in}{0.983178in}}%
\pgfpathlineto{\pgfqpoint{2.242003in}{0.983178in}}%
\pgfpathlineto{\pgfqpoint{2.242003in}{0.972726in}}%
\pgfpathlineto{\pgfqpoint{2.243151in}{1.045896in}}%
\pgfpathlineto{\pgfqpoint{2.243534in}{0.988405in}}%
\pgfpathlineto{\pgfqpoint{2.243917in}{0.988405in}}%
\pgfpathlineto{\pgfqpoint{2.244300in}{0.915234in}}%
\pgfpathlineto{\pgfqpoint{2.245448in}{0.941367in}}%
\pgfpathlineto{\pgfqpoint{2.245831in}{0.941367in}}%
\pgfpathlineto{\pgfqpoint{2.245831in}{1.061576in}}%
\pgfpathlineto{\pgfqpoint{2.247362in}{0.910008in}}%
\pgfpathlineto{\pgfqpoint{2.247745in}{0.910008in}}%
\pgfpathlineto{\pgfqpoint{2.249276in}{1.024990in}}%
\pgfpathlineto{\pgfqpoint{2.249659in}{1.024990in}}%
\pgfpathlineto{\pgfqpoint{2.250042in}{1.040670in}}%
\pgfpathlineto{\pgfqpoint{2.250808in}{0.951820in}}%
\pgfpathlineto{\pgfqpoint{2.251191in}{1.014537in}}%
\pgfpathlineto{\pgfqpoint{2.251573in}{1.014537in}}%
\pgfpathlineto{\pgfqpoint{2.253105in}{0.951820in}}%
\pgfpathlineto{\pgfqpoint{2.253487in}{0.951820in}}%
\pgfpathlineto{\pgfqpoint{2.253870in}{1.004084in}}%
\pgfpathlineto{\pgfqpoint{2.254636in}{0.936140in}}%
\pgfpathlineto{\pgfqpoint{2.255019in}{0.977952in}}%
\pgfpathlineto{\pgfqpoint{2.255784in}{0.977952in}}%
\pgfpathlineto{\pgfqpoint{2.255784in}{0.951820in}}%
\pgfpathlineto{\pgfqpoint{2.256167in}{1.056349in}}%
\pgfpathlineto{\pgfqpoint{2.257316in}{0.957046in}}%
\pgfpathlineto{\pgfqpoint{2.257699in}{0.957046in}}%
\pgfpathlineto{\pgfqpoint{2.258081in}{1.056349in}}%
\pgfpathlineto{\pgfqpoint{2.258847in}{0.904781in}}%
\pgfpathlineto{\pgfqpoint{2.259230in}{0.910008in}}%
\pgfpathlineto{\pgfqpoint{2.259613in}{0.910008in}}%
\pgfpathlineto{\pgfqpoint{2.261144in}{1.024990in}}%
\pgfpathlineto{\pgfqpoint{2.261527in}{1.024990in}}%
\pgfpathlineto{\pgfqpoint{2.262293in}{0.915234in}}%
\pgfpathlineto{\pgfqpoint{2.261910in}{1.045896in}}%
\pgfpathlineto{\pgfqpoint{2.263058in}{0.993631in}}%
\pgfpathlineto{\pgfqpoint{2.263441in}{0.993631in}}%
\pgfpathlineto{\pgfqpoint{2.264972in}{0.899555in}}%
\pgfpathlineto{\pgfqpoint{2.265355in}{0.899555in}}%
\pgfpathlineto{\pgfqpoint{2.266504in}{1.051123in}}%
\pgfpathlineto{\pgfqpoint{2.265738in}{0.889102in}}%
\pgfpathlineto{\pgfqpoint{2.266887in}{0.899555in}}%
\pgfpathlineto{\pgfqpoint{2.267269in}{0.899555in}}%
\pgfpathlineto{\pgfqpoint{2.267652in}{0.993631in}}%
\pgfpathlineto{\pgfqpoint{2.268801in}{0.988405in}}%
\pgfpathlineto{\pgfqpoint{2.269184in}{0.988405in}}%
\pgfpathlineto{\pgfqpoint{2.270332in}{1.009311in}}%
\pgfpathlineto{\pgfqpoint{2.270715in}{0.925687in}}%
\pgfpathlineto{\pgfqpoint{2.271098in}{0.925687in}}%
\pgfpathlineto{\pgfqpoint{2.271098in}{0.993631in}}%
\pgfpathlineto{\pgfqpoint{2.271481in}{0.899555in}}%
\pgfpathlineto{\pgfqpoint{2.272629in}{0.972726in}}%
\pgfpathlineto{\pgfqpoint{2.273012in}{0.972726in}}%
\pgfpathlineto{\pgfqpoint{2.273395in}{0.925687in}}%
\pgfpathlineto{\pgfqpoint{2.273777in}{1.009311in}}%
\pgfpathlineto{\pgfqpoint{2.274543in}{0.983178in}}%
\pgfpathlineto{\pgfqpoint{2.274926in}{0.983178in}}%
\pgfpathlineto{\pgfqpoint{2.276074in}{0.847290in}}%
\pgfpathlineto{\pgfqpoint{2.276457in}{0.930914in}}%
\pgfpathlineto{\pgfqpoint{2.276840in}{0.930914in}}%
\pgfpathlineto{\pgfqpoint{2.276840in}{0.883875in}}%
\pgfpathlineto{\pgfqpoint{2.278371in}{0.962273in}}%
\pgfpathlineto{\pgfqpoint{2.278754in}{0.962273in}}%
\pgfpathlineto{\pgfqpoint{2.279903in}{0.936140in}}%
\pgfpathlineto{\pgfqpoint{2.280286in}{0.983178in}}%
\pgfpathlineto{\pgfqpoint{2.280668in}{0.983178in}}%
\pgfpathlineto{\pgfqpoint{2.281434in}{0.910008in}}%
\pgfpathlineto{\pgfqpoint{2.281051in}{0.993631in}}%
\pgfpathlineto{\pgfqpoint{2.282200in}{0.915234in}}%
\pgfpathlineto{\pgfqpoint{2.282583in}{0.915234in}}%
\pgfpathlineto{\pgfqpoint{2.282583in}{0.910008in}}%
\pgfpathlineto{\pgfqpoint{2.284114in}{0.998858in}}%
\pgfpathlineto{\pgfqpoint{2.284497in}{0.998858in}}%
\pgfpathlineto{\pgfqpoint{2.284880in}{0.899555in}}%
\pgfpathlineto{\pgfqpoint{2.286028in}{0.941367in}}%
\pgfpathlineto{\pgfqpoint{2.286411in}{0.941367in}}%
\pgfpathlineto{\pgfqpoint{2.286411in}{1.009311in}}%
\pgfpathlineto{\pgfqpoint{2.286794in}{0.936140in}}%
\pgfpathlineto{\pgfqpoint{2.287942in}{0.941367in}}%
\pgfpathlineto{\pgfqpoint{2.288325in}{0.941367in}}%
\pgfpathlineto{\pgfqpoint{2.288325in}{0.899555in}}%
\pgfpathlineto{\pgfqpoint{2.289474in}{1.009311in}}%
\pgfpathlineto{\pgfqpoint{2.289856in}{1.009311in}}%
\pgfpathlineto{\pgfqpoint{2.290239in}{1.009311in}}%
\pgfpathlineto{\pgfqpoint{2.291388in}{0.904781in}}%
\pgfpathlineto{\pgfqpoint{2.291770in}{0.941367in}}%
\pgfpathlineto{\pgfqpoint{2.292153in}{0.941367in}}%
\pgfpathlineto{\pgfqpoint{2.292536in}{0.972726in}}%
\pgfpathlineto{\pgfqpoint{2.293685in}{0.925687in}}%
\pgfpathlineto{\pgfqpoint{2.294067in}{0.925687in}}%
\pgfpathlineto{\pgfqpoint{2.294067in}{0.915234in}}%
\pgfpathlineto{\pgfqpoint{2.295599in}{1.030217in}}%
\pgfpathlineto{\pgfqpoint{2.295982in}{1.030217in}}%
\pgfpathlineto{\pgfqpoint{2.297130in}{0.915234in}}%
\pgfpathlineto{\pgfqpoint{2.297513in}{0.930914in}}%
\pgfpathlineto{\pgfqpoint{2.297896in}{0.930914in}}%
\pgfpathlineto{\pgfqpoint{2.299427in}{0.972726in}}%
\pgfpathlineto{\pgfqpoint{2.299810in}{0.972726in}}%
\pgfpathlineto{\pgfqpoint{2.299810in}{0.988405in}}%
\pgfpathlineto{\pgfqpoint{2.300193in}{0.910008in}}%
\pgfpathlineto{\pgfqpoint{2.301341in}{0.925687in}}%
\pgfpathlineto{\pgfqpoint{2.301724in}{0.925687in}}%
\pgfpathlineto{\pgfqpoint{2.302107in}{0.889102in}}%
\pgfpathlineto{\pgfqpoint{2.303255in}{1.019764in}}%
\pgfpathlineto{\pgfqpoint{2.303638in}{1.019764in}}%
\pgfpathlineto{\pgfqpoint{2.304404in}{0.904781in}}%
\pgfpathlineto{\pgfqpoint{2.305170in}{0.977952in}}%
\pgfpathlineto{\pgfqpoint{2.305552in}{0.977952in}}%
\pgfpathlineto{\pgfqpoint{2.306318in}{0.883875in}}%
\pgfpathlineto{\pgfqpoint{2.307084in}{1.030217in}}%
\pgfpathlineto{\pgfqpoint{2.307467in}{1.030217in}}%
\pgfpathlineto{\pgfqpoint{2.308232in}{0.899555in}}%
\pgfpathlineto{\pgfqpoint{2.308998in}{0.957046in}}%
\pgfpathlineto{\pgfqpoint{2.309381in}{0.957046in}}%
\pgfpathlineto{\pgfqpoint{2.309764in}{0.889102in}}%
\pgfpathlineto{\pgfqpoint{2.310912in}{0.910008in}}%
\pgfpathlineto{\pgfqpoint{2.311295in}{0.910008in}}%
\pgfpathlineto{\pgfqpoint{2.312060in}{0.868196in}}%
\pgfpathlineto{\pgfqpoint{2.311678in}{0.941367in}}%
\pgfpathlineto{\pgfqpoint{2.312826in}{0.899555in}}%
\pgfpathlineto{\pgfqpoint{2.313209in}{0.899555in}}%
\pgfpathlineto{\pgfqpoint{2.313592in}{0.967499in}}%
\pgfpathlineto{\pgfqpoint{2.314740in}{0.810705in}}%
\pgfpathlineto{\pgfqpoint{2.315123in}{0.810705in}}%
\pgfpathlineto{\pgfqpoint{2.315123in}{0.998858in}}%
\pgfpathlineto{\pgfqpoint{2.316654in}{0.920461in}}%
\pgfpathlineto{\pgfqpoint{2.317037in}{0.920461in}}%
\pgfpathlineto{\pgfqpoint{2.317420in}{0.873422in}}%
\pgfpathlineto{\pgfqpoint{2.318569in}{1.009311in}}%
\pgfpathlineto{\pgfqpoint{2.318951in}{1.009311in}}%
\pgfpathlineto{\pgfqpoint{2.318951in}{0.925687in}}%
\pgfpathlineto{\pgfqpoint{2.320483in}{0.941367in}}%
\pgfpathlineto{\pgfqpoint{2.320866in}{0.941367in}}%
\pgfpathlineto{\pgfqpoint{2.320866in}{0.951820in}}%
\pgfpathlineto{\pgfqpoint{2.321248in}{0.915234in}}%
\pgfpathlineto{\pgfqpoint{2.322397in}{0.930914in}}%
\pgfpathlineto{\pgfqpoint{2.322780in}{0.930914in}}%
\pgfpathlineto{\pgfqpoint{2.323163in}{0.831611in}}%
\pgfpathlineto{\pgfqpoint{2.323545in}{0.972726in}}%
\pgfpathlineto{\pgfqpoint{2.324311in}{0.878649in}}%
\pgfpathlineto{\pgfqpoint{2.324694in}{0.878649in}}%
\pgfpathlineto{\pgfqpoint{2.324694in}{0.857743in}}%
\pgfpathlineto{\pgfqpoint{2.325077in}{0.951820in}}%
\pgfpathlineto{\pgfqpoint{2.326225in}{0.894328in}}%
\pgfpathlineto{\pgfqpoint{2.326608in}{0.894328in}}%
\pgfpathlineto{\pgfqpoint{2.326991in}{0.946593in}}%
\pgfpathlineto{\pgfqpoint{2.328139in}{0.852516in}}%
\pgfpathlineto{\pgfqpoint{2.328522in}{0.852516in}}%
\pgfpathlineto{\pgfqpoint{2.330053in}{0.967499in}}%
\pgfpathlineto{\pgfqpoint{2.330436in}{0.967499in}}%
\pgfpathlineto{\pgfqpoint{2.331968in}{0.889102in}}%
\pgfpathlineto{\pgfqpoint{2.332350in}{0.889102in}}%
\pgfpathlineto{\pgfqpoint{2.333882in}{0.993631in}}%
\pgfpathlineto{\pgfqpoint{2.334265in}{0.993631in}}%
\pgfpathlineto{\pgfqpoint{2.335413in}{0.910008in}}%
\pgfpathlineto{\pgfqpoint{2.335796in}{0.920461in}}%
\pgfpathlineto{\pgfqpoint{2.336179in}{0.920461in}}%
\pgfpathlineto{\pgfqpoint{2.336179in}{0.857743in}}%
\pgfpathlineto{\pgfqpoint{2.336562in}{0.972726in}}%
\pgfpathlineto{\pgfqpoint{2.337710in}{0.972726in}}%
\pgfpathlineto{\pgfqpoint{2.338093in}{0.972726in}}%
\pgfpathlineto{\pgfqpoint{2.338859in}{0.847290in}}%
\pgfpathlineto{\pgfqpoint{2.339241in}{1.030217in}}%
\pgfpathlineto{\pgfqpoint{2.339624in}{0.936140in}}%
\pgfpathlineto{\pgfqpoint{2.340007in}{0.936140in}}%
\pgfpathlineto{\pgfqpoint{2.340007in}{0.957046in}}%
\pgfpathlineto{\pgfqpoint{2.341538in}{0.899555in}}%
\pgfpathlineto{\pgfqpoint{2.341921in}{0.899555in}}%
\pgfpathlineto{\pgfqpoint{2.343070in}{0.930914in}}%
\pgfpathlineto{\pgfqpoint{2.343453in}{0.852516in}}%
\pgfpathlineto{\pgfqpoint{2.343835in}{0.852516in}}%
\pgfpathlineto{\pgfqpoint{2.344601in}{0.941367in}}%
\pgfpathlineto{\pgfqpoint{2.345367in}{0.925687in}}%
\pgfpathlineto{\pgfqpoint{2.345750in}{0.925687in}}%
\pgfpathlineto{\pgfqpoint{2.345750in}{0.962273in}}%
\pgfpathlineto{\pgfqpoint{2.346132in}{0.873422in}}%
\pgfpathlineto{\pgfqpoint{2.347281in}{0.941367in}}%
\pgfpathlineto{\pgfqpoint{2.347664in}{0.941367in}}%
\pgfpathlineto{\pgfqpoint{2.347664in}{0.988405in}}%
\pgfpathlineto{\pgfqpoint{2.348047in}{0.868196in}}%
\pgfpathlineto{\pgfqpoint{2.349195in}{0.878649in}}%
\pgfpathlineto{\pgfqpoint{2.349578in}{0.878649in}}%
\pgfpathlineto{\pgfqpoint{2.350343in}{0.904781in}}%
\pgfpathlineto{\pgfqpoint{2.351109in}{0.826384in}}%
\pgfpathlineto{\pgfqpoint{2.351492in}{0.826384in}}%
\pgfpathlineto{\pgfqpoint{2.351492in}{0.988405in}}%
\pgfpathlineto{\pgfqpoint{2.353023in}{0.915234in}}%
\pgfpathlineto{\pgfqpoint{2.353789in}{0.915234in}}%
\pgfpathlineto{\pgfqpoint{2.354172in}{0.815931in}}%
\pgfpathlineto{\pgfqpoint{2.355320in}{0.951820in}}%
\pgfpathlineto{\pgfqpoint{2.355703in}{0.951820in}}%
\pgfpathlineto{\pgfqpoint{2.356852in}{0.878649in}}%
\pgfpathlineto{\pgfqpoint{2.356086in}{0.957046in}}%
\pgfpathlineto{\pgfqpoint{2.357234in}{0.904781in}}%
\pgfpathlineto{\pgfqpoint{2.357617in}{0.904781in}}%
\pgfpathlineto{\pgfqpoint{2.358383in}{0.930914in}}%
\pgfpathlineto{\pgfqpoint{2.358766in}{0.852516in}}%
\pgfpathlineto{\pgfqpoint{2.359149in}{0.883875in}}%
\pgfpathlineto{\pgfqpoint{2.359531in}{0.883875in}}%
\pgfpathlineto{\pgfqpoint{2.359914in}{0.983178in}}%
\pgfpathlineto{\pgfqpoint{2.361063in}{0.925687in}}%
\pgfpathlineto{\pgfqpoint{2.361446in}{0.925687in}}%
\pgfpathlineto{\pgfqpoint{2.362977in}{0.847290in}}%
\pgfpathlineto{\pgfqpoint{2.363360in}{0.847290in}}%
\pgfpathlineto{\pgfqpoint{2.363743in}{0.936140in}}%
\pgfpathlineto{\pgfqpoint{2.364891in}{0.899555in}}%
\pgfpathlineto{\pgfqpoint{2.365274in}{0.899555in}}%
\pgfpathlineto{\pgfqpoint{2.366422in}{0.862969in}}%
\pgfpathlineto{\pgfqpoint{2.365657in}{0.930914in}}%
\pgfpathlineto{\pgfqpoint{2.366805in}{0.873422in}}%
\pgfpathlineto{\pgfqpoint{2.367188in}{0.873422in}}%
\pgfpathlineto{\pgfqpoint{2.367571in}{0.946593in}}%
\pgfpathlineto{\pgfqpoint{2.368719in}{0.836837in}}%
\pgfpathlineto{\pgfqpoint{2.369102in}{0.836837in}}%
\pgfpathlineto{\pgfqpoint{2.370633in}{0.951820in}}%
\pgfpathlineto{\pgfqpoint{2.371016in}{0.951820in}}%
\pgfpathlineto{\pgfqpoint{2.371016in}{0.836837in}}%
\pgfpathlineto{\pgfqpoint{2.372548in}{0.857743in}}%
\pgfpathlineto{\pgfqpoint{2.373313in}{0.857743in}}%
\pgfpathlineto{\pgfqpoint{2.373313in}{0.805478in}}%
\pgfpathlineto{\pgfqpoint{2.373696in}{0.946593in}}%
\pgfpathlineto{\pgfqpoint{2.374845in}{0.883875in}}%
\pgfpathlineto{\pgfqpoint{2.375227in}{0.883875in}}%
\pgfpathlineto{\pgfqpoint{2.375227in}{0.826384in}}%
\pgfpathlineto{\pgfqpoint{2.375610in}{0.899555in}}%
\pgfpathlineto{\pgfqpoint{2.376759in}{0.836837in}}%
\pgfpathlineto{\pgfqpoint{2.377142in}{0.836837in}}%
\pgfpathlineto{\pgfqpoint{2.377142in}{0.941367in}}%
\pgfpathlineto{\pgfqpoint{2.378290in}{0.821158in}}%
\pgfpathlineto{\pgfqpoint{2.378673in}{0.868196in}}%
\pgfpathlineto{\pgfqpoint{2.379056in}{0.868196in}}%
\pgfpathlineto{\pgfqpoint{2.380204in}{0.826384in}}%
\pgfpathlineto{\pgfqpoint{2.380587in}{0.936140in}}%
\pgfpathlineto{\pgfqpoint{2.380970in}{0.936140in}}%
\pgfpathlineto{\pgfqpoint{2.380970in}{0.868196in}}%
\pgfpathlineto{\pgfqpoint{2.382501in}{0.930914in}}%
\pgfpathlineto{\pgfqpoint{2.382884in}{0.930914in}}%
\pgfpathlineto{\pgfqpoint{2.384033in}{0.836837in}}%
\pgfpathlineto{\pgfqpoint{2.384415in}{0.868196in}}%
\pgfpathlineto{\pgfqpoint{2.384798in}{0.868196in}}%
\pgfpathlineto{\pgfqpoint{2.385564in}{0.941367in}}%
\pgfpathlineto{\pgfqpoint{2.385947in}{0.831611in}}%
\pgfpathlineto{\pgfqpoint{2.386330in}{0.857743in}}%
\pgfpathlineto{\pgfqpoint{2.386712in}{0.857743in}}%
\pgfpathlineto{\pgfqpoint{2.386712in}{0.915234in}}%
\pgfpathlineto{\pgfqpoint{2.387478in}{0.852516in}}%
\pgfpathlineto{\pgfqpoint{2.388244in}{0.862969in}}%
\pgfpathlineto{\pgfqpoint{2.388626in}{0.862969in}}%
\pgfpathlineto{\pgfqpoint{2.389775in}{0.842063in}}%
\pgfpathlineto{\pgfqpoint{2.390158in}{0.883875in}}%
\pgfpathlineto{\pgfqpoint{2.390541in}{0.883875in}}%
\pgfpathlineto{\pgfqpoint{2.390923in}{0.899555in}}%
\pgfpathlineto{\pgfqpoint{2.392072in}{0.842063in}}%
\pgfpathlineto{\pgfqpoint{2.392455in}{0.842063in}}%
\pgfpathlineto{\pgfqpoint{2.393220in}{0.868196in}}%
\pgfpathlineto{\pgfqpoint{2.393603in}{0.795025in}}%
\pgfpathlineto{\pgfqpoint{2.393986in}{0.857743in}}%
\pgfpathlineto{\pgfqpoint{2.394369in}{0.857743in}}%
\pgfpathlineto{\pgfqpoint{2.395517in}{0.941367in}}%
\pgfpathlineto{\pgfqpoint{2.395900in}{0.810705in}}%
\pgfpathlineto{\pgfqpoint{2.396283in}{0.810705in}}%
\pgfpathlineto{\pgfqpoint{2.397814in}{0.894328in}}%
\pgfpathlineto{\pgfqpoint{2.398197in}{0.894328in}}%
\pgfpathlineto{\pgfqpoint{2.398580in}{0.821158in}}%
\pgfpathlineto{\pgfqpoint{2.398963in}{0.930914in}}%
\pgfpathlineto{\pgfqpoint{2.399729in}{0.826384in}}%
\pgfpathlineto{\pgfqpoint{2.400111in}{0.826384in}}%
\pgfpathlineto{\pgfqpoint{2.400494in}{0.894328in}}%
\pgfpathlineto{\pgfqpoint{2.401643in}{0.852516in}}%
\pgfpathlineto{\pgfqpoint{2.402026in}{0.852516in}}%
\pgfpathlineto{\pgfqpoint{2.402026in}{0.789799in}}%
\pgfpathlineto{\pgfqpoint{2.403174in}{0.920461in}}%
\pgfpathlineto{\pgfqpoint{2.403557in}{0.821158in}}%
\pgfpathlineto{\pgfqpoint{2.403940in}{0.821158in}}%
\pgfpathlineto{\pgfqpoint{2.404705in}{0.789799in}}%
\pgfpathlineto{\pgfqpoint{2.405471in}{0.878649in}}%
\pgfpathlineto{\pgfqpoint{2.405854in}{0.878649in}}%
\pgfpathlineto{\pgfqpoint{2.406237in}{0.810705in}}%
\pgfpathlineto{\pgfqpoint{2.406620in}{0.883875in}}%
\pgfpathlineto{\pgfqpoint{2.407385in}{0.847290in}}%
\pgfpathlineto{\pgfqpoint{2.407768in}{0.847290in}}%
\pgfpathlineto{\pgfqpoint{2.408534in}{0.910008in}}%
\pgfpathlineto{\pgfqpoint{2.409299in}{0.831611in}}%
\pgfpathlineto{\pgfqpoint{2.409682in}{0.831611in}}%
\pgfpathlineto{\pgfqpoint{2.409682in}{0.810705in}}%
\pgfpathlineto{\pgfqpoint{2.410448in}{0.889102in}}%
\pgfpathlineto{\pgfqpoint{2.411213in}{0.810705in}}%
\pgfpathlineto{\pgfqpoint{2.411596in}{0.810705in}}%
\pgfpathlineto{\pgfqpoint{2.412362in}{0.904781in}}%
\pgfpathlineto{\pgfqpoint{2.413128in}{0.894328in}}%
\pgfpathlineto{\pgfqpoint{2.413510in}{0.894328in}}%
\pgfpathlineto{\pgfqpoint{2.413893in}{0.821158in}}%
\pgfpathlineto{\pgfqpoint{2.415042in}{0.847290in}}%
\pgfpathlineto{\pgfqpoint{2.415425in}{0.847290in}}%
\pgfpathlineto{\pgfqpoint{2.416190in}{0.862969in}}%
\pgfpathlineto{\pgfqpoint{2.416956in}{0.800252in}}%
\pgfpathlineto{\pgfqpoint{2.417339in}{0.800252in}}%
\pgfpathlineto{\pgfqpoint{2.417722in}{0.784572in}}%
\pgfpathlineto{\pgfqpoint{2.418870in}{0.962273in}}%
\pgfpathlineto{\pgfqpoint{2.419253in}{0.962273in}}%
\pgfpathlineto{\pgfqpoint{2.419636in}{0.810705in}}%
\pgfpathlineto{\pgfqpoint{2.420784in}{0.842063in}}%
\pgfpathlineto{\pgfqpoint{2.421167in}{0.842063in}}%
\pgfpathlineto{\pgfqpoint{2.421933in}{0.910008in}}%
\pgfpathlineto{\pgfqpoint{2.422698in}{0.842063in}}%
\pgfpathlineto{\pgfqpoint{2.423081in}{0.842063in}}%
\pgfpathlineto{\pgfqpoint{2.423464in}{0.915234in}}%
\pgfpathlineto{\pgfqpoint{2.423464in}{0.836837in}}%
\pgfpathlineto{\pgfqpoint{2.424613in}{0.862969in}}%
\pgfpathlineto{\pgfqpoint{2.425378in}{0.862969in}}%
\pgfpathlineto{\pgfqpoint{2.425378in}{0.873422in}}%
\pgfpathlineto{\pgfqpoint{2.425761in}{0.826384in}}%
\pgfpathlineto{\pgfqpoint{2.426909in}{0.842063in}}%
\pgfpathlineto{\pgfqpoint{2.427675in}{0.842063in}}%
\pgfpathlineto{\pgfqpoint{2.428824in}{0.941367in}}%
\pgfpathlineto{\pgfqpoint{2.429206in}{0.910008in}}%
\pgfpathlineto{\pgfqpoint{2.429589in}{0.910008in}}%
\pgfpathlineto{\pgfqpoint{2.429589in}{0.925687in}}%
\pgfpathlineto{\pgfqpoint{2.431121in}{0.795025in}}%
\pgfpathlineto{\pgfqpoint{2.431503in}{0.795025in}}%
\pgfpathlineto{\pgfqpoint{2.431503in}{0.915234in}}%
\pgfpathlineto{\pgfqpoint{2.433035in}{0.831611in}}%
\pgfpathlineto{\pgfqpoint{2.433418in}{0.831611in}}%
\pgfpathlineto{\pgfqpoint{2.434566in}{0.805478in}}%
\pgfpathlineto{\pgfqpoint{2.434949in}{0.894328in}}%
\pgfpathlineto{\pgfqpoint{2.435332in}{0.894328in}}%
\pgfpathlineto{\pgfqpoint{2.435332in}{0.847290in}}%
\pgfpathlineto{\pgfqpoint{2.435715in}{0.957046in}}%
\pgfpathlineto{\pgfqpoint{2.436863in}{0.862969in}}%
\pgfpathlineto{\pgfqpoint{2.437246in}{0.862969in}}%
\pgfpathlineto{\pgfqpoint{2.438012in}{0.821158in}}%
\pgfpathlineto{\pgfqpoint{2.438777in}{0.946593in}}%
\pgfpathlineto{\pgfqpoint{2.439160in}{0.946593in}}%
\pgfpathlineto{\pgfqpoint{2.439926in}{0.831611in}}%
\pgfpathlineto{\pgfqpoint{2.440691in}{0.899555in}}%
\pgfpathlineto{\pgfqpoint{2.441074in}{0.899555in}}%
\pgfpathlineto{\pgfqpoint{2.441074in}{0.915234in}}%
\pgfpathlineto{\pgfqpoint{2.442606in}{0.821158in}}%
\pgfpathlineto{\pgfqpoint{2.443371in}{0.821158in}}%
\pgfpathlineto{\pgfqpoint{2.444137in}{0.894328in}}%
\pgfpathlineto{\pgfqpoint{2.444903in}{0.868196in}}%
\pgfpathlineto{\pgfqpoint{2.445285in}{0.868196in}}%
\pgfpathlineto{\pgfqpoint{2.446817in}{0.784572in}}%
\pgfpathlineto{\pgfqpoint{2.447199in}{0.784572in}}%
\pgfpathlineto{\pgfqpoint{2.447582in}{0.904781in}}%
\pgfpathlineto{\pgfqpoint{2.448731in}{0.889102in}}%
\pgfpathlineto{\pgfqpoint{2.449114in}{0.889102in}}%
\pgfpathlineto{\pgfqpoint{2.449879in}{0.789799in}}%
\pgfpathlineto{\pgfqpoint{2.450645in}{0.857743in}}%
\pgfpathlineto{\pgfqpoint{2.451028in}{0.857743in}}%
\pgfpathlineto{\pgfqpoint{2.451793in}{0.878649in}}%
\pgfpathlineto{\pgfqpoint{2.452559in}{0.826384in}}%
\pgfpathlineto{\pgfqpoint{2.452942in}{0.826384in}}%
\pgfpathlineto{\pgfqpoint{2.452942in}{0.904781in}}%
\pgfpathlineto{\pgfqpoint{2.454090in}{0.795025in}}%
\pgfpathlineto{\pgfqpoint{2.454473in}{0.868196in}}%
\pgfpathlineto{\pgfqpoint{2.454856in}{0.868196in}}%
\pgfpathlineto{\pgfqpoint{2.455239in}{0.805478in}}%
\pgfpathlineto{\pgfqpoint{2.456387in}{0.868196in}}%
\pgfpathlineto{\pgfqpoint{2.456770in}{0.868196in}}%
\pgfpathlineto{\pgfqpoint{2.457536in}{0.904781in}}%
\pgfpathlineto{\pgfqpoint{2.457153in}{0.842063in}}%
\pgfpathlineto{\pgfqpoint{2.458302in}{0.878649in}}%
\pgfpathlineto{\pgfqpoint{2.458684in}{0.878649in}}%
\pgfpathlineto{\pgfqpoint{2.459067in}{0.763666in}}%
\pgfpathlineto{\pgfqpoint{2.460216in}{0.826384in}}%
\pgfpathlineto{\pgfqpoint{2.460599in}{0.826384in}}%
\pgfpathlineto{\pgfqpoint{2.460599in}{0.868196in}}%
\pgfpathlineto{\pgfqpoint{2.462130in}{0.805478in}}%
\pgfpathlineto{\pgfqpoint{2.462513in}{0.805478in}}%
\pgfpathlineto{\pgfqpoint{2.462513in}{0.857743in}}%
\pgfpathlineto{\pgfqpoint{2.462896in}{0.789799in}}%
\pgfpathlineto{\pgfqpoint{2.464044in}{0.815931in}}%
\pgfpathlineto{\pgfqpoint{2.464427in}{0.815931in}}%
\pgfpathlineto{\pgfqpoint{2.464427in}{0.873422in}}%
\pgfpathlineto{\pgfqpoint{2.465958in}{0.873422in}}%
\pgfpathlineto{\pgfqpoint{2.466341in}{0.873422in}}%
\pgfpathlineto{\pgfqpoint{2.466341in}{0.805478in}}%
\pgfpathlineto{\pgfqpoint{2.466724in}{0.883875in}}%
\pgfpathlineto{\pgfqpoint{2.467872in}{0.836837in}}%
\pgfpathlineto{\pgfqpoint{2.468255in}{0.836837in}}%
\pgfpathlineto{\pgfqpoint{2.468638in}{0.899555in}}%
\pgfpathlineto{\pgfqpoint{2.469786in}{0.815931in}}%
\pgfpathlineto{\pgfqpoint{2.470169in}{0.815931in}}%
\pgfpathlineto{\pgfqpoint{2.471318in}{0.894328in}}%
\pgfpathlineto{\pgfqpoint{2.471701in}{0.810705in}}%
\pgfpathlineto{\pgfqpoint{2.472083in}{0.810705in}}%
\pgfpathlineto{\pgfqpoint{2.472466in}{0.894328in}}%
\pgfpathlineto{\pgfqpoint{2.473615in}{0.826384in}}%
\pgfpathlineto{\pgfqpoint{2.473998in}{0.826384in}}%
\pgfpathlineto{\pgfqpoint{2.473998in}{0.883875in}}%
\pgfpathlineto{\pgfqpoint{2.475529in}{0.800252in}}%
\pgfpathlineto{\pgfqpoint{2.475912in}{0.800252in}}%
\pgfpathlineto{\pgfqpoint{2.476295in}{0.847290in}}%
\pgfpathlineto{\pgfqpoint{2.477443in}{0.815931in}}%
\pgfpathlineto{\pgfqpoint{2.477826in}{0.815931in}}%
\pgfpathlineto{\pgfqpoint{2.478974in}{0.779346in}}%
\pgfpathlineto{\pgfqpoint{2.479357in}{0.873422in}}%
\pgfpathlineto{\pgfqpoint{2.479740in}{0.873422in}}%
\pgfpathlineto{\pgfqpoint{2.480889in}{0.789799in}}%
\pgfpathlineto{\pgfqpoint{2.481271in}{0.883875in}}%
\pgfpathlineto{\pgfqpoint{2.481654in}{0.883875in}}%
\pgfpathlineto{\pgfqpoint{2.481654in}{0.899555in}}%
\pgfpathlineto{\pgfqpoint{2.482803in}{0.821158in}}%
\pgfpathlineto{\pgfqpoint{2.483186in}{0.836837in}}%
\pgfpathlineto{\pgfqpoint{2.483568in}{0.836837in}}%
\pgfpathlineto{\pgfqpoint{2.483951in}{0.889102in}}%
\pgfpathlineto{\pgfqpoint{2.484717in}{0.774119in}}%
\pgfpathlineto{\pgfqpoint{2.485100in}{0.873422in}}%
\pgfpathlineto{\pgfqpoint{2.485482in}{0.873422in}}%
\pgfpathlineto{\pgfqpoint{2.485482in}{0.815931in}}%
\pgfpathlineto{\pgfqpoint{2.487014in}{0.915234in}}%
\pgfpathlineto{\pgfqpoint{2.487397in}{0.915234in}}%
\pgfpathlineto{\pgfqpoint{2.487397in}{0.789799in}}%
\pgfpathlineto{\pgfqpoint{2.488928in}{0.810705in}}%
\pgfpathlineto{\pgfqpoint{2.489311in}{0.810705in}}%
\pgfpathlineto{\pgfqpoint{2.490459in}{0.920461in}}%
\pgfpathlineto{\pgfqpoint{2.490842in}{0.836837in}}%
\pgfpathlineto{\pgfqpoint{2.491225in}{0.836837in}}%
\pgfpathlineto{\pgfqpoint{2.491225in}{0.779346in}}%
\pgfpathlineto{\pgfqpoint{2.491608in}{0.889102in}}%
\pgfpathlineto{\pgfqpoint{2.492756in}{0.810705in}}%
\pgfpathlineto{\pgfqpoint{2.493139in}{0.810705in}}%
\pgfpathlineto{\pgfqpoint{2.493522in}{0.873422in}}%
\pgfpathlineto{\pgfqpoint{2.494670in}{0.815931in}}%
\pgfpathlineto{\pgfqpoint{2.495053in}{0.815931in}}%
\pgfpathlineto{\pgfqpoint{2.495053in}{0.800252in}}%
\pgfpathlineto{\pgfqpoint{2.496202in}{0.862969in}}%
\pgfpathlineto{\pgfqpoint{2.496585in}{0.826384in}}%
\pgfpathlineto{\pgfqpoint{2.496967in}{0.826384in}}%
\pgfpathlineto{\pgfqpoint{2.497350in}{0.774119in}}%
\pgfpathlineto{\pgfqpoint{2.498499in}{0.889102in}}%
\pgfpathlineto{\pgfqpoint{2.498882in}{0.889102in}}%
\pgfpathlineto{\pgfqpoint{2.500413in}{0.768893in}}%
\pgfpathlineto{\pgfqpoint{2.500796in}{0.768893in}}%
\pgfpathlineto{\pgfqpoint{2.500796in}{0.873422in}}%
\pgfpathlineto{\pgfqpoint{2.502327in}{0.847290in}}%
\pgfpathlineto{\pgfqpoint{2.502710in}{0.847290in}}%
\pgfpathlineto{\pgfqpoint{2.503476in}{0.795025in}}%
\pgfpathlineto{\pgfqpoint{2.504241in}{0.795025in}}%
\pgfpathlineto{\pgfqpoint{2.504624in}{0.795025in}}%
\pgfpathlineto{\pgfqpoint{2.505390in}{0.789799in}}%
\pgfpathlineto{\pgfqpoint{2.506155in}{0.852516in}}%
\pgfpathlineto{\pgfqpoint{2.506538in}{0.852516in}}%
\pgfpathlineto{\pgfqpoint{2.506921in}{0.779346in}}%
\pgfpathlineto{\pgfqpoint{2.507687in}{0.862969in}}%
\pgfpathlineto{\pgfqpoint{2.508069in}{0.795025in}}%
\pgfpathlineto{\pgfqpoint{2.508452in}{0.795025in}}%
\pgfpathlineto{\pgfqpoint{2.509601in}{0.784572in}}%
\pgfpathlineto{\pgfqpoint{2.509984in}{0.868196in}}%
\pgfpathlineto{\pgfqpoint{2.510366in}{0.868196in}}%
\pgfpathlineto{\pgfqpoint{2.511132in}{0.784572in}}%
\pgfpathlineto{\pgfqpoint{2.510749in}{0.910008in}}%
\pgfpathlineto{\pgfqpoint{2.511898in}{0.800252in}}%
\pgfpathlineto{\pgfqpoint{2.512281in}{0.800252in}}%
\pgfpathlineto{\pgfqpoint{2.512281in}{0.899555in}}%
\pgfpathlineto{\pgfqpoint{2.513812in}{0.789799in}}%
\pgfpathlineto{\pgfqpoint{2.514195in}{0.789799in}}%
\pgfpathlineto{\pgfqpoint{2.514195in}{0.779346in}}%
\pgfpathlineto{\pgfqpoint{2.514578in}{0.862969in}}%
\pgfpathlineto{\pgfqpoint{2.515726in}{0.821158in}}%
\pgfpathlineto{\pgfqpoint{2.516109in}{0.821158in}}%
\pgfpathlineto{\pgfqpoint{2.516109in}{0.836837in}}%
\pgfpathlineto{\pgfqpoint{2.517640in}{0.758440in}}%
\pgfpathlineto{\pgfqpoint{2.518023in}{0.758440in}}%
\pgfpathlineto{\pgfqpoint{2.518406in}{0.826384in}}%
\pgfpathlineto{\pgfqpoint{2.519554in}{0.789799in}}%
\pgfpathlineto{\pgfqpoint{2.519937in}{0.789799in}}%
\pgfpathlineto{\pgfqpoint{2.519937in}{0.784572in}}%
\pgfpathlineto{\pgfqpoint{2.520703in}{0.852516in}}%
\pgfpathlineto{\pgfqpoint{2.521469in}{0.852516in}}%
\pgfpathlineto{\pgfqpoint{2.521851in}{0.852516in}}%
\pgfpathlineto{\pgfqpoint{2.521851in}{0.784572in}}%
\pgfpathlineto{\pgfqpoint{2.523383in}{0.800252in}}%
\pgfpathlineto{\pgfqpoint{2.523765in}{0.800252in}}%
\pgfpathlineto{\pgfqpoint{2.524914in}{0.789799in}}%
\pgfpathlineto{\pgfqpoint{2.525297in}{0.847290in}}%
\pgfpathlineto{\pgfqpoint{2.525680in}{0.847290in}}%
\pgfpathlineto{\pgfqpoint{2.526445in}{0.789799in}}%
\pgfpathlineto{\pgfqpoint{2.527211in}{0.857743in}}%
\pgfpathlineto{\pgfqpoint{2.527594in}{0.857743in}}%
\pgfpathlineto{\pgfqpoint{2.528359in}{0.758440in}}%
\pgfpathlineto{\pgfqpoint{2.529125in}{0.862969in}}%
\pgfpathlineto{\pgfqpoint{2.529508in}{0.862969in}}%
\pgfpathlineto{\pgfqpoint{2.530274in}{0.789799in}}%
\pgfpathlineto{\pgfqpoint{2.531039in}{0.836837in}}%
\pgfpathlineto{\pgfqpoint{2.531422in}{0.836837in}}%
\pgfpathlineto{\pgfqpoint{2.531422in}{0.857743in}}%
\pgfpathlineto{\pgfqpoint{2.531805in}{0.768893in}}%
\pgfpathlineto{\pgfqpoint{2.532953in}{0.815931in}}%
\pgfpathlineto{\pgfqpoint{2.533336in}{0.815931in}}%
\pgfpathlineto{\pgfqpoint{2.533336in}{0.795025in}}%
\pgfpathlineto{\pgfqpoint{2.533719in}{0.831611in}}%
\pgfpathlineto{\pgfqpoint{2.534868in}{0.805478in}}%
\pgfpathlineto{\pgfqpoint{2.535250in}{0.805478in}}%
\pgfpathlineto{\pgfqpoint{2.535250in}{0.852516in}}%
\pgfpathlineto{\pgfqpoint{2.536782in}{0.774119in}}%
\pgfpathlineto{\pgfqpoint{2.537165in}{0.774119in}}%
\pgfpathlineto{\pgfqpoint{2.537547in}{0.836837in}}%
\pgfpathlineto{\pgfqpoint{2.538696in}{0.784572in}}%
\pgfpathlineto{\pgfqpoint{2.539079in}{0.784572in}}%
\pgfpathlineto{\pgfqpoint{2.540610in}{0.842063in}}%
\pgfpathlineto{\pgfqpoint{2.540993in}{0.842063in}}%
\pgfpathlineto{\pgfqpoint{2.541376in}{0.862969in}}%
\pgfpathlineto{\pgfqpoint{2.542524in}{0.763666in}}%
\pgfpathlineto{\pgfqpoint{2.542907in}{0.763666in}}%
\pgfpathlineto{\pgfqpoint{2.543290in}{0.862969in}}%
\pgfpathlineto{\pgfqpoint{2.544438in}{0.815931in}}%
\pgfpathlineto{\pgfqpoint{2.544821in}{0.815931in}}%
\pgfpathlineto{\pgfqpoint{2.545970in}{0.763666in}}%
\pgfpathlineto{\pgfqpoint{2.545587in}{0.826384in}}%
\pgfpathlineto{\pgfqpoint{2.546352in}{0.826384in}}%
\pgfpathlineto{\pgfqpoint{2.546735in}{0.826384in}}%
\pgfpathlineto{\pgfqpoint{2.547118in}{0.784572in}}%
\pgfpathlineto{\pgfqpoint{2.548267in}{0.805478in}}%
\pgfpathlineto{\pgfqpoint{2.548649in}{0.805478in}}%
\pgfpathlineto{\pgfqpoint{2.549415in}{0.826384in}}%
\pgfpathlineto{\pgfqpoint{2.549798in}{0.763666in}}%
\pgfpathlineto{\pgfqpoint{2.550181in}{0.768893in}}%
\pgfpathlineto{\pgfqpoint{2.550564in}{0.768893in}}%
\pgfpathlineto{\pgfqpoint{2.550564in}{0.826384in}}%
\pgfpathlineto{\pgfqpoint{2.552095in}{0.763666in}}%
\pgfpathlineto{\pgfqpoint{2.552478in}{0.763666in}}%
\pgfpathlineto{\pgfqpoint{2.554009in}{0.836837in}}%
\pgfpathlineto{\pgfqpoint{2.554392in}{0.836837in}}%
\pgfpathlineto{\pgfqpoint{2.554392in}{0.789799in}}%
\pgfpathlineto{\pgfqpoint{2.554775in}{0.852516in}}%
\pgfpathlineto{\pgfqpoint{2.555923in}{0.836837in}}%
\pgfpathlineto{\pgfqpoint{2.556306in}{0.836837in}}%
\pgfpathlineto{\pgfqpoint{2.557072in}{0.805478in}}%
\pgfpathlineto{\pgfqpoint{2.557837in}{0.873422in}}%
\pgfpathlineto{\pgfqpoint{2.558220in}{0.873422in}}%
\pgfpathlineto{\pgfqpoint{2.558986in}{0.795025in}}%
\pgfpathlineto{\pgfqpoint{2.559752in}{0.821158in}}%
\pgfpathlineto{\pgfqpoint{2.560134in}{0.821158in}}%
\pgfpathlineto{\pgfqpoint{2.560517in}{0.774119in}}%
\pgfpathlineto{\pgfqpoint{2.561666in}{0.836837in}}%
\pgfpathlineto{\pgfqpoint{2.562048in}{0.836837in}}%
\pgfpathlineto{\pgfqpoint{2.562048in}{0.795025in}}%
\pgfpathlineto{\pgfqpoint{2.563197in}{0.847290in}}%
\pgfpathlineto{\pgfqpoint{2.563580in}{0.810705in}}%
\pgfpathlineto{\pgfqpoint{2.563963in}{0.810705in}}%
\pgfpathlineto{\pgfqpoint{2.565111in}{0.768893in}}%
\pgfpathlineto{\pgfqpoint{2.564345in}{0.815931in}}%
\pgfpathlineto{\pgfqpoint{2.565494in}{0.784572in}}%
\pgfpathlineto{\pgfqpoint{2.565877in}{0.784572in}}%
\pgfpathlineto{\pgfqpoint{2.565877in}{0.899555in}}%
\pgfpathlineto{\pgfqpoint{2.567025in}{0.753213in}}%
\pgfpathlineto{\pgfqpoint{2.567408in}{0.847290in}}%
\pgfpathlineto{\pgfqpoint{2.568174in}{0.847290in}}%
\pgfpathlineto{\pgfqpoint{2.569705in}{0.774119in}}%
\pgfpathlineto{\pgfqpoint{2.570088in}{0.774119in}}%
\pgfpathlineto{\pgfqpoint{2.571619in}{0.868196in}}%
\pgfpathlineto{\pgfqpoint{2.572002in}{0.868196in}}%
\pgfpathlineto{\pgfqpoint{2.572768in}{0.795025in}}%
\pgfpathlineto{\pgfqpoint{2.573533in}{0.810705in}}%
\pgfpathlineto{\pgfqpoint{2.574299in}{0.810705in}}%
\pgfpathlineto{\pgfqpoint{2.575448in}{0.831611in}}%
\pgfpathlineto{\pgfqpoint{2.575065in}{0.716628in}}%
\pgfpathlineto{\pgfqpoint{2.575830in}{0.774119in}}%
\pgfpathlineto{\pgfqpoint{2.576213in}{0.774119in}}%
\pgfpathlineto{\pgfqpoint{2.577362in}{0.847290in}}%
\pgfpathlineto{\pgfqpoint{2.577745in}{0.784572in}}%
\pgfpathlineto{\pgfqpoint{2.578127in}{0.784572in}}%
\pgfpathlineto{\pgfqpoint{2.579276in}{0.852516in}}%
\pgfpathlineto{\pgfqpoint{2.579659in}{0.768893in}}%
\pgfpathlineto{\pgfqpoint{2.580042in}{0.768893in}}%
\pgfpathlineto{\pgfqpoint{2.580042in}{0.862969in}}%
\pgfpathlineto{\pgfqpoint{2.581190in}{0.727081in}}%
\pgfpathlineto{\pgfqpoint{2.581573in}{0.795025in}}%
\pgfpathlineto{\pgfqpoint{2.581956in}{0.795025in}}%
\pgfpathlineto{\pgfqpoint{2.582721in}{0.836837in}}%
\pgfpathlineto{\pgfqpoint{2.583104in}{0.779346in}}%
\pgfpathlineto{\pgfqpoint{2.583487in}{0.810705in}}%
\pgfpathlineto{\pgfqpoint{2.583870in}{0.810705in}}%
\pgfpathlineto{\pgfqpoint{2.583870in}{0.842063in}}%
\pgfpathlineto{\pgfqpoint{2.585018in}{0.779346in}}%
\pgfpathlineto{\pgfqpoint{2.585401in}{0.836837in}}%
\pgfpathlineto{\pgfqpoint{2.585784in}{0.836837in}}%
\pgfpathlineto{\pgfqpoint{2.585784in}{0.774119in}}%
\pgfpathlineto{\pgfqpoint{2.587315in}{0.831611in}}%
\pgfpathlineto{\pgfqpoint{2.587698in}{0.831611in}}%
\pgfpathlineto{\pgfqpoint{2.588464in}{0.774119in}}%
\pgfpathlineto{\pgfqpoint{2.589229in}{0.784572in}}%
\pgfpathlineto{\pgfqpoint{2.589612in}{0.784572in}}%
\pgfpathlineto{\pgfqpoint{2.589612in}{0.878649in}}%
\pgfpathlineto{\pgfqpoint{2.591144in}{0.815931in}}%
\pgfpathlineto{\pgfqpoint{2.591526in}{0.815931in}}%
\pgfpathlineto{\pgfqpoint{2.591909in}{0.763666in}}%
\pgfpathlineto{\pgfqpoint{2.592292in}{0.857743in}}%
\pgfpathlineto{\pgfqpoint{2.593058in}{0.768893in}}%
\pgfpathlineto{\pgfqpoint{2.593441in}{0.768893in}}%
\pgfpathlineto{\pgfqpoint{2.594589in}{0.815931in}}%
\pgfpathlineto{\pgfqpoint{2.593823in}{0.758440in}}%
\pgfpathlineto{\pgfqpoint{2.594972in}{0.779346in}}%
\pgfpathlineto{\pgfqpoint{2.595355in}{0.779346in}}%
\pgfpathlineto{\pgfqpoint{2.596886in}{0.815931in}}%
\pgfpathlineto{\pgfqpoint{2.597269in}{0.815931in}}%
\pgfpathlineto{\pgfqpoint{2.598035in}{0.774119in}}%
\pgfpathlineto{\pgfqpoint{2.597652in}{0.852516in}}%
\pgfpathlineto{\pgfqpoint{2.598800in}{0.800252in}}%
\pgfpathlineto{\pgfqpoint{2.599183in}{0.800252in}}%
\pgfpathlineto{\pgfqpoint{2.599949in}{0.742760in}}%
\pgfpathlineto{\pgfqpoint{2.600714in}{0.815931in}}%
\pgfpathlineto{\pgfqpoint{2.601097in}{0.815931in}}%
\pgfpathlineto{\pgfqpoint{2.601480in}{0.747987in}}%
\pgfpathlineto{\pgfqpoint{2.601863in}{0.852516in}}%
\pgfpathlineto{\pgfqpoint{2.602628in}{0.758440in}}%
\pgfpathlineto{\pgfqpoint{2.603011in}{0.758440in}}%
\pgfpathlineto{\pgfqpoint{2.604543in}{0.831611in}}%
\pgfpathlineto{\pgfqpoint{2.604925in}{0.831611in}}%
\pgfpathlineto{\pgfqpoint{2.606457in}{0.747987in}}%
\pgfpathlineto{\pgfqpoint{2.606840in}{0.747987in}}%
\pgfpathlineto{\pgfqpoint{2.607605in}{0.821158in}}%
\pgfpathlineto{\pgfqpoint{2.608371in}{0.763666in}}%
\pgfpathlineto{\pgfqpoint{2.608754in}{0.763666in}}%
\pgfpathlineto{\pgfqpoint{2.608754in}{0.753213in}}%
\pgfpathlineto{\pgfqpoint{2.609519in}{0.821158in}}%
\pgfpathlineto{\pgfqpoint{2.610285in}{0.800252in}}%
\pgfpathlineto{\pgfqpoint{2.610668in}{0.800252in}}%
\pgfpathlineto{\pgfqpoint{2.610668in}{0.831611in}}%
\pgfpathlineto{\pgfqpoint{2.612199in}{0.753213in}}%
\pgfpathlineto{\pgfqpoint{2.612582in}{0.753213in}}%
\pgfpathlineto{\pgfqpoint{2.613731in}{0.810705in}}%
\pgfpathlineto{\pgfqpoint{2.614113in}{0.789799in}}%
\pgfpathlineto{\pgfqpoint{2.614879in}{0.789799in}}%
\pgfpathlineto{\pgfqpoint{2.614879in}{0.810705in}}%
\pgfpathlineto{\pgfqpoint{2.616028in}{0.768893in}}%
\pgfpathlineto{\pgfqpoint{2.616410in}{0.784572in}}%
\pgfpathlineto{\pgfqpoint{2.616793in}{0.784572in}}%
\pgfpathlineto{\pgfqpoint{2.616793in}{0.742760in}}%
\pgfpathlineto{\pgfqpoint{2.618325in}{0.821158in}}%
\pgfpathlineto{\pgfqpoint{2.618707in}{0.821158in}}%
\pgfpathlineto{\pgfqpoint{2.619090in}{0.763666in}}%
\pgfpathlineto{\pgfqpoint{2.619473in}{0.826384in}}%
\pgfpathlineto{\pgfqpoint{2.620239in}{0.768893in}}%
\pgfpathlineto{\pgfqpoint{2.620621in}{0.768893in}}%
\pgfpathlineto{\pgfqpoint{2.621770in}{0.831611in}}%
\pgfpathlineto{\pgfqpoint{2.622153in}{0.742760in}}%
\pgfpathlineto{\pgfqpoint{2.622536in}{0.742760in}}%
\pgfpathlineto{\pgfqpoint{2.622536in}{0.821158in}}%
\pgfpathlineto{\pgfqpoint{2.624067in}{0.768893in}}%
\pgfpathlineto{\pgfqpoint{2.624450in}{0.768893in}}%
\pgfpathlineto{\pgfqpoint{2.624833in}{0.826384in}}%
\pgfpathlineto{\pgfqpoint{2.625981in}{0.732307in}}%
\pgfpathlineto{\pgfqpoint{2.626364in}{0.732307in}}%
\pgfpathlineto{\pgfqpoint{2.626747in}{0.810705in}}%
\pgfpathlineto{\pgfqpoint{2.627895in}{0.810705in}}%
\pgfpathlineto{\pgfqpoint{2.628278in}{0.810705in}}%
\pgfpathlineto{\pgfqpoint{2.628278in}{0.826384in}}%
\pgfpathlineto{\pgfqpoint{2.628661in}{0.763666in}}%
\pgfpathlineto{\pgfqpoint{2.629809in}{0.789799in}}%
\pgfpathlineto{\pgfqpoint{2.630192in}{0.789799in}}%
\pgfpathlineto{\pgfqpoint{2.630192in}{0.711401in}}%
\pgfpathlineto{\pgfqpoint{2.630958in}{0.800252in}}%
\pgfpathlineto{\pgfqpoint{2.631724in}{0.768893in}}%
\pgfpathlineto{\pgfqpoint{2.632106in}{0.768893in}}%
\pgfpathlineto{\pgfqpoint{2.632106in}{0.784572in}}%
\pgfpathlineto{\pgfqpoint{2.633255in}{0.742760in}}%
\pgfpathlineto{\pgfqpoint{2.633638in}{0.779346in}}%
\pgfpathlineto{\pgfqpoint{2.634021in}{0.779346in}}%
\pgfpathlineto{\pgfqpoint{2.634403in}{0.774119in}}%
\pgfpathlineto{\pgfqpoint{2.635552in}{0.831611in}}%
\pgfpathlineto{\pgfqpoint{2.635935in}{0.831611in}}%
\pgfpathlineto{\pgfqpoint{2.637083in}{0.800252in}}%
\pgfpathlineto{\pgfqpoint{2.637466in}{0.831611in}}%
\pgfpathlineto{\pgfqpoint{2.637849in}{0.831611in}}%
\pgfpathlineto{\pgfqpoint{2.637849in}{0.763666in}}%
\pgfpathlineto{\pgfqpoint{2.639380in}{0.821158in}}%
\pgfpathlineto{\pgfqpoint{2.639763in}{0.821158in}}%
\pgfpathlineto{\pgfqpoint{2.639763in}{0.795025in}}%
\pgfpathlineto{\pgfqpoint{2.641294in}{0.873422in}}%
\pgfpathlineto{\pgfqpoint{2.641677in}{0.873422in}}%
\pgfpathlineto{\pgfqpoint{2.643208in}{0.753213in}}%
\pgfpathlineto{\pgfqpoint{2.643591in}{0.753213in}}%
\pgfpathlineto{\pgfqpoint{2.644740in}{0.815931in}}%
\pgfpathlineto{\pgfqpoint{2.645123in}{0.800252in}}%
\pgfpathlineto{\pgfqpoint{2.645505in}{0.800252in}}%
\pgfpathlineto{\pgfqpoint{2.646271in}{0.831611in}}%
\pgfpathlineto{\pgfqpoint{2.647037in}{0.768893in}}%
\pgfpathlineto{\pgfqpoint{2.647420in}{0.768893in}}%
\pgfpathlineto{\pgfqpoint{2.647420in}{0.742760in}}%
\pgfpathlineto{\pgfqpoint{2.648185in}{0.836837in}}%
\pgfpathlineto{\pgfqpoint{2.648951in}{0.774119in}}%
\pgfpathlineto{\pgfqpoint{2.649717in}{0.774119in}}%
\pgfpathlineto{\pgfqpoint{2.649717in}{0.800252in}}%
\pgfpathlineto{\pgfqpoint{2.650865in}{0.753213in}}%
\pgfpathlineto{\pgfqpoint{2.651248in}{0.774119in}}%
\pgfpathlineto{\pgfqpoint{2.651631in}{0.774119in}}%
\pgfpathlineto{\pgfqpoint{2.651631in}{0.831611in}}%
\pgfpathlineto{\pgfqpoint{2.652779in}{0.768893in}}%
\pgfpathlineto{\pgfqpoint{2.653162in}{0.768893in}}%
\pgfpathlineto{\pgfqpoint{2.653545in}{0.768893in}}%
\pgfpathlineto{\pgfqpoint{2.654311in}{0.815931in}}%
\pgfpathlineto{\pgfqpoint{2.655076in}{0.742760in}}%
\pgfpathlineto{\pgfqpoint{2.655459in}{0.742760in}}%
\pgfpathlineto{\pgfqpoint{2.656990in}{0.815931in}}%
\pgfpathlineto{\pgfqpoint{2.657373in}{0.815931in}}%
\pgfpathlineto{\pgfqpoint{2.658522in}{0.758440in}}%
\pgfpathlineto{\pgfqpoint{2.658904in}{0.831611in}}%
\pgfpathlineto{\pgfqpoint{2.659287in}{0.831611in}}%
\pgfpathlineto{\pgfqpoint{2.660053in}{0.763666in}}%
\pgfpathlineto{\pgfqpoint{2.660436in}{0.836837in}}%
\pgfpathlineto{\pgfqpoint{2.660819in}{0.800252in}}%
\pgfpathlineto{\pgfqpoint{2.661201in}{0.800252in}}%
\pgfpathlineto{\pgfqpoint{2.662733in}{0.758440in}}%
\pgfpathlineto{\pgfqpoint{2.663116in}{0.758440in}}%
\pgfpathlineto{\pgfqpoint{2.663116in}{0.878649in}}%
\pgfpathlineto{\pgfqpoint{2.664647in}{0.721854in}}%
\pgfpathlineto{\pgfqpoint{2.665030in}{0.721854in}}%
\pgfpathlineto{\pgfqpoint{2.665413in}{0.842063in}}%
\pgfpathlineto{\pgfqpoint{2.666561in}{0.779346in}}%
\pgfpathlineto{\pgfqpoint{2.666944in}{0.779346in}}%
\pgfpathlineto{\pgfqpoint{2.667327in}{0.857743in}}%
\pgfpathlineto{\pgfqpoint{2.668475in}{0.774119in}}%
\pgfpathlineto{\pgfqpoint{2.668858in}{0.774119in}}%
\pgfpathlineto{\pgfqpoint{2.670007in}{0.842063in}}%
\pgfpathlineto{\pgfqpoint{2.670389in}{0.784572in}}%
\pgfpathlineto{\pgfqpoint{2.670772in}{0.784572in}}%
\pgfpathlineto{\pgfqpoint{2.671155in}{0.821158in}}%
\pgfpathlineto{\pgfqpoint{2.672304in}{0.758440in}}%
\pgfpathlineto{\pgfqpoint{2.672686in}{0.758440in}}%
\pgfpathlineto{\pgfqpoint{2.673452in}{0.810705in}}%
\pgfpathlineto{\pgfqpoint{2.673069in}{0.742760in}}%
\pgfpathlineto{\pgfqpoint{2.674218in}{0.753213in}}%
\pgfpathlineto{\pgfqpoint{2.674601in}{0.753213in}}%
\pgfpathlineto{\pgfqpoint{2.675749in}{0.836837in}}%
\pgfpathlineto{\pgfqpoint{2.676132in}{0.779346in}}%
\pgfpathlineto{\pgfqpoint{2.676515in}{0.779346in}}%
\pgfpathlineto{\pgfqpoint{2.677280in}{0.795025in}}%
\pgfpathlineto{\pgfqpoint{2.676898in}{0.758440in}}%
\pgfpathlineto{\pgfqpoint{2.678046in}{0.758440in}}%
\pgfpathlineto{\pgfqpoint{2.678429in}{0.758440in}}%
\pgfpathlineto{\pgfqpoint{2.678812in}{0.847290in}}%
\pgfpathlineto{\pgfqpoint{2.679577in}{0.742760in}}%
\pgfpathlineto{\pgfqpoint{2.679960in}{0.768893in}}%
\pgfpathlineto{\pgfqpoint{2.680343in}{0.768893in}}%
\pgfpathlineto{\pgfqpoint{2.680343in}{0.821158in}}%
\pgfpathlineto{\pgfqpoint{2.681491in}{0.758440in}}%
\pgfpathlineto{\pgfqpoint{2.681874in}{0.784572in}}%
\pgfpathlineto{\pgfqpoint{2.682257in}{0.784572in}}%
\pgfpathlineto{\pgfqpoint{2.683023in}{0.826384in}}%
\pgfpathlineto{\pgfqpoint{2.682640in}{0.732307in}}%
\pgfpathlineto{\pgfqpoint{2.683788in}{0.795025in}}%
\pgfpathlineto{\pgfqpoint{2.684171in}{0.795025in}}%
\pgfpathlineto{\pgfqpoint{2.684937in}{0.742760in}}%
\pgfpathlineto{\pgfqpoint{2.685320in}{0.821158in}}%
\pgfpathlineto{\pgfqpoint{2.685703in}{0.742760in}}%
\pgfpathlineto{\pgfqpoint{2.686085in}{0.742760in}}%
\pgfpathlineto{\pgfqpoint{2.686085in}{0.810705in}}%
\pgfpathlineto{\pgfqpoint{2.687617in}{0.758440in}}%
\pgfpathlineto{\pgfqpoint{2.688000in}{0.758440in}}%
\pgfpathlineto{\pgfqpoint{2.689531in}{0.826384in}}%
\pgfpathlineto{\pgfqpoint{2.689914in}{0.826384in}}%
\pgfpathlineto{\pgfqpoint{2.691062in}{0.721854in}}%
\pgfpathlineto{\pgfqpoint{2.691445in}{0.779346in}}%
\pgfpathlineto{\pgfqpoint{2.691828in}{0.779346in}}%
\pgfpathlineto{\pgfqpoint{2.692594in}{0.747987in}}%
\pgfpathlineto{\pgfqpoint{2.692211in}{0.810705in}}%
\pgfpathlineto{\pgfqpoint{2.693359in}{0.774119in}}%
\pgfpathlineto{\pgfqpoint{2.693742in}{0.774119in}}%
\pgfpathlineto{\pgfqpoint{2.693742in}{0.753213in}}%
\pgfpathlineto{\pgfqpoint{2.694508in}{0.789799in}}%
\pgfpathlineto{\pgfqpoint{2.695273in}{0.784572in}}%
\pgfpathlineto{\pgfqpoint{2.695656in}{0.784572in}}%
\pgfpathlineto{\pgfqpoint{2.696805in}{0.810705in}}%
\pgfpathlineto{\pgfqpoint{2.696422in}{0.753213in}}%
\pgfpathlineto{\pgfqpoint{2.697188in}{0.805478in}}%
\pgfpathlineto{\pgfqpoint{2.697570in}{0.805478in}}%
\pgfpathlineto{\pgfqpoint{2.698719in}{0.768893in}}%
\pgfpathlineto{\pgfqpoint{2.698336in}{0.815931in}}%
\pgfpathlineto{\pgfqpoint{2.699102in}{0.805478in}}%
\pgfpathlineto{\pgfqpoint{2.699484in}{0.805478in}}%
\pgfpathlineto{\pgfqpoint{2.699484in}{0.753213in}}%
\pgfpathlineto{\pgfqpoint{2.701016in}{0.768893in}}%
\pgfpathlineto{\pgfqpoint{2.701399in}{0.768893in}}%
\pgfpathlineto{\pgfqpoint{2.702164in}{0.737534in}}%
\pgfpathlineto{\pgfqpoint{2.702547in}{0.815931in}}%
\pgfpathlineto{\pgfqpoint{2.702930in}{0.805478in}}%
\pgfpathlineto{\pgfqpoint{2.703313in}{0.805478in}}%
\pgfpathlineto{\pgfqpoint{2.703313in}{0.815931in}}%
\pgfpathlineto{\pgfqpoint{2.704844in}{0.763666in}}%
\pgfpathlineto{\pgfqpoint{2.705227in}{0.763666in}}%
\pgfpathlineto{\pgfqpoint{2.705610in}{0.836837in}}%
\pgfpathlineto{\pgfqpoint{2.705993in}{0.758440in}}%
\pgfpathlineto{\pgfqpoint{2.706758in}{0.768893in}}%
\pgfpathlineto{\pgfqpoint{2.707141in}{0.768893in}}%
\pgfpathlineto{\pgfqpoint{2.707141in}{0.852516in}}%
\pgfpathlineto{\pgfqpoint{2.708290in}{0.747987in}}%
\pgfpathlineto{\pgfqpoint{2.708672in}{0.831611in}}%
\pgfpathlineto{\pgfqpoint{2.709055in}{0.831611in}}%
\pgfpathlineto{\pgfqpoint{2.710204in}{0.747987in}}%
\pgfpathlineto{\pgfqpoint{2.710587in}{0.800252in}}%
\pgfpathlineto{\pgfqpoint{2.710969in}{0.800252in}}%
\pgfpathlineto{\pgfqpoint{2.711352in}{0.737534in}}%
\pgfpathlineto{\pgfqpoint{2.712501in}{0.795025in}}%
\pgfpathlineto{\pgfqpoint{2.712884in}{0.795025in}}%
\pgfpathlineto{\pgfqpoint{2.714032in}{0.732307in}}%
\pgfpathlineto{\pgfqpoint{2.714415in}{0.795025in}}%
\pgfpathlineto{\pgfqpoint{2.714798in}{0.795025in}}%
\pgfpathlineto{\pgfqpoint{2.715563in}{0.716628in}}%
\pgfpathlineto{\pgfqpoint{2.715181in}{0.800252in}}%
\pgfpathlineto{\pgfqpoint{2.716329in}{0.758440in}}%
\pgfpathlineto{\pgfqpoint{2.716712in}{0.758440in}}%
\pgfpathlineto{\pgfqpoint{2.716712in}{0.737534in}}%
\pgfpathlineto{\pgfqpoint{2.717477in}{0.795025in}}%
\pgfpathlineto{\pgfqpoint{2.718243in}{0.768893in}}%
\pgfpathlineto{\pgfqpoint{2.718626in}{0.768893in}}%
\pgfpathlineto{\pgfqpoint{2.719392in}{0.742760in}}%
\pgfpathlineto{\pgfqpoint{2.719009in}{0.774119in}}%
\pgfpathlineto{\pgfqpoint{2.720157in}{0.774119in}}%
\pgfpathlineto{\pgfqpoint{2.720540in}{0.774119in}}%
\pgfpathlineto{\pgfqpoint{2.721689in}{0.805478in}}%
\pgfpathlineto{\pgfqpoint{2.722071in}{0.763666in}}%
\pgfpathlineto{\pgfqpoint{2.722454in}{0.763666in}}%
\pgfpathlineto{\pgfqpoint{2.722837in}{0.826384in}}%
\pgfpathlineto{\pgfqpoint{2.723986in}{0.763666in}}%
\pgfpathlineto{\pgfqpoint{2.724368in}{0.763666in}}%
\pgfpathlineto{\pgfqpoint{2.725517in}{0.784572in}}%
\pgfpathlineto{\pgfqpoint{2.725900in}{0.768893in}}%
\pgfpathlineto{\pgfqpoint{2.726283in}{0.768893in}}%
\pgfpathlineto{\pgfqpoint{2.727048in}{0.810705in}}%
\pgfpathlineto{\pgfqpoint{2.727814in}{0.758440in}}%
\pgfpathlineto{\pgfqpoint{2.728197in}{0.758440in}}%
\pgfpathlineto{\pgfqpoint{2.729345in}{0.800252in}}%
\pgfpathlineto{\pgfqpoint{2.729728in}{0.768893in}}%
\pgfpathlineto{\pgfqpoint{2.730111in}{0.768893in}}%
\pgfpathlineto{\pgfqpoint{2.731259in}{0.810705in}}%
\pgfpathlineto{\pgfqpoint{2.730494in}{0.747987in}}%
\pgfpathlineto{\pgfqpoint{2.731642in}{0.774119in}}%
\pgfpathlineto{\pgfqpoint{2.732025in}{0.774119in}}%
\pgfpathlineto{\pgfqpoint{2.732025in}{0.805478in}}%
\pgfpathlineto{\pgfqpoint{2.732408in}{0.732307in}}%
\pgfpathlineto{\pgfqpoint{2.733556in}{0.768893in}}%
\pgfpathlineto{\pgfqpoint{2.733939in}{0.768893in}}%
\pgfpathlineto{\pgfqpoint{2.734322in}{0.784572in}}%
\pgfpathlineto{\pgfqpoint{2.735471in}{0.742760in}}%
\pgfpathlineto{\pgfqpoint{2.735853in}{0.742760in}}%
\pgfpathlineto{\pgfqpoint{2.736236in}{0.826384in}}%
\pgfpathlineto{\pgfqpoint{2.737385in}{0.768893in}}%
\pgfpathlineto{\pgfqpoint{2.738150in}{0.768893in}}%
\pgfpathlineto{\pgfqpoint{2.738533in}{0.810705in}}%
\pgfpathlineto{\pgfqpoint{2.739682in}{0.716628in}}%
\pgfpathlineto{\pgfqpoint{2.740064in}{0.716628in}}%
\pgfpathlineto{\pgfqpoint{2.740447in}{0.810705in}}%
\pgfpathlineto{\pgfqpoint{2.741596in}{0.789799in}}%
\pgfpathlineto{\pgfqpoint{2.741979in}{0.789799in}}%
\pgfpathlineto{\pgfqpoint{2.741979in}{0.810705in}}%
\pgfpathlineto{\pgfqpoint{2.742744in}{0.763666in}}%
\pgfpathlineto{\pgfqpoint{2.743510in}{0.784572in}}%
\pgfpathlineto{\pgfqpoint{2.743893in}{0.784572in}}%
\pgfpathlineto{\pgfqpoint{2.744276in}{0.721854in}}%
\pgfpathlineto{\pgfqpoint{2.745424in}{0.789799in}}%
\pgfpathlineto{\pgfqpoint{2.745807in}{0.789799in}}%
\pgfpathlineto{\pgfqpoint{2.747338in}{0.737534in}}%
\pgfpathlineto{\pgfqpoint{2.747721in}{0.737534in}}%
\pgfpathlineto{\pgfqpoint{2.749252in}{0.779346in}}%
\pgfpathlineto{\pgfqpoint{2.749635in}{0.779346in}}%
\pgfpathlineto{\pgfqpoint{2.750018in}{0.763666in}}%
\pgfpathlineto{\pgfqpoint{2.751167in}{0.800252in}}%
\pgfpathlineto{\pgfqpoint{2.751549in}{0.800252in}}%
\pgfpathlineto{\pgfqpoint{2.753081in}{0.758440in}}%
\pgfpathlineto{\pgfqpoint{2.753464in}{0.758440in}}%
\pgfpathlineto{\pgfqpoint{2.754995in}{0.815931in}}%
\pgfpathlineto{\pgfqpoint{2.755378in}{0.815931in}}%
\pgfpathlineto{\pgfqpoint{2.756909in}{0.747987in}}%
\pgfpathlineto{\pgfqpoint{2.757292in}{0.747987in}}%
\pgfpathlineto{\pgfqpoint{2.757292in}{0.742760in}}%
\pgfpathlineto{\pgfqpoint{2.758823in}{0.815931in}}%
\pgfpathlineto{\pgfqpoint{2.759206in}{0.815931in}}%
\pgfpathlineto{\pgfqpoint{2.759972in}{0.727081in}}%
\pgfpathlineto{\pgfqpoint{2.760737in}{0.789799in}}%
\pgfpathlineto{\pgfqpoint{2.761120in}{0.789799in}}%
\pgfpathlineto{\pgfqpoint{2.761886in}{0.732307in}}%
\pgfpathlineto{\pgfqpoint{2.762651in}{0.779346in}}%
\pgfpathlineto{\pgfqpoint{2.763417in}{0.779346in}}%
\pgfpathlineto{\pgfqpoint{2.763800in}{0.768893in}}%
\pgfpathlineto{\pgfqpoint{2.764566in}{0.789799in}}%
\pgfpathlineto{\pgfqpoint{2.764948in}{0.774119in}}%
\pgfpathlineto{\pgfqpoint{2.765331in}{0.774119in}}%
\pgfpathlineto{\pgfqpoint{2.766480in}{0.805478in}}%
\pgfpathlineto{\pgfqpoint{2.766863in}{0.795025in}}%
\pgfpathlineto{\pgfqpoint{2.767245in}{0.795025in}}%
\pgfpathlineto{\pgfqpoint{2.768011in}{0.753213in}}%
\pgfpathlineto{\pgfqpoint{2.768394in}{0.831611in}}%
\pgfpathlineto{\pgfqpoint{2.768777in}{0.768893in}}%
\pgfpathlineto{\pgfqpoint{2.769160in}{0.768893in}}%
\pgfpathlineto{\pgfqpoint{2.770308in}{0.711401in}}%
\pgfpathlineto{\pgfqpoint{2.769925in}{0.805478in}}%
\pgfpathlineto{\pgfqpoint{2.770691in}{0.758440in}}%
\pgfpathlineto{\pgfqpoint{2.771074in}{0.758440in}}%
\pgfpathlineto{\pgfqpoint{2.771074in}{0.815931in}}%
\pgfpathlineto{\pgfqpoint{2.772605in}{0.774119in}}%
\pgfpathlineto{\pgfqpoint{2.772988in}{0.774119in}}%
\pgfpathlineto{\pgfqpoint{2.772988in}{0.805478in}}%
\pgfpathlineto{\pgfqpoint{2.773371in}{0.732307in}}%
\pgfpathlineto{\pgfqpoint{2.774519in}{0.779346in}}%
\pgfpathlineto{\pgfqpoint{2.774902in}{0.779346in}}%
\pgfpathlineto{\pgfqpoint{2.774902in}{0.836837in}}%
\pgfpathlineto{\pgfqpoint{2.776433in}{0.774119in}}%
\pgfpathlineto{\pgfqpoint{2.777199in}{0.774119in}}%
\pgfpathlineto{\pgfqpoint{2.777199in}{0.742760in}}%
\pgfpathlineto{\pgfqpoint{2.778730in}{0.810705in}}%
\pgfpathlineto{\pgfqpoint{2.779113in}{0.810705in}}%
\pgfpathlineto{\pgfqpoint{2.779113in}{0.831611in}}%
\pgfpathlineto{\pgfqpoint{2.779879in}{0.721854in}}%
\pgfpathlineto{\pgfqpoint{2.780644in}{0.753213in}}%
\pgfpathlineto{\pgfqpoint{2.781027in}{0.753213in}}%
\pgfpathlineto{\pgfqpoint{2.781027in}{0.747987in}}%
\pgfpathlineto{\pgfqpoint{2.782176in}{0.779346in}}%
\pgfpathlineto{\pgfqpoint{2.782559in}{0.753213in}}%
\pgfpathlineto{\pgfqpoint{2.782941in}{0.753213in}}%
\pgfpathlineto{\pgfqpoint{2.783707in}{0.795025in}}%
\pgfpathlineto{\pgfqpoint{2.784473in}{0.747987in}}%
\pgfpathlineto{\pgfqpoint{2.784856in}{0.747987in}}%
\pgfpathlineto{\pgfqpoint{2.785238in}{0.732307in}}%
\pgfpathlineto{\pgfqpoint{2.785621in}{0.784572in}}%
\pgfpathlineto{\pgfqpoint{2.786387in}{0.763666in}}%
\pgfpathlineto{\pgfqpoint{2.786770in}{0.763666in}}%
\pgfpathlineto{\pgfqpoint{2.787918in}{0.789799in}}%
\pgfpathlineto{\pgfqpoint{2.787535in}{0.727081in}}%
\pgfpathlineto{\pgfqpoint{2.788301in}{0.747987in}}%
\pgfpathlineto{\pgfqpoint{2.788684in}{0.747987in}}%
\pgfpathlineto{\pgfqpoint{2.788684in}{0.737534in}}%
\pgfpathlineto{\pgfqpoint{2.790215in}{0.810705in}}%
\pgfpathlineto{\pgfqpoint{2.790598in}{0.810705in}}%
\pgfpathlineto{\pgfqpoint{2.792129in}{0.727081in}}%
\pgfpathlineto{\pgfqpoint{2.792512in}{0.727081in}}%
\pgfpathlineto{\pgfqpoint{2.792512in}{0.779346in}}%
\pgfpathlineto{\pgfqpoint{2.794044in}{0.768893in}}%
\pgfpathlineto{\pgfqpoint{2.794426in}{0.768893in}}%
\pgfpathlineto{\pgfqpoint{2.794426in}{0.784572in}}%
\pgfpathlineto{\pgfqpoint{2.795192in}{0.753213in}}%
\pgfpathlineto{\pgfqpoint{2.795958in}{0.763666in}}%
\pgfpathlineto{\pgfqpoint{2.796340in}{0.763666in}}%
\pgfpathlineto{\pgfqpoint{2.796340in}{0.716628in}}%
\pgfpathlineto{\pgfqpoint{2.797489in}{0.831611in}}%
\pgfpathlineto{\pgfqpoint{2.797872in}{0.753213in}}%
\pgfpathlineto{\pgfqpoint{2.798255in}{0.753213in}}%
\pgfpathlineto{\pgfqpoint{2.798255in}{0.810705in}}%
\pgfpathlineto{\pgfqpoint{2.799403in}{0.747987in}}%
\pgfpathlineto{\pgfqpoint{2.799786in}{0.758440in}}%
\pgfpathlineto{\pgfqpoint{2.800169in}{0.758440in}}%
\pgfpathlineto{\pgfqpoint{2.800169in}{0.805478in}}%
\pgfpathlineto{\pgfqpoint{2.801317in}{0.716628in}}%
\pgfpathlineto{\pgfqpoint{2.801700in}{0.779346in}}%
\pgfpathlineto{\pgfqpoint{2.802083in}{0.779346in}}%
\pgfpathlineto{\pgfqpoint{2.802466in}{0.727081in}}%
\pgfpathlineto{\pgfqpoint{2.803614in}{0.779346in}}%
\pgfpathlineto{\pgfqpoint{2.803997in}{0.779346in}}%
\pgfpathlineto{\pgfqpoint{2.803997in}{0.753213in}}%
\pgfpathlineto{\pgfqpoint{2.804380in}{0.800252in}}%
\pgfpathlineto{\pgfqpoint{2.805528in}{0.758440in}}%
\pgfpathlineto{\pgfqpoint{2.805911in}{0.758440in}}%
\pgfpathlineto{\pgfqpoint{2.806677in}{0.732307in}}%
\pgfpathlineto{\pgfqpoint{2.806294in}{0.800252in}}%
\pgfpathlineto{\pgfqpoint{2.807443in}{0.732307in}}%
\pgfpathlineto{\pgfqpoint{2.807825in}{0.732307in}}%
\pgfpathlineto{\pgfqpoint{2.807825in}{0.721854in}}%
\pgfpathlineto{\pgfqpoint{2.808591in}{0.795025in}}%
\pgfpathlineto{\pgfqpoint{2.809357in}{0.779346in}}%
\pgfpathlineto{\pgfqpoint{2.809740in}{0.779346in}}%
\pgfpathlineto{\pgfqpoint{2.809740in}{0.826384in}}%
\pgfpathlineto{\pgfqpoint{2.810505in}{0.727081in}}%
\pgfpathlineto{\pgfqpoint{2.811271in}{0.821158in}}%
\pgfpathlineto{\pgfqpoint{2.811654in}{0.821158in}}%
\pgfpathlineto{\pgfqpoint{2.812037in}{0.742760in}}%
\pgfpathlineto{\pgfqpoint{2.813185in}{0.789799in}}%
\pgfpathlineto{\pgfqpoint{2.813568in}{0.789799in}}%
\pgfpathlineto{\pgfqpoint{2.813951in}{0.747987in}}%
\pgfpathlineto{\pgfqpoint{2.815099in}{0.779346in}}%
\pgfpathlineto{\pgfqpoint{2.815482in}{0.779346in}}%
\pgfpathlineto{\pgfqpoint{2.815482in}{0.842063in}}%
\pgfpathlineto{\pgfqpoint{2.817013in}{0.753213in}}%
\pgfpathlineto{\pgfqpoint{2.817396in}{0.753213in}}%
\pgfpathlineto{\pgfqpoint{2.818162in}{0.805478in}}%
\pgfpathlineto{\pgfqpoint{2.818927in}{0.779346in}}%
\pgfpathlineto{\pgfqpoint{2.819310in}{0.779346in}}%
\pgfpathlineto{\pgfqpoint{2.820076in}{0.700949in}}%
\pgfpathlineto{\pgfqpoint{2.820842in}{0.800252in}}%
\pgfpathlineto{\pgfqpoint{2.821224in}{0.800252in}}%
\pgfpathlineto{\pgfqpoint{2.822373in}{0.737534in}}%
\pgfpathlineto{\pgfqpoint{2.822756in}{0.779346in}}%
\pgfpathlineto{\pgfqpoint{2.823521in}{0.779346in}}%
\pgfpathlineto{\pgfqpoint{2.824287in}{0.810705in}}%
\pgfpathlineto{\pgfqpoint{2.824670in}{0.727081in}}%
\pgfpathlineto{\pgfqpoint{2.825053in}{0.758440in}}%
\pgfpathlineto{\pgfqpoint{2.825818in}{0.758440in}}%
\pgfpathlineto{\pgfqpoint{2.826584in}{0.716628in}}%
\pgfpathlineto{\pgfqpoint{2.826967in}{0.774119in}}%
\pgfpathlineto{\pgfqpoint{2.827350in}{0.747987in}}%
\pgfpathlineto{\pgfqpoint{2.827733in}{0.747987in}}%
\pgfpathlineto{\pgfqpoint{2.827733in}{0.721854in}}%
\pgfpathlineto{\pgfqpoint{2.828498in}{0.784572in}}%
\pgfpathlineto{\pgfqpoint{2.829264in}{0.768893in}}%
\pgfpathlineto{\pgfqpoint{2.829647in}{0.768893in}}%
\pgfpathlineto{\pgfqpoint{2.830030in}{0.795025in}}%
\pgfpathlineto{\pgfqpoint{2.830795in}{0.742760in}}%
\pgfpathlineto{\pgfqpoint{2.831178in}{0.779346in}}%
\pgfpathlineto{\pgfqpoint{2.831561in}{0.779346in}}%
\pgfpathlineto{\pgfqpoint{2.832327in}{0.795025in}}%
\pgfpathlineto{\pgfqpoint{2.833092in}{0.768893in}}%
\pgfpathlineto{\pgfqpoint{2.833475in}{0.768893in}}%
\pgfpathlineto{\pgfqpoint{2.834241in}{0.732307in}}%
\pgfpathlineto{\pgfqpoint{2.835006in}{0.815931in}}%
\pgfpathlineto{\pgfqpoint{2.835389in}{0.815931in}}%
\pgfpathlineto{\pgfqpoint{2.836155in}{0.737534in}}%
\pgfpathlineto{\pgfqpoint{2.836920in}{0.753213in}}%
\pgfpathlineto{\pgfqpoint{2.837303in}{0.753213in}}%
\pgfpathlineto{\pgfqpoint{2.838069in}{0.727081in}}%
\pgfpathlineto{\pgfqpoint{2.838835in}{0.800252in}}%
\pgfpathlineto{\pgfqpoint{2.839217in}{0.800252in}}%
\pgfpathlineto{\pgfqpoint{2.840749in}{0.727081in}}%
\pgfpathlineto{\pgfqpoint{2.841132in}{0.727081in}}%
\pgfpathlineto{\pgfqpoint{2.842663in}{0.779346in}}%
\pgfpathlineto{\pgfqpoint{2.843046in}{0.779346in}}%
\pgfpathlineto{\pgfqpoint{2.843046in}{0.789799in}}%
\pgfpathlineto{\pgfqpoint{2.843811in}{0.732307in}}%
\pgfpathlineto{\pgfqpoint{2.844577in}{0.753213in}}%
\pgfpathlineto{\pgfqpoint{2.844960in}{0.753213in}}%
\pgfpathlineto{\pgfqpoint{2.845343in}{0.732307in}}%
\pgfpathlineto{\pgfqpoint{2.846108in}{0.779346in}}%
\pgfpathlineto{\pgfqpoint{2.846491in}{0.737534in}}%
\pgfpathlineto{\pgfqpoint{2.846874in}{0.737534in}}%
\pgfpathlineto{\pgfqpoint{2.847640in}{0.784572in}}%
\pgfpathlineto{\pgfqpoint{2.848023in}{0.721854in}}%
\pgfpathlineto{\pgfqpoint{2.848405in}{0.727081in}}%
\pgfpathlineto{\pgfqpoint{2.848788in}{0.727081in}}%
\pgfpathlineto{\pgfqpoint{2.848788in}{0.721854in}}%
\pgfpathlineto{\pgfqpoint{2.850320in}{0.789799in}}%
\pgfpathlineto{\pgfqpoint{2.850702in}{0.789799in}}%
\pgfpathlineto{\pgfqpoint{2.851085in}{0.779346in}}%
\pgfpathlineto{\pgfqpoint{2.851468in}{0.821158in}}%
\pgfpathlineto{\pgfqpoint{2.852234in}{0.789799in}}%
\pgfpathlineto{\pgfqpoint{2.852616in}{0.789799in}}%
\pgfpathlineto{\pgfqpoint{2.852999in}{0.826384in}}%
\pgfpathlineto{\pgfqpoint{2.854148in}{0.737534in}}%
\pgfpathlineto{\pgfqpoint{2.854531in}{0.737534in}}%
\pgfpathlineto{\pgfqpoint{2.854913in}{0.805478in}}%
\pgfpathlineto{\pgfqpoint{2.855679in}{0.727081in}}%
\pgfpathlineto{\pgfqpoint{2.856062in}{0.768893in}}%
\pgfpathlineto{\pgfqpoint{2.856445in}{0.768893in}}%
\pgfpathlineto{\pgfqpoint{2.856445in}{0.784572in}}%
\pgfpathlineto{\pgfqpoint{2.856828in}{0.747987in}}%
\pgfpathlineto{\pgfqpoint{2.857976in}{0.753213in}}%
\pgfpathlineto{\pgfqpoint{2.858359in}{0.753213in}}%
\pgfpathlineto{\pgfqpoint{2.858359in}{0.727081in}}%
\pgfpathlineto{\pgfqpoint{2.859890in}{0.727081in}}%
\pgfpathlineto{\pgfqpoint{2.860273in}{0.727081in}}%
\pgfpathlineto{\pgfqpoint{2.860273in}{0.789799in}}%
\pgfpathlineto{\pgfqpoint{2.861804in}{0.763666in}}%
\pgfpathlineto{\pgfqpoint{2.862187in}{0.763666in}}%
\pgfpathlineto{\pgfqpoint{2.862570in}{0.805478in}}%
\pgfpathlineto{\pgfqpoint{2.863719in}{0.732307in}}%
\pgfpathlineto{\pgfqpoint{2.864101in}{0.732307in}}%
\pgfpathlineto{\pgfqpoint{2.864484in}{0.795025in}}%
\pgfpathlineto{\pgfqpoint{2.865633in}{0.784572in}}%
\pgfpathlineto{\pgfqpoint{2.866016in}{0.784572in}}%
\pgfpathlineto{\pgfqpoint{2.866016in}{0.753213in}}%
\pgfpathlineto{\pgfqpoint{2.867547in}{0.789799in}}%
\pgfpathlineto{\pgfqpoint{2.868695in}{0.789799in}}%
\pgfpathlineto{\pgfqpoint{2.870227in}{0.721854in}}%
\pgfpathlineto{\pgfqpoint{2.870610in}{0.721854in}}%
\pgfpathlineto{\pgfqpoint{2.871375in}{0.774119in}}%
\pgfpathlineto{\pgfqpoint{2.872141in}{0.774119in}}%
\pgfpathlineto{\pgfqpoint{2.872524in}{0.774119in}}%
\pgfpathlineto{\pgfqpoint{2.872906in}{0.721854in}}%
\pgfpathlineto{\pgfqpoint{2.873672in}{0.795025in}}%
\pgfpathlineto{\pgfqpoint{2.874055in}{0.774119in}}%
\pgfpathlineto{\pgfqpoint{2.874438in}{0.774119in}}%
\pgfpathlineto{\pgfqpoint{2.874438in}{0.742760in}}%
\pgfpathlineto{\pgfqpoint{2.875969in}{0.795025in}}%
\pgfpathlineto{\pgfqpoint{2.876352in}{0.795025in}}%
\pgfpathlineto{\pgfqpoint{2.876735in}{0.747987in}}%
\pgfpathlineto{\pgfqpoint{2.877883in}{0.758440in}}%
\pgfpathlineto{\pgfqpoint{2.878266in}{0.758440in}}%
\pgfpathlineto{\pgfqpoint{2.879415in}{0.805478in}}%
\pgfpathlineto{\pgfqpoint{2.878649in}{0.742760in}}%
\pgfpathlineto{\pgfqpoint{2.879797in}{0.742760in}}%
\pgfpathlineto{\pgfqpoint{2.880180in}{0.742760in}}%
\pgfpathlineto{\pgfqpoint{2.880946in}{0.768893in}}%
\pgfpathlineto{\pgfqpoint{2.881329in}{0.727081in}}%
\pgfpathlineto{\pgfqpoint{2.881712in}{0.753213in}}%
\pgfpathlineto{\pgfqpoint{2.882094in}{0.753213in}}%
\pgfpathlineto{\pgfqpoint{2.882477in}{0.716628in}}%
\pgfpathlineto{\pgfqpoint{2.882477in}{0.789799in}}%
\pgfpathlineto{\pgfqpoint{2.883626in}{0.716628in}}%
\pgfpathlineto{\pgfqpoint{2.884009in}{0.716628in}}%
\pgfpathlineto{\pgfqpoint{2.884009in}{0.779346in}}%
\pgfpathlineto{\pgfqpoint{2.885540in}{0.716628in}}%
\pgfpathlineto{\pgfqpoint{2.885923in}{0.716628in}}%
\pgfpathlineto{\pgfqpoint{2.887454in}{0.763666in}}%
\pgfpathlineto{\pgfqpoint{2.887837in}{0.763666in}}%
\pgfpathlineto{\pgfqpoint{2.888220in}{0.805478in}}%
\pgfpathlineto{\pgfqpoint{2.888985in}{0.737534in}}%
\pgfpathlineto{\pgfqpoint{2.889368in}{0.747987in}}%
\pgfpathlineto{\pgfqpoint{2.889751in}{0.747987in}}%
\pgfpathlineto{\pgfqpoint{2.890899in}{0.711401in}}%
\pgfpathlineto{\pgfqpoint{2.891282in}{0.800252in}}%
\pgfpathlineto{\pgfqpoint{2.891665in}{0.800252in}}%
\pgfpathlineto{\pgfqpoint{2.891665in}{0.747987in}}%
\pgfpathlineto{\pgfqpoint{2.892431in}{0.805478in}}%
\pgfpathlineto{\pgfqpoint{2.893196in}{0.768893in}}%
\pgfpathlineto{\pgfqpoint{2.893579in}{0.768893in}}%
\pgfpathlineto{\pgfqpoint{2.894345in}{0.779346in}}%
\pgfpathlineto{\pgfqpoint{2.895111in}{0.700949in}}%
\pgfpathlineto{\pgfqpoint{2.895493in}{0.700949in}}%
\pgfpathlineto{\pgfqpoint{2.896642in}{0.779346in}}%
\pgfpathlineto{\pgfqpoint{2.897025in}{0.737534in}}%
\pgfpathlineto{\pgfqpoint{2.897408in}{0.737534in}}%
\pgfpathlineto{\pgfqpoint{2.897408in}{0.747987in}}%
\pgfpathlineto{\pgfqpoint{2.898173in}{0.727081in}}%
\pgfpathlineto{\pgfqpoint{2.898939in}{0.737534in}}%
\pgfpathlineto{\pgfqpoint{2.899322in}{0.737534in}}%
\pgfpathlineto{\pgfqpoint{2.900470in}{0.721854in}}%
\pgfpathlineto{\pgfqpoint{2.900853in}{0.779346in}}%
\pgfpathlineto{\pgfqpoint{2.901236in}{0.779346in}}%
\pgfpathlineto{\pgfqpoint{2.902002in}{0.789799in}}%
\pgfpathlineto{\pgfqpoint{2.902767in}{0.706175in}}%
\pgfpathlineto{\pgfqpoint{2.903150in}{0.706175in}}%
\pgfpathlineto{\pgfqpoint{2.903533in}{0.763666in}}%
\pgfpathlineto{\pgfqpoint{2.904681in}{0.737534in}}%
\pgfpathlineto{\pgfqpoint{2.905064in}{0.737534in}}%
\pgfpathlineto{\pgfqpoint{2.906213in}{0.753213in}}%
\pgfpathlineto{\pgfqpoint{2.905830in}{0.695722in}}%
\pgfpathlineto{\pgfqpoint{2.906596in}{0.747987in}}%
\pgfpathlineto{\pgfqpoint{2.906978in}{0.747987in}}%
\pgfpathlineto{\pgfqpoint{2.907361in}{0.716628in}}%
\pgfpathlineto{\pgfqpoint{2.908510in}{0.784572in}}%
\pgfpathlineto{\pgfqpoint{2.908893in}{0.784572in}}%
\pgfpathlineto{\pgfqpoint{2.908893in}{0.737534in}}%
\pgfpathlineto{\pgfqpoint{2.910041in}{0.789799in}}%
\pgfpathlineto{\pgfqpoint{2.910424in}{0.758440in}}%
\pgfpathlineto{\pgfqpoint{2.910807in}{0.758440in}}%
\pgfpathlineto{\pgfqpoint{2.911572in}{0.779346in}}%
\pgfpathlineto{\pgfqpoint{2.912338in}{0.727081in}}%
\pgfpathlineto{\pgfqpoint{2.912721in}{0.727081in}}%
\pgfpathlineto{\pgfqpoint{2.913104in}{0.758440in}}%
\pgfpathlineto{\pgfqpoint{2.913869in}{0.706175in}}%
\pgfpathlineto{\pgfqpoint{2.914252in}{0.742760in}}%
\pgfpathlineto{\pgfqpoint{2.914635in}{0.742760in}}%
\pgfpathlineto{\pgfqpoint{2.914635in}{0.700949in}}%
\pgfpathlineto{\pgfqpoint{2.915401in}{0.763666in}}%
\pgfpathlineto{\pgfqpoint{2.916166in}{0.732307in}}%
\pgfpathlineto{\pgfqpoint{2.916549in}{0.732307in}}%
\pgfpathlineto{\pgfqpoint{2.917698in}{0.742760in}}%
\pgfpathlineto{\pgfqpoint{2.918080in}{0.706175in}}%
\pgfpathlineto{\pgfqpoint{2.918463in}{0.706175in}}%
\pgfpathlineto{\pgfqpoint{2.918463in}{0.847290in}}%
\pgfpathlineto{\pgfqpoint{2.919995in}{0.747987in}}%
\pgfpathlineto{\pgfqpoint{2.920377in}{0.747987in}}%
\pgfpathlineto{\pgfqpoint{2.920377in}{0.716628in}}%
\pgfpathlineto{\pgfqpoint{2.920760in}{0.716628in}}%
\pgfusepath{stroke}%
\end{pgfscope}%
\begin{pgfscope}%
\pgfsetrectcap%
\pgfsetmiterjoin%
\pgfsetlinewidth{0.803000pt}%
\definecolor{currentstroke}{rgb}{0.000000,0.000000,0.000000}%
\pgfsetstrokecolor{currentstroke}%
\pgfsetdash{}{0pt}%
\pgfpathmoveto{\pgfqpoint{0.781944in}{0.552778in}}%
\pgfpathlineto{\pgfqpoint{0.781944in}{2.202778in}}%
\pgfusepath{stroke}%
\end{pgfscope}%
\begin{pgfscope}%
\pgfsetrectcap%
\pgfsetmiterjoin%
\pgfsetlinewidth{0.803000pt}%
\definecolor{currentstroke}{rgb}{0.000000,0.000000,0.000000}%
\pgfsetstrokecolor{currentstroke}%
\pgfsetdash{}{0pt}%
\pgfpathmoveto{\pgfqpoint{2.920660in}{0.552778in}}%
\pgfpathlineto{\pgfqpoint{2.920660in}{2.202778in}}%
\pgfusepath{stroke}%
\end{pgfscope}%
\begin{pgfscope}%
\pgfsetrectcap%
\pgfsetmiterjoin%
\pgfsetlinewidth{0.803000pt}%
\definecolor{currentstroke}{rgb}{0.000000,0.000000,0.000000}%
\pgfsetstrokecolor{currentstroke}%
\pgfsetdash{}{0pt}%
\pgfpathmoveto{\pgfqpoint{0.781944in}{0.552778in}}%
\pgfpathlineto{\pgfqpoint{2.920660in}{0.552778in}}%
\pgfusepath{stroke}%
\end{pgfscope}%
\begin{pgfscope}%
\pgfsetrectcap%
\pgfsetmiterjoin%
\pgfsetlinewidth{0.803000pt}%
\definecolor{currentstroke}{rgb}{0.000000,0.000000,0.000000}%
\pgfsetstrokecolor{currentstroke}%
\pgfsetdash{}{0pt}%
\pgfpathmoveto{\pgfqpoint{0.781944in}{2.202778in}}%
\pgfpathlineto{\pgfqpoint{2.920660in}{2.202778in}}%
\pgfusepath{stroke}%
\end{pgfscope}%
\begin{pgfscope}%
\definecolor{textcolor}{rgb}{0.000000,0.000000,0.000000}%
\pgfsetstrokecolor{textcolor}%
\pgfsetfillcolor{textcolor}%
\pgftext[x=1.851302in,y=2.286111in,,base]{\color{textcolor}\rmfamily\fontsize{12.000000}{14.400000}\selectfont Energiespektrum A}%
\end{pgfscope}%
\begin{pgfscope}%
\pgfsetbuttcap%
\pgfsetmiterjoin%
\definecolor{currentfill}{rgb}{1.000000,1.000000,1.000000}%
\pgfsetfillcolor{currentfill}%
\pgfsetfillopacity{0.800000}%
\pgfsetlinewidth{1.003750pt}%
\definecolor{currentstroke}{rgb}{0.800000,0.800000,0.800000}%
\pgfsetstrokecolor{currentstroke}%
\pgfsetstrokeopacity{0.800000}%
\pgfsetdash{}{0pt}%
\pgfpathmoveto{\pgfqpoint{0.879167in}{1.510834in}}%
\pgfpathlineto{\pgfqpoint{1.727361in}{1.510834in}}%
\pgfpathquadraticcurveto{\pgfqpoint{1.755139in}{1.510834in}}{\pgfqpoint{1.755139in}{1.538612in}}%
\pgfpathlineto{\pgfqpoint{1.755139in}{2.105556in}}%
\pgfpathquadraticcurveto{\pgfqpoint{1.755139in}{2.133333in}}{\pgfqpoint{1.727361in}{2.133333in}}%
\pgfpathlineto{\pgfqpoint{0.879167in}{2.133333in}}%
\pgfpathquadraticcurveto{\pgfqpoint{0.851389in}{2.133333in}}{\pgfqpoint{0.851389in}{2.105556in}}%
\pgfpathlineto{\pgfqpoint{0.851389in}{1.538612in}}%
\pgfpathquadraticcurveto{\pgfqpoint{0.851389in}{1.510834in}}{\pgfqpoint{0.879167in}{1.510834in}}%
\pgfpathclose%
\pgfusepath{stroke,fill}%
\end{pgfscope}%
\begin{pgfscope}%
\pgfsetrectcap%
\pgfsetroundjoin%
\pgfsetlinewidth{1.505625pt}%
\definecolor{currentstroke}{rgb}{0.121569,0.466667,0.705882}%
\pgfsetstrokecolor{currentstroke}%
\pgfsetdash{}{0pt}%
\pgfpathmoveto{\pgfqpoint{0.906944in}{2.029167in}}%
\pgfpathlineto{\pgfqpoint{1.184722in}{2.029167in}}%
\pgfusepath{stroke}%
\end{pgfscope}%
\begin{pgfscope}%
\definecolor{textcolor}{rgb}{0.000000,0.000000,0.000000}%
\pgfsetstrokecolor{textcolor}%
\pgfsetfillcolor{textcolor}%
\pgftext[x=1.295833in,y=1.980556in,left,base]{\color{textcolor}\rmfamily\fontsize{10.000000}{12.000000}\selectfont Mitte}%
\end{pgfscope}%
\begin{pgfscope}%
\pgfsetrectcap%
\pgfsetroundjoin%
\pgfsetlinewidth{1.505625pt}%
\definecolor{currentstroke}{rgb}{1.000000,0.498039,0.054902}%
\pgfsetstrokecolor{currentstroke}%
\pgfsetstrokeopacity{0.800000}%
\pgfsetdash{}{0pt}%
\pgfpathmoveto{\pgfqpoint{0.906944in}{1.835556in}}%
\pgfpathlineto{\pgfqpoint{1.184722in}{1.835556in}}%
\pgfusepath{stroke}%
\end{pgfscope}%
\begin{pgfscope}%
\definecolor{textcolor}{rgb}{0.000000,0.000000,0.000000}%
\pgfsetstrokecolor{textcolor}%
\pgfsetfillcolor{textcolor}%
\pgftext[x=1.295833in,y=1.786945in,left,base]{\color{textcolor}\rmfamily\fontsize{10.000000}{12.000000}\selectfont Rechts}%
\end{pgfscope}%
\begin{pgfscope}%
\pgfsetrectcap%
\pgfsetroundjoin%
\pgfsetlinewidth{1.505625pt}%
\definecolor{currentstroke}{rgb}{0.172549,0.627451,0.172549}%
\pgfsetstrokecolor{currentstroke}%
\pgfsetstrokeopacity{0.800000}%
\pgfsetdash{}{0pt}%
\pgfpathmoveto{\pgfqpoint{0.906944in}{1.641945in}}%
\pgfpathlineto{\pgfqpoint{1.184722in}{1.641945in}}%
\pgfusepath{stroke}%
\end{pgfscope}%
\begin{pgfscope}%
\definecolor{textcolor}{rgb}{0.000000,0.000000,0.000000}%
\pgfsetstrokecolor{textcolor}%
\pgfsetfillcolor{textcolor}%
\pgftext[x=1.295833in,y=1.593334in,left,base]{\color{textcolor}\rmfamily\fontsize{10.000000}{12.000000}\selectfont Links}%
\end{pgfscope}%
\begin{pgfscope}%
\pgfsetbuttcap%
\pgfsetmiterjoin%
\definecolor{currentfill}{rgb}{1.000000,1.000000,1.000000}%
\pgfsetfillcolor{currentfill}%
\pgfsetlinewidth{0.000000pt}%
\definecolor{currentstroke}{rgb}{0.000000,0.000000,0.000000}%
\pgfsetstrokecolor{currentstroke}%
\pgfsetstrokeopacity{0.000000}%
\pgfsetdash{}{0pt}%
\pgfpathmoveto{\pgfqpoint{3.662674in}{0.552778in}}%
\pgfpathlineto{\pgfqpoint{5.801389in}{0.552778in}}%
\pgfpathlineto{\pgfqpoint{5.801389in}{2.202778in}}%
\pgfpathlineto{\pgfqpoint{3.662674in}{2.202778in}}%
\pgfpathclose%
\pgfusepath{fill}%
\end{pgfscope}%
\begin{pgfscope}%
\pgfpathrectangle{\pgfqpoint{3.662674in}{0.552778in}}{\pgfqpoint{2.138715in}{1.650000in}}%
\pgfusepath{clip}%
\pgfsetrectcap%
\pgfsetroundjoin%
\pgfsetlinewidth{0.803000pt}%
\definecolor{currentstroke}{rgb}{0.690196,0.690196,0.690196}%
\pgfsetstrokecolor{currentstroke}%
\pgfsetstrokeopacity{0.800000}%
\pgfsetdash{}{0pt}%
\pgfpathmoveto{\pgfqpoint{3.815439in}{0.552778in}}%
\pgfpathlineto{\pgfqpoint{3.815439in}{2.202778in}}%
\pgfusepath{stroke}%
\end{pgfscope}%
\begin{pgfscope}%
\pgfsetbuttcap%
\pgfsetroundjoin%
\definecolor{currentfill}{rgb}{0.000000,0.000000,0.000000}%
\pgfsetfillcolor{currentfill}%
\pgfsetlinewidth{0.803000pt}%
\definecolor{currentstroke}{rgb}{0.000000,0.000000,0.000000}%
\pgfsetstrokecolor{currentstroke}%
\pgfsetdash{}{0pt}%
\pgfsys@defobject{currentmarker}{\pgfqpoint{0.000000in}{-0.048611in}}{\pgfqpoint{0.000000in}{0.000000in}}{%
\pgfpathmoveto{\pgfqpoint{0.000000in}{0.000000in}}%
\pgfpathlineto{\pgfqpoint{0.000000in}{-0.048611in}}%
\pgfusepath{stroke,fill}%
}%
\begin{pgfscope}%
\pgfsys@transformshift{3.815439in}{0.552778in}%
\pgfsys@useobject{currentmarker}{}%
\end{pgfscope}%
\end{pgfscope}%
\begin{pgfscope}%
\pgfsetbuttcap%
\pgfsetroundjoin%
\definecolor{currentfill}{rgb}{0.000000,0.000000,0.000000}%
\pgfsetfillcolor{currentfill}%
\pgfsetlinewidth{0.803000pt}%
\definecolor{currentstroke}{rgb}{0.000000,0.000000,0.000000}%
\pgfsetstrokecolor{currentstroke}%
\pgfsetdash{}{0pt}%
\pgfsys@defobject{currentmarker}{\pgfqpoint{0.000000in}{0.000000in}}{\pgfqpoint{0.000000in}{0.048611in}}{%
\pgfpathmoveto{\pgfqpoint{0.000000in}{0.000000in}}%
\pgfpathlineto{\pgfqpoint{0.000000in}{0.048611in}}%
\pgfusepath{stroke,fill}%
}%
\begin{pgfscope}%
\pgfsys@transformshift{3.815439in}{2.202778in}%
\pgfsys@useobject{currentmarker}{}%
\end{pgfscope}%
\end{pgfscope}%
\begin{pgfscope}%
\definecolor{textcolor}{rgb}{0.000000,0.000000,0.000000}%
\pgfsetstrokecolor{textcolor}%
\pgfsetfillcolor{textcolor}%
\pgftext[x=3.815439in,y=0.455556in,,top]{\color{textcolor}\rmfamily\fontsize{10.000000}{12.000000}\selectfont 150}%
\end{pgfscope}%
\begin{pgfscope}%
\pgfpathrectangle{\pgfqpoint{3.662674in}{0.552778in}}{\pgfqpoint{2.138715in}{1.650000in}}%
\pgfusepath{clip}%
\pgfsetrectcap%
\pgfsetroundjoin%
\pgfsetlinewidth{0.803000pt}%
\definecolor{currentstroke}{rgb}{0.690196,0.690196,0.690196}%
\pgfsetstrokecolor{currentstroke}%
\pgfsetstrokeopacity{0.800000}%
\pgfsetdash{}{0pt}%
\pgfpathmoveto{\pgfqpoint{4.273735in}{0.552778in}}%
\pgfpathlineto{\pgfqpoint{4.273735in}{2.202778in}}%
\pgfusepath{stroke}%
\end{pgfscope}%
\begin{pgfscope}%
\pgfsetbuttcap%
\pgfsetroundjoin%
\definecolor{currentfill}{rgb}{0.000000,0.000000,0.000000}%
\pgfsetfillcolor{currentfill}%
\pgfsetlinewidth{0.803000pt}%
\definecolor{currentstroke}{rgb}{0.000000,0.000000,0.000000}%
\pgfsetstrokecolor{currentstroke}%
\pgfsetdash{}{0pt}%
\pgfsys@defobject{currentmarker}{\pgfqpoint{0.000000in}{-0.048611in}}{\pgfqpoint{0.000000in}{0.000000in}}{%
\pgfpathmoveto{\pgfqpoint{0.000000in}{0.000000in}}%
\pgfpathlineto{\pgfqpoint{0.000000in}{-0.048611in}}%
\pgfusepath{stroke,fill}%
}%
\begin{pgfscope}%
\pgfsys@transformshift{4.273735in}{0.552778in}%
\pgfsys@useobject{currentmarker}{}%
\end{pgfscope}%
\end{pgfscope}%
\begin{pgfscope}%
\pgfsetbuttcap%
\pgfsetroundjoin%
\definecolor{currentfill}{rgb}{0.000000,0.000000,0.000000}%
\pgfsetfillcolor{currentfill}%
\pgfsetlinewidth{0.803000pt}%
\definecolor{currentstroke}{rgb}{0.000000,0.000000,0.000000}%
\pgfsetstrokecolor{currentstroke}%
\pgfsetdash{}{0pt}%
\pgfsys@defobject{currentmarker}{\pgfqpoint{0.000000in}{0.000000in}}{\pgfqpoint{0.000000in}{0.048611in}}{%
\pgfpathmoveto{\pgfqpoint{0.000000in}{0.000000in}}%
\pgfpathlineto{\pgfqpoint{0.000000in}{0.048611in}}%
\pgfusepath{stroke,fill}%
}%
\begin{pgfscope}%
\pgfsys@transformshift{4.273735in}{2.202778in}%
\pgfsys@useobject{currentmarker}{}%
\end{pgfscope}%
\end{pgfscope}%
\begin{pgfscope}%
\definecolor{textcolor}{rgb}{0.000000,0.000000,0.000000}%
\pgfsetstrokecolor{textcolor}%
\pgfsetfillcolor{textcolor}%
\pgftext[x=4.273735in,y=0.455556in,,top]{\color{textcolor}\rmfamily\fontsize{10.000000}{12.000000}\selectfont 300}%
\end{pgfscope}%
\begin{pgfscope}%
\pgfpathrectangle{\pgfqpoint{3.662674in}{0.552778in}}{\pgfqpoint{2.138715in}{1.650000in}}%
\pgfusepath{clip}%
\pgfsetrectcap%
\pgfsetroundjoin%
\pgfsetlinewidth{0.803000pt}%
\definecolor{currentstroke}{rgb}{0.690196,0.690196,0.690196}%
\pgfsetstrokecolor{currentstroke}%
\pgfsetstrokeopacity{0.800000}%
\pgfsetdash{}{0pt}%
\pgfpathmoveto{\pgfqpoint{4.732031in}{0.552778in}}%
\pgfpathlineto{\pgfqpoint{4.732031in}{2.202778in}}%
\pgfusepath{stroke}%
\end{pgfscope}%
\begin{pgfscope}%
\pgfsetbuttcap%
\pgfsetroundjoin%
\definecolor{currentfill}{rgb}{0.000000,0.000000,0.000000}%
\pgfsetfillcolor{currentfill}%
\pgfsetlinewidth{0.803000pt}%
\definecolor{currentstroke}{rgb}{0.000000,0.000000,0.000000}%
\pgfsetstrokecolor{currentstroke}%
\pgfsetdash{}{0pt}%
\pgfsys@defobject{currentmarker}{\pgfqpoint{0.000000in}{-0.048611in}}{\pgfqpoint{0.000000in}{0.000000in}}{%
\pgfpathmoveto{\pgfqpoint{0.000000in}{0.000000in}}%
\pgfpathlineto{\pgfqpoint{0.000000in}{-0.048611in}}%
\pgfusepath{stroke,fill}%
}%
\begin{pgfscope}%
\pgfsys@transformshift{4.732031in}{0.552778in}%
\pgfsys@useobject{currentmarker}{}%
\end{pgfscope}%
\end{pgfscope}%
\begin{pgfscope}%
\pgfsetbuttcap%
\pgfsetroundjoin%
\definecolor{currentfill}{rgb}{0.000000,0.000000,0.000000}%
\pgfsetfillcolor{currentfill}%
\pgfsetlinewidth{0.803000pt}%
\definecolor{currentstroke}{rgb}{0.000000,0.000000,0.000000}%
\pgfsetstrokecolor{currentstroke}%
\pgfsetdash{}{0pt}%
\pgfsys@defobject{currentmarker}{\pgfqpoint{0.000000in}{0.000000in}}{\pgfqpoint{0.000000in}{0.048611in}}{%
\pgfpathmoveto{\pgfqpoint{0.000000in}{0.000000in}}%
\pgfpathlineto{\pgfqpoint{0.000000in}{0.048611in}}%
\pgfusepath{stroke,fill}%
}%
\begin{pgfscope}%
\pgfsys@transformshift{4.732031in}{2.202778in}%
\pgfsys@useobject{currentmarker}{}%
\end{pgfscope}%
\end{pgfscope}%
\begin{pgfscope}%
\definecolor{textcolor}{rgb}{0.000000,0.000000,0.000000}%
\pgfsetstrokecolor{textcolor}%
\pgfsetfillcolor{textcolor}%
\pgftext[x=4.732031in,y=0.455556in,,top]{\color{textcolor}\rmfamily\fontsize{10.000000}{12.000000}\selectfont 450}%
\end{pgfscope}%
\begin{pgfscope}%
\pgfpathrectangle{\pgfqpoint{3.662674in}{0.552778in}}{\pgfqpoint{2.138715in}{1.650000in}}%
\pgfusepath{clip}%
\pgfsetrectcap%
\pgfsetroundjoin%
\pgfsetlinewidth{0.803000pt}%
\definecolor{currentstroke}{rgb}{0.690196,0.690196,0.690196}%
\pgfsetstrokecolor{currentstroke}%
\pgfsetstrokeopacity{0.800000}%
\pgfsetdash{}{0pt}%
\pgfpathmoveto{\pgfqpoint{5.190327in}{0.552778in}}%
\pgfpathlineto{\pgfqpoint{5.190327in}{2.202778in}}%
\pgfusepath{stroke}%
\end{pgfscope}%
\begin{pgfscope}%
\pgfsetbuttcap%
\pgfsetroundjoin%
\definecolor{currentfill}{rgb}{0.000000,0.000000,0.000000}%
\pgfsetfillcolor{currentfill}%
\pgfsetlinewidth{0.803000pt}%
\definecolor{currentstroke}{rgb}{0.000000,0.000000,0.000000}%
\pgfsetstrokecolor{currentstroke}%
\pgfsetdash{}{0pt}%
\pgfsys@defobject{currentmarker}{\pgfqpoint{0.000000in}{-0.048611in}}{\pgfqpoint{0.000000in}{0.000000in}}{%
\pgfpathmoveto{\pgfqpoint{0.000000in}{0.000000in}}%
\pgfpathlineto{\pgfqpoint{0.000000in}{-0.048611in}}%
\pgfusepath{stroke,fill}%
}%
\begin{pgfscope}%
\pgfsys@transformshift{5.190327in}{0.552778in}%
\pgfsys@useobject{currentmarker}{}%
\end{pgfscope}%
\end{pgfscope}%
\begin{pgfscope}%
\pgfsetbuttcap%
\pgfsetroundjoin%
\definecolor{currentfill}{rgb}{0.000000,0.000000,0.000000}%
\pgfsetfillcolor{currentfill}%
\pgfsetlinewidth{0.803000pt}%
\definecolor{currentstroke}{rgb}{0.000000,0.000000,0.000000}%
\pgfsetstrokecolor{currentstroke}%
\pgfsetdash{}{0pt}%
\pgfsys@defobject{currentmarker}{\pgfqpoint{0.000000in}{0.000000in}}{\pgfqpoint{0.000000in}{0.048611in}}{%
\pgfpathmoveto{\pgfqpoint{0.000000in}{0.000000in}}%
\pgfpathlineto{\pgfqpoint{0.000000in}{0.048611in}}%
\pgfusepath{stroke,fill}%
}%
\begin{pgfscope}%
\pgfsys@transformshift{5.190327in}{2.202778in}%
\pgfsys@useobject{currentmarker}{}%
\end{pgfscope}%
\end{pgfscope}%
\begin{pgfscope}%
\definecolor{textcolor}{rgb}{0.000000,0.000000,0.000000}%
\pgfsetstrokecolor{textcolor}%
\pgfsetfillcolor{textcolor}%
\pgftext[x=5.190327in,y=0.455556in,,top]{\color{textcolor}\rmfamily\fontsize{10.000000}{12.000000}\selectfont 600}%
\end{pgfscope}%
\begin{pgfscope}%
\pgfpathrectangle{\pgfqpoint{3.662674in}{0.552778in}}{\pgfqpoint{2.138715in}{1.650000in}}%
\pgfusepath{clip}%
\pgfsetrectcap%
\pgfsetroundjoin%
\pgfsetlinewidth{0.803000pt}%
\definecolor{currentstroke}{rgb}{0.690196,0.690196,0.690196}%
\pgfsetstrokecolor{currentstroke}%
\pgfsetstrokeopacity{0.800000}%
\pgfsetdash{}{0pt}%
\pgfpathmoveto{\pgfqpoint{5.648624in}{0.552778in}}%
\pgfpathlineto{\pgfqpoint{5.648624in}{2.202778in}}%
\pgfusepath{stroke}%
\end{pgfscope}%
\begin{pgfscope}%
\pgfsetbuttcap%
\pgfsetroundjoin%
\definecolor{currentfill}{rgb}{0.000000,0.000000,0.000000}%
\pgfsetfillcolor{currentfill}%
\pgfsetlinewidth{0.803000pt}%
\definecolor{currentstroke}{rgb}{0.000000,0.000000,0.000000}%
\pgfsetstrokecolor{currentstroke}%
\pgfsetdash{}{0pt}%
\pgfsys@defobject{currentmarker}{\pgfqpoint{0.000000in}{-0.048611in}}{\pgfqpoint{0.000000in}{0.000000in}}{%
\pgfpathmoveto{\pgfqpoint{0.000000in}{0.000000in}}%
\pgfpathlineto{\pgfqpoint{0.000000in}{-0.048611in}}%
\pgfusepath{stroke,fill}%
}%
\begin{pgfscope}%
\pgfsys@transformshift{5.648624in}{0.552778in}%
\pgfsys@useobject{currentmarker}{}%
\end{pgfscope}%
\end{pgfscope}%
\begin{pgfscope}%
\pgfsetbuttcap%
\pgfsetroundjoin%
\definecolor{currentfill}{rgb}{0.000000,0.000000,0.000000}%
\pgfsetfillcolor{currentfill}%
\pgfsetlinewidth{0.803000pt}%
\definecolor{currentstroke}{rgb}{0.000000,0.000000,0.000000}%
\pgfsetstrokecolor{currentstroke}%
\pgfsetdash{}{0pt}%
\pgfsys@defobject{currentmarker}{\pgfqpoint{0.000000in}{0.000000in}}{\pgfqpoint{0.000000in}{0.048611in}}{%
\pgfpathmoveto{\pgfqpoint{0.000000in}{0.000000in}}%
\pgfpathlineto{\pgfqpoint{0.000000in}{0.048611in}}%
\pgfusepath{stroke,fill}%
}%
\begin{pgfscope}%
\pgfsys@transformshift{5.648624in}{2.202778in}%
\pgfsys@useobject{currentmarker}{}%
\end{pgfscope}%
\end{pgfscope}%
\begin{pgfscope}%
\definecolor{textcolor}{rgb}{0.000000,0.000000,0.000000}%
\pgfsetstrokecolor{textcolor}%
\pgfsetfillcolor{textcolor}%
\pgftext[x=5.648624in,y=0.455556in,,top]{\color{textcolor}\rmfamily\fontsize{10.000000}{12.000000}\selectfont 750}%
\end{pgfscope}%
\begin{pgfscope}%
\pgfpathrectangle{\pgfqpoint{3.662674in}{0.552778in}}{\pgfqpoint{2.138715in}{1.650000in}}%
\pgfusepath{clip}%
\pgfsetrectcap%
\pgfsetroundjoin%
\pgfsetlinewidth{0.803000pt}%
\definecolor{currentstroke}{rgb}{0.690196,0.690196,0.690196}%
\pgfsetstrokecolor{currentstroke}%
\pgfsetstrokeopacity{0.300000}%
\pgfsetdash{}{0pt}%
\pgfpathmoveto{\pgfqpoint{3.677950in}{0.552778in}}%
\pgfpathlineto{\pgfqpoint{3.677950in}{2.202778in}}%
\pgfusepath{stroke}%
\end{pgfscope}%
\begin{pgfscope}%
\pgfsetbuttcap%
\pgfsetroundjoin%
\definecolor{currentfill}{rgb}{0.000000,0.000000,0.000000}%
\pgfsetfillcolor{currentfill}%
\pgfsetlinewidth{0.602250pt}%
\definecolor{currentstroke}{rgb}{0.000000,0.000000,0.000000}%
\pgfsetstrokecolor{currentstroke}%
\pgfsetdash{}{0pt}%
\pgfsys@defobject{currentmarker}{\pgfqpoint{0.000000in}{-0.027778in}}{\pgfqpoint{0.000000in}{0.000000in}}{%
\pgfpathmoveto{\pgfqpoint{0.000000in}{0.000000in}}%
\pgfpathlineto{\pgfqpoint{0.000000in}{-0.027778in}}%
\pgfusepath{stroke,fill}%
}%
\begin{pgfscope}%
\pgfsys@transformshift{3.677950in}{0.552778in}%
\pgfsys@useobject{currentmarker}{}%
\end{pgfscope}%
\end{pgfscope}%
\begin{pgfscope}%
\pgfsetbuttcap%
\pgfsetroundjoin%
\definecolor{currentfill}{rgb}{0.000000,0.000000,0.000000}%
\pgfsetfillcolor{currentfill}%
\pgfsetlinewidth{0.602250pt}%
\definecolor{currentstroke}{rgb}{0.000000,0.000000,0.000000}%
\pgfsetstrokecolor{currentstroke}%
\pgfsetdash{}{0pt}%
\pgfsys@defobject{currentmarker}{\pgfqpoint{0.000000in}{0.000000in}}{\pgfqpoint{0.000000in}{0.027778in}}{%
\pgfpathmoveto{\pgfqpoint{0.000000in}{0.000000in}}%
\pgfpathlineto{\pgfqpoint{0.000000in}{0.027778in}}%
\pgfusepath{stroke,fill}%
}%
\begin{pgfscope}%
\pgfsys@transformshift{3.677950in}{2.202778in}%
\pgfsys@useobject{currentmarker}{}%
\end{pgfscope}%
\end{pgfscope}%
\begin{pgfscope}%
\pgfpathrectangle{\pgfqpoint{3.662674in}{0.552778in}}{\pgfqpoint{2.138715in}{1.650000in}}%
\pgfusepath{clip}%
\pgfsetrectcap%
\pgfsetroundjoin%
\pgfsetlinewidth{0.803000pt}%
\definecolor{currentstroke}{rgb}{0.690196,0.690196,0.690196}%
\pgfsetstrokecolor{currentstroke}%
\pgfsetstrokeopacity{0.300000}%
\pgfsetdash{}{0pt}%
\pgfpathmoveto{\pgfqpoint{3.723780in}{0.552778in}}%
\pgfpathlineto{\pgfqpoint{3.723780in}{2.202778in}}%
\pgfusepath{stroke}%
\end{pgfscope}%
\begin{pgfscope}%
\pgfsetbuttcap%
\pgfsetroundjoin%
\definecolor{currentfill}{rgb}{0.000000,0.000000,0.000000}%
\pgfsetfillcolor{currentfill}%
\pgfsetlinewidth{0.602250pt}%
\definecolor{currentstroke}{rgb}{0.000000,0.000000,0.000000}%
\pgfsetstrokecolor{currentstroke}%
\pgfsetdash{}{0pt}%
\pgfsys@defobject{currentmarker}{\pgfqpoint{0.000000in}{-0.027778in}}{\pgfqpoint{0.000000in}{0.000000in}}{%
\pgfpathmoveto{\pgfqpoint{0.000000in}{0.000000in}}%
\pgfpathlineto{\pgfqpoint{0.000000in}{-0.027778in}}%
\pgfusepath{stroke,fill}%
}%
\begin{pgfscope}%
\pgfsys@transformshift{3.723780in}{0.552778in}%
\pgfsys@useobject{currentmarker}{}%
\end{pgfscope}%
\end{pgfscope}%
\begin{pgfscope}%
\pgfsetbuttcap%
\pgfsetroundjoin%
\definecolor{currentfill}{rgb}{0.000000,0.000000,0.000000}%
\pgfsetfillcolor{currentfill}%
\pgfsetlinewidth{0.602250pt}%
\definecolor{currentstroke}{rgb}{0.000000,0.000000,0.000000}%
\pgfsetstrokecolor{currentstroke}%
\pgfsetdash{}{0pt}%
\pgfsys@defobject{currentmarker}{\pgfqpoint{0.000000in}{0.000000in}}{\pgfqpoint{0.000000in}{0.027778in}}{%
\pgfpathmoveto{\pgfqpoint{0.000000in}{0.000000in}}%
\pgfpathlineto{\pgfqpoint{0.000000in}{0.027778in}}%
\pgfusepath{stroke,fill}%
}%
\begin{pgfscope}%
\pgfsys@transformshift{3.723780in}{2.202778in}%
\pgfsys@useobject{currentmarker}{}%
\end{pgfscope}%
\end{pgfscope}%
\begin{pgfscope}%
\pgfpathrectangle{\pgfqpoint{3.662674in}{0.552778in}}{\pgfqpoint{2.138715in}{1.650000in}}%
\pgfusepath{clip}%
\pgfsetrectcap%
\pgfsetroundjoin%
\pgfsetlinewidth{0.803000pt}%
\definecolor{currentstroke}{rgb}{0.690196,0.690196,0.690196}%
\pgfsetstrokecolor{currentstroke}%
\pgfsetstrokeopacity{0.300000}%
\pgfsetdash{}{0pt}%
\pgfpathmoveto{\pgfqpoint{3.769609in}{0.552778in}}%
\pgfpathlineto{\pgfqpoint{3.769609in}{2.202778in}}%
\pgfusepath{stroke}%
\end{pgfscope}%
\begin{pgfscope}%
\pgfsetbuttcap%
\pgfsetroundjoin%
\definecolor{currentfill}{rgb}{0.000000,0.000000,0.000000}%
\pgfsetfillcolor{currentfill}%
\pgfsetlinewidth{0.602250pt}%
\definecolor{currentstroke}{rgb}{0.000000,0.000000,0.000000}%
\pgfsetstrokecolor{currentstroke}%
\pgfsetdash{}{0pt}%
\pgfsys@defobject{currentmarker}{\pgfqpoint{0.000000in}{-0.027778in}}{\pgfqpoint{0.000000in}{0.000000in}}{%
\pgfpathmoveto{\pgfqpoint{0.000000in}{0.000000in}}%
\pgfpathlineto{\pgfqpoint{0.000000in}{-0.027778in}}%
\pgfusepath{stroke,fill}%
}%
\begin{pgfscope}%
\pgfsys@transformshift{3.769609in}{0.552778in}%
\pgfsys@useobject{currentmarker}{}%
\end{pgfscope}%
\end{pgfscope}%
\begin{pgfscope}%
\pgfsetbuttcap%
\pgfsetroundjoin%
\definecolor{currentfill}{rgb}{0.000000,0.000000,0.000000}%
\pgfsetfillcolor{currentfill}%
\pgfsetlinewidth{0.602250pt}%
\definecolor{currentstroke}{rgb}{0.000000,0.000000,0.000000}%
\pgfsetstrokecolor{currentstroke}%
\pgfsetdash{}{0pt}%
\pgfsys@defobject{currentmarker}{\pgfqpoint{0.000000in}{0.000000in}}{\pgfqpoint{0.000000in}{0.027778in}}{%
\pgfpathmoveto{\pgfqpoint{0.000000in}{0.000000in}}%
\pgfpathlineto{\pgfqpoint{0.000000in}{0.027778in}}%
\pgfusepath{stroke,fill}%
}%
\begin{pgfscope}%
\pgfsys@transformshift{3.769609in}{2.202778in}%
\pgfsys@useobject{currentmarker}{}%
\end{pgfscope}%
\end{pgfscope}%
\begin{pgfscope}%
\pgfpathrectangle{\pgfqpoint{3.662674in}{0.552778in}}{\pgfqpoint{2.138715in}{1.650000in}}%
\pgfusepath{clip}%
\pgfsetrectcap%
\pgfsetroundjoin%
\pgfsetlinewidth{0.803000pt}%
\definecolor{currentstroke}{rgb}{0.690196,0.690196,0.690196}%
\pgfsetstrokecolor{currentstroke}%
\pgfsetstrokeopacity{0.300000}%
\pgfsetdash{}{0pt}%
\pgfpathmoveto{\pgfqpoint{3.861269in}{0.552778in}}%
\pgfpathlineto{\pgfqpoint{3.861269in}{2.202778in}}%
\pgfusepath{stroke}%
\end{pgfscope}%
\begin{pgfscope}%
\pgfsetbuttcap%
\pgfsetroundjoin%
\definecolor{currentfill}{rgb}{0.000000,0.000000,0.000000}%
\pgfsetfillcolor{currentfill}%
\pgfsetlinewidth{0.602250pt}%
\definecolor{currentstroke}{rgb}{0.000000,0.000000,0.000000}%
\pgfsetstrokecolor{currentstroke}%
\pgfsetdash{}{0pt}%
\pgfsys@defobject{currentmarker}{\pgfqpoint{0.000000in}{-0.027778in}}{\pgfqpoint{0.000000in}{0.000000in}}{%
\pgfpathmoveto{\pgfqpoint{0.000000in}{0.000000in}}%
\pgfpathlineto{\pgfqpoint{0.000000in}{-0.027778in}}%
\pgfusepath{stroke,fill}%
}%
\begin{pgfscope}%
\pgfsys@transformshift{3.861269in}{0.552778in}%
\pgfsys@useobject{currentmarker}{}%
\end{pgfscope}%
\end{pgfscope}%
\begin{pgfscope}%
\pgfsetbuttcap%
\pgfsetroundjoin%
\definecolor{currentfill}{rgb}{0.000000,0.000000,0.000000}%
\pgfsetfillcolor{currentfill}%
\pgfsetlinewidth{0.602250pt}%
\definecolor{currentstroke}{rgb}{0.000000,0.000000,0.000000}%
\pgfsetstrokecolor{currentstroke}%
\pgfsetdash{}{0pt}%
\pgfsys@defobject{currentmarker}{\pgfqpoint{0.000000in}{0.000000in}}{\pgfqpoint{0.000000in}{0.027778in}}{%
\pgfpathmoveto{\pgfqpoint{0.000000in}{0.000000in}}%
\pgfpathlineto{\pgfqpoint{0.000000in}{0.027778in}}%
\pgfusepath{stroke,fill}%
}%
\begin{pgfscope}%
\pgfsys@transformshift{3.861269in}{2.202778in}%
\pgfsys@useobject{currentmarker}{}%
\end{pgfscope}%
\end{pgfscope}%
\begin{pgfscope}%
\pgfpathrectangle{\pgfqpoint{3.662674in}{0.552778in}}{\pgfqpoint{2.138715in}{1.650000in}}%
\pgfusepath{clip}%
\pgfsetrectcap%
\pgfsetroundjoin%
\pgfsetlinewidth{0.803000pt}%
\definecolor{currentstroke}{rgb}{0.690196,0.690196,0.690196}%
\pgfsetstrokecolor{currentstroke}%
\pgfsetstrokeopacity{0.300000}%
\pgfsetdash{}{0pt}%
\pgfpathmoveto{\pgfqpoint{3.907098in}{0.552778in}}%
\pgfpathlineto{\pgfqpoint{3.907098in}{2.202778in}}%
\pgfusepath{stroke}%
\end{pgfscope}%
\begin{pgfscope}%
\pgfsetbuttcap%
\pgfsetroundjoin%
\definecolor{currentfill}{rgb}{0.000000,0.000000,0.000000}%
\pgfsetfillcolor{currentfill}%
\pgfsetlinewidth{0.602250pt}%
\definecolor{currentstroke}{rgb}{0.000000,0.000000,0.000000}%
\pgfsetstrokecolor{currentstroke}%
\pgfsetdash{}{0pt}%
\pgfsys@defobject{currentmarker}{\pgfqpoint{0.000000in}{-0.027778in}}{\pgfqpoint{0.000000in}{0.000000in}}{%
\pgfpathmoveto{\pgfqpoint{0.000000in}{0.000000in}}%
\pgfpathlineto{\pgfqpoint{0.000000in}{-0.027778in}}%
\pgfusepath{stroke,fill}%
}%
\begin{pgfscope}%
\pgfsys@transformshift{3.907098in}{0.552778in}%
\pgfsys@useobject{currentmarker}{}%
\end{pgfscope}%
\end{pgfscope}%
\begin{pgfscope}%
\pgfsetbuttcap%
\pgfsetroundjoin%
\definecolor{currentfill}{rgb}{0.000000,0.000000,0.000000}%
\pgfsetfillcolor{currentfill}%
\pgfsetlinewidth{0.602250pt}%
\definecolor{currentstroke}{rgb}{0.000000,0.000000,0.000000}%
\pgfsetstrokecolor{currentstroke}%
\pgfsetdash{}{0pt}%
\pgfsys@defobject{currentmarker}{\pgfqpoint{0.000000in}{0.000000in}}{\pgfqpoint{0.000000in}{0.027778in}}{%
\pgfpathmoveto{\pgfqpoint{0.000000in}{0.000000in}}%
\pgfpathlineto{\pgfqpoint{0.000000in}{0.027778in}}%
\pgfusepath{stroke,fill}%
}%
\begin{pgfscope}%
\pgfsys@transformshift{3.907098in}{2.202778in}%
\pgfsys@useobject{currentmarker}{}%
\end{pgfscope}%
\end{pgfscope}%
\begin{pgfscope}%
\pgfpathrectangle{\pgfqpoint{3.662674in}{0.552778in}}{\pgfqpoint{2.138715in}{1.650000in}}%
\pgfusepath{clip}%
\pgfsetrectcap%
\pgfsetroundjoin%
\pgfsetlinewidth{0.803000pt}%
\definecolor{currentstroke}{rgb}{0.690196,0.690196,0.690196}%
\pgfsetstrokecolor{currentstroke}%
\pgfsetstrokeopacity{0.300000}%
\pgfsetdash{}{0pt}%
\pgfpathmoveto{\pgfqpoint{3.952928in}{0.552778in}}%
\pgfpathlineto{\pgfqpoint{3.952928in}{2.202778in}}%
\pgfusepath{stroke}%
\end{pgfscope}%
\begin{pgfscope}%
\pgfsetbuttcap%
\pgfsetroundjoin%
\definecolor{currentfill}{rgb}{0.000000,0.000000,0.000000}%
\pgfsetfillcolor{currentfill}%
\pgfsetlinewidth{0.602250pt}%
\definecolor{currentstroke}{rgb}{0.000000,0.000000,0.000000}%
\pgfsetstrokecolor{currentstroke}%
\pgfsetdash{}{0pt}%
\pgfsys@defobject{currentmarker}{\pgfqpoint{0.000000in}{-0.027778in}}{\pgfqpoint{0.000000in}{0.000000in}}{%
\pgfpathmoveto{\pgfqpoint{0.000000in}{0.000000in}}%
\pgfpathlineto{\pgfqpoint{0.000000in}{-0.027778in}}%
\pgfusepath{stroke,fill}%
}%
\begin{pgfscope}%
\pgfsys@transformshift{3.952928in}{0.552778in}%
\pgfsys@useobject{currentmarker}{}%
\end{pgfscope}%
\end{pgfscope}%
\begin{pgfscope}%
\pgfsetbuttcap%
\pgfsetroundjoin%
\definecolor{currentfill}{rgb}{0.000000,0.000000,0.000000}%
\pgfsetfillcolor{currentfill}%
\pgfsetlinewidth{0.602250pt}%
\definecolor{currentstroke}{rgb}{0.000000,0.000000,0.000000}%
\pgfsetstrokecolor{currentstroke}%
\pgfsetdash{}{0pt}%
\pgfsys@defobject{currentmarker}{\pgfqpoint{0.000000in}{0.000000in}}{\pgfqpoint{0.000000in}{0.027778in}}{%
\pgfpathmoveto{\pgfqpoint{0.000000in}{0.000000in}}%
\pgfpathlineto{\pgfqpoint{0.000000in}{0.027778in}}%
\pgfusepath{stroke,fill}%
}%
\begin{pgfscope}%
\pgfsys@transformshift{3.952928in}{2.202778in}%
\pgfsys@useobject{currentmarker}{}%
\end{pgfscope}%
\end{pgfscope}%
\begin{pgfscope}%
\pgfpathrectangle{\pgfqpoint{3.662674in}{0.552778in}}{\pgfqpoint{2.138715in}{1.650000in}}%
\pgfusepath{clip}%
\pgfsetrectcap%
\pgfsetroundjoin%
\pgfsetlinewidth{0.803000pt}%
\definecolor{currentstroke}{rgb}{0.690196,0.690196,0.690196}%
\pgfsetstrokecolor{currentstroke}%
\pgfsetstrokeopacity{0.300000}%
\pgfsetdash{}{0pt}%
\pgfpathmoveto{\pgfqpoint{3.998757in}{0.552778in}}%
\pgfpathlineto{\pgfqpoint{3.998757in}{2.202778in}}%
\pgfusepath{stroke}%
\end{pgfscope}%
\begin{pgfscope}%
\pgfsetbuttcap%
\pgfsetroundjoin%
\definecolor{currentfill}{rgb}{0.000000,0.000000,0.000000}%
\pgfsetfillcolor{currentfill}%
\pgfsetlinewidth{0.602250pt}%
\definecolor{currentstroke}{rgb}{0.000000,0.000000,0.000000}%
\pgfsetstrokecolor{currentstroke}%
\pgfsetdash{}{0pt}%
\pgfsys@defobject{currentmarker}{\pgfqpoint{0.000000in}{-0.027778in}}{\pgfqpoint{0.000000in}{0.000000in}}{%
\pgfpathmoveto{\pgfqpoint{0.000000in}{0.000000in}}%
\pgfpathlineto{\pgfqpoint{0.000000in}{-0.027778in}}%
\pgfusepath{stroke,fill}%
}%
\begin{pgfscope}%
\pgfsys@transformshift{3.998757in}{0.552778in}%
\pgfsys@useobject{currentmarker}{}%
\end{pgfscope}%
\end{pgfscope}%
\begin{pgfscope}%
\pgfsetbuttcap%
\pgfsetroundjoin%
\definecolor{currentfill}{rgb}{0.000000,0.000000,0.000000}%
\pgfsetfillcolor{currentfill}%
\pgfsetlinewidth{0.602250pt}%
\definecolor{currentstroke}{rgb}{0.000000,0.000000,0.000000}%
\pgfsetstrokecolor{currentstroke}%
\pgfsetdash{}{0pt}%
\pgfsys@defobject{currentmarker}{\pgfqpoint{0.000000in}{0.000000in}}{\pgfqpoint{0.000000in}{0.027778in}}{%
\pgfpathmoveto{\pgfqpoint{0.000000in}{0.000000in}}%
\pgfpathlineto{\pgfqpoint{0.000000in}{0.027778in}}%
\pgfusepath{stroke,fill}%
}%
\begin{pgfscope}%
\pgfsys@transformshift{3.998757in}{2.202778in}%
\pgfsys@useobject{currentmarker}{}%
\end{pgfscope}%
\end{pgfscope}%
\begin{pgfscope}%
\pgfpathrectangle{\pgfqpoint{3.662674in}{0.552778in}}{\pgfqpoint{2.138715in}{1.650000in}}%
\pgfusepath{clip}%
\pgfsetrectcap%
\pgfsetroundjoin%
\pgfsetlinewidth{0.803000pt}%
\definecolor{currentstroke}{rgb}{0.690196,0.690196,0.690196}%
\pgfsetstrokecolor{currentstroke}%
\pgfsetstrokeopacity{0.300000}%
\pgfsetdash{}{0pt}%
\pgfpathmoveto{\pgfqpoint{4.044587in}{0.552778in}}%
\pgfpathlineto{\pgfqpoint{4.044587in}{2.202778in}}%
\pgfusepath{stroke}%
\end{pgfscope}%
\begin{pgfscope}%
\pgfsetbuttcap%
\pgfsetroundjoin%
\definecolor{currentfill}{rgb}{0.000000,0.000000,0.000000}%
\pgfsetfillcolor{currentfill}%
\pgfsetlinewidth{0.602250pt}%
\definecolor{currentstroke}{rgb}{0.000000,0.000000,0.000000}%
\pgfsetstrokecolor{currentstroke}%
\pgfsetdash{}{0pt}%
\pgfsys@defobject{currentmarker}{\pgfqpoint{0.000000in}{-0.027778in}}{\pgfqpoint{0.000000in}{0.000000in}}{%
\pgfpathmoveto{\pgfqpoint{0.000000in}{0.000000in}}%
\pgfpathlineto{\pgfqpoint{0.000000in}{-0.027778in}}%
\pgfusepath{stroke,fill}%
}%
\begin{pgfscope}%
\pgfsys@transformshift{4.044587in}{0.552778in}%
\pgfsys@useobject{currentmarker}{}%
\end{pgfscope}%
\end{pgfscope}%
\begin{pgfscope}%
\pgfsetbuttcap%
\pgfsetroundjoin%
\definecolor{currentfill}{rgb}{0.000000,0.000000,0.000000}%
\pgfsetfillcolor{currentfill}%
\pgfsetlinewidth{0.602250pt}%
\definecolor{currentstroke}{rgb}{0.000000,0.000000,0.000000}%
\pgfsetstrokecolor{currentstroke}%
\pgfsetdash{}{0pt}%
\pgfsys@defobject{currentmarker}{\pgfqpoint{0.000000in}{0.000000in}}{\pgfqpoint{0.000000in}{0.027778in}}{%
\pgfpathmoveto{\pgfqpoint{0.000000in}{0.000000in}}%
\pgfpathlineto{\pgfqpoint{0.000000in}{0.027778in}}%
\pgfusepath{stroke,fill}%
}%
\begin{pgfscope}%
\pgfsys@transformshift{4.044587in}{2.202778in}%
\pgfsys@useobject{currentmarker}{}%
\end{pgfscope}%
\end{pgfscope}%
\begin{pgfscope}%
\pgfpathrectangle{\pgfqpoint{3.662674in}{0.552778in}}{\pgfqpoint{2.138715in}{1.650000in}}%
\pgfusepath{clip}%
\pgfsetrectcap%
\pgfsetroundjoin%
\pgfsetlinewidth{0.803000pt}%
\definecolor{currentstroke}{rgb}{0.690196,0.690196,0.690196}%
\pgfsetstrokecolor{currentstroke}%
\pgfsetstrokeopacity{0.300000}%
\pgfsetdash{}{0pt}%
\pgfpathmoveto{\pgfqpoint{4.090417in}{0.552778in}}%
\pgfpathlineto{\pgfqpoint{4.090417in}{2.202778in}}%
\pgfusepath{stroke}%
\end{pgfscope}%
\begin{pgfscope}%
\pgfsetbuttcap%
\pgfsetroundjoin%
\definecolor{currentfill}{rgb}{0.000000,0.000000,0.000000}%
\pgfsetfillcolor{currentfill}%
\pgfsetlinewidth{0.602250pt}%
\definecolor{currentstroke}{rgb}{0.000000,0.000000,0.000000}%
\pgfsetstrokecolor{currentstroke}%
\pgfsetdash{}{0pt}%
\pgfsys@defobject{currentmarker}{\pgfqpoint{0.000000in}{-0.027778in}}{\pgfqpoint{0.000000in}{0.000000in}}{%
\pgfpathmoveto{\pgfqpoint{0.000000in}{0.000000in}}%
\pgfpathlineto{\pgfqpoint{0.000000in}{-0.027778in}}%
\pgfusepath{stroke,fill}%
}%
\begin{pgfscope}%
\pgfsys@transformshift{4.090417in}{0.552778in}%
\pgfsys@useobject{currentmarker}{}%
\end{pgfscope}%
\end{pgfscope}%
\begin{pgfscope}%
\pgfsetbuttcap%
\pgfsetroundjoin%
\definecolor{currentfill}{rgb}{0.000000,0.000000,0.000000}%
\pgfsetfillcolor{currentfill}%
\pgfsetlinewidth{0.602250pt}%
\definecolor{currentstroke}{rgb}{0.000000,0.000000,0.000000}%
\pgfsetstrokecolor{currentstroke}%
\pgfsetdash{}{0pt}%
\pgfsys@defobject{currentmarker}{\pgfqpoint{0.000000in}{0.000000in}}{\pgfqpoint{0.000000in}{0.027778in}}{%
\pgfpathmoveto{\pgfqpoint{0.000000in}{0.000000in}}%
\pgfpathlineto{\pgfqpoint{0.000000in}{0.027778in}}%
\pgfusepath{stroke,fill}%
}%
\begin{pgfscope}%
\pgfsys@transformshift{4.090417in}{2.202778in}%
\pgfsys@useobject{currentmarker}{}%
\end{pgfscope}%
\end{pgfscope}%
\begin{pgfscope}%
\pgfpathrectangle{\pgfqpoint{3.662674in}{0.552778in}}{\pgfqpoint{2.138715in}{1.650000in}}%
\pgfusepath{clip}%
\pgfsetrectcap%
\pgfsetroundjoin%
\pgfsetlinewidth{0.803000pt}%
\definecolor{currentstroke}{rgb}{0.690196,0.690196,0.690196}%
\pgfsetstrokecolor{currentstroke}%
\pgfsetstrokeopacity{0.300000}%
\pgfsetdash{}{0pt}%
\pgfpathmoveto{\pgfqpoint{4.136246in}{0.552778in}}%
\pgfpathlineto{\pgfqpoint{4.136246in}{2.202778in}}%
\pgfusepath{stroke}%
\end{pgfscope}%
\begin{pgfscope}%
\pgfsetbuttcap%
\pgfsetroundjoin%
\definecolor{currentfill}{rgb}{0.000000,0.000000,0.000000}%
\pgfsetfillcolor{currentfill}%
\pgfsetlinewidth{0.602250pt}%
\definecolor{currentstroke}{rgb}{0.000000,0.000000,0.000000}%
\pgfsetstrokecolor{currentstroke}%
\pgfsetdash{}{0pt}%
\pgfsys@defobject{currentmarker}{\pgfqpoint{0.000000in}{-0.027778in}}{\pgfqpoint{0.000000in}{0.000000in}}{%
\pgfpathmoveto{\pgfqpoint{0.000000in}{0.000000in}}%
\pgfpathlineto{\pgfqpoint{0.000000in}{-0.027778in}}%
\pgfusepath{stroke,fill}%
}%
\begin{pgfscope}%
\pgfsys@transformshift{4.136246in}{0.552778in}%
\pgfsys@useobject{currentmarker}{}%
\end{pgfscope}%
\end{pgfscope}%
\begin{pgfscope}%
\pgfsetbuttcap%
\pgfsetroundjoin%
\definecolor{currentfill}{rgb}{0.000000,0.000000,0.000000}%
\pgfsetfillcolor{currentfill}%
\pgfsetlinewidth{0.602250pt}%
\definecolor{currentstroke}{rgb}{0.000000,0.000000,0.000000}%
\pgfsetstrokecolor{currentstroke}%
\pgfsetdash{}{0pt}%
\pgfsys@defobject{currentmarker}{\pgfqpoint{0.000000in}{0.000000in}}{\pgfqpoint{0.000000in}{0.027778in}}{%
\pgfpathmoveto{\pgfqpoint{0.000000in}{0.000000in}}%
\pgfpathlineto{\pgfqpoint{0.000000in}{0.027778in}}%
\pgfusepath{stroke,fill}%
}%
\begin{pgfscope}%
\pgfsys@transformshift{4.136246in}{2.202778in}%
\pgfsys@useobject{currentmarker}{}%
\end{pgfscope}%
\end{pgfscope}%
\begin{pgfscope}%
\pgfpathrectangle{\pgfqpoint{3.662674in}{0.552778in}}{\pgfqpoint{2.138715in}{1.650000in}}%
\pgfusepath{clip}%
\pgfsetrectcap%
\pgfsetroundjoin%
\pgfsetlinewidth{0.803000pt}%
\definecolor{currentstroke}{rgb}{0.690196,0.690196,0.690196}%
\pgfsetstrokecolor{currentstroke}%
\pgfsetstrokeopacity{0.300000}%
\pgfsetdash{}{0pt}%
\pgfpathmoveto{\pgfqpoint{4.182076in}{0.552778in}}%
\pgfpathlineto{\pgfqpoint{4.182076in}{2.202778in}}%
\pgfusepath{stroke}%
\end{pgfscope}%
\begin{pgfscope}%
\pgfsetbuttcap%
\pgfsetroundjoin%
\definecolor{currentfill}{rgb}{0.000000,0.000000,0.000000}%
\pgfsetfillcolor{currentfill}%
\pgfsetlinewidth{0.602250pt}%
\definecolor{currentstroke}{rgb}{0.000000,0.000000,0.000000}%
\pgfsetstrokecolor{currentstroke}%
\pgfsetdash{}{0pt}%
\pgfsys@defobject{currentmarker}{\pgfqpoint{0.000000in}{-0.027778in}}{\pgfqpoint{0.000000in}{0.000000in}}{%
\pgfpathmoveto{\pgfqpoint{0.000000in}{0.000000in}}%
\pgfpathlineto{\pgfqpoint{0.000000in}{-0.027778in}}%
\pgfusepath{stroke,fill}%
}%
\begin{pgfscope}%
\pgfsys@transformshift{4.182076in}{0.552778in}%
\pgfsys@useobject{currentmarker}{}%
\end{pgfscope}%
\end{pgfscope}%
\begin{pgfscope}%
\pgfsetbuttcap%
\pgfsetroundjoin%
\definecolor{currentfill}{rgb}{0.000000,0.000000,0.000000}%
\pgfsetfillcolor{currentfill}%
\pgfsetlinewidth{0.602250pt}%
\definecolor{currentstroke}{rgb}{0.000000,0.000000,0.000000}%
\pgfsetstrokecolor{currentstroke}%
\pgfsetdash{}{0pt}%
\pgfsys@defobject{currentmarker}{\pgfqpoint{0.000000in}{0.000000in}}{\pgfqpoint{0.000000in}{0.027778in}}{%
\pgfpathmoveto{\pgfqpoint{0.000000in}{0.000000in}}%
\pgfpathlineto{\pgfqpoint{0.000000in}{0.027778in}}%
\pgfusepath{stroke,fill}%
}%
\begin{pgfscope}%
\pgfsys@transformshift{4.182076in}{2.202778in}%
\pgfsys@useobject{currentmarker}{}%
\end{pgfscope}%
\end{pgfscope}%
\begin{pgfscope}%
\pgfpathrectangle{\pgfqpoint{3.662674in}{0.552778in}}{\pgfqpoint{2.138715in}{1.650000in}}%
\pgfusepath{clip}%
\pgfsetrectcap%
\pgfsetroundjoin%
\pgfsetlinewidth{0.803000pt}%
\definecolor{currentstroke}{rgb}{0.690196,0.690196,0.690196}%
\pgfsetstrokecolor{currentstroke}%
\pgfsetstrokeopacity{0.300000}%
\pgfsetdash{}{0pt}%
\pgfpathmoveto{\pgfqpoint{4.227906in}{0.552778in}}%
\pgfpathlineto{\pgfqpoint{4.227906in}{2.202778in}}%
\pgfusepath{stroke}%
\end{pgfscope}%
\begin{pgfscope}%
\pgfsetbuttcap%
\pgfsetroundjoin%
\definecolor{currentfill}{rgb}{0.000000,0.000000,0.000000}%
\pgfsetfillcolor{currentfill}%
\pgfsetlinewidth{0.602250pt}%
\definecolor{currentstroke}{rgb}{0.000000,0.000000,0.000000}%
\pgfsetstrokecolor{currentstroke}%
\pgfsetdash{}{0pt}%
\pgfsys@defobject{currentmarker}{\pgfqpoint{0.000000in}{-0.027778in}}{\pgfqpoint{0.000000in}{0.000000in}}{%
\pgfpathmoveto{\pgfqpoint{0.000000in}{0.000000in}}%
\pgfpathlineto{\pgfqpoint{0.000000in}{-0.027778in}}%
\pgfusepath{stroke,fill}%
}%
\begin{pgfscope}%
\pgfsys@transformshift{4.227906in}{0.552778in}%
\pgfsys@useobject{currentmarker}{}%
\end{pgfscope}%
\end{pgfscope}%
\begin{pgfscope}%
\pgfsetbuttcap%
\pgfsetroundjoin%
\definecolor{currentfill}{rgb}{0.000000,0.000000,0.000000}%
\pgfsetfillcolor{currentfill}%
\pgfsetlinewidth{0.602250pt}%
\definecolor{currentstroke}{rgb}{0.000000,0.000000,0.000000}%
\pgfsetstrokecolor{currentstroke}%
\pgfsetdash{}{0pt}%
\pgfsys@defobject{currentmarker}{\pgfqpoint{0.000000in}{0.000000in}}{\pgfqpoint{0.000000in}{0.027778in}}{%
\pgfpathmoveto{\pgfqpoint{0.000000in}{0.000000in}}%
\pgfpathlineto{\pgfqpoint{0.000000in}{0.027778in}}%
\pgfusepath{stroke,fill}%
}%
\begin{pgfscope}%
\pgfsys@transformshift{4.227906in}{2.202778in}%
\pgfsys@useobject{currentmarker}{}%
\end{pgfscope}%
\end{pgfscope}%
\begin{pgfscope}%
\pgfpathrectangle{\pgfqpoint{3.662674in}{0.552778in}}{\pgfqpoint{2.138715in}{1.650000in}}%
\pgfusepath{clip}%
\pgfsetrectcap%
\pgfsetroundjoin%
\pgfsetlinewidth{0.803000pt}%
\definecolor{currentstroke}{rgb}{0.690196,0.690196,0.690196}%
\pgfsetstrokecolor{currentstroke}%
\pgfsetstrokeopacity{0.300000}%
\pgfsetdash{}{0pt}%
\pgfpathmoveto{\pgfqpoint{4.319565in}{0.552778in}}%
\pgfpathlineto{\pgfqpoint{4.319565in}{2.202778in}}%
\pgfusepath{stroke}%
\end{pgfscope}%
\begin{pgfscope}%
\pgfsetbuttcap%
\pgfsetroundjoin%
\definecolor{currentfill}{rgb}{0.000000,0.000000,0.000000}%
\pgfsetfillcolor{currentfill}%
\pgfsetlinewidth{0.602250pt}%
\definecolor{currentstroke}{rgb}{0.000000,0.000000,0.000000}%
\pgfsetstrokecolor{currentstroke}%
\pgfsetdash{}{0pt}%
\pgfsys@defobject{currentmarker}{\pgfqpoint{0.000000in}{-0.027778in}}{\pgfqpoint{0.000000in}{0.000000in}}{%
\pgfpathmoveto{\pgfqpoint{0.000000in}{0.000000in}}%
\pgfpathlineto{\pgfqpoint{0.000000in}{-0.027778in}}%
\pgfusepath{stroke,fill}%
}%
\begin{pgfscope}%
\pgfsys@transformshift{4.319565in}{0.552778in}%
\pgfsys@useobject{currentmarker}{}%
\end{pgfscope}%
\end{pgfscope}%
\begin{pgfscope}%
\pgfsetbuttcap%
\pgfsetroundjoin%
\definecolor{currentfill}{rgb}{0.000000,0.000000,0.000000}%
\pgfsetfillcolor{currentfill}%
\pgfsetlinewidth{0.602250pt}%
\definecolor{currentstroke}{rgb}{0.000000,0.000000,0.000000}%
\pgfsetstrokecolor{currentstroke}%
\pgfsetdash{}{0pt}%
\pgfsys@defobject{currentmarker}{\pgfqpoint{0.000000in}{0.000000in}}{\pgfqpoint{0.000000in}{0.027778in}}{%
\pgfpathmoveto{\pgfqpoint{0.000000in}{0.000000in}}%
\pgfpathlineto{\pgfqpoint{0.000000in}{0.027778in}}%
\pgfusepath{stroke,fill}%
}%
\begin{pgfscope}%
\pgfsys@transformshift{4.319565in}{2.202778in}%
\pgfsys@useobject{currentmarker}{}%
\end{pgfscope}%
\end{pgfscope}%
\begin{pgfscope}%
\pgfpathrectangle{\pgfqpoint{3.662674in}{0.552778in}}{\pgfqpoint{2.138715in}{1.650000in}}%
\pgfusepath{clip}%
\pgfsetrectcap%
\pgfsetroundjoin%
\pgfsetlinewidth{0.803000pt}%
\definecolor{currentstroke}{rgb}{0.690196,0.690196,0.690196}%
\pgfsetstrokecolor{currentstroke}%
\pgfsetstrokeopacity{0.300000}%
\pgfsetdash{}{0pt}%
\pgfpathmoveto{\pgfqpoint{4.365394in}{0.552778in}}%
\pgfpathlineto{\pgfqpoint{4.365394in}{2.202778in}}%
\pgfusepath{stroke}%
\end{pgfscope}%
\begin{pgfscope}%
\pgfsetbuttcap%
\pgfsetroundjoin%
\definecolor{currentfill}{rgb}{0.000000,0.000000,0.000000}%
\pgfsetfillcolor{currentfill}%
\pgfsetlinewidth{0.602250pt}%
\definecolor{currentstroke}{rgb}{0.000000,0.000000,0.000000}%
\pgfsetstrokecolor{currentstroke}%
\pgfsetdash{}{0pt}%
\pgfsys@defobject{currentmarker}{\pgfqpoint{0.000000in}{-0.027778in}}{\pgfqpoint{0.000000in}{0.000000in}}{%
\pgfpathmoveto{\pgfqpoint{0.000000in}{0.000000in}}%
\pgfpathlineto{\pgfqpoint{0.000000in}{-0.027778in}}%
\pgfusepath{stroke,fill}%
}%
\begin{pgfscope}%
\pgfsys@transformshift{4.365394in}{0.552778in}%
\pgfsys@useobject{currentmarker}{}%
\end{pgfscope}%
\end{pgfscope}%
\begin{pgfscope}%
\pgfsetbuttcap%
\pgfsetroundjoin%
\definecolor{currentfill}{rgb}{0.000000,0.000000,0.000000}%
\pgfsetfillcolor{currentfill}%
\pgfsetlinewidth{0.602250pt}%
\definecolor{currentstroke}{rgb}{0.000000,0.000000,0.000000}%
\pgfsetstrokecolor{currentstroke}%
\pgfsetdash{}{0pt}%
\pgfsys@defobject{currentmarker}{\pgfqpoint{0.000000in}{0.000000in}}{\pgfqpoint{0.000000in}{0.027778in}}{%
\pgfpathmoveto{\pgfqpoint{0.000000in}{0.000000in}}%
\pgfpathlineto{\pgfqpoint{0.000000in}{0.027778in}}%
\pgfusepath{stroke,fill}%
}%
\begin{pgfscope}%
\pgfsys@transformshift{4.365394in}{2.202778in}%
\pgfsys@useobject{currentmarker}{}%
\end{pgfscope}%
\end{pgfscope}%
\begin{pgfscope}%
\pgfpathrectangle{\pgfqpoint{3.662674in}{0.552778in}}{\pgfqpoint{2.138715in}{1.650000in}}%
\pgfusepath{clip}%
\pgfsetrectcap%
\pgfsetroundjoin%
\pgfsetlinewidth{0.803000pt}%
\definecolor{currentstroke}{rgb}{0.690196,0.690196,0.690196}%
\pgfsetstrokecolor{currentstroke}%
\pgfsetstrokeopacity{0.300000}%
\pgfsetdash{}{0pt}%
\pgfpathmoveto{\pgfqpoint{4.411224in}{0.552778in}}%
\pgfpathlineto{\pgfqpoint{4.411224in}{2.202778in}}%
\pgfusepath{stroke}%
\end{pgfscope}%
\begin{pgfscope}%
\pgfsetbuttcap%
\pgfsetroundjoin%
\definecolor{currentfill}{rgb}{0.000000,0.000000,0.000000}%
\pgfsetfillcolor{currentfill}%
\pgfsetlinewidth{0.602250pt}%
\definecolor{currentstroke}{rgb}{0.000000,0.000000,0.000000}%
\pgfsetstrokecolor{currentstroke}%
\pgfsetdash{}{0pt}%
\pgfsys@defobject{currentmarker}{\pgfqpoint{0.000000in}{-0.027778in}}{\pgfqpoint{0.000000in}{0.000000in}}{%
\pgfpathmoveto{\pgfqpoint{0.000000in}{0.000000in}}%
\pgfpathlineto{\pgfqpoint{0.000000in}{-0.027778in}}%
\pgfusepath{stroke,fill}%
}%
\begin{pgfscope}%
\pgfsys@transformshift{4.411224in}{0.552778in}%
\pgfsys@useobject{currentmarker}{}%
\end{pgfscope}%
\end{pgfscope}%
\begin{pgfscope}%
\pgfsetbuttcap%
\pgfsetroundjoin%
\definecolor{currentfill}{rgb}{0.000000,0.000000,0.000000}%
\pgfsetfillcolor{currentfill}%
\pgfsetlinewidth{0.602250pt}%
\definecolor{currentstroke}{rgb}{0.000000,0.000000,0.000000}%
\pgfsetstrokecolor{currentstroke}%
\pgfsetdash{}{0pt}%
\pgfsys@defobject{currentmarker}{\pgfqpoint{0.000000in}{0.000000in}}{\pgfqpoint{0.000000in}{0.027778in}}{%
\pgfpathmoveto{\pgfqpoint{0.000000in}{0.000000in}}%
\pgfpathlineto{\pgfqpoint{0.000000in}{0.027778in}}%
\pgfusepath{stroke,fill}%
}%
\begin{pgfscope}%
\pgfsys@transformshift{4.411224in}{2.202778in}%
\pgfsys@useobject{currentmarker}{}%
\end{pgfscope}%
\end{pgfscope}%
\begin{pgfscope}%
\pgfpathrectangle{\pgfqpoint{3.662674in}{0.552778in}}{\pgfqpoint{2.138715in}{1.650000in}}%
\pgfusepath{clip}%
\pgfsetrectcap%
\pgfsetroundjoin%
\pgfsetlinewidth{0.803000pt}%
\definecolor{currentstroke}{rgb}{0.690196,0.690196,0.690196}%
\pgfsetstrokecolor{currentstroke}%
\pgfsetstrokeopacity{0.300000}%
\pgfsetdash{}{0pt}%
\pgfpathmoveto{\pgfqpoint{4.457054in}{0.552778in}}%
\pgfpathlineto{\pgfqpoint{4.457054in}{2.202778in}}%
\pgfusepath{stroke}%
\end{pgfscope}%
\begin{pgfscope}%
\pgfsetbuttcap%
\pgfsetroundjoin%
\definecolor{currentfill}{rgb}{0.000000,0.000000,0.000000}%
\pgfsetfillcolor{currentfill}%
\pgfsetlinewidth{0.602250pt}%
\definecolor{currentstroke}{rgb}{0.000000,0.000000,0.000000}%
\pgfsetstrokecolor{currentstroke}%
\pgfsetdash{}{0pt}%
\pgfsys@defobject{currentmarker}{\pgfqpoint{0.000000in}{-0.027778in}}{\pgfqpoint{0.000000in}{0.000000in}}{%
\pgfpathmoveto{\pgfqpoint{0.000000in}{0.000000in}}%
\pgfpathlineto{\pgfqpoint{0.000000in}{-0.027778in}}%
\pgfusepath{stroke,fill}%
}%
\begin{pgfscope}%
\pgfsys@transformshift{4.457054in}{0.552778in}%
\pgfsys@useobject{currentmarker}{}%
\end{pgfscope}%
\end{pgfscope}%
\begin{pgfscope}%
\pgfsetbuttcap%
\pgfsetroundjoin%
\definecolor{currentfill}{rgb}{0.000000,0.000000,0.000000}%
\pgfsetfillcolor{currentfill}%
\pgfsetlinewidth{0.602250pt}%
\definecolor{currentstroke}{rgb}{0.000000,0.000000,0.000000}%
\pgfsetstrokecolor{currentstroke}%
\pgfsetdash{}{0pt}%
\pgfsys@defobject{currentmarker}{\pgfqpoint{0.000000in}{0.000000in}}{\pgfqpoint{0.000000in}{0.027778in}}{%
\pgfpathmoveto{\pgfqpoint{0.000000in}{0.000000in}}%
\pgfpathlineto{\pgfqpoint{0.000000in}{0.027778in}}%
\pgfusepath{stroke,fill}%
}%
\begin{pgfscope}%
\pgfsys@transformshift{4.457054in}{2.202778in}%
\pgfsys@useobject{currentmarker}{}%
\end{pgfscope}%
\end{pgfscope}%
\begin{pgfscope}%
\pgfpathrectangle{\pgfqpoint{3.662674in}{0.552778in}}{\pgfqpoint{2.138715in}{1.650000in}}%
\pgfusepath{clip}%
\pgfsetrectcap%
\pgfsetroundjoin%
\pgfsetlinewidth{0.803000pt}%
\definecolor{currentstroke}{rgb}{0.690196,0.690196,0.690196}%
\pgfsetstrokecolor{currentstroke}%
\pgfsetstrokeopacity{0.300000}%
\pgfsetdash{}{0pt}%
\pgfpathmoveto{\pgfqpoint{4.502883in}{0.552778in}}%
\pgfpathlineto{\pgfqpoint{4.502883in}{2.202778in}}%
\pgfusepath{stroke}%
\end{pgfscope}%
\begin{pgfscope}%
\pgfsetbuttcap%
\pgfsetroundjoin%
\definecolor{currentfill}{rgb}{0.000000,0.000000,0.000000}%
\pgfsetfillcolor{currentfill}%
\pgfsetlinewidth{0.602250pt}%
\definecolor{currentstroke}{rgb}{0.000000,0.000000,0.000000}%
\pgfsetstrokecolor{currentstroke}%
\pgfsetdash{}{0pt}%
\pgfsys@defobject{currentmarker}{\pgfqpoint{0.000000in}{-0.027778in}}{\pgfqpoint{0.000000in}{0.000000in}}{%
\pgfpathmoveto{\pgfqpoint{0.000000in}{0.000000in}}%
\pgfpathlineto{\pgfqpoint{0.000000in}{-0.027778in}}%
\pgfusepath{stroke,fill}%
}%
\begin{pgfscope}%
\pgfsys@transformshift{4.502883in}{0.552778in}%
\pgfsys@useobject{currentmarker}{}%
\end{pgfscope}%
\end{pgfscope}%
\begin{pgfscope}%
\pgfsetbuttcap%
\pgfsetroundjoin%
\definecolor{currentfill}{rgb}{0.000000,0.000000,0.000000}%
\pgfsetfillcolor{currentfill}%
\pgfsetlinewidth{0.602250pt}%
\definecolor{currentstroke}{rgb}{0.000000,0.000000,0.000000}%
\pgfsetstrokecolor{currentstroke}%
\pgfsetdash{}{0pt}%
\pgfsys@defobject{currentmarker}{\pgfqpoint{0.000000in}{0.000000in}}{\pgfqpoint{0.000000in}{0.027778in}}{%
\pgfpathmoveto{\pgfqpoint{0.000000in}{0.000000in}}%
\pgfpathlineto{\pgfqpoint{0.000000in}{0.027778in}}%
\pgfusepath{stroke,fill}%
}%
\begin{pgfscope}%
\pgfsys@transformshift{4.502883in}{2.202778in}%
\pgfsys@useobject{currentmarker}{}%
\end{pgfscope}%
\end{pgfscope}%
\begin{pgfscope}%
\pgfpathrectangle{\pgfqpoint{3.662674in}{0.552778in}}{\pgfqpoint{2.138715in}{1.650000in}}%
\pgfusepath{clip}%
\pgfsetrectcap%
\pgfsetroundjoin%
\pgfsetlinewidth{0.803000pt}%
\definecolor{currentstroke}{rgb}{0.690196,0.690196,0.690196}%
\pgfsetstrokecolor{currentstroke}%
\pgfsetstrokeopacity{0.300000}%
\pgfsetdash{}{0pt}%
\pgfpathmoveto{\pgfqpoint{4.548713in}{0.552778in}}%
\pgfpathlineto{\pgfqpoint{4.548713in}{2.202778in}}%
\pgfusepath{stroke}%
\end{pgfscope}%
\begin{pgfscope}%
\pgfsetbuttcap%
\pgfsetroundjoin%
\definecolor{currentfill}{rgb}{0.000000,0.000000,0.000000}%
\pgfsetfillcolor{currentfill}%
\pgfsetlinewidth{0.602250pt}%
\definecolor{currentstroke}{rgb}{0.000000,0.000000,0.000000}%
\pgfsetstrokecolor{currentstroke}%
\pgfsetdash{}{0pt}%
\pgfsys@defobject{currentmarker}{\pgfqpoint{0.000000in}{-0.027778in}}{\pgfqpoint{0.000000in}{0.000000in}}{%
\pgfpathmoveto{\pgfqpoint{0.000000in}{0.000000in}}%
\pgfpathlineto{\pgfqpoint{0.000000in}{-0.027778in}}%
\pgfusepath{stroke,fill}%
}%
\begin{pgfscope}%
\pgfsys@transformshift{4.548713in}{0.552778in}%
\pgfsys@useobject{currentmarker}{}%
\end{pgfscope}%
\end{pgfscope}%
\begin{pgfscope}%
\pgfsetbuttcap%
\pgfsetroundjoin%
\definecolor{currentfill}{rgb}{0.000000,0.000000,0.000000}%
\pgfsetfillcolor{currentfill}%
\pgfsetlinewidth{0.602250pt}%
\definecolor{currentstroke}{rgb}{0.000000,0.000000,0.000000}%
\pgfsetstrokecolor{currentstroke}%
\pgfsetdash{}{0pt}%
\pgfsys@defobject{currentmarker}{\pgfqpoint{0.000000in}{0.000000in}}{\pgfqpoint{0.000000in}{0.027778in}}{%
\pgfpathmoveto{\pgfqpoint{0.000000in}{0.000000in}}%
\pgfpathlineto{\pgfqpoint{0.000000in}{0.027778in}}%
\pgfusepath{stroke,fill}%
}%
\begin{pgfscope}%
\pgfsys@transformshift{4.548713in}{2.202778in}%
\pgfsys@useobject{currentmarker}{}%
\end{pgfscope}%
\end{pgfscope}%
\begin{pgfscope}%
\pgfpathrectangle{\pgfqpoint{3.662674in}{0.552778in}}{\pgfqpoint{2.138715in}{1.650000in}}%
\pgfusepath{clip}%
\pgfsetrectcap%
\pgfsetroundjoin%
\pgfsetlinewidth{0.803000pt}%
\definecolor{currentstroke}{rgb}{0.690196,0.690196,0.690196}%
\pgfsetstrokecolor{currentstroke}%
\pgfsetstrokeopacity{0.300000}%
\pgfsetdash{}{0pt}%
\pgfpathmoveto{\pgfqpoint{4.594542in}{0.552778in}}%
\pgfpathlineto{\pgfqpoint{4.594542in}{2.202778in}}%
\pgfusepath{stroke}%
\end{pgfscope}%
\begin{pgfscope}%
\pgfsetbuttcap%
\pgfsetroundjoin%
\definecolor{currentfill}{rgb}{0.000000,0.000000,0.000000}%
\pgfsetfillcolor{currentfill}%
\pgfsetlinewidth{0.602250pt}%
\definecolor{currentstroke}{rgb}{0.000000,0.000000,0.000000}%
\pgfsetstrokecolor{currentstroke}%
\pgfsetdash{}{0pt}%
\pgfsys@defobject{currentmarker}{\pgfqpoint{0.000000in}{-0.027778in}}{\pgfqpoint{0.000000in}{0.000000in}}{%
\pgfpathmoveto{\pgfqpoint{0.000000in}{0.000000in}}%
\pgfpathlineto{\pgfqpoint{0.000000in}{-0.027778in}}%
\pgfusepath{stroke,fill}%
}%
\begin{pgfscope}%
\pgfsys@transformshift{4.594542in}{0.552778in}%
\pgfsys@useobject{currentmarker}{}%
\end{pgfscope}%
\end{pgfscope}%
\begin{pgfscope}%
\pgfsetbuttcap%
\pgfsetroundjoin%
\definecolor{currentfill}{rgb}{0.000000,0.000000,0.000000}%
\pgfsetfillcolor{currentfill}%
\pgfsetlinewidth{0.602250pt}%
\definecolor{currentstroke}{rgb}{0.000000,0.000000,0.000000}%
\pgfsetstrokecolor{currentstroke}%
\pgfsetdash{}{0pt}%
\pgfsys@defobject{currentmarker}{\pgfqpoint{0.000000in}{0.000000in}}{\pgfqpoint{0.000000in}{0.027778in}}{%
\pgfpathmoveto{\pgfqpoint{0.000000in}{0.000000in}}%
\pgfpathlineto{\pgfqpoint{0.000000in}{0.027778in}}%
\pgfusepath{stroke,fill}%
}%
\begin{pgfscope}%
\pgfsys@transformshift{4.594542in}{2.202778in}%
\pgfsys@useobject{currentmarker}{}%
\end{pgfscope}%
\end{pgfscope}%
\begin{pgfscope}%
\pgfpathrectangle{\pgfqpoint{3.662674in}{0.552778in}}{\pgfqpoint{2.138715in}{1.650000in}}%
\pgfusepath{clip}%
\pgfsetrectcap%
\pgfsetroundjoin%
\pgfsetlinewidth{0.803000pt}%
\definecolor{currentstroke}{rgb}{0.690196,0.690196,0.690196}%
\pgfsetstrokecolor{currentstroke}%
\pgfsetstrokeopacity{0.300000}%
\pgfsetdash{}{0pt}%
\pgfpathmoveto{\pgfqpoint{4.640372in}{0.552778in}}%
\pgfpathlineto{\pgfqpoint{4.640372in}{2.202778in}}%
\pgfusepath{stroke}%
\end{pgfscope}%
\begin{pgfscope}%
\pgfsetbuttcap%
\pgfsetroundjoin%
\definecolor{currentfill}{rgb}{0.000000,0.000000,0.000000}%
\pgfsetfillcolor{currentfill}%
\pgfsetlinewidth{0.602250pt}%
\definecolor{currentstroke}{rgb}{0.000000,0.000000,0.000000}%
\pgfsetstrokecolor{currentstroke}%
\pgfsetdash{}{0pt}%
\pgfsys@defobject{currentmarker}{\pgfqpoint{0.000000in}{-0.027778in}}{\pgfqpoint{0.000000in}{0.000000in}}{%
\pgfpathmoveto{\pgfqpoint{0.000000in}{0.000000in}}%
\pgfpathlineto{\pgfqpoint{0.000000in}{-0.027778in}}%
\pgfusepath{stroke,fill}%
}%
\begin{pgfscope}%
\pgfsys@transformshift{4.640372in}{0.552778in}%
\pgfsys@useobject{currentmarker}{}%
\end{pgfscope}%
\end{pgfscope}%
\begin{pgfscope}%
\pgfsetbuttcap%
\pgfsetroundjoin%
\definecolor{currentfill}{rgb}{0.000000,0.000000,0.000000}%
\pgfsetfillcolor{currentfill}%
\pgfsetlinewidth{0.602250pt}%
\definecolor{currentstroke}{rgb}{0.000000,0.000000,0.000000}%
\pgfsetstrokecolor{currentstroke}%
\pgfsetdash{}{0pt}%
\pgfsys@defobject{currentmarker}{\pgfqpoint{0.000000in}{0.000000in}}{\pgfqpoint{0.000000in}{0.027778in}}{%
\pgfpathmoveto{\pgfqpoint{0.000000in}{0.000000in}}%
\pgfpathlineto{\pgfqpoint{0.000000in}{0.027778in}}%
\pgfusepath{stroke,fill}%
}%
\begin{pgfscope}%
\pgfsys@transformshift{4.640372in}{2.202778in}%
\pgfsys@useobject{currentmarker}{}%
\end{pgfscope}%
\end{pgfscope}%
\begin{pgfscope}%
\pgfpathrectangle{\pgfqpoint{3.662674in}{0.552778in}}{\pgfqpoint{2.138715in}{1.650000in}}%
\pgfusepath{clip}%
\pgfsetrectcap%
\pgfsetroundjoin%
\pgfsetlinewidth{0.803000pt}%
\definecolor{currentstroke}{rgb}{0.690196,0.690196,0.690196}%
\pgfsetstrokecolor{currentstroke}%
\pgfsetstrokeopacity{0.300000}%
\pgfsetdash{}{0pt}%
\pgfpathmoveto{\pgfqpoint{4.686202in}{0.552778in}}%
\pgfpathlineto{\pgfqpoint{4.686202in}{2.202778in}}%
\pgfusepath{stroke}%
\end{pgfscope}%
\begin{pgfscope}%
\pgfsetbuttcap%
\pgfsetroundjoin%
\definecolor{currentfill}{rgb}{0.000000,0.000000,0.000000}%
\pgfsetfillcolor{currentfill}%
\pgfsetlinewidth{0.602250pt}%
\definecolor{currentstroke}{rgb}{0.000000,0.000000,0.000000}%
\pgfsetstrokecolor{currentstroke}%
\pgfsetdash{}{0pt}%
\pgfsys@defobject{currentmarker}{\pgfqpoint{0.000000in}{-0.027778in}}{\pgfqpoint{0.000000in}{0.000000in}}{%
\pgfpathmoveto{\pgfqpoint{0.000000in}{0.000000in}}%
\pgfpathlineto{\pgfqpoint{0.000000in}{-0.027778in}}%
\pgfusepath{stroke,fill}%
}%
\begin{pgfscope}%
\pgfsys@transformshift{4.686202in}{0.552778in}%
\pgfsys@useobject{currentmarker}{}%
\end{pgfscope}%
\end{pgfscope}%
\begin{pgfscope}%
\pgfsetbuttcap%
\pgfsetroundjoin%
\definecolor{currentfill}{rgb}{0.000000,0.000000,0.000000}%
\pgfsetfillcolor{currentfill}%
\pgfsetlinewidth{0.602250pt}%
\definecolor{currentstroke}{rgb}{0.000000,0.000000,0.000000}%
\pgfsetstrokecolor{currentstroke}%
\pgfsetdash{}{0pt}%
\pgfsys@defobject{currentmarker}{\pgfqpoint{0.000000in}{0.000000in}}{\pgfqpoint{0.000000in}{0.027778in}}{%
\pgfpathmoveto{\pgfqpoint{0.000000in}{0.000000in}}%
\pgfpathlineto{\pgfqpoint{0.000000in}{0.027778in}}%
\pgfusepath{stroke,fill}%
}%
\begin{pgfscope}%
\pgfsys@transformshift{4.686202in}{2.202778in}%
\pgfsys@useobject{currentmarker}{}%
\end{pgfscope}%
\end{pgfscope}%
\begin{pgfscope}%
\pgfpathrectangle{\pgfqpoint{3.662674in}{0.552778in}}{\pgfqpoint{2.138715in}{1.650000in}}%
\pgfusepath{clip}%
\pgfsetrectcap%
\pgfsetroundjoin%
\pgfsetlinewidth{0.803000pt}%
\definecolor{currentstroke}{rgb}{0.690196,0.690196,0.690196}%
\pgfsetstrokecolor{currentstroke}%
\pgfsetstrokeopacity{0.300000}%
\pgfsetdash{}{0pt}%
\pgfpathmoveto{\pgfqpoint{4.777861in}{0.552778in}}%
\pgfpathlineto{\pgfqpoint{4.777861in}{2.202778in}}%
\pgfusepath{stroke}%
\end{pgfscope}%
\begin{pgfscope}%
\pgfsetbuttcap%
\pgfsetroundjoin%
\definecolor{currentfill}{rgb}{0.000000,0.000000,0.000000}%
\pgfsetfillcolor{currentfill}%
\pgfsetlinewidth{0.602250pt}%
\definecolor{currentstroke}{rgb}{0.000000,0.000000,0.000000}%
\pgfsetstrokecolor{currentstroke}%
\pgfsetdash{}{0pt}%
\pgfsys@defobject{currentmarker}{\pgfqpoint{0.000000in}{-0.027778in}}{\pgfqpoint{0.000000in}{0.000000in}}{%
\pgfpathmoveto{\pgfqpoint{0.000000in}{0.000000in}}%
\pgfpathlineto{\pgfqpoint{0.000000in}{-0.027778in}}%
\pgfusepath{stroke,fill}%
}%
\begin{pgfscope}%
\pgfsys@transformshift{4.777861in}{0.552778in}%
\pgfsys@useobject{currentmarker}{}%
\end{pgfscope}%
\end{pgfscope}%
\begin{pgfscope}%
\pgfsetbuttcap%
\pgfsetroundjoin%
\definecolor{currentfill}{rgb}{0.000000,0.000000,0.000000}%
\pgfsetfillcolor{currentfill}%
\pgfsetlinewidth{0.602250pt}%
\definecolor{currentstroke}{rgb}{0.000000,0.000000,0.000000}%
\pgfsetstrokecolor{currentstroke}%
\pgfsetdash{}{0pt}%
\pgfsys@defobject{currentmarker}{\pgfqpoint{0.000000in}{0.000000in}}{\pgfqpoint{0.000000in}{0.027778in}}{%
\pgfpathmoveto{\pgfqpoint{0.000000in}{0.000000in}}%
\pgfpathlineto{\pgfqpoint{0.000000in}{0.027778in}}%
\pgfusepath{stroke,fill}%
}%
\begin{pgfscope}%
\pgfsys@transformshift{4.777861in}{2.202778in}%
\pgfsys@useobject{currentmarker}{}%
\end{pgfscope}%
\end{pgfscope}%
\begin{pgfscope}%
\pgfpathrectangle{\pgfqpoint{3.662674in}{0.552778in}}{\pgfqpoint{2.138715in}{1.650000in}}%
\pgfusepath{clip}%
\pgfsetrectcap%
\pgfsetroundjoin%
\pgfsetlinewidth{0.803000pt}%
\definecolor{currentstroke}{rgb}{0.690196,0.690196,0.690196}%
\pgfsetstrokecolor{currentstroke}%
\pgfsetstrokeopacity{0.300000}%
\pgfsetdash{}{0pt}%
\pgfpathmoveto{\pgfqpoint{4.823690in}{0.552778in}}%
\pgfpathlineto{\pgfqpoint{4.823690in}{2.202778in}}%
\pgfusepath{stroke}%
\end{pgfscope}%
\begin{pgfscope}%
\pgfsetbuttcap%
\pgfsetroundjoin%
\definecolor{currentfill}{rgb}{0.000000,0.000000,0.000000}%
\pgfsetfillcolor{currentfill}%
\pgfsetlinewidth{0.602250pt}%
\definecolor{currentstroke}{rgb}{0.000000,0.000000,0.000000}%
\pgfsetstrokecolor{currentstroke}%
\pgfsetdash{}{0pt}%
\pgfsys@defobject{currentmarker}{\pgfqpoint{0.000000in}{-0.027778in}}{\pgfqpoint{0.000000in}{0.000000in}}{%
\pgfpathmoveto{\pgfqpoint{0.000000in}{0.000000in}}%
\pgfpathlineto{\pgfqpoint{0.000000in}{-0.027778in}}%
\pgfusepath{stroke,fill}%
}%
\begin{pgfscope}%
\pgfsys@transformshift{4.823690in}{0.552778in}%
\pgfsys@useobject{currentmarker}{}%
\end{pgfscope}%
\end{pgfscope}%
\begin{pgfscope}%
\pgfsetbuttcap%
\pgfsetroundjoin%
\definecolor{currentfill}{rgb}{0.000000,0.000000,0.000000}%
\pgfsetfillcolor{currentfill}%
\pgfsetlinewidth{0.602250pt}%
\definecolor{currentstroke}{rgb}{0.000000,0.000000,0.000000}%
\pgfsetstrokecolor{currentstroke}%
\pgfsetdash{}{0pt}%
\pgfsys@defobject{currentmarker}{\pgfqpoint{0.000000in}{0.000000in}}{\pgfqpoint{0.000000in}{0.027778in}}{%
\pgfpathmoveto{\pgfqpoint{0.000000in}{0.000000in}}%
\pgfpathlineto{\pgfqpoint{0.000000in}{0.027778in}}%
\pgfusepath{stroke,fill}%
}%
\begin{pgfscope}%
\pgfsys@transformshift{4.823690in}{2.202778in}%
\pgfsys@useobject{currentmarker}{}%
\end{pgfscope}%
\end{pgfscope}%
\begin{pgfscope}%
\pgfpathrectangle{\pgfqpoint{3.662674in}{0.552778in}}{\pgfqpoint{2.138715in}{1.650000in}}%
\pgfusepath{clip}%
\pgfsetrectcap%
\pgfsetroundjoin%
\pgfsetlinewidth{0.803000pt}%
\definecolor{currentstroke}{rgb}{0.690196,0.690196,0.690196}%
\pgfsetstrokecolor{currentstroke}%
\pgfsetstrokeopacity{0.300000}%
\pgfsetdash{}{0pt}%
\pgfpathmoveto{\pgfqpoint{4.869520in}{0.552778in}}%
\pgfpathlineto{\pgfqpoint{4.869520in}{2.202778in}}%
\pgfusepath{stroke}%
\end{pgfscope}%
\begin{pgfscope}%
\pgfsetbuttcap%
\pgfsetroundjoin%
\definecolor{currentfill}{rgb}{0.000000,0.000000,0.000000}%
\pgfsetfillcolor{currentfill}%
\pgfsetlinewidth{0.602250pt}%
\definecolor{currentstroke}{rgb}{0.000000,0.000000,0.000000}%
\pgfsetstrokecolor{currentstroke}%
\pgfsetdash{}{0pt}%
\pgfsys@defobject{currentmarker}{\pgfqpoint{0.000000in}{-0.027778in}}{\pgfqpoint{0.000000in}{0.000000in}}{%
\pgfpathmoveto{\pgfqpoint{0.000000in}{0.000000in}}%
\pgfpathlineto{\pgfqpoint{0.000000in}{-0.027778in}}%
\pgfusepath{stroke,fill}%
}%
\begin{pgfscope}%
\pgfsys@transformshift{4.869520in}{0.552778in}%
\pgfsys@useobject{currentmarker}{}%
\end{pgfscope}%
\end{pgfscope}%
\begin{pgfscope}%
\pgfsetbuttcap%
\pgfsetroundjoin%
\definecolor{currentfill}{rgb}{0.000000,0.000000,0.000000}%
\pgfsetfillcolor{currentfill}%
\pgfsetlinewidth{0.602250pt}%
\definecolor{currentstroke}{rgb}{0.000000,0.000000,0.000000}%
\pgfsetstrokecolor{currentstroke}%
\pgfsetdash{}{0pt}%
\pgfsys@defobject{currentmarker}{\pgfqpoint{0.000000in}{0.000000in}}{\pgfqpoint{0.000000in}{0.027778in}}{%
\pgfpathmoveto{\pgfqpoint{0.000000in}{0.000000in}}%
\pgfpathlineto{\pgfqpoint{0.000000in}{0.027778in}}%
\pgfusepath{stroke,fill}%
}%
\begin{pgfscope}%
\pgfsys@transformshift{4.869520in}{2.202778in}%
\pgfsys@useobject{currentmarker}{}%
\end{pgfscope}%
\end{pgfscope}%
\begin{pgfscope}%
\pgfpathrectangle{\pgfqpoint{3.662674in}{0.552778in}}{\pgfqpoint{2.138715in}{1.650000in}}%
\pgfusepath{clip}%
\pgfsetrectcap%
\pgfsetroundjoin%
\pgfsetlinewidth{0.803000pt}%
\definecolor{currentstroke}{rgb}{0.690196,0.690196,0.690196}%
\pgfsetstrokecolor{currentstroke}%
\pgfsetstrokeopacity{0.300000}%
\pgfsetdash{}{0pt}%
\pgfpathmoveto{\pgfqpoint{4.915350in}{0.552778in}}%
\pgfpathlineto{\pgfqpoint{4.915350in}{2.202778in}}%
\pgfusepath{stroke}%
\end{pgfscope}%
\begin{pgfscope}%
\pgfsetbuttcap%
\pgfsetroundjoin%
\definecolor{currentfill}{rgb}{0.000000,0.000000,0.000000}%
\pgfsetfillcolor{currentfill}%
\pgfsetlinewidth{0.602250pt}%
\definecolor{currentstroke}{rgb}{0.000000,0.000000,0.000000}%
\pgfsetstrokecolor{currentstroke}%
\pgfsetdash{}{0pt}%
\pgfsys@defobject{currentmarker}{\pgfqpoint{0.000000in}{-0.027778in}}{\pgfqpoint{0.000000in}{0.000000in}}{%
\pgfpathmoveto{\pgfqpoint{0.000000in}{0.000000in}}%
\pgfpathlineto{\pgfqpoint{0.000000in}{-0.027778in}}%
\pgfusepath{stroke,fill}%
}%
\begin{pgfscope}%
\pgfsys@transformshift{4.915350in}{0.552778in}%
\pgfsys@useobject{currentmarker}{}%
\end{pgfscope}%
\end{pgfscope}%
\begin{pgfscope}%
\pgfsetbuttcap%
\pgfsetroundjoin%
\definecolor{currentfill}{rgb}{0.000000,0.000000,0.000000}%
\pgfsetfillcolor{currentfill}%
\pgfsetlinewidth{0.602250pt}%
\definecolor{currentstroke}{rgb}{0.000000,0.000000,0.000000}%
\pgfsetstrokecolor{currentstroke}%
\pgfsetdash{}{0pt}%
\pgfsys@defobject{currentmarker}{\pgfqpoint{0.000000in}{0.000000in}}{\pgfqpoint{0.000000in}{0.027778in}}{%
\pgfpathmoveto{\pgfqpoint{0.000000in}{0.000000in}}%
\pgfpathlineto{\pgfqpoint{0.000000in}{0.027778in}}%
\pgfusepath{stroke,fill}%
}%
\begin{pgfscope}%
\pgfsys@transformshift{4.915350in}{2.202778in}%
\pgfsys@useobject{currentmarker}{}%
\end{pgfscope}%
\end{pgfscope}%
\begin{pgfscope}%
\pgfpathrectangle{\pgfqpoint{3.662674in}{0.552778in}}{\pgfqpoint{2.138715in}{1.650000in}}%
\pgfusepath{clip}%
\pgfsetrectcap%
\pgfsetroundjoin%
\pgfsetlinewidth{0.803000pt}%
\definecolor{currentstroke}{rgb}{0.690196,0.690196,0.690196}%
\pgfsetstrokecolor{currentstroke}%
\pgfsetstrokeopacity{0.300000}%
\pgfsetdash{}{0pt}%
\pgfpathmoveto{\pgfqpoint{4.961179in}{0.552778in}}%
\pgfpathlineto{\pgfqpoint{4.961179in}{2.202778in}}%
\pgfusepath{stroke}%
\end{pgfscope}%
\begin{pgfscope}%
\pgfsetbuttcap%
\pgfsetroundjoin%
\definecolor{currentfill}{rgb}{0.000000,0.000000,0.000000}%
\pgfsetfillcolor{currentfill}%
\pgfsetlinewidth{0.602250pt}%
\definecolor{currentstroke}{rgb}{0.000000,0.000000,0.000000}%
\pgfsetstrokecolor{currentstroke}%
\pgfsetdash{}{0pt}%
\pgfsys@defobject{currentmarker}{\pgfqpoint{0.000000in}{-0.027778in}}{\pgfqpoint{0.000000in}{0.000000in}}{%
\pgfpathmoveto{\pgfqpoint{0.000000in}{0.000000in}}%
\pgfpathlineto{\pgfqpoint{0.000000in}{-0.027778in}}%
\pgfusepath{stroke,fill}%
}%
\begin{pgfscope}%
\pgfsys@transformshift{4.961179in}{0.552778in}%
\pgfsys@useobject{currentmarker}{}%
\end{pgfscope}%
\end{pgfscope}%
\begin{pgfscope}%
\pgfsetbuttcap%
\pgfsetroundjoin%
\definecolor{currentfill}{rgb}{0.000000,0.000000,0.000000}%
\pgfsetfillcolor{currentfill}%
\pgfsetlinewidth{0.602250pt}%
\definecolor{currentstroke}{rgb}{0.000000,0.000000,0.000000}%
\pgfsetstrokecolor{currentstroke}%
\pgfsetdash{}{0pt}%
\pgfsys@defobject{currentmarker}{\pgfqpoint{0.000000in}{0.000000in}}{\pgfqpoint{0.000000in}{0.027778in}}{%
\pgfpathmoveto{\pgfqpoint{0.000000in}{0.000000in}}%
\pgfpathlineto{\pgfqpoint{0.000000in}{0.027778in}}%
\pgfusepath{stroke,fill}%
}%
\begin{pgfscope}%
\pgfsys@transformshift{4.961179in}{2.202778in}%
\pgfsys@useobject{currentmarker}{}%
\end{pgfscope}%
\end{pgfscope}%
\begin{pgfscope}%
\pgfpathrectangle{\pgfqpoint{3.662674in}{0.552778in}}{\pgfqpoint{2.138715in}{1.650000in}}%
\pgfusepath{clip}%
\pgfsetrectcap%
\pgfsetroundjoin%
\pgfsetlinewidth{0.803000pt}%
\definecolor{currentstroke}{rgb}{0.690196,0.690196,0.690196}%
\pgfsetstrokecolor{currentstroke}%
\pgfsetstrokeopacity{0.300000}%
\pgfsetdash{}{0pt}%
\pgfpathmoveto{\pgfqpoint{5.007009in}{0.552778in}}%
\pgfpathlineto{\pgfqpoint{5.007009in}{2.202778in}}%
\pgfusepath{stroke}%
\end{pgfscope}%
\begin{pgfscope}%
\pgfsetbuttcap%
\pgfsetroundjoin%
\definecolor{currentfill}{rgb}{0.000000,0.000000,0.000000}%
\pgfsetfillcolor{currentfill}%
\pgfsetlinewidth{0.602250pt}%
\definecolor{currentstroke}{rgb}{0.000000,0.000000,0.000000}%
\pgfsetstrokecolor{currentstroke}%
\pgfsetdash{}{0pt}%
\pgfsys@defobject{currentmarker}{\pgfqpoint{0.000000in}{-0.027778in}}{\pgfqpoint{0.000000in}{0.000000in}}{%
\pgfpathmoveto{\pgfqpoint{0.000000in}{0.000000in}}%
\pgfpathlineto{\pgfqpoint{0.000000in}{-0.027778in}}%
\pgfusepath{stroke,fill}%
}%
\begin{pgfscope}%
\pgfsys@transformshift{5.007009in}{0.552778in}%
\pgfsys@useobject{currentmarker}{}%
\end{pgfscope}%
\end{pgfscope}%
\begin{pgfscope}%
\pgfsetbuttcap%
\pgfsetroundjoin%
\definecolor{currentfill}{rgb}{0.000000,0.000000,0.000000}%
\pgfsetfillcolor{currentfill}%
\pgfsetlinewidth{0.602250pt}%
\definecolor{currentstroke}{rgb}{0.000000,0.000000,0.000000}%
\pgfsetstrokecolor{currentstroke}%
\pgfsetdash{}{0pt}%
\pgfsys@defobject{currentmarker}{\pgfqpoint{0.000000in}{0.000000in}}{\pgfqpoint{0.000000in}{0.027778in}}{%
\pgfpathmoveto{\pgfqpoint{0.000000in}{0.000000in}}%
\pgfpathlineto{\pgfqpoint{0.000000in}{0.027778in}}%
\pgfusepath{stroke,fill}%
}%
\begin{pgfscope}%
\pgfsys@transformshift{5.007009in}{2.202778in}%
\pgfsys@useobject{currentmarker}{}%
\end{pgfscope}%
\end{pgfscope}%
\begin{pgfscope}%
\pgfpathrectangle{\pgfqpoint{3.662674in}{0.552778in}}{\pgfqpoint{2.138715in}{1.650000in}}%
\pgfusepath{clip}%
\pgfsetrectcap%
\pgfsetroundjoin%
\pgfsetlinewidth{0.803000pt}%
\definecolor{currentstroke}{rgb}{0.690196,0.690196,0.690196}%
\pgfsetstrokecolor{currentstroke}%
\pgfsetstrokeopacity{0.300000}%
\pgfsetdash{}{0pt}%
\pgfpathmoveto{\pgfqpoint{5.052839in}{0.552778in}}%
\pgfpathlineto{\pgfqpoint{5.052839in}{2.202778in}}%
\pgfusepath{stroke}%
\end{pgfscope}%
\begin{pgfscope}%
\pgfsetbuttcap%
\pgfsetroundjoin%
\definecolor{currentfill}{rgb}{0.000000,0.000000,0.000000}%
\pgfsetfillcolor{currentfill}%
\pgfsetlinewidth{0.602250pt}%
\definecolor{currentstroke}{rgb}{0.000000,0.000000,0.000000}%
\pgfsetstrokecolor{currentstroke}%
\pgfsetdash{}{0pt}%
\pgfsys@defobject{currentmarker}{\pgfqpoint{0.000000in}{-0.027778in}}{\pgfqpoint{0.000000in}{0.000000in}}{%
\pgfpathmoveto{\pgfqpoint{0.000000in}{0.000000in}}%
\pgfpathlineto{\pgfqpoint{0.000000in}{-0.027778in}}%
\pgfusepath{stroke,fill}%
}%
\begin{pgfscope}%
\pgfsys@transformshift{5.052839in}{0.552778in}%
\pgfsys@useobject{currentmarker}{}%
\end{pgfscope}%
\end{pgfscope}%
\begin{pgfscope}%
\pgfsetbuttcap%
\pgfsetroundjoin%
\definecolor{currentfill}{rgb}{0.000000,0.000000,0.000000}%
\pgfsetfillcolor{currentfill}%
\pgfsetlinewidth{0.602250pt}%
\definecolor{currentstroke}{rgb}{0.000000,0.000000,0.000000}%
\pgfsetstrokecolor{currentstroke}%
\pgfsetdash{}{0pt}%
\pgfsys@defobject{currentmarker}{\pgfqpoint{0.000000in}{0.000000in}}{\pgfqpoint{0.000000in}{0.027778in}}{%
\pgfpathmoveto{\pgfqpoint{0.000000in}{0.000000in}}%
\pgfpathlineto{\pgfqpoint{0.000000in}{0.027778in}}%
\pgfusepath{stroke,fill}%
}%
\begin{pgfscope}%
\pgfsys@transformshift{5.052839in}{2.202778in}%
\pgfsys@useobject{currentmarker}{}%
\end{pgfscope}%
\end{pgfscope}%
\begin{pgfscope}%
\pgfpathrectangle{\pgfqpoint{3.662674in}{0.552778in}}{\pgfqpoint{2.138715in}{1.650000in}}%
\pgfusepath{clip}%
\pgfsetrectcap%
\pgfsetroundjoin%
\pgfsetlinewidth{0.803000pt}%
\definecolor{currentstroke}{rgb}{0.690196,0.690196,0.690196}%
\pgfsetstrokecolor{currentstroke}%
\pgfsetstrokeopacity{0.300000}%
\pgfsetdash{}{0pt}%
\pgfpathmoveto{\pgfqpoint{5.098668in}{0.552778in}}%
\pgfpathlineto{\pgfqpoint{5.098668in}{2.202778in}}%
\pgfusepath{stroke}%
\end{pgfscope}%
\begin{pgfscope}%
\pgfsetbuttcap%
\pgfsetroundjoin%
\definecolor{currentfill}{rgb}{0.000000,0.000000,0.000000}%
\pgfsetfillcolor{currentfill}%
\pgfsetlinewidth{0.602250pt}%
\definecolor{currentstroke}{rgb}{0.000000,0.000000,0.000000}%
\pgfsetstrokecolor{currentstroke}%
\pgfsetdash{}{0pt}%
\pgfsys@defobject{currentmarker}{\pgfqpoint{0.000000in}{-0.027778in}}{\pgfqpoint{0.000000in}{0.000000in}}{%
\pgfpathmoveto{\pgfqpoint{0.000000in}{0.000000in}}%
\pgfpathlineto{\pgfqpoint{0.000000in}{-0.027778in}}%
\pgfusepath{stroke,fill}%
}%
\begin{pgfscope}%
\pgfsys@transformshift{5.098668in}{0.552778in}%
\pgfsys@useobject{currentmarker}{}%
\end{pgfscope}%
\end{pgfscope}%
\begin{pgfscope}%
\pgfsetbuttcap%
\pgfsetroundjoin%
\definecolor{currentfill}{rgb}{0.000000,0.000000,0.000000}%
\pgfsetfillcolor{currentfill}%
\pgfsetlinewidth{0.602250pt}%
\definecolor{currentstroke}{rgb}{0.000000,0.000000,0.000000}%
\pgfsetstrokecolor{currentstroke}%
\pgfsetdash{}{0pt}%
\pgfsys@defobject{currentmarker}{\pgfqpoint{0.000000in}{0.000000in}}{\pgfqpoint{0.000000in}{0.027778in}}{%
\pgfpathmoveto{\pgfqpoint{0.000000in}{0.000000in}}%
\pgfpathlineto{\pgfqpoint{0.000000in}{0.027778in}}%
\pgfusepath{stroke,fill}%
}%
\begin{pgfscope}%
\pgfsys@transformshift{5.098668in}{2.202778in}%
\pgfsys@useobject{currentmarker}{}%
\end{pgfscope}%
\end{pgfscope}%
\begin{pgfscope}%
\pgfpathrectangle{\pgfqpoint{3.662674in}{0.552778in}}{\pgfqpoint{2.138715in}{1.650000in}}%
\pgfusepath{clip}%
\pgfsetrectcap%
\pgfsetroundjoin%
\pgfsetlinewidth{0.803000pt}%
\definecolor{currentstroke}{rgb}{0.690196,0.690196,0.690196}%
\pgfsetstrokecolor{currentstroke}%
\pgfsetstrokeopacity{0.300000}%
\pgfsetdash{}{0pt}%
\pgfpathmoveto{\pgfqpoint{5.144498in}{0.552778in}}%
\pgfpathlineto{\pgfqpoint{5.144498in}{2.202778in}}%
\pgfusepath{stroke}%
\end{pgfscope}%
\begin{pgfscope}%
\pgfsetbuttcap%
\pgfsetroundjoin%
\definecolor{currentfill}{rgb}{0.000000,0.000000,0.000000}%
\pgfsetfillcolor{currentfill}%
\pgfsetlinewidth{0.602250pt}%
\definecolor{currentstroke}{rgb}{0.000000,0.000000,0.000000}%
\pgfsetstrokecolor{currentstroke}%
\pgfsetdash{}{0pt}%
\pgfsys@defobject{currentmarker}{\pgfqpoint{0.000000in}{-0.027778in}}{\pgfqpoint{0.000000in}{0.000000in}}{%
\pgfpathmoveto{\pgfqpoint{0.000000in}{0.000000in}}%
\pgfpathlineto{\pgfqpoint{0.000000in}{-0.027778in}}%
\pgfusepath{stroke,fill}%
}%
\begin{pgfscope}%
\pgfsys@transformshift{5.144498in}{0.552778in}%
\pgfsys@useobject{currentmarker}{}%
\end{pgfscope}%
\end{pgfscope}%
\begin{pgfscope}%
\pgfsetbuttcap%
\pgfsetroundjoin%
\definecolor{currentfill}{rgb}{0.000000,0.000000,0.000000}%
\pgfsetfillcolor{currentfill}%
\pgfsetlinewidth{0.602250pt}%
\definecolor{currentstroke}{rgb}{0.000000,0.000000,0.000000}%
\pgfsetstrokecolor{currentstroke}%
\pgfsetdash{}{0pt}%
\pgfsys@defobject{currentmarker}{\pgfqpoint{0.000000in}{0.000000in}}{\pgfqpoint{0.000000in}{0.027778in}}{%
\pgfpathmoveto{\pgfqpoint{0.000000in}{0.000000in}}%
\pgfpathlineto{\pgfqpoint{0.000000in}{0.027778in}}%
\pgfusepath{stroke,fill}%
}%
\begin{pgfscope}%
\pgfsys@transformshift{5.144498in}{2.202778in}%
\pgfsys@useobject{currentmarker}{}%
\end{pgfscope}%
\end{pgfscope}%
\begin{pgfscope}%
\pgfpathrectangle{\pgfqpoint{3.662674in}{0.552778in}}{\pgfqpoint{2.138715in}{1.650000in}}%
\pgfusepath{clip}%
\pgfsetrectcap%
\pgfsetroundjoin%
\pgfsetlinewidth{0.803000pt}%
\definecolor{currentstroke}{rgb}{0.690196,0.690196,0.690196}%
\pgfsetstrokecolor{currentstroke}%
\pgfsetstrokeopacity{0.300000}%
\pgfsetdash{}{0pt}%
\pgfpathmoveto{\pgfqpoint{5.236157in}{0.552778in}}%
\pgfpathlineto{\pgfqpoint{5.236157in}{2.202778in}}%
\pgfusepath{stroke}%
\end{pgfscope}%
\begin{pgfscope}%
\pgfsetbuttcap%
\pgfsetroundjoin%
\definecolor{currentfill}{rgb}{0.000000,0.000000,0.000000}%
\pgfsetfillcolor{currentfill}%
\pgfsetlinewidth{0.602250pt}%
\definecolor{currentstroke}{rgb}{0.000000,0.000000,0.000000}%
\pgfsetstrokecolor{currentstroke}%
\pgfsetdash{}{0pt}%
\pgfsys@defobject{currentmarker}{\pgfqpoint{0.000000in}{-0.027778in}}{\pgfqpoint{0.000000in}{0.000000in}}{%
\pgfpathmoveto{\pgfqpoint{0.000000in}{0.000000in}}%
\pgfpathlineto{\pgfqpoint{0.000000in}{-0.027778in}}%
\pgfusepath{stroke,fill}%
}%
\begin{pgfscope}%
\pgfsys@transformshift{5.236157in}{0.552778in}%
\pgfsys@useobject{currentmarker}{}%
\end{pgfscope}%
\end{pgfscope}%
\begin{pgfscope}%
\pgfsetbuttcap%
\pgfsetroundjoin%
\definecolor{currentfill}{rgb}{0.000000,0.000000,0.000000}%
\pgfsetfillcolor{currentfill}%
\pgfsetlinewidth{0.602250pt}%
\definecolor{currentstroke}{rgb}{0.000000,0.000000,0.000000}%
\pgfsetstrokecolor{currentstroke}%
\pgfsetdash{}{0pt}%
\pgfsys@defobject{currentmarker}{\pgfqpoint{0.000000in}{0.000000in}}{\pgfqpoint{0.000000in}{0.027778in}}{%
\pgfpathmoveto{\pgfqpoint{0.000000in}{0.000000in}}%
\pgfpathlineto{\pgfqpoint{0.000000in}{0.027778in}}%
\pgfusepath{stroke,fill}%
}%
\begin{pgfscope}%
\pgfsys@transformshift{5.236157in}{2.202778in}%
\pgfsys@useobject{currentmarker}{}%
\end{pgfscope}%
\end{pgfscope}%
\begin{pgfscope}%
\pgfpathrectangle{\pgfqpoint{3.662674in}{0.552778in}}{\pgfqpoint{2.138715in}{1.650000in}}%
\pgfusepath{clip}%
\pgfsetrectcap%
\pgfsetroundjoin%
\pgfsetlinewidth{0.803000pt}%
\definecolor{currentstroke}{rgb}{0.690196,0.690196,0.690196}%
\pgfsetstrokecolor{currentstroke}%
\pgfsetstrokeopacity{0.300000}%
\pgfsetdash{}{0pt}%
\pgfpathmoveto{\pgfqpoint{5.281987in}{0.552778in}}%
\pgfpathlineto{\pgfqpoint{5.281987in}{2.202778in}}%
\pgfusepath{stroke}%
\end{pgfscope}%
\begin{pgfscope}%
\pgfsetbuttcap%
\pgfsetroundjoin%
\definecolor{currentfill}{rgb}{0.000000,0.000000,0.000000}%
\pgfsetfillcolor{currentfill}%
\pgfsetlinewidth{0.602250pt}%
\definecolor{currentstroke}{rgb}{0.000000,0.000000,0.000000}%
\pgfsetstrokecolor{currentstroke}%
\pgfsetdash{}{0pt}%
\pgfsys@defobject{currentmarker}{\pgfqpoint{0.000000in}{-0.027778in}}{\pgfqpoint{0.000000in}{0.000000in}}{%
\pgfpathmoveto{\pgfqpoint{0.000000in}{0.000000in}}%
\pgfpathlineto{\pgfqpoint{0.000000in}{-0.027778in}}%
\pgfusepath{stroke,fill}%
}%
\begin{pgfscope}%
\pgfsys@transformshift{5.281987in}{0.552778in}%
\pgfsys@useobject{currentmarker}{}%
\end{pgfscope}%
\end{pgfscope}%
\begin{pgfscope}%
\pgfsetbuttcap%
\pgfsetroundjoin%
\definecolor{currentfill}{rgb}{0.000000,0.000000,0.000000}%
\pgfsetfillcolor{currentfill}%
\pgfsetlinewidth{0.602250pt}%
\definecolor{currentstroke}{rgb}{0.000000,0.000000,0.000000}%
\pgfsetstrokecolor{currentstroke}%
\pgfsetdash{}{0pt}%
\pgfsys@defobject{currentmarker}{\pgfqpoint{0.000000in}{0.000000in}}{\pgfqpoint{0.000000in}{0.027778in}}{%
\pgfpathmoveto{\pgfqpoint{0.000000in}{0.000000in}}%
\pgfpathlineto{\pgfqpoint{0.000000in}{0.027778in}}%
\pgfusepath{stroke,fill}%
}%
\begin{pgfscope}%
\pgfsys@transformshift{5.281987in}{2.202778in}%
\pgfsys@useobject{currentmarker}{}%
\end{pgfscope}%
\end{pgfscope}%
\begin{pgfscope}%
\pgfpathrectangle{\pgfqpoint{3.662674in}{0.552778in}}{\pgfqpoint{2.138715in}{1.650000in}}%
\pgfusepath{clip}%
\pgfsetrectcap%
\pgfsetroundjoin%
\pgfsetlinewidth{0.803000pt}%
\definecolor{currentstroke}{rgb}{0.690196,0.690196,0.690196}%
\pgfsetstrokecolor{currentstroke}%
\pgfsetstrokeopacity{0.300000}%
\pgfsetdash{}{0pt}%
\pgfpathmoveto{\pgfqpoint{5.327816in}{0.552778in}}%
\pgfpathlineto{\pgfqpoint{5.327816in}{2.202778in}}%
\pgfusepath{stroke}%
\end{pgfscope}%
\begin{pgfscope}%
\pgfsetbuttcap%
\pgfsetroundjoin%
\definecolor{currentfill}{rgb}{0.000000,0.000000,0.000000}%
\pgfsetfillcolor{currentfill}%
\pgfsetlinewidth{0.602250pt}%
\definecolor{currentstroke}{rgb}{0.000000,0.000000,0.000000}%
\pgfsetstrokecolor{currentstroke}%
\pgfsetdash{}{0pt}%
\pgfsys@defobject{currentmarker}{\pgfqpoint{0.000000in}{-0.027778in}}{\pgfqpoint{0.000000in}{0.000000in}}{%
\pgfpathmoveto{\pgfqpoint{0.000000in}{0.000000in}}%
\pgfpathlineto{\pgfqpoint{0.000000in}{-0.027778in}}%
\pgfusepath{stroke,fill}%
}%
\begin{pgfscope}%
\pgfsys@transformshift{5.327816in}{0.552778in}%
\pgfsys@useobject{currentmarker}{}%
\end{pgfscope}%
\end{pgfscope}%
\begin{pgfscope}%
\pgfsetbuttcap%
\pgfsetroundjoin%
\definecolor{currentfill}{rgb}{0.000000,0.000000,0.000000}%
\pgfsetfillcolor{currentfill}%
\pgfsetlinewidth{0.602250pt}%
\definecolor{currentstroke}{rgb}{0.000000,0.000000,0.000000}%
\pgfsetstrokecolor{currentstroke}%
\pgfsetdash{}{0pt}%
\pgfsys@defobject{currentmarker}{\pgfqpoint{0.000000in}{0.000000in}}{\pgfqpoint{0.000000in}{0.027778in}}{%
\pgfpathmoveto{\pgfqpoint{0.000000in}{0.000000in}}%
\pgfpathlineto{\pgfqpoint{0.000000in}{0.027778in}}%
\pgfusepath{stroke,fill}%
}%
\begin{pgfscope}%
\pgfsys@transformshift{5.327816in}{2.202778in}%
\pgfsys@useobject{currentmarker}{}%
\end{pgfscope}%
\end{pgfscope}%
\begin{pgfscope}%
\pgfpathrectangle{\pgfqpoint{3.662674in}{0.552778in}}{\pgfqpoint{2.138715in}{1.650000in}}%
\pgfusepath{clip}%
\pgfsetrectcap%
\pgfsetroundjoin%
\pgfsetlinewidth{0.803000pt}%
\definecolor{currentstroke}{rgb}{0.690196,0.690196,0.690196}%
\pgfsetstrokecolor{currentstroke}%
\pgfsetstrokeopacity{0.300000}%
\pgfsetdash{}{0pt}%
\pgfpathmoveto{\pgfqpoint{5.373646in}{0.552778in}}%
\pgfpathlineto{\pgfqpoint{5.373646in}{2.202778in}}%
\pgfusepath{stroke}%
\end{pgfscope}%
\begin{pgfscope}%
\pgfsetbuttcap%
\pgfsetroundjoin%
\definecolor{currentfill}{rgb}{0.000000,0.000000,0.000000}%
\pgfsetfillcolor{currentfill}%
\pgfsetlinewidth{0.602250pt}%
\definecolor{currentstroke}{rgb}{0.000000,0.000000,0.000000}%
\pgfsetstrokecolor{currentstroke}%
\pgfsetdash{}{0pt}%
\pgfsys@defobject{currentmarker}{\pgfqpoint{0.000000in}{-0.027778in}}{\pgfqpoint{0.000000in}{0.000000in}}{%
\pgfpathmoveto{\pgfqpoint{0.000000in}{0.000000in}}%
\pgfpathlineto{\pgfqpoint{0.000000in}{-0.027778in}}%
\pgfusepath{stroke,fill}%
}%
\begin{pgfscope}%
\pgfsys@transformshift{5.373646in}{0.552778in}%
\pgfsys@useobject{currentmarker}{}%
\end{pgfscope}%
\end{pgfscope}%
\begin{pgfscope}%
\pgfsetbuttcap%
\pgfsetroundjoin%
\definecolor{currentfill}{rgb}{0.000000,0.000000,0.000000}%
\pgfsetfillcolor{currentfill}%
\pgfsetlinewidth{0.602250pt}%
\definecolor{currentstroke}{rgb}{0.000000,0.000000,0.000000}%
\pgfsetstrokecolor{currentstroke}%
\pgfsetdash{}{0pt}%
\pgfsys@defobject{currentmarker}{\pgfqpoint{0.000000in}{0.000000in}}{\pgfqpoint{0.000000in}{0.027778in}}{%
\pgfpathmoveto{\pgfqpoint{0.000000in}{0.000000in}}%
\pgfpathlineto{\pgfqpoint{0.000000in}{0.027778in}}%
\pgfusepath{stroke,fill}%
}%
\begin{pgfscope}%
\pgfsys@transformshift{5.373646in}{2.202778in}%
\pgfsys@useobject{currentmarker}{}%
\end{pgfscope}%
\end{pgfscope}%
\begin{pgfscope}%
\pgfpathrectangle{\pgfqpoint{3.662674in}{0.552778in}}{\pgfqpoint{2.138715in}{1.650000in}}%
\pgfusepath{clip}%
\pgfsetrectcap%
\pgfsetroundjoin%
\pgfsetlinewidth{0.803000pt}%
\definecolor{currentstroke}{rgb}{0.690196,0.690196,0.690196}%
\pgfsetstrokecolor{currentstroke}%
\pgfsetstrokeopacity{0.300000}%
\pgfsetdash{}{0pt}%
\pgfpathmoveto{\pgfqpoint{5.419475in}{0.552778in}}%
\pgfpathlineto{\pgfqpoint{5.419475in}{2.202778in}}%
\pgfusepath{stroke}%
\end{pgfscope}%
\begin{pgfscope}%
\pgfsetbuttcap%
\pgfsetroundjoin%
\definecolor{currentfill}{rgb}{0.000000,0.000000,0.000000}%
\pgfsetfillcolor{currentfill}%
\pgfsetlinewidth{0.602250pt}%
\definecolor{currentstroke}{rgb}{0.000000,0.000000,0.000000}%
\pgfsetstrokecolor{currentstroke}%
\pgfsetdash{}{0pt}%
\pgfsys@defobject{currentmarker}{\pgfqpoint{0.000000in}{-0.027778in}}{\pgfqpoint{0.000000in}{0.000000in}}{%
\pgfpathmoveto{\pgfqpoint{0.000000in}{0.000000in}}%
\pgfpathlineto{\pgfqpoint{0.000000in}{-0.027778in}}%
\pgfusepath{stroke,fill}%
}%
\begin{pgfscope}%
\pgfsys@transformshift{5.419475in}{0.552778in}%
\pgfsys@useobject{currentmarker}{}%
\end{pgfscope}%
\end{pgfscope}%
\begin{pgfscope}%
\pgfsetbuttcap%
\pgfsetroundjoin%
\definecolor{currentfill}{rgb}{0.000000,0.000000,0.000000}%
\pgfsetfillcolor{currentfill}%
\pgfsetlinewidth{0.602250pt}%
\definecolor{currentstroke}{rgb}{0.000000,0.000000,0.000000}%
\pgfsetstrokecolor{currentstroke}%
\pgfsetdash{}{0pt}%
\pgfsys@defobject{currentmarker}{\pgfqpoint{0.000000in}{0.000000in}}{\pgfqpoint{0.000000in}{0.027778in}}{%
\pgfpathmoveto{\pgfqpoint{0.000000in}{0.000000in}}%
\pgfpathlineto{\pgfqpoint{0.000000in}{0.027778in}}%
\pgfusepath{stroke,fill}%
}%
\begin{pgfscope}%
\pgfsys@transformshift{5.419475in}{2.202778in}%
\pgfsys@useobject{currentmarker}{}%
\end{pgfscope}%
\end{pgfscope}%
\begin{pgfscope}%
\pgfpathrectangle{\pgfqpoint{3.662674in}{0.552778in}}{\pgfqpoint{2.138715in}{1.650000in}}%
\pgfusepath{clip}%
\pgfsetrectcap%
\pgfsetroundjoin%
\pgfsetlinewidth{0.803000pt}%
\definecolor{currentstroke}{rgb}{0.690196,0.690196,0.690196}%
\pgfsetstrokecolor{currentstroke}%
\pgfsetstrokeopacity{0.300000}%
\pgfsetdash{}{0pt}%
\pgfpathmoveto{\pgfqpoint{5.465305in}{0.552778in}}%
\pgfpathlineto{\pgfqpoint{5.465305in}{2.202778in}}%
\pgfusepath{stroke}%
\end{pgfscope}%
\begin{pgfscope}%
\pgfsetbuttcap%
\pgfsetroundjoin%
\definecolor{currentfill}{rgb}{0.000000,0.000000,0.000000}%
\pgfsetfillcolor{currentfill}%
\pgfsetlinewidth{0.602250pt}%
\definecolor{currentstroke}{rgb}{0.000000,0.000000,0.000000}%
\pgfsetstrokecolor{currentstroke}%
\pgfsetdash{}{0pt}%
\pgfsys@defobject{currentmarker}{\pgfqpoint{0.000000in}{-0.027778in}}{\pgfqpoint{0.000000in}{0.000000in}}{%
\pgfpathmoveto{\pgfqpoint{0.000000in}{0.000000in}}%
\pgfpathlineto{\pgfqpoint{0.000000in}{-0.027778in}}%
\pgfusepath{stroke,fill}%
}%
\begin{pgfscope}%
\pgfsys@transformshift{5.465305in}{0.552778in}%
\pgfsys@useobject{currentmarker}{}%
\end{pgfscope}%
\end{pgfscope}%
\begin{pgfscope}%
\pgfsetbuttcap%
\pgfsetroundjoin%
\definecolor{currentfill}{rgb}{0.000000,0.000000,0.000000}%
\pgfsetfillcolor{currentfill}%
\pgfsetlinewidth{0.602250pt}%
\definecolor{currentstroke}{rgb}{0.000000,0.000000,0.000000}%
\pgfsetstrokecolor{currentstroke}%
\pgfsetdash{}{0pt}%
\pgfsys@defobject{currentmarker}{\pgfqpoint{0.000000in}{0.000000in}}{\pgfqpoint{0.000000in}{0.027778in}}{%
\pgfpathmoveto{\pgfqpoint{0.000000in}{0.000000in}}%
\pgfpathlineto{\pgfqpoint{0.000000in}{0.027778in}}%
\pgfusepath{stroke,fill}%
}%
\begin{pgfscope}%
\pgfsys@transformshift{5.465305in}{2.202778in}%
\pgfsys@useobject{currentmarker}{}%
\end{pgfscope}%
\end{pgfscope}%
\begin{pgfscope}%
\pgfpathrectangle{\pgfqpoint{3.662674in}{0.552778in}}{\pgfqpoint{2.138715in}{1.650000in}}%
\pgfusepath{clip}%
\pgfsetrectcap%
\pgfsetroundjoin%
\pgfsetlinewidth{0.803000pt}%
\definecolor{currentstroke}{rgb}{0.690196,0.690196,0.690196}%
\pgfsetstrokecolor{currentstroke}%
\pgfsetstrokeopacity{0.300000}%
\pgfsetdash{}{0pt}%
\pgfpathmoveto{\pgfqpoint{5.511135in}{0.552778in}}%
\pgfpathlineto{\pgfqpoint{5.511135in}{2.202778in}}%
\pgfusepath{stroke}%
\end{pgfscope}%
\begin{pgfscope}%
\pgfsetbuttcap%
\pgfsetroundjoin%
\definecolor{currentfill}{rgb}{0.000000,0.000000,0.000000}%
\pgfsetfillcolor{currentfill}%
\pgfsetlinewidth{0.602250pt}%
\definecolor{currentstroke}{rgb}{0.000000,0.000000,0.000000}%
\pgfsetstrokecolor{currentstroke}%
\pgfsetdash{}{0pt}%
\pgfsys@defobject{currentmarker}{\pgfqpoint{0.000000in}{-0.027778in}}{\pgfqpoint{0.000000in}{0.000000in}}{%
\pgfpathmoveto{\pgfqpoint{0.000000in}{0.000000in}}%
\pgfpathlineto{\pgfqpoint{0.000000in}{-0.027778in}}%
\pgfusepath{stroke,fill}%
}%
\begin{pgfscope}%
\pgfsys@transformshift{5.511135in}{0.552778in}%
\pgfsys@useobject{currentmarker}{}%
\end{pgfscope}%
\end{pgfscope}%
\begin{pgfscope}%
\pgfsetbuttcap%
\pgfsetroundjoin%
\definecolor{currentfill}{rgb}{0.000000,0.000000,0.000000}%
\pgfsetfillcolor{currentfill}%
\pgfsetlinewidth{0.602250pt}%
\definecolor{currentstroke}{rgb}{0.000000,0.000000,0.000000}%
\pgfsetstrokecolor{currentstroke}%
\pgfsetdash{}{0pt}%
\pgfsys@defobject{currentmarker}{\pgfqpoint{0.000000in}{0.000000in}}{\pgfqpoint{0.000000in}{0.027778in}}{%
\pgfpathmoveto{\pgfqpoint{0.000000in}{0.000000in}}%
\pgfpathlineto{\pgfqpoint{0.000000in}{0.027778in}}%
\pgfusepath{stroke,fill}%
}%
\begin{pgfscope}%
\pgfsys@transformshift{5.511135in}{2.202778in}%
\pgfsys@useobject{currentmarker}{}%
\end{pgfscope}%
\end{pgfscope}%
\begin{pgfscope}%
\pgfpathrectangle{\pgfqpoint{3.662674in}{0.552778in}}{\pgfqpoint{2.138715in}{1.650000in}}%
\pgfusepath{clip}%
\pgfsetrectcap%
\pgfsetroundjoin%
\pgfsetlinewidth{0.803000pt}%
\definecolor{currentstroke}{rgb}{0.690196,0.690196,0.690196}%
\pgfsetstrokecolor{currentstroke}%
\pgfsetstrokeopacity{0.300000}%
\pgfsetdash{}{0pt}%
\pgfpathmoveto{\pgfqpoint{5.556964in}{0.552778in}}%
\pgfpathlineto{\pgfqpoint{5.556964in}{2.202778in}}%
\pgfusepath{stroke}%
\end{pgfscope}%
\begin{pgfscope}%
\pgfsetbuttcap%
\pgfsetroundjoin%
\definecolor{currentfill}{rgb}{0.000000,0.000000,0.000000}%
\pgfsetfillcolor{currentfill}%
\pgfsetlinewidth{0.602250pt}%
\definecolor{currentstroke}{rgb}{0.000000,0.000000,0.000000}%
\pgfsetstrokecolor{currentstroke}%
\pgfsetdash{}{0pt}%
\pgfsys@defobject{currentmarker}{\pgfqpoint{0.000000in}{-0.027778in}}{\pgfqpoint{0.000000in}{0.000000in}}{%
\pgfpathmoveto{\pgfqpoint{0.000000in}{0.000000in}}%
\pgfpathlineto{\pgfqpoint{0.000000in}{-0.027778in}}%
\pgfusepath{stroke,fill}%
}%
\begin{pgfscope}%
\pgfsys@transformshift{5.556964in}{0.552778in}%
\pgfsys@useobject{currentmarker}{}%
\end{pgfscope}%
\end{pgfscope}%
\begin{pgfscope}%
\pgfsetbuttcap%
\pgfsetroundjoin%
\definecolor{currentfill}{rgb}{0.000000,0.000000,0.000000}%
\pgfsetfillcolor{currentfill}%
\pgfsetlinewidth{0.602250pt}%
\definecolor{currentstroke}{rgb}{0.000000,0.000000,0.000000}%
\pgfsetstrokecolor{currentstroke}%
\pgfsetdash{}{0pt}%
\pgfsys@defobject{currentmarker}{\pgfqpoint{0.000000in}{0.000000in}}{\pgfqpoint{0.000000in}{0.027778in}}{%
\pgfpathmoveto{\pgfqpoint{0.000000in}{0.000000in}}%
\pgfpathlineto{\pgfqpoint{0.000000in}{0.027778in}}%
\pgfusepath{stroke,fill}%
}%
\begin{pgfscope}%
\pgfsys@transformshift{5.556964in}{2.202778in}%
\pgfsys@useobject{currentmarker}{}%
\end{pgfscope}%
\end{pgfscope}%
\begin{pgfscope}%
\pgfpathrectangle{\pgfqpoint{3.662674in}{0.552778in}}{\pgfqpoint{2.138715in}{1.650000in}}%
\pgfusepath{clip}%
\pgfsetrectcap%
\pgfsetroundjoin%
\pgfsetlinewidth{0.803000pt}%
\definecolor{currentstroke}{rgb}{0.690196,0.690196,0.690196}%
\pgfsetstrokecolor{currentstroke}%
\pgfsetstrokeopacity{0.300000}%
\pgfsetdash{}{0pt}%
\pgfpathmoveto{\pgfqpoint{5.602794in}{0.552778in}}%
\pgfpathlineto{\pgfqpoint{5.602794in}{2.202778in}}%
\pgfusepath{stroke}%
\end{pgfscope}%
\begin{pgfscope}%
\pgfsetbuttcap%
\pgfsetroundjoin%
\definecolor{currentfill}{rgb}{0.000000,0.000000,0.000000}%
\pgfsetfillcolor{currentfill}%
\pgfsetlinewidth{0.602250pt}%
\definecolor{currentstroke}{rgb}{0.000000,0.000000,0.000000}%
\pgfsetstrokecolor{currentstroke}%
\pgfsetdash{}{0pt}%
\pgfsys@defobject{currentmarker}{\pgfqpoint{0.000000in}{-0.027778in}}{\pgfqpoint{0.000000in}{0.000000in}}{%
\pgfpathmoveto{\pgfqpoint{0.000000in}{0.000000in}}%
\pgfpathlineto{\pgfqpoint{0.000000in}{-0.027778in}}%
\pgfusepath{stroke,fill}%
}%
\begin{pgfscope}%
\pgfsys@transformshift{5.602794in}{0.552778in}%
\pgfsys@useobject{currentmarker}{}%
\end{pgfscope}%
\end{pgfscope}%
\begin{pgfscope}%
\pgfsetbuttcap%
\pgfsetroundjoin%
\definecolor{currentfill}{rgb}{0.000000,0.000000,0.000000}%
\pgfsetfillcolor{currentfill}%
\pgfsetlinewidth{0.602250pt}%
\definecolor{currentstroke}{rgb}{0.000000,0.000000,0.000000}%
\pgfsetstrokecolor{currentstroke}%
\pgfsetdash{}{0pt}%
\pgfsys@defobject{currentmarker}{\pgfqpoint{0.000000in}{0.000000in}}{\pgfqpoint{0.000000in}{0.027778in}}{%
\pgfpathmoveto{\pgfqpoint{0.000000in}{0.000000in}}%
\pgfpathlineto{\pgfqpoint{0.000000in}{0.027778in}}%
\pgfusepath{stroke,fill}%
}%
\begin{pgfscope}%
\pgfsys@transformshift{5.602794in}{2.202778in}%
\pgfsys@useobject{currentmarker}{}%
\end{pgfscope}%
\end{pgfscope}%
\begin{pgfscope}%
\pgfpathrectangle{\pgfqpoint{3.662674in}{0.552778in}}{\pgfqpoint{2.138715in}{1.650000in}}%
\pgfusepath{clip}%
\pgfsetrectcap%
\pgfsetroundjoin%
\pgfsetlinewidth{0.803000pt}%
\definecolor{currentstroke}{rgb}{0.690196,0.690196,0.690196}%
\pgfsetstrokecolor{currentstroke}%
\pgfsetstrokeopacity{0.300000}%
\pgfsetdash{}{0pt}%
\pgfpathmoveto{\pgfqpoint{5.694453in}{0.552778in}}%
\pgfpathlineto{\pgfqpoint{5.694453in}{2.202778in}}%
\pgfusepath{stroke}%
\end{pgfscope}%
\begin{pgfscope}%
\pgfsetbuttcap%
\pgfsetroundjoin%
\definecolor{currentfill}{rgb}{0.000000,0.000000,0.000000}%
\pgfsetfillcolor{currentfill}%
\pgfsetlinewidth{0.602250pt}%
\definecolor{currentstroke}{rgb}{0.000000,0.000000,0.000000}%
\pgfsetstrokecolor{currentstroke}%
\pgfsetdash{}{0pt}%
\pgfsys@defobject{currentmarker}{\pgfqpoint{0.000000in}{-0.027778in}}{\pgfqpoint{0.000000in}{0.000000in}}{%
\pgfpathmoveto{\pgfqpoint{0.000000in}{0.000000in}}%
\pgfpathlineto{\pgfqpoint{0.000000in}{-0.027778in}}%
\pgfusepath{stroke,fill}%
}%
\begin{pgfscope}%
\pgfsys@transformshift{5.694453in}{0.552778in}%
\pgfsys@useobject{currentmarker}{}%
\end{pgfscope}%
\end{pgfscope}%
\begin{pgfscope}%
\pgfsetbuttcap%
\pgfsetroundjoin%
\definecolor{currentfill}{rgb}{0.000000,0.000000,0.000000}%
\pgfsetfillcolor{currentfill}%
\pgfsetlinewidth{0.602250pt}%
\definecolor{currentstroke}{rgb}{0.000000,0.000000,0.000000}%
\pgfsetstrokecolor{currentstroke}%
\pgfsetdash{}{0pt}%
\pgfsys@defobject{currentmarker}{\pgfqpoint{0.000000in}{0.000000in}}{\pgfqpoint{0.000000in}{0.027778in}}{%
\pgfpathmoveto{\pgfqpoint{0.000000in}{0.000000in}}%
\pgfpathlineto{\pgfqpoint{0.000000in}{0.027778in}}%
\pgfusepath{stroke,fill}%
}%
\begin{pgfscope}%
\pgfsys@transformshift{5.694453in}{2.202778in}%
\pgfsys@useobject{currentmarker}{}%
\end{pgfscope}%
\end{pgfscope}%
\begin{pgfscope}%
\pgfpathrectangle{\pgfqpoint{3.662674in}{0.552778in}}{\pgfqpoint{2.138715in}{1.650000in}}%
\pgfusepath{clip}%
\pgfsetrectcap%
\pgfsetroundjoin%
\pgfsetlinewidth{0.803000pt}%
\definecolor{currentstroke}{rgb}{0.690196,0.690196,0.690196}%
\pgfsetstrokecolor{currentstroke}%
\pgfsetstrokeopacity{0.300000}%
\pgfsetdash{}{0pt}%
\pgfpathmoveto{\pgfqpoint{5.740283in}{0.552778in}}%
\pgfpathlineto{\pgfqpoint{5.740283in}{2.202778in}}%
\pgfusepath{stroke}%
\end{pgfscope}%
\begin{pgfscope}%
\pgfsetbuttcap%
\pgfsetroundjoin%
\definecolor{currentfill}{rgb}{0.000000,0.000000,0.000000}%
\pgfsetfillcolor{currentfill}%
\pgfsetlinewidth{0.602250pt}%
\definecolor{currentstroke}{rgb}{0.000000,0.000000,0.000000}%
\pgfsetstrokecolor{currentstroke}%
\pgfsetdash{}{0pt}%
\pgfsys@defobject{currentmarker}{\pgfqpoint{0.000000in}{-0.027778in}}{\pgfqpoint{0.000000in}{0.000000in}}{%
\pgfpathmoveto{\pgfqpoint{0.000000in}{0.000000in}}%
\pgfpathlineto{\pgfqpoint{0.000000in}{-0.027778in}}%
\pgfusepath{stroke,fill}%
}%
\begin{pgfscope}%
\pgfsys@transformshift{5.740283in}{0.552778in}%
\pgfsys@useobject{currentmarker}{}%
\end{pgfscope}%
\end{pgfscope}%
\begin{pgfscope}%
\pgfsetbuttcap%
\pgfsetroundjoin%
\definecolor{currentfill}{rgb}{0.000000,0.000000,0.000000}%
\pgfsetfillcolor{currentfill}%
\pgfsetlinewidth{0.602250pt}%
\definecolor{currentstroke}{rgb}{0.000000,0.000000,0.000000}%
\pgfsetstrokecolor{currentstroke}%
\pgfsetdash{}{0pt}%
\pgfsys@defobject{currentmarker}{\pgfqpoint{0.000000in}{0.000000in}}{\pgfqpoint{0.000000in}{0.027778in}}{%
\pgfpathmoveto{\pgfqpoint{0.000000in}{0.000000in}}%
\pgfpathlineto{\pgfqpoint{0.000000in}{0.027778in}}%
\pgfusepath{stroke,fill}%
}%
\begin{pgfscope}%
\pgfsys@transformshift{5.740283in}{2.202778in}%
\pgfsys@useobject{currentmarker}{}%
\end{pgfscope}%
\end{pgfscope}%
\begin{pgfscope}%
\pgfpathrectangle{\pgfqpoint{3.662674in}{0.552778in}}{\pgfqpoint{2.138715in}{1.650000in}}%
\pgfusepath{clip}%
\pgfsetrectcap%
\pgfsetroundjoin%
\pgfsetlinewidth{0.803000pt}%
\definecolor{currentstroke}{rgb}{0.690196,0.690196,0.690196}%
\pgfsetstrokecolor{currentstroke}%
\pgfsetstrokeopacity{0.300000}%
\pgfsetdash{}{0pt}%
\pgfpathmoveto{\pgfqpoint{5.786112in}{0.552778in}}%
\pgfpathlineto{\pgfqpoint{5.786112in}{2.202778in}}%
\pgfusepath{stroke}%
\end{pgfscope}%
\begin{pgfscope}%
\pgfsetbuttcap%
\pgfsetroundjoin%
\definecolor{currentfill}{rgb}{0.000000,0.000000,0.000000}%
\pgfsetfillcolor{currentfill}%
\pgfsetlinewidth{0.602250pt}%
\definecolor{currentstroke}{rgb}{0.000000,0.000000,0.000000}%
\pgfsetstrokecolor{currentstroke}%
\pgfsetdash{}{0pt}%
\pgfsys@defobject{currentmarker}{\pgfqpoint{0.000000in}{-0.027778in}}{\pgfqpoint{0.000000in}{0.000000in}}{%
\pgfpathmoveto{\pgfqpoint{0.000000in}{0.000000in}}%
\pgfpathlineto{\pgfqpoint{0.000000in}{-0.027778in}}%
\pgfusepath{stroke,fill}%
}%
\begin{pgfscope}%
\pgfsys@transformshift{5.786112in}{0.552778in}%
\pgfsys@useobject{currentmarker}{}%
\end{pgfscope}%
\end{pgfscope}%
\begin{pgfscope}%
\pgfsetbuttcap%
\pgfsetroundjoin%
\definecolor{currentfill}{rgb}{0.000000,0.000000,0.000000}%
\pgfsetfillcolor{currentfill}%
\pgfsetlinewidth{0.602250pt}%
\definecolor{currentstroke}{rgb}{0.000000,0.000000,0.000000}%
\pgfsetstrokecolor{currentstroke}%
\pgfsetdash{}{0pt}%
\pgfsys@defobject{currentmarker}{\pgfqpoint{0.000000in}{0.000000in}}{\pgfqpoint{0.000000in}{0.027778in}}{%
\pgfpathmoveto{\pgfqpoint{0.000000in}{0.000000in}}%
\pgfpathlineto{\pgfqpoint{0.000000in}{0.027778in}}%
\pgfusepath{stroke,fill}%
}%
\begin{pgfscope}%
\pgfsys@transformshift{5.786112in}{2.202778in}%
\pgfsys@useobject{currentmarker}{}%
\end{pgfscope}%
\end{pgfscope}%
\begin{pgfscope}%
\definecolor{textcolor}{rgb}{0.000000,0.000000,0.000000}%
\pgfsetstrokecolor{textcolor}%
\pgfsetfillcolor{textcolor}%
\pgftext[x=4.732031in,y=0.276667in,,top]{\color{textcolor}\rmfamily\fontsize{10.000000}{12.000000}\selectfont Energie [keV]}%
\end{pgfscope}%
\begin{pgfscope}%
\pgfpathrectangle{\pgfqpoint{3.662674in}{0.552778in}}{\pgfqpoint{2.138715in}{1.650000in}}%
\pgfusepath{clip}%
\pgfsetrectcap%
\pgfsetroundjoin%
\pgfsetlinewidth{0.803000pt}%
\definecolor{currentstroke}{rgb}{0.690196,0.690196,0.690196}%
\pgfsetstrokecolor{currentstroke}%
\pgfsetstrokeopacity{0.800000}%
\pgfsetdash{}{0pt}%
\pgfpathmoveto{\pgfqpoint{3.662674in}{0.627778in}}%
\pgfpathlineto{\pgfqpoint{5.801389in}{0.627778in}}%
\pgfusepath{stroke}%
\end{pgfscope}%
\begin{pgfscope}%
\pgfsetbuttcap%
\pgfsetroundjoin%
\definecolor{currentfill}{rgb}{0.000000,0.000000,0.000000}%
\pgfsetfillcolor{currentfill}%
\pgfsetlinewidth{0.803000pt}%
\definecolor{currentstroke}{rgb}{0.000000,0.000000,0.000000}%
\pgfsetstrokecolor{currentstroke}%
\pgfsetdash{}{0pt}%
\pgfsys@defobject{currentmarker}{\pgfqpoint{-0.048611in}{0.000000in}}{\pgfqpoint{0.000000in}{0.000000in}}{%
\pgfpathmoveto{\pgfqpoint{0.000000in}{0.000000in}}%
\pgfpathlineto{\pgfqpoint{-0.048611in}{0.000000in}}%
\pgfusepath{stroke,fill}%
}%
\begin{pgfscope}%
\pgfsys@transformshift{3.662674in}{0.627778in}%
\pgfsys@useobject{currentmarker}{}%
\end{pgfscope}%
\end{pgfscope}%
\begin{pgfscope}%
\pgfsetbuttcap%
\pgfsetroundjoin%
\definecolor{currentfill}{rgb}{0.000000,0.000000,0.000000}%
\pgfsetfillcolor{currentfill}%
\pgfsetlinewidth{0.803000pt}%
\definecolor{currentstroke}{rgb}{0.000000,0.000000,0.000000}%
\pgfsetstrokecolor{currentstroke}%
\pgfsetdash{}{0pt}%
\pgfsys@defobject{currentmarker}{\pgfqpoint{0.000000in}{0.000000in}}{\pgfqpoint{0.048611in}{0.000000in}}{%
\pgfpathmoveto{\pgfqpoint{0.000000in}{0.000000in}}%
\pgfpathlineto{\pgfqpoint{0.048611in}{0.000000in}}%
\pgfusepath{stroke,fill}%
}%
\begin{pgfscope}%
\pgfsys@transformshift{5.801389in}{0.627778in}%
\pgfsys@useobject{currentmarker}{}%
\end{pgfscope}%
\end{pgfscope}%
\begin{pgfscope}%
\definecolor{textcolor}{rgb}{0.000000,0.000000,0.000000}%
\pgfsetstrokecolor{textcolor}%
\pgfsetfillcolor{textcolor}%
\pgftext[x=3.496007in,y=0.579583in,left,base]{\color{textcolor}\rmfamily\fontsize{10.000000}{12.000000}\selectfont 0}%
\end{pgfscope}%
\begin{pgfscope}%
\pgfpathrectangle{\pgfqpoint{3.662674in}{0.552778in}}{\pgfqpoint{2.138715in}{1.650000in}}%
\pgfusepath{clip}%
\pgfsetrectcap%
\pgfsetroundjoin%
\pgfsetlinewidth{0.803000pt}%
\definecolor{currentstroke}{rgb}{0.690196,0.690196,0.690196}%
\pgfsetstrokecolor{currentstroke}%
\pgfsetstrokeopacity{0.800000}%
\pgfsetdash{}{0pt}%
\pgfpathmoveto{\pgfqpoint{3.662674in}{1.025304in}}%
\pgfpathlineto{\pgfqpoint{5.801389in}{1.025304in}}%
\pgfusepath{stroke}%
\end{pgfscope}%
\begin{pgfscope}%
\pgfsetbuttcap%
\pgfsetroundjoin%
\definecolor{currentfill}{rgb}{0.000000,0.000000,0.000000}%
\pgfsetfillcolor{currentfill}%
\pgfsetlinewidth{0.803000pt}%
\definecolor{currentstroke}{rgb}{0.000000,0.000000,0.000000}%
\pgfsetstrokecolor{currentstroke}%
\pgfsetdash{}{0pt}%
\pgfsys@defobject{currentmarker}{\pgfqpoint{-0.048611in}{0.000000in}}{\pgfqpoint{0.000000in}{0.000000in}}{%
\pgfpathmoveto{\pgfqpoint{0.000000in}{0.000000in}}%
\pgfpathlineto{\pgfqpoint{-0.048611in}{0.000000in}}%
\pgfusepath{stroke,fill}%
}%
\begin{pgfscope}%
\pgfsys@transformshift{3.662674in}{1.025304in}%
\pgfsys@useobject{currentmarker}{}%
\end{pgfscope}%
\end{pgfscope}%
\begin{pgfscope}%
\pgfsetbuttcap%
\pgfsetroundjoin%
\definecolor{currentfill}{rgb}{0.000000,0.000000,0.000000}%
\pgfsetfillcolor{currentfill}%
\pgfsetlinewidth{0.803000pt}%
\definecolor{currentstroke}{rgb}{0.000000,0.000000,0.000000}%
\pgfsetstrokecolor{currentstroke}%
\pgfsetdash{}{0pt}%
\pgfsys@defobject{currentmarker}{\pgfqpoint{0.000000in}{0.000000in}}{\pgfqpoint{0.048611in}{0.000000in}}{%
\pgfpathmoveto{\pgfqpoint{0.000000in}{0.000000in}}%
\pgfpathlineto{\pgfqpoint{0.048611in}{0.000000in}}%
\pgfusepath{stroke,fill}%
}%
\begin{pgfscope}%
\pgfsys@transformshift{5.801389in}{1.025304in}%
\pgfsys@useobject{currentmarker}{}%
\end{pgfscope}%
\end{pgfscope}%
\begin{pgfscope}%
\definecolor{textcolor}{rgb}{0.000000,0.000000,0.000000}%
\pgfsetstrokecolor{textcolor}%
\pgfsetfillcolor{textcolor}%
\pgftext[x=3.357118in,y=0.977110in,left,base]{\color{textcolor}\rmfamily\fontsize{10.000000}{12.000000}\selectfont 150}%
\end{pgfscope}%
\begin{pgfscope}%
\pgfpathrectangle{\pgfqpoint{3.662674in}{0.552778in}}{\pgfqpoint{2.138715in}{1.650000in}}%
\pgfusepath{clip}%
\pgfsetrectcap%
\pgfsetroundjoin%
\pgfsetlinewidth{0.803000pt}%
\definecolor{currentstroke}{rgb}{0.690196,0.690196,0.690196}%
\pgfsetstrokecolor{currentstroke}%
\pgfsetstrokeopacity{0.800000}%
\pgfsetdash{}{0pt}%
\pgfpathmoveto{\pgfqpoint{3.662674in}{1.422831in}}%
\pgfpathlineto{\pgfqpoint{5.801389in}{1.422831in}}%
\pgfusepath{stroke}%
\end{pgfscope}%
\begin{pgfscope}%
\pgfsetbuttcap%
\pgfsetroundjoin%
\definecolor{currentfill}{rgb}{0.000000,0.000000,0.000000}%
\pgfsetfillcolor{currentfill}%
\pgfsetlinewidth{0.803000pt}%
\definecolor{currentstroke}{rgb}{0.000000,0.000000,0.000000}%
\pgfsetstrokecolor{currentstroke}%
\pgfsetdash{}{0pt}%
\pgfsys@defobject{currentmarker}{\pgfqpoint{-0.048611in}{0.000000in}}{\pgfqpoint{0.000000in}{0.000000in}}{%
\pgfpathmoveto{\pgfqpoint{0.000000in}{0.000000in}}%
\pgfpathlineto{\pgfqpoint{-0.048611in}{0.000000in}}%
\pgfusepath{stroke,fill}%
}%
\begin{pgfscope}%
\pgfsys@transformshift{3.662674in}{1.422831in}%
\pgfsys@useobject{currentmarker}{}%
\end{pgfscope}%
\end{pgfscope}%
\begin{pgfscope}%
\pgfsetbuttcap%
\pgfsetroundjoin%
\definecolor{currentfill}{rgb}{0.000000,0.000000,0.000000}%
\pgfsetfillcolor{currentfill}%
\pgfsetlinewidth{0.803000pt}%
\definecolor{currentstroke}{rgb}{0.000000,0.000000,0.000000}%
\pgfsetstrokecolor{currentstroke}%
\pgfsetdash{}{0pt}%
\pgfsys@defobject{currentmarker}{\pgfqpoint{0.000000in}{0.000000in}}{\pgfqpoint{0.048611in}{0.000000in}}{%
\pgfpathmoveto{\pgfqpoint{0.000000in}{0.000000in}}%
\pgfpathlineto{\pgfqpoint{0.048611in}{0.000000in}}%
\pgfusepath{stroke,fill}%
}%
\begin{pgfscope}%
\pgfsys@transformshift{5.801389in}{1.422831in}%
\pgfsys@useobject{currentmarker}{}%
\end{pgfscope}%
\end{pgfscope}%
\begin{pgfscope}%
\definecolor{textcolor}{rgb}{0.000000,0.000000,0.000000}%
\pgfsetstrokecolor{textcolor}%
\pgfsetfillcolor{textcolor}%
\pgftext[x=3.357118in,y=1.374636in,left,base]{\color{textcolor}\rmfamily\fontsize{10.000000}{12.000000}\selectfont 300}%
\end{pgfscope}%
\begin{pgfscope}%
\pgfpathrectangle{\pgfqpoint{3.662674in}{0.552778in}}{\pgfqpoint{2.138715in}{1.650000in}}%
\pgfusepath{clip}%
\pgfsetrectcap%
\pgfsetroundjoin%
\pgfsetlinewidth{0.803000pt}%
\definecolor{currentstroke}{rgb}{0.690196,0.690196,0.690196}%
\pgfsetstrokecolor{currentstroke}%
\pgfsetstrokeopacity{0.800000}%
\pgfsetdash{}{0pt}%
\pgfpathmoveto{\pgfqpoint{3.662674in}{1.820357in}}%
\pgfpathlineto{\pgfqpoint{5.801389in}{1.820357in}}%
\pgfusepath{stroke}%
\end{pgfscope}%
\begin{pgfscope}%
\pgfsetbuttcap%
\pgfsetroundjoin%
\definecolor{currentfill}{rgb}{0.000000,0.000000,0.000000}%
\pgfsetfillcolor{currentfill}%
\pgfsetlinewidth{0.803000pt}%
\definecolor{currentstroke}{rgb}{0.000000,0.000000,0.000000}%
\pgfsetstrokecolor{currentstroke}%
\pgfsetdash{}{0pt}%
\pgfsys@defobject{currentmarker}{\pgfqpoint{-0.048611in}{0.000000in}}{\pgfqpoint{0.000000in}{0.000000in}}{%
\pgfpathmoveto{\pgfqpoint{0.000000in}{0.000000in}}%
\pgfpathlineto{\pgfqpoint{-0.048611in}{0.000000in}}%
\pgfusepath{stroke,fill}%
}%
\begin{pgfscope}%
\pgfsys@transformshift{3.662674in}{1.820357in}%
\pgfsys@useobject{currentmarker}{}%
\end{pgfscope}%
\end{pgfscope}%
\begin{pgfscope}%
\pgfsetbuttcap%
\pgfsetroundjoin%
\definecolor{currentfill}{rgb}{0.000000,0.000000,0.000000}%
\pgfsetfillcolor{currentfill}%
\pgfsetlinewidth{0.803000pt}%
\definecolor{currentstroke}{rgb}{0.000000,0.000000,0.000000}%
\pgfsetstrokecolor{currentstroke}%
\pgfsetdash{}{0pt}%
\pgfsys@defobject{currentmarker}{\pgfqpoint{0.000000in}{0.000000in}}{\pgfqpoint{0.048611in}{0.000000in}}{%
\pgfpathmoveto{\pgfqpoint{0.000000in}{0.000000in}}%
\pgfpathlineto{\pgfqpoint{0.048611in}{0.000000in}}%
\pgfusepath{stroke,fill}%
}%
\begin{pgfscope}%
\pgfsys@transformshift{5.801389in}{1.820357in}%
\pgfsys@useobject{currentmarker}{}%
\end{pgfscope}%
\end{pgfscope}%
\begin{pgfscope}%
\definecolor{textcolor}{rgb}{0.000000,0.000000,0.000000}%
\pgfsetstrokecolor{textcolor}%
\pgfsetfillcolor{textcolor}%
\pgftext[x=3.357118in,y=1.772163in,left,base]{\color{textcolor}\rmfamily\fontsize{10.000000}{12.000000}\selectfont 450}%
\end{pgfscope}%
\begin{pgfscope}%
\pgfpathrectangle{\pgfqpoint{3.662674in}{0.552778in}}{\pgfqpoint{2.138715in}{1.650000in}}%
\pgfusepath{clip}%
\pgfsetrectcap%
\pgfsetroundjoin%
\pgfsetlinewidth{0.803000pt}%
\definecolor{currentstroke}{rgb}{0.690196,0.690196,0.690196}%
\pgfsetstrokecolor{currentstroke}%
\pgfsetstrokeopacity{0.300000}%
\pgfsetdash{}{0pt}%
\pgfpathmoveto{\pgfqpoint{3.662674in}{0.588025in}}%
\pgfpathlineto{\pgfqpoint{5.801389in}{0.588025in}}%
\pgfusepath{stroke}%
\end{pgfscope}%
\begin{pgfscope}%
\pgfsetbuttcap%
\pgfsetroundjoin%
\definecolor{currentfill}{rgb}{0.000000,0.000000,0.000000}%
\pgfsetfillcolor{currentfill}%
\pgfsetlinewidth{0.602250pt}%
\definecolor{currentstroke}{rgb}{0.000000,0.000000,0.000000}%
\pgfsetstrokecolor{currentstroke}%
\pgfsetdash{}{0pt}%
\pgfsys@defobject{currentmarker}{\pgfqpoint{-0.027778in}{0.000000in}}{\pgfqpoint{0.000000in}{0.000000in}}{%
\pgfpathmoveto{\pgfqpoint{0.000000in}{0.000000in}}%
\pgfpathlineto{\pgfqpoint{-0.027778in}{0.000000in}}%
\pgfusepath{stroke,fill}%
}%
\begin{pgfscope}%
\pgfsys@transformshift{3.662674in}{0.588025in}%
\pgfsys@useobject{currentmarker}{}%
\end{pgfscope}%
\end{pgfscope}%
\begin{pgfscope}%
\pgfsetbuttcap%
\pgfsetroundjoin%
\definecolor{currentfill}{rgb}{0.000000,0.000000,0.000000}%
\pgfsetfillcolor{currentfill}%
\pgfsetlinewidth{0.602250pt}%
\definecolor{currentstroke}{rgb}{0.000000,0.000000,0.000000}%
\pgfsetstrokecolor{currentstroke}%
\pgfsetdash{}{0pt}%
\pgfsys@defobject{currentmarker}{\pgfqpoint{0.000000in}{0.000000in}}{\pgfqpoint{0.027778in}{0.000000in}}{%
\pgfpathmoveto{\pgfqpoint{0.000000in}{0.000000in}}%
\pgfpathlineto{\pgfqpoint{0.027778in}{0.000000in}}%
\pgfusepath{stroke,fill}%
}%
\begin{pgfscope}%
\pgfsys@transformshift{5.801389in}{0.588025in}%
\pgfsys@useobject{currentmarker}{}%
\end{pgfscope}%
\end{pgfscope}%
\begin{pgfscope}%
\pgfpathrectangle{\pgfqpoint{3.662674in}{0.552778in}}{\pgfqpoint{2.138715in}{1.650000in}}%
\pgfusepath{clip}%
\pgfsetrectcap%
\pgfsetroundjoin%
\pgfsetlinewidth{0.803000pt}%
\definecolor{currentstroke}{rgb}{0.690196,0.690196,0.690196}%
\pgfsetstrokecolor{currentstroke}%
\pgfsetstrokeopacity{0.300000}%
\pgfsetdash{}{0pt}%
\pgfpathmoveto{\pgfqpoint{3.662674in}{0.667530in}}%
\pgfpathlineto{\pgfqpoint{5.801389in}{0.667530in}}%
\pgfusepath{stroke}%
\end{pgfscope}%
\begin{pgfscope}%
\pgfsetbuttcap%
\pgfsetroundjoin%
\definecolor{currentfill}{rgb}{0.000000,0.000000,0.000000}%
\pgfsetfillcolor{currentfill}%
\pgfsetlinewidth{0.602250pt}%
\definecolor{currentstroke}{rgb}{0.000000,0.000000,0.000000}%
\pgfsetstrokecolor{currentstroke}%
\pgfsetdash{}{0pt}%
\pgfsys@defobject{currentmarker}{\pgfqpoint{-0.027778in}{0.000000in}}{\pgfqpoint{0.000000in}{0.000000in}}{%
\pgfpathmoveto{\pgfqpoint{0.000000in}{0.000000in}}%
\pgfpathlineto{\pgfqpoint{-0.027778in}{0.000000in}}%
\pgfusepath{stroke,fill}%
}%
\begin{pgfscope}%
\pgfsys@transformshift{3.662674in}{0.667530in}%
\pgfsys@useobject{currentmarker}{}%
\end{pgfscope}%
\end{pgfscope}%
\begin{pgfscope}%
\pgfsetbuttcap%
\pgfsetroundjoin%
\definecolor{currentfill}{rgb}{0.000000,0.000000,0.000000}%
\pgfsetfillcolor{currentfill}%
\pgfsetlinewidth{0.602250pt}%
\definecolor{currentstroke}{rgb}{0.000000,0.000000,0.000000}%
\pgfsetstrokecolor{currentstroke}%
\pgfsetdash{}{0pt}%
\pgfsys@defobject{currentmarker}{\pgfqpoint{0.000000in}{0.000000in}}{\pgfqpoint{0.027778in}{0.000000in}}{%
\pgfpathmoveto{\pgfqpoint{0.000000in}{0.000000in}}%
\pgfpathlineto{\pgfqpoint{0.027778in}{0.000000in}}%
\pgfusepath{stroke,fill}%
}%
\begin{pgfscope}%
\pgfsys@transformshift{5.801389in}{0.667530in}%
\pgfsys@useobject{currentmarker}{}%
\end{pgfscope}%
\end{pgfscope}%
\begin{pgfscope}%
\pgfpathrectangle{\pgfqpoint{3.662674in}{0.552778in}}{\pgfqpoint{2.138715in}{1.650000in}}%
\pgfusepath{clip}%
\pgfsetrectcap%
\pgfsetroundjoin%
\pgfsetlinewidth{0.803000pt}%
\definecolor{currentstroke}{rgb}{0.690196,0.690196,0.690196}%
\pgfsetstrokecolor{currentstroke}%
\pgfsetstrokeopacity{0.300000}%
\pgfsetdash{}{0pt}%
\pgfpathmoveto{\pgfqpoint{3.662674in}{0.707283in}}%
\pgfpathlineto{\pgfqpoint{5.801389in}{0.707283in}}%
\pgfusepath{stroke}%
\end{pgfscope}%
\begin{pgfscope}%
\pgfsetbuttcap%
\pgfsetroundjoin%
\definecolor{currentfill}{rgb}{0.000000,0.000000,0.000000}%
\pgfsetfillcolor{currentfill}%
\pgfsetlinewidth{0.602250pt}%
\definecolor{currentstroke}{rgb}{0.000000,0.000000,0.000000}%
\pgfsetstrokecolor{currentstroke}%
\pgfsetdash{}{0pt}%
\pgfsys@defobject{currentmarker}{\pgfqpoint{-0.027778in}{0.000000in}}{\pgfqpoint{0.000000in}{0.000000in}}{%
\pgfpathmoveto{\pgfqpoint{0.000000in}{0.000000in}}%
\pgfpathlineto{\pgfqpoint{-0.027778in}{0.000000in}}%
\pgfusepath{stroke,fill}%
}%
\begin{pgfscope}%
\pgfsys@transformshift{3.662674in}{0.707283in}%
\pgfsys@useobject{currentmarker}{}%
\end{pgfscope}%
\end{pgfscope}%
\begin{pgfscope}%
\pgfsetbuttcap%
\pgfsetroundjoin%
\definecolor{currentfill}{rgb}{0.000000,0.000000,0.000000}%
\pgfsetfillcolor{currentfill}%
\pgfsetlinewidth{0.602250pt}%
\definecolor{currentstroke}{rgb}{0.000000,0.000000,0.000000}%
\pgfsetstrokecolor{currentstroke}%
\pgfsetdash{}{0pt}%
\pgfsys@defobject{currentmarker}{\pgfqpoint{0.000000in}{0.000000in}}{\pgfqpoint{0.027778in}{0.000000in}}{%
\pgfpathmoveto{\pgfqpoint{0.000000in}{0.000000in}}%
\pgfpathlineto{\pgfqpoint{0.027778in}{0.000000in}}%
\pgfusepath{stroke,fill}%
}%
\begin{pgfscope}%
\pgfsys@transformshift{5.801389in}{0.707283in}%
\pgfsys@useobject{currentmarker}{}%
\end{pgfscope}%
\end{pgfscope}%
\begin{pgfscope}%
\pgfpathrectangle{\pgfqpoint{3.662674in}{0.552778in}}{\pgfqpoint{2.138715in}{1.650000in}}%
\pgfusepath{clip}%
\pgfsetrectcap%
\pgfsetroundjoin%
\pgfsetlinewidth{0.803000pt}%
\definecolor{currentstroke}{rgb}{0.690196,0.690196,0.690196}%
\pgfsetstrokecolor{currentstroke}%
\pgfsetstrokeopacity{0.300000}%
\pgfsetdash{}{0pt}%
\pgfpathmoveto{\pgfqpoint{3.662674in}{0.747036in}}%
\pgfpathlineto{\pgfqpoint{5.801389in}{0.747036in}}%
\pgfusepath{stroke}%
\end{pgfscope}%
\begin{pgfscope}%
\pgfsetbuttcap%
\pgfsetroundjoin%
\definecolor{currentfill}{rgb}{0.000000,0.000000,0.000000}%
\pgfsetfillcolor{currentfill}%
\pgfsetlinewidth{0.602250pt}%
\definecolor{currentstroke}{rgb}{0.000000,0.000000,0.000000}%
\pgfsetstrokecolor{currentstroke}%
\pgfsetdash{}{0pt}%
\pgfsys@defobject{currentmarker}{\pgfqpoint{-0.027778in}{0.000000in}}{\pgfqpoint{0.000000in}{0.000000in}}{%
\pgfpathmoveto{\pgfqpoint{0.000000in}{0.000000in}}%
\pgfpathlineto{\pgfqpoint{-0.027778in}{0.000000in}}%
\pgfusepath{stroke,fill}%
}%
\begin{pgfscope}%
\pgfsys@transformshift{3.662674in}{0.747036in}%
\pgfsys@useobject{currentmarker}{}%
\end{pgfscope}%
\end{pgfscope}%
\begin{pgfscope}%
\pgfsetbuttcap%
\pgfsetroundjoin%
\definecolor{currentfill}{rgb}{0.000000,0.000000,0.000000}%
\pgfsetfillcolor{currentfill}%
\pgfsetlinewidth{0.602250pt}%
\definecolor{currentstroke}{rgb}{0.000000,0.000000,0.000000}%
\pgfsetstrokecolor{currentstroke}%
\pgfsetdash{}{0pt}%
\pgfsys@defobject{currentmarker}{\pgfqpoint{0.000000in}{0.000000in}}{\pgfqpoint{0.027778in}{0.000000in}}{%
\pgfpathmoveto{\pgfqpoint{0.000000in}{0.000000in}}%
\pgfpathlineto{\pgfqpoint{0.027778in}{0.000000in}}%
\pgfusepath{stroke,fill}%
}%
\begin{pgfscope}%
\pgfsys@transformshift{5.801389in}{0.747036in}%
\pgfsys@useobject{currentmarker}{}%
\end{pgfscope}%
\end{pgfscope}%
\begin{pgfscope}%
\pgfpathrectangle{\pgfqpoint{3.662674in}{0.552778in}}{\pgfqpoint{2.138715in}{1.650000in}}%
\pgfusepath{clip}%
\pgfsetrectcap%
\pgfsetroundjoin%
\pgfsetlinewidth{0.803000pt}%
\definecolor{currentstroke}{rgb}{0.690196,0.690196,0.690196}%
\pgfsetstrokecolor{currentstroke}%
\pgfsetstrokeopacity{0.300000}%
\pgfsetdash{}{0pt}%
\pgfpathmoveto{\pgfqpoint{3.662674in}{0.786788in}}%
\pgfpathlineto{\pgfqpoint{5.801389in}{0.786788in}}%
\pgfusepath{stroke}%
\end{pgfscope}%
\begin{pgfscope}%
\pgfsetbuttcap%
\pgfsetroundjoin%
\definecolor{currentfill}{rgb}{0.000000,0.000000,0.000000}%
\pgfsetfillcolor{currentfill}%
\pgfsetlinewidth{0.602250pt}%
\definecolor{currentstroke}{rgb}{0.000000,0.000000,0.000000}%
\pgfsetstrokecolor{currentstroke}%
\pgfsetdash{}{0pt}%
\pgfsys@defobject{currentmarker}{\pgfqpoint{-0.027778in}{0.000000in}}{\pgfqpoint{0.000000in}{0.000000in}}{%
\pgfpathmoveto{\pgfqpoint{0.000000in}{0.000000in}}%
\pgfpathlineto{\pgfqpoint{-0.027778in}{0.000000in}}%
\pgfusepath{stroke,fill}%
}%
\begin{pgfscope}%
\pgfsys@transformshift{3.662674in}{0.786788in}%
\pgfsys@useobject{currentmarker}{}%
\end{pgfscope}%
\end{pgfscope}%
\begin{pgfscope}%
\pgfsetbuttcap%
\pgfsetroundjoin%
\definecolor{currentfill}{rgb}{0.000000,0.000000,0.000000}%
\pgfsetfillcolor{currentfill}%
\pgfsetlinewidth{0.602250pt}%
\definecolor{currentstroke}{rgb}{0.000000,0.000000,0.000000}%
\pgfsetstrokecolor{currentstroke}%
\pgfsetdash{}{0pt}%
\pgfsys@defobject{currentmarker}{\pgfqpoint{0.000000in}{0.000000in}}{\pgfqpoint{0.027778in}{0.000000in}}{%
\pgfpathmoveto{\pgfqpoint{0.000000in}{0.000000in}}%
\pgfpathlineto{\pgfqpoint{0.027778in}{0.000000in}}%
\pgfusepath{stroke,fill}%
}%
\begin{pgfscope}%
\pgfsys@transformshift{5.801389in}{0.786788in}%
\pgfsys@useobject{currentmarker}{}%
\end{pgfscope}%
\end{pgfscope}%
\begin{pgfscope}%
\pgfpathrectangle{\pgfqpoint{3.662674in}{0.552778in}}{\pgfqpoint{2.138715in}{1.650000in}}%
\pgfusepath{clip}%
\pgfsetrectcap%
\pgfsetroundjoin%
\pgfsetlinewidth{0.803000pt}%
\definecolor{currentstroke}{rgb}{0.690196,0.690196,0.690196}%
\pgfsetstrokecolor{currentstroke}%
\pgfsetstrokeopacity{0.300000}%
\pgfsetdash{}{0pt}%
\pgfpathmoveto{\pgfqpoint{3.662674in}{0.826541in}}%
\pgfpathlineto{\pgfqpoint{5.801389in}{0.826541in}}%
\pgfusepath{stroke}%
\end{pgfscope}%
\begin{pgfscope}%
\pgfsetbuttcap%
\pgfsetroundjoin%
\definecolor{currentfill}{rgb}{0.000000,0.000000,0.000000}%
\pgfsetfillcolor{currentfill}%
\pgfsetlinewidth{0.602250pt}%
\definecolor{currentstroke}{rgb}{0.000000,0.000000,0.000000}%
\pgfsetstrokecolor{currentstroke}%
\pgfsetdash{}{0pt}%
\pgfsys@defobject{currentmarker}{\pgfqpoint{-0.027778in}{0.000000in}}{\pgfqpoint{0.000000in}{0.000000in}}{%
\pgfpathmoveto{\pgfqpoint{0.000000in}{0.000000in}}%
\pgfpathlineto{\pgfqpoint{-0.027778in}{0.000000in}}%
\pgfusepath{stroke,fill}%
}%
\begin{pgfscope}%
\pgfsys@transformshift{3.662674in}{0.826541in}%
\pgfsys@useobject{currentmarker}{}%
\end{pgfscope}%
\end{pgfscope}%
\begin{pgfscope}%
\pgfsetbuttcap%
\pgfsetroundjoin%
\definecolor{currentfill}{rgb}{0.000000,0.000000,0.000000}%
\pgfsetfillcolor{currentfill}%
\pgfsetlinewidth{0.602250pt}%
\definecolor{currentstroke}{rgb}{0.000000,0.000000,0.000000}%
\pgfsetstrokecolor{currentstroke}%
\pgfsetdash{}{0pt}%
\pgfsys@defobject{currentmarker}{\pgfqpoint{0.000000in}{0.000000in}}{\pgfqpoint{0.027778in}{0.000000in}}{%
\pgfpathmoveto{\pgfqpoint{0.000000in}{0.000000in}}%
\pgfpathlineto{\pgfqpoint{0.027778in}{0.000000in}}%
\pgfusepath{stroke,fill}%
}%
\begin{pgfscope}%
\pgfsys@transformshift{5.801389in}{0.826541in}%
\pgfsys@useobject{currentmarker}{}%
\end{pgfscope}%
\end{pgfscope}%
\begin{pgfscope}%
\pgfpathrectangle{\pgfqpoint{3.662674in}{0.552778in}}{\pgfqpoint{2.138715in}{1.650000in}}%
\pgfusepath{clip}%
\pgfsetrectcap%
\pgfsetroundjoin%
\pgfsetlinewidth{0.803000pt}%
\definecolor{currentstroke}{rgb}{0.690196,0.690196,0.690196}%
\pgfsetstrokecolor{currentstroke}%
\pgfsetstrokeopacity{0.300000}%
\pgfsetdash{}{0pt}%
\pgfpathmoveto{\pgfqpoint{3.662674in}{0.866294in}}%
\pgfpathlineto{\pgfqpoint{5.801389in}{0.866294in}}%
\pgfusepath{stroke}%
\end{pgfscope}%
\begin{pgfscope}%
\pgfsetbuttcap%
\pgfsetroundjoin%
\definecolor{currentfill}{rgb}{0.000000,0.000000,0.000000}%
\pgfsetfillcolor{currentfill}%
\pgfsetlinewidth{0.602250pt}%
\definecolor{currentstroke}{rgb}{0.000000,0.000000,0.000000}%
\pgfsetstrokecolor{currentstroke}%
\pgfsetdash{}{0pt}%
\pgfsys@defobject{currentmarker}{\pgfqpoint{-0.027778in}{0.000000in}}{\pgfqpoint{0.000000in}{0.000000in}}{%
\pgfpathmoveto{\pgfqpoint{0.000000in}{0.000000in}}%
\pgfpathlineto{\pgfqpoint{-0.027778in}{0.000000in}}%
\pgfusepath{stroke,fill}%
}%
\begin{pgfscope}%
\pgfsys@transformshift{3.662674in}{0.866294in}%
\pgfsys@useobject{currentmarker}{}%
\end{pgfscope}%
\end{pgfscope}%
\begin{pgfscope}%
\pgfsetbuttcap%
\pgfsetroundjoin%
\definecolor{currentfill}{rgb}{0.000000,0.000000,0.000000}%
\pgfsetfillcolor{currentfill}%
\pgfsetlinewidth{0.602250pt}%
\definecolor{currentstroke}{rgb}{0.000000,0.000000,0.000000}%
\pgfsetstrokecolor{currentstroke}%
\pgfsetdash{}{0pt}%
\pgfsys@defobject{currentmarker}{\pgfqpoint{0.000000in}{0.000000in}}{\pgfqpoint{0.027778in}{0.000000in}}{%
\pgfpathmoveto{\pgfqpoint{0.000000in}{0.000000in}}%
\pgfpathlineto{\pgfqpoint{0.027778in}{0.000000in}}%
\pgfusepath{stroke,fill}%
}%
\begin{pgfscope}%
\pgfsys@transformshift{5.801389in}{0.866294in}%
\pgfsys@useobject{currentmarker}{}%
\end{pgfscope}%
\end{pgfscope}%
\begin{pgfscope}%
\pgfpathrectangle{\pgfqpoint{3.662674in}{0.552778in}}{\pgfqpoint{2.138715in}{1.650000in}}%
\pgfusepath{clip}%
\pgfsetrectcap%
\pgfsetroundjoin%
\pgfsetlinewidth{0.803000pt}%
\definecolor{currentstroke}{rgb}{0.690196,0.690196,0.690196}%
\pgfsetstrokecolor{currentstroke}%
\pgfsetstrokeopacity{0.300000}%
\pgfsetdash{}{0pt}%
\pgfpathmoveto{\pgfqpoint{3.662674in}{0.906046in}}%
\pgfpathlineto{\pgfqpoint{5.801389in}{0.906046in}}%
\pgfusepath{stroke}%
\end{pgfscope}%
\begin{pgfscope}%
\pgfsetbuttcap%
\pgfsetroundjoin%
\definecolor{currentfill}{rgb}{0.000000,0.000000,0.000000}%
\pgfsetfillcolor{currentfill}%
\pgfsetlinewidth{0.602250pt}%
\definecolor{currentstroke}{rgb}{0.000000,0.000000,0.000000}%
\pgfsetstrokecolor{currentstroke}%
\pgfsetdash{}{0pt}%
\pgfsys@defobject{currentmarker}{\pgfqpoint{-0.027778in}{0.000000in}}{\pgfqpoint{0.000000in}{0.000000in}}{%
\pgfpathmoveto{\pgfqpoint{0.000000in}{0.000000in}}%
\pgfpathlineto{\pgfqpoint{-0.027778in}{0.000000in}}%
\pgfusepath{stroke,fill}%
}%
\begin{pgfscope}%
\pgfsys@transformshift{3.662674in}{0.906046in}%
\pgfsys@useobject{currentmarker}{}%
\end{pgfscope}%
\end{pgfscope}%
\begin{pgfscope}%
\pgfsetbuttcap%
\pgfsetroundjoin%
\definecolor{currentfill}{rgb}{0.000000,0.000000,0.000000}%
\pgfsetfillcolor{currentfill}%
\pgfsetlinewidth{0.602250pt}%
\definecolor{currentstroke}{rgb}{0.000000,0.000000,0.000000}%
\pgfsetstrokecolor{currentstroke}%
\pgfsetdash{}{0pt}%
\pgfsys@defobject{currentmarker}{\pgfqpoint{0.000000in}{0.000000in}}{\pgfqpoint{0.027778in}{0.000000in}}{%
\pgfpathmoveto{\pgfqpoint{0.000000in}{0.000000in}}%
\pgfpathlineto{\pgfqpoint{0.027778in}{0.000000in}}%
\pgfusepath{stroke,fill}%
}%
\begin{pgfscope}%
\pgfsys@transformshift{5.801389in}{0.906046in}%
\pgfsys@useobject{currentmarker}{}%
\end{pgfscope}%
\end{pgfscope}%
\begin{pgfscope}%
\pgfpathrectangle{\pgfqpoint{3.662674in}{0.552778in}}{\pgfqpoint{2.138715in}{1.650000in}}%
\pgfusepath{clip}%
\pgfsetrectcap%
\pgfsetroundjoin%
\pgfsetlinewidth{0.803000pt}%
\definecolor{currentstroke}{rgb}{0.690196,0.690196,0.690196}%
\pgfsetstrokecolor{currentstroke}%
\pgfsetstrokeopacity{0.300000}%
\pgfsetdash{}{0pt}%
\pgfpathmoveto{\pgfqpoint{3.662674in}{0.945799in}}%
\pgfpathlineto{\pgfqpoint{5.801389in}{0.945799in}}%
\pgfusepath{stroke}%
\end{pgfscope}%
\begin{pgfscope}%
\pgfsetbuttcap%
\pgfsetroundjoin%
\definecolor{currentfill}{rgb}{0.000000,0.000000,0.000000}%
\pgfsetfillcolor{currentfill}%
\pgfsetlinewidth{0.602250pt}%
\definecolor{currentstroke}{rgb}{0.000000,0.000000,0.000000}%
\pgfsetstrokecolor{currentstroke}%
\pgfsetdash{}{0pt}%
\pgfsys@defobject{currentmarker}{\pgfqpoint{-0.027778in}{0.000000in}}{\pgfqpoint{0.000000in}{0.000000in}}{%
\pgfpathmoveto{\pgfqpoint{0.000000in}{0.000000in}}%
\pgfpathlineto{\pgfqpoint{-0.027778in}{0.000000in}}%
\pgfusepath{stroke,fill}%
}%
\begin{pgfscope}%
\pgfsys@transformshift{3.662674in}{0.945799in}%
\pgfsys@useobject{currentmarker}{}%
\end{pgfscope}%
\end{pgfscope}%
\begin{pgfscope}%
\pgfsetbuttcap%
\pgfsetroundjoin%
\definecolor{currentfill}{rgb}{0.000000,0.000000,0.000000}%
\pgfsetfillcolor{currentfill}%
\pgfsetlinewidth{0.602250pt}%
\definecolor{currentstroke}{rgb}{0.000000,0.000000,0.000000}%
\pgfsetstrokecolor{currentstroke}%
\pgfsetdash{}{0pt}%
\pgfsys@defobject{currentmarker}{\pgfqpoint{0.000000in}{0.000000in}}{\pgfqpoint{0.027778in}{0.000000in}}{%
\pgfpathmoveto{\pgfqpoint{0.000000in}{0.000000in}}%
\pgfpathlineto{\pgfqpoint{0.027778in}{0.000000in}}%
\pgfusepath{stroke,fill}%
}%
\begin{pgfscope}%
\pgfsys@transformshift{5.801389in}{0.945799in}%
\pgfsys@useobject{currentmarker}{}%
\end{pgfscope}%
\end{pgfscope}%
\begin{pgfscope}%
\pgfpathrectangle{\pgfqpoint{3.662674in}{0.552778in}}{\pgfqpoint{2.138715in}{1.650000in}}%
\pgfusepath{clip}%
\pgfsetrectcap%
\pgfsetroundjoin%
\pgfsetlinewidth{0.803000pt}%
\definecolor{currentstroke}{rgb}{0.690196,0.690196,0.690196}%
\pgfsetstrokecolor{currentstroke}%
\pgfsetstrokeopacity{0.300000}%
\pgfsetdash{}{0pt}%
\pgfpathmoveto{\pgfqpoint{3.662674in}{0.985552in}}%
\pgfpathlineto{\pgfqpoint{5.801389in}{0.985552in}}%
\pgfusepath{stroke}%
\end{pgfscope}%
\begin{pgfscope}%
\pgfsetbuttcap%
\pgfsetroundjoin%
\definecolor{currentfill}{rgb}{0.000000,0.000000,0.000000}%
\pgfsetfillcolor{currentfill}%
\pgfsetlinewidth{0.602250pt}%
\definecolor{currentstroke}{rgb}{0.000000,0.000000,0.000000}%
\pgfsetstrokecolor{currentstroke}%
\pgfsetdash{}{0pt}%
\pgfsys@defobject{currentmarker}{\pgfqpoint{-0.027778in}{0.000000in}}{\pgfqpoint{0.000000in}{0.000000in}}{%
\pgfpathmoveto{\pgfqpoint{0.000000in}{0.000000in}}%
\pgfpathlineto{\pgfqpoint{-0.027778in}{0.000000in}}%
\pgfusepath{stroke,fill}%
}%
\begin{pgfscope}%
\pgfsys@transformshift{3.662674in}{0.985552in}%
\pgfsys@useobject{currentmarker}{}%
\end{pgfscope}%
\end{pgfscope}%
\begin{pgfscope}%
\pgfsetbuttcap%
\pgfsetroundjoin%
\definecolor{currentfill}{rgb}{0.000000,0.000000,0.000000}%
\pgfsetfillcolor{currentfill}%
\pgfsetlinewidth{0.602250pt}%
\definecolor{currentstroke}{rgb}{0.000000,0.000000,0.000000}%
\pgfsetstrokecolor{currentstroke}%
\pgfsetdash{}{0pt}%
\pgfsys@defobject{currentmarker}{\pgfqpoint{0.000000in}{0.000000in}}{\pgfqpoint{0.027778in}{0.000000in}}{%
\pgfpathmoveto{\pgfqpoint{0.000000in}{0.000000in}}%
\pgfpathlineto{\pgfqpoint{0.027778in}{0.000000in}}%
\pgfusepath{stroke,fill}%
}%
\begin{pgfscope}%
\pgfsys@transformshift{5.801389in}{0.985552in}%
\pgfsys@useobject{currentmarker}{}%
\end{pgfscope}%
\end{pgfscope}%
\begin{pgfscope}%
\pgfpathrectangle{\pgfqpoint{3.662674in}{0.552778in}}{\pgfqpoint{2.138715in}{1.650000in}}%
\pgfusepath{clip}%
\pgfsetrectcap%
\pgfsetroundjoin%
\pgfsetlinewidth{0.803000pt}%
\definecolor{currentstroke}{rgb}{0.690196,0.690196,0.690196}%
\pgfsetstrokecolor{currentstroke}%
\pgfsetstrokeopacity{0.300000}%
\pgfsetdash{}{0pt}%
\pgfpathmoveto{\pgfqpoint{3.662674in}{1.065057in}}%
\pgfpathlineto{\pgfqpoint{5.801389in}{1.065057in}}%
\pgfusepath{stroke}%
\end{pgfscope}%
\begin{pgfscope}%
\pgfsetbuttcap%
\pgfsetroundjoin%
\definecolor{currentfill}{rgb}{0.000000,0.000000,0.000000}%
\pgfsetfillcolor{currentfill}%
\pgfsetlinewidth{0.602250pt}%
\definecolor{currentstroke}{rgb}{0.000000,0.000000,0.000000}%
\pgfsetstrokecolor{currentstroke}%
\pgfsetdash{}{0pt}%
\pgfsys@defobject{currentmarker}{\pgfqpoint{-0.027778in}{0.000000in}}{\pgfqpoint{0.000000in}{0.000000in}}{%
\pgfpathmoveto{\pgfqpoint{0.000000in}{0.000000in}}%
\pgfpathlineto{\pgfqpoint{-0.027778in}{0.000000in}}%
\pgfusepath{stroke,fill}%
}%
\begin{pgfscope}%
\pgfsys@transformshift{3.662674in}{1.065057in}%
\pgfsys@useobject{currentmarker}{}%
\end{pgfscope}%
\end{pgfscope}%
\begin{pgfscope}%
\pgfsetbuttcap%
\pgfsetroundjoin%
\definecolor{currentfill}{rgb}{0.000000,0.000000,0.000000}%
\pgfsetfillcolor{currentfill}%
\pgfsetlinewidth{0.602250pt}%
\definecolor{currentstroke}{rgb}{0.000000,0.000000,0.000000}%
\pgfsetstrokecolor{currentstroke}%
\pgfsetdash{}{0pt}%
\pgfsys@defobject{currentmarker}{\pgfqpoint{0.000000in}{0.000000in}}{\pgfqpoint{0.027778in}{0.000000in}}{%
\pgfpathmoveto{\pgfqpoint{0.000000in}{0.000000in}}%
\pgfpathlineto{\pgfqpoint{0.027778in}{0.000000in}}%
\pgfusepath{stroke,fill}%
}%
\begin{pgfscope}%
\pgfsys@transformshift{5.801389in}{1.065057in}%
\pgfsys@useobject{currentmarker}{}%
\end{pgfscope}%
\end{pgfscope}%
\begin{pgfscope}%
\pgfpathrectangle{\pgfqpoint{3.662674in}{0.552778in}}{\pgfqpoint{2.138715in}{1.650000in}}%
\pgfusepath{clip}%
\pgfsetrectcap%
\pgfsetroundjoin%
\pgfsetlinewidth{0.803000pt}%
\definecolor{currentstroke}{rgb}{0.690196,0.690196,0.690196}%
\pgfsetstrokecolor{currentstroke}%
\pgfsetstrokeopacity{0.300000}%
\pgfsetdash{}{0pt}%
\pgfpathmoveto{\pgfqpoint{3.662674in}{1.104810in}}%
\pgfpathlineto{\pgfqpoint{5.801389in}{1.104810in}}%
\pgfusepath{stroke}%
\end{pgfscope}%
\begin{pgfscope}%
\pgfsetbuttcap%
\pgfsetroundjoin%
\definecolor{currentfill}{rgb}{0.000000,0.000000,0.000000}%
\pgfsetfillcolor{currentfill}%
\pgfsetlinewidth{0.602250pt}%
\definecolor{currentstroke}{rgb}{0.000000,0.000000,0.000000}%
\pgfsetstrokecolor{currentstroke}%
\pgfsetdash{}{0pt}%
\pgfsys@defobject{currentmarker}{\pgfqpoint{-0.027778in}{0.000000in}}{\pgfqpoint{0.000000in}{0.000000in}}{%
\pgfpathmoveto{\pgfqpoint{0.000000in}{0.000000in}}%
\pgfpathlineto{\pgfqpoint{-0.027778in}{0.000000in}}%
\pgfusepath{stroke,fill}%
}%
\begin{pgfscope}%
\pgfsys@transformshift{3.662674in}{1.104810in}%
\pgfsys@useobject{currentmarker}{}%
\end{pgfscope}%
\end{pgfscope}%
\begin{pgfscope}%
\pgfsetbuttcap%
\pgfsetroundjoin%
\definecolor{currentfill}{rgb}{0.000000,0.000000,0.000000}%
\pgfsetfillcolor{currentfill}%
\pgfsetlinewidth{0.602250pt}%
\definecolor{currentstroke}{rgb}{0.000000,0.000000,0.000000}%
\pgfsetstrokecolor{currentstroke}%
\pgfsetdash{}{0pt}%
\pgfsys@defobject{currentmarker}{\pgfqpoint{0.000000in}{0.000000in}}{\pgfqpoint{0.027778in}{0.000000in}}{%
\pgfpathmoveto{\pgfqpoint{0.000000in}{0.000000in}}%
\pgfpathlineto{\pgfqpoint{0.027778in}{0.000000in}}%
\pgfusepath{stroke,fill}%
}%
\begin{pgfscope}%
\pgfsys@transformshift{5.801389in}{1.104810in}%
\pgfsys@useobject{currentmarker}{}%
\end{pgfscope}%
\end{pgfscope}%
\begin{pgfscope}%
\pgfpathrectangle{\pgfqpoint{3.662674in}{0.552778in}}{\pgfqpoint{2.138715in}{1.650000in}}%
\pgfusepath{clip}%
\pgfsetrectcap%
\pgfsetroundjoin%
\pgfsetlinewidth{0.803000pt}%
\definecolor{currentstroke}{rgb}{0.690196,0.690196,0.690196}%
\pgfsetstrokecolor{currentstroke}%
\pgfsetstrokeopacity{0.300000}%
\pgfsetdash{}{0pt}%
\pgfpathmoveto{\pgfqpoint{3.662674in}{1.144562in}}%
\pgfpathlineto{\pgfqpoint{5.801389in}{1.144562in}}%
\pgfusepath{stroke}%
\end{pgfscope}%
\begin{pgfscope}%
\pgfsetbuttcap%
\pgfsetroundjoin%
\definecolor{currentfill}{rgb}{0.000000,0.000000,0.000000}%
\pgfsetfillcolor{currentfill}%
\pgfsetlinewidth{0.602250pt}%
\definecolor{currentstroke}{rgb}{0.000000,0.000000,0.000000}%
\pgfsetstrokecolor{currentstroke}%
\pgfsetdash{}{0pt}%
\pgfsys@defobject{currentmarker}{\pgfqpoint{-0.027778in}{0.000000in}}{\pgfqpoint{0.000000in}{0.000000in}}{%
\pgfpathmoveto{\pgfqpoint{0.000000in}{0.000000in}}%
\pgfpathlineto{\pgfqpoint{-0.027778in}{0.000000in}}%
\pgfusepath{stroke,fill}%
}%
\begin{pgfscope}%
\pgfsys@transformshift{3.662674in}{1.144562in}%
\pgfsys@useobject{currentmarker}{}%
\end{pgfscope}%
\end{pgfscope}%
\begin{pgfscope}%
\pgfsetbuttcap%
\pgfsetroundjoin%
\definecolor{currentfill}{rgb}{0.000000,0.000000,0.000000}%
\pgfsetfillcolor{currentfill}%
\pgfsetlinewidth{0.602250pt}%
\definecolor{currentstroke}{rgb}{0.000000,0.000000,0.000000}%
\pgfsetstrokecolor{currentstroke}%
\pgfsetdash{}{0pt}%
\pgfsys@defobject{currentmarker}{\pgfqpoint{0.000000in}{0.000000in}}{\pgfqpoint{0.027778in}{0.000000in}}{%
\pgfpathmoveto{\pgfqpoint{0.000000in}{0.000000in}}%
\pgfpathlineto{\pgfqpoint{0.027778in}{0.000000in}}%
\pgfusepath{stroke,fill}%
}%
\begin{pgfscope}%
\pgfsys@transformshift{5.801389in}{1.144562in}%
\pgfsys@useobject{currentmarker}{}%
\end{pgfscope}%
\end{pgfscope}%
\begin{pgfscope}%
\pgfpathrectangle{\pgfqpoint{3.662674in}{0.552778in}}{\pgfqpoint{2.138715in}{1.650000in}}%
\pgfusepath{clip}%
\pgfsetrectcap%
\pgfsetroundjoin%
\pgfsetlinewidth{0.803000pt}%
\definecolor{currentstroke}{rgb}{0.690196,0.690196,0.690196}%
\pgfsetstrokecolor{currentstroke}%
\pgfsetstrokeopacity{0.300000}%
\pgfsetdash{}{0pt}%
\pgfpathmoveto{\pgfqpoint{3.662674in}{1.184315in}}%
\pgfpathlineto{\pgfqpoint{5.801389in}{1.184315in}}%
\pgfusepath{stroke}%
\end{pgfscope}%
\begin{pgfscope}%
\pgfsetbuttcap%
\pgfsetroundjoin%
\definecolor{currentfill}{rgb}{0.000000,0.000000,0.000000}%
\pgfsetfillcolor{currentfill}%
\pgfsetlinewidth{0.602250pt}%
\definecolor{currentstroke}{rgb}{0.000000,0.000000,0.000000}%
\pgfsetstrokecolor{currentstroke}%
\pgfsetdash{}{0pt}%
\pgfsys@defobject{currentmarker}{\pgfqpoint{-0.027778in}{0.000000in}}{\pgfqpoint{0.000000in}{0.000000in}}{%
\pgfpathmoveto{\pgfqpoint{0.000000in}{0.000000in}}%
\pgfpathlineto{\pgfqpoint{-0.027778in}{0.000000in}}%
\pgfusepath{stroke,fill}%
}%
\begin{pgfscope}%
\pgfsys@transformshift{3.662674in}{1.184315in}%
\pgfsys@useobject{currentmarker}{}%
\end{pgfscope}%
\end{pgfscope}%
\begin{pgfscope}%
\pgfsetbuttcap%
\pgfsetroundjoin%
\definecolor{currentfill}{rgb}{0.000000,0.000000,0.000000}%
\pgfsetfillcolor{currentfill}%
\pgfsetlinewidth{0.602250pt}%
\definecolor{currentstroke}{rgb}{0.000000,0.000000,0.000000}%
\pgfsetstrokecolor{currentstroke}%
\pgfsetdash{}{0pt}%
\pgfsys@defobject{currentmarker}{\pgfqpoint{0.000000in}{0.000000in}}{\pgfqpoint{0.027778in}{0.000000in}}{%
\pgfpathmoveto{\pgfqpoint{0.000000in}{0.000000in}}%
\pgfpathlineto{\pgfqpoint{0.027778in}{0.000000in}}%
\pgfusepath{stroke,fill}%
}%
\begin{pgfscope}%
\pgfsys@transformshift{5.801389in}{1.184315in}%
\pgfsys@useobject{currentmarker}{}%
\end{pgfscope}%
\end{pgfscope}%
\begin{pgfscope}%
\pgfpathrectangle{\pgfqpoint{3.662674in}{0.552778in}}{\pgfqpoint{2.138715in}{1.650000in}}%
\pgfusepath{clip}%
\pgfsetrectcap%
\pgfsetroundjoin%
\pgfsetlinewidth{0.803000pt}%
\definecolor{currentstroke}{rgb}{0.690196,0.690196,0.690196}%
\pgfsetstrokecolor{currentstroke}%
\pgfsetstrokeopacity{0.300000}%
\pgfsetdash{}{0pt}%
\pgfpathmoveto{\pgfqpoint{3.662674in}{1.224068in}}%
\pgfpathlineto{\pgfqpoint{5.801389in}{1.224068in}}%
\pgfusepath{stroke}%
\end{pgfscope}%
\begin{pgfscope}%
\pgfsetbuttcap%
\pgfsetroundjoin%
\definecolor{currentfill}{rgb}{0.000000,0.000000,0.000000}%
\pgfsetfillcolor{currentfill}%
\pgfsetlinewidth{0.602250pt}%
\definecolor{currentstroke}{rgb}{0.000000,0.000000,0.000000}%
\pgfsetstrokecolor{currentstroke}%
\pgfsetdash{}{0pt}%
\pgfsys@defobject{currentmarker}{\pgfqpoint{-0.027778in}{0.000000in}}{\pgfqpoint{0.000000in}{0.000000in}}{%
\pgfpathmoveto{\pgfqpoint{0.000000in}{0.000000in}}%
\pgfpathlineto{\pgfqpoint{-0.027778in}{0.000000in}}%
\pgfusepath{stroke,fill}%
}%
\begin{pgfscope}%
\pgfsys@transformshift{3.662674in}{1.224068in}%
\pgfsys@useobject{currentmarker}{}%
\end{pgfscope}%
\end{pgfscope}%
\begin{pgfscope}%
\pgfsetbuttcap%
\pgfsetroundjoin%
\definecolor{currentfill}{rgb}{0.000000,0.000000,0.000000}%
\pgfsetfillcolor{currentfill}%
\pgfsetlinewidth{0.602250pt}%
\definecolor{currentstroke}{rgb}{0.000000,0.000000,0.000000}%
\pgfsetstrokecolor{currentstroke}%
\pgfsetdash{}{0pt}%
\pgfsys@defobject{currentmarker}{\pgfqpoint{0.000000in}{0.000000in}}{\pgfqpoint{0.027778in}{0.000000in}}{%
\pgfpathmoveto{\pgfqpoint{0.000000in}{0.000000in}}%
\pgfpathlineto{\pgfqpoint{0.027778in}{0.000000in}}%
\pgfusepath{stroke,fill}%
}%
\begin{pgfscope}%
\pgfsys@transformshift{5.801389in}{1.224068in}%
\pgfsys@useobject{currentmarker}{}%
\end{pgfscope}%
\end{pgfscope}%
\begin{pgfscope}%
\pgfpathrectangle{\pgfqpoint{3.662674in}{0.552778in}}{\pgfqpoint{2.138715in}{1.650000in}}%
\pgfusepath{clip}%
\pgfsetrectcap%
\pgfsetroundjoin%
\pgfsetlinewidth{0.803000pt}%
\definecolor{currentstroke}{rgb}{0.690196,0.690196,0.690196}%
\pgfsetstrokecolor{currentstroke}%
\pgfsetstrokeopacity{0.300000}%
\pgfsetdash{}{0pt}%
\pgfpathmoveto{\pgfqpoint{3.662674in}{1.263820in}}%
\pgfpathlineto{\pgfqpoint{5.801389in}{1.263820in}}%
\pgfusepath{stroke}%
\end{pgfscope}%
\begin{pgfscope}%
\pgfsetbuttcap%
\pgfsetroundjoin%
\definecolor{currentfill}{rgb}{0.000000,0.000000,0.000000}%
\pgfsetfillcolor{currentfill}%
\pgfsetlinewidth{0.602250pt}%
\definecolor{currentstroke}{rgb}{0.000000,0.000000,0.000000}%
\pgfsetstrokecolor{currentstroke}%
\pgfsetdash{}{0pt}%
\pgfsys@defobject{currentmarker}{\pgfqpoint{-0.027778in}{0.000000in}}{\pgfqpoint{0.000000in}{0.000000in}}{%
\pgfpathmoveto{\pgfqpoint{0.000000in}{0.000000in}}%
\pgfpathlineto{\pgfqpoint{-0.027778in}{0.000000in}}%
\pgfusepath{stroke,fill}%
}%
\begin{pgfscope}%
\pgfsys@transformshift{3.662674in}{1.263820in}%
\pgfsys@useobject{currentmarker}{}%
\end{pgfscope}%
\end{pgfscope}%
\begin{pgfscope}%
\pgfsetbuttcap%
\pgfsetroundjoin%
\definecolor{currentfill}{rgb}{0.000000,0.000000,0.000000}%
\pgfsetfillcolor{currentfill}%
\pgfsetlinewidth{0.602250pt}%
\definecolor{currentstroke}{rgb}{0.000000,0.000000,0.000000}%
\pgfsetstrokecolor{currentstroke}%
\pgfsetdash{}{0pt}%
\pgfsys@defobject{currentmarker}{\pgfqpoint{0.000000in}{0.000000in}}{\pgfqpoint{0.027778in}{0.000000in}}{%
\pgfpathmoveto{\pgfqpoint{0.000000in}{0.000000in}}%
\pgfpathlineto{\pgfqpoint{0.027778in}{0.000000in}}%
\pgfusepath{stroke,fill}%
}%
\begin{pgfscope}%
\pgfsys@transformshift{5.801389in}{1.263820in}%
\pgfsys@useobject{currentmarker}{}%
\end{pgfscope}%
\end{pgfscope}%
\begin{pgfscope}%
\pgfpathrectangle{\pgfqpoint{3.662674in}{0.552778in}}{\pgfqpoint{2.138715in}{1.650000in}}%
\pgfusepath{clip}%
\pgfsetrectcap%
\pgfsetroundjoin%
\pgfsetlinewidth{0.803000pt}%
\definecolor{currentstroke}{rgb}{0.690196,0.690196,0.690196}%
\pgfsetstrokecolor{currentstroke}%
\pgfsetstrokeopacity{0.300000}%
\pgfsetdash{}{0pt}%
\pgfpathmoveto{\pgfqpoint{3.662674in}{1.303573in}}%
\pgfpathlineto{\pgfqpoint{5.801389in}{1.303573in}}%
\pgfusepath{stroke}%
\end{pgfscope}%
\begin{pgfscope}%
\pgfsetbuttcap%
\pgfsetroundjoin%
\definecolor{currentfill}{rgb}{0.000000,0.000000,0.000000}%
\pgfsetfillcolor{currentfill}%
\pgfsetlinewidth{0.602250pt}%
\definecolor{currentstroke}{rgb}{0.000000,0.000000,0.000000}%
\pgfsetstrokecolor{currentstroke}%
\pgfsetdash{}{0pt}%
\pgfsys@defobject{currentmarker}{\pgfqpoint{-0.027778in}{0.000000in}}{\pgfqpoint{0.000000in}{0.000000in}}{%
\pgfpathmoveto{\pgfqpoint{0.000000in}{0.000000in}}%
\pgfpathlineto{\pgfqpoint{-0.027778in}{0.000000in}}%
\pgfusepath{stroke,fill}%
}%
\begin{pgfscope}%
\pgfsys@transformshift{3.662674in}{1.303573in}%
\pgfsys@useobject{currentmarker}{}%
\end{pgfscope}%
\end{pgfscope}%
\begin{pgfscope}%
\pgfsetbuttcap%
\pgfsetroundjoin%
\definecolor{currentfill}{rgb}{0.000000,0.000000,0.000000}%
\pgfsetfillcolor{currentfill}%
\pgfsetlinewidth{0.602250pt}%
\definecolor{currentstroke}{rgb}{0.000000,0.000000,0.000000}%
\pgfsetstrokecolor{currentstroke}%
\pgfsetdash{}{0pt}%
\pgfsys@defobject{currentmarker}{\pgfqpoint{0.000000in}{0.000000in}}{\pgfqpoint{0.027778in}{0.000000in}}{%
\pgfpathmoveto{\pgfqpoint{0.000000in}{0.000000in}}%
\pgfpathlineto{\pgfqpoint{0.027778in}{0.000000in}}%
\pgfusepath{stroke,fill}%
}%
\begin{pgfscope}%
\pgfsys@transformshift{5.801389in}{1.303573in}%
\pgfsys@useobject{currentmarker}{}%
\end{pgfscope}%
\end{pgfscope}%
\begin{pgfscope}%
\pgfpathrectangle{\pgfqpoint{3.662674in}{0.552778in}}{\pgfqpoint{2.138715in}{1.650000in}}%
\pgfusepath{clip}%
\pgfsetrectcap%
\pgfsetroundjoin%
\pgfsetlinewidth{0.803000pt}%
\definecolor{currentstroke}{rgb}{0.690196,0.690196,0.690196}%
\pgfsetstrokecolor{currentstroke}%
\pgfsetstrokeopacity{0.300000}%
\pgfsetdash{}{0pt}%
\pgfpathmoveto{\pgfqpoint{3.662674in}{1.343325in}}%
\pgfpathlineto{\pgfqpoint{5.801389in}{1.343325in}}%
\pgfusepath{stroke}%
\end{pgfscope}%
\begin{pgfscope}%
\pgfsetbuttcap%
\pgfsetroundjoin%
\definecolor{currentfill}{rgb}{0.000000,0.000000,0.000000}%
\pgfsetfillcolor{currentfill}%
\pgfsetlinewidth{0.602250pt}%
\definecolor{currentstroke}{rgb}{0.000000,0.000000,0.000000}%
\pgfsetstrokecolor{currentstroke}%
\pgfsetdash{}{0pt}%
\pgfsys@defobject{currentmarker}{\pgfqpoint{-0.027778in}{0.000000in}}{\pgfqpoint{0.000000in}{0.000000in}}{%
\pgfpathmoveto{\pgfqpoint{0.000000in}{0.000000in}}%
\pgfpathlineto{\pgfqpoint{-0.027778in}{0.000000in}}%
\pgfusepath{stroke,fill}%
}%
\begin{pgfscope}%
\pgfsys@transformshift{3.662674in}{1.343325in}%
\pgfsys@useobject{currentmarker}{}%
\end{pgfscope}%
\end{pgfscope}%
\begin{pgfscope}%
\pgfsetbuttcap%
\pgfsetroundjoin%
\definecolor{currentfill}{rgb}{0.000000,0.000000,0.000000}%
\pgfsetfillcolor{currentfill}%
\pgfsetlinewidth{0.602250pt}%
\definecolor{currentstroke}{rgb}{0.000000,0.000000,0.000000}%
\pgfsetstrokecolor{currentstroke}%
\pgfsetdash{}{0pt}%
\pgfsys@defobject{currentmarker}{\pgfqpoint{0.000000in}{0.000000in}}{\pgfqpoint{0.027778in}{0.000000in}}{%
\pgfpathmoveto{\pgfqpoint{0.000000in}{0.000000in}}%
\pgfpathlineto{\pgfqpoint{0.027778in}{0.000000in}}%
\pgfusepath{stroke,fill}%
}%
\begin{pgfscope}%
\pgfsys@transformshift{5.801389in}{1.343325in}%
\pgfsys@useobject{currentmarker}{}%
\end{pgfscope}%
\end{pgfscope}%
\begin{pgfscope}%
\pgfpathrectangle{\pgfqpoint{3.662674in}{0.552778in}}{\pgfqpoint{2.138715in}{1.650000in}}%
\pgfusepath{clip}%
\pgfsetrectcap%
\pgfsetroundjoin%
\pgfsetlinewidth{0.803000pt}%
\definecolor{currentstroke}{rgb}{0.690196,0.690196,0.690196}%
\pgfsetstrokecolor{currentstroke}%
\pgfsetstrokeopacity{0.300000}%
\pgfsetdash{}{0pt}%
\pgfpathmoveto{\pgfqpoint{3.662674in}{1.383078in}}%
\pgfpathlineto{\pgfqpoint{5.801389in}{1.383078in}}%
\pgfusepath{stroke}%
\end{pgfscope}%
\begin{pgfscope}%
\pgfsetbuttcap%
\pgfsetroundjoin%
\definecolor{currentfill}{rgb}{0.000000,0.000000,0.000000}%
\pgfsetfillcolor{currentfill}%
\pgfsetlinewidth{0.602250pt}%
\definecolor{currentstroke}{rgb}{0.000000,0.000000,0.000000}%
\pgfsetstrokecolor{currentstroke}%
\pgfsetdash{}{0pt}%
\pgfsys@defobject{currentmarker}{\pgfqpoint{-0.027778in}{0.000000in}}{\pgfqpoint{0.000000in}{0.000000in}}{%
\pgfpathmoveto{\pgfqpoint{0.000000in}{0.000000in}}%
\pgfpathlineto{\pgfqpoint{-0.027778in}{0.000000in}}%
\pgfusepath{stroke,fill}%
}%
\begin{pgfscope}%
\pgfsys@transformshift{3.662674in}{1.383078in}%
\pgfsys@useobject{currentmarker}{}%
\end{pgfscope}%
\end{pgfscope}%
\begin{pgfscope}%
\pgfsetbuttcap%
\pgfsetroundjoin%
\definecolor{currentfill}{rgb}{0.000000,0.000000,0.000000}%
\pgfsetfillcolor{currentfill}%
\pgfsetlinewidth{0.602250pt}%
\definecolor{currentstroke}{rgb}{0.000000,0.000000,0.000000}%
\pgfsetstrokecolor{currentstroke}%
\pgfsetdash{}{0pt}%
\pgfsys@defobject{currentmarker}{\pgfqpoint{0.000000in}{0.000000in}}{\pgfqpoint{0.027778in}{0.000000in}}{%
\pgfpathmoveto{\pgfqpoint{0.000000in}{0.000000in}}%
\pgfpathlineto{\pgfqpoint{0.027778in}{0.000000in}}%
\pgfusepath{stroke,fill}%
}%
\begin{pgfscope}%
\pgfsys@transformshift{5.801389in}{1.383078in}%
\pgfsys@useobject{currentmarker}{}%
\end{pgfscope}%
\end{pgfscope}%
\begin{pgfscope}%
\pgfpathrectangle{\pgfqpoint{3.662674in}{0.552778in}}{\pgfqpoint{2.138715in}{1.650000in}}%
\pgfusepath{clip}%
\pgfsetrectcap%
\pgfsetroundjoin%
\pgfsetlinewidth{0.803000pt}%
\definecolor{currentstroke}{rgb}{0.690196,0.690196,0.690196}%
\pgfsetstrokecolor{currentstroke}%
\pgfsetstrokeopacity{0.300000}%
\pgfsetdash{}{0pt}%
\pgfpathmoveto{\pgfqpoint{3.662674in}{1.462583in}}%
\pgfpathlineto{\pgfqpoint{5.801389in}{1.462583in}}%
\pgfusepath{stroke}%
\end{pgfscope}%
\begin{pgfscope}%
\pgfsetbuttcap%
\pgfsetroundjoin%
\definecolor{currentfill}{rgb}{0.000000,0.000000,0.000000}%
\pgfsetfillcolor{currentfill}%
\pgfsetlinewidth{0.602250pt}%
\definecolor{currentstroke}{rgb}{0.000000,0.000000,0.000000}%
\pgfsetstrokecolor{currentstroke}%
\pgfsetdash{}{0pt}%
\pgfsys@defobject{currentmarker}{\pgfqpoint{-0.027778in}{0.000000in}}{\pgfqpoint{0.000000in}{0.000000in}}{%
\pgfpathmoveto{\pgfqpoint{0.000000in}{0.000000in}}%
\pgfpathlineto{\pgfqpoint{-0.027778in}{0.000000in}}%
\pgfusepath{stroke,fill}%
}%
\begin{pgfscope}%
\pgfsys@transformshift{3.662674in}{1.462583in}%
\pgfsys@useobject{currentmarker}{}%
\end{pgfscope}%
\end{pgfscope}%
\begin{pgfscope}%
\pgfsetbuttcap%
\pgfsetroundjoin%
\definecolor{currentfill}{rgb}{0.000000,0.000000,0.000000}%
\pgfsetfillcolor{currentfill}%
\pgfsetlinewidth{0.602250pt}%
\definecolor{currentstroke}{rgb}{0.000000,0.000000,0.000000}%
\pgfsetstrokecolor{currentstroke}%
\pgfsetdash{}{0pt}%
\pgfsys@defobject{currentmarker}{\pgfqpoint{0.000000in}{0.000000in}}{\pgfqpoint{0.027778in}{0.000000in}}{%
\pgfpathmoveto{\pgfqpoint{0.000000in}{0.000000in}}%
\pgfpathlineto{\pgfqpoint{0.027778in}{0.000000in}}%
\pgfusepath{stroke,fill}%
}%
\begin{pgfscope}%
\pgfsys@transformshift{5.801389in}{1.462583in}%
\pgfsys@useobject{currentmarker}{}%
\end{pgfscope}%
\end{pgfscope}%
\begin{pgfscope}%
\pgfpathrectangle{\pgfqpoint{3.662674in}{0.552778in}}{\pgfqpoint{2.138715in}{1.650000in}}%
\pgfusepath{clip}%
\pgfsetrectcap%
\pgfsetroundjoin%
\pgfsetlinewidth{0.803000pt}%
\definecolor{currentstroke}{rgb}{0.690196,0.690196,0.690196}%
\pgfsetstrokecolor{currentstroke}%
\pgfsetstrokeopacity{0.300000}%
\pgfsetdash{}{0pt}%
\pgfpathmoveto{\pgfqpoint{3.662674in}{1.502336in}}%
\pgfpathlineto{\pgfqpoint{5.801389in}{1.502336in}}%
\pgfusepath{stroke}%
\end{pgfscope}%
\begin{pgfscope}%
\pgfsetbuttcap%
\pgfsetroundjoin%
\definecolor{currentfill}{rgb}{0.000000,0.000000,0.000000}%
\pgfsetfillcolor{currentfill}%
\pgfsetlinewidth{0.602250pt}%
\definecolor{currentstroke}{rgb}{0.000000,0.000000,0.000000}%
\pgfsetstrokecolor{currentstroke}%
\pgfsetdash{}{0pt}%
\pgfsys@defobject{currentmarker}{\pgfqpoint{-0.027778in}{0.000000in}}{\pgfqpoint{0.000000in}{0.000000in}}{%
\pgfpathmoveto{\pgfqpoint{0.000000in}{0.000000in}}%
\pgfpathlineto{\pgfqpoint{-0.027778in}{0.000000in}}%
\pgfusepath{stroke,fill}%
}%
\begin{pgfscope}%
\pgfsys@transformshift{3.662674in}{1.502336in}%
\pgfsys@useobject{currentmarker}{}%
\end{pgfscope}%
\end{pgfscope}%
\begin{pgfscope}%
\pgfsetbuttcap%
\pgfsetroundjoin%
\definecolor{currentfill}{rgb}{0.000000,0.000000,0.000000}%
\pgfsetfillcolor{currentfill}%
\pgfsetlinewidth{0.602250pt}%
\definecolor{currentstroke}{rgb}{0.000000,0.000000,0.000000}%
\pgfsetstrokecolor{currentstroke}%
\pgfsetdash{}{0pt}%
\pgfsys@defobject{currentmarker}{\pgfqpoint{0.000000in}{0.000000in}}{\pgfqpoint{0.027778in}{0.000000in}}{%
\pgfpathmoveto{\pgfqpoint{0.000000in}{0.000000in}}%
\pgfpathlineto{\pgfqpoint{0.027778in}{0.000000in}}%
\pgfusepath{stroke,fill}%
}%
\begin{pgfscope}%
\pgfsys@transformshift{5.801389in}{1.502336in}%
\pgfsys@useobject{currentmarker}{}%
\end{pgfscope}%
\end{pgfscope}%
\begin{pgfscope}%
\pgfpathrectangle{\pgfqpoint{3.662674in}{0.552778in}}{\pgfqpoint{2.138715in}{1.650000in}}%
\pgfusepath{clip}%
\pgfsetrectcap%
\pgfsetroundjoin%
\pgfsetlinewidth{0.803000pt}%
\definecolor{currentstroke}{rgb}{0.690196,0.690196,0.690196}%
\pgfsetstrokecolor{currentstroke}%
\pgfsetstrokeopacity{0.300000}%
\pgfsetdash{}{0pt}%
\pgfpathmoveto{\pgfqpoint{3.662674in}{1.542089in}}%
\pgfpathlineto{\pgfqpoint{5.801389in}{1.542089in}}%
\pgfusepath{stroke}%
\end{pgfscope}%
\begin{pgfscope}%
\pgfsetbuttcap%
\pgfsetroundjoin%
\definecolor{currentfill}{rgb}{0.000000,0.000000,0.000000}%
\pgfsetfillcolor{currentfill}%
\pgfsetlinewidth{0.602250pt}%
\definecolor{currentstroke}{rgb}{0.000000,0.000000,0.000000}%
\pgfsetstrokecolor{currentstroke}%
\pgfsetdash{}{0pt}%
\pgfsys@defobject{currentmarker}{\pgfqpoint{-0.027778in}{0.000000in}}{\pgfqpoint{0.000000in}{0.000000in}}{%
\pgfpathmoveto{\pgfqpoint{0.000000in}{0.000000in}}%
\pgfpathlineto{\pgfqpoint{-0.027778in}{0.000000in}}%
\pgfusepath{stroke,fill}%
}%
\begin{pgfscope}%
\pgfsys@transformshift{3.662674in}{1.542089in}%
\pgfsys@useobject{currentmarker}{}%
\end{pgfscope}%
\end{pgfscope}%
\begin{pgfscope}%
\pgfsetbuttcap%
\pgfsetroundjoin%
\definecolor{currentfill}{rgb}{0.000000,0.000000,0.000000}%
\pgfsetfillcolor{currentfill}%
\pgfsetlinewidth{0.602250pt}%
\definecolor{currentstroke}{rgb}{0.000000,0.000000,0.000000}%
\pgfsetstrokecolor{currentstroke}%
\pgfsetdash{}{0pt}%
\pgfsys@defobject{currentmarker}{\pgfqpoint{0.000000in}{0.000000in}}{\pgfqpoint{0.027778in}{0.000000in}}{%
\pgfpathmoveto{\pgfqpoint{0.000000in}{0.000000in}}%
\pgfpathlineto{\pgfqpoint{0.027778in}{0.000000in}}%
\pgfusepath{stroke,fill}%
}%
\begin{pgfscope}%
\pgfsys@transformshift{5.801389in}{1.542089in}%
\pgfsys@useobject{currentmarker}{}%
\end{pgfscope}%
\end{pgfscope}%
\begin{pgfscope}%
\pgfpathrectangle{\pgfqpoint{3.662674in}{0.552778in}}{\pgfqpoint{2.138715in}{1.650000in}}%
\pgfusepath{clip}%
\pgfsetrectcap%
\pgfsetroundjoin%
\pgfsetlinewidth{0.803000pt}%
\definecolor{currentstroke}{rgb}{0.690196,0.690196,0.690196}%
\pgfsetstrokecolor{currentstroke}%
\pgfsetstrokeopacity{0.300000}%
\pgfsetdash{}{0pt}%
\pgfpathmoveto{\pgfqpoint{3.662674in}{1.581841in}}%
\pgfpathlineto{\pgfqpoint{5.801389in}{1.581841in}}%
\pgfusepath{stroke}%
\end{pgfscope}%
\begin{pgfscope}%
\pgfsetbuttcap%
\pgfsetroundjoin%
\definecolor{currentfill}{rgb}{0.000000,0.000000,0.000000}%
\pgfsetfillcolor{currentfill}%
\pgfsetlinewidth{0.602250pt}%
\definecolor{currentstroke}{rgb}{0.000000,0.000000,0.000000}%
\pgfsetstrokecolor{currentstroke}%
\pgfsetdash{}{0pt}%
\pgfsys@defobject{currentmarker}{\pgfqpoint{-0.027778in}{0.000000in}}{\pgfqpoint{0.000000in}{0.000000in}}{%
\pgfpathmoveto{\pgfqpoint{0.000000in}{0.000000in}}%
\pgfpathlineto{\pgfqpoint{-0.027778in}{0.000000in}}%
\pgfusepath{stroke,fill}%
}%
\begin{pgfscope}%
\pgfsys@transformshift{3.662674in}{1.581841in}%
\pgfsys@useobject{currentmarker}{}%
\end{pgfscope}%
\end{pgfscope}%
\begin{pgfscope}%
\pgfsetbuttcap%
\pgfsetroundjoin%
\definecolor{currentfill}{rgb}{0.000000,0.000000,0.000000}%
\pgfsetfillcolor{currentfill}%
\pgfsetlinewidth{0.602250pt}%
\definecolor{currentstroke}{rgb}{0.000000,0.000000,0.000000}%
\pgfsetstrokecolor{currentstroke}%
\pgfsetdash{}{0pt}%
\pgfsys@defobject{currentmarker}{\pgfqpoint{0.000000in}{0.000000in}}{\pgfqpoint{0.027778in}{0.000000in}}{%
\pgfpathmoveto{\pgfqpoint{0.000000in}{0.000000in}}%
\pgfpathlineto{\pgfqpoint{0.027778in}{0.000000in}}%
\pgfusepath{stroke,fill}%
}%
\begin{pgfscope}%
\pgfsys@transformshift{5.801389in}{1.581841in}%
\pgfsys@useobject{currentmarker}{}%
\end{pgfscope}%
\end{pgfscope}%
\begin{pgfscope}%
\pgfpathrectangle{\pgfqpoint{3.662674in}{0.552778in}}{\pgfqpoint{2.138715in}{1.650000in}}%
\pgfusepath{clip}%
\pgfsetrectcap%
\pgfsetroundjoin%
\pgfsetlinewidth{0.803000pt}%
\definecolor{currentstroke}{rgb}{0.690196,0.690196,0.690196}%
\pgfsetstrokecolor{currentstroke}%
\pgfsetstrokeopacity{0.300000}%
\pgfsetdash{}{0pt}%
\pgfpathmoveto{\pgfqpoint{3.662674in}{1.621594in}}%
\pgfpathlineto{\pgfqpoint{5.801389in}{1.621594in}}%
\pgfusepath{stroke}%
\end{pgfscope}%
\begin{pgfscope}%
\pgfsetbuttcap%
\pgfsetroundjoin%
\definecolor{currentfill}{rgb}{0.000000,0.000000,0.000000}%
\pgfsetfillcolor{currentfill}%
\pgfsetlinewidth{0.602250pt}%
\definecolor{currentstroke}{rgb}{0.000000,0.000000,0.000000}%
\pgfsetstrokecolor{currentstroke}%
\pgfsetdash{}{0pt}%
\pgfsys@defobject{currentmarker}{\pgfqpoint{-0.027778in}{0.000000in}}{\pgfqpoint{0.000000in}{0.000000in}}{%
\pgfpathmoveto{\pgfqpoint{0.000000in}{0.000000in}}%
\pgfpathlineto{\pgfqpoint{-0.027778in}{0.000000in}}%
\pgfusepath{stroke,fill}%
}%
\begin{pgfscope}%
\pgfsys@transformshift{3.662674in}{1.621594in}%
\pgfsys@useobject{currentmarker}{}%
\end{pgfscope}%
\end{pgfscope}%
\begin{pgfscope}%
\pgfsetbuttcap%
\pgfsetroundjoin%
\definecolor{currentfill}{rgb}{0.000000,0.000000,0.000000}%
\pgfsetfillcolor{currentfill}%
\pgfsetlinewidth{0.602250pt}%
\definecolor{currentstroke}{rgb}{0.000000,0.000000,0.000000}%
\pgfsetstrokecolor{currentstroke}%
\pgfsetdash{}{0pt}%
\pgfsys@defobject{currentmarker}{\pgfqpoint{0.000000in}{0.000000in}}{\pgfqpoint{0.027778in}{0.000000in}}{%
\pgfpathmoveto{\pgfqpoint{0.000000in}{0.000000in}}%
\pgfpathlineto{\pgfqpoint{0.027778in}{0.000000in}}%
\pgfusepath{stroke,fill}%
}%
\begin{pgfscope}%
\pgfsys@transformshift{5.801389in}{1.621594in}%
\pgfsys@useobject{currentmarker}{}%
\end{pgfscope}%
\end{pgfscope}%
\begin{pgfscope}%
\pgfpathrectangle{\pgfqpoint{3.662674in}{0.552778in}}{\pgfqpoint{2.138715in}{1.650000in}}%
\pgfusepath{clip}%
\pgfsetrectcap%
\pgfsetroundjoin%
\pgfsetlinewidth{0.803000pt}%
\definecolor{currentstroke}{rgb}{0.690196,0.690196,0.690196}%
\pgfsetstrokecolor{currentstroke}%
\pgfsetstrokeopacity{0.300000}%
\pgfsetdash{}{0pt}%
\pgfpathmoveto{\pgfqpoint{3.662674in}{1.661347in}}%
\pgfpathlineto{\pgfqpoint{5.801389in}{1.661347in}}%
\pgfusepath{stroke}%
\end{pgfscope}%
\begin{pgfscope}%
\pgfsetbuttcap%
\pgfsetroundjoin%
\definecolor{currentfill}{rgb}{0.000000,0.000000,0.000000}%
\pgfsetfillcolor{currentfill}%
\pgfsetlinewidth{0.602250pt}%
\definecolor{currentstroke}{rgb}{0.000000,0.000000,0.000000}%
\pgfsetstrokecolor{currentstroke}%
\pgfsetdash{}{0pt}%
\pgfsys@defobject{currentmarker}{\pgfqpoint{-0.027778in}{0.000000in}}{\pgfqpoint{0.000000in}{0.000000in}}{%
\pgfpathmoveto{\pgfqpoint{0.000000in}{0.000000in}}%
\pgfpathlineto{\pgfqpoint{-0.027778in}{0.000000in}}%
\pgfusepath{stroke,fill}%
}%
\begin{pgfscope}%
\pgfsys@transformshift{3.662674in}{1.661347in}%
\pgfsys@useobject{currentmarker}{}%
\end{pgfscope}%
\end{pgfscope}%
\begin{pgfscope}%
\pgfsetbuttcap%
\pgfsetroundjoin%
\definecolor{currentfill}{rgb}{0.000000,0.000000,0.000000}%
\pgfsetfillcolor{currentfill}%
\pgfsetlinewidth{0.602250pt}%
\definecolor{currentstroke}{rgb}{0.000000,0.000000,0.000000}%
\pgfsetstrokecolor{currentstroke}%
\pgfsetdash{}{0pt}%
\pgfsys@defobject{currentmarker}{\pgfqpoint{0.000000in}{0.000000in}}{\pgfqpoint{0.027778in}{0.000000in}}{%
\pgfpathmoveto{\pgfqpoint{0.000000in}{0.000000in}}%
\pgfpathlineto{\pgfqpoint{0.027778in}{0.000000in}}%
\pgfusepath{stroke,fill}%
}%
\begin{pgfscope}%
\pgfsys@transformshift{5.801389in}{1.661347in}%
\pgfsys@useobject{currentmarker}{}%
\end{pgfscope}%
\end{pgfscope}%
\begin{pgfscope}%
\pgfpathrectangle{\pgfqpoint{3.662674in}{0.552778in}}{\pgfqpoint{2.138715in}{1.650000in}}%
\pgfusepath{clip}%
\pgfsetrectcap%
\pgfsetroundjoin%
\pgfsetlinewidth{0.803000pt}%
\definecolor{currentstroke}{rgb}{0.690196,0.690196,0.690196}%
\pgfsetstrokecolor{currentstroke}%
\pgfsetstrokeopacity{0.300000}%
\pgfsetdash{}{0pt}%
\pgfpathmoveto{\pgfqpoint{3.662674in}{1.701099in}}%
\pgfpathlineto{\pgfqpoint{5.801389in}{1.701099in}}%
\pgfusepath{stroke}%
\end{pgfscope}%
\begin{pgfscope}%
\pgfsetbuttcap%
\pgfsetroundjoin%
\definecolor{currentfill}{rgb}{0.000000,0.000000,0.000000}%
\pgfsetfillcolor{currentfill}%
\pgfsetlinewidth{0.602250pt}%
\definecolor{currentstroke}{rgb}{0.000000,0.000000,0.000000}%
\pgfsetstrokecolor{currentstroke}%
\pgfsetdash{}{0pt}%
\pgfsys@defobject{currentmarker}{\pgfqpoint{-0.027778in}{0.000000in}}{\pgfqpoint{0.000000in}{0.000000in}}{%
\pgfpathmoveto{\pgfqpoint{0.000000in}{0.000000in}}%
\pgfpathlineto{\pgfqpoint{-0.027778in}{0.000000in}}%
\pgfusepath{stroke,fill}%
}%
\begin{pgfscope}%
\pgfsys@transformshift{3.662674in}{1.701099in}%
\pgfsys@useobject{currentmarker}{}%
\end{pgfscope}%
\end{pgfscope}%
\begin{pgfscope}%
\pgfsetbuttcap%
\pgfsetroundjoin%
\definecolor{currentfill}{rgb}{0.000000,0.000000,0.000000}%
\pgfsetfillcolor{currentfill}%
\pgfsetlinewidth{0.602250pt}%
\definecolor{currentstroke}{rgb}{0.000000,0.000000,0.000000}%
\pgfsetstrokecolor{currentstroke}%
\pgfsetdash{}{0pt}%
\pgfsys@defobject{currentmarker}{\pgfqpoint{0.000000in}{0.000000in}}{\pgfqpoint{0.027778in}{0.000000in}}{%
\pgfpathmoveto{\pgfqpoint{0.000000in}{0.000000in}}%
\pgfpathlineto{\pgfqpoint{0.027778in}{0.000000in}}%
\pgfusepath{stroke,fill}%
}%
\begin{pgfscope}%
\pgfsys@transformshift{5.801389in}{1.701099in}%
\pgfsys@useobject{currentmarker}{}%
\end{pgfscope}%
\end{pgfscope}%
\begin{pgfscope}%
\pgfpathrectangle{\pgfqpoint{3.662674in}{0.552778in}}{\pgfqpoint{2.138715in}{1.650000in}}%
\pgfusepath{clip}%
\pgfsetrectcap%
\pgfsetroundjoin%
\pgfsetlinewidth{0.803000pt}%
\definecolor{currentstroke}{rgb}{0.690196,0.690196,0.690196}%
\pgfsetstrokecolor{currentstroke}%
\pgfsetstrokeopacity{0.300000}%
\pgfsetdash{}{0pt}%
\pgfpathmoveto{\pgfqpoint{3.662674in}{1.740852in}}%
\pgfpathlineto{\pgfqpoint{5.801389in}{1.740852in}}%
\pgfusepath{stroke}%
\end{pgfscope}%
\begin{pgfscope}%
\pgfsetbuttcap%
\pgfsetroundjoin%
\definecolor{currentfill}{rgb}{0.000000,0.000000,0.000000}%
\pgfsetfillcolor{currentfill}%
\pgfsetlinewidth{0.602250pt}%
\definecolor{currentstroke}{rgb}{0.000000,0.000000,0.000000}%
\pgfsetstrokecolor{currentstroke}%
\pgfsetdash{}{0pt}%
\pgfsys@defobject{currentmarker}{\pgfqpoint{-0.027778in}{0.000000in}}{\pgfqpoint{0.000000in}{0.000000in}}{%
\pgfpathmoveto{\pgfqpoint{0.000000in}{0.000000in}}%
\pgfpathlineto{\pgfqpoint{-0.027778in}{0.000000in}}%
\pgfusepath{stroke,fill}%
}%
\begin{pgfscope}%
\pgfsys@transformshift{3.662674in}{1.740852in}%
\pgfsys@useobject{currentmarker}{}%
\end{pgfscope}%
\end{pgfscope}%
\begin{pgfscope}%
\pgfsetbuttcap%
\pgfsetroundjoin%
\definecolor{currentfill}{rgb}{0.000000,0.000000,0.000000}%
\pgfsetfillcolor{currentfill}%
\pgfsetlinewidth{0.602250pt}%
\definecolor{currentstroke}{rgb}{0.000000,0.000000,0.000000}%
\pgfsetstrokecolor{currentstroke}%
\pgfsetdash{}{0pt}%
\pgfsys@defobject{currentmarker}{\pgfqpoint{0.000000in}{0.000000in}}{\pgfqpoint{0.027778in}{0.000000in}}{%
\pgfpathmoveto{\pgfqpoint{0.000000in}{0.000000in}}%
\pgfpathlineto{\pgfqpoint{0.027778in}{0.000000in}}%
\pgfusepath{stroke,fill}%
}%
\begin{pgfscope}%
\pgfsys@transformshift{5.801389in}{1.740852in}%
\pgfsys@useobject{currentmarker}{}%
\end{pgfscope}%
\end{pgfscope}%
\begin{pgfscope}%
\pgfpathrectangle{\pgfqpoint{3.662674in}{0.552778in}}{\pgfqpoint{2.138715in}{1.650000in}}%
\pgfusepath{clip}%
\pgfsetrectcap%
\pgfsetroundjoin%
\pgfsetlinewidth{0.803000pt}%
\definecolor{currentstroke}{rgb}{0.690196,0.690196,0.690196}%
\pgfsetstrokecolor{currentstroke}%
\pgfsetstrokeopacity{0.300000}%
\pgfsetdash{}{0pt}%
\pgfpathmoveto{\pgfqpoint{3.662674in}{1.780605in}}%
\pgfpathlineto{\pgfqpoint{5.801389in}{1.780605in}}%
\pgfusepath{stroke}%
\end{pgfscope}%
\begin{pgfscope}%
\pgfsetbuttcap%
\pgfsetroundjoin%
\definecolor{currentfill}{rgb}{0.000000,0.000000,0.000000}%
\pgfsetfillcolor{currentfill}%
\pgfsetlinewidth{0.602250pt}%
\definecolor{currentstroke}{rgb}{0.000000,0.000000,0.000000}%
\pgfsetstrokecolor{currentstroke}%
\pgfsetdash{}{0pt}%
\pgfsys@defobject{currentmarker}{\pgfqpoint{-0.027778in}{0.000000in}}{\pgfqpoint{0.000000in}{0.000000in}}{%
\pgfpathmoveto{\pgfqpoint{0.000000in}{0.000000in}}%
\pgfpathlineto{\pgfqpoint{-0.027778in}{0.000000in}}%
\pgfusepath{stroke,fill}%
}%
\begin{pgfscope}%
\pgfsys@transformshift{3.662674in}{1.780605in}%
\pgfsys@useobject{currentmarker}{}%
\end{pgfscope}%
\end{pgfscope}%
\begin{pgfscope}%
\pgfsetbuttcap%
\pgfsetroundjoin%
\definecolor{currentfill}{rgb}{0.000000,0.000000,0.000000}%
\pgfsetfillcolor{currentfill}%
\pgfsetlinewidth{0.602250pt}%
\definecolor{currentstroke}{rgb}{0.000000,0.000000,0.000000}%
\pgfsetstrokecolor{currentstroke}%
\pgfsetdash{}{0pt}%
\pgfsys@defobject{currentmarker}{\pgfqpoint{0.000000in}{0.000000in}}{\pgfqpoint{0.027778in}{0.000000in}}{%
\pgfpathmoveto{\pgfqpoint{0.000000in}{0.000000in}}%
\pgfpathlineto{\pgfqpoint{0.027778in}{0.000000in}}%
\pgfusepath{stroke,fill}%
}%
\begin{pgfscope}%
\pgfsys@transformshift{5.801389in}{1.780605in}%
\pgfsys@useobject{currentmarker}{}%
\end{pgfscope}%
\end{pgfscope}%
\begin{pgfscope}%
\pgfpathrectangle{\pgfqpoint{3.662674in}{0.552778in}}{\pgfqpoint{2.138715in}{1.650000in}}%
\pgfusepath{clip}%
\pgfsetrectcap%
\pgfsetroundjoin%
\pgfsetlinewidth{0.803000pt}%
\definecolor{currentstroke}{rgb}{0.690196,0.690196,0.690196}%
\pgfsetstrokecolor{currentstroke}%
\pgfsetstrokeopacity{0.300000}%
\pgfsetdash{}{0pt}%
\pgfpathmoveto{\pgfqpoint{3.662674in}{1.860110in}}%
\pgfpathlineto{\pgfqpoint{5.801389in}{1.860110in}}%
\pgfusepath{stroke}%
\end{pgfscope}%
\begin{pgfscope}%
\pgfsetbuttcap%
\pgfsetroundjoin%
\definecolor{currentfill}{rgb}{0.000000,0.000000,0.000000}%
\pgfsetfillcolor{currentfill}%
\pgfsetlinewidth{0.602250pt}%
\definecolor{currentstroke}{rgb}{0.000000,0.000000,0.000000}%
\pgfsetstrokecolor{currentstroke}%
\pgfsetdash{}{0pt}%
\pgfsys@defobject{currentmarker}{\pgfqpoint{-0.027778in}{0.000000in}}{\pgfqpoint{0.000000in}{0.000000in}}{%
\pgfpathmoveto{\pgfqpoint{0.000000in}{0.000000in}}%
\pgfpathlineto{\pgfqpoint{-0.027778in}{0.000000in}}%
\pgfusepath{stroke,fill}%
}%
\begin{pgfscope}%
\pgfsys@transformshift{3.662674in}{1.860110in}%
\pgfsys@useobject{currentmarker}{}%
\end{pgfscope}%
\end{pgfscope}%
\begin{pgfscope}%
\pgfsetbuttcap%
\pgfsetroundjoin%
\definecolor{currentfill}{rgb}{0.000000,0.000000,0.000000}%
\pgfsetfillcolor{currentfill}%
\pgfsetlinewidth{0.602250pt}%
\definecolor{currentstroke}{rgb}{0.000000,0.000000,0.000000}%
\pgfsetstrokecolor{currentstroke}%
\pgfsetdash{}{0pt}%
\pgfsys@defobject{currentmarker}{\pgfqpoint{0.000000in}{0.000000in}}{\pgfqpoint{0.027778in}{0.000000in}}{%
\pgfpathmoveto{\pgfqpoint{0.000000in}{0.000000in}}%
\pgfpathlineto{\pgfqpoint{0.027778in}{0.000000in}}%
\pgfusepath{stroke,fill}%
}%
\begin{pgfscope}%
\pgfsys@transformshift{5.801389in}{1.860110in}%
\pgfsys@useobject{currentmarker}{}%
\end{pgfscope}%
\end{pgfscope}%
\begin{pgfscope}%
\pgfpathrectangle{\pgfqpoint{3.662674in}{0.552778in}}{\pgfqpoint{2.138715in}{1.650000in}}%
\pgfusepath{clip}%
\pgfsetrectcap%
\pgfsetroundjoin%
\pgfsetlinewidth{0.803000pt}%
\definecolor{currentstroke}{rgb}{0.690196,0.690196,0.690196}%
\pgfsetstrokecolor{currentstroke}%
\pgfsetstrokeopacity{0.300000}%
\pgfsetdash{}{0pt}%
\pgfpathmoveto{\pgfqpoint{3.662674in}{1.899863in}}%
\pgfpathlineto{\pgfqpoint{5.801389in}{1.899863in}}%
\pgfusepath{stroke}%
\end{pgfscope}%
\begin{pgfscope}%
\pgfsetbuttcap%
\pgfsetroundjoin%
\definecolor{currentfill}{rgb}{0.000000,0.000000,0.000000}%
\pgfsetfillcolor{currentfill}%
\pgfsetlinewidth{0.602250pt}%
\definecolor{currentstroke}{rgb}{0.000000,0.000000,0.000000}%
\pgfsetstrokecolor{currentstroke}%
\pgfsetdash{}{0pt}%
\pgfsys@defobject{currentmarker}{\pgfqpoint{-0.027778in}{0.000000in}}{\pgfqpoint{0.000000in}{0.000000in}}{%
\pgfpathmoveto{\pgfqpoint{0.000000in}{0.000000in}}%
\pgfpathlineto{\pgfqpoint{-0.027778in}{0.000000in}}%
\pgfusepath{stroke,fill}%
}%
\begin{pgfscope}%
\pgfsys@transformshift{3.662674in}{1.899863in}%
\pgfsys@useobject{currentmarker}{}%
\end{pgfscope}%
\end{pgfscope}%
\begin{pgfscope}%
\pgfsetbuttcap%
\pgfsetroundjoin%
\definecolor{currentfill}{rgb}{0.000000,0.000000,0.000000}%
\pgfsetfillcolor{currentfill}%
\pgfsetlinewidth{0.602250pt}%
\definecolor{currentstroke}{rgb}{0.000000,0.000000,0.000000}%
\pgfsetstrokecolor{currentstroke}%
\pgfsetdash{}{0pt}%
\pgfsys@defobject{currentmarker}{\pgfqpoint{0.000000in}{0.000000in}}{\pgfqpoint{0.027778in}{0.000000in}}{%
\pgfpathmoveto{\pgfqpoint{0.000000in}{0.000000in}}%
\pgfpathlineto{\pgfqpoint{0.027778in}{0.000000in}}%
\pgfusepath{stroke,fill}%
}%
\begin{pgfscope}%
\pgfsys@transformshift{5.801389in}{1.899863in}%
\pgfsys@useobject{currentmarker}{}%
\end{pgfscope}%
\end{pgfscope}%
\begin{pgfscope}%
\pgfpathrectangle{\pgfqpoint{3.662674in}{0.552778in}}{\pgfqpoint{2.138715in}{1.650000in}}%
\pgfusepath{clip}%
\pgfsetrectcap%
\pgfsetroundjoin%
\pgfsetlinewidth{0.803000pt}%
\definecolor{currentstroke}{rgb}{0.690196,0.690196,0.690196}%
\pgfsetstrokecolor{currentstroke}%
\pgfsetstrokeopacity{0.300000}%
\pgfsetdash{}{0pt}%
\pgfpathmoveto{\pgfqpoint{3.662674in}{1.939615in}}%
\pgfpathlineto{\pgfqpoint{5.801389in}{1.939615in}}%
\pgfusepath{stroke}%
\end{pgfscope}%
\begin{pgfscope}%
\pgfsetbuttcap%
\pgfsetroundjoin%
\definecolor{currentfill}{rgb}{0.000000,0.000000,0.000000}%
\pgfsetfillcolor{currentfill}%
\pgfsetlinewidth{0.602250pt}%
\definecolor{currentstroke}{rgb}{0.000000,0.000000,0.000000}%
\pgfsetstrokecolor{currentstroke}%
\pgfsetdash{}{0pt}%
\pgfsys@defobject{currentmarker}{\pgfqpoint{-0.027778in}{0.000000in}}{\pgfqpoint{0.000000in}{0.000000in}}{%
\pgfpathmoveto{\pgfqpoint{0.000000in}{0.000000in}}%
\pgfpathlineto{\pgfqpoint{-0.027778in}{0.000000in}}%
\pgfusepath{stroke,fill}%
}%
\begin{pgfscope}%
\pgfsys@transformshift{3.662674in}{1.939615in}%
\pgfsys@useobject{currentmarker}{}%
\end{pgfscope}%
\end{pgfscope}%
\begin{pgfscope}%
\pgfsetbuttcap%
\pgfsetroundjoin%
\definecolor{currentfill}{rgb}{0.000000,0.000000,0.000000}%
\pgfsetfillcolor{currentfill}%
\pgfsetlinewidth{0.602250pt}%
\definecolor{currentstroke}{rgb}{0.000000,0.000000,0.000000}%
\pgfsetstrokecolor{currentstroke}%
\pgfsetdash{}{0pt}%
\pgfsys@defobject{currentmarker}{\pgfqpoint{0.000000in}{0.000000in}}{\pgfqpoint{0.027778in}{0.000000in}}{%
\pgfpathmoveto{\pgfqpoint{0.000000in}{0.000000in}}%
\pgfpathlineto{\pgfqpoint{0.027778in}{0.000000in}}%
\pgfusepath{stroke,fill}%
}%
\begin{pgfscope}%
\pgfsys@transformshift{5.801389in}{1.939615in}%
\pgfsys@useobject{currentmarker}{}%
\end{pgfscope}%
\end{pgfscope}%
\begin{pgfscope}%
\pgfpathrectangle{\pgfqpoint{3.662674in}{0.552778in}}{\pgfqpoint{2.138715in}{1.650000in}}%
\pgfusepath{clip}%
\pgfsetrectcap%
\pgfsetroundjoin%
\pgfsetlinewidth{0.803000pt}%
\definecolor{currentstroke}{rgb}{0.690196,0.690196,0.690196}%
\pgfsetstrokecolor{currentstroke}%
\pgfsetstrokeopacity{0.300000}%
\pgfsetdash{}{0pt}%
\pgfpathmoveto{\pgfqpoint{3.662674in}{1.979368in}}%
\pgfpathlineto{\pgfqpoint{5.801389in}{1.979368in}}%
\pgfusepath{stroke}%
\end{pgfscope}%
\begin{pgfscope}%
\pgfsetbuttcap%
\pgfsetroundjoin%
\definecolor{currentfill}{rgb}{0.000000,0.000000,0.000000}%
\pgfsetfillcolor{currentfill}%
\pgfsetlinewidth{0.602250pt}%
\definecolor{currentstroke}{rgb}{0.000000,0.000000,0.000000}%
\pgfsetstrokecolor{currentstroke}%
\pgfsetdash{}{0pt}%
\pgfsys@defobject{currentmarker}{\pgfqpoint{-0.027778in}{0.000000in}}{\pgfqpoint{0.000000in}{0.000000in}}{%
\pgfpathmoveto{\pgfqpoint{0.000000in}{0.000000in}}%
\pgfpathlineto{\pgfqpoint{-0.027778in}{0.000000in}}%
\pgfusepath{stroke,fill}%
}%
\begin{pgfscope}%
\pgfsys@transformshift{3.662674in}{1.979368in}%
\pgfsys@useobject{currentmarker}{}%
\end{pgfscope}%
\end{pgfscope}%
\begin{pgfscope}%
\pgfsetbuttcap%
\pgfsetroundjoin%
\definecolor{currentfill}{rgb}{0.000000,0.000000,0.000000}%
\pgfsetfillcolor{currentfill}%
\pgfsetlinewidth{0.602250pt}%
\definecolor{currentstroke}{rgb}{0.000000,0.000000,0.000000}%
\pgfsetstrokecolor{currentstroke}%
\pgfsetdash{}{0pt}%
\pgfsys@defobject{currentmarker}{\pgfqpoint{0.000000in}{0.000000in}}{\pgfqpoint{0.027778in}{0.000000in}}{%
\pgfpathmoveto{\pgfqpoint{0.000000in}{0.000000in}}%
\pgfpathlineto{\pgfqpoint{0.027778in}{0.000000in}}%
\pgfusepath{stroke,fill}%
}%
\begin{pgfscope}%
\pgfsys@transformshift{5.801389in}{1.979368in}%
\pgfsys@useobject{currentmarker}{}%
\end{pgfscope}%
\end{pgfscope}%
\begin{pgfscope}%
\pgfpathrectangle{\pgfqpoint{3.662674in}{0.552778in}}{\pgfqpoint{2.138715in}{1.650000in}}%
\pgfusepath{clip}%
\pgfsetrectcap%
\pgfsetroundjoin%
\pgfsetlinewidth{0.803000pt}%
\definecolor{currentstroke}{rgb}{0.690196,0.690196,0.690196}%
\pgfsetstrokecolor{currentstroke}%
\pgfsetstrokeopacity{0.300000}%
\pgfsetdash{}{0pt}%
\pgfpathmoveto{\pgfqpoint{3.662674in}{2.019121in}}%
\pgfpathlineto{\pgfqpoint{5.801389in}{2.019121in}}%
\pgfusepath{stroke}%
\end{pgfscope}%
\begin{pgfscope}%
\pgfsetbuttcap%
\pgfsetroundjoin%
\definecolor{currentfill}{rgb}{0.000000,0.000000,0.000000}%
\pgfsetfillcolor{currentfill}%
\pgfsetlinewidth{0.602250pt}%
\definecolor{currentstroke}{rgb}{0.000000,0.000000,0.000000}%
\pgfsetstrokecolor{currentstroke}%
\pgfsetdash{}{0pt}%
\pgfsys@defobject{currentmarker}{\pgfqpoint{-0.027778in}{0.000000in}}{\pgfqpoint{0.000000in}{0.000000in}}{%
\pgfpathmoveto{\pgfqpoint{0.000000in}{0.000000in}}%
\pgfpathlineto{\pgfqpoint{-0.027778in}{0.000000in}}%
\pgfusepath{stroke,fill}%
}%
\begin{pgfscope}%
\pgfsys@transformshift{3.662674in}{2.019121in}%
\pgfsys@useobject{currentmarker}{}%
\end{pgfscope}%
\end{pgfscope}%
\begin{pgfscope}%
\pgfsetbuttcap%
\pgfsetroundjoin%
\definecolor{currentfill}{rgb}{0.000000,0.000000,0.000000}%
\pgfsetfillcolor{currentfill}%
\pgfsetlinewidth{0.602250pt}%
\definecolor{currentstroke}{rgb}{0.000000,0.000000,0.000000}%
\pgfsetstrokecolor{currentstroke}%
\pgfsetdash{}{0pt}%
\pgfsys@defobject{currentmarker}{\pgfqpoint{0.000000in}{0.000000in}}{\pgfqpoint{0.027778in}{0.000000in}}{%
\pgfpathmoveto{\pgfqpoint{0.000000in}{0.000000in}}%
\pgfpathlineto{\pgfqpoint{0.027778in}{0.000000in}}%
\pgfusepath{stroke,fill}%
}%
\begin{pgfscope}%
\pgfsys@transformshift{5.801389in}{2.019121in}%
\pgfsys@useobject{currentmarker}{}%
\end{pgfscope}%
\end{pgfscope}%
\begin{pgfscope}%
\pgfpathrectangle{\pgfqpoint{3.662674in}{0.552778in}}{\pgfqpoint{2.138715in}{1.650000in}}%
\pgfusepath{clip}%
\pgfsetrectcap%
\pgfsetroundjoin%
\pgfsetlinewidth{0.803000pt}%
\definecolor{currentstroke}{rgb}{0.690196,0.690196,0.690196}%
\pgfsetstrokecolor{currentstroke}%
\pgfsetstrokeopacity{0.300000}%
\pgfsetdash{}{0pt}%
\pgfpathmoveto{\pgfqpoint{3.662674in}{2.058873in}}%
\pgfpathlineto{\pgfqpoint{5.801389in}{2.058873in}}%
\pgfusepath{stroke}%
\end{pgfscope}%
\begin{pgfscope}%
\pgfsetbuttcap%
\pgfsetroundjoin%
\definecolor{currentfill}{rgb}{0.000000,0.000000,0.000000}%
\pgfsetfillcolor{currentfill}%
\pgfsetlinewidth{0.602250pt}%
\definecolor{currentstroke}{rgb}{0.000000,0.000000,0.000000}%
\pgfsetstrokecolor{currentstroke}%
\pgfsetdash{}{0pt}%
\pgfsys@defobject{currentmarker}{\pgfqpoint{-0.027778in}{0.000000in}}{\pgfqpoint{0.000000in}{0.000000in}}{%
\pgfpathmoveto{\pgfqpoint{0.000000in}{0.000000in}}%
\pgfpathlineto{\pgfqpoint{-0.027778in}{0.000000in}}%
\pgfusepath{stroke,fill}%
}%
\begin{pgfscope}%
\pgfsys@transformshift{3.662674in}{2.058873in}%
\pgfsys@useobject{currentmarker}{}%
\end{pgfscope}%
\end{pgfscope}%
\begin{pgfscope}%
\pgfsetbuttcap%
\pgfsetroundjoin%
\definecolor{currentfill}{rgb}{0.000000,0.000000,0.000000}%
\pgfsetfillcolor{currentfill}%
\pgfsetlinewidth{0.602250pt}%
\definecolor{currentstroke}{rgb}{0.000000,0.000000,0.000000}%
\pgfsetstrokecolor{currentstroke}%
\pgfsetdash{}{0pt}%
\pgfsys@defobject{currentmarker}{\pgfqpoint{0.000000in}{0.000000in}}{\pgfqpoint{0.027778in}{0.000000in}}{%
\pgfpathmoveto{\pgfqpoint{0.000000in}{0.000000in}}%
\pgfpathlineto{\pgfqpoint{0.027778in}{0.000000in}}%
\pgfusepath{stroke,fill}%
}%
\begin{pgfscope}%
\pgfsys@transformshift{5.801389in}{2.058873in}%
\pgfsys@useobject{currentmarker}{}%
\end{pgfscope}%
\end{pgfscope}%
\begin{pgfscope}%
\pgfpathrectangle{\pgfqpoint{3.662674in}{0.552778in}}{\pgfqpoint{2.138715in}{1.650000in}}%
\pgfusepath{clip}%
\pgfsetrectcap%
\pgfsetroundjoin%
\pgfsetlinewidth{0.803000pt}%
\definecolor{currentstroke}{rgb}{0.690196,0.690196,0.690196}%
\pgfsetstrokecolor{currentstroke}%
\pgfsetstrokeopacity{0.300000}%
\pgfsetdash{}{0pt}%
\pgfpathmoveto{\pgfqpoint{3.662674in}{2.098626in}}%
\pgfpathlineto{\pgfqpoint{5.801389in}{2.098626in}}%
\pgfusepath{stroke}%
\end{pgfscope}%
\begin{pgfscope}%
\pgfsetbuttcap%
\pgfsetroundjoin%
\definecolor{currentfill}{rgb}{0.000000,0.000000,0.000000}%
\pgfsetfillcolor{currentfill}%
\pgfsetlinewidth{0.602250pt}%
\definecolor{currentstroke}{rgb}{0.000000,0.000000,0.000000}%
\pgfsetstrokecolor{currentstroke}%
\pgfsetdash{}{0pt}%
\pgfsys@defobject{currentmarker}{\pgfqpoint{-0.027778in}{0.000000in}}{\pgfqpoint{0.000000in}{0.000000in}}{%
\pgfpathmoveto{\pgfqpoint{0.000000in}{0.000000in}}%
\pgfpathlineto{\pgfqpoint{-0.027778in}{0.000000in}}%
\pgfusepath{stroke,fill}%
}%
\begin{pgfscope}%
\pgfsys@transformshift{3.662674in}{2.098626in}%
\pgfsys@useobject{currentmarker}{}%
\end{pgfscope}%
\end{pgfscope}%
\begin{pgfscope}%
\pgfsetbuttcap%
\pgfsetroundjoin%
\definecolor{currentfill}{rgb}{0.000000,0.000000,0.000000}%
\pgfsetfillcolor{currentfill}%
\pgfsetlinewidth{0.602250pt}%
\definecolor{currentstroke}{rgb}{0.000000,0.000000,0.000000}%
\pgfsetstrokecolor{currentstroke}%
\pgfsetdash{}{0pt}%
\pgfsys@defobject{currentmarker}{\pgfqpoint{0.000000in}{0.000000in}}{\pgfqpoint{0.027778in}{0.000000in}}{%
\pgfpathmoveto{\pgfqpoint{0.000000in}{0.000000in}}%
\pgfpathlineto{\pgfqpoint{0.027778in}{0.000000in}}%
\pgfusepath{stroke,fill}%
}%
\begin{pgfscope}%
\pgfsys@transformshift{5.801389in}{2.098626in}%
\pgfsys@useobject{currentmarker}{}%
\end{pgfscope}%
\end{pgfscope}%
\begin{pgfscope}%
\pgfpathrectangle{\pgfqpoint{3.662674in}{0.552778in}}{\pgfqpoint{2.138715in}{1.650000in}}%
\pgfusepath{clip}%
\pgfsetrectcap%
\pgfsetroundjoin%
\pgfsetlinewidth{0.803000pt}%
\definecolor{currentstroke}{rgb}{0.690196,0.690196,0.690196}%
\pgfsetstrokecolor{currentstroke}%
\pgfsetstrokeopacity{0.300000}%
\pgfsetdash{}{0pt}%
\pgfpathmoveto{\pgfqpoint{3.662674in}{2.138378in}}%
\pgfpathlineto{\pgfqpoint{5.801389in}{2.138378in}}%
\pgfusepath{stroke}%
\end{pgfscope}%
\begin{pgfscope}%
\pgfsetbuttcap%
\pgfsetroundjoin%
\definecolor{currentfill}{rgb}{0.000000,0.000000,0.000000}%
\pgfsetfillcolor{currentfill}%
\pgfsetlinewidth{0.602250pt}%
\definecolor{currentstroke}{rgb}{0.000000,0.000000,0.000000}%
\pgfsetstrokecolor{currentstroke}%
\pgfsetdash{}{0pt}%
\pgfsys@defobject{currentmarker}{\pgfqpoint{-0.027778in}{0.000000in}}{\pgfqpoint{0.000000in}{0.000000in}}{%
\pgfpathmoveto{\pgfqpoint{0.000000in}{0.000000in}}%
\pgfpathlineto{\pgfqpoint{-0.027778in}{0.000000in}}%
\pgfusepath{stroke,fill}%
}%
\begin{pgfscope}%
\pgfsys@transformshift{3.662674in}{2.138378in}%
\pgfsys@useobject{currentmarker}{}%
\end{pgfscope}%
\end{pgfscope}%
\begin{pgfscope}%
\pgfsetbuttcap%
\pgfsetroundjoin%
\definecolor{currentfill}{rgb}{0.000000,0.000000,0.000000}%
\pgfsetfillcolor{currentfill}%
\pgfsetlinewidth{0.602250pt}%
\definecolor{currentstroke}{rgb}{0.000000,0.000000,0.000000}%
\pgfsetstrokecolor{currentstroke}%
\pgfsetdash{}{0pt}%
\pgfsys@defobject{currentmarker}{\pgfqpoint{0.000000in}{0.000000in}}{\pgfqpoint{0.027778in}{0.000000in}}{%
\pgfpathmoveto{\pgfqpoint{0.000000in}{0.000000in}}%
\pgfpathlineto{\pgfqpoint{0.027778in}{0.000000in}}%
\pgfusepath{stroke,fill}%
}%
\begin{pgfscope}%
\pgfsys@transformshift{5.801389in}{2.138378in}%
\pgfsys@useobject{currentmarker}{}%
\end{pgfscope}%
\end{pgfscope}%
\begin{pgfscope}%
\pgfpathrectangle{\pgfqpoint{3.662674in}{0.552778in}}{\pgfqpoint{2.138715in}{1.650000in}}%
\pgfusepath{clip}%
\pgfsetrectcap%
\pgfsetroundjoin%
\pgfsetlinewidth{0.803000pt}%
\definecolor{currentstroke}{rgb}{0.690196,0.690196,0.690196}%
\pgfsetstrokecolor{currentstroke}%
\pgfsetstrokeopacity{0.300000}%
\pgfsetdash{}{0pt}%
\pgfpathmoveto{\pgfqpoint{3.662674in}{2.178131in}}%
\pgfpathlineto{\pgfqpoint{5.801389in}{2.178131in}}%
\pgfusepath{stroke}%
\end{pgfscope}%
\begin{pgfscope}%
\pgfsetbuttcap%
\pgfsetroundjoin%
\definecolor{currentfill}{rgb}{0.000000,0.000000,0.000000}%
\pgfsetfillcolor{currentfill}%
\pgfsetlinewidth{0.602250pt}%
\definecolor{currentstroke}{rgb}{0.000000,0.000000,0.000000}%
\pgfsetstrokecolor{currentstroke}%
\pgfsetdash{}{0pt}%
\pgfsys@defobject{currentmarker}{\pgfqpoint{-0.027778in}{0.000000in}}{\pgfqpoint{0.000000in}{0.000000in}}{%
\pgfpathmoveto{\pgfqpoint{0.000000in}{0.000000in}}%
\pgfpathlineto{\pgfqpoint{-0.027778in}{0.000000in}}%
\pgfusepath{stroke,fill}%
}%
\begin{pgfscope}%
\pgfsys@transformshift{3.662674in}{2.178131in}%
\pgfsys@useobject{currentmarker}{}%
\end{pgfscope}%
\end{pgfscope}%
\begin{pgfscope}%
\pgfsetbuttcap%
\pgfsetroundjoin%
\definecolor{currentfill}{rgb}{0.000000,0.000000,0.000000}%
\pgfsetfillcolor{currentfill}%
\pgfsetlinewidth{0.602250pt}%
\definecolor{currentstroke}{rgb}{0.000000,0.000000,0.000000}%
\pgfsetstrokecolor{currentstroke}%
\pgfsetdash{}{0pt}%
\pgfsys@defobject{currentmarker}{\pgfqpoint{0.000000in}{0.000000in}}{\pgfqpoint{0.027778in}{0.000000in}}{%
\pgfpathmoveto{\pgfqpoint{0.000000in}{0.000000in}}%
\pgfpathlineto{\pgfqpoint{0.027778in}{0.000000in}}%
\pgfusepath{stroke,fill}%
}%
\begin{pgfscope}%
\pgfsys@transformshift{5.801389in}{2.178131in}%
\pgfsys@useobject{currentmarker}{}%
\end{pgfscope}%
\end{pgfscope}%
\begin{pgfscope}%
\definecolor{textcolor}{rgb}{0.000000,0.000000,0.000000}%
\pgfsetstrokecolor{textcolor}%
\pgfsetfillcolor{textcolor}%
\pgftext[x=3.301563in,y=1.377778in,,bottom,rotate=90.000000]{\color{textcolor}\rmfamily\fontsize{10.000000}{12.000000}\selectfont Ereignisszahl}%
\end{pgfscope}%
\begin{pgfscope}%
\pgfpathrectangle{\pgfqpoint{3.662674in}{0.552778in}}{\pgfqpoint{2.138715in}{1.650000in}}%
\pgfusepath{clip}%
\pgfsetrectcap%
\pgfsetroundjoin%
\pgfsetlinewidth{1.505625pt}%
\definecolor{currentstroke}{rgb}{0.121569,0.466667,0.705882}%
\pgfsetstrokecolor{currentstroke}%
\pgfsetdash{}{0pt}%
\pgfpathmoveto{\pgfqpoint{3.663346in}{0.638378in}}%
\pgfpathlineto{\pgfqpoint{3.664018in}{0.630428in}}%
\pgfpathlineto{\pgfqpoint{3.664690in}{0.630428in}}%
\pgfpathlineto{\pgfqpoint{3.665362in}{0.630428in}}%
\pgfpathlineto{\pgfqpoint{3.666707in}{0.648979in}}%
\pgfpathlineto{\pgfqpoint{3.667379in}{0.648979in}}%
\pgfpathlineto{\pgfqpoint{3.668723in}{0.635728in}}%
\pgfpathlineto{\pgfqpoint{3.670067in}{0.635728in}}%
\pgfpathlineto{\pgfqpoint{3.670740in}{0.646329in}}%
\pgfpathlineto{\pgfqpoint{3.671412in}{0.643679in}}%
\pgfpathlineto{\pgfqpoint{3.672084in}{0.643679in}}%
\pgfpathlineto{\pgfqpoint{3.672084in}{0.630428in}}%
\pgfpathlineto{\pgfqpoint{3.673428in}{0.641029in}}%
\pgfpathlineto{\pgfqpoint{3.674100in}{0.641029in}}%
\pgfpathlineto{\pgfqpoint{3.674100in}{0.646329in}}%
\pgfpathlineto{\pgfqpoint{3.674773in}{0.630428in}}%
\pgfpathlineto{\pgfqpoint{3.675445in}{0.635728in}}%
\pgfpathlineto{\pgfqpoint{3.676117in}{0.635728in}}%
\pgfpathlineto{\pgfqpoint{3.676117in}{0.638378in}}%
\pgfpathlineto{\pgfqpoint{3.677461in}{0.633078in}}%
\pgfpathlineto{\pgfqpoint{3.678133in}{0.633078in}}%
\pgfpathlineto{\pgfqpoint{3.678133in}{0.635728in}}%
\pgfpathlineto{\pgfqpoint{3.679478in}{0.635728in}}%
\pgfpathlineto{\pgfqpoint{3.680150in}{0.635728in}}%
\pgfpathlineto{\pgfqpoint{3.680822in}{0.648979in}}%
\pgfpathlineto{\pgfqpoint{3.681494in}{0.635728in}}%
\pgfpathlineto{\pgfqpoint{3.682166in}{0.635728in}}%
\pgfpathlineto{\pgfqpoint{3.682166in}{0.638378in}}%
\pgfpathlineto{\pgfqpoint{3.682839in}{0.630428in}}%
\pgfpathlineto{\pgfqpoint{3.683511in}{0.630428in}}%
\pgfpathlineto{\pgfqpoint{3.684183in}{0.630428in}}%
\pgfpathlineto{\pgfqpoint{3.684183in}{0.635728in}}%
\pgfpathlineto{\pgfqpoint{3.685527in}{0.633078in}}%
\pgfpathlineto{\pgfqpoint{3.686199in}{0.633078in}}%
\pgfpathlineto{\pgfqpoint{3.686199in}{0.638378in}}%
\pgfpathlineto{\pgfqpoint{3.687544in}{0.635728in}}%
\pgfpathlineto{\pgfqpoint{3.688216in}{0.635728in}}%
\pgfpathlineto{\pgfqpoint{3.689560in}{0.643679in}}%
\pgfpathlineto{\pgfqpoint{3.690232in}{0.643679in}}%
\pgfpathlineto{\pgfqpoint{3.690232in}{0.638378in}}%
\pgfpathlineto{\pgfqpoint{3.691577in}{0.638378in}}%
\pgfpathlineto{\pgfqpoint{3.692249in}{0.638378in}}%
\pgfpathlineto{\pgfqpoint{3.692921in}{0.630428in}}%
\pgfpathlineto{\pgfqpoint{3.693593in}{0.635728in}}%
\pgfpathlineto{\pgfqpoint{3.696282in}{0.635728in}}%
\pgfpathlineto{\pgfqpoint{3.696282in}{0.633078in}}%
\pgfpathlineto{\pgfqpoint{3.696954in}{0.643679in}}%
\pgfpathlineto{\pgfqpoint{3.697626in}{0.635728in}}%
\pgfpathlineto{\pgfqpoint{3.698298in}{0.635728in}}%
\pgfpathlineto{\pgfqpoint{3.698298in}{0.630428in}}%
\pgfpathlineto{\pgfqpoint{3.699643in}{0.641029in}}%
\pgfpathlineto{\pgfqpoint{3.700315in}{0.641029in}}%
\pgfpathlineto{\pgfqpoint{3.700315in}{0.635728in}}%
\pgfpathlineto{\pgfqpoint{3.701659in}{0.641029in}}%
\pgfpathlineto{\pgfqpoint{3.702332in}{0.641029in}}%
\pgfpathlineto{\pgfqpoint{3.702332in}{0.633078in}}%
\pgfpathlineto{\pgfqpoint{3.703676in}{0.638378in}}%
\pgfpathlineto{\pgfqpoint{3.705692in}{0.638378in}}%
\pgfpathlineto{\pgfqpoint{3.705692in}{0.627778in}}%
\pgfpathlineto{\pgfqpoint{3.707037in}{0.641029in}}%
\pgfpathlineto{\pgfqpoint{3.707709in}{0.641029in}}%
\pgfpathlineto{\pgfqpoint{3.707709in}{0.643679in}}%
\pgfpathlineto{\pgfqpoint{3.709053in}{0.635728in}}%
\pgfpathlineto{\pgfqpoint{3.709725in}{0.635728in}}%
\pgfpathlineto{\pgfqpoint{3.709725in}{0.646329in}}%
\pgfpathlineto{\pgfqpoint{3.710398in}{0.633078in}}%
\pgfpathlineto{\pgfqpoint{3.711070in}{0.638378in}}%
\pgfpathlineto{\pgfqpoint{3.711742in}{0.638378in}}%
\pgfpathlineto{\pgfqpoint{3.712414in}{0.648979in}}%
\pgfpathlineto{\pgfqpoint{3.713086in}{0.633078in}}%
\pgfpathlineto{\pgfqpoint{3.713758in}{0.633078in}}%
\pgfpathlineto{\pgfqpoint{3.713758in}{0.630428in}}%
\pgfpathlineto{\pgfqpoint{3.714431in}{0.635728in}}%
\pgfpathlineto{\pgfqpoint{3.715103in}{0.630428in}}%
\pgfpathlineto{\pgfqpoint{3.715775in}{0.630428in}}%
\pgfpathlineto{\pgfqpoint{3.716447in}{0.638378in}}%
\pgfpathlineto{\pgfqpoint{3.717119in}{0.633078in}}%
\pgfpathlineto{\pgfqpoint{3.717791in}{0.633078in}}%
\pgfpathlineto{\pgfqpoint{3.718464in}{0.643679in}}%
\pgfpathlineto{\pgfqpoint{3.719136in}{0.638378in}}%
\pgfpathlineto{\pgfqpoint{3.720480in}{0.638378in}}%
\pgfpathlineto{\pgfqpoint{3.721824in}{0.630428in}}%
\pgfpathlineto{\pgfqpoint{3.722497in}{0.630428in}}%
\pgfpathlineto{\pgfqpoint{3.722497in}{0.641029in}}%
\pgfpathlineto{\pgfqpoint{3.723841in}{0.641029in}}%
\pgfpathlineto{\pgfqpoint{3.724513in}{0.641029in}}%
\pgfpathlineto{\pgfqpoint{3.725185in}{0.630428in}}%
\pgfpathlineto{\pgfqpoint{3.725857in}{0.643679in}}%
\pgfpathlineto{\pgfqpoint{3.726530in}{0.643679in}}%
\pgfpathlineto{\pgfqpoint{3.726530in}{0.635728in}}%
\pgfpathlineto{\pgfqpoint{3.727874in}{0.646329in}}%
\pgfpathlineto{\pgfqpoint{3.729218in}{0.646329in}}%
\pgfpathlineto{\pgfqpoint{3.729890in}{0.633078in}}%
\pgfpathlineto{\pgfqpoint{3.730563in}{0.638378in}}%
\pgfpathlineto{\pgfqpoint{3.731235in}{0.638378in}}%
\pgfpathlineto{\pgfqpoint{3.731235in}{0.633078in}}%
\pgfpathlineto{\pgfqpoint{3.732579in}{0.638378in}}%
\pgfpathlineto{\pgfqpoint{3.733251in}{0.638378in}}%
\pgfpathlineto{\pgfqpoint{3.733923in}{0.643679in}}%
\pgfpathlineto{\pgfqpoint{3.734596in}{0.638378in}}%
\pgfpathlineto{\pgfqpoint{3.735268in}{0.638378in}}%
\pgfpathlineto{\pgfqpoint{3.735268in}{0.646329in}}%
\pgfpathlineto{\pgfqpoint{3.736612in}{0.641029in}}%
\pgfpathlineto{\pgfqpoint{3.737284in}{0.641029in}}%
\pgfpathlineto{\pgfqpoint{3.737956in}{0.633078in}}%
\pgfpathlineto{\pgfqpoint{3.738629in}{0.638378in}}%
\pgfpathlineto{\pgfqpoint{3.739301in}{0.638378in}}%
\pgfpathlineto{\pgfqpoint{3.739973in}{0.659580in}}%
\pgfpathlineto{\pgfqpoint{3.740645in}{0.651629in}}%
\pgfpathlineto{\pgfqpoint{3.741317in}{0.651629in}}%
\pgfpathlineto{\pgfqpoint{3.741989in}{0.638378in}}%
\pgfpathlineto{\pgfqpoint{3.742662in}{0.651629in}}%
\pgfpathlineto{\pgfqpoint{3.743334in}{0.651629in}}%
\pgfpathlineto{\pgfqpoint{3.743334in}{0.641029in}}%
\pgfpathlineto{\pgfqpoint{3.744678in}{0.641029in}}%
\pgfpathlineto{\pgfqpoint{3.745350in}{0.641029in}}%
\pgfpathlineto{\pgfqpoint{3.746695in}{0.630428in}}%
\pgfpathlineto{\pgfqpoint{3.747367in}{0.630428in}}%
\pgfpathlineto{\pgfqpoint{3.747367in}{0.646329in}}%
\pgfpathlineto{\pgfqpoint{3.748711in}{0.646329in}}%
\pgfpathlineto{\pgfqpoint{3.749383in}{0.646329in}}%
\pgfpathlineto{\pgfqpoint{3.750055in}{0.638378in}}%
\pgfpathlineto{\pgfqpoint{3.750728in}{0.659580in}}%
\pgfpathlineto{\pgfqpoint{3.751400in}{0.659580in}}%
\pgfpathlineto{\pgfqpoint{3.752744in}{0.633078in}}%
\pgfpathlineto{\pgfqpoint{3.753416in}{0.633078in}}%
\pgfpathlineto{\pgfqpoint{3.754761in}{0.648979in}}%
\pgfpathlineto{\pgfqpoint{3.755433in}{0.648979in}}%
\pgfpathlineto{\pgfqpoint{3.756105in}{0.638378in}}%
\pgfpathlineto{\pgfqpoint{3.756777in}{0.643679in}}%
\pgfpathlineto{\pgfqpoint{3.757449in}{0.643679in}}%
\pgfpathlineto{\pgfqpoint{3.757449in}{0.638378in}}%
\pgfpathlineto{\pgfqpoint{3.758121in}{0.654280in}}%
\pgfpathlineto{\pgfqpoint{3.758794in}{0.643679in}}%
\pgfpathlineto{\pgfqpoint{3.760138in}{0.643679in}}%
\pgfpathlineto{\pgfqpoint{3.760138in}{0.646329in}}%
\pgfpathlineto{\pgfqpoint{3.760810in}{0.630428in}}%
\pgfpathlineto{\pgfqpoint{3.761482in}{0.646329in}}%
\pgfpathlineto{\pgfqpoint{3.762154in}{0.646329in}}%
\pgfpathlineto{\pgfqpoint{3.762827in}{0.638378in}}%
\pgfpathlineto{\pgfqpoint{3.763499in}{0.648979in}}%
\pgfpathlineto{\pgfqpoint{3.764171in}{0.648979in}}%
\pgfpathlineto{\pgfqpoint{3.765515in}{0.638378in}}%
\pgfpathlineto{\pgfqpoint{3.766187in}{0.638378in}}%
\pgfpathlineto{\pgfqpoint{3.766187in}{0.651629in}}%
\pgfpathlineto{\pgfqpoint{3.766860in}{0.635728in}}%
\pgfpathlineto{\pgfqpoint{3.767532in}{0.643679in}}%
\pgfpathlineto{\pgfqpoint{3.768876in}{0.643679in}}%
\pgfpathlineto{\pgfqpoint{3.768876in}{0.656930in}}%
\pgfpathlineto{\pgfqpoint{3.769548in}{0.635728in}}%
\pgfpathlineto{\pgfqpoint{3.770220in}{0.646329in}}%
\pgfpathlineto{\pgfqpoint{3.770893in}{0.646329in}}%
\pgfpathlineto{\pgfqpoint{3.770893in}{0.656930in}}%
\pgfpathlineto{\pgfqpoint{3.771565in}{0.638378in}}%
\pgfpathlineto{\pgfqpoint{3.772237in}{0.656930in}}%
\pgfpathlineto{\pgfqpoint{3.772909in}{0.656930in}}%
\pgfpathlineto{\pgfqpoint{3.772909in}{0.638378in}}%
\pgfpathlineto{\pgfqpoint{3.774253in}{0.651629in}}%
\pgfpathlineto{\pgfqpoint{3.774926in}{0.651629in}}%
\pgfpathlineto{\pgfqpoint{3.775598in}{0.633078in}}%
\pgfpathlineto{\pgfqpoint{3.776270in}{0.654280in}}%
\pgfpathlineto{\pgfqpoint{3.776942in}{0.654280in}}%
\pgfpathlineto{\pgfqpoint{3.777614in}{0.635728in}}%
\pgfpathlineto{\pgfqpoint{3.778286in}{0.651629in}}%
\pgfpathlineto{\pgfqpoint{3.778959in}{0.651629in}}%
\pgfpathlineto{\pgfqpoint{3.778959in}{0.659580in}}%
\pgfpathlineto{\pgfqpoint{3.780303in}{0.641029in}}%
\pgfpathlineto{\pgfqpoint{3.780975in}{0.641029in}}%
\pgfpathlineto{\pgfqpoint{3.780975in}{0.643679in}}%
\pgfpathlineto{\pgfqpoint{3.782319in}{0.643679in}}%
\pgfpathlineto{\pgfqpoint{3.782992in}{0.643679in}}%
\pgfpathlineto{\pgfqpoint{3.783664in}{0.664880in}}%
\pgfpathlineto{\pgfqpoint{3.784336in}{0.651629in}}%
\pgfpathlineto{\pgfqpoint{3.785008in}{0.651629in}}%
\pgfpathlineto{\pgfqpoint{3.785680in}{0.638378in}}%
\pgfpathlineto{\pgfqpoint{3.786352in}{0.651629in}}%
\pgfpathlineto{\pgfqpoint{3.787025in}{0.651629in}}%
\pgfpathlineto{\pgfqpoint{3.788369in}{0.635728in}}%
\pgfpathlineto{\pgfqpoint{3.789041in}{0.635728in}}%
\pgfpathlineto{\pgfqpoint{3.789713in}{0.654280in}}%
\pgfpathlineto{\pgfqpoint{3.790385in}{0.654280in}}%
\pgfpathlineto{\pgfqpoint{3.791058in}{0.654280in}}%
\pgfpathlineto{\pgfqpoint{3.791730in}{0.643679in}}%
\pgfpathlineto{\pgfqpoint{3.792402in}{0.659580in}}%
\pgfpathlineto{\pgfqpoint{3.793746in}{0.659580in}}%
\pgfpathlineto{\pgfqpoint{3.795091in}{0.646329in}}%
\pgfpathlineto{\pgfqpoint{3.795763in}{0.646329in}}%
\pgfpathlineto{\pgfqpoint{3.797107in}{0.659580in}}%
\pgfpathlineto{\pgfqpoint{3.797779in}{0.659580in}}%
\pgfpathlineto{\pgfqpoint{3.797779in}{0.664880in}}%
\pgfpathlineto{\pgfqpoint{3.798451in}{0.646329in}}%
\pgfpathlineto{\pgfqpoint{3.799124in}{0.648979in}}%
\pgfpathlineto{\pgfqpoint{3.799796in}{0.648979in}}%
\pgfpathlineto{\pgfqpoint{3.801140in}{0.659580in}}%
\pgfpathlineto{\pgfqpoint{3.801812in}{0.659580in}}%
\pgfpathlineto{\pgfqpoint{3.803157in}{0.643679in}}%
\pgfpathlineto{\pgfqpoint{3.803829in}{0.643679in}}%
\pgfpathlineto{\pgfqpoint{3.804501in}{0.664880in}}%
\pgfpathlineto{\pgfqpoint{3.805173in}{0.662230in}}%
\pgfpathlineto{\pgfqpoint{3.805845in}{0.662230in}}%
\pgfpathlineto{\pgfqpoint{3.805845in}{0.664880in}}%
\pgfpathlineto{\pgfqpoint{3.806517in}{0.659580in}}%
\pgfpathlineto{\pgfqpoint{3.807190in}{0.664880in}}%
\pgfpathlineto{\pgfqpoint{3.807862in}{0.664880in}}%
\pgfpathlineto{\pgfqpoint{3.809206in}{0.648979in}}%
\pgfpathlineto{\pgfqpoint{3.809878in}{0.648979in}}%
\pgfpathlineto{\pgfqpoint{3.809878in}{0.646329in}}%
\pgfpathlineto{\pgfqpoint{3.810550in}{0.662230in}}%
\pgfpathlineto{\pgfqpoint{3.811223in}{0.662230in}}%
\pgfpathlineto{\pgfqpoint{3.811895in}{0.662230in}}%
\pgfpathlineto{\pgfqpoint{3.812567in}{0.683431in}}%
\pgfpathlineto{\pgfqpoint{3.813239in}{0.651629in}}%
\pgfpathlineto{\pgfqpoint{3.813911in}{0.651629in}}%
\pgfpathlineto{\pgfqpoint{3.814584in}{0.662230in}}%
\pgfpathlineto{\pgfqpoint{3.815256in}{0.651629in}}%
\pgfpathlineto{\pgfqpoint{3.816600in}{0.651629in}}%
\pgfpathlineto{\pgfqpoint{3.817944in}{0.672831in}}%
\pgfpathlineto{\pgfqpoint{3.818617in}{0.672831in}}%
\pgfpathlineto{\pgfqpoint{3.819289in}{0.651629in}}%
\pgfpathlineto{\pgfqpoint{3.819961in}{0.654280in}}%
\pgfpathlineto{\pgfqpoint{3.820633in}{0.654280in}}%
\pgfpathlineto{\pgfqpoint{3.820633in}{0.667530in}}%
\pgfpathlineto{\pgfqpoint{3.821977in}{0.662230in}}%
\pgfpathlineto{\pgfqpoint{3.822650in}{0.662230in}}%
\pgfpathlineto{\pgfqpoint{3.823322in}{0.648979in}}%
\pgfpathlineto{\pgfqpoint{3.823994in}{0.662230in}}%
\pgfpathlineto{\pgfqpoint{3.824666in}{0.662230in}}%
\pgfpathlineto{\pgfqpoint{3.824666in}{0.656930in}}%
\pgfpathlineto{\pgfqpoint{3.825338in}{0.672831in}}%
\pgfpathlineto{\pgfqpoint{3.826010in}{0.662230in}}%
\pgfpathlineto{\pgfqpoint{3.826683in}{0.662230in}}%
\pgfpathlineto{\pgfqpoint{3.826683in}{0.667530in}}%
\pgfpathlineto{\pgfqpoint{3.828027in}{0.659580in}}%
\pgfpathlineto{\pgfqpoint{3.829371in}{0.659580in}}%
\pgfpathlineto{\pgfqpoint{3.829371in}{0.664880in}}%
\pgfpathlineto{\pgfqpoint{3.830716in}{0.651629in}}%
\pgfpathlineto{\pgfqpoint{3.831388in}{0.651629in}}%
\pgfpathlineto{\pgfqpoint{3.832732in}{0.675481in}}%
\pgfpathlineto{\pgfqpoint{3.833404in}{0.675481in}}%
\pgfpathlineto{\pgfqpoint{3.834749in}{0.688732in}}%
\pgfpathlineto{\pgfqpoint{3.835421in}{0.688732in}}%
\pgfpathlineto{\pgfqpoint{3.835421in}{0.694032in}}%
\pgfpathlineto{\pgfqpoint{3.836765in}{0.667530in}}%
\pgfpathlineto{\pgfqpoint{3.837437in}{0.667530in}}%
\pgfpathlineto{\pgfqpoint{3.837437in}{0.662230in}}%
\pgfpathlineto{\pgfqpoint{3.838782in}{0.683431in}}%
\pgfpathlineto{\pgfqpoint{3.839454in}{0.683431in}}%
\pgfpathlineto{\pgfqpoint{3.840126in}{0.664880in}}%
\pgfpathlineto{\pgfqpoint{3.840798in}{0.670181in}}%
\pgfpathlineto{\pgfqpoint{3.841470in}{0.670181in}}%
\pgfpathlineto{\pgfqpoint{3.842142in}{0.667530in}}%
\pgfpathlineto{\pgfqpoint{3.842815in}{0.678131in}}%
\pgfpathlineto{\pgfqpoint{3.843487in}{0.678131in}}%
\pgfpathlineto{\pgfqpoint{3.843487in}{0.686082in}}%
\pgfpathlineto{\pgfqpoint{3.844159in}{0.648979in}}%
\pgfpathlineto{\pgfqpoint{3.844831in}{0.672831in}}%
\pgfpathlineto{\pgfqpoint{3.845503in}{0.672831in}}%
\pgfpathlineto{\pgfqpoint{3.845503in}{0.654280in}}%
\pgfpathlineto{\pgfqpoint{3.846175in}{0.678131in}}%
\pgfpathlineto{\pgfqpoint{3.846848in}{0.670181in}}%
\pgfpathlineto{\pgfqpoint{3.847520in}{0.670181in}}%
\pgfpathlineto{\pgfqpoint{3.847520in}{0.648979in}}%
\pgfpathlineto{\pgfqpoint{3.848192in}{0.675481in}}%
\pgfpathlineto{\pgfqpoint{3.848864in}{0.672831in}}%
\pgfpathlineto{\pgfqpoint{3.849536in}{0.672831in}}%
\pgfpathlineto{\pgfqpoint{3.849536in}{0.664880in}}%
\pgfpathlineto{\pgfqpoint{3.850881in}{0.664880in}}%
\pgfpathlineto{\pgfqpoint{3.851553in}{0.664880in}}%
\pgfpathlineto{\pgfqpoint{3.852225in}{0.694032in}}%
\pgfpathlineto{\pgfqpoint{3.852897in}{0.667530in}}%
\pgfpathlineto{\pgfqpoint{3.853569in}{0.667530in}}%
\pgfpathlineto{\pgfqpoint{3.854914in}{0.675481in}}%
\pgfpathlineto{\pgfqpoint{3.855586in}{0.675481in}}%
\pgfpathlineto{\pgfqpoint{3.855586in}{0.694032in}}%
\pgfpathlineto{\pgfqpoint{3.856930in}{0.651629in}}%
\pgfpathlineto{\pgfqpoint{3.857602in}{0.651629in}}%
\pgfpathlineto{\pgfqpoint{3.858947in}{0.699333in}}%
\pgfpathlineto{\pgfqpoint{3.859619in}{0.699333in}}%
\pgfpathlineto{\pgfqpoint{3.859619in}{0.672831in}}%
\pgfpathlineto{\pgfqpoint{3.860963in}{0.683431in}}%
\pgfpathlineto{\pgfqpoint{3.861635in}{0.683431in}}%
\pgfpathlineto{\pgfqpoint{3.862307in}{0.680781in}}%
\pgfpathlineto{\pgfqpoint{3.862980in}{0.699333in}}%
\pgfpathlineto{\pgfqpoint{3.863652in}{0.699333in}}%
\pgfpathlineto{\pgfqpoint{3.864324in}{0.688732in}}%
\pgfpathlineto{\pgfqpoint{3.864996in}{0.694032in}}%
\pgfpathlineto{\pgfqpoint{3.865668in}{0.694032in}}%
\pgfpathlineto{\pgfqpoint{3.866340in}{0.656930in}}%
\pgfpathlineto{\pgfqpoint{3.867013in}{0.675481in}}%
\pgfpathlineto{\pgfqpoint{3.867685in}{0.675481in}}%
\pgfpathlineto{\pgfqpoint{3.867685in}{0.701983in}}%
\pgfpathlineto{\pgfqpoint{3.868357in}{0.654280in}}%
\pgfpathlineto{\pgfqpoint{3.869029in}{0.683431in}}%
\pgfpathlineto{\pgfqpoint{3.869701in}{0.683431in}}%
\pgfpathlineto{\pgfqpoint{3.869701in}{0.694032in}}%
\pgfpathlineto{\pgfqpoint{3.870373in}{0.656930in}}%
\pgfpathlineto{\pgfqpoint{3.871046in}{0.667530in}}%
\pgfpathlineto{\pgfqpoint{3.871718in}{0.667530in}}%
\pgfpathlineto{\pgfqpoint{3.873062in}{0.688732in}}%
\pgfpathlineto{\pgfqpoint{3.873734in}{0.688732in}}%
\pgfpathlineto{\pgfqpoint{3.875079in}{0.670181in}}%
\pgfpathlineto{\pgfqpoint{3.875751in}{0.670181in}}%
\pgfpathlineto{\pgfqpoint{3.877095in}{0.686082in}}%
\pgfpathlineto{\pgfqpoint{3.877767in}{0.686082in}}%
\pgfpathlineto{\pgfqpoint{3.877767in}{0.670181in}}%
\pgfpathlineto{\pgfqpoint{3.878439in}{0.688732in}}%
\pgfpathlineto{\pgfqpoint{3.879112in}{0.672831in}}%
\pgfpathlineto{\pgfqpoint{3.879784in}{0.672831in}}%
\pgfpathlineto{\pgfqpoint{3.881128in}{0.699333in}}%
\pgfpathlineto{\pgfqpoint{3.881800in}{0.699333in}}%
\pgfpathlineto{\pgfqpoint{3.881800in}{0.683431in}}%
\pgfpathlineto{\pgfqpoint{3.882472in}{0.717884in}}%
\pgfpathlineto{\pgfqpoint{3.883145in}{0.704633in}}%
\pgfpathlineto{\pgfqpoint{3.883817in}{0.704633in}}%
\pgfpathlineto{\pgfqpoint{3.885161in}{0.675481in}}%
\pgfpathlineto{\pgfqpoint{3.885833in}{0.675481in}}%
\pgfpathlineto{\pgfqpoint{3.887178in}{0.701983in}}%
\pgfpathlineto{\pgfqpoint{3.887850in}{0.701983in}}%
\pgfpathlineto{\pgfqpoint{3.887850in}{0.672831in}}%
\pgfpathlineto{\pgfqpoint{3.889194in}{0.696682in}}%
\pgfpathlineto{\pgfqpoint{3.889866in}{0.696682in}}%
\pgfpathlineto{\pgfqpoint{3.889866in}{0.704633in}}%
\pgfpathlineto{\pgfqpoint{3.891211in}{0.672831in}}%
\pgfpathlineto{\pgfqpoint{3.891883in}{0.672831in}}%
\pgfpathlineto{\pgfqpoint{3.892555in}{0.696682in}}%
\pgfpathlineto{\pgfqpoint{3.893227in}{0.686082in}}%
\pgfpathlineto{\pgfqpoint{3.893899in}{0.686082in}}%
\pgfpathlineto{\pgfqpoint{3.894571in}{0.680781in}}%
\pgfpathlineto{\pgfqpoint{3.895244in}{0.704633in}}%
\pgfpathlineto{\pgfqpoint{3.895916in}{0.704633in}}%
\pgfpathlineto{\pgfqpoint{3.896588in}{0.678131in}}%
\pgfpathlineto{\pgfqpoint{3.897260in}{0.686082in}}%
\pgfpathlineto{\pgfqpoint{3.897932in}{0.686082in}}%
\pgfpathlineto{\pgfqpoint{3.899277in}{0.704633in}}%
\pgfpathlineto{\pgfqpoint{3.899949in}{0.704633in}}%
\pgfpathlineto{\pgfqpoint{3.900621in}{0.709933in}}%
\pgfpathlineto{\pgfqpoint{3.901293in}{0.699333in}}%
\pgfpathlineto{\pgfqpoint{3.901965in}{0.699333in}}%
\pgfpathlineto{\pgfqpoint{3.901965in}{0.691382in}}%
\pgfpathlineto{\pgfqpoint{3.903310in}{0.707283in}}%
\pgfpathlineto{\pgfqpoint{3.904654in}{0.707283in}}%
\pgfpathlineto{\pgfqpoint{3.905998in}{0.691382in}}%
\pgfpathlineto{\pgfqpoint{3.906670in}{0.691382in}}%
\pgfpathlineto{\pgfqpoint{3.906670in}{0.728484in}}%
\pgfpathlineto{\pgfqpoint{3.908015in}{0.704633in}}%
\pgfpathlineto{\pgfqpoint{3.909359in}{0.704633in}}%
\pgfpathlineto{\pgfqpoint{3.910703in}{0.715234in}}%
\pgfpathlineto{\pgfqpoint{3.911376in}{0.715234in}}%
\pgfpathlineto{\pgfqpoint{3.912048in}{0.739085in}}%
\pgfpathlineto{\pgfqpoint{3.912048in}{0.688732in}}%
\pgfpathlineto{\pgfqpoint{3.912720in}{0.696682in}}%
\pgfpathlineto{\pgfqpoint{3.913392in}{0.696682in}}%
\pgfpathlineto{\pgfqpoint{3.914064in}{0.691382in}}%
\pgfpathlineto{\pgfqpoint{3.914736in}{0.715234in}}%
\pgfpathlineto{\pgfqpoint{3.915409in}{0.715234in}}%
\pgfpathlineto{\pgfqpoint{3.915409in}{0.694032in}}%
\pgfpathlineto{\pgfqpoint{3.916753in}{0.707283in}}%
\pgfpathlineto{\pgfqpoint{3.917425in}{0.707283in}}%
\pgfpathlineto{\pgfqpoint{3.918097in}{0.725834in}}%
\pgfpathlineto{\pgfqpoint{3.918769in}{0.709933in}}%
\pgfpathlineto{\pgfqpoint{3.919442in}{0.709933in}}%
\pgfpathlineto{\pgfqpoint{3.920114in}{0.701983in}}%
\pgfpathlineto{\pgfqpoint{3.920786in}{0.725834in}}%
\pgfpathlineto{\pgfqpoint{3.921458in}{0.725834in}}%
\pgfpathlineto{\pgfqpoint{3.921458in}{0.680781in}}%
\pgfpathlineto{\pgfqpoint{3.922802in}{0.709933in}}%
\pgfpathlineto{\pgfqpoint{3.923475in}{0.709933in}}%
\pgfpathlineto{\pgfqpoint{3.923475in}{0.712583in}}%
\pgfpathlineto{\pgfqpoint{3.924819in}{0.712583in}}%
\pgfpathlineto{\pgfqpoint{3.925491in}{0.712583in}}%
\pgfpathlineto{\pgfqpoint{3.925491in}{0.720534in}}%
\pgfpathlineto{\pgfqpoint{3.926836in}{0.691382in}}%
\pgfpathlineto{\pgfqpoint{3.927508in}{0.691382in}}%
\pgfpathlineto{\pgfqpoint{3.927508in}{0.712583in}}%
\pgfpathlineto{\pgfqpoint{3.928852in}{0.701983in}}%
\pgfpathlineto{\pgfqpoint{3.929524in}{0.701983in}}%
\pgfpathlineto{\pgfqpoint{3.930196in}{0.720534in}}%
\pgfpathlineto{\pgfqpoint{3.930869in}{0.701983in}}%
\pgfpathlineto{\pgfqpoint{3.931541in}{0.701983in}}%
\pgfpathlineto{\pgfqpoint{3.932213in}{0.725834in}}%
\pgfpathlineto{\pgfqpoint{3.932885in}{0.712583in}}%
\pgfpathlineto{\pgfqpoint{3.933557in}{0.712583in}}%
\pgfpathlineto{\pgfqpoint{3.933557in}{0.683431in}}%
\pgfpathlineto{\pgfqpoint{3.934902in}{0.752336in}}%
\pgfpathlineto{\pgfqpoint{3.935574in}{0.752336in}}%
\pgfpathlineto{\pgfqpoint{3.935574in}{0.691382in}}%
\pgfpathlineto{\pgfqpoint{3.936918in}{0.741735in}}%
\pgfpathlineto{\pgfqpoint{3.937590in}{0.741735in}}%
\pgfpathlineto{\pgfqpoint{3.937590in}{0.709933in}}%
\pgfpathlineto{\pgfqpoint{3.938935in}{0.715234in}}%
\pgfpathlineto{\pgfqpoint{3.939607in}{0.715234in}}%
\pgfpathlineto{\pgfqpoint{3.940279in}{0.694032in}}%
\pgfpathlineto{\pgfqpoint{3.940951in}{0.739085in}}%
\pgfpathlineto{\pgfqpoint{3.941623in}{0.739085in}}%
\pgfpathlineto{\pgfqpoint{3.941623in}{0.704633in}}%
\pgfpathlineto{\pgfqpoint{3.942968in}{0.717884in}}%
\pgfpathlineto{\pgfqpoint{3.944312in}{0.717884in}}%
\pgfpathlineto{\pgfqpoint{3.944312in}{0.709933in}}%
\pgfpathlineto{\pgfqpoint{3.944984in}{0.725834in}}%
\pgfpathlineto{\pgfqpoint{3.945656in}{0.712583in}}%
\pgfpathlineto{\pgfqpoint{3.946328in}{0.712583in}}%
\pgfpathlineto{\pgfqpoint{3.946328in}{0.715234in}}%
\pgfpathlineto{\pgfqpoint{3.947673in}{0.704633in}}%
\pgfpathlineto{\pgfqpoint{3.948345in}{0.704633in}}%
\pgfpathlineto{\pgfqpoint{3.949017in}{0.701983in}}%
\pgfpathlineto{\pgfqpoint{3.949689in}{0.723184in}}%
\pgfpathlineto{\pgfqpoint{3.950361in}{0.723184in}}%
\pgfpathlineto{\pgfqpoint{3.950361in}{0.715234in}}%
\pgfpathlineto{\pgfqpoint{3.951706in}{0.739085in}}%
\pgfpathlineto{\pgfqpoint{3.952378in}{0.739085in}}%
\pgfpathlineto{\pgfqpoint{3.952378in}{0.712583in}}%
\pgfpathlineto{\pgfqpoint{3.953722in}{0.744386in}}%
\pgfpathlineto{\pgfqpoint{3.954394in}{0.744386in}}%
\pgfpathlineto{\pgfqpoint{3.955739in}{0.725834in}}%
\pgfpathlineto{\pgfqpoint{3.956411in}{0.725834in}}%
\pgfpathlineto{\pgfqpoint{3.956411in}{0.694032in}}%
\pgfpathlineto{\pgfqpoint{3.957755in}{0.717884in}}%
\pgfpathlineto{\pgfqpoint{3.958427in}{0.717884in}}%
\pgfpathlineto{\pgfqpoint{3.958427in}{0.739085in}}%
\pgfpathlineto{\pgfqpoint{3.959772in}{0.709933in}}%
\pgfpathlineto{\pgfqpoint{3.960444in}{0.709933in}}%
\pgfpathlineto{\pgfqpoint{3.961116in}{0.736435in}}%
\pgfpathlineto{\pgfqpoint{3.961788in}{0.725834in}}%
\pgfpathlineto{\pgfqpoint{3.962460in}{0.725834in}}%
\pgfpathlineto{\pgfqpoint{3.962460in}{0.736435in}}%
\pgfpathlineto{\pgfqpoint{3.963805in}{0.728484in}}%
\pgfpathlineto{\pgfqpoint{3.964477in}{0.728484in}}%
\pgfpathlineto{\pgfqpoint{3.964477in}{0.733785in}}%
\pgfpathlineto{\pgfqpoint{3.965821in}{0.701983in}}%
\pgfpathlineto{\pgfqpoint{3.966493in}{0.701983in}}%
\pgfpathlineto{\pgfqpoint{3.967166in}{0.741735in}}%
\pgfpathlineto{\pgfqpoint{3.967838in}{0.739085in}}%
\pgfpathlineto{\pgfqpoint{3.968510in}{0.739085in}}%
\pgfpathlineto{\pgfqpoint{3.968510in}{0.723184in}}%
\pgfpathlineto{\pgfqpoint{3.969854in}{0.765587in}}%
\pgfpathlineto{\pgfqpoint{3.970526in}{0.765587in}}%
\pgfpathlineto{\pgfqpoint{3.970526in}{0.712583in}}%
\pgfpathlineto{\pgfqpoint{3.971871in}{0.733785in}}%
\pgfpathlineto{\pgfqpoint{3.972543in}{0.733785in}}%
\pgfpathlineto{\pgfqpoint{3.973215in}{0.744386in}}%
\pgfpathlineto{\pgfqpoint{3.973887in}{0.707283in}}%
\pgfpathlineto{\pgfqpoint{3.974559in}{0.707283in}}%
\pgfpathlineto{\pgfqpoint{3.974559in}{0.747036in}}%
\pgfpathlineto{\pgfqpoint{3.975904in}{0.723184in}}%
\pgfpathlineto{\pgfqpoint{3.976576in}{0.723184in}}%
\pgfpathlineto{\pgfqpoint{3.976576in}{0.733785in}}%
\pgfpathlineto{\pgfqpoint{3.977248in}{0.715234in}}%
\pgfpathlineto{\pgfqpoint{3.977920in}{0.723184in}}%
\pgfpathlineto{\pgfqpoint{3.978592in}{0.723184in}}%
\pgfpathlineto{\pgfqpoint{3.979265in}{0.715234in}}%
\pgfpathlineto{\pgfqpoint{3.979937in}{0.741735in}}%
\pgfpathlineto{\pgfqpoint{3.981281in}{0.741735in}}%
\pgfpathlineto{\pgfqpoint{3.981281in}{0.752336in}}%
\pgfpathlineto{\pgfqpoint{3.981953in}{0.709933in}}%
\pgfpathlineto{\pgfqpoint{3.982625in}{0.752336in}}%
\pgfpathlineto{\pgfqpoint{3.983298in}{0.752336in}}%
\pgfpathlineto{\pgfqpoint{3.983298in}{0.744386in}}%
\pgfpathlineto{\pgfqpoint{3.984642in}{0.760287in}}%
\pgfpathlineto{\pgfqpoint{3.985314in}{0.760287in}}%
\pgfpathlineto{\pgfqpoint{3.985986in}{0.731135in}}%
\pgfpathlineto{\pgfqpoint{3.986658in}{0.776188in}}%
\pgfpathlineto{\pgfqpoint{3.987331in}{0.776188in}}%
\pgfpathlineto{\pgfqpoint{3.987331in}{0.723184in}}%
\pgfpathlineto{\pgfqpoint{3.988675in}{0.741735in}}%
\pgfpathlineto{\pgfqpoint{3.989347in}{0.741735in}}%
\pgfpathlineto{\pgfqpoint{3.989347in}{0.749686in}}%
\pgfpathlineto{\pgfqpoint{3.990691in}{0.691382in}}%
\pgfpathlineto{\pgfqpoint{3.991364in}{0.691382in}}%
\pgfpathlineto{\pgfqpoint{3.991364in}{0.760287in}}%
\pgfpathlineto{\pgfqpoint{3.992708in}{0.736435in}}%
\pgfpathlineto{\pgfqpoint{3.993380in}{0.736435in}}%
\pgfpathlineto{\pgfqpoint{3.993380in}{0.725834in}}%
\pgfpathlineto{\pgfqpoint{3.994724in}{0.728484in}}%
\pgfpathlineto{\pgfqpoint{3.995397in}{0.728484in}}%
\pgfpathlineto{\pgfqpoint{3.995397in}{0.723184in}}%
\pgfpathlineto{\pgfqpoint{3.996741in}{0.754986in}}%
\pgfpathlineto{\pgfqpoint{3.998085in}{0.754986in}}%
\pgfpathlineto{\pgfqpoint{3.998085in}{0.765587in}}%
\pgfpathlineto{\pgfqpoint{3.999430in}{0.760287in}}%
\pgfpathlineto{\pgfqpoint{4.000102in}{0.760287in}}%
\pgfpathlineto{\pgfqpoint{4.000102in}{0.717884in}}%
\pgfpathlineto{\pgfqpoint{4.001446in}{0.765587in}}%
\pgfpathlineto{\pgfqpoint{4.002118in}{0.765587in}}%
\pgfpathlineto{\pgfqpoint{4.002118in}{0.736435in}}%
\pgfpathlineto{\pgfqpoint{4.003463in}{0.762937in}}%
\pgfpathlineto{\pgfqpoint{4.004807in}{0.762937in}}%
\pgfpathlineto{\pgfqpoint{4.004807in}{0.768237in}}%
\pgfpathlineto{\pgfqpoint{4.005479in}{0.728484in}}%
\pgfpathlineto{\pgfqpoint{4.006151in}{0.733785in}}%
\pgfpathlineto{\pgfqpoint{4.006823in}{0.733785in}}%
\pgfpathlineto{\pgfqpoint{4.006823in}{0.760287in}}%
\pgfpathlineto{\pgfqpoint{4.008168in}{0.754986in}}%
\pgfpathlineto{\pgfqpoint{4.008840in}{0.754986in}}%
\pgfpathlineto{\pgfqpoint{4.008840in}{0.770887in}}%
\pgfpathlineto{\pgfqpoint{4.010184in}{0.749686in}}%
\pgfpathlineto{\pgfqpoint{4.010856in}{0.749686in}}%
\pgfpathlineto{\pgfqpoint{4.010856in}{0.744386in}}%
\pgfpathlineto{\pgfqpoint{4.011529in}{0.765587in}}%
\pgfpathlineto{\pgfqpoint{4.012201in}{0.765587in}}%
\pgfpathlineto{\pgfqpoint{4.012873in}{0.765587in}}%
\pgfpathlineto{\pgfqpoint{4.012873in}{0.760287in}}%
\pgfpathlineto{\pgfqpoint{4.014217in}{0.762937in}}%
\pgfpathlineto{\pgfqpoint{4.014889in}{0.762937in}}%
\pgfpathlineto{\pgfqpoint{4.015562in}{0.725834in}}%
\pgfpathlineto{\pgfqpoint{4.016234in}{0.752336in}}%
\pgfpathlineto{\pgfqpoint{4.016906in}{0.752336in}}%
\pgfpathlineto{\pgfqpoint{4.016906in}{0.741735in}}%
\pgfpathlineto{\pgfqpoint{4.017578in}{0.778838in}}%
\pgfpathlineto{\pgfqpoint{4.018250in}{0.744386in}}%
\pgfpathlineto{\pgfqpoint{4.018922in}{0.744386in}}%
\pgfpathlineto{\pgfqpoint{4.019595in}{0.762937in}}%
\pgfpathlineto{\pgfqpoint{4.020267in}{0.725834in}}%
\pgfpathlineto{\pgfqpoint{4.020939in}{0.725834in}}%
\pgfpathlineto{\pgfqpoint{4.020939in}{0.723184in}}%
\pgfpathlineto{\pgfqpoint{4.022283in}{0.784138in}}%
\pgfpathlineto{\pgfqpoint{4.022955in}{0.784138in}}%
\pgfpathlineto{\pgfqpoint{4.023628in}{0.723184in}}%
\pgfpathlineto{\pgfqpoint{4.024300in}{0.770887in}}%
\pgfpathlineto{\pgfqpoint{4.024972in}{0.770887in}}%
\pgfpathlineto{\pgfqpoint{4.024972in}{0.781488in}}%
\pgfpathlineto{\pgfqpoint{4.025644in}{0.741735in}}%
\pgfpathlineto{\pgfqpoint{4.026316in}{0.757636in}}%
\pgfpathlineto{\pgfqpoint{4.026988in}{0.757636in}}%
\pgfpathlineto{\pgfqpoint{4.026988in}{0.752336in}}%
\pgfpathlineto{\pgfqpoint{4.028333in}{0.768237in}}%
\pgfpathlineto{\pgfqpoint{4.029005in}{0.768237in}}%
\pgfpathlineto{\pgfqpoint{4.029677in}{0.749686in}}%
\pgfpathlineto{\pgfqpoint{4.030349in}{0.752336in}}%
\pgfpathlineto{\pgfqpoint{4.031694in}{0.752336in}}%
\pgfpathlineto{\pgfqpoint{4.031694in}{0.770887in}}%
\pgfpathlineto{\pgfqpoint{4.032366in}{0.728484in}}%
\pgfpathlineto{\pgfqpoint{4.033038in}{0.739085in}}%
\pgfpathlineto{\pgfqpoint{4.033710in}{0.739085in}}%
\pgfpathlineto{\pgfqpoint{4.033710in}{0.786788in}}%
\pgfpathlineto{\pgfqpoint{4.035054in}{0.770887in}}%
\pgfpathlineto{\pgfqpoint{4.035727in}{0.770887in}}%
\pgfpathlineto{\pgfqpoint{4.035727in}{0.728484in}}%
\pgfpathlineto{\pgfqpoint{4.037071in}{0.778838in}}%
\pgfpathlineto{\pgfqpoint{4.037743in}{0.778838in}}%
\pgfpathlineto{\pgfqpoint{4.037743in}{0.784138in}}%
\pgfpathlineto{\pgfqpoint{4.039088in}{0.733785in}}%
\pgfpathlineto{\pgfqpoint{4.039760in}{0.733785in}}%
\pgfpathlineto{\pgfqpoint{4.039760in}{0.770887in}}%
\pgfpathlineto{\pgfqpoint{4.041104in}{0.754986in}}%
\pgfpathlineto{\pgfqpoint{4.041776in}{0.754986in}}%
\pgfpathlineto{\pgfqpoint{4.042448in}{0.731135in}}%
\pgfpathlineto{\pgfqpoint{4.043121in}{0.770887in}}%
\pgfpathlineto{\pgfqpoint{4.043793in}{0.770887in}}%
\pgfpathlineto{\pgfqpoint{4.045137in}{0.739085in}}%
\pgfpathlineto{\pgfqpoint{4.045809in}{0.739085in}}%
\pgfpathlineto{\pgfqpoint{4.047154in}{0.789439in}}%
\pgfpathlineto{\pgfqpoint{4.047826in}{0.789439in}}%
\pgfpathlineto{\pgfqpoint{4.048498in}{0.749686in}}%
\pgfpathlineto{\pgfqpoint{4.049170in}{0.794739in}}%
\pgfpathlineto{\pgfqpoint{4.049842in}{0.794739in}}%
\pgfpathlineto{\pgfqpoint{4.051187in}{0.749686in}}%
\pgfpathlineto{\pgfqpoint{4.051859in}{0.749686in}}%
\pgfpathlineto{\pgfqpoint{4.053203in}{0.805340in}}%
\pgfpathlineto{\pgfqpoint{4.053875in}{0.805340in}}%
\pgfpathlineto{\pgfqpoint{4.055220in}{0.731135in}}%
\pgfpathlineto{\pgfqpoint{4.055892in}{0.731135in}}%
\pgfpathlineto{\pgfqpoint{4.056564in}{0.778838in}}%
\pgfpathlineto{\pgfqpoint{4.057236in}{0.768237in}}%
\pgfpathlineto{\pgfqpoint{4.057908in}{0.768237in}}%
\pgfpathlineto{\pgfqpoint{4.057908in}{0.773537in}}%
\pgfpathlineto{\pgfqpoint{4.059253in}{0.754986in}}%
\pgfpathlineto{\pgfqpoint{4.059925in}{0.754986in}}%
\pgfpathlineto{\pgfqpoint{4.060597in}{0.778838in}}%
\pgfpathlineto{\pgfqpoint{4.061269in}{0.747036in}}%
\pgfpathlineto{\pgfqpoint{4.061941in}{0.747036in}}%
\pgfpathlineto{\pgfqpoint{4.062613in}{0.797389in}}%
\pgfpathlineto{\pgfqpoint{4.063286in}{0.741735in}}%
\pgfpathlineto{\pgfqpoint{4.063958in}{0.741735in}}%
\pgfpathlineto{\pgfqpoint{4.064630in}{0.770887in}}%
\pgfpathlineto{\pgfqpoint{4.065302in}{0.744386in}}%
\pgfpathlineto{\pgfqpoint{4.065974in}{0.744386in}}%
\pgfpathlineto{\pgfqpoint{4.065974in}{0.781488in}}%
\pgfpathlineto{\pgfqpoint{4.067319in}{0.754986in}}%
\pgfpathlineto{\pgfqpoint{4.067991in}{0.754986in}}%
\pgfpathlineto{\pgfqpoint{4.067991in}{0.747036in}}%
\pgfpathlineto{\pgfqpoint{4.068663in}{0.773537in}}%
\pgfpathlineto{\pgfqpoint{4.069335in}{0.747036in}}%
\pgfpathlineto{\pgfqpoint{4.070007in}{0.747036in}}%
\pgfpathlineto{\pgfqpoint{4.070007in}{0.770887in}}%
\pgfpathlineto{\pgfqpoint{4.070679in}{0.736435in}}%
\pgfpathlineto{\pgfqpoint{4.071352in}{0.744386in}}%
\pgfpathlineto{\pgfqpoint{4.072024in}{0.744386in}}%
\pgfpathlineto{\pgfqpoint{4.072024in}{0.736435in}}%
\pgfpathlineto{\pgfqpoint{4.072696in}{0.765587in}}%
\pgfpathlineto{\pgfqpoint{4.073368in}{0.754986in}}%
\pgfpathlineto{\pgfqpoint{4.074040in}{0.754986in}}%
\pgfpathlineto{\pgfqpoint{4.074040in}{0.744386in}}%
\pgfpathlineto{\pgfqpoint{4.075385in}{0.786788in}}%
\pgfpathlineto{\pgfqpoint{4.076057in}{0.786788in}}%
\pgfpathlineto{\pgfqpoint{4.077401in}{0.760287in}}%
\pgfpathlineto{\pgfqpoint{4.078073in}{0.760287in}}%
\pgfpathlineto{\pgfqpoint{4.079418in}{0.805340in}}%
\pgfpathlineto{\pgfqpoint{4.080090in}{0.805340in}}%
\pgfpathlineto{\pgfqpoint{4.080090in}{0.760287in}}%
\pgfpathlineto{\pgfqpoint{4.081434in}{0.762937in}}%
\pgfpathlineto{\pgfqpoint{4.082106in}{0.762937in}}%
\pgfpathlineto{\pgfqpoint{4.082106in}{0.749686in}}%
\pgfpathlineto{\pgfqpoint{4.083451in}{0.770887in}}%
\pgfpathlineto{\pgfqpoint{4.084795in}{0.770887in}}%
\pgfpathlineto{\pgfqpoint{4.085467in}{0.752336in}}%
\pgfpathlineto{\pgfqpoint{4.086139in}{0.754986in}}%
\pgfpathlineto{\pgfqpoint{4.086811in}{0.754986in}}%
\pgfpathlineto{\pgfqpoint{4.087484in}{0.813290in}}%
\pgfpathlineto{\pgfqpoint{4.088156in}{0.760287in}}%
\pgfpathlineto{\pgfqpoint{4.088828in}{0.760287in}}%
\pgfpathlineto{\pgfqpoint{4.088828in}{0.757636in}}%
\pgfpathlineto{\pgfqpoint{4.089500in}{0.786788in}}%
\pgfpathlineto{\pgfqpoint{4.090172in}{0.765587in}}%
\pgfpathlineto{\pgfqpoint{4.090844in}{0.765587in}}%
\pgfpathlineto{\pgfqpoint{4.090844in}{0.757636in}}%
\pgfpathlineto{\pgfqpoint{4.091517in}{0.800039in}}%
\pgfpathlineto{\pgfqpoint{4.092189in}{0.765587in}}%
\pgfpathlineto{\pgfqpoint{4.092861in}{0.765587in}}%
\pgfpathlineto{\pgfqpoint{4.092861in}{0.752336in}}%
\pgfpathlineto{\pgfqpoint{4.093533in}{0.776188in}}%
\pgfpathlineto{\pgfqpoint{4.094205in}{0.770887in}}%
\pgfpathlineto{\pgfqpoint{4.095550in}{0.770887in}}%
\pgfpathlineto{\pgfqpoint{4.095550in}{0.768237in}}%
\pgfpathlineto{\pgfqpoint{4.096894in}{0.778838in}}%
\pgfpathlineto{\pgfqpoint{4.097566in}{0.778838in}}%
\pgfpathlineto{\pgfqpoint{4.097566in}{0.781488in}}%
\pgfpathlineto{\pgfqpoint{4.098238in}{0.739085in}}%
\pgfpathlineto{\pgfqpoint{4.098910in}{0.749686in}}%
\pgfpathlineto{\pgfqpoint{4.099583in}{0.749686in}}%
\pgfpathlineto{\pgfqpoint{4.099583in}{0.741735in}}%
\pgfpathlineto{\pgfqpoint{4.100927in}{0.781488in}}%
\pgfpathlineto{\pgfqpoint{4.101599in}{0.781488in}}%
\pgfpathlineto{\pgfqpoint{4.102271in}{0.784138in}}%
\pgfpathlineto{\pgfqpoint{4.102943in}{0.765587in}}%
\pgfpathlineto{\pgfqpoint{4.103616in}{0.765587in}}%
\pgfpathlineto{\pgfqpoint{4.103616in}{0.749686in}}%
\pgfpathlineto{\pgfqpoint{4.104960in}{0.768237in}}%
\pgfpathlineto{\pgfqpoint{4.105632in}{0.768237in}}%
\pgfpathlineto{\pgfqpoint{4.105632in}{0.752336in}}%
\pgfpathlineto{\pgfqpoint{4.106304in}{0.770887in}}%
\pgfpathlineto{\pgfqpoint{4.106976in}{0.762937in}}%
\pgfpathlineto{\pgfqpoint{4.107649in}{0.762937in}}%
\pgfpathlineto{\pgfqpoint{4.107649in}{0.786788in}}%
\pgfpathlineto{\pgfqpoint{4.108993in}{0.762937in}}%
\pgfpathlineto{\pgfqpoint{4.110337in}{0.762937in}}%
\pgfpathlineto{\pgfqpoint{4.110337in}{0.754986in}}%
\pgfpathlineto{\pgfqpoint{4.111009in}{0.792089in}}%
\pgfpathlineto{\pgfqpoint{4.111682in}{0.770887in}}%
\pgfpathlineto{\pgfqpoint{4.112354in}{0.770887in}}%
\pgfpathlineto{\pgfqpoint{4.112354in}{0.815940in}}%
\pgfpathlineto{\pgfqpoint{4.113698in}{0.776188in}}%
\pgfpathlineto{\pgfqpoint{4.114370in}{0.776188in}}%
\pgfpathlineto{\pgfqpoint{4.114370in}{0.760287in}}%
\pgfpathlineto{\pgfqpoint{4.115715in}{0.786788in}}%
\pgfpathlineto{\pgfqpoint{4.116387in}{0.786788in}}%
\pgfpathlineto{\pgfqpoint{4.117731in}{0.826541in}}%
\pgfpathlineto{\pgfqpoint{4.118403in}{0.826541in}}%
\pgfpathlineto{\pgfqpoint{4.119075in}{0.717884in}}%
\pgfpathlineto{\pgfqpoint{4.119748in}{0.754986in}}%
\pgfpathlineto{\pgfqpoint{4.120420in}{0.754986in}}%
\pgfpathlineto{\pgfqpoint{4.121092in}{0.797389in}}%
\pgfpathlineto{\pgfqpoint{4.121764in}{0.781488in}}%
\pgfpathlineto{\pgfqpoint{4.122436in}{0.781488in}}%
\pgfpathlineto{\pgfqpoint{4.123781in}{0.747036in}}%
\pgfpathlineto{\pgfqpoint{4.124453in}{0.747036in}}%
\pgfpathlineto{\pgfqpoint{4.124453in}{0.784138in}}%
\pgfpathlineto{\pgfqpoint{4.125797in}{0.773537in}}%
\pgfpathlineto{\pgfqpoint{4.126469in}{0.773537in}}%
\pgfpathlineto{\pgfqpoint{4.126469in}{0.789439in}}%
\pgfpathlineto{\pgfqpoint{4.127814in}{0.739085in}}%
\pgfpathlineto{\pgfqpoint{4.128486in}{0.739085in}}%
\pgfpathlineto{\pgfqpoint{4.128486in}{0.810640in}}%
\pgfpathlineto{\pgfqpoint{4.129830in}{0.754986in}}%
\pgfpathlineto{\pgfqpoint{4.130502in}{0.754986in}}%
\pgfpathlineto{\pgfqpoint{4.131847in}{0.797389in}}%
\pgfpathlineto{\pgfqpoint{4.132519in}{0.797389in}}%
\pgfpathlineto{\pgfqpoint{4.133863in}{0.754986in}}%
\pgfpathlineto{\pgfqpoint{4.134535in}{0.754986in}}%
\pgfpathlineto{\pgfqpoint{4.135207in}{0.773537in}}%
\pgfpathlineto{\pgfqpoint{4.135880in}{0.728484in}}%
\pgfpathlineto{\pgfqpoint{4.136552in}{0.728484in}}%
\pgfpathlineto{\pgfqpoint{4.136552in}{0.784138in}}%
\pgfpathlineto{\pgfqpoint{4.137896in}{0.765587in}}%
\pgfpathlineto{\pgfqpoint{4.138568in}{0.765587in}}%
\pgfpathlineto{\pgfqpoint{4.138568in}{0.728484in}}%
\pgfpathlineto{\pgfqpoint{4.139240in}{0.773537in}}%
\pgfpathlineto{\pgfqpoint{4.139913in}{0.773537in}}%
\pgfpathlineto{\pgfqpoint{4.140585in}{0.773537in}}%
\pgfpathlineto{\pgfqpoint{4.141257in}{0.757636in}}%
\pgfpathlineto{\pgfqpoint{4.141929in}{0.794739in}}%
\pgfpathlineto{\pgfqpoint{4.142601in}{0.794739in}}%
\pgfpathlineto{\pgfqpoint{4.142601in}{0.815940in}}%
\pgfpathlineto{\pgfqpoint{4.143273in}{0.754986in}}%
\pgfpathlineto{\pgfqpoint{4.143946in}{0.754986in}}%
\pgfpathlineto{\pgfqpoint{4.144618in}{0.754986in}}%
\pgfpathlineto{\pgfqpoint{4.144618in}{0.739085in}}%
\pgfpathlineto{\pgfqpoint{4.145962in}{0.789439in}}%
\pgfpathlineto{\pgfqpoint{4.146634in}{0.789439in}}%
\pgfpathlineto{\pgfqpoint{4.146634in}{0.778838in}}%
\pgfpathlineto{\pgfqpoint{4.147979in}{0.784138in}}%
\pgfpathlineto{\pgfqpoint{4.149323in}{0.784138in}}%
\pgfpathlineto{\pgfqpoint{4.149995in}{0.786788in}}%
\pgfpathlineto{\pgfqpoint{4.150667in}{0.768237in}}%
\pgfpathlineto{\pgfqpoint{4.151339in}{0.768237in}}%
\pgfpathlineto{\pgfqpoint{4.152012in}{0.770887in}}%
\pgfpathlineto{\pgfqpoint{4.152684in}{0.752336in}}%
\pgfpathlineto{\pgfqpoint{4.153356in}{0.752336in}}%
\pgfpathlineto{\pgfqpoint{4.153356in}{0.747036in}}%
\pgfpathlineto{\pgfqpoint{4.154028in}{0.765587in}}%
\pgfpathlineto{\pgfqpoint{4.154700in}{0.749686in}}%
\pgfpathlineto{\pgfqpoint{4.155373in}{0.749686in}}%
\pgfpathlineto{\pgfqpoint{4.155373in}{0.789439in}}%
\pgfpathlineto{\pgfqpoint{4.156717in}{0.778838in}}%
\pgfpathlineto{\pgfqpoint{4.157389in}{0.778838in}}%
\pgfpathlineto{\pgfqpoint{4.157389in}{0.773537in}}%
\pgfpathlineto{\pgfqpoint{4.158733in}{0.786788in}}%
\pgfpathlineto{\pgfqpoint{4.159406in}{0.786788in}}%
\pgfpathlineto{\pgfqpoint{4.160078in}{0.765587in}}%
\pgfpathlineto{\pgfqpoint{4.160750in}{0.770887in}}%
\pgfpathlineto{\pgfqpoint{4.161422in}{0.770887in}}%
\pgfpathlineto{\pgfqpoint{4.161422in}{0.747036in}}%
\pgfpathlineto{\pgfqpoint{4.162766in}{0.765587in}}%
\pgfpathlineto{\pgfqpoint{4.163439in}{0.765587in}}%
\pgfpathlineto{\pgfqpoint{4.163439in}{0.821241in}}%
\pgfpathlineto{\pgfqpoint{4.164783in}{0.773537in}}%
\pgfpathlineto{\pgfqpoint{4.165455in}{0.773537in}}%
\pgfpathlineto{\pgfqpoint{4.165455in}{0.800039in}}%
\pgfpathlineto{\pgfqpoint{4.166799in}{0.736435in}}%
\pgfpathlineto{\pgfqpoint{4.167472in}{0.736435in}}%
\pgfpathlineto{\pgfqpoint{4.168816in}{0.786788in}}%
\pgfpathlineto{\pgfqpoint{4.169488in}{0.786788in}}%
\pgfpathlineto{\pgfqpoint{4.170160in}{0.754986in}}%
\pgfpathlineto{\pgfqpoint{4.170832in}{0.789439in}}%
\pgfpathlineto{\pgfqpoint{4.171505in}{0.789439in}}%
\pgfpathlineto{\pgfqpoint{4.172177in}{0.765587in}}%
\pgfpathlineto{\pgfqpoint{4.172849in}{0.768237in}}%
\pgfpathlineto{\pgfqpoint{4.173521in}{0.768237in}}%
\pgfpathlineto{\pgfqpoint{4.173521in}{0.807990in}}%
\pgfpathlineto{\pgfqpoint{4.174193in}{0.752336in}}%
\pgfpathlineto{\pgfqpoint{4.174865in}{0.762937in}}%
\pgfpathlineto{\pgfqpoint{4.175538in}{0.762937in}}%
\pgfpathlineto{\pgfqpoint{4.175538in}{0.731135in}}%
\pgfpathlineto{\pgfqpoint{4.176210in}{0.781488in}}%
\pgfpathlineto{\pgfqpoint{4.176882in}{0.762937in}}%
\pgfpathlineto{\pgfqpoint{4.177554in}{0.762937in}}%
\pgfpathlineto{\pgfqpoint{4.178226in}{0.797389in}}%
\pgfpathlineto{\pgfqpoint{4.178898in}{0.786788in}}%
\pgfpathlineto{\pgfqpoint{4.179571in}{0.786788in}}%
\pgfpathlineto{\pgfqpoint{4.179571in}{0.792089in}}%
\pgfpathlineto{\pgfqpoint{4.180243in}{0.728484in}}%
\pgfpathlineto{\pgfqpoint{4.180915in}{0.768237in}}%
\pgfpathlineto{\pgfqpoint{4.181587in}{0.768237in}}%
\pgfpathlineto{\pgfqpoint{4.182259in}{0.752336in}}%
\pgfpathlineto{\pgfqpoint{4.182931in}{0.792089in}}%
\pgfpathlineto{\pgfqpoint{4.183604in}{0.792089in}}%
\pgfpathlineto{\pgfqpoint{4.184276in}{0.797389in}}%
\pgfpathlineto{\pgfqpoint{4.184948in}{0.741735in}}%
\pgfpathlineto{\pgfqpoint{4.185620in}{0.741735in}}%
\pgfpathlineto{\pgfqpoint{4.185620in}{0.794739in}}%
\pgfpathlineto{\pgfqpoint{4.186964in}{0.757636in}}%
\pgfpathlineto{\pgfqpoint{4.187637in}{0.757636in}}%
\pgfpathlineto{\pgfqpoint{4.188309in}{0.784138in}}%
\pgfpathlineto{\pgfqpoint{4.188981in}{0.757636in}}%
\pgfpathlineto{\pgfqpoint{4.189653in}{0.757636in}}%
\pgfpathlineto{\pgfqpoint{4.190325in}{0.778838in}}%
\pgfpathlineto{\pgfqpoint{4.190997in}{0.778838in}}%
\pgfpathlineto{\pgfqpoint{4.191670in}{0.778838in}}%
\pgfpathlineto{\pgfqpoint{4.193014in}{0.757636in}}%
\pgfpathlineto{\pgfqpoint{4.193686in}{0.757636in}}%
\pgfpathlineto{\pgfqpoint{4.193686in}{0.770887in}}%
\pgfpathlineto{\pgfqpoint{4.194358in}{0.731135in}}%
\pgfpathlineto{\pgfqpoint{4.195030in}{0.760287in}}%
\pgfpathlineto{\pgfqpoint{4.195703in}{0.760287in}}%
\pgfpathlineto{\pgfqpoint{4.196375in}{0.770887in}}%
\pgfpathlineto{\pgfqpoint{4.197047in}{0.752336in}}%
\pgfpathlineto{\pgfqpoint{4.198391in}{0.752336in}}%
\pgfpathlineto{\pgfqpoint{4.198391in}{0.770887in}}%
\pgfpathlineto{\pgfqpoint{4.199736in}{0.757636in}}%
\pgfpathlineto{\pgfqpoint{4.200408in}{0.757636in}}%
\pgfpathlineto{\pgfqpoint{4.200408in}{0.784138in}}%
\pgfpathlineto{\pgfqpoint{4.201752in}{0.776188in}}%
\pgfpathlineto{\pgfqpoint{4.202424in}{0.776188in}}%
\pgfpathlineto{\pgfqpoint{4.202424in}{0.741735in}}%
\pgfpathlineto{\pgfqpoint{4.203096in}{0.784138in}}%
\pgfpathlineto{\pgfqpoint{4.203769in}{0.778838in}}%
\pgfpathlineto{\pgfqpoint{4.204441in}{0.778838in}}%
\pgfpathlineto{\pgfqpoint{4.205113in}{0.731135in}}%
\pgfpathlineto{\pgfqpoint{4.205113in}{0.781488in}}%
\pgfpathlineto{\pgfqpoint{4.205785in}{0.757636in}}%
\pgfpathlineto{\pgfqpoint{4.206457in}{0.757636in}}%
\pgfpathlineto{\pgfqpoint{4.207129in}{0.765587in}}%
\pgfpathlineto{\pgfqpoint{4.207802in}{0.744386in}}%
\pgfpathlineto{\pgfqpoint{4.208474in}{0.744386in}}%
\pgfpathlineto{\pgfqpoint{4.208474in}{0.802689in}}%
\pgfpathlineto{\pgfqpoint{4.209818in}{0.765587in}}%
\pgfpathlineto{\pgfqpoint{4.210490in}{0.765587in}}%
\pgfpathlineto{\pgfqpoint{4.210490in}{0.749686in}}%
\pgfpathlineto{\pgfqpoint{4.211162in}{0.802689in}}%
\pgfpathlineto{\pgfqpoint{4.211835in}{0.786788in}}%
\pgfpathlineto{\pgfqpoint{4.212507in}{0.786788in}}%
\pgfpathlineto{\pgfqpoint{4.213179in}{0.810640in}}%
\pgfpathlineto{\pgfqpoint{4.213851in}{0.768237in}}%
\pgfpathlineto{\pgfqpoint{4.214523in}{0.768237in}}%
\pgfpathlineto{\pgfqpoint{4.215195in}{0.792089in}}%
\pgfpathlineto{\pgfqpoint{4.215868in}{0.747036in}}%
\pgfpathlineto{\pgfqpoint{4.216540in}{0.747036in}}%
\pgfpathlineto{\pgfqpoint{4.217212in}{0.781488in}}%
\pgfpathlineto{\pgfqpoint{4.217884in}{0.757636in}}%
\pgfpathlineto{\pgfqpoint{4.219228in}{0.757636in}}%
\pgfpathlineto{\pgfqpoint{4.219901in}{0.786788in}}%
\pgfpathlineto{\pgfqpoint{4.220573in}{0.747036in}}%
\pgfpathlineto{\pgfqpoint{4.221245in}{0.747036in}}%
\pgfpathlineto{\pgfqpoint{4.222589in}{0.770887in}}%
\pgfpathlineto{\pgfqpoint{4.223261in}{0.770887in}}%
\pgfpathlineto{\pgfqpoint{4.223261in}{0.792089in}}%
\pgfpathlineto{\pgfqpoint{4.224606in}{0.784138in}}%
\pgfpathlineto{\pgfqpoint{4.225278in}{0.784138in}}%
\pgfpathlineto{\pgfqpoint{4.225278in}{0.797389in}}%
\pgfpathlineto{\pgfqpoint{4.225950in}{0.739085in}}%
\pgfpathlineto{\pgfqpoint{4.226622in}{0.773537in}}%
\pgfpathlineto{\pgfqpoint{4.227294in}{0.773537in}}%
\pgfpathlineto{\pgfqpoint{4.227967in}{0.786788in}}%
\pgfpathlineto{\pgfqpoint{4.228639in}{0.781488in}}%
\pgfpathlineto{\pgfqpoint{4.229311in}{0.781488in}}%
\pgfpathlineto{\pgfqpoint{4.229311in}{0.762937in}}%
\pgfpathlineto{\pgfqpoint{4.230655in}{0.792089in}}%
\pgfpathlineto{\pgfqpoint{4.231327in}{0.792089in}}%
\pgfpathlineto{\pgfqpoint{4.232672in}{0.749686in}}%
\pgfpathlineto{\pgfqpoint{4.233344in}{0.749686in}}%
\pgfpathlineto{\pgfqpoint{4.233344in}{0.744386in}}%
\pgfpathlineto{\pgfqpoint{4.234016in}{0.757636in}}%
\pgfpathlineto{\pgfqpoint{4.234688in}{0.752336in}}%
\pgfpathlineto{\pgfqpoint{4.235360in}{0.752336in}}%
\pgfpathlineto{\pgfqpoint{4.235360in}{0.760287in}}%
\pgfpathlineto{\pgfqpoint{4.236705in}{0.731135in}}%
\pgfpathlineto{\pgfqpoint{4.237377in}{0.731135in}}%
\pgfpathlineto{\pgfqpoint{4.238049in}{0.754986in}}%
\pgfpathlineto{\pgfqpoint{4.238721in}{0.739085in}}%
\pgfpathlineto{\pgfqpoint{4.239393in}{0.739085in}}%
\pgfpathlineto{\pgfqpoint{4.240738in}{0.784138in}}%
\pgfpathlineto{\pgfqpoint{4.241410in}{0.784138in}}%
\pgfpathlineto{\pgfqpoint{4.241410in}{0.752336in}}%
\pgfpathlineto{\pgfqpoint{4.242754in}{0.786788in}}%
\pgfpathlineto{\pgfqpoint{4.243426in}{0.786788in}}%
\pgfpathlineto{\pgfqpoint{4.243426in}{0.818590in}}%
\pgfpathlineto{\pgfqpoint{4.244099in}{0.762937in}}%
\pgfpathlineto{\pgfqpoint{4.244771in}{0.778838in}}%
\pgfpathlineto{\pgfqpoint{4.245443in}{0.778838in}}%
\pgfpathlineto{\pgfqpoint{4.245443in}{0.762937in}}%
\pgfpathlineto{\pgfqpoint{4.246787in}{0.813290in}}%
\pgfpathlineto{\pgfqpoint{4.247459in}{0.813290in}}%
\pgfpathlineto{\pgfqpoint{4.247459in}{0.741735in}}%
\pgfpathlineto{\pgfqpoint{4.248804in}{0.792089in}}%
\pgfpathlineto{\pgfqpoint{4.249476in}{0.792089in}}%
\pgfpathlineto{\pgfqpoint{4.250820in}{0.736435in}}%
\pgfpathlineto{\pgfqpoint{4.251492in}{0.736435in}}%
\pgfpathlineto{\pgfqpoint{4.252165in}{0.786788in}}%
\pgfpathlineto{\pgfqpoint{4.252837in}{0.720534in}}%
\pgfpathlineto{\pgfqpoint{4.253509in}{0.720534in}}%
\pgfpathlineto{\pgfqpoint{4.254181in}{0.770887in}}%
\pgfpathlineto{\pgfqpoint{4.254853in}{0.757636in}}%
\pgfpathlineto{\pgfqpoint{4.255525in}{0.757636in}}%
\pgfpathlineto{\pgfqpoint{4.256198in}{0.741735in}}%
\pgfpathlineto{\pgfqpoint{4.256870in}{0.784138in}}%
\pgfpathlineto{\pgfqpoint{4.257542in}{0.784138in}}%
\pgfpathlineto{\pgfqpoint{4.258214in}{0.749686in}}%
\pgfpathlineto{\pgfqpoint{4.258886in}{0.805340in}}%
\pgfpathlineto{\pgfqpoint{4.259558in}{0.805340in}}%
\pgfpathlineto{\pgfqpoint{4.259558in}{0.765587in}}%
\pgfpathlineto{\pgfqpoint{4.260903in}{0.773537in}}%
\pgfpathlineto{\pgfqpoint{4.261575in}{0.773537in}}%
\pgfpathlineto{\pgfqpoint{4.262247in}{0.723184in}}%
\pgfpathlineto{\pgfqpoint{4.262919in}{0.760287in}}%
\pgfpathlineto{\pgfqpoint{4.264264in}{0.760287in}}%
\pgfpathlineto{\pgfqpoint{4.264264in}{0.752336in}}%
\pgfpathlineto{\pgfqpoint{4.264936in}{0.765587in}}%
\pgfpathlineto{\pgfqpoint{4.265608in}{0.752336in}}%
\pgfpathlineto{\pgfqpoint{4.266280in}{0.752336in}}%
\pgfpathlineto{\pgfqpoint{4.266280in}{0.770887in}}%
\pgfpathlineto{\pgfqpoint{4.267625in}{0.733785in}}%
\pgfpathlineto{\pgfqpoint{4.268297in}{0.733785in}}%
\pgfpathlineto{\pgfqpoint{4.268969in}{0.784138in}}%
\pgfpathlineto{\pgfqpoint{4.269641in}{0.733785in}}%
\pgfpathlineto{\pgfqpoint{4.270313in}{0.733785in}}%
\pgfpathlineto{\pgfqpoint{4.270985in}{0.794739in}}%
\pgfpathlineto{\pgfqpoint{4.271658in}{0.741735in}}%
\pgfpathlineto{\pgfqpoint{4.272330in}{0.741735in}}%
\pgfpathlineto{\pgfqpoint{4.273002in}{0.789439in}}%
\pgfpathlineto{\pgfqpoint{4.273674in}{0.728484in}}%
\pgfpathlineto{\pgfqpoint{4.274346in}{0.728484in}}%
\pgfpathlineto{\pgfqpoint{4.274346in}{0.754986in}}%
\pgfpathlineto{\pgfqpoint{4.275691in}{0.747036in}}%
\pgfpathlineto{\pgfqpoint{4.276363in}{0.747036in}}%
\pgfpathlineto{\pgfqpoint{4.277035in}{0.800039in}}%
\pgfpathlineto{\pgfqpoint{4.277707in}{0.752336in}}%
\pgfpathlineto{\pgfqpoint{4.278379in}{0.752336in}}%
\pgfpathlineto{\pgfqpoint{4.278379in}{0.768237in}}%
\pgfpathlineto{\pgfqpoint{4.279051in}{0.736435in}}%
\pgfpathlineto{\pgfqpoint{4.279724in}{0.760287in}}%
\pgfpathlineto{\pgfqpoint{4.280396in}{0.760287in}}%
\pgfpathlineto{\pgfqpoint{4.280396in}{0.752336in}}%
\pgfpathlineto{\pgfqpoint{4.281068in}{0.778838in}}%
\pgfpathlineto{\pgfqpoint{4.281740in}{0.757636in}}%
\pgfpathlineto{\pgfqpoint{4.282412in}{0.757636in}}%
\pgfpathlineto{\pgfqpoint{4.283084in}{0.765587in}}%
\pgfpathlineto{\pgfqpoint{4.283757in}{0.739085in}}%
\pgfpathlineto{\pgfqpoint{4.284429in}{0.739085in}}%
\pgfpathlineto{\pgfqpoint{4.284429in}{0.736435in}}%
\pgfpathlineto{\pgfqpoint{4.285773in}{0.778838in}}%
\pgfpathlineto{\pgfqpoint{4.286445in}{0.778838in}}%
\pgfpathlineto{\pgfqpoint{4.286445in}{0.749686in}}%
\pgfpathlineto{\pgfqpoint{4.287790in}{0.773537in}}%
\pgfpathlineto{\pgfqpoint{4.288462in}{0.773537in}}%
\pgfpathlineto{\pgfqpoint{4.289134in}{0.778838in}}%
\pgfpathlineto{\pgfqpoint{4.289806in}{0.757636in}}%
\pgfpathlineto{\pgfqpoint{4.290478in}{0.757636in}}%
\pgfpathlineto{\pgfqpoint{4.290478in}{0.747036in}}%
\pgfpathlineto{\pgfqpoint{4.291823in}{0.778838in}}%
\pgfpathlineto{\pgfqpoint{4.292495in}{0.778838in}}%
\pgfpathlineto{\pgfqpoint{4.293167in}{0.792089in}}%
\pgfpathlineto{\pgfqpoint{4.293839in}{0.747036in}}%
\pgfpathlineto{\pgfqpoint{4.294511in}{0.747036in}}%
\pgfpathlineto{\pgfqpoint{4.295183in}{0.778838in}}%
\pgfpathlineto{\pgfqpoint{4.295856in}{0.778838in}}%
\pgfpathlineto{\pgfqpoint{4.297200in}{0.778838in}}%
\pgfpathlineto{\pgfqpoint{4.297872in}{0.731135in}}%
\pgfpathlineto{\pgfqpoint{4.297872in}{0.781488in}}%
\pgfpathlineto{\pgfqpoint{4.298544in}{0.741735in}}%
\pgfpathlineto{\pgfqpoint{4.299216in}{0.741735in}}%
\pgfpathlineto{\pgfqpoint{4.300561in}{0.770887in}}%
\pgfpathlineto{\pgfqpoint{4.301233in}{0.770887in}}%
\pgfpathlineto{\pgfqpoint{4.301905in}{0.731135in}}%
\pgfpathlineto{\pgfqpoint{4.302577in}{0.749686in}}%
\pgfpathlineto{\pgfqpoint{4.303249in}{0.749686in}}%
\pgfpathlineto{\pgfqpoint{4.303922in}{0.776188in}}%
\pgfpathlineto{\pgfqpoint{4.304594in}{0.760287in}}%
\pgfpathlineto{\pgfqpoint{4.305266in}{0.760287in}}%
\pgfpathlineto{\pgfqpoint{4.305266in}{0.741735in}}%
\pgfpathlineto{\pgfqpoint{4.306610in}{0.770887in}}%
\pgfpathlineto{\pgfqpoint{4.307282in}{0.770887in}}%
\pgfpathlineto{\pgfqpoint{4.307282in}{0.757636in}}%
\pgfpathlineto{\pgfqpoint{4.307955in}{0.773537in}}%
\pgfpathlineto{\pgfqpoint{4.308627in}{0.773537in}}%
\pgfpathlineto{\pgfqpoint{4.309299in}{0.773537in}}%
\pgfpathlineto{\pgfqpoint{4.309971in}{0.776188in}}%
\pgfpathlineto{\pgfqpoint{4.310643in}{0.731135in}}%
\pgfpathlineto{\pgfqpoint{4.311315in}{0.731135in}}%
\pgfpathlineto{\pgfqpoint{4.311315in}{0.805340in}}%
\pgfpathlineto{\pgfqpoint{4.312660in}{0.747036in}}%
\pgfpathlineto{\pgfqpoint{4.313332in}{0.747036in}}%
\pgfpathlineto{\pgfqpoint{4.314004in}{0.794739in}}%
\pgfpathlineto{\pgfqpoint{4.314676in}{0.739085in}}%
\pgfpathlineto{\pgfqpoint{4.315348in}{0.739085in}}%
\pgfpathlineto{\pgfqpoint{4.315348in}{0.754986in}}%
\pgfpathlineto{\pgfqpoint{4.316693in}{0.739085in}}%
\pgfpathlineto{\pgfqpoint{4.317365in}{0.739085in}}%
\pgfpathlineto{\pgfqpoint{4.318037in}{0.778838in}}%
\pgfpathlineto{\pgfqpoint{4.318709in}{0.773537in}}%
\pgfpathlineto{\pgfqpoint{4.319381in}{0.773537in}}%
\pgfpathlineto{\pgfqpoint{4.319381in}{0.789439in}}%
\pgfpathlineto{\pgfqpoint{4.320726in}{0.757636in}}%
\pgfpathlineto{\pgfqpoint{4.321398in}{0.757636in}}%
\pgfpathlineto{\pgfqpoint{4.322070in}{0.754986in}}%
\pgfpathlineto{\pgfqpoint{4.322742in}{0.781488in}}%
\pgfpathlineto{\pgfqpoint{4.323414in}{0.781488in}}%
\pgfpathlineto{\pgfqpoint{4.324087in}{0.810640in}}%
\pgfpathlineto{\pgfqpoint{4.324759in}{0.741735in}}%
\pgfpathlineto{\pgfqpoint{4.325431in}{0.741735in}}%
\pgfpathlineto{\pgfqpoint{4.325431in}{0.757636in}}%
\pgfpathlineto{\pgfqpoint{4.326103in}{0.739085in}}%
\pgfpathlineto{\pgfqpoint{4.326775in}{0.749686in}}%
\pgfpathlineto{\pgfqpoint{4.327447in}{0.749686in}}%
\pgfpathlineto{\pgfqpoint{4.328120in}{0.739085in}}%
\pgfpathlineto{\pgfqpoint{4.328792in}{0.770887in}}%
\pgfpathlineto{\pgfqpoint{4.329464in}{0.770887in}}%
\pgfpathlineto{\pgfqpoint{4.329464in}{0.731135in}}%
\pgfpathlineto{\pgfqpoint{4.330808in}{0.754986in}}%
\pgfpathlineto{\pgfqpoint{4.331480in}{0.754986in}}%
\pgfpathlineto{\pgfqpoint{4.331480in}{0.739085in}}%
\pgfpathlineto{\pgfqpoint{4.332825in}{0.776188in}}%
\pgfpathlineto{\pgfqpoint{4.333497in}{0.776188in}}%
\pgfpathlineto{\pgfqpoint{4.334169in}{0.800039in}}%
\pgfpathlineto{\pgfqpoint{4.334841in}{0.736435in}}%
\pgfpathlineto{\pgfqpoint{4.335513in}{0.736435in}}%
\pgfpathlineto{\pgfqpoint{4.336186in}{0.805340in}}%
\pgfpathlineto{\pgfqpoint{4.336858in}{0.736435in}}%
\pgfpathlineto{\pgfqpoint{4.337530in}{0.736435in}}%
\pgfpathlineto{\pgfqpoint{4.337530in}{0.731135in}}%
\pgfpathlineto{\pgfqpoint{4.338874in}{0.776188in}}%
\pgfpathlineto{\pgfqpoint{4.339546in}{0.776188in}}%
\pgfpathlineto{\pgfqpoint{4.339546in}{0.754986in}}%
\pgfpathlineto{\pgfqpoint{4.340891in}{0.794739in}}%
\pgfpathlineto{\pgfqpoint{4.341563in}{0.794739in}}%
\pgfpathlineto{\pgfqpoint{4.341563in}{0.752336in}}%
\pgfpathlineto{\pgfqpoint{4.342907in}{0.765587in}}%
\pgfpathlineto{\pgfqpoint{4.343579in}{0.765587in}}%
\pgfpathlineto{\pgfqpoint{4.343579in}{0.786788in}}%
\pgfpathlineto{\pgfqpoint{4.344924in}{0.733785in}}%
\pgfpathlineto{\pgfqpoint{4.345596in}{0.733785in}}%
\pgfpathlineto{\pgfqpoint{4.346268in}{0.776188in}}%
\pgfpathlineto{\pgfqpoint{4.346940in}{0.765587in}}%
\pgfpathlineto{\pgfqpoint{4.347612in}{0.765587in}}%
\pgfpathlineto{\pgfqpoint{4.348285in}{0.749686in}}%
\pgfpathlineto{\pgfqpoint{4.348957in}{0.749686in}}%
\pgfpathlineto{\pgfqpoint{4.349629in}{0.749686in}}%
\pgfpathlineto{\pgfqpoint{4.350973in}{0.807990in}}%
\pgfpathlineto{\pgfqpoint{4.351645in}{0.807990in}}%
\pgfpathlineto{\pgfqpoint{4.352990in}{0.731135in}}%
\pgfpathlineto{\pgfqpoint{4.353662in}{0.731135in}}%
\pgfpathlineto{\pgfqpoint{4.355006in}{0.781488in}}%
\pgfpathlineto{\pgfqpoint{4.355678in}{0.781488in}}%
\pgfpathlineto{\pgfqpoint{4.355678in}{0.794739in}}%
\pgfpathlineto{\pgfqpoint{4.357023in}{0.736435in}}%
\pgfpathlineto{\pgfqpoint{4.357695in}{0.736435in}}%
\pgfpathlineto{\pgfqpoint{4.357695in}{0.731135in}}%
\pgfpathlineto{\pgfqpoint{4.359039in}{0.757636in}}%
\pgfpathlineto{\pgfqpoint{4.359711in}{0.757636in}}%
\pgfpathlineto{\pgfqpoint{4.360384in}{0.770887in}}%
\pgfpathlineto{\pgfqpoint{4.361056in}{0.741735in}}%
\pgfpathlineto{\pgfqpoint{4.361728in}{0.741735in}}%
\pgfpathlineto{\pgfqpoint{4.361728in}{0.723184in}}%
\pgfpathlineto{\pgfqpoint{4.363072in}{0.752336in}}%
\pgfpathlineto{\pgfqpoint{4.363744in}{0.752336in}}%
\pgfpathlineto{\pgfqpoint{4.363744in}{0.744386in}}%
\pgfpathlineto{\pgfqpoint{4.365089in}{0.773537in}}%
\pgfpathlineto{\pgfqpoint{4.365761in}{0.773537in}}%
\pgfpathlineto{\pgfqpoint{4.365761in}{0.781488in}}%
\pgfpathlineto{\pgfqpoint{4.367105in}{0.752336in}}%
\pgfpathlineto{\pgfqpoint{4.369122in}{0.752336in}}%
\pgfpathlineto{\pgfqpoint{4.369794in}{0.731135in}}%
\pgfpathlineto{\pgfqpoint{4.370466in}{0.805340in}}%
\pgfpathlineto{\pgfqpoint{4.371138in}{0.805340in}}%
\pgfpathlineto{\pgfqpoint{4.371138in}{0.741735in}}%
\pgfpathlineto{\pgfqpoint{4.372483in}{0.747036in}}%
\pgfpathlineto{\pgfqpoint{4.373155in}{0.747036in}}%
\pgfpathlineto{\pgfqpoint{4.373155in}{0.762937in}}%
\pgfpathlineto{\pgfqpoint{4.374499in}{0.725834in}}%
\pgfpathlineto{\pgfqpoint{4.375171in}{0.725834in}}%
\pgfpathlineto{\pgfqpoint{4.376516in}{0.768237in}}%
\pgfpathlineto{\pgfqpoint{4.377188in}{0.768237in}}%
\pgfpathlineto{\pgfqpoint{4.377860in}{0.739085in}}%
\pgfpathlineto{\pgfqpoint{4.378532in}{0.757636in}}%
\pgfpathlineto{\pgfqpoint{4.379204in}{0.757636in}}%
\pgfpathlineto{\pgfqpoint{4.379204in}{0.770887in}}%
\pgfpathlineto{\pgfqpoint{4.379877in}{0.723184in}}%
\pgfpathlineto{\pgfqpoint{4.380549in}{0.760287in}}%
\pgfpathlineto{\pgfqpoint{4.381221in}{0.760287in}}%
\pgfpathlineto{\pgfqpoint{4.381221in}{0.792089in}}%
\pgfpathlineto{\pgfqpoint{4.382565in}{0.768237in}}%
\pgfpathlineto{\pgfqpoint{4.383237in}{0.768237in}}%
\pgfpathlineto{\pgfqpoint{4.383910in}{0.717884in}}%
\pgfpathlineto{\pgfqpoint{4.384582in}{0.749686in}}%
\pgfpathlineto{\pgfqpoint{4.385254in}{0.749686in}}%
\pgfpathlineto{\pgfqpoint{4.385254in}{0.741735in}}%
\pgfpathlineto{\pgfqpoint{4.385926in}{0.776188in}}%
\pgfpathlineto{\pgfqpoint{4.386598in}{0.768237in}}%
\pgfpathlineto{\pgfqpoint{4.387270in}{0.768237in}}%
\pgfpathlineto{\pgfqpoint{4.387943in}{0.744386in}}%
\pgfpathlineto{\pgfqpoint{4.388615in}{0.770887in}}%
\pgfpathlineto{\pgfqpoint{4.389287in}{0.770887in}}%
\pgfpathlineto{\pgfqpoint{4.389287in}{0.728484in}}%
\pgfpathlineto{\pgfqpoint{4.390631in}{0.765587in}}%
\pgfpathlineto{\pgfqpoint{4.391303in}{0.765587in}}%
\pgfpathlineto{\pgfqpoint{4.391303in}{0.723184in}}%
\pgfpathlineto{\pgfqpoint{4.392648in}{0.765587in}}%
\pgfpathlineto{\pgfqpoint{4.393320in}{0.765587in}}%
\pgfpathlineto{\pgfqpoint{4.393320in}{0.728484in}}%
\pgfpathlineto{\pgfqpoint{4.394664in}{0.741735in}}%
\pgfpathlineto{\pgfqpoint{4.395336in}{0.741735in}}%
\pgfpathlineto{\pgfqpoint{4.395336in}{0.757636in}}%
\pgfpathlineto{\pgfqpoint{4.396009in}{0.736435in}}%
\pgfpathlineto{\pgfqpoint{4.396681in}{0.736435in}}%
\pgfpathlineto{\pgfqpoint{4.397353in}{0.736435in}}%
\pgfpathlineto{\pgfqpoint{4.398025in}{0.760287in}}%
\pgfpathlineto{\pgfqpoint{4.398697in}{0.731135in}}%
\pgfpathlineto{\pgfqpoint{4.399369in}{0.731135in}}%
\pgfpathlineto{\pgfqpoint{4.399369in}{0.762937in}}%
\pgfpathlineto{\pgfqpoint{4.400714in}{0.731135in}}%
\pgfpathlineto{\pgfqpoint{4.401386in}{0.731135in}}%
\pgfpathlineto{\pgfqpoint{4.402058in}{0.723184in}}%
\pgfpathlineto{\pgfqpoint{4.402730in}{0.781488in}}%
\pgfpathlineto{\pgfqpoint{4.403402in}{0.781488in}}%
\pgfpathlineto{\pgfqpoint{4.403402in}{0.789439in}}%
\pgfpathlineto{\pgfqpoint{4.404075in}{0.760287in}}%
\pgfpathlineto{\pgfqpoint{4.404747in}{0.781488in}}%
\pgfpathlineto{\pgfqpoint{4.405419in}{0.781488in}}%
\pgfpathlineto{\pgfqpoint{4.406763in}{0.739085in}}%
\pgfpathlineto{\pgfqpoint{4.407435in}{0.739085in}}%
\pgfpathlineto{\pgfqpoint{4.408780in}{0.770887in}}%
\pgfpathlineto{\pgfqpoint{4.409452in}{0.770887in}}%
\pgfpathlineto{\pgfqpoint{4.410124in}{0.739085in}}%
\pgfpathlineto{\pgfqpoint{4.410796in}{0.773537in}}%
\pgfpathlineto{\pgfqpoint{4.411468in}{0.773537in}}%
\pgfpathlineto{\pgfqpoint{4.412141in}{0.749686in}}%
\pgfpathlineto{\pgfqpoint{4.412813in}{0.762937in}}%
\pgfpathlineto{\pgfqpoint{4.413485in}{0.762937in}}%
\pgfpathlineto{\pgfqpoint{4.414157in}{0.749686in}}%
\pgfpathlineto{\pgfqpoint{4.414829in}{0.778838in}}%
\pgfpathlineto{\pgfqpoint{4.415501in}{0.778838in}}%
\pgfpathlineto{\pgfqpoint{4.415501in}{0.741735in}}%
\pgfpathlineto{\pgfqpoint{4.416846in}{0.776188in}}%
\pgfpathlineto{\pgfqpoint{4.417518in}{0.776188in}}%
\pgfpathlineto{\pgfqpoint{4.418190in}{0.741735in}}%
\pgfpathlineto{\pgfqpoint{4.418862in}{0.762937in}}%
\pgfpathlineto{\pgfqpoint{4.419534in}{0.762937in}}%
\pgfpathlineto{\pgfqpoint{4.420207in}{0.715234in}}%
\pgfpathlineto{\pgfqpoint{4.420879in}{0.744386in}}%
\pgfpathlineto{\pgfqpoint{4.421551in}{0.744386in}}%
\pgfpathlineto{\pgfqpoint{4.421551in}{0.776188in}}%
\pgfpathlineto{\pgfqpoint{4.422895in}{0.733785in}}%
\pgfpathlineto{\pgfqpoint{4.423567in}{0.733785in}}%
\pgfpathlineto{\pgfqpoint{4.423567in}{0.757636in}}%
\pgfpathlineto{\pgfqpoint{4.424912in}{0.752336in}}%
\pgfpathlineto{\pgfqpoint{4.425584in}{0.752336in}}%
\pgfpathlineto{\pgfqpoint{4.426256in}{0.754986in}}%
\pgfpathlineto{\pgfqpoint{4.426928in}{0.739085in}}%
\pgfpathlineto{\pgfqpoint{4.427600in}{0.739085in}}%
\pgfpathlineto{\pgfqpoint{4.427600in}{0.736435in}}%
\pgfpathlineto{\pgfqpoint{4.428945in}{0.776188in}}%
\pgfpathlineto{\pgfqpoint{4.429617in}{0.776188in}}%
\pgfpathlineto{\pgfqpoint{4.430289in}{0.739085in}}%
\pgfpathlineto{\pgfqpoint{4.430961in}{0.744386in}}%
\pgfpathlineto{\pgfqpoint{4.431633in}{0.744386in}}%
\pgfpathlineto{\pgfqpoint{4.431633in}{0.778838in}}%
\pgfpathlineto{\pgfqpoint{4.432978in}{0.712583in}}%
\pgfpathlineto{\pgfqpoint{4.433650in}{0.712583in}}%
\pgfpathlineto{\pgfqpoint{4.434322in}{0.778838in}}%
\pgfpathlineto{\pgfqpoint{4.434994in}{0.770887in}}%
\pgfpathlineto{\pgfqpoint{4.435666in}{0.770887in}}%
\pgfpathlineto{\pgfqpoint{4.436339in}{0.786788in}}%
\pgfpathlineto{\pgfqpoint{4.437011in}{0.728484in}}%
\pgfpathlineto{\pgfqpoint{4.437683in}{0.728484in}}%
\pgfpathlineto{\pgfqpoint{4.438355in}{0.770887in}}%
\pgfpathlineto{\pgfqpoint{4.439027in}{0.765587in}}%
\pgfpathlineto{\pgfqpoint{4.439699in}{0.765587in}}%
\pgfpathlineto{\pgfqpoint{4.439699in}{0.744386in}}%
\pgfpathlineto{\pgfqpoint{4.441044in}{0.747036in}}%
\pgfpathlineto{\pgfqpoint{4.441716in}{0.747036in}}%
\pgfpathlineto{\pgfqpoint{4.442388in}{0.789439in}}%
\pgfpathlineto{\pgfqpoint{4.443060in}{0.765587in}}%
\pgfpathlineto{\pgfqpoint{4.443732in}{0.765587in}}%
\pgfpathlineto{\pgfqpoint{4.443732in}{0.786788in}}%
\pgfpathlineto{\pgfqpoint{4.445077in}{0.781488in}}%
\pgfpathlineto{\pgfqpoint{4.445749in}{0.781488in}}%
\pgfpathlineto{\pgfqpoint{4.446421in}{0.736435in}}%
\pgfpathlineto{\pgfqpoint{4.447093in}{0.760287in}}%
\pgfpathlineto{\pgfqpoint{4.447765in}{0.760287in}}%
\pgfpathlineto{\pgfqpoint{4.448438in}{0.807990in}}%
\pgfpathlineto{\pgfqpoint{4.448438in}{0.715234in}}%
\pgfpathlineto{\pgfqpoint{4.449110in}{0.715234in}}%
\pgfpathlineto{\pgfqpoint{4.449782in}{0.715234in}}%
\pgfpathlineto{\pgfqpoint{4.449782in}{0.765587in}}%
\pgfpathlineto{\pgfqpoint{4.451126in}{0.747036in}}%
\pgfpathlineto{\pgfqpoint{4.451798in}{0.747036in}}%
\pgfpathlineto{\pgfqpoint{4.451798in}{0.744386in}}%
\pgfpathlineto{\pgfqpoint{4.452471in}{0.770887in}}%
\pgfpathlineto{\pgfqpoint{4.453143in}{0.754986in}}%
\pgfpathlineto{\pgfqpoint{4.453815in}{0.754986in}}%
\pgfpathlineto{\pgfqpoint{4.454487in}{0.725834in}}%
\pgfpathlineto{\pgfqpoint{4.455159in}{0.800039in}}%
\pgfpathlineto{\pgfqpoint{4.455831in}{0.800039in}}%
\pgfpathlineto{\pgfqpoint{4.456504in}{0.754986in}}%
\pgfpathlineto{\pgfqpoint{4.457176in}{0.776188in}}%
\pgfpathlineto{\pgfqpoint{4.457848in}{0.776188in}}%
\pgfpathlineto{\pgfqpoint{4.457848in}{0.731135in}}%
\pgfpathlineto{\pgfqpoint{4.459192in}{0.792089in}}%
\pgfpathlineto{\pgfqpoint{4.459864in}{0.792089in}}%
\pgfpathlineto{\pgfqpoint{4.460537in}{0.736435in}}%
\pgfpathlineto{\pgfqpoint{4.461209in}{0.757636in}}%
\pgfpathlineto{\pgfqpoint{4.461881in}{0.757636in}}%
\pgfpathlineto{\pgfqpoint{4.461881in}{0.773537in}}%
\pgfpathlineto{\pgfqpoint{4.462553in}{0.720534in}}%
\pgfpathlineto{\pgfqpoint{4.463225in}{0.744386in}}%
\pgfpathlineto{\pgfqpoint{4.463897in}{0.744386in}}%
\pgfpathlineto{\pgfqpoint{4.464570in}{0.768237in}}%
\pgfpathlineto{\pgfqpoint{4.465242in}{0.752336in}}%
\pgfpathlineto{\pgfqpoint{4.465914in}{0.752336in}}%
\pgfpathlineto{\pgfqpoint{4.465914in}{0.760287in}}%
\pgfpathlineto{\pgfqpoint{4.467258in}{0.736435in}}%
\pgfpathlineto{\pgfqpoint{4.467930in}{0.736435in}}%
\pgfpathlineto{\pgfqpoint{4.467930in}{0.784138in}}%
\pgfpathlineto{\pgfqpoint{4.469275in}{0.760287in}}%
\pgfpathlineto{\pgfqpoint{4.469947in}{0.760287in}}%
\pgfpathlineto{\pgfqpoint{4.469947in}{0.776188in}}%
\pgfpathlineto{\pgfqpoint{4.471291in}{0.773537in}}%
\pgfpathlineto{\pgfqpoint{4.471963in}{0.773537in}}%
\pgfpathlineto{\pgfqpoint{4.472636in}{0.731135in}}%
\pgfpathlineto{\pgfqpoint{4.473308in}{0.741735in}}%
\pgfpathlineto{\pgfqpoint{4.473980in}{0.741735in}}%
\pgfpathlineto{\pgfqpoint{4.474652in}{0.778838in}}%
\pgfpathlineto{\pgfqpoint{4.475324in}{0.749686in}}%
\pgfpathlineto{\pgfqpoint{4.475996in}{0.749686in}}%
\pgfpathlineto{\pgfqpoint{4.475996in}{0.752336in}}%
\pgfpathlineto{\pgfqpoint{4.476669in}{0.733785in}}%
\pgfpathlineto{\pgfqpoint{4.477341in}{0.752336in}}%
\pgfpathlineto{\pgfqpoint{4.478013in}{0.752336in}}%
\pgfpathlineto{\pgfqpoint{4.478013in}{0.786788in}}%
\pgfpathlineto{\pgfqpoint{4.479357in}{0.762937in}}%
\pgfpathlineto{\pgfqpoint{4.480029in}{0.762937in}}%
\pgfpathlineto{\pgfqpoint{4.480702in}{0.805340in}}%
\pgfpathlineto{\pgfqpoint{4.481374in}{0.760287in}}%
\pgfpathlineto{\pgfqpoint{4.482046in}{0.760287in}}%
\pgfpathlineto{\pgfqpoint{4.482046in}{0.773537in}}%
\pgfpathlineto{\pgfqpoint{4.482718in}{0.752336in}}%
\pgfpathlineto{\pgfqpoint{4.483390in}{0.768237in}}%
\pgfpathlineto{\pgfqpoint{4.484062in}{0.768237in}}%
\pgfpathlineto{\pgfqpoint{4.484062in}{0.757636in}}%
\pgfpathlineto{\pgfqpoint{4.485407in}{0.794739in}}%
\pgfpathlineto{\pgfqpoint{4.486079in}{0.794739in}}%
\pgfpathlineto{\pgfqpoint{4.486079in}{0.815940in}}%
\pgfpathlineto{\pgfqpoint{4.486751in}{0.747036in}}%
\pgfpathlineto{\pgfqpoint{4.487423in}{0.792089in}}%
\pgfpathlineto{\pgfqpoint{4.488095in}{0.792089in}}%
\pgfpathlineto{\pgfqpoint{4.488768in}{0.741735in}}%
\pgfpathlineto{\pgfqpoint{4.489440in}{0.802689in}}%
\pgfpathlineto{\pgfqpoint{4.490112in}{0.802689in}}%
\pgfpathlineto{\pgfqpoint{4.490112in}{0.765587in}}%
\pgfpathlineto{\pgfqpoint{4.491456in}{0.768237in}}%
\pgfpathlineto{\pgfqpoint{4.492129in}{0.768237in}}%
\pgfpathlineto{\pgfqpoint{4.492801in}{0.805340in}}%
\pgfpathlineto{\pgfqpoint{4.493473in}{0.749686in}}%
\pgfpathlineto{\pgfqpoint{4.494145in}{0.749686in}}%
\pgfpathlineto{\pgfqpoint{4.495489in}{0.784138in}}%
\pgfpathlineto{\pgfqpoint{4.496162in}{0.784138in}}%
\pgfpathlineto{\pgfqpoint{4.496834in}{0.762937in}}%
\pgfpathlineto{\pgfqpoint{4.497506in}{0.792089in}}%
\pgfpathlineto{\pgfqpoint{4.498178in}{0.792089in}}%
\pgfpathlineto{\pgfqpoint{4.499522in}{0.760287in}}%
\pgfpathlineto{\pgfqpoint{4.500195in}{0.760287in}}%
\pgfpathlineto{\pgfqpoint{4.500195in}{0.754986in}}%
\pgfpathlineto{\pgfqpoint{4.501539in}{0.797389in}}%
\pgfpathlineto{\pgfqpoint{4.502211in}{0.797389in}}%
\pgfpathlineto{\pgfqpoint{4.502211in}{0.752336in}}%
\pgfpathlineto{\pgfqpoint{4.503555in}{0.770887in}}%
\pgfpathlineto{\pgfqpoint{4.504228in}{0.770887in}}%
\pgfpathlineto{\pgfqpoint{4.504228in}{0.749686in}}%
\pgfpathlineto{\pgfqpoint{4.504900in}{0.778838in}}%
\pgfpathlineto{\pgfqpoint{4.505572in}{0.762937in}}%
\pgfpathlineto{\pgfqpoint{4.506244in}{0.762937in}}%
\pgfpathlineto{\pgfqpoint{4.507588in}{0.786788in}}%
\pgfpathlineto{\pgfqpoint{4.508261in}{0.786788in}}%
\pgfpathlineto{\pgfqpoint{4.508261in}{0.760287in}}%
\pgfpathlineto{\pgfqpoint{4.509605in}{0.778838in}}%
\pgfpathlineto{\pgfqpoint{4.510277in}{0.778838in}}%
\pgfpathlineto{\pgfqpoint{4.510277in}{0.747036in}}%
\pgfpathlineto{\pgfqpoint{4.511621in}{0.747036in}}%
\pgfpathlineto{\pgfqpoint{4.512294in}{0.747036in}}%
\pgfpathlineto{\pgfqpoint{4.512966in}{0.768237in}}%
\pgfpathlineto{\pgfqpoint{4.513638in}{0.768237in}}%
\pgfpathlineto{\pgfqpoint{4.514310in}{0.768237in}}%
\pgfpathlineto{\pgfqpoint{4.515654in}{0.749686in}}%
\pgfpathlineto{\pgfqpoint{4.516327in}{0.749686in}}%
\pgfpathlineto{\pgfqpoint{4.516327in}{0.770887in}}%
\pgfpathlineto{\pgfqpoint{4.517671in}{0.749686in}}%
\pgfpathlineto{\pgfqpoint{4.518343in}{0.749686in}}%
\pgfpathlineto{\pgfqpoint{4.519015in}{0.807990in}}%
\pgfpathlineto{\pgfqpoint{4.519687in}{0.789439in}}%
\pgfpathlineto{\pgfqpoint{4.520360in}{0.789439in}}%
\pgfpathlineto{\pgfqpoint{4.521704in}{0.752336in}}%
\pgfpathlineto{\pgfqpoint{4.522376in}{0.752336in}}%
\pgfpathlineto{\pgfqpoint{4.522376in}{0.778838in}}%
\pgfpathlineto{\pgfqpoint{4.523720in}{0.760287in}}%
\pgfpathlineto{\pgfqpoint{4.524393in}{0.760287in}}%
\pgfpathlineto{\pgfqpoint{4.525065in}{0.810640in}}%
\pgfpathlineto{\pgfqpoint{4.525737in}{0.807990in}}%
\pgfpathlineto{\pgfqpoint{4.526409in}{0.807990in}}%
\pgfpathlineto{\pgfqpoint{4.526409in}{0.747036in}}%
\pgfpathlineto{\pgfqpoint{4.527753in}{0.765587in}}%
\pgfpathlineto{\pgfqpoint{4.528426in}{0.765587in}}%
\pgfpathlineto{\pgfqpoint{4.529098in}{0.739085in}}%
\pgfpathlineto{\pgfqpoint{4.529770in}{0.786788in}}%
\pgfpathlineto{\pgfqpoint{4.530442in}{0.786788in}}%
\pgfpathlineto{\pgfqpoint{4.531114in}{0.789439in}}%
\pgfpathlineto{\pgfqpoint{4.531786in}{0.762937in}}%
\pgfpathlineto{\pgfqpoint{4.532459in}{0.762937in}}%
\pgfpathlineto{\pgfqpoint{4.533803in}{0.789439in}}%
\pgfpathlineto{\pgfqpoint{4.534475in}{0.789439in}}%
\pgfpathlineto{\pgfqpoint{4.534475in}{0.762937in}}%
\pgfpathlineto{\pgfqpoint{4.535819in}{0.805340in}}%
\pgfpathlineto{\pgfqpoint{4.536492in}{0.805340in}}%
\pgfpathlineto{\pgfqpoint{4.537164in}{0.786788in}}%
\pgfpathlineto{\pgfqpoint{4.537836in}{0.802689in}}%
\pgfpathlineto{\pgfqpoint{4.538508in}{0.802689in}}%
\pgfpathlineto{\pgfqpoint{4.538508in}{0.744386in}}%
\pgfpathlineto{\pgfqpoint{4.539852in}{0.768237in}}%
\pgfpathlineto{\pgfqpoint{4.540525in}{0.768237in}}%
\pgfpathlineto{\pgfqpoint{4.540525in}{0.757636in}}%
\pgfpathlineto{\pgfqpoint{4.541869in}{0.797389in}}%
\pgfpathlineto{\pgfqpoint{4.542541in}{0.797389in}}%
\pgfpathlineto{\pgfqpoint{4.543885in}{0.760287in}}%
\pgfpathlineto{\pgfqpoint{4.544558in}{0.760287in}}%
\pgfpathlineto{\pgfqpoint{4.545230in}{0.784138in}}%
\pgfpathlineto{\pgfqpoint{4.545902in}{0.778838in}}%
\pgfpathlineto{\pgfqpoint{4.546574in}{0.778838in}}%
\pgfpathlineto{\pgfqpoint{4.547246in}{0.760287in}}%
\pgfpathlineto{\pgfqpoint{4.547918in}{0.815940in}}%
\pgfpathlineto{\pgfqpoint{4.548591in}{0.815940in}}%
\pgfpathlineto{\pgfqpoint{4.549263in}{0.762937in}}%
\pgfpathlineto{\pgfqpoint{4.549935in}{0.781488in}}%
\pgfpathlineto{\pgfqpoint{4.550607in}{0.781488in}}%
\pgfpathlineto{\pgfqpoint{4.551279in}{0.853043in}}%
\pgfpathlineto{\pgfqpoint{4.551951in}{0.834492in}}%
\pgfpathlineto{\pgfqpoint{4.552624in}{0.834492in}}%
\pgfpathlineto{\pgfqpoint{4.553968in}{0.765587in}}%
\pgfpathlineto{\pgfqpoint{4.554640in}{0.765587in}}%
\pgfpathlineto{\pgfqpoint{4.554640in}{0.757636in}}%
\pgfpathlineto{\pgfqpoint{4.555984in}{0.805340in}}%
\pgfpathlineto{\pgfqpoint{4.556657in}{0.805340in}}%
\pgfpathlineto{\pgfqpoint{4.556657in}{0.770887in}}%
\pgfpathlineto{\pgfqpoint{4.558001in}{0.810640in}}%
\pgfpathlineto{\pgfqpoint{4.558673in}{0.810640in}}%
\pgfpathlineto{\pgfqpoint{4.558673in}{0.802689in}}%
\pgfpathlineto{\pgfqpoint{4.559345in}{0.882195in}}%
\pgfpathlineto{\pgfqpoint{4.560017in}{0.810640in}}%
\pgfpathlineto{\pgfqpoint{4.560690in}{0.810640in}}%
\pgfpathlineto{\pgfqpoint{4.560690in}{0.826541in}}%
\pgfpathlineto{\pgfqpoint{4.562034in}{0.797389in}}%
\pgfpathlineto{\pgfqpoint{4.562706in}{0.797389in}}%
\pgfpathlineto{\pgfqpoint{4.562706in}{0.805340in}}%
\pgfpathlineto{\pgfqpoint{4.563378in}{0.794739in}}%
\pgfpathlineto{\pgfqpoint{4.564050in}{0.794739in}}%
\pgfpathlineto{\pgfqpoint{4.564723in}{0.794739in}}%
\pgfpathlineto{\pgfqpoint{4.564723in}{0.786788in}}%
\pgfpathlineto{\pgfqpoint{4.566067in}{0.826541in}}%
\pgfpathlineto{\pgfqpoint{4.566739in}{0.826541in}}%
\pgfpathlineto{\pgfqpoint{4.567411in}{0.837142in}}%
\pgfpathlineto{\pgfqpoint{4.568083in}{0.792089in}}%
\pgfpathlineto{\pgfqpoint{4.568756in}{0.792089in}}%
\pgfpathlineto{\pgfqpoint{4.568756in}{0.858343in}}%
\pgfpathlineto{\pgfqpoint{4.570100in}{0.805340in}}%
\pgfpathlineto{\pgfqpoint{4.570772in}{0.805340in}}%
\pgfpathlineto{\pgfqpoint{4.570772in}{0.842442in}}%
\pgfpathlineto{\pgfqpoint{4.572116in}{0.815940in}}%
\pgfpathlineto{\pgfqpoint{4.572789in}{0.815940in}}%
\pgfpathlineto{\pgfqpoint{4.573461in}{0.781488in}}%
\pgfpathlineto{\pgfqpoint{4.574133in}{0.845092in}}%
\pgfpathlineto{\pgfqpoint{4.574805in}{0.845092in}}%
\pgfpathlineto{\pgfqpoint{4.575477in}{0.781488in}}%
\pgfpathlineto{\pgfqpoint{4.576149in}{0.847742in}}%
\pgfpathlineto{\pgfqpoint{4.576822in}{0.847742in}}%
\pgfpathlineto{\pgfqpoint{4.576822in}{0.850393in}}%
\pgfpathlineto{\pgfqpoint{4.577494in}{0.839792in}}%
\pgfpathlineto{\pgfqpoint{4.578166in}{0.845092in}}%
\pgfpathlineto{\pgfqpoint{4.578838in}{0.845092in}}%
\pgfpathlineto{\pgfqpoint{4.578838in}{0.871594in}}%
\pgfpathlineto{\pgfqpoint{4.579510in}{0.815940in}}%
\pgfpathlineto{\pgfqpoint{4.580182in}{0.860993in}}%
\pgfpathlineto{\pgfqpoint{4.580855in}{0.860993in}}%
\pgfpathlineto{\pgfqpoint{4.582199in}{0.821241in}}%
\pgfpathlineto{\pgfqpoint{4.582871in}{0.821241in}}%
\pgfpathlineto{\pgfqpoint{4.583543in}{0.853043in}}%
\pgfpathlineto{\pgfqpoint{4.584215in}{0.842442in}}%
\pgfpathlineto{\pgfqpoint{4.584888in}{0.842442in}}%
\pgfpathlineto{\pgfqpoint{4.585560in}{0.831841in}}%
\pgfpathlineto{\pgfqpoint{4.586232in}{0.866294in}}%
\pgfpathlineto{\pgfqpoint{4.586904in}{0.866294in}}%
\pgfpathlineto{\pgfqpoint{4.586904in}{0.871594in}}%
\pgfpathlineto{\pgfqpoint{4.588248in}{0.842442in}}%
\pgfpathlineto{\pgfqpoint{4.588921in}{0.842442in}}%
\pgfpathlineto{\pgfqpoint{4.590265in}{0.887495in}}%
\pgfpathlineto{\pgfqpoint{4.590937in}{0.887495in}}%
\pgfpathlineto{\pgfqpoint{4.591609in}{0.810640in}}%
\pgfpathlineto{\pgfqpoint{4.592281in}{0.866294in}}%
\pgfpathlineto{\pgfqpoint{4.592954in}{0.866294in}}%
\pgfpathlineto{\pgfqpoint{4.592954in}{0.810640in}}%
\pgfpathlineto{\pgfqpoint{4.593626in}{0.924598in}}%
\pgfpathlineto{\pgfqpoint{4.594298in}{0.850393in}}%
\pgfpathlineto{\pgfqpoint{4.594970in}{0.850393in}}%
\pgfpathlineto{\pgfqpoint{4.595642in}{0.892795in}}%
\pgfpathlineto{\pgfqpoint{4.596314in}{0.845092in}}%
\pgfpathlineto{\pgfqpoint{4.596987in}{0.845092in}}%
\pgfpathlineto{\pgfqpoint{4.596987in}{0.929898in}}%
\pgfpathlineto{\pgfqpoint{4.598331in}{0.831841in}}%
\pgfpathlineto{\pgfqpoint{4.599003in}{0.831841in}}%
\pgfpathlineto{\pgfqpoint{4.599003in}{0.908697in}}%
\pgfpathlineto{\pgfqpoint{4.600347in}{0.898096in}}%
\pgfpathlineto{\pgfqpoint{4.601692in}{0.898096in}}%
\pgfpathlineto{\pgfqpoint{4.602364in}{0.858343in}}%
\pgfpathlineto{\pgfqpoint{4.603036in}{0.874244in}}%
\pgfpathlineto{\pgfqpoint{4.603708in}{0.874244in}}%
\pgfpathlineto{\pgfqpoint{4.603708in}{0.871594in}}%
\pgfpathlineto{\pgfqpoint{4.604381in}{0.906046in}}%
\pgfpathlineto{\pgfqpoint{4.605053in}{0.884845in}}%
\pgfpathlineto{\pgfqpoint{4.605725in}{0.884845in}}%
\pgfpathlineto{\pgfqpoint{4.605725in}{0.898096in}}%
\pgfpathlineto{\pgfqpoint{4.607069in}{0.876894in}}%
\pgfpathlineto{\pgfqpoint{4.607741in}{0.876894in}}%
\pgfpathlineto{\pgfqpoint{4.607741in}{0.908697in}}%
\pgfpathlineto{\pgfqpoint{4.609086in}{0.879545in}}%
\pgfpathlineto{\pgfqpoint{4.609758in}{0.879545in}}%
\pgfpathlineto{\pgfqpoint{4.611102in}{0.961700in}}%
\pgfpathlineto{\pgfqpoint{4.611774in}{0.961700in}}%
\pgfpathlineto{\pgfqpoint{4.611774in}{0.887495in}}%
\pgfpathlineto{\pgfqpoint{4.613119in}{0.951099in}}%
\pgfpathlineto{\pgfqpoint{4.613791in}{0.951099in}}%
\pgfpathlineto{\pgfqpoint{4.614463in}{0.953750in}}%
\pgfpathlineto{\pgfqpoint{4.615135in}{0.850393in}}%
\pgfpathlineto{\pgfqpoint{4.615807in}{0.850393in}}%
\pgfpathlineto{\pgfqpoint{4.616480in}{0.935198in}}%
\pgfpathlineto{\pgfqpoint{4.617152in}{0.860993in}}%
\pgfpathlineto{\pgfqpoint{4.617824in}{0.860993in}}%
\pgfpathlineto{\pgfqpoint{4.617824in}{0.932548in}}%
\pgfpathlineto{\pgfqpoint{4.618496in}{0.855693in}}%
\pgfpathlineto{\pgfqpoint{4.619168in}{0.900746in}}%
\pgfpathlineto{\pgfqpoint{4.619840in}{0.900746in}}%
\pgfpathlineto{\pgfqpoint{4.619840in}{0.937848in}}%
\pgfpathlineto{\pgfqpoint{4.620513in}{0.895446in}}%
\pgfpathlineto{\pgfqpoint{4.621185in}{0.911347in}}%
\pgfpathlineto{\pgfqpoint{4.621857in}{0.911347in}}%
\pgfpathlineto{\pgfqpoint{4.622529in}{0.953750in}}%
\pgfpathlineto{\pgfqpoint{4.623201in}{0.953750in}}%
\pgfpathlineto{\pgfqpoint{4.623873in}{0.953750in}}%
\pgfpathlineto{\pgfqpoint{4.623873in}{0.982901in}}%
\pgfpathlineto{\pgfqpoint{4.625218in}{0.924598in}}%
\pgfpathlineto{\pgfqpoint{4.625890in}{0.924598in}}%
\pgfpathlineto{\pgfqpoint{4.626562in}{0.969651in}}%
\pgfpathlineto{\pgfqpoint{4.627234in}{0.895446in}}%
\pgfpathlineto{\pgfqpoint{4.627906in}{0.895446in}}%
\pgfpathlineto{\pgfqpoint{4.627906in}{0.998803in}}%
\pgfpathlineto{\pgfqpoint{4.629251in}{0.935198in}}%
\pgfpathlineto{\pgfqpoint{4.629923in}{0.935198in}}%
\pgfpathlineto{\pgfqpoint{4.629923in}{0.961700in}}%
\pgfpathlineto{\pgfqpoint{4.630595in}{0.898096in}}%
\pgfpathlineto{\pgfqpoint{4.631267in}{0.937848in}}%
\pgfpathlineto{\pgfqpoint{4.631939in}{0.937848in}}%
\pgfpathlineto{\pgfqpoint{4.631939in}{0.924598in}}%
\pgfpathlineto{\pgfqpoint{4.633284in}{0.998803in}}%
\pgfpathlineto{\pgfqpoint{4.633956in}{0.998803in}}%
\pgfpathlineto{\pgfqpoint{4.633956in}{0.943149in}}%
\pgfpathlineto{\pgfqpoint{4.635300in}{0.956400in}}%
\pgfpathlineto{\pgfqpoint{4.635972in}{0.956400in}}%
\pgfpathlineto{\pgfqpoint{4.636645in}{1.004103in}}%
\pgfpathlineto{\pgfqpoint{4.637317in}{0.977601in}}%
\pgfpathlineto{\pgfqpoint{4.637989in}{0.977601in}}%
\pgfpathlineto{\pgfqpoint{4.638661in}{0.948449in}}%
\pgfpathlineto{\pgfqpoint{4.639333in}{0.990852in}}%
\pgfpathlineto{\pgfqpoint{4.640005in}{0.990852in}}%
\pgfpathlineto{\pgfqpoint{4.640005in}{0.993502in}}%
\pgfpathlineto{\pgfqpoint{4.640678in}{0.964350in}}%
\pgfpathlineto{\pgfqpoint{4.641350in}{0.993502in}}%
\pgfpathlineto{\pgfqpoint{4.642022in}{0.993502in}}%
\pgfpathlineto{\pgfqpoint{4.642022in}{0.998803in}}%
\pgfpathlineto{\pgfqpoint{4.643366in}{0.967000in}}%
\pgfpathlineto{\pgfqpoint{4.644038in}{0.967000in}}%
\pgfpathlineto{\pgfqpoint{4.644038in}{0.988202in}}%
\pgfpathlineto{\pgfqpoint{4.645383in}{0.974951in}}%
\pgfpathlineto{\pgfqpoint{4.646055in}{0.974951in}}%
\pgfpathlineto{\pgfqpoint{4.646055in}{1.012053in}}%
\pgfpathlineto{\pgfqpoint{4.646727in}{0.943149in}}%
\pgfpathlineto{\pgfqpoint{4.647399in}{0.985552in}}%
\pgfpathlineto{\pgfqpoint{4.648071in}{0.985552in}}%
\pgfpathlineto{\pgfqpoint{4.648071in}{0.921947in}}%
\pgfpathlineto{\pgfqpoint{4.649416in}{0.956400in}}%
\pgfpathlineto{\pgfqpoint{4.650088in}{0.956400in}}%
\pgfpathlineto{\pgfqpoint{4.650088in}{1.020004in}}%
\pgfpathlineto{\pgfqpoint{4.650760in}{0.953750in}}%
\pgfpathlineto{\pgfqpoint{4.651432in}{1.009403in}}%
\pgfpathlineto{\pgfqpoint{4.652104in}{1.009403in}}%
\pgfpathlineto{\pgfqpoint{4.652104in}{1.035905in}}%
\pgfpathlineto{\pgfqpoint{4.652777in}{0.982901in}}%
\pgfpathlineto{\pgfqpoint{4.653449in}{1.030605in}}%
\pgfpathlineto{\pgfqpoint{4.654121in}{1.030605in}}%
\pgfpathlineto{\pgfqpoint{4.654121in}{1.009403in}}%
\pgfpathlineto{\pgfqpoint{4.655465in}{1.014704in}}%
\pgfpathlineto{\pgfqpoint{4.656137in}{1.014704in}}%
\pgfpathlineto{\pgfqpoint{4.656137in}{1.067707in}}%
\pgfpathlineto{\pgfqpoint{4.656810in}{0.956400in}}%
\pgfpathlineto{\pgfqpoint{4.657482in}{0.982901in}}%
\pgfpathlineto{\pgfqpoint{4.658154in}{0.982901in}}%
\pgfpathlineto{\pgfqpoint{4.659498in}{1.046506in}}%
\pgfpathlineto{\pgfqpoint{4.660170in}{1.046506in}}%
\pgfpathlineto{\pgfqpoint{4.660170in}{1.041205in}}%
\pgfpathlineto{\pgfqpoint{4.661515in}{1.062407in}}%
\pgfpathlineto{\pgfqpoint{4.662187in}{1.062407in}}%
\pgfpathlineto{\pgfqpoint{4.662187in}{1.094209in}}%
\pgfpathlineto{\pgfqpoint{4.662859in}{0.977601in}}%
\pgfpathlineto{\pgfqpoint{4.663531in}{1.078308in}}%
\pgfpathlineto{\pgfqpoint{4.664203in}{1.078308in}}%
\pgfpathlineto{\pgfqpoint{4.665548in}{1.020004in}}%
\pgfpathlineto{\pgfqpoint{4.666220in}{1.020004in}}%
\pgfpathlineto{\pgfqpoint{4.666220in}{1.054456in}}%
\pgfpathlineto{\pgfqpoint{4.666892in}{0.974951in}}%
\pgfpathlineto{\pgfqpoint{4.667564in}{1.012053in}}%
\pgfpathlineto{\pgfqpoint{4.668236in}{1.012053in}}%
\pgfpathlineto{\pgfqpoint{4.668909in}{1.043856in}}%
\pgfpathlineto{\pgfqpoint{4.669581in}{1.001453in}}%
\pgfpathlineto{\pgfqpoint{4.670253in}{1.001453in}}%
\pgfpathlineto{\pgfqpoint{4.670925in}{1.075658in}}%
\pgfpathlineto{\pgfqpoint{4.671597in}{1.067707in}}%
\pgfpathlineto{\pgfqpoint{4.672269in}{1.067707in}}%
\pgfpathlineto{\pgfqpoint{4.672269in}{1.083608in}}%
\pgfpathlineto{\pgfqpoint{4.672942in}{1.051806in}}%
\pgfpathlineto{\pgfqpoint{4.673614in}{1.051806in}}%
\pgfpathlineto{\pgfqpoint{4.674286in}{1.051806in}}%
\pgfpathlineto{\pgfqpoint{4.674958in}{1.043856in}}%
\pgfpathlineto{\pgfqpoint{4.675630in}{1.067707in}}%
\pgfpathlineto{\pgfqpoint{4.676302in}{1.067707in}}%
\pgfpathlineto{\pgfqpoint{4.677647in}{1.141912in}}%
\pgfpathlineto{\pgfqpoint{4.678319in}{1.141912in}}%
\pgfpathlineto{\pgfqpoint{4.678319in}{1.059757in}}%
\pgfpathlineto{\pgfqpoint{4.679663in}{1.123361in}}%
\pgfpathlineto{\pgfqpoint{4.680335in}{1.123361in}}%
\pgfpathlineto{\pgfqpoint{4.680335in}{1.043856in}}%
\pgfpathlineto{\pgfqpoint{4.681680in}{1.160463in}}%
\pgfpathlineto{\pgfqpoint{4.682352in}{1.160463in}}%
\pgfpathlineto{\pgfqpoint{4.682352in}{1.033255in}}%
\pgfpathlineto{\pgfqpoint{4.683696in}{1.099509in}}%
\pgfpathlineto{\pgfqpoint{4.684368in}{1.099509in}}%
\pgfpathlineto{\pgfqpoint{4.684368in}{1.131311in}}%
\pgfpathlineto{\pgfqpoint{4.685713in}{1.118060in}}%
\pgfpathlineto{\pgfqpoint{4.686385in}{1.118060in}}%
\pgfpathlineto{\pgfqpoint{4.687057in}{1.086258in}}%
\pgfpathlineto{\pgfqpoint{4.687729in}{1.094209in}}%
\pgfpathlineto{\pgfqpoint{4.688401in}{1.094209in}}%
\pgfpathlineto{\pgfqpoint{4.689074in}{1.046506in}}%
\pgfpathlineto{\pgfqpoint{4.689746in}{1.149863in}}%
\pgfpathlineto{\pgfqpoint{4.691090in}{1.149863in}}%
\pgfpathlineto{\pgfqpoint{4.691762in}{1.065057in}}%
\pgfpathlineto{\pgfqpoint{4.692434in}{1.136612in}}%
\pgfpathlineto{\pgfqpoint{4.693107in}{1.136612in}}%
\pgfpathlineto{\pgfqpoint{4.693779in}{1.168414in}}%
\pgfpathlineto{\pgfqpoint{4.694451in}{1.147212in}}%
\pgfpathlineto{\pgfqpoint{4.695123in}{1.147212in}}%
\pgfpathlineto{\pgfqpoint{4.695795in}{1.133962in}}%
\pgfpathlineto{\pgfqpoint{4.696467in}{1.176364in}}%
\pgfpathlineto{\pgfqpoint{4.697140in}{1.176364in}}%
\pgfpathlineto{\pgfqpoint{4.697812in}{1.123361in}}%
\pgfpathlineto{\pgfqpoint{4.698484in}{1.176364in}}%
\pgfpathlineto{\pgfqpoint{4.699156in}{1.176364in}}%
\pgfpathlineto{\pgfqpoint{4.699828in}{1.234668in}}%
\pgfpathlineto{\pgfqpoint{4.700500in}{1.155163in}}%
\pgfpathlineto{\pgfqpoint{4.701173in}{1.155163in}}%
\pgfpathlineto{\pgfqpoint{4.701173in}{1.112760in}}%
\pgfpathlineto{\pgfqpoint{4.702517in}{1.202866in}}%
\pgfpathlineto{\pgfqpoint{4.703189in}{1.202866in}}%
\pgfpathlineto{\pgfqpoint{4.703861in}{1.208166in}}%
\pgfpathlineto{\pgfqpoint{4.704533in}{1.171064in}}%
\pgfpathlineto{\pgfqpoint{4.705206in}{1.171064in}}%
\pgfpathlineto{\pgfqpoint{4.705878in}{1.133962in}}%
\pgfpathlineto{\pgfqpoint{4.706550in}{1.179015in}}%
\pgfpathlineto{\pgfqpoint{4.707222in}{1.179015in}}%
\pgfpathlineto{\pgfqpoint{4.707894in}{1.155163in}}%
\pgfpathlineto{\pgfqpoint{4.708566in}{1.261170in}}%
\pgfpathlineto{\pgfqpoint{4.709239in}{1.261170in}}%
\pgfpathlineto{\pgfqpoint{4.709911in}{1.139262in}}%
\pgfpathlineto{\pgfqpoint{4.710583in}{1.221417in}}%
\pgfpathlineto{\pgfqpoint{4.711255in}{1.221417in}}%
\pgfpathlineto{\pgfqpoint{4.711255in}{1.163113in}}%
\pgfpathlineto{\pgfqpoint{4.711927in}{1.255870in}}%
\pgfpathlineto{\pgfqpoint{4.712599in}{1.197566in}}%
\pgfpathlineto{\pgfqpoint{4.713272in}{1.197566in}}%
\pgfpathlineto{\pgfqpoint{4.713944in}{1.216117in}}%
\pgfpathlineto{\pgfqpoint{4.714616in}{1.168414in}}%
\pgfpathlineto{\pgfqpoint{4.715288in}{1.168414in}}%
\pgfpathlineto{\pgfqpoint{4.715960in}{1.266470in}}%
\pgfpathlineto{\pgfqpoint{4.716633in}{1.200216in}}%
\pgfpathlineto{\pgfqpoint{4.717305in}{1.200216in}}%
\pgfpathlineto{\pgfqpoint{4.717305in}{1.308873in}}%
\pgfpathlineto{\pgfqpoint{4.718649in}{1.173714in}}%
\pgfpathlineto{\pgfqpoint{4.719321in}{1.173714in}}%
\pgfpathlineto{\pgfqpoint{4.719993in}{1.290322in}}%
\pgfpathlineto{\pgfqpoint{4.720666in}{1.253219in}}%
\pgfpathlineto{\pgfqpoint{4.721338in}{1.253219in}}%
\pgfpathlineto{\pgfqpoint{4.722682in}{1.216117in}}%
\pgfpathlineto{\pgfqpoint{4.723354in}{1.216117in}}%
\pgfpathlineto{\pgfqpoint{4.723354in}{1.316824in}}%
\pgfpathlineto{\pgfqpoint{4.724699in}{1.298272in}}%
\pgfpathlineto{\pgfqpoint{4.725371in}{1.298272in}}%
\pgfpathlineto{\pgfqpoint{4.725371in}{1.356576in}}%
\pgfpathlineto{\pgfqpoint{4.726043in}{1.250569in}}%
\pgfpathlineto{\pgfqpoint{4.726715in}{1.261170in}}%
\pgfpathlineto{\pgfqpoint{4.727387in}{1.261170in}}%
\pgfpathlineto{\pgfqpoint{4.728732in}{1.300923in}}%
\pgfpathlineto{\pgfqpoint{4.729404in}{1.300923in}}%
\pgfpathlineto{\pgfqpoint{4.730076in}{1.218767in}}%
\pgfpathlineto{\pgfqpoint{4.730748in}{1.327424in}}%
\pgfpathlineto{\pgfqpoint{4.731420in}{1.327424in}}%
\pgfpathlineto{\pgfqpoint{4.731420in}{1.380428in}}%
\pgfpathlineto{\pgfqpoint{4.732765in}{1.369827in}}%
\pgfpathlineto{\pgfqpoint{4.733437in}{1.369827in}}%
\pgfpathlineto{\pgfqpoint{4.733437in}{1.298272in}}%
\pgfpathlineto{\pgfqpoint{4.734781in}{1.311523in}}%
\pgfpathlineto{\pgfqpoint{4.735453in}{1.311523in}}%
\pgfpathlineto{\pgfqpoint{4.735453in}{1.261170in}}%
\pgfpathlineto{\pgfqpoint{4.736798in}{1.340675in}}%
\pgfpathlineto{\pgfqpoint{4.737470in}{1.340675in}}%
\pgfpathlineto{\pgfqpoint{4.738142in}{1.377778in}}%
\pgfpathlineto{\pgfqpoint{4.738814in}{1.285022in}}%
\pgfpathlineto{\pgfqpoint{4.739486in}{1.285022in}}%
\pgfpathlineto{\pgfqpoint{4.740831in}{1.391029in}}%
\pgfpathlineto{\pgfqpoint{4.741503in}{1.391029in}}%
\pgfpathlineto{\pgfqpoint{4.742847in}{1.287672in}}%
\pgfpathlineto{\pgfqpoint{4.743519in}{1.287672in}}%
\pgfpathlineto{\pgfqpoint{4.743519in}{1.322124in}}%
\pgfpathlineto{\pgfqpoint{4.744864in}{1.250569in}}%
\pgfpathlineto{\pgfqpoint{4.745536in}{1.250569in}}%
\pgfpathlineto{\pgfqpoint{4.745536in}{1.375128in}}%
\pgfpathlineto{\pgfqpoint{4.746880in}{1.285022in}}%
\pgfpathlineto{\pgfqpoint{4.747552in}{1.285022in}}%
\pgfpathlineto{\pgfqpoint{4.748897in}{1.375128in}}%
\pgfpathlineto{\pgfqpoint{4.749569in}{1.375128in}}%
\pgfpathlineto{\pgfqpoint{4.749569in}{1.303573in}}%
\pgfpathlineto{\pgfqpoint{4.750913in}{1.338025in}}%
\pgfpathlineto{\pgfqpoint{4.751585in}{1.338025in}}%
\pgfpathlineto{\pgfqpoint{4.752930in}{1.462583in}}%
\pgfpathlineto{\pgfqpoint{4.753602in}{1.462583in}}%
\pgfpathlineto{\pgfqpoint{4.753602in}{1.324774in}}%
\pgfpathlineto{\pgfqpoint{4.754946in}{1.398979in}}%
\pgfpathlineto{\pgfqpoint{4.755618in}{1.398979in}}%
\pgfpathlineto{\pgfqpoint{4.756290in}{1.433431in}}%
\pgfpathlineto{\pgfqpoint{4.756963in}{1.343325in}}%
\pgfpathlineto{\pgfqpoint{4.757635in}{1.343325in}}%
\pgfpathlineto{\pgfqpoint{4.758307in}{1.451983in}}%
\pgfpathlineto{\pgfqpoint{4.758979in}{1.396329in}}%
\pgfpathlineto{\pgfqpoint{4.759651in}{1.396329in}}%
\pgfpathlineto{\pgfqpoint{4.759651in}{1.364527in}}%
\pgfpathlineto{\pgfqpoint{4.760996in}{1.409580in}}%
\pgfpathlineto{\pgfqpoint{4.761668in}{1.409580in}}%
\pgfpathlineto{\pgfqpoint{4.762340in}{1.398979in}}%
\pgfpathlineto{\pgfqpoint{4.763012in}{1.457283in}}%
\pgfpathlineto{\pgfqpoint{4.763684in}{1.457283in}}%
\pgfpathlineto{\pgfqpoint{4.763684in}{1.486435in}}%
\pgfpathlineto{\pgfqpoint{4.764356in}{1.396329in}}%
\pgfpathlineto{\pgfqpoint{4.765029in}{1.425481in}}%
\pgfpathlineto{\pgfqpoint{4.765701in}{1.425481in}}%
\pgfpathlineto{\pgfqpoint{4.765701in}{1.478484in}}%
\pgfpathlineto{\pgfqpoint{4.767045in}{1.380428in}}%
\pgfpathlineto{\pgfqpoint{4.767717in}{1.380428in}}%
\pgfpathlineto{\pgfqpoint{4.768389in}{1.510287in}}%
\pgfpathlineto{\pgfqpoint{4.769062in}{1.457283in}}%
\pgfpathlineto{\pgfqpoint{4.769734in}{1.457283in}}%
\pgfpathlineto{\pgfqpoint{4.769734in}{1.462583in}}%
\pgfpathlineto{\pgfqpoint{4.771078in}{1.409580in}}%
\pgfpathlineto{\pgfqpoint{4.771750in}{1.409580in}}%
\pgfpathlineto{\pgfqpoint{4.773095in}{1.526188in}}%
\pgfpathlineto{\pgfqpoint{4.773767in}{1.526188in}}%
\pgfpathlineto{\pgfqpoint{4.773767in}{1.528838in}}%
\pgfpathlineto{\pgfqpoint{4.774439in}{1.375128in}}%
\pgfpathlineto{\pgfqpoint{4.775111in}{1.438732in}}%
\pgfpathlineto{\pgfqpoint{4.775783in}{1.438732in}}%
\pgfpathlineto{\pgfqpoint{4.775783in}{1.515587in}}%
\pgfpathlineto{\pgfqpoint{4.776455in}{1.436082in}}%
\pgfpathlineto{\pgfqpoint{4.777128in}{1.457283in}}%
\pgfpathlineto{\pgfqpoint{4.777800in}{1.457283in}}%
\pgfpathlineto{\pgfqpoint{4.778472in}{1.433431in}}%
\pgfpathlineto{\pgfqpoint{4.778472in}{1.483785in}}%
\pgfpathlineto{\pgfqpoint{4.779144in}{1.483785in}}%
\pgfpathlineto{\pgfqpoint{4.779816in}{1.483785in}}%
\pgfpathlineto{\pgfqpoint{4.779816in}{1.478484in}}%
\pgfpathlineto{\pgfqpoint{4.781161in}{1.555340in}}%
\pgfpathlineto{\pgfqpoint{4.781833in}{1.555340in}}%
\pgfpathlineto{\pgfqpoint{4.782505in}{1.467884in}}%
\pgfpathlineto{\pgfqpoint{4.783177in}{1.486435in}}%
\pgfpathlineto{\pgfqpoint{4.783849in}{1.486435in}}%
\pgfpathlineto{\pgfqpoint{4.783849in}{1.579191in}}%
\pgfpathlineto{\pgfqpoint{4.785194in}{1.497036in}}%
\pgfpathlineto{\pgfqpoint{4.785866in}{1.497036in}}%
\pgfpathlineto{\pgfqpoint{4.786538in}{1.526188in}}%
\pgfpathlineto{\pgfqpoint{4.787210in}{1.465234in}}%
\pgfpathlineto{\pgfqpoint{4.787882in}{1.465234in}}%
\pgfpathlineto{\pgfqpoint{4.787882in}{1.404280in}}%
\pgfpathlineto{\pgfqpoint{4.788554in}{1.573891in}}%
\pgfpathlineto{\pgfqpoint{4.789227in}{1.465234in}}%
\pgfpathlineto{\pgfqpoint{4.789899in}{1.465234in}}%
\pgfpathlineto{\pgfqpoint{4.789899in}{1.539439in}}%
\pgfpathlineto{\pgfqpoint{4.791243in}{1.489085in}}%
\pgfpathlineto{\pgfqpoint{4.791915in}{1.489085in}}%
\pgfpathlineto{\pgfqpoint{4.791915in}{1.412230in}}%
\pgfpathlineto{\pgfqpoint{4.792587in}{1.571241in}}%
\pgfpathlineto{\pgfqpoint{4.793260in}{1.544739in}}%
\pgfpathlineto{\pgfqpoint{4.793932in}{1.544739in}}%
\pgfpathlineto{\pgfqpoint{4.793932in}{1.531488in}}%
\pgfpathlineto{\pgfqpoint{4.795276in}{1.621594in}}%
\pgfpathlineto{\pgfqpoint{4.795948in}{1.621594in}}%
\pgfpathlineto{\pgfqpoint{4.795948in}{1.504986in}}%
\pgfpathlineto{\pgfqpoint{4.797293in}{1.573891in}}%
\pgfpathlineto{\pgfqpoint{4.797965in}{1.573891in}}%
\pgfpathlineto{\pgfqpoint{4.798637in}{1.515587in}}%
\pgfpathlineto{\pgfqpoint{4.799309in}{1.555340in}}%
\pgfpathlineto{\pgfqpoint{4.799981in}{1.555340in}}%
\pgfpathlineto{\pgfqpoint{4.799981in}{1.520887in}}%
\pgfpathlineto{\pgfqpoint{4.801326in}{1.597742in}}%
\pgfpathlineto{\pgfqpoint{4.801998in}{1.597742in}}%
\pgfpathlineto{\pgfqpoint{4.801998in}{1.698449in}}%
\pgfpathlineto{\pgfqpoint{4.802670in}{1.528838in}}%
\pgfpathlineto{\pgfqpoint{4.803342in}{1.584492in}}%
\pgfpathlineto{\pgfqpoint{4.804014in}{1.584492in}}%
\pgfpathlineto{\pgfqpoint{4.804014in}{1.534138in}}%
\pgfpathlineto{\pgfqpoint{4.804686in}{1.642795in}}%
\pgfpathlineto{\pgfqpoint{4.805359in}{1.581841in}}%
\pgfpathlineto{\pgfqpoint{4.806031in}{1.581841in}}%
\pgfpathlineto{\pgfqpoint{4.806031in}{1.547389in}}%
\pgfpathlineto{\pgfqpoint{4.807375in}{1.695799in}}%
\pgfpathlineto{\pgfqpoint{4.808047in}{1.695799in}}%
\pgfpathlineto{\pgfqpoint{4.809392in}{1.587142in}}%
\pgfpathlineto{\pgfqpoint{4.810064in}{1.587142in}}%
\pgfpathlineto{\pgfqpoint{4.810064in}{1.740852in}}%
\pgfpathlineto{\pgfqpoint{4.810736in}{1.552689in}}%
\pgfpathlineto{\pgfqpoint{4.811408in}{1.687848in}}%
\pgfpathlineto{\pgfqpoint{4.812080in}{1.687848in}}%
\pgfpathlineto{\pgfqpoint{4.812080in}{1.626894in}}%
\pgfpathlineto{\pgfqpoint{4.812752in}{1.690499in}}%
\pgfpathlineto{\pgfqpoint{4.813425in}{1.671947in}}%
\pgfpathlineto{\pgfqpoint{4.814097in}{1.671947in}}%
\pgfpathlineto{\pgfqpoint{4.814097in}{1.534138in}}%
\pgfpathlineto{\pgfqpoint{4.815441in}{1.685198in}}%
\pgfpathlineto{\pgfqpoint{4.816113in}{1.685198in}}%
\pgfpathlineto{\pgfqpoint{4.817458in}{1.568590in}}%
\pgfpathlineto{\pgfqpoint{4.818130in}{1.568590in}}%
\pgfpathlineto{\pgfqpoint{4.818130in}{1.717000in}}%
\pgfpathlineto{\pgfqpoint{4.819474in}{1.640145in}}%
\pgfpathlineto{\pgfqpoint{4.820146in}{1.640145in}}%
\pgfpathlineto{\pgfqpoint{4.820146in}{1.698449in}}%
\pgfpathlineto{\pgfqpoint{4.820818in}{1.634845in}}%
\pgfpathlineto{\pgfqpoint{4.821491in}{1.677248in}}%
\pgfpathlineto{\pgfqpoint{4.822163in}{1.677248in}}%
\pgfpathlineto{\pgfqpoint{4.822163in}{1.727601in}}%
\pgfpathlineto{\pgfqpoint{4.823507in}{1.719651in}}%
\pgfpathlineto{\pgfqpoint{4.824179in}{1.719651in}}%
\pgfpathlineto{\pgfqpoint{4.825524in}{1.613644in}}%
\pgfpathlineto{\pgfqpoint{4.826196in}{1.613644in}}%
\pgfpathlineto{\pgfqpoint{4.826196in}{1.748803in}}%
\pgfpathlineto{\pgfqpoint{4.827540in}{1.709050in}}%
\pgfpathlineto{\pgfqpoint{4.828212in}{1.709050in}}%
\pgfpathlineto{\pgfqpoint{4.828884in}{1.661347in}}%
\pgfpathlineto{\pgfqpoint{4.829557in}{1.717000in}}%
\pgfpathlineto{\pgfqpoint{4.830229in}{1.717000in}}%
\pgfpathlineto{\pgfqpoint{4.830901in}{1.597742in}}%
\pgfpathlineto{\pgfqpoint{4.831573in}{1.738202in}}%
\pgfpathlineto{\pgfqpoint{4.832245in}{1.738202in}}%
\pgfpathlineto{\pgfqpoint{4.832918in}{1.648096in}}%
\pgfpathlineto{\pgfqpoint{4.833590in}{1.717000in}}%
\pgfpathlineto{\pgfqpoint{4.834262in}{1.717000in}}%
\pgfpathlineto{\pgfqpoint{4.834262in}{1.754103in}}%
\pgfpathlineto{\pgfqpoint{4.834934in}{1.613644in}}%
\pgfpathlineto{\pgfqpoint{4.835606in}{1.706400in}}%
\pgfpathlineto{\pgfqpoint{4.836278in}{1.706400in}}%
\pgfpathlineto{\pgfqpoint{4.836278in}{1.735552in}}%
\pgfpathlineto{\pgfqpoint{4.837623in}{1.648096in}}%
\pgfpathlineto{\pgfqpoint{4.838295in}{1.648096in}}%
\pgfpathlineto{\pgfqpoint{4.838967in}{1.785905in}}%
\pgfpathlineto{\pgfqpoint{4.839639in}{1.777954in}}%
\pgfpathlineto{\pgfqpoint{4.840311in}{1.777954in}}%
\pgfpathlineto{\pgfqpoint{4.840311in}{1.719651in}}%
\pgfpathlineto{\pgfqpoint{4.840984in}{1.862760in}}%
\pgfpathlineto{\pgfqpoint{4.841656in}{1.762053in}}%
\pgfpathlineto{\pgfqpoint{4.842328in}{1.762053in}}%
\pgfpathlineto{\pgfqpoint{4.843000in}{1.844209in}}%
\pgfpathlineto{\pgfqpoint{4.843672in}{1.656046in}}%
\pgfpathlineto{\pgfqpoint{4.844344in}{1.656046in}}%
\pgfpathlineto{\pgfqpoint{4.844344in}{1.841559in}}%
\pgfpathlineto{\pgfqpoint{4.845689in}{1.799156in}}%
\pgfpathlineto{\pgfqpoint{4.846361in}{1.799156in}}%
\pgfpathlineto{\pgfqpoint{4.846361in}{1.751453in}}%
\pgfpathlineto{\pgfqpoint{4.847033in}{1.854810in}}%
\pgfpathlineto{\pgfqpoint{4.847705in}{1.767354in}}%
\pgfpathlineto{\pgfqpoint{4.848377in}{1.767354in}}%
\pgfpathlineto{\pgfqpoint{4.848377in}{1.754103in}}%
\pgfpathlineto{\pgfqpoint{4.849722in}{1.828308in}}%
\pgfpathlineto{\pgfqpoint{4.850394in}{1.828308in}}%
\pgfpathlineto{\pgfqpoint{4.850394in}{1.658697in}}%
\pgfpathlineto{\pgfqpoint{4.851738in}{1.724951in}}%
\pgfpathlineto{\pgfqpoint{4.852410in}{1.724951in}}%
\pgfpathlineto{\pgfqpoint{4.852410in}{1.796506in}}%
\pgfpathlineto{\pgfqpoint{4.853083in}{1.671947in}}%
\pgfpathlineto{\pgfqpoint{4.853755in}{1.679898in}}%
\pgfpathlineto{\pgfqpoint{4.854427in}{1.679898in}}%
\pgfpathlineto{\pgfqpoint{4.854427in}{1.817707in}}%
\pgfpathlineto{\pgfqpoint{4.855771in}{1.772654in}}%
\pgfpathlineto{\pgfqpoint{4.856443in}{1.772654in}}%
\pgfpathlineto{\pgfqpoint{4.856443in}{1.966117in}}%
\pgfpathlineto{\pgfqpoint{4.857116in}{1.762053in}}%
\pgfpathlineto{\pgfqpoint{4.857788in}{1.783255in}}%
\pgfpathlineto{\pgfqpoint{4.858460in}{1.783255in}}%
\pgfpathlineto{\pgfqpoint{4.859132in}{1.897212in}}%
\pgfpathlineto{\pgfqpoint{4.859804in}{1.873361in}}%
\pgfpathlineto{\pgfqpoint{4.860476in}{1.873361in}}%
\pgfpathlineto{\pgfqpoint{4.861821in}{1.724951in}}%
\pgfpathlineto{\pgfqpoint{4.862493in}{1.724951in}}%
\pgfpathlineto{\pgfqpoint{4.863837in}{1.883962in}}%
\pgfpathlineto{\pgfqpoint{4.864509in}{1.883962in}}%
\pgfpathlineto{\pgfqpoint{4.865854in}{1.788555in}}%
\pgfpathlineto{\pgfqpoint{4.866526in}{1.788555in}}%
\pgfpathlineto{\pgfqpoint{4.866526in}{1.838909in}}%
\pgfpathlineto{\pgfqpoint{4.867198in}{1.751453in}}%
\pgfpathlineto{\pgfqpoint{4.867870in}{1.807106in}}%
\pgfpathlineto{\pgfqpoint{4.868542in}{1.807106in}}%
\pgfpathlineto{\pgfqpoint{4.868542in}{1.865410in}}%
\pgfpathlineto{\pgfqpoint{4.869887in}{1.788555in}}%
\pgfpathlineto{\pgfqpoint{4.870559in}{1.788555in}}%
\pgfpathlineto{\pgfqpoint{4.871231in}{1.780605in}}%
\pgfpathlineto{\pgfqpoint{4.871903in}{1.870711in}}%
\pgfpathlineto{\pgfqpoint{4.872575in}{1.870711in}}%
\pgfpathlineto{\pgfqpoint{4.873248in}{1.809757in}}%
\pgfpathlineto{\pgfqpoint{4.873920in}{1.820357in}}%
\pgfpathlineto{\pgfqpoint{4.874592in}{1.820357in}}%
\pgfpathlineto{\pgfqpoint{4.875264in}{1.796506in}}%
\pgfpathlineto{\pgfqpoint{4.875264in}{1.902513in}}%
\pgfpathlineto{\pgfqpoint{4.875936in}{1.815057in}}%
\pgfpathlineto{\pgfqpoint{4.876608in}{1.815057in}}%
\pgfpathlineto{\pgfqpoint{4.876608in}{1.894562in}}%
\pgfpathlineto{\pgfqpoint{4.877953in}{1.894562in}}%
\pgfpathlineto{\pgfqpoint{4.878625in}{1.894562in}}%
\pgfpathlineto{\pgfqpoint{4.879297in}{1.939615in}}%
\pgfpathlineto{\pgfqpoint{4.879969in}{1.775304in}}%
\pgfpathlineto{\pgfqpoint{4.880641in}{1.775304in}}%
\pgfpathlineto{\pgfqpoint{4.880641in}{1.913113in}}%
\pgfpathlineto{\pgfqpoint{4.881986in}{1.886612in}}%
\pgfpathlineto{\pgfqpoint{4.883330in}{1.886612in}}%
\pgfpathlineto{\pgfqpoint{4.884002in}{1.913113in}}%
\pgfpathlineto{\pgfqpoint{4.884674in}{1.780605in}}%
\pgfpathlineto{\pgfqpoint{4.885347in}{1.780605in}}%
\pgfpathlineto{\pgfqpoint{4.885347in}{1.876011in}}%
\pgfpathlineto{\pgfqpoint{4.886691in}{1.812407in}}%
\pgfpathlineto{\pgfqpoint{4.887363in}{1.812407in}}%
\pgfpathlineto{\pgfqpoint{4.887363in}{1.883962in}}%
\pgfpathlineto{\pgfqpoint{4.888707in}{1.860110in}}%
\pgfpathlineto{\pgfqpoint{4.889380in}{1.860110in}}%
\pgfpathlineto{\pgfqpoint{4.890052in}{1.936965in}}%
\pgfpathlineto{\pgfqpoint{4.890724in}{1.923714in}}%
\pgfpathlineto{\pgfqpoint{4.891396in}{1.923714in}}%
\pgfpathlineto{\pgfqpoint{4.892068in}{1.730251in}}%
\pgfpathlineto{\pgfqpoint{4.892740in}{1.796506in}}%
\pgfpathlineto{\pgfqpoint{4.893413in}{1.796506in}}%
\pgfpathlineto{\pgfqpoint{4.894757in}{1.891912in}}%
\pgfpathlineto{\pgfqpoint{4.895429in}{1.891912in}}%
\pgfpathlineto{\pgfqpoint{4.896101in}{1.854810in}}%
\pgfpathlineto{\pgfqpoint{4.896773in}{1.934315in}}%
\pgfpathlineto{\pgfqpoint{4.897446in}{1.934315in}}%
\pgfpathlineto{\pgfqpoint{4.898118in}{2.032371in}}%
\pgfpathlineto{\pgfqpoint{4.898790in}{1.979368in}}%
\pgfpathlineto{\pgfqpoint{4.899462in}{1.979368in}}%
\pgfpathlineto{\pgfqpoint{4.900134in}{1.799156in}}%
\pgfpathlineto{\pgfqpoint{4.900806in}{1.833608in}}%
\pgfpathlineto{\pgfqpoint{4.901479in}{1.833608in}}%
\pgfpathlineto{\pgfqpoint{4.901479in}{1.971417in}}%
\pgfpathlineto{\pgfqpoint{4.902823in}{1.854810in}}%
\pgfpathlineto{\pgfqpoint{4.903495in}{1.854810in}}%
\pgfpathlineto{\pgfqpoint{4.904839in}{2.011170in}}%
\pgfpathlineto{\pgfqpoint{4.905512in}{2.011170in}}%
\pgfpathlineto{\pgfqpoint{4.905512in}{1.952866in}}%
\pgfpathlineto{\pgfqpoint{4.906856in}{2.042972in}}%
\pgfpathlineto{\pgfqpoint{4.907528in}{2.042972in}}%
\pgfpathlineto{\pgfqpoint{4.908200in}{1.936965in}}%
\pgfpathlineto{\pgfqpoint{4.908872in}{2.003219in}}%
\pgfpathlineto{\pgfqpoint{4.909545in}{2.003219in}}%
\pgfpathlineto{\pgfqpoint{4.909545in}{1.936965in}}%
\pgfpathlineto{\pgfqpoint{4.910889in}{1.952866in}}%
\pgfpathlineto{\pgfqpoint{4.911561in}{1.952866in}}%
\pgfpathlineto{\pgfqpoint{4.911561in}{1.966117in}}%
\pgfpathlineto{\pgfqpoint{4.912233in}{1.915764in}}%
\pgfpathlineto{\pgfqpoint{4.912905in}{1.950216in}}%
\pgfpathlineto{\pgfqpoint{4.913578in}{1.950216in}}%
\pgfpathlineto{\pgfqpoint{4.913578in}{1.891912in}}%
\pgfpathlineto{\pgfqpoint{4.914250in}{1.971417in}}%
\pgfpathlineto{\pgfqpoint{4.914922in}{1.913113in}}%
\pgfpathlineto{\pgfqpoint{4.915594in}{1.913113in}}%
\pgfpathlineto{\pgfqpoint{4.916938in}{2.029721in}}%
\pgfpathlineto{\pgfqpoint{4.917611in}{2.029721in}}%
\pgfpathlineto{\pgfqpoint{4.917611in}{1.889262in}}%
\pgfpathlineto{\pgfqpoint{4.918955in}{1.952866in}}%
\pgfpathlineto{\pgfqpoint{4.919627in}{1.952866in}}%
\pgfpathlineto{\pgfqpoint{4.920299in}{1.979368in}}%
\pgfpathlineto{\pgfqpoint{4.920971in}{1.958166in}}%
\pgfpathlineto{\pgfqpoint{4.921644in}{1.958166in}}%
\pgfpathlineto{\pgfqpoint{4.921644in}{1.974068in}}%
\pgfpathlineto{\pgfqpoint{4.922988in}{1.960817in}}%
\pgfpathlineto{\pgfqpoint{4.923660in}{1.960817in}}%
\pgfpathlineto{\pgfqpoint{4.924332in}{1.881311in}}%
\pgfpathlineto{\pgfqpoint{4.925004in}{1.929015in}}%
\pgfpathlineto{\pgfqpoint{4.925677in}{1.929015in}}%
\pgfpathlineto{\pgfqpoint{4.925677in}{2.080075in}}%
\pgfpathlineto{\pgfqpoint{4.926349in}{1.926364in}}%
\pgfpathlineto{\pgfqpoint{4.927021in}{1.976718in}}%
\pgfpathlineto{\pgfqpoint{4.927693in}{1.976718in}}%
\pgfpathlineto{\pgfqpoint{4.929037in}{1.923714in}}%
\pgfpathlineto{\pgfqpoint{4.929710in}{1.923714in}}%
\pgfpathlineto{\pgfqpoint{4.929710in}{2.011170in}}%
\pgfpathlineto{\pgfqpoint{4.931054in}{1.955516in}}%
\pgfpathlineto{\pgfqpoint{4.931726in}{1.955516in}}%
\pgfpathlineto{\pgfqpoint{4.933070in}{1.870711in}}%
\pgfpathlineto{\pgfqpoint{4.933743in}{1.870711in}}%
\pgfpathlineto{\pgfqpoint{4.934415in}{2.003219in}}%
\pgfpathlineto{\pgfqpoint{4.935087in}{1.931665in}}%
\pgfpathlineto{\pgfqpoint{4.935759in}{1.931665in}}%
\pgfpathlineto{\pgfqpoint{4.936431in}{1.889262in}}%
\pgfpathlineto{\pgfqpoint{4.937103in}{2.035022in}}%
\pgfpathlineto{\pgfqpoint{4.937776in}{2.035022in}}%
\pgfpathlineto{\pgfqpoint{4.937776in}{1.918414in}}%
\pgfpathlineto{\pgfqpoint{4.939120in}{1.947566in}}%
\pgfpathlineto{\pgfqpoint{4.939792in}{1.947566in}}%
\pgfpathlineto{\pgfqpoint{4.941136in}{2.058873in}}%
\pgfpathlineto{\pgfqpoint{4.941809in}{2.058873in}}%
\pgfpathlineto{\pgfqpoint{4.942481in}{1.934315in}}%
\pgfpathlineto{\pgfqpoint{4.943153in}{2.013820in}}%
\pgfpathlineto{\pgfqpoint{4.943825in}{2.013820in}}%
\pgfpathlineto{\pgfqpoint{4.943825in}{2.029721in}}%
\pgfpathlineto{\pgfqpoint{4.945170in}{1.881311in}}%
\pgfpathlineto{\pgfqpoint{4.945842in}{1.881311in}}%
\pgfpathlineto{\pgfqpoint{4.946514in}{2.072124in}}%
\pgfpathlineto{\pgfqpoint{4.947186in}{1.997919in}}%
\pgfpathlineto{\pgfqpoint{4.947858in}{1.997919in}}%
\pgfpathlineto{\pgfqpoint{4.947858in}{2.029721in}}%
\pgfpathlineto{\pgfqpoint{4.949203in}{2.019121in}}%
\pgfpathlineto{\pgfqpoint{4.949875in}{2.019121in}}%
\pgfpathlineto{\pgfqpoint{4.951219in}{1.950216in}}%
\pgfpathlineto{\pgfqpoint{4.951891in}{1.950216in}}%
\pgfpathlineto{\pgfqpoint{4.952563in}{2.024421in}}%
\pgfpathlineto{\pgfqpoint{4.953236in}{2.024421in}}%
\pgfpathlineto{\pgfqpoint{4.953908in}{2.024421in}}%
\pgfpathlineto{\pgfqpoint{4.954580in}{2.127778in}}%
\pgfpathlineto{\pgfqpoint{4.955252in}{1.915764in}}%
\pgfpathlineto{\pgfqpoint{4.955924in}{1.915764in}}%
\pgfpathlineto{\pgfqpoint{4.955924in}{2.029721in}}%
\pgfpathlineto{\pgfqpoint{4.957269in}{1.976718in}}%
\pgfpathlineto{\pgfqpoint{4.957941in}{1.976718in}}%
\pgfpathlineto{\pgfqpoint{4.959285in}{2.005870in}}%
\pgfpathlineto{\pgfqpoint{4.959957in}{2.005870in}}%
\pgfpathlineto{\pgfqpoint{4.959957in}{2.040322in}}%
\pgfpathlineto{\pgfqpoint{4.961302in}{1.958166in}}%
\pgfpathlineto{\pgfqpoint{4.961974in}{1.958166in}}%
\pgfpathlineto{\pgfqpoint{4.961974in}{1.968767in}}%
\pgfpathlineto{\pgfqpoint{4.963318in}{1.870711in}}%
\pgfpathlineto{\pgfqpoint{4.963990in}{1.870711in}}%
\pgfpathlineto{\pgfqpoint{4.965335in}{2.013820in}}%
\pgfpathlineto{\pgfqpoint{4.966007in}{2.013820in}}%
\pgfpathlineto{\pgfqpoint{4.966007in}{1.889262in}}%
\pgfpathlineto{\pgfqpoint{4.967351in}{2.019121in}}%
\pgfpathlineto{\pgfqpoint{4.968023in}{2.019121in}}%
\pgfpathlineto{\pgfqpoint{4.968023in}{2.109227in}}%
\pgfpathlineto{\pgfqpoint{4.969368in}{1.952866in}}%
\pgfpathlineto{\pgfqpoint{4.970040in}{1.952866in}}%
\pgfpathlineto{\pgfqpoint{4.970712in}{1.881311in}}%
\pgfpathlineto{\pgfqpoint{4.971384in}{2.019121in}}%
\pgfpathlineto{\pgfqpoint{4.972056in}{2.019121in}}%
\pgfpathlineto{\pgfqpoint{4.972056in}{1.868060in}}%
\pgfpathlineto{\pgfqpoint{4.973401in}{1.886612in}}%
\pgfpathlineto{\pgfqpoint{4.974073in}{1.886612in}}%
\pgfpathlineto{\pgfqpoint{4.974073in}{1.844209in}}%
\pgfpathlineto{\pgfqpoint{4.975417in}{2.042972in}}%
\pgfpathlineto{\pgfqpoint{4.976089in}{2.042972in}}%
\pgfpathlineto{\pgfqpoint{4.977434in}{1.910463in}}%
\pgfpathlineto{\pgfqpoint{4.978106in}{1.910463in}}%
\pgfpathlineto{\pgfqpoint{4.978106in}{1.791205in}}%
\pgfpathlineto{\pgfqpoint{4.979450in}{1.976718in}}%
\pgfpathlineto{\pgfqpoint{4.980122in}{1.976718in}}%
\pgfpathlineto{\pgfqpoint{4.980122in}{2.045622in}}%
\pgfpathlineto{\pgfqpoint{4.981467in}{1.971417in}}%
\pgfpathlineto{\pgfqpoint{4.982139in}{1.971417in}}%
\pgfpathlineto{\pgfqpoint{4.982139in}{1.963467in}}%
\pgfpathlineto{\pgfqpoint{4.983483in}{2.040322in}}%
\pgfpathlineto{\pgfqpoint{4.984155in}{2.040322in}}%
\pgfpathlineto{\pgfqpoint{4.984827in}{1.936965in}}%
\pgfpathlineto{\pgfqpoint{4.985500in}{1.942265in}}%
\pgfpathlineto{\pgfqpoint{4.986172in}{1.942265in}}%
\pgfpathlineto{\pgfqpoint{4.986844in}{2.029721in}}%
\pgfpathlineto{\pgfqpoint{4.987516in}{2.019121in}}%
\pgfpathlineto{\pgfqpoint{4.988188in}{2.019121in}}%
\pgfpathlineto{\pgfqpoint{4.988860in}{1.942265in}}%
\pgfpathlineto{\pgfqpoint{4.989533in}{2.000569in}}%
\pgfpathlineto{\pgfqpoint{4.990205in}{2.000569in}}%
\pgfpathlineto{\pgfqpoint{4.990205in}{1.862760in}}%
\pgfpathlineto{\pgfqpoint{4.991549in}{1.918414in}}%
\pgfpathlineto{\pgfqpoint{4.992221in}{1.918414in}}%
\pgfpathlineto{\pgfqpoint{4.993566in}{2.045622in}}%
\pgfpathlineto{\pgfqpoint{4.994238in}{2.045622in}}%
\pgfpathlineto{\pgfqpoint{4.994910in}{1.891912in}}%
\pgfpathlineto{\pgfqpoint{4.995582in}{1.992619in}}%
\pgfpathlineto{\pgfqpoint{4.996254in}{1.992619in}}%
\pgfpathlineto{\pgfqpoint{4.997599in}{1.860110in}}%
\pgfpathlineto{\pgfqpoint{4.998271in}{1.860110in}}%
\pgfpathlineto{\pgfqpoint{4.998271in}{1.968767in}}%
\pgfpathlineto{\pgfqpoint{4.999615in}{1.926364in}}%
\pgfpathlineto{\pgfqpoint{5.000287in}{1.926364in}}%
\pgfpathlineto{\pgfqpoint{5.000287in}{2.005870in}}%
\pgfpathlineto{\pgfqpoint{5.000959in}{1.836258in}}%
\pgfpathlineto{\pgfqpoint{5.001632in}{1.838909in}}%
\pgfpathlineto{\pgfqpoint{5.002304in}{1.838909in}}%
\pgfpathlineto{\pgfqpoint{5.002304in}{1.801806in}}%
\pgfpathlineto{\pgfqpoint{5.002976in}{1.899863in}}%
\pgfpathlineto{\pgfqpoint{5.003648in}{1.868060in}}%
\pgfpathlineto{\pgfqpoint{5.004320in}{1.868060in}}%
\pgfpathlineto{\pgfqpoint{5.004992in}{2.050923in}}%
\pgfpathlineto{\pgfqpoint{5.005665in}{1.844209in}}%
\pgfpathlineto{\pgfqpoint{5.006337in}{1.844209in}}%
\pgfpathlineto{\pgfqpoint{5.007009in}{2.011170in}}%
\pgfpathlineto{\pgfqpoint{5.007681in}{1.950216in}}%
\pgfpathlineto{\pgfqpoint{5.009025in}{1.950216in}}%
\pgfpathlineto{\pgfqpoint{5.009698in}{1.820357in}}%
\pgfpathlineto{\pgfqpoint{5.010370in}{1.939615in}}%
\pgfpathlineto{\pgfqpoint{5.011042in}{1.939615in}}%
\pgfpathlineto{\pgfqpoint{5.011042in}{1.823007in}}%
\pgfpathlineto{\pgfqpoint{5.011714in}{1.944916in}}%
\pgfpathlineto{\pgfqpoint{5.012386in}{1.873361in}}%
\pgfpathlineto{\pgfqpoint{5.013058in}{1.873361in}}%
\pgfpathlineto{\pgfqpoint{5.013731in}{1.815057in}}%
\pgfpathlineto{\pgfqpoint{5.014403in}{1.923714in}}%
\pgfpathlineto{\pgfqpoint{5.015075in}{1.923714in}}%
\pgfpathlineto{\pgfqpoint{5.015747in}{1.820357in}}%
\pgfpathlineto{\pgfqpoint{5.015747in}{1.950216in}}%
\pgfpathlineto{\pgfqpoint{5.016419in}{1.902513in}}%
\pgfpathlineto{\pgfqpoint{5.017091in}{1.902513in}}%
\pgfpathlineto{\pgfqpoint{5.017764in}{1.992619in}}%
\pgfpathlineto{\pgfqpoint{5.018436in}{1.894562in}}%
\pgfpathlineto{\pgfqpoint{5.019108in}{1.894562in}}%
\pgfpathlineto{\pgfqpoint{5.019780in}{1.860110in}}%
\pgfpathlineto{\pgfqpoint{5.020452in}{1.979368in}}%
\pgfpathlineto{\pgfqpoint{5.021124in}{1.979368in}}%
\pgfpathlineto{\pgfqpoint{5.022469in}{1.836258in}}%
\pgfpathlineto{\pgfqpoint{5.023141in}{1.836258in}}%
\pgfpathlineto{\pgfqpoint{5.023141in}{1.823007in}}%
\pgfpathlineto{\pgfqpoint{5.023813in}{2.008520in}}%
\pgfpathlineto{\pgfqpoint{5.024485in}{1.915764in}}%
\pgfpathlineto{\pgfqpoint{5.025157in}{1.915764in}}%
\pgfpathlineto{\pgfqpoint{5.025830in}{1.809757in}}%
\pgfpathlineto{\pgfqpoint{5.026502in}{1.944916in}}%
\pgfpathlineto{\pgfqpoint{5.027174in}{1.944916in}}%
\pgfpathlineto{\pgfqpoint{5.027174in}{1.796506in}}%
\pgfpathlineto{\pgfqpoint{5.028518in}{1.905163in}}%
\pgfpathlineto{\pgfqpoint{5.029190in}{1.905163in}}%
\pgfpathlineto{\pgfqpoint{5.029190in}{1.772654in}}%
\pgfpathlineto{\pgfqpoint{5.030535in}{1.873361in}}%
\pgfpathlineto{\pgfqpoint{5.031207in}{1.873361in}}%
\pgfpathlineto{\pgfqpoint{5.031207in}{1.788555in}}%
\pgfpathlineto{\pgfqpoint{5.032551in}{1.817707in}}%
\pgfpathlineto{\pgfqpoint{5.033223in}{1.817707in}}%
\pgfpathlineto{\pgfqpoint{5.033223in}{1.799156in}}%
\pgfpathlineto{\pgfqpoint{5.034568in}{1.918414in}}%
\pgfpathlineto{\pgfqpoint{5.035240in}{1.918414in}}%
\pgfpathlineto{\pgfqpoint{5.036584in}{1.868060in}}%
\pgfpathlineto{\pgfqpoint{5.037256in}{1.868060in}}%
\pgfpathlineto{\pgfqpoint{5.037929in}{1.754103in}}%
\pgfpathlineto{\pgfqpoint{5.038601in}{1.854810in}}%
\pgfpathlineto{\pgfqpoint{5.039273in}{1.854810in}}%
\pgfpathlineto{\pgfqpoint{5.040617in}{1.770004in}}%
\pgfpathlineto{\pgfqpoint{5.041289in}{1.770004in}}%
\pgfpathlineto{\pgfqpoint{5.041962in}{1.717000in}}%
\pgfpathlineto{\pgfqpoint{5.042634in}{1.883962in}}%
\pgfpathlineto{\pgfqpoint{5.043306in}{1.883962in}}%
\pgfpathlineto{\pgfqpoint{5.043306in}{1.767354in}}%
\pgfpathlineto{\pgfqpoint{5.044650in}{1.783255in}}%
\pgfpathlineto{\pgfqpoint{5.045322in}{1.783255in}}%
\pgfpathlineto{\pgfqpoint{5.045322in}{1.748803in}}%
\pgfpathlineto{\pgfqpoint{5.045995in}{1.902513in}}%
\pgfpathlineto{\pgfqpoint{5.046667in}{1.812407in}}%
\pgfpathlineto{\pgfqpoint{5.047339in}{1.812407in}}%
\pgfpathlineto{\pgfqpoint{5.048683in}{1.785905in}}%
\pgfpathlineto{\pgfqpoint{5.049355in}{1.785905in}}%
\pgfpathlineto{\pgfqpoint{5.050700in}{1.897212in}}%
\pgfpathlineto{\pgfqpoint{5.051372in}{1.897212in}}%
\pgfpathlineto{\pgfqpoint{5.052044in}{1.762053in}}%
\pgfpathlineto{\pgfqpoint{5.052716in}{1.899863in}}%
\pgfpathlineto{\pgfqpoint{5.053388in}{1.899863in}}%
\pgfpathlineto{\pgfqpoint{5.054061in}{1.687848in}}%
\pgfpathlineto{\pgfqpoint{5.054733in}{1.770004in}}%
\pgfpathlineto{\pgfqpoint{5.055405in}{1.770004in}}%
\pgfpathlineto{\pgfqpoint{5.056077in}{1.709050in}}%
\pgfpathlineto{\pgfqpoint{5.056749in}{1.743502in}}%
\pgfpathlineto{\pgfqpoint{5.057422in}{1.743502in}}%
\pgfpathlineto{\pgfqpoint{5.057422in}{1.799156in}}%
\pgfpathlineto{\pgfqpoint{5.058766in}{1.695799in}}%
\pgfpathlineto{\pgfqpoint{5.059438in}{1.695799in}}%
\pgfpathlineto{\pgfqpoint{5.060782in}{1.891912in}}%
\pgfpathlineto{\pgfqpoint{5.061455in}{1.891912in}}%
\pgfpathlineto{\pgfqpoint{5.061455in}{1.669297in}}%
\pgfpathlineto{\pgfqpoint{5.062799in}{1.754103in}}%
\pgfpathlineto{\pgfqpoint{5.063471in}{1.754103in}}%
\pgfpathlineto{\pgfqpoint{5.064143in}{1.709050in}}%
\pgfpathlineto{\pgfqpoint{5.064815in}{1.804456in}}%
\pgfpathlineto{\pgfqpoint{5.065488in}{1.804456in}}%
\pgfpathlineto{\pgfqpoint{5.066832in}{1.658697in}}%
\pgfpathlineto{\pgfqpoint{5.067504in}{1.658697in}}%
\pgfpathlineto{\pgfqpoint{5.067504in}{1.762053in}}%
\pgfpathlineto{\pgfqpoint{5.068848in}{1.690499in}}%
\pgfpathlineto{\pgfqpoint{5.069521in}{1.690499in}}%
\pgfpathlineto{\pgfqpoint{5.070193in}{1.730251in}}%
\pgfpathlineto{\pgfqpoint{5.070865in}{1.610993in}}%
\pgfpathlineto{\pgfqpoint{5.071537in}{1.610993in}}%
\pgfpathlineto{\pgfqpoint{5.071537in}{1.785905in}}%
\pgfpathlineto{\pgfqpoint{5.072881in}{1.743502in}}%
\pgfpathlineto{\pgfqpoint{5.074226in}{1.743502in}}%
\pgfpathlineto{\pgfqpoint{5.074226in}{1.740852in}}%
\pgfpathlineto{\pgfqpoint{5.075570in}{1.751453in}}%
\pgfpathlineto{\pgfqpoint{5.076242in}{1.751453in}}%
\pgfpathlineto{\pgfqpoint{5.076914in}{1.687848in}}%
\pgfpathlineto{\pgfqpoint{5.077587in}{1.825658in}}%
\pgfpathlineto{\pgfqpoint{5.078259in}{1.825658in}}%
\pgfpathlineto{\pgfqpoint{5.078259in}{1.677248in}}%
\pgfpathlineto{\pgfqpoint{5.079603in}{1.695799in}}%
\pgfpathlineto{\pgfqpoint{5.080275in}{1.695799in}}%
\pgfpathlineto{\pgfqpoint{5.080275in}{1.626894in}}%
\pgfpathlineto{\pgfqpoint{5.081620in}{1.770004in}}%
\pgfpathlineto{\pgfqpoint{5.082292in}{1.770004in}}%
\pgfpathlineto{\pgfqpoint{5.082964in}{1.663997in}}%
\pgfpathlineto{\pgfqpoint{5.083636in}{1.791205in}}%
\pgfpathlineto{\pgfqpoint{5.084308in}{1.791205in}}%
\pgfpathlineto{\pgfqpoint{5.084980in}{1.584492in}}%
\pgfpathlineto{\pgfqpoint{5.085653in}{1.648096in}}%
\pgfpathlineto{\pgfqpoint{5.086325in}{1.648096in}}%
\pgfpathlineto{\pgfqpoint{5.086325in}{1.687848in}}%
\pgfpathlineto{\pgfqpoint{5.086997in}{1.584492in}}%
\pgfpathlineto{\pgfqpoint{5.087669in}{1.634845in}}%
\pgfpathlineto{\pgfqpoint{5.088341in}{1.634845in}}%
\pgfpathlineto{\pgfqpoint{5.089013in}{1.581841in}}%
\pgfpathlineto{\pgfqpoint{5.089686in}{1.624244in}}%
\pgfpathlineto{\pgfqpoint{5.090358in}{1.624244in}}%
\pgfpathlineto{\pgfqpoint{5.090358in}{1.738202in}}%
\pgfpathlineto{\pgfqpoint{5.091030in}{1.523537in}}%
\pgfpathlineto{\pgfqpoint{5.091702in}{1.685198in}}%
\pgfpathlineto{\pgfqpoint{5.092374in}{1.685198in}}%
\pgfpathlineto{\pgfqpoint{5.093046in}{1.642795in}}%
\pgfpathlineto{\pgfqpoint{5.093719in}{1.642795in}}%
\pgfpathlineto{\pgfqpoint{5.094391in}{1.642795in}}%
\pgfpathlineto{\pgfqpoint{5.095735in}{1.584492in}}%
\pgfpathlineto{\pgfqpoint{5.096407in}{1.584492in}}%
\pgfpathlineto{\pgfqpoint{5.097752in}{1.626894in}}%
\pgfpathlineto{\pgfqpoint{5.098424in}{1.626894in}}%
\pgfpathlineto{\pgfqpoint{5.098424in}{1.520887in}}%
\pgfpathlineto{\pgfqpoint{5.099768in}{1.568590in}}%
\pgfpathlineto{\pgfqpoint{5.101112in}{1.568590in}}%
\pgfpathlineto{\pgfqpoint{5.101785in}{1.616294in}}%
\pgfpathlineto{\pgfqpoint{5.102457in}{1.459933in}}%
\pgfpathlineto{\pgfqpoint{5.103129in}{1.459933in}}%
\pgfpathlineto{\pgfqpoint{5.103129in}{1.626894in}}%
\pgfpathlineto{\pgfqpoint{5.104473in}{1.595092in}}%
\pgfpathlineto{\pgfqpoint{5.105145in}{1.595092in}}%
\pgfpathlineto{\pgfqpoint{5.105145in}{1.433431in}}%
\pgfpathlineto{\pgfqpoint{5.106490in}{1.528838in}}%
\pgfpathlineto{\pgfqpoint{5.107162in}{1.528838in}}%
\pgfpathlineto{\pgfqpoint{5.107162in}{1.534138in}}%
\pgfpathlineto{\pgfqpoint{5.108506in}{1.467884in}}%
\pgfpathlineto{\pgfqpoint{5.109178in}{1.467884in}}%
\pgfpathlineto{\pgfqpoint{5.109851in}{1.589792in}}%
\pgfpathlineto{\pgfqpoint{5.110523in}{1.536788in}}%
\pgfpathlineto{\pgfqpoint{5.111195in}{1.536788in}}%
\pgfpathlineto{\pgfqpoint{5.111195in}{1.563290in}}%
\pgfpathlineto{\pgfqpoint{5.111867in}{1.475834in}}%
\pgfpathlineto{\pgfqpoint{5.112539in}{1.550039in}}%
\pgfpathlineto{\pgfqpoint{5.113211in}{1.550039in}}%
\pgfpathlineto{\pgfqpoint{5.113884in}{1.489085in}}%
\pgfpathlineto{\pgfqpoint{5.114556in}{1.589792in}}%
\pgfpathlineto{\pgfqpoint{5.115228in}{1.589792in}}%
\pgfpathlineto{\pgfqpoint{5.115900in}{1.515587in}}%
\pgfpathlineto{\pgfqpoint{5.116572in}{1.624244in}}%
\pgfpathlineto{\pgfqpoint{5.117244in}{1.624244in}}%
\pgfpathlineto{\pgfqpoint{5.117917in}{1.534138in}}%
\pgfpathlineto{\pgfqpoint{5.118589in}{1.573891in}}%
\pgfpathlineto{\pgfqpoint{5.119261in}{1.573891in}}%
\pgfpathlineto{\pgfqpoint{5.119933in}{1.412230in}}%
\pgfpathlineto{\pgfqpoint{5.120605in}{1.457283in}}%
\pgfpathlineto{\pgfqpoint{5.121277in}{1.457283in}}%
\pgfpathlineto{\pgfqpoint{5.121277in}{1.441382in}}%
\pgfpathlineto{\pgfqpoint{5.122622in}{1.571241in}}%
\pgfpathlineto{\pgfqpoint{5.123294in}{1.571241in}}%
\pgfpathlineto{\pgfqpoint{5.123294in}{1.457283in}}%
\pgfpathlineto{\pgfqpoint{5.124638in}{1.499686in}}%
\pgfpathlineto{\pgfqpoint{5.125310in}{1.499686in}}%
\pgfpathlineto{\pgfqpoint{5.125310in}{1.555340in}}%
\pgfpathlineto{\pgfqpoint{5.126655in}{1.454633in}}%
\pgfpathlineto{\pgfqpoint{5.127327in}{1.454633in}}%
\pgfpathlineto{\pgfqpoint{5.127327in}{1.489085in}}%
\pgfpathlineto{\pgfqpoint{5.127999in}{1.359227in}}%
\pgfpathlineto{\pgfqpoint{5.128671in}{1.451983in}}%
\pgfpathlineto{\pgfqpoint{5.129343in}{1.451983in}}%
\pgfpathlineto{\pgfqpoint{5.130016in}{1.518237in}}%
\pgfpathlineto{\pgfqpoint{5.130688in}{1.398979in}}%
\pgfpathlineto{\pgfqpoint{5.131360in}{1.398979in}}%
\pgfpathlineto{\pgfqpoint{5.132032in}{1.489085in}}%
\pgfpathlineto{\pgfqpoint{5.132704in}{1.377778in}}%
\pgfpathlineto{\pgfqpoint{5.133376in}{1.377778in}}%
\pgfpathlineto{\pgfqpoint{5.134049in}{1.499686in}}%
\pgfpathlineto{\pgfqpoint{5.134721in}{1.446682in}}%
\pgfpathlineto{\pgfqpoint{5.135393in}{1.446682in}}%
\pgfpathlineto{\pgfqpoint{5.135393in}{1.409580in}}%
\pgfpathlineto{\pgfqpoint{5.136065in}{1.491735in}}%
\pgfpathlineto{\pgfqpoint{5.136737in}{1.475834in}}%
\pgfpathlineto{\pgfqpoint{5.137409in}{1.475834in}}%
\pgfpathlineto{\pgfqpoint{5.137409in}{1.369827in}}%
\pgfpathlineto{\pgfqpoint{5.138754in}{1.417530in}}%
\pgfpathlineto{\pgfqpoint{5.139426in}{1.417530in}}%
\pgfpathlineto{\pgfqpoint{5.140098in}{1.441382in}}%
\pgfpathlineto{\pgfqpoint{5.140770in}{1.391029in}}%
\pgfpathlineto{\pgfqpoint{5.141442in}{1.391029in}}%
\pgfpathlineto{\pgfqpoint{5.141442in}{1.433431in}}%
\pgfpathlineto{\pgfqpoint{5.142787in}{1.388378in}}%
\pgfpathlineto{\pgfqpoint{5.143459in}{1.388378in}}%
\pgfpathlineto{\pgfqpoint{5.143459in}{1.409580in}}%
\pgfpathlineto{\pgfqpoint{5.144131in}{1.377778in}}%
\pgfpathlineto{\pgfqpoint{5.144803in}{1.404280in}}%
\pgfpathlineto{\pgfqpoint{5.145475in}{1.404280in}}%
\pgfpathlineto{\pgfqpoint{5.145475in}{1.348626in}}%
\pgfpathlineto{\pgfqpoint{5.146820in}{1.441382in}}%
\pgfpathlineto{\pgfqpoint{5.147492in}{1.441382in}}%
\pgfpathlineto{\pgfqpoint{5.147492in}{1.330075in}}%
\pgfpathlineto{\pgfqpoint{5.148836in}{1.398979in}}%
\pgfpathlineto{\pgfqpoint{5.149508in}{1.398979in}}%
\pgfpathlineto{\pgfqpoint{5.150181in}{1.412230in}}%
\pgfpathlineto{\pgfqpoint{5.150853in}{1.295622in}}%
\pgfpathlineto{\pgfqpoint{5.151525in}{1.295622in}}%
\pgfpathlineto{\pgfqpoint{5.152869in}{1.388378in}}%
\pgfpathlineto{\pgfqpoint{5.153541in}{1.388378in}}%
\pgfpathlineto{\pgfqpoint{5.154886in}{1.303573in}}%
\pgfpathlineto{\pgfqpoint{5.155558in}{1.303573in}}%
\pgfpathlineto{\pgfqpoint{5.155558in}{1.470534in}}%
\pgfpathlineto{\pgfqpoint{5.156902in}{1.285022in}}%
\pgfpathlineto{\pgfqpoint{5.158247in}{1.285022in}}%
\pgfpathlineto{\pgfqpoint{5.159591in}{1.375128in}}%
\pgfpathlineto{\pgfqpoint{5.160263in}{1.375128in}}%
\pgfpathlineto{\pgfqpoint{5.160935in}{1.300923in}}%
\pgfpathlineto{\pgfqpoint{5.161607in}{1.425481in}}%
\pgfpathlineto{\pgfqpoint{5.162280in}{1.425481in}}%
\pgfpathlineto{\pgfqpoint{5.162952in}{1.290322in}}%
\pgfpathlineto{\pgfqpoint{5.163624in}{1.369827in}}%
\pgfpathlineto{\pgfqpoint{5.164296in}{1.369827in}}%
\pgfpathlineto{\pgfqpoint{5.164968in}{1.277071in}}%
\pgfpathlineto{\pgfqpoint{5.165640in}{1.303573in}}%
\pgfpathlineto{\pgfqpoint{5.166313in}{1.303573in}}%
\pgfpathlineto{\pgfqpoint{5.166985in}{1.369827in}}%
\pgfpathlineto{\pgfqpoint{5.167657in}{1.237318in}}%
\pgfpathlineto{\pgfqpoint{5.168329in}{1.237318in}}%
\pgfpathlineto{\pgfqpoint{5.169001in}{1.295622in}}%
\pgfpathlineto{\pgfqpoint{5.169674in}{1.239969in}}%
\pgfpathlineto{\pgfqpoint{5.170346in}{1.239969in}}%
\pgfpathlineto{\pgfqpoint{5.170346in}{1.229368in}}%
\pgfpathlineto{\pgfqpoint{5.171018in}{1.356576in}}%
\pgfpathlineto{\pgfqpoint{5.171690in}{1.287672in}}%
\pgfpathlineto{\pgfqpoint{5.172362in}{1.287672in}}%
\pgfpathlineto{\pgfqpoint{5.172362in}{1.229368in}}%
\pgfpathlineto{\pgfqpoint{5.173707in}{1.303573in}}%
\pgfpathlineto{\pgfqpoint{5.174379in}{1.303573in}}%
\pgfpathlineto{\pgfqpoint{5.174379in}{1.245269in}}%
\pgfpathlineto{\pgfqpoint{5.175051in}{1.375128in}}%
\pgfpathlineto{\pgfqpoint{5.175723in}{1.335375in}}%
\pgfpathlineto{\pgfqpoint{5.176395in}{1.335375in}}%
\pgfpathlineto{\pgfqpoint{5.177740in}{1.186965in}}%
\pgfpathlineto{\pgfqpoint{5.178412in}{1.186965in}}%
\pgfpathlineto{\pgfqpoint{5.179756in}{1.261170in}}%
\pgfpathlineto{\pgfqpoint{5.180428in}{1.261170in}}%
\pgfpathlineto{\pgfqpoint{5.180428in}{1.221417in}}%
\pgfpathlineto{\pgfqpoint{5.181100in}{1.298272in}}%
\pgfpathlineto{\pgfqpoint{5.181773in}{1.224068in}}%
\pgfpathlineto{\pgfqpoint{5.182445in}{1.224068in}}%
\pgfpathlineto{\pgfqpoint{5.183117in}{1.279721in}}%
\pgfpathlineto{\pgfqpoint{5.183789in}{1.258520in}}%
\pgfpathlineto{\pgfqpoint{5.184461in}{1.258520in}}%
\pgfpathlineto{\pgfqpoint{5.184461in}{1.332725in}}%
\pgfpathlineto{\pgfqpoint{5.185806in}{1.218767in}}%
\pgfpathlineto{\pgfqpoint{5.186478in}{1.218767in}}%
\pgfpathlineto{\pgfqpoint{5.186478in}{1.123361in}}%
\pgfpathlineto{\pgfqpoint{5.187822in}{1.123361in}}%
\pgfpathlineto{\pgfqpoint{5.188494in}{1.123361in}}%
\pgfpathlineto{\pgfqpoint{5.189839in}{1.237318in}}%
\pgfpathlineto{\pgfqpoint{5.190511in}{1.237318in}}%
\pgfpathlineto{\pgfqpoint{5.190511in}{1.163113in}}%
\pgfpathlineto{\pgfqpoint{5.191183in}{1.239969in}}%
\pgfpathlineto{\pgfqpoint{5.191855in}{1.186965in}}%
\pgfpathlineto{\pgfqpoint{5.192527in}{1.186965in}}%
\pgfpathlineto{\pgfqpoint{5.192527in}{1.253219in}}%
\pgfpathlineto{\pgfqpoint{5.193199in}{1.139262in}}%
\pgfpathlineto{\pgfqpoint{5.193872in}{1.250569in}}%
\pgfpathlineto{\pgfqpoint{5.194544in}{1.250569in}}%
\pgfpathlineto{\pgfqpoint{5.194544in}{1.263820in}}%
\pgfpathlineto{\pgfqpoint{5.195888in}{1.202866in}}%
\pgfpathlineto{\pgfqpoint{5.196560in}{1.202866in}}%
\pgfpathlineto{\pgfqpoint{5.197232in}{1.224068in}}%
\pgfpathlineto{\pgfqpoint{5.197905in}{1.139262in}}%
\pgfpathlineto{\pgfqpoint{5.198577in}{1.139262in}}%
\pgfpathlineto{\pgfqpoint{5.198577in}{1.245269in}}%
\pgfpathlineto{\pgfqpoint{5.199921in}{1.181665in}}%
\pgfpathlineto{\pgfqpoint{5.200593in}{1.181665in}}%
\pgfpathlineto{\pgfqpoint{5.200593in}{1.128661in}}%
\pgfpathlineto{\pgfqpoint{5.201265in}{1.224068in}}%
\pgfpathlineto{\pgfqpoint{5.201938in}{1.208166in}}%
\pgfpathlineto{\pgfqpoint{5.202610in}{1.208166in}}%
\pgfpathlineto{\pgfqpoint{5.202610in}{1.118060in}}%
\pgfpathlineto{\pgfqpoint{5.203954in}{1.155163in}}%
\pgfpathlineto{\pgfqpoint{5.204626in}{1.155163in}}%
\pgfpathlineto{\pgfqpoint{5.204626in}{1.107460in}}%
\pgfpathlineto{\pgfqpoint{5.205971in}{1.163113in}}%
\pgfpathlineto{\pgfqpoint{5.206643in}{1.163113in}}%
\pgfpathlineto{\pgfqpoint{5.207987in}{1.088909in}}%
\pgfpathlineto{\pgfqpoint{5.208659in}{1.088909in}}%
\pgfpathlineto{\pgfqpoint{5.209331in}{1.083608in}}%
\pgfpathlineto{\pgfqpoint{5.210004in}{1.208166in}}%
\pgfpathlineto{\pgfqpoint{5.210676in}{1.208166in}}%
\pgfpathlineto{\pgfqpoint{5.210676in}{1.118060in}}%
\pgfpathlineto{\pgfqpoint{5.212020in}{1.184315in}}%
\pgfpathlineto{\pgfqpoint{5.212692in}{1.184315in}}%
\pgfpathlineto{\pgfqpoint{5.213364in}{1.049156in}}%
\pgfpathlineto{\pgfqpoint{5.214037in}{1.189615in}}%
\pgfpathlineto{\pgfqpoint{5.214709in}{1.189615in}}%
\pgfpathlineto{\pgfqpoint{5.215381in}{1.046506in}}%
\pgfpathlineto{\pgfqpoint{5.216053in}{1.171064in}}%
\pgfpathlineto{\pgfqpoint{5.216725in}{1.171064in}}%
\pgfpathlineto{\pgfqpoint{5.216725in}{1.078308in}}%
\pgfpathlineto{\pgfqpoint{5.218070in}{1.115410in}}%
\pgfpathlineto{\pgfqpoint{5.218742in}{1.115410in}}%
\pgfpathlineto{\pgfqpoint{5.219414in}{1.075658in}}%
\pgfpathlineto{\pgfqpoint{5.220086in}{1.123361in}}%
\pgfpathlineto{\pgfqpoint{5.220758in}{1.123361in}}%
\pgfpathlineto{\pgfqpoint{5.222103in}{1.080958in}}%
\pgfpathlineto{\pgfqpoint{5.222775in}{1.080958in}}%
\pgfpathlineto{\pgfqpoint{5.224119in}{1.001453in}}%
\pgfpathlineto{\pgfqpoint{5.224791in}{1.001453in}}%
\pgfpathlineto{\pgfqpoint{5.224791in}{1.086258in}}%
\pgfpathlineto{\pgfqpoint{5.226136in}{1.067707in}}%
\pgfpathlineto{\pgfqpoint{5.226808in}{1.067707in}}%
\pgfpathlineto{\pgfqpoint{5.227480in}{1.088909in}}%
\pgfpathlineto{\pgfqpoint{5.228152in}{1.065057in}}%
\pgfpathlineto{\pgfqpoint{5.229496in}{1.065057in}}%
\pgfpathlineto{\pgfqpoint{5.230169in}{0.982901in}}%
\pgfpathlineto{\pgfqpoint{5.230841in}{1.041205in}}%
\pgfpathlineto{\pgfqpoint{5.231513in}{1.041205in}}%
\pgfpathlineto{\pgfqpoint{5.231513in}{0.996152in}}%
\pgfpathlineto{\pgfqpoint{5.232857in}{1.014704in}}%
\pgfpathlineto{\pgfqpoint{5.233529in}{1.014704in}}%
\pgfpathlineto{\pgfqpoint{5.233529in}{1.054456in}}%
\pgfpathlineto{\pgfqpoint{5.234874in}{1.006753in}}%
\pgfpathlineto{\pgfqpoint{5.235546in}{1.006753in}}%
\pgfpathlineto{\pgfqpoint{5.235546in}{1.057106in}}%
\pgfpathlineto{\pgfqpoint{5.236890in}{1.014704in}}%
\pgfpathlineto{\pgfqpoint{5.237562in}{1.014704in}}%
\pgfpathlineto{\pgfqpoint{5.237562in}{1.057106in}}%
\pgfpathlineto{\pgfqpoint{5.238907in}{1.054456in}}%
\pgfpathlineto{\pgfqpoint{5.239579in}{1.054456in}}%
\pgfpathlineto{\pgfqpoint{5.240251in}{1.057106in}}%
\pgfpathlineto{\pgfqpoint{5.240923in}{0.996152in}}%
\pgfpathlineto{\pgfqpoint{5.241595in}{0.996152in}}%
\pgfpathlineto{\pgfqpoint{5.242268in}{1.004103in}}%
\pgfpathlineto{\pgfqpoint{5.242940in}{0.982901in}}%
\pgfpathlineto{\pgfqpoint{5.243612in}{0.982901in}}%
\pgfpathlineto{\pgfqpoint{5.243612in}{1.083608in}}%
\pgfpathlineto{\pgfqpoint{5.244956in}{0.980251in}}%
\pgfpathlineto{\pgfqpoint{5.245628in}{0.980251in}}%
\pgfpathlineto{\pgfqpoint{5.245628in}{0.996152in}}%
\pgfpathlineto{\pgfqpoint{5.246301in}{0.935198in}}%
\pgfpathlineto{\pgfqpoint{5.246973in}{0.988202in}}%
\pgfpathlineto{\pgfqpoint{5.247645in}{0.988202in}}%
\pgfpathlineto{\pgfqpoint{5.248317in}{0.974951in}}%
\pgfpathlineto{\pgfqpoint{5.248989in}{1.049156in}}%
\pgfpathlineto{\pgfqpoint{5.249661in}{1.049156in}}%
\pgfpathlineto{\pgfqpoint{5.250334in}{0.964350in}}%
\pgfpathlineto{\pgfqpoint{5.251006in}{0.990852in}}%
\pgfpathlineto{\pgfqpoint{5.251678in}{0.990852in}}%
\pgfpathlineto{\pgfqpoint{5.251678in}{0.985552in}}%
\pgfpathlineto{\pgfqpoint{5.252350in}{1.022654in}}%
\pgfpathlineto{\pgfqpoint{5.253022in}{1.014704in}}%
\pgfpathlineto{\pgfqpoint{5.253694in}{1.014704in}}%
\pgfpathlineto{\pgfqpoint{5.254367in}{0.935198in}}%
\pgfpathlineto{\pgfqpoint{5.255039in}{0.982901in}}%
\pgfpathlineto{\pgfqpoint{5.255711in}{0.982901in}}%
\pgfpathlineto{\pgfqpoint{5.256383in}{1.030605in}}%
\pgfpathlineto{\pgfqpoint{5.257055in}{0.940499in}}%
\pgfpathlineto{\pgfqpoint{5.257727in}{0.940499in}}%
\pgfpathlineto{\pgfqpoint{5.257727in}{0.980251in}}%
\pgfpathlineto{\pgfqpoint{5.259072in}{0.961700in}}%
\pgfpathlineto{\pgfqpoint{5.259744in}{0.961700in}}%
\pgfpathlineto{\pgfqpoint{5.260416in}{0.985552in}}%
\pgfpathlineto{\pgfqpoint{5.261088in}{0.964350in}}%
\pgfpathlineto{\pgfqpoint{5.262433in}{0.964350in}}%
\pgfpathlineto{\pgfqpoint{5.263105in}{0.924598in}}%
\pgfpathlineto{\pgfqpoint{5.263777in}{0.988202in}}%
\pgfpathlineto{\pgfqpoint{5.264449in}{0.988202in}}%
\pgfpathlineto{\pgfqpoint{5.265121in}{0.906046in}}%
\pgfpathlineto{\pgfqpoint{5.265793in}{0.908697in}}%
\pgfpathlineto{\pgfqpoint{5.266466in}{0.908697in}}%
\pgfpathlineto{\pgfqpoint{5.266466in}{0.921947in}}%
\pgfpathlineto{\pgfqpoint{5.267138in}{0.900746in}}%
\pgfpathlineto{\pgfqpoint{5.267810in}{0.916647in}}%
\pgfpathlineto{\pgfqpoint{5.268482in}{0.916647in}}%
\pgfpathlineto{\pgfqpoint{5.269154in}{0.898096in}}%
\pgfpathlineto{\pgfqpoint{5.269826in}{0.937848in}}%
\pgfpathlineto{\pgfqpoint{5.270499in}{0.937848in}}%
\pgfpathlineto{\pgfqpoint{5.270499in}{0.951099in}}%
\pgfpathlineto{\pgfqpoint{5.271843in}{0.924598in}}%
\pgfpathlineto{\pgfqpoint{5.272515in}{0.924598in}}%
\pgfpathlineto{\pgfqpoint{5.272515in}{0.898096in}}%
\pgfpathlineto{\pgfqpoint{5.273859in}{0.932548in}}%
\pgfpathlineto{\pgfqpoint{5.274532in}{0.932548in}}%
\pgfpathlineto{\pgfqpoint{5.275876in}{0.887495in}}%
\pgfpathlineto{\pgfqpoint{5.276548in}{0.887495in}}%
\pgfpathlineto{\pgfqpoint{5.277220in}{0.868944in}}%
\pgfpathlineto{\pgfqpoint{5.277892in}{0.932548in}}%
\pgfpathlineto{\pgfqpoint{5.278565in}{0.932548in}}%
\pgfpathlineto{\pgfqpoint{5.279237in}{0.879545in}}%
\pgfpathlineto{\pgfqpoint{5.279909in}{0.882195in}}%
\pgfpathlineto{\pgfqpoint{5.280581in}{0.882195in}}%
\pgfpathlineto{\pgfqpoint{5.281926in}{0.921947in}}%
\pgfpathlineto{\pgfqpoint{5.282598in}{0.921947in}}%
\pgfpathlineto{\pgfqpoint{5.282598in}{0.879545in}}%
\pgfpathlineto{\pgfqpoint{5.283270in}{0.935198in}}%
\pgfpathlineto{\pgfqpoint{5.283942in}{0.887495in}}%
\pgfpathlineto{\pgfqpoint{5.284614in}{0.887495in}}%
\pgfpathlineto{\pgfqpoint{5.285286in}{0.900746in}}%
\pgfpathlineto{\pgfqpoint{5.285959in}{0.839792in}}%
\pgfpathlineto{\pgfqpoint{5.286631in}{0.839792in}}%
\pgfpathlineto{\pgfqpoint{5.287303in}{0.900746in}}%
\pgfpathlineto{\pgfqpoint{5.287975in}{0.831841in}}%
\pgfpathlineto{\pgfqpoint{5.288647in}{0.831841in}}%
\pgfpathlineto{\pgfqpoint{5.289992in}{0.860993in}}%
\pgfpathlineto{\pgfqpoint{5.290664in}{0.860993in}}%
\pgfpathlineto{\pgfqpoint{5.291336in}{0.908697in}}%
\pgfpathlineto{\pgfqpoint{5.292008in}{0.876894in}}%
\pgfpathlineto{\pgfqpoint{5.292680in}{0.876894in}}%
\pgfpathlineto{\pgfqpoint{5.292680in}{0.853043in}}%
\pgfpathlineto{\pgfqpoint{5.293352in}{0.900746in}}%
\pgfpathlineto{\pgfqpoint{5.294025in}{0.874244in}}%
\pgfpathlineto{\pgfqpoint{5.294697in}{0.874244in}}%
\pgfpathlineto{\pgfqpoint{5.295369in}{0.818590in}}%
\pgfpathlineto{\pgfqpoint{5.296041in}{0.858343in}}%
\pgfpathlineto{\pgfqpoint{5.296713in}{0.858343in}}%
\pgfpathlineto{\pgfqpoint{5.296713in}{0.892795in}}%
\pgfpathlineto{\pgfqpoint{5.298058in}{0.837142in}}%
\pgfpathlineto{\pgfqpoint{5.298730in}{0.837142in}}%
\pgfpathlineto{\pgfqpoint{5.300074in}{0.903396in}}%
\pgfpathlineto{\pgfqpoint{5.300746in}{0.903396in}}%
\pgfpathlineto{\pgfqpoint{5.301418in}{0.807990in}}%
\pgfpathlineto{\pgfqpoint{5.302091in}{0.850393in}}%
\pgfpathlineto{\pgfqpoint{5.302763in}{0.850393in}}%
\pgfpathlineto{\pgfqpoint{5.303435in}{0.818590in}}%
\pgfpathlineto{\pgfqpoint{5.304107in}{0.890145in}}%
\pgfpathlineto{\pgfqpoint{5.304779in}{0.890145in}}%
\pgfpathlineto{\pgfqpoint{5.305451in}{0.834492in}}%
\pgfpathlineto{\pgfqpoint{5.306124in}{0.834492in}}%
\pgfpathlineto{\pgfqpoint{5.306796in}{0.834492in}}%
\pgfpathlineto{\pgfqpoint{5.306796in}{0.821241in}}%
\pgfpathlineto{\pgfqpoint{5.308140in}{0.858343in}}%
\pgfpathlineto{\pgfqpoint{5.308812in}{0.858343in}}%
\pgfpathlineto{\pgfqpoint{5.309484in}{0.810640in}}%
\pgfpathlineto{\pgfqpoint{5.310157in}{0.823891in}}%
\pgfpathlineto{\pgfqpoint{5.310829in}{0.823891in}}%
\pgfpathlineto{\pgfqpoint{5.310829in}{0.800039in}}%
\pgfpathlineto{\pgfqpoint{5.311501in}{0.850393in}}%
\pgfpathlineto{\pgfqpoint{5.312173in}{0.815940in}}%
\pgfpathlineto{\pgfqpoint{5.312845in}{0.815940in}}%
\pgfpathlineto{\pgfqpoint{5.313517in}{0.778838in}}%
\pgfpathlineto{\pgfqpoint{5.314190in}{0.858343in}}%
\pgfpathlineto{\pgfqpoint{5.314862in}{0.858343in}}%
\pgfpathlineto{\pgfqpoint{5.315534in}{0.800039in}}%
\pgfpathlineto{\pgfqpoint{5.316206in}{0.826541in}}%
\pgfpathlineto{\pgfqpoint{5.316878in}{0.826541in}}%
\pgfpathlineto{\pgfqpoint{5.317550in}{0.776188in}}%
\pgfpathlineto{\pgfqpoint{5.318223in}{0.839792in}}%
\pgfpathlineto{\pgfqpoint{5.318895in}{0.839792in}}%
\pgfpathlineto{\pgfqpoint{5.320239in}{0.807990in}}%
\pgfpathlineto{\pgfqpoint{5.320911in}{0.807990in}}%
\pgfpathlineto{\pgfqpoint{5.321583in}{0.855693in}}%
\pgfpathlineto{\pgfqpoint{5.321583in}{0.773537in}}%
\pgfpathlineto{\pgfqpoint{5.322256in}{0.829191in}}%
\pgfpathlineto{\pgfqpoint{5.322928in}{0.829191in}}%
\pgfpathlineto{\pgfqpoint{5.322928in}{0.773537in}}%
\pgfpathlineto{\pgfqpoint{5.324272in}{0.794739in}}%
\pgfpathlineto{\pgfqpoint{5.324944in}{0.794739in}}%
\pgfpathlineto{\pgfqpoint{5.324944in}{0.776188in}}%
\pgfpathlineto{\pgfqpoint{5.326289in}{0.821241in}}%
\pgfpathlineto{\pgfqpoint{5.326961in}{0.821241in}}%
\pgfpathlineto{\pgfqpoint{5.328305in}{0.768237in}}%
\pgfpathlineto{\pgfqpoint{5.328977in}{0.768237in}}%
\pgfpathlineto{\pgfqpoint{5.328977in}{0.749686in}}%
\pgfpathlineto{\pgfqpoint{5.329649in}{0.786788in}}%
\pgfpathlineto{\pgfqpoint{5.330322in}{0.757636in}}%
\pgfpathlineto{\pgfqpoint{5.330994in}{0.757636in}}%
\pgfpathlineto{\pgfqpoint{5.331666in}{0.784138in}}%
\pgfpathlineto{\pgfqpoint{5.332338in}{0.760287in}}%
\pgfpathlineto{\pgfqpoint{5.333010in}{0.760287in}}%
\pgfpathlineto{\pgfqpoint{5.333682in}{0.754986in}}%
\pgfpathlineto{\pgfqpoint{5.334355in}{0.813290in}}%
\pgfpathlineto{\pgfqpoint{5.335027in}{0.813290in}}%
\pgfpathlineto{\pgfqpoint{5.336371in}{0.765587in}}%
\pgfpathlineto{\pgfqpoint{5.337043in}{0.765587in}}%
\pgfpathlineto{\pgfqpoint{5.337043in}{0.781488in}}%
\pgfpathlineto{\pgfqpoint{5.338388in}{0.762937in}}%
\pgfpathlineto{\pgfqpoint{5.339060in}{0.762937in}}%
\pgfpathlineto{\pgfqpoint{5.339060in}{0.760287in}}%
\pgfpathlineto{\pgfqpoint{5.340404in}{0.781488in}}%
\pgfpathlineto{\pgfqpoint{5.341076in}{0.781488in}}%
\pgfpathlineto{\pgfqpoint{5.341076in}{0.723184in}}%
\pgfpathlineto{\pgfqpoint{5.342421in}{0.754986in}}%
\pgfpathlineto{\pgfqpoint{5.343093in}{0.754986in}}%
\pgfpathlineto{\pgfqpoint{5.343765in}{0.762937in}}%
\pgfpathlineto{\pgfqpoint{5.344437in}{0.762937in}}%
\pgfpathlineto{\pgfqpoint{5.345109in}{0.762937in}}%
\pgfpathlineto{\pgfqpoint{5.345109in}{0.800039in}}%
\pgfpathlineto{\pgfqpoint{5.345781in}{0.739085in}}%
\pgfpathlineto{\pgfqpoint{5.346454in}{0.754986in}}%
\pgfpathlineto{\pgfqpoint{5.347126in}{0.754986in}}%
\pgfpathlineto{\pgfqpoint{5.347798in}{0.784138in}}%
\pgfpathlineto{\pgfqpoint{5.348470in}{0.760287in}}%
\pgfpathlineto{\pgfqpoint{5.349142in}{0.760287in}}%
\pgfpathlineto{\pgfqpoint{5.350487in}{0.723184in}}%
\pgfpathlineto{\pgfqpoint{5.351159in}{0.723184in}}%
\pgfpathlineto{\pgfqpoint{5.351159in}{0.720534in}}%
\pgfpathlineto{\pgfqpoint{5.351831in}{0.747036in}}%
\pgfpathlineto{\pgfqpoint{5.352503in}{0.741735in}}%
\pgfpathlineto{\pgfqpoint{5.353175in}{0.741735in}}%
\pgfpathlineto{\pgfqpoint{5.353175in}{0.728484in}}%
\pgfpathlineto{\pgfqpoint{5.353847in}{0.765587in}}%
\pgfpathlineto{\pgfqpoint{5.354520in}{0.736435in}}%
\pgfpathlineto{\pgfqpoint{5.355192in}{0.736435in}}%
\pgfpathlineto{\pgfqpoint{5.355192in}{0.725834in}}%
\pgfpathlineto{\pgfqpoint{5.355864in}{0.739085in}}%
\pgfpathlineto{\pgfqpoint{5.356536in}{0.731135in}}%
\pgfpathlineto{\pgfqpoint{5.357208in}{0.731135in}}%
\pgfpathlineto{\pgfqpoint{5.357208in}{0.762937in}}%
\pgfpathlineto{\pgfqpoint{5.358553in}{0.717884in}}%
\pgfpathlineto{\pgfqpoint{5.359225in}{0.717884in}}%
\pgfpathlineto{\pgfqpoint{5.359225in}{0.760287in}}%
\pgfpathlineto{\pgfqpoint{5.360569in}{0.733785in}}%
\pgfpathlineto{\pgfqpoint{5.361241in}{0.733785in}}%
\pgfpathlineto{\pgfqpoint{5.361241in}{0.720534in}}%
\pgfpathlineto{\pgfqpoint{5.361913in}{0.747036in}}%
\pgfpathlineto{\pgfqpoint{5.362586in}{0.736435in}}%
\pgfpathlineto{\pgfqpoint{5.363930in}{0.736435in}}%
\pgfpathlineto{\pgfqpoint{5.363930in}{0.744386in}}%
\pgfpathlineto{\pgfqpoint{5.365274in}{0.715234in}}%
\pgfpathlineto{\pgfqpoint{5.365946in}{0.715234in}}%
\pgfpathlineto{\pgfqpoint{5.367291in}{0.749686in}}%
\pgfpathlineto{\pgfqpoint{5.367963in}{0.749686in}}%
\pgfpathlineto{\pgfqpoint{5.369307in}{0.691382in}}%
\pgfpathlineto{\pgfqpoint{5.369979in}{0.691382in}}%
\pgfpathlineto{\pgfqpoint{5.370652in}{0.709933in}}%
\pgfpathlineto{\pgfqpoint{5.371324in}{0.704633in}}%
\pgfpathlineto{\pgfqpoint{5.371996in}{0.704633in}}%
\pgfpathlineto{\pgfqpoint{5.371996in}{0.699333in}}%
\pgfpathlineto{\pgfqpoint{5.372668in}{0.725834in}}%
\pgfpathlineto{\pgfqpoint{5.373340in}{0.717884in}}%
\pgfpathlineto{\pgfqpoint{5.374012in}{0.717884in}}%
\pgfpathlineto{\pgfqpoint{5.374685in}{0.725834in}}%
\pgfpathlineto{\pgfqpoint{5.375357in}{0.704633in}}%
\pgfpathlineto{\pgfqpoint{5.376029in}{0.704633in}}%
\pgfpathlineto{\pgfqpoint{5.376701in}{0.733785in}}%
\pgfpathlineto{\pgfqpoint{5.377373in}{0.712583in}}%
\pgfpathlineto{\pgfqpoint{5.378045in}{0.712583in}}%
\pgfpathlineto{\pgfqpoint{5.378045in}{0.731135in}}%
\pgfpathlineto{\pgfqpoint{5.379390in}{0.712583in}}%
\pgfpathlineto{\pgfqpoint{5.380062in}{0.712583in}}%
\pgfpathlineto{\pgfqpoint{5.380734in}{0.723184in}}%
\pgfpathlineto{\pgfqpoint{5.381406in}{0.688732in}}%
\pgfpathlineto{\pgfqpoint{5.382078in}{0.688732in}}%
\pgfpathlineto{\pgfqpoint{5.382751in}{0.747036in}}%
\pgfpathlineto{\pgfqpoint{5.383423in}{0.712583in}}%
\pgfpathlineto{\pgfqpoint{5.384095in}{0.712583in}}%
\pgfpathlineto{\pgfqpoint{5.384095in}{0.715234in}}%
\pgfpathlineto{\pgfqpoint{5.385439in}{0.683431in}}%
\pgfpathlineto{\pgfqpoint{5.386784in}{0.683431in}}%
\pgfpathlineto{\pgfqpoint{5.386784in}{0.712583in}}%
\pgfpathlineto{\pgfqpoint{5.388128in}{0.691382in}}%
\pgfpathlineto{\pgfqpoint{5.388800in}{0.691382in}}%
\pgfpathlineto{\pgfqpoint{5.388800in}{0.675481in}}%
\pgfpathlineto{\pgfqpoint{5.390144in}{0.701983in}}%
\pgfpathlineto{\pgfqpoint{5.390817in}{0.701983in}}%
\pgfpathlineto{\pgfqpoint{5.392161in}{0.725834in}}%
\pgfpathlineto{\pgfqpoint{5.392833in}{0.725834in}}%
\pgfpathlineto{\pgfqpoint{5.392833in}{0.696682in}}%
\pgfpathlineto{\pgfqpoint{5.394178in}{0.723184in}}%
\pgfpathlineto{\pgfqpoint{5.394850in}{0.723184in}}%
\pgfpathlineto{\pgfqpoint{5.394850in}{0.672831in}}%
\pgfpathlineto{\pgfqpoint{5.396194in}{0.720534in}}%
\pgfpathlineto{\pgfqpoint{5.396866in}{0.720534in}}%
\pgfpathlineto{\pgfqpoint{5.396866in}{0.728484in}}%
\pgfpathlineto{\pgfqpoint{5.398211in}{0.683431in}}%
\pgfpathlineto{\pgfqpoint{5.398883in}{0.683431in}}%
\pgfpathlineto{\pgfqpoint{5.399555in}{0.712583in}}%
\pgfpathlineto{\pgfqpoint{5.400227in}{0.699333in}}%
\pgfpathlineto{\pgfqpoint{5.400899in}{0.699333in}}%
\pgfpathlineto{\pgfqpoint{5.400899in}{0.664880in}}%
\pgfpathlineto{\pgfqpoint{5.401571in}{0.701983in}}%
\pgfpathlineto{\pgfqpoint{5.402244in}{0.694032in}}%
\pgfpathlineto{\pgfqpoint{5.403588in}{0.694032in}}%
\pgfpathlineto{\pgfqpoint{5.404260in}{0.712583in}}%
\pgfpathlineto{\pgfqpoint{5.404932in}{0.683431in}}%
\pgfpathlineto{\pgfqpoint{5.405604in}{0.683431in}}%
\pgfpathlineto{\pgfqpoint{5.406949in}{0.704633in}}%
\pgfpathlineto{\pgfqpoint{5.407621in}{0.704633in}}%
\pgfpathlineto{\pgfqpoint{5.408293in}{0.678131in}}%
\pgfpathlineto{\pgfqpoint{5.408965in}{0.688732in}}%
\pgfpathlineto{\pgfqpoint{5.409637in}{0.688732in}}%
\pgfpathlineto{\pgfqpoint{5.410982in}{0.667530in}}%
\pgfpathlineto{\pgfqpoint{5.411654in}{0.667530in}}%
\pgfpathlineto{\pgfqpoint{5.412326in}{0.701983in}}%
\pgfpathlineto{\pgfqpoint{5.412998in}{0.667530in}}%
\pgfpathlineto{\pgfqpoint{5.413670in}{0.667530in}}%
\pgfpathlineto{\pgfqpoint{5.415015in}{0.688732in}}%
\pgfpathlineto{\pgfqpoint{5.416359in}{0.688732in}}%
\pgfpathlineto{\pgfqpoint{5.416359in}{0.662230in}}%
\pgfpathlineto{\pgfqpoint{5.417703in}{0.696682in}}%
\pgfpathlineto{\pgfqpoint{5.418376in}{0.696682in}}%
\pgfpathlineto{\pgfqpoint{5.419720in}{0.667530in}}%
\pgfpathlineto{\pgfqpoint{5.420392in}{0.667530in}}%
\pgfpathlineto{\pgfqpoint{5.421736in}{0.701983in}}%
\pgfpathlineto{\pgfqpoint{5.422409in}{0.701983in}}%
\pgfpathlineto{\pgfqpoint{5.422409in}{0.656930in}}%
\pgfpathlineto{\pgfqpoint{5.423753in}{0.688732in}}%
\pgfpathlineto{\pgfqpoint{5.424425in}{0.688732in}}%
\pgfpathlineto{\pgfqpoint{5.425097in}{0.662230in}}%
\pgfpathlineto{\pgfqpoint{5.425769in}{0.683431in}}%
\pgfpathlineto{\pgfqpoint{5.427114in}{0.683431in}}%
\pgfpathlineto{\pgfqpoint{5.427114in}{0.678131in}}%
\pgfpathlineto{\pgfqpoint{5.427786in}{0.709933in}}%
\pgfpathlineto{\pgfqpoint{5.428458in}{0.686082in}}%
\pgfpathlineto{\pgfqpoint{5.429130in}{0.686082in}}%
\pgfpathlineto{\pgfqpoint{5.429130in}{0.691382in}}%
\pgfpathlineto{\pgfqpoint{5.430475in}{0.664880in}}%
\pgfpathlineto{\pgfqpoint{5.431147in}{0.664880in}}%
\pgfpathlineto{\pgfqpoint{5.431819in}{0.683431in}}%
\pgfpathlineto{\pgfqpoint{5.432491in}{0.678131in}}%
\pgfpathlineto{\pgfqpoint{5.433163in}{0.678131in}}%
\pgfpathlineto{\pgfqpoint{5.433163in}{0.686082in}}%
\pgfpathlineto{\pgfqpoint{5.433835in}{0.664880in}}%
\pgfpathlineto{\pgfqpoint{5.434508in}{0.683431in}}%
\pgfpathlineto{\pgfqpoint{5.435180in}{0.683431in}}%
\pgfpathlineto{\pgfqpoint{5.435180in}{0.656930in}}%
\pgfpathlineto{\pgfqpoint{5.436524in}{0.667530in}}%
\pgfpathlineto{\pgfqpoint{5.437196in}{0.667530in}}%
\pgfpathlineto{\pgfqpoint{5.437196in}{0.678131in}}%
\pgfpathlineto{\pgfqpoint{5.438541in}{0.664880in}}%
\pgfpathlineto{\pgfqpoint{5.439213in}{0.664880in}}%
\pgfpathlineto{\pgfqpoint{5.439213in}{0.672831in}}%
\pgfpathlineto{\pgfqpoint{5.440557in}{0.664880in}}%
\pgfpathlineto{\pgfqpoint{5.442574in}{0.664880in}}%
\pgfpathlineto{\pgfqpoint{5.442574in}{0.672831in}}%
\pgfpathlineto{\pgfqpoint{5.443246in}{0.648979in}}%
\pgfpathlineto{\pgfqpoint{5.443918in}{0.667530in}}%
\pgfpathlineto{\pgfqpoint{5.444590in}{0.667530in}}%
\pgfpathlineto{\pgfqpoint{5.444590in}{0.656930in}}%
\pgfpathlineto{\pgfqpoint{5.445934in}{0.659580in}}%
\pgfpathlineto{\pgfqpoint{5.446607in}{0.659580in}}%
\pgfpathlineto{\pgfqpoint{5.446607in}{0.670181in}}%
\pgfpathlineto{\pgfqpoint{5.447951in}{0.662230in}}%
\pgfpathlineto{\pgfqpoint{5.448623in}{0.662230in}}%
\pgfpathlineto{\pgfqpoint{5.448623in}{0.678131in}}%
\pgfpathlineto{\pgfqpoint{5.449967in}{0.664880in}}%
\pgfpathlineto{\pgfqpoint{5.450640in}{0.664880in}}%
\pgfpathlineto{\pgfqpoint{5.450640in}{0.672831in}}%
\pgfpathlineto{\pgfqpoint{5.451984in}{0.651629in}}%
\pgfpathlineto{\pgfqpoint{5.452656in}{0.651629in}}%
\pgfpathlineto{\pgfqpoint{5.453328in}{0.667530in}}%
\pgfpathlineto{\pgfqpoint{5.454000in}{0.664880in}}%
\pgfpathlineto{\pgfqpoint{5.454673in}{0.664880in}}%
\pgfpathlineto{\pgfqpoint{5.454673in}{0.654280in}}%
\pgfpathlineto{\pgfqpoint{5.456017in}{0.654280in}}%
\pgfpathlineto{\pgfqpoint{5.456689in}{0.654280in}}%
\pgfpathlineto{\pgfqpoint{5.456689in}{0.646329in}}%
\pgfpathlineto{\pgfqpoint{5.457361in}{0.656930in}}%
\pgfpathlineto{\pgfqpoint{5.458033in}{0.656930in}}%
\pgfpathlineto{\pgfqpoint{5.458706in}{0.656930in}}%
\pgfpathlineto{\pgfqpoint{5.458706in}{0.648979in}}%
\pgfpathlineto{\pgfqpoint{5.460050in}{0.667530in}}%
\pgfpathlineto{\pgfqpoint{5.460722in}{0.667530in}}%
\pgfpathlineto{\pgfqpoint{5.460722in}{0.659580in}}%
\pgfpathlineto{\pgfqpoint{5.462066in}{0.664880in}}%
\pgfpathlineto{\pgfqpoint{5.462739in}{0.664880in}}%
\pgfpathlineto{\pgfqpoint{5.462739in}{0.651629in}}%
\pgfpathlineto{\pgfqpoint{5.463411in}{0.672831in}}%
\pgfpathlineto{\pgfqpoint{5.464083in}{0.659580in}}%
\pgfpathlineto{\pgfqpoint{5.464755in}{0.659580in}}%
\pgfpathlineto{\pgfqpoint{5.464755in}{0.654280in}}%
\pgfpathlineto{\pgfqpoint{5.465427in}{0.664880in}}%
\pgfpathlineto{\pgfqpoint{5.466099in}{0.664880in}}%
\pgfpathlineto{\pgfqpoint{5.466772in}{0.664880in}}%
\pgfpathlineto{\pgfqpoint{5.466772in}{0.654280in}}%
\pgfpathlineto{\pgfqpoint{5.467444in}{0.670181in}}%
\pgfpathlineto{\pgfqpoint{5.468116in}{0.656930in}}%
\pgfpathlineto{\pgfqpoint{5.468788in}{0.656930in}}%
\pgfpathlineto{\pgfqpoint{5.468788in}{0.651629in}}%
\pgfpathlineto{\pgfqpoint{5.469460in}{0.667530in}}%
\pgfpathlineto{\pgfqpoint{5.470132in}{0.662230in}}%
\pgfpathlineto{\pgfqpoint{5.470805in}{0.662230in}}%
\pgfpathlineto{\pgfqpoint{5.471477in}{0.643679in}}%
\pgfpathlineto{\pgfqpoint{5.472149in}{0.643679in}}%
\pgfpathlineto{\pgfqpoint{5.472821in}{0.643679in}}%
\pgfpathlineto{\pgfqpoint{5.473493in}{0.656930in}}%
\pgfpathlineto{\pgfqpoint{5.474165in}{0.646329in}}%
\pgfpathlineto{\pgfqpoint{5.474838in}{0.646329in}}%
\pgfpathlineto{\pgfqpoint{5.476182in}{0.656930in}}%
\pgfpathlineto{\pgfqpoint{5.476854in}{0.656930in}}%
\pgfpathlineto{\pgfqpoint{5.476854in}{0.651629in}}%
\pgfpathlineto{\pgfqpoint{5.478198in}{0.659580in}}%
\pgfpathlineto{\pgfqpoint{5.478871in}{0.659580in}}%
\pgfpathlineto{\pgfqpoint{5.480215in}{0.646329in}}%
\pgfpathlineto{\pgfqpoint{5.480887in}{0.646329in}}%
\pgfpathlineto{\pgfqpoint{5.480887in}{0.648979in}}%
\pgfpathlineto{\pgfqpoint{5.482231in}{0.638378in}}%
\pgfpathlineto{\pgfqpoint{5.482904in}{0.638378in}}%
\pgfpathlineto{\pgfqpoint{5.482904in}{0.651629in}}%
\pgfpathlineto{\pgfqpoint{5.484248in}{0.651629in}}%
\pgfpathlineto{\pgfqpoint{5.484920in}{0.651629in}}%
\pgfpathlineto{\pgfqpoint{5.486264in}{0.635728in}}%
\pgfpathlineto{\pgfqpoint{5.486937in}{0.635728in}}%
\pgfpathlineto{\pgfqpoint{5.487609in}{0.659580in}}%
\pgfpathlineto{\pgfqpoint{5.488281in}{0.643679in}}%
\pgfpathlineto{\pgfqpoint{5.489625in}{0.643679in}}%
\pgfpathlineto{\pgfqpoint{5.489625in}{0.641029in}}%
\pgfpathlineto{\pgfqpoint{5.490297in}{0.656930in}}%
\pgfpathlineto{\pgfqpoint{5.490970in}{0.654280in}}%
\pgfpathlineto{\pgfqpoint{5.491642in}{0.654280in}}%
\pgfpathlineto{\pgfqpoint{5.491642in}{0.638378in}}%
\pgfpathlineto{\pgfqpoint{5.492986in}{0.643679in}}%
\pgfpathlineto{\pgfqpoint{5.495003in}{0.643679in}}%
\pgfpathlineto{\pgfqpoint{5.496347in}{0.656930in}}%
\pgfpathlineto{\pgfqpoint{5.497019in}{0.656930in}}%
\pgfpathlineto{\pgfqpoint{5.497019in}{0.641029in}}%
\pgfpathlineto{\pgfqpoint{5.498363in}{0.643679in}}%
\pgfpathlineto{\pgfqpoint{5.499036in}{0.643679in}}%
\pgfpathlineto{\pgfqpoint{5.499036in}{0.635728in}}%
\pgfpathlineto{\pgfqpoint{5.500380in}{0.670181in}}%
\pgfpathlineto{\pgfqpoint{5.501052in}{0.670181in}}%
\pgfpathlineto{\pgfqpoint{5.501724in}{0.627778in}}%
\pgfpathlineto{\pgfqpoint{5.502396in}{0.648979in}}%
\pgfpathlineto{\pgfqpoint{5.503069in}{0.648979in}}%
\pgfpathlineto{\pgfqpoint{5.503069in}{0.651629in}}%
\pgfpathlineto{\pgfqpoint{5.503741in}{0.641029in}}%
\pgfpathlineto{\pgfqpoint{5.504413in}{0.643679in}}%
\pgfpathlineto{\pgfqpoint{5.505085in}{0.643679in}}%
\pgfpathlineto{\pgfqpoint{5.505085in}{0.635728in}}%
\pgfpathlineto{\pgfqpoint{5.506429in}{0.656930in}}%
\pgfpathlineto{\pgfqpoint{5.507102in}{0.656930in}}%
\pgfpathlineto{\pgfqpoint{5.507774in}{0.638378in}}%
\pgfpathlineto{\pgfqpoint{5.508446in}{0.641029in}}%
\pgfpathlineto{\pgfqpoint{5.509118in}{0.641029in}}%
\pgfpathlineto{\pgfqpoint{5.509118in}{0.638378in}}%
\pgfpathlineto{\pgfqpoint{5.510463in}{0.662230in}}%
\pgfpathlineto{\pgfqpoint{5.511135in}{0.662230in}}%
\pgfpathlineto{\pgfqpoint{5.511135in}{0.635728in}}%
\pgfpathlineto{\pgfqpoint{5.512479in}{0.635728in}}%
\pgfpathlineto{\pgfqpoint{5.513151in}{0.635728in}}%
\pgfpathlineto{\pgfqpoint{5.513151in}{0.656930in}}%
\pgfpathlineto{\pgfqpoint{5.514496in}{0.643679in}}%
\pgfpathlineto{\pgfqpoint{5.515168in}{0.643679in}}%
\pgfpathlineto{\pgfqpoint{5.515168in}{0.641029in}}%
\pgfpathlineto{\pgfqpoint{5.516512in}{0.643679in}}%
\pgfpathlineto{\pgfqpoint{5.517184in}{0.643679in}}%
\pgfpathlineto{\pgfqpoint{5.517184in}{0.641029in}}%
\pgfpathlineto{\pgfqpoint{5.517856in}{0.648979in}}%
\pgfpathlineto{\pgfqpoint{5.518529in}{0.641029in}}%
\pgfpathlineto{\pgfqpoint{5.519201in}{0.641029in}}%
\pgfpathlineto{\pgfqpoint{5.519201in}{0.633078in}}%
\pgfpathlineto{\pgfqpoint{5.520545in}{0.646329in}}%
\pgfpathlineto{\pgfqpoint{5.521217in}{0.646329in}}%
\pgfpathlineto{\pgfqpoint{5.521217in}{0.633078in}}%
\pgfpathlineto{\pgfqpoint{5.522562in}{0.646329in}}%
\pgfpathlineto{\pgfqpoint{5.523234in}{0.646329in}}%
\pgfpathlineto{\pgfqpoint{5.523234in}{0.630428in}}%
\pgfpathlineto{\pgfqpoint{5.524578in}{0.648979in}}%
\pgfpathlineto{\pgfqpoint{5.525250in}{0.648979in}}%
\pgfpathlineto{\pgfqpoint{5.525250in}{0.641029in}}%
\pgfpathlineto{\pgfqpoint{5.526595in}{0.641029in}}%
\pgfpathlineto{\pgfqpoint{5.527267in}{0.641029in}}%
\pgfpathlineto{\pgfqpoint{5.527267in}{0.633078in}}%
\pgfpathlineto{\pgfqpoint{5.527939in}{0.643679in}}%
\pgfpathlineto{\pgfqpoint{5.528611in}{0.643679in}}%
\pgfpathlineto{\pgfqpoint{5.529283in}{0.643679in}}%
\pgfpathlineto{\pgfqpoint{5.529955in}{0.638378in}}%
\pgfpathlineto{\pgfqpoint{5.530628in}{0.638378in}}%
\pgfpathlineto{\pgfqpoint{5.531300in}{0.638378in}}%
\pgfpathlineto{\pgfqpoint{5.531972in}{0.651629in}}%
\pgfpathlineto{\pgfqpoint{5.532644in}{0.643679in}}%
\pgfpathlineto{\pgfqpoint{5.534661in}{0.643679in}}%
\pgfpathlineto{\pgfqpoint{5.534661in}{0.633078in}}%
\pgfpathlineto{\pgfqpoint{5.536005in}{0.641029in}}%
\pgfpathlineto{\pgfqpoint{5.536677in}{0.641029in}}%
\pgfpathlineto{\pgfqpoint{5.537349in}{0.627778in}}%
\pgfpathlineto{\pgfqpoint{5.538021in}{0.633078in}}%
\pgfpathlineto{\pgfqpoint{5.538694in}{0.633078in}}%
\pgfpathlineto{\pgfqpoint{5.540038in}{0.643679in}}%
\pgfpathlineto{\pgfqpoint{5.540710in}{0.643679in}}%
\pgfpathlineto{\pgfqpoint{5.540710in}{0.633078in}}%
\pgfpathlineto{\pgfqpoint{5.542054in}{0.643679in}}%
\pgfpathlineto{\pgfqpoint{5.542727in}{0.643679in}}%
\pgfpathlineto{\pgfqpoint{5.543399in}{0.633078in}}%
\pgfpathlineto{\pgfqpoint{5.544071in}{0.638378in}}%
\pgfpathlineto{\pgfqpoint{5.544743in}{0.638378in}}%
\pgfpathlineto{\pgfqpoint{5.544743in}{0.633078in}}%
\pgfpathlineto{\pgfqpoint{5.545415in}{0.646329in}}%
\pgfpathlineto{\pgfqpoint{5.546087in}{0.638378in}}%
\pgfpathlineto{\pgfqpoint{5.546760in}{0.638378in}}%
\pgfpathlineto{\pgfqpoint{5.547432in}{0.633078in}}%
\pgfpathlineto{\pgfqpoint{5.548104in}{0.638378in}}%
\pgfpathlineto{\pgfqpoint{5.548776in}{0.638378in}}%
\pgfpathlineto{\pgfqpoint{5.549448in}{0.633078in}}%
\pgfpathlineto{\pgfqpoint{5.550120in}{0.648979in}}%
\pgfpathlineto{\pgfqpoint{5.550793in}{0.648979in}}%
\pgfpathlineto{\pgfqpoint{5.551465in}{0.630428in}}%
\pgfpathlineto{\pgfqpoint{5.552137in}{0.633078in}}%
\pgfpathlineto{\pgfqpoint{5.552809in}{0.633078in}}%
\pgfpathlineto{\pgfqpoint{5.553481in}{0.630428in}}%
\pgfpathlineto{\pgfqpoint{5.554153in}{0.646329in}}%
\pgfpathlineto{\pgfqpoint{5.554826in}{0.646329in}}%
\pgfpathlineto{\pgfqpoint{5.554826in}{0.633078in}}%
\pgfpathlineto{\pgfqpoint{5.556170in}{0.638378in}}%
\pgfpathlineto{\pgfqpoint{5.556842in}{0.638378in}}%
\pgfpathlineto{\pgfqpoint{5.556842in}{0.635728in}}%
\pgfpathlineto{\pgfqpoint{5.558186in}{0.635728in}}%
\pgfpathlineto{\pgfqpoint{5.558859in}{0.635728in}}%
\pgfpathlineto{\pgfqpoint{5.560203in}{0.641029in}}%
\pgfpathlineto{\pgfqpoint{5.560875in}{0.641029in}}%
\pgfpathlineto{\pgfqpoint{5.560875in}{0.633078in}}%
\pgfpathlineto{\pgfqpoint{5.562219in}{0.633078in}}%
\pgfpathlineto{\pgfqpoint{5.562892in}{0.633078in}}%
\pgfpathlineto{\pgfqpoint{5.562892in}{0.630428in}}%
\pgfpathlineto{\pgfqpoint{5.563564in}{0.643679in}}%
\pgfpathlineto{\pgfqpoint{5.564236in}{0.643679in}}%
\pgfpathlineto{\pgfqpoint{5.564908in}{0.643679in}}%
\pgfpathlineto{\pgfqpoint{5.565580in}{0.633078in}}%
\pgfpathlineto{\pgfqpoint{5.566252in}{0.633078in}}%
\pgfpathlineto{\pgfqpoint{5.566925in}{0.633078in}}%
\pgfpathlineto{\pgfqpoint{5.566925in}{0.638378in}}%
\pgfpathlineto{\pgfqpoint{5.568269in}{0.630428in}}%
\pgfpathlineto{\pgfqpoint{5.568941in}{0.630428in}}%
\pgfpathlineto{\pgfqpoint{5.570285in}{0.638378in}}%
\pgfpathlineto{\pgfqpoint{5.571630in}{0.638378in}}%
\pgfpathlineto{\pgfqpoint{5.571630in}{0.633078in}}%
\pgfpathlineto{\pgfqpoint{5.572974in}{0.635728in}}%
\pgfpathlineto{\pgfqpoint{5.573646in}{0.635728in}}%
\pgfpathlineto{\pgfqpoint{5.573646in}{0.630428in}}%
\pgfpathlineto{\pgfqpoint{5.574991in}{0.643679in}}%
\pgfpathlineto{\pgfqpoint{5.575663in}{0.643679in}}%
\pgfpathlineto{\pgfqpoint{5.575663in}{0.635728in}}%
\pgfpathlineto{\pgfqpoint{5.577007in}{0.635728in}}%
\pgfpathlineto{\pgfqpoint{5.578351in}{0.635728in}}%
\pgfpathlineto{\pgfqpoint{5.578351in}{0.630428in}}%
\pgfpathlineto{\pgfqpoint{5.579024in}{0.643679in}}%
\pgfpathlineto{\pgfqpoint{5.579696in}{0.638378in}}%
\pgfpathlineto{\pgfqpoint{5.580368in}{0.638378in}}%
\pgfpathlineto{\pgfqpoint{5.580368in}{0.635728in}}%
\pgfpathlineto{\pgfqpoint{5.581712in}{0.638378in}}%
\pgfpathlineto{\pgfqpoint{5.582384in}{0.638378in}}%
\pgfpathlineto{\pgfqpoint{5.582384in}{0.643679in}}%
\pgfpathlineto{\pgfqpoint{5.583729in}{0.633078in}}%
\pgfpathlineto{\pgfqpoint{5.584401in}{0.633078in}}%
\pgfpathlineto{\pgfqpoint{5.584401in}{0.630428in}}%
\pgfpathlineto{\pgfqpoint{5.585745in}{0.630428in}}%
\pgfpathlineto{\pgfqpoint{5.586417in}{0.630428in}}%
\pgfpathlineto{\pgfqpoint{5.586417in}{0.641029in}}%
\pgfpathlineto{\pgfqpoint{5.587762in}{0.633078in}}%
\pgfpathlineto{\pgfqpoint{5.588434in}{0.633078in}}%
\pgfpathlineto{\pgfqpoint{5.588434in}{0.635728in}}%
\pgfpathlineto{\pgfqpoint{5.589106in}{0.627778in}}%
\pgfpathlineto{\pgfqpoint{5.589778in}{0.630428in}}%
\pgfpathlineto{\pgfqpoint{5.591123in}{0.630428in}}%
\pgfpathlineto{\pgfqpoint{5.591123in}{0.627778in}}%
\pgfpathlineto{\pgfqpoint{5.592467in}{0.643679in}}%
\pgfpathlineto{\pgfqpoint{5.593139in}{0.643679in}}%
\pgfpathlineto{\pgfqpoint{5.594483in}{0.630428in}}%
\pgfpathlineto{\pgfqpoint{5.595156in}{0.630428in}}%
\pgfpathlineto{\pgfqpoint{5.595156in}{0.638378in}}%
\pgfpathlineto{\pgfqpoint{5.596500in}{0.633078in}}%
\pgfpathlineto{\pgfqpoint{5.597172in}{0.633078in}}%
\pgfpathlineto{\pgfqpoint{5.597172in}{0.627778in}}%
\pgfpathlineto{\pgfqpoint{5.598516in}{0.646329in}}%
\pgfpathlineto{\pgfqpoint{5.599189in}{0.646329in}}%
\pgfpathlineto{\pgfqpoint{5.600533in}{0.638378in}}%
\pgfpathlineto{\pgfqpoint{5.601205in}{0.638378in}}%
\pgfpathlineto{\pgfqpoint{5.601205in}{0.633078in}}%
\pgfpathlineto{\pgfqpoint{5.602549in}{0.641029in}}%
\pgfpathlineto{\pgfqpoint{5.603222in}{0.641029in}}%
\pgfpathlineto{\pgfqpoint{5.603222in}{0.630428in}}%
\pgfpathlineto{\pgfqpoint{5.604566in}{0.638378in}}%
\pgfpathlineto{\pgfqpoint{5.605910in}{0.638378in}}%
\pgfpathlineto{\pgfqpoint{5.605910in}{0.633078in}}%
\pgfpathlineto{\pgfqpoint{5.607255in}{0.638378in}}%
\pgfpathlineto{\pgfqpoint{5.607927in}{0.638378in}}%
\pgfpathlineto{\pgfqpoint{5.607927in}{0.646329in}}%
\pgfpathlineto{\pgfqpoint{5.608599in}{0.630428in}}%
\pgfpathlineto{\pgfqpoint{5.609271in}{0.638378in}}%
\pgfpathlineto{\pgfqpoint{5.609943in}{0.638378in}}%
\pgfpathlineto{\pgfqpoint{5.609943in}{0.633078in}}%
\pgfpathlineto{\pgfqpoint{5.611288in}{0.635728in}}%
\pgfpathlineto{\pgfqpoint{5.611960in}{0.635728in}}%
\pgfpathlineto{\pgfqpoint{5.612632in}{0.630428in}}%
\pgfpathlineto{\pgfqpoint{5.613304in}{0.635728in}}%
\pgfpathlineto{\pgfqpoint{5.614648in}{0.635728in}}%
\pgfpathlineto{\pgfqpoint{5.614648in}{0.633078in}}%
\pgfpathlineto{\pgfqpoint{5.615993in}{0.633078in}}%
\pgfpathlineto{\pgfqpoint{5.616665in}{0.633078in}}%
\pgfpathlineto{\pgfqpoint{5.617337in}{0.641029in}}%
\pgfpathlineto{\pgfqpoint{5.618009in}{0.635728in}}%
\pgfpathlineto{\pgfqpoint{5.619354in}{0.635728in}}%
\pgfpathlineto{\pgfqpoint{5.619354in}{0.627778in}}%
\pgfpathlineto{\pgfqpoint{5.620026in}{0.641029in}}%
\pgfpathlineto{\pgfqpoint{5.620698in}{0.630428in}}%
\pgfpathlineto{\pgfqpoint{5.623387in}{0.630428in}}%
\pgfpathlineto{\pgfqpoint{5.623387in}{0.635728in}}%
\pgfpathlineto{\pgfqpoint{5.624731in}{0.627778in}}%
\pgfpathlineto{\pgfqpoint{5.625403in}{0.627778in}}%
\pgfpathlineto{\pgfqpoint{5.625403in}{0.638378in}}%
\pgfpathlineto{\pgfqpoint{5.626748in}{0.638378in}}%
\pgfpathlineto{\pgfqpoint{5.627420in}{0.638378in}}%
\pgfpathlineto{\pgfqpoint{5.627420in}{0.630428in}}%
\pgfpathlineto{\pgfqpoint{5.628092in}{0.641029in}}%
\pgfpathlineto{\pgfqpoint{5.628764in}{0.635728in}}%
\pgfpathlineto{\pgfqpoint{5.629436in}{0.635728in}}%
\pgfpathlineto{\pgfqpoint{5.629436in}{0.638378in}}%
\pgfpathlineto{\pgfqpoint{5.630108in}{0.630428in}}%
\pgfpathlineto{\pgfqpoint{5.630781in}{0.633078in}}%
\pgfpathlineto{\pgfqpoint{5.631453in}{0.633078in}}%
\pgfpathlineto{\pgfqpoint{5.631453in}{0.630428in}}%
\pgfpathlineto{\pgfqpoint{5.632797in}{0.643679in}}%
\pgfpathlineto{\pgfqpoint{5.633469in}{0.643679in}}%
\pgfpathlineto{\pgfqpoint{5.634141in}{0.630428in}}%
\pgfpathlineto{\pgfqpoint{5.634814in}{0.630428in}}%
\pgfpathlineto{\pgfqpoint{5.635486in}{0.630428in}}%
\pgfpathlineto{\pgfqpoint{5.636830in}{0.638378in}}%
\pgfpathlineto{\pgfqpoint{5.637502in}{0.638378in}}%
\pgfpathlineto{\pgfqpoint{5.637502in}{0.635728in}}%
\pgfpathlineto{\pgfqpoint{5.638174in}{0.641029in}}%
\pgfpathlineto{\pgfqpoint{5.638847in}{0.635728in}}%
\pgfpathlineto{\pgfqpoint{5.639519in}{0.635728in}}%
\pgfpathlineto{\pgfqpoint{5.639519in}{0.638378in}}%
\pgfpathlineto{\pgfqpoint{5.640863in}{0.635728in}}%
\pgfpathlineto{\pgfqpoint{5.641535in}{0.635728in}}%
\pgfpathlineto{\pgfqpoint{5.641535in}{0.641029in}}%
\pgfpathlineto{\pgfqpoint{5.642880in}{0.630428in}}%
\pgfpathlineto{\pgfqpoint{5.643552in}{0.630428in}}%
\pgfpathlineto{\pgfqpoint{5.644224in}{0.641029in}}%
\pgfpathlineto{\pgfqpoint{5.644896in}{0.633078in}}%
\pgfpathlineto{\pgfqpoint{5.645568in}{0.633078in}}%
\pgfpathlineto{\pgfqpoint{5.645568in}{0.638378in}}%
\pgfpathlineto{\pgfqpoint{5.646913in}{0.633078in}}%
\pgfpathlineto{\pgfqpoint{5.647585in}{0.633078in}}%
\pgfpathlineto{\pgfqpoint{5.647585in}{0.638378in}}%
\pgfpathlineto{\pgfqpoint{5.648929in}{0.630428in}}%
\pgfpathlineto{\pgfqpoint{5.650273in}{0.630428in}}%
\pgfpathlineto{\pgfqpoint{5.651618in}{0.635728in}}%
\pgfpathlineto{\pgfqpoint{5.652290in}{0.635728in}}%
\pgfpathlineto{\pgfqpoint{5.653634in}{0.630428in}}%
\pgfpathlineto{\pgfqpoint{5.654979in}{0.630428in}}%
\pgfpathlineto{\pgfqpoint{5.654979in}{0.635728in}}%
\pgfpathlineto{\pgfqpoint{5.656323in}{0.630428in}}%
\pgfpathlineto{\pgfqpoint{5.656995in}{0.630428in}}%
\pgfpathlineto{\pgfqpoint{5.658339in}{0.648979in}}%
\pgfpathlineto{\pgfqpoint{5.659012in}{0.648979in}}%
\pgfpathlineto{\pgfqpoint{5.659684in}{0.633078in}}%
\pgfpathlineto{\pgfqpoint{5.660356in}{0.635728in}}%
\pgfpathlineto{\pgfqpoint{5.661028in}{0.635728in}}%
\pgfpathlineto{\pgfqpoint{5.661028in}{0.627778in}}%
\pgfpathlineto{\pgfqpoint{5.662372in}{0.635728in}}%
\pgfpathlineto{\pgfqpoint{5.663045in}{0.635728in}}%
\pgfpathlineto{\pgfqpoint{5.663045in}{0.630428in}}%
\pgfpathlineto{\pgfqpoint{5.663717in}{0.643679in}}%
\pgfpathlineto{\pgfqpoint{5.664389in}{0.635728in}}%
\pgfpathlineto{\pgfqpoint{5.665061in}{0.635728in}}%
\pgfpathlineto{\pgfqpoint{5.665061in}{0.651629in}}%
\pgfpathlineto{\pgfqpoint{5.665733in}{0.633078in}}%
\pgfpathlineto{\pgfqpoint{5.666405in}{0.633078in}}%
\pgfpathlineto{\pgfqpoint{5.667078in}{0.633078in}}%
\pgfpathlineto{\pgfqpoint{5.667078in}{0.635728in}}%
\pgfpathlineto{\pgfqpoint{5.668422in}{0.635728in}}%
\pgfpathlineto{\pgfqpoint{5.669094in}{0.635728in}}%
\pgfpathlineto{\pgfqpoint{5.669094in}{0.627778in}}%
\pgfpathlineto{\pgfqpoint{5.670438in}{0.627778in}}%
\pgfpathlineto{\pgfqpoint{5.671111in}{0.627778in}}%
\pgfpathlineto{\pgfqpoint{5.671783in}{0.638378in}}%
\pgfpathlineto{\pgfqpoint{5.672455in}{0.633078in}}%
\pgfpathlineto{\pgfqpoint{5.673127in}{0.633078in}}%
\pgfpathlineto{\pgfqpoint{5.674471in}{0.648979in}}%
\pgfpathlineto{\pgfqpoint{5.675144in}{0.648979in}}%
\pgfpathlineto{\pgfqpoint{5.675816in}{0.627778in}}%
\pgfpathlineto{\pgfqpoint{5.676488in}{0.635728in}}%
\pgfpathlineto{\pgfqpoint{5.677160in}{0.635728in}}%
\pgfpathlineto{\pgfqpoint{5.677160in}{0.638378in}}%
\pgfpathlineto{\pgfqpoint{5.678504in}{0.627778in}}%
\pgfpathlineto{\pgfqpoint{5.679177in}{0.627778in}}%
\pgfpathlineto{\pgfqpoint{5.679849in}{0.635728in}}%
\pgfpathlineto{\pgfqpoint{5.680521in}{0.633078in}}%
\pgfpathlineto{\pgfqpoint{5.681193in}{0.633078in}}%
\pgfpathlineto{\pgfqpoint{5.681193in}{0.643679in}}%
\pgfpathlineto{\pgfqpoint{5.682537in}{0.635728in}}%
\pgfpathlineto{\pgfqpoint{5.683210in}{0.635728in}}%
\pgfpathlineto{\pgfqpoint{5.683210in}{0.633078in}}%
\pgfpathlineto{\pgfqpoint{5.684554in}{0.633078in}}%
\pgfpathlineto{\pgfqpoint{5.685226in}{0.633078in}}%
\pgfpathlineto{\pgfqpoint{5.685226in}{0.641029in}}%
\pgfpathlineto{\pgfqpoint{5.686570in}{0.633078in}}%
\pgfpathlineto{\pgfqpoint{5.687243in}{0.633078in}}%
\pgfpathlineto{\pgfqpoint{5.687915in}{0.627778in}}%
\pgfpathlineto{\pgfqpoint{5.688587in}{0.635728in}}%
\pgfpathlineto{\pgfqpoint{5.689259in}{0.635728in}}%
\pgfpathlineto{\pgfqpoint{5.690603in}{0.627778in}}%
\pgfpathlineto{\pgfqpoint{5.691276in}{0.627778in}}%
\pgfpathlineto{\pgfqpoint{5.691948in}{0.633078in}}%
\pgfpathlineto{\pgfqpoint{5.692620in}{0.627778in}}%
\pgfpathlineto{\pgfqpoint{5.693292in}{0.627778in}}%
\pgfpathlineto{\pgfqpoint{5.693292in}{0.638378in}}%
\pgfpathlineto{\pgfqpoint{5.694636in}{0.633078in}}%
\pgfpathlineto{\pgfqpoint{5.695309in}{0.633078in}}%
\pgfpathlineto{\pgfqpoint{5.695309in}{0.635728in}}%
\pgfpathlineto{\pgfqpoint{5.695981in}{0.630428in}}%
\pgfpathlineto{\pgfqpoint{5.696653in}{0.635728in}}%
\pgfpathlineto{\pgfqpoint{5.697325in}{0.635728in}}%
\pgfpathlineto{\pgfqpoint{5.698669in}{0.627778in}}%
\pgfpathlineto{\pgfqpoint{5.699342in}{0.627778in}}%
\pgfpathlineto{\pgfqpoint{5.700014in}{0.638378in}}%
\pgfpathlineto{\pgfqpoint{5.700686in}{0.633078in}}%
\pgfpathlineto{\pgfqpoint{5.702030in}{0.633078in}}%
\pgfpathlineto{\pgfqpoint{5.702030in}{0.627778in}}%
\pgfpathlineto{\pgfqpoint{5.703375in}{0.627778in}}%
\pgfpathlineto{\pgfqpoint{5.704047in}{0.627778in}}%
\pgfpathlineto{\pgfqpoint{5.704719in}{0.635728in}}%
\pgfpathlineto{\pgfqpoint{5.705391in}{0.633078in}}%
\pgfpathlineto{\pgfqpoint{5.706063in}{0.633078in}}%
\pgfpathlineto{\pgfqpoint{5.706063in}{0.627778in}}%
\pgfpathlineto{\pgfqpoint{5.706735in}{0.635728in}}%
\pgfpathlineto{\pgfqpoint{5.707408in}{0.630428in}}%
\pgfpathlineto{\pgfqpoint{5.708080in}{0.630428in}}%
\pgfpathlineto{\pgfqpoint{5.709424in}{0.641029in}}%
\pgfpathlineto{\pgfqpoint{5.710096in}{0.641029in}}%
\pgfpathlineto{\pgfqpoint{5.710096in}{0.630428in}}%
\pgfpathlineto{\pgfqpoint{5.711441in}{0.635728in}}%
\pgfpathlineto{\pgfqpoint{5.712785in}{0.635728in}}%
\pgfpathlineto{\pgfqpoint{5.712785in}{0.627778in}}%
\pgfpathlineto{\pgfqpoint{5.714129in}{0.630428in}}%
\pgfpathlineto{\pgfqpoint{5.714801in}{0.630428in}}%
\pgfpathlineto{\pgfqpoint{5.714801in}{0.627778in}}%
\pgfpathlineto{\pgfqpoint{5.716146in}{0.638378in}}%
\pgfpathlineto{\pgfqpoint{5.716818in}{0.638378in}}%
\pgfpathlineto{\pgfqpoint{5.718162in}{0.633078in}}%
\pgfpathlineto{\pgfqpoint{5.718834in}{0.633078in}}%
\pgfpathlineto{\pgfqpoint{5.718834in}{0.635728in}}%
\pgfpathlineto{\pgfqpoint{5.719507in}{0.627778in}}%
\pgfpathlineto{\pgfqpoint{5.720179in}{0.635728in}}%
\pgfpathlineto{\pgfqpoint{5.720851in}{0.635728in}}%
\pgfpathlineto{\pgfqpoint{5.721523in}{0.630428in}}%
\pgfpathlineto{\pgfqpoint{5.722195in}{0.630428in}}%
\pgfpathlineto{\pgfqpoint{5.724212in}{0.630428in}}%
\pgfpathlineto{\pgfqpoint{5.724212in}{0.633078in}}%
\pgfpathlineto{\pgfqpoint{5.725556in}{0.633078in}}%
\pgfpathlineto{\pgfqpoint{5.726900in}{0.633078in}}%
\pgfpathlineto{\pgfqpoint{5.726900in}{0.635728in}}%
\pgfpathlineto{\pgfqpoint{5.728245in}{0.630428in}}%
\pgfpathlineto{\pgfqpoint{5.728917in}{0.630428in}}%
\pgfpathlineto{\pgfqpoint{5.730261in}{0.635728in}}%
\pgfpathlineto{\pgfqpoint{5.732278in}{0.635728in}}%
\pgfpathlineto{\pgfqpoint{5.732278in}{0.627778in}}%
\pgfpathlineto{\pgfqpoint{5.733622in}{0.630428in}}%
\pgfpathlineto{\pgfqpoint{5.734294in}{0.630428in}}%
\pgfpathlineto{\pgfqpoint{5.734294in}{0.633078in}}%
\pgfpathlineto{\pgfqpoint{5.734967in}{0.627778in}}%
\pgfpathlineto{\pgfqpoint{5.735639in}{0.633078in}}%
\pgfpathlineto{\pgfqpoint{5.736983in}{0.633078in}}%
\pgfpathlineto{\pgfqpoint{5.736983in}{0.630428in}}%
\pgfpathlineto{\pgfqpoint{5.737655in}{0.635728in}}%
\pgfpathlineto{\pgfqpoint{5.738327in}{0.633078in}}%
\pgfpathlineto{\pgfqpoint{5.739000in}{0.633078in}}%
\pgfpathlineto{\pgfqpoint{5.739000in}{0.627778in}}%
\pgfpathlineto{\pgfqpoint{5.740344in}{0.630428in}}%
\pgfpathlineto{\pgfqpoint{5.741688in}{0.630428in}}%
\pgfpathlineto{\pgfqpoint{5.741688in}{0.633078in}}%
\pgfpathlineto{\pgfqpoint{5.742360in}{0.627778in}}%
\pgfpathlineto{\pgfqpoint{5.743033in}{0.627778in}}%
\pgfpathlineto{\pgfqpoint{5.744377in}{0.627778in}}%
\pgfpathlineto{\pgfqpoint{5.745049in}{0.633078in}}%
\pgfpathlineto{\pgfqpoint{5.745721in}{0.630428in}}%
\pgfpathlineto{\pgfqpoint{5.746393in}{0.630428in}}%
\pgfpathlineto{\pgfqpoint{5.746393in}{0.641029in}}%
\pgfpathlineto{\pgfqpoint{5.747738in}{0.627778in}}%
\pgfpathlineto{\pgfqpoint{5.748410in}{0.627778in}}%
\pgfpathlineto{\pgfqpoint{5.749754in}{0.638378in}}%
\pgfpathlineto{\pgfqpoint{5.750426in}{0.638378in}}%
\pgfpathlineto{\pgfqpoint{5.750426in}{0.630428in}}%
\pgfpathlineto{\pgfqpoint{5.751771in}{0.630428in}}%
\pgfpathlineto{\pgfqpoint{5.752443in}{0.630428in}}%
\pgfpathlineto{\pgfqpoint{5.752443in}{0.635728in}}%
\pgfpathlineto{\pgfqpoint{5.753787in}{0.635728in}}%
\pgfpathlineto{\pgfqpoint{5.754459in}{0.635728in}}%
\pgfpathlineto{\pgfqpoint{5.754459in}{0.630428in}}%
\pgfpathlineto{\pgfqpoint{5.755804in}{0.635728in}}%
\pgfpathlineto{\pgfqpoint{5.756476in}{0.635728in}}%
\pgfpathlineto{\pgfqpoint{5.757148in}{0.627778in}}%
\pgfpathlineto{\pgfqpoint{5.757820in}{0.630428in}}%
\pgfpathlineto{\pgfqpoint{5.759165in}{0.630428in}}%
\pgfpathlineto{\pgfqpoint{5.759165in}{0.638378in}}%
\pgfpathlineto{\pgfqpoint{5.760509in}{0.633078in}}%
\pgfpathlineto{\pgfqpoint{5.761181in}{0.633078in}}%
\pgfpathlineto{\pgfqpoint{5.761181in}{0.635728in}}%
\pgfpathlineto{\pgfqpoint{5.761853in}{0.630428in}}%
\pgfpathlineto{\pgfqpoint{5.762525in}{0.633078in}}%
\pgfpathlineto{\pgfqpoint{5.763198in}{0.633078in}}%
\pgfpathlineto{\pgfqpoint{5.763870in}{0.638378in}}%
\pgfpathlineto{\pgfqpoint{5.764542in}{0.633078in}}%
\pgfpathlineto{\pgfqpoint{5.765214in}{0.633078in}}%
\pgfpathlineto{\pgfqpoint{5.765214in}{0.643679in}}%
\pgfpathlineto{\pgfqpoint{5.766558in}{0.630428in}}%
\pgfpathlineto{\pgfqpoint{5.767231in}{0.630428in}}%
\pgfpathlineto{\pgfqpoint{5.767231in}{0.635728in}}%
\pgfpathlineto{\pgfqpoint{5.768575in}{0.633078in}}%
\pgfpathlineto{\pgfqpoint{5.769247in}{0.633078in}}%
\pgfpathlineto{\pgfqpoint{5.769247in}{0.635728in}}%
\pgfpathlineto{\pgfqpoint{5.770591in}{0.633078in}}%
\pgfpathlineto{\pgfqpoint{5.772608in}{0.633078in}}%
\pgfpathlineto{\pgfqpoint{5.772608in}{0.627778in}}%
\pgfpathlineto{\pgfqpoint{5.773280in}{0.641029in}}%
\pgfpathlineto{\pgfqpoint{5.773952in}{0.635728in}}%
\pgfpathlineto{\pgfqpoint{5.775297in}{0.635728in}}%
\pgfpathlineto{\pgfqpoint{5.776641in}{0.627778in}}%
\pgfpathlineto{\pgfqpoint{5.777313in}{0.627778in}}%
\pgfpathlineto{\pgfqpoint{5.777313in}{0.638378in}}%
\pgfpathlineto{\pgfqpoint{5.778657in}{0.630428in}}%
\pgfpathlineto{\pgfqpoint{5.779330in}{0.630428in}}%
\pgfpathlineto{\pgfqpoint{5.779330in}{0.641029in}}%
\pgfpathlineto{\pgfqpoint{5.780674in}{0.641029in}}%
\pgfpathlineto{\pgfqpoint{5.781346in}{0.641029in}}%
\pgfpathlineto{\pgfqpoint{5.781346in}{0.630428in}}%
\pgfpathlineto{\pgfqpoint{5.782690in}{0.635728in}}%
\pgfpathlineto{\pgfqpoint{5.783363in}{0.635728in}}%
\pgfpathlineto{\pgfqpoint{5.783363in}{0.638378in}}%
\pgfpathlineto{\pgfqpoint{5.784707in}{0.627778in}}%
\pgfpathlineto{\pgfqpoint{5.785379in}{0.627778in}}%
\pgfpathlineto{\pgfqpoint{5.785379in}{0.633078in}}%
\pgfpathlineto{\pgfqpoint{5.786723in}{0.630428in}}%
\pgfpathlineto{\pgfqpoint{5.787396in}{0.630428in}}%
\pgfpathlineto{\pgfqpoint{5.787396in}{0.627778in}}%
\pgfpathlineto{\pgfqpoint{5.788740in}{0.633078in}}%
\pgfpathlineto{\pgfqpoint{5.790084in}{0.633078in}}%
\pgfpathlineto{\pgfqpoint{5.790084in}{0.630428in}}%
\pgfpathlineto{\pgfqpoint{5.791429in}{0.633078in}}%
\pgfpathlineto{\pgfqpoint{5.792101in}{0.633078in}}%
\pgfpathlineto{\pgfqpoint{5.792773in}{0.641029in}}%
\pgfpathlineto{\pgfqpoint{5.793445in}{0.638378in}}%
\pgfpathlineto{\pgfqpoint{5.794117in}{0.638378in}}%
\pgfpathlineto{\pgfqpoint{5.794117in}{0.627778in}}%
\pgfpathlineto{\pgfqpoint{5.795462in}{0.633078in}}%
\pgfpathlineto{\pgfqpoint{5.796134in}{0.633078in}}%
\pgfpathlineto{\pgfqpoint{5.796134in}{0.635728in}}%
\pgfpathlineto{\pgfqpoint{5.796806in}{0.630428in}}%
\pgfpathlineto{\pgfqpoint{5.797478in}{0.630428in}}%
\pgfpathlineto{\pgfqpoint{5.798150in}{0.630428in}}%
\pgfpathlineto{\pgfqpoint{5.798150in}{0.627778in}}%
\pgfpathlineto{\pgfqpoint{5.798822in}{0.635728in}}%
\pgfpathlineto{\pgfqpoint{5.799495in}{0.627778in}}%
\pgfpathlineto{\pgfqpoint{5.800167in}{0.627778in}}%
\pgfpathlineto{\pgfqpoint{5.800839in}{0.633078in}}%
\pgfpathlineto{\pgfqpoint{5.801511in}{0.633078in}}%
\pgfusepath{stroke}%
\end{pgfscope}%
\begin{pgfscope}%
\pgfpathrectangle{\pgfqpoint{3.662674in}{0.552778in}}{\pgfqpoint{2.138715in}{1.650000in}}%
\pgfusepath{clip}%
\pgfsetrectcap%
\pgfsetroundjoin%
\pgfsetlinewidth{1.505625pt}%
\definecolor{currentstroke}{rgb}{1.000000,0.498039,0.054902}%
\pgfsetstrokecolor{currentstroke}%
\pgfsetstrokeopacity{0.800000}%
\pgfsetdash{}{0pt}%
\pgfpathmoveto{\pgfqpoint{3.663346in}{0.638378in}}%
\pgfpathlineto{\pgfqpoint{3.663346in}{0.641029in}}%
\pgfpathlineto{\pgfqpoint{3.664690in}{0.633078in}}%
\pgfpathlineto{\pgfqpoint{3.665362in}{0.633078in}}%
\pgfpathlineto{\pgfqpoint{3.665362in}{0.651629in}}%
\pgfpathlineto{\pgfqpoint{3.666707in}{0.651629in}}%
\pgfpathlineto{\pgfqpoint{3.667379in}{0.651629in}}%
\pgfpathlineto{\pgfqpoint{3.668723in}{0.633078in}}%
\pgfpathlineto{\pgfqpoint{3.669395in}{0.633078in}}%
\pgfpathlineto{\pgfqpoint{3.670740in}{0.643679in}}%
\pgfpathlineto{\pgfqpoint{3.671412in}{0.643679in}}%
\pgfpathlineto{\pgfqpoint{3.671412in}{0.651629in}}%
\pgfpathlineto{\pgfqpoint{3.672084in}{0.638378in}}%
\pgfpathlineto{\pgfqpoint{3.672756in}{0.648979in}}%
\pgfpathlineto{\pgfqpoint{3.673428in}{0.648979in}}%
\pgfpathlineto{\pgfqpoint{3.673428in}{0.635728in}}%
\pgfpathlineto{\pgfqpoint{3.674773in}{0.643679in}}%
\pgfpathlineto{\pgfqpoint{3.675445in}{0.643679in}}%
\pgfpathlineto{\pgfqpoint{3.675445in}{0.635728in}}%
\pgfpathlineto{\pgfqpoint{3.676117in}{0.659580in}}%
\pgfpathlineto{\pgfqpoint{3.676789in}{0.643679in}}%
\pgfpathlineto{\pgfqpoint{3.677461in}{0.643679in}}%
\pgfpathlineto{\pgfqpoint{3.677461in}{0.646329in}}%
\pgfpathlineto{\pgfqpoint{3.678133in}{0.638378in}}%
\pgfpathlineto{\pgfqpoint{3.678806in}{0.646329in}}%
\pgfpathlineto{\pgfqpoint{3.680150in}{0.646329in}}%
\pgfpathlineto{\pgfqpoint{3.680150in}{0.641029in}}%
\pgfpathlineto{\pgfqpoint{3.680822in}{0.648979in}}%
\pgfpathlineto{\pgfqpoint{3.681494in}{0.643679in}}%
\pgfpathlineto{\pgfqpoint{3.682166in}{0.643679in}}%
\pgfpathlineto{\pgfqpoint{3.683511in}{0.656930in}}%
\pgfpathlineto{\pgfqpoint{3.684183in}{0.656930in}}%
\pgfpathlineto{\pgfqpoint{3.685527in}{0.633078in}}%
\pgfpathlineto{\pgfqpoint{3.686199in}{0.633078in}}%
\pgfpathlineto{\pgfqpoint{3.686199in}{0.641029in}}%
\pgfpathlineto{\pgfqpoint{3.687544in}{0.638378in}}%
\pgfpathlineto{\pgfqpoint{3.688216in}{0.638378in}}%
\pgfpathlineto{\pgfqpoint{3.688216in}{0.641029in}}%
\pgfpathlineto{\pgfqpoint{3.689560in}{0.641029in}}%
\pgfpathlineto{\pgfqpoint{3.690232in}{0.641029in}}%
\pgfpathlineto{\pgfqpoint{3.690232in}{0.638378in}}%
\pgfpathlineto{\pgfqpoint{3.691577in}{0.646329in}}%
\pgfpathlineto{\pgfqpoint{3.692249in}{0.646329in}}%
\pgfpathlineto{\pgfqpoint{3.693593in}{0.633078in}}%
\pgfpathlineto{\pgfqpoint{3.694265in}{0.633078in}}%
\pgfpathlineto{\pgfqpoint{3.695610in}{0.638378in}}%
\pgfpathlineto{\pgfqpoint{3.696954in}{0.638378in}}%
\pgfpathlineto{\pgfqpoint{3.698298in}{0.654280in}}%
\pgfpathlineto{\pgfqpoint{3.698971in}{0.654280in}}%
\pgfpathlineto{\pgfqpoint{3.699643in}{0.641029in}}%
\pgfpathlineto{\pgfqpoint{3.700315in}{0.641029in}}%
\pgfpathlineto{\pgfqpoint{3.700987in}{0.641029in}}%
\pgfpathlineto{\pgfqpoint{3.700987in}{0.638378in}}%
\pgfpathlineto{\pgfqpoint{3.701659in}{0.651629in}}%
\pgfpathlineto{\pgfqpoint{3.702332in}{0.646329in}}%
\pgfpathlineto{\pgfqpoint{3.703004in}{0.646329in}}%
\pgfpathlineto{\pgfqpoint{3.703004in}{0.641029in}}%
\pgfpathlineto{\pgfqpoint{3.704348in}{0.643679in}}%
\pgfpathlineto{\pgfqpoint{3.706365in}{0.643679in}}%
\pgfpathlineto{\pgfqpoint{3.706365in}{0.648979in}}%
\pgfpathlineto{\pgfqpoint{3.707709in}{0.646329in}}%
\pgfpathlineto{\pgfqpoint{3.708381in}{0.646329in}}%
\pgfpathlineto{\pgfqpoint{3.709053in}{0.641029in}}%
\pgfpathlineto{\pgfqpoint{3.709725in}{0.656930in}}%
\pgfpathlineto{\pgfqpoint{3.710398in}{0.656930in}}%
\pgfpathlineto{\pgfqpoint{3.710398in}{0.635728in}}%
\pgfpathlineto{\pgfqpoint{3.711742in}{0.654280in}}%
\pgfpathlineto{\pgfqpoint{3.712414in}{0.654280in}}%
\pgfpathlineto{\pgfqpoint{3.713758in}{0.638378in}}%
\pgfpathlineto{\pgfqpoint{3.714431in}{0.638378in}}%
\pgfpathlineto{\pgfqpoint{3.715103in}{0.648979in}}%
\pgfpathlineto{\pgfqpoint{3.715775in}{0.641029in}}%
\pgfpathlineto{\pgfqpoint{3.716447in}{0.641029in}}%
\pgfpathlineto{\pgfqpoint{3.717119in}{0.638378in}}%
\pgfpathlineto{\pgfqpoint{3.717791in}{0.654280in}}%
\pgfpathlineto{\pgfqpoint{3.718464in}{0.654280in}}%
\pgfpathlineto{\pgfqpoint{3.718464in}{0.638378in}}%
\pgfpathlineto{\pgfqpoint{3.719808in}{0.654280in}}%
\pgfpathlineto{\pgfqpoint{3.720480in}{0.654280in}}%
\pgfpathlineto{\pgfqpoint{3.720480in}{0.648979in}}%
\pgfpathlineto{\pgfqpoint{3.721152in}{0.659580in}}%
\pgfpathlineto{\pgfqpoint{3.721824in}{0.648979in}}%
\pgfpathlineto{\pgfqpoint{3.722497in}{0.648979in}}%
\pgfpathlineto{\pgfqpoint{3.722497in}{0.656930in}}%
\pgfpathlineto{\pgfqpoint{3.723169in}{0.641029in}}%
\pgfpathlineto{\pgfqpoint{3.723841in}{0.648979in}}%
\pgfpathlineto{\pgfqpoint{3.724513in}{0.648979in}}%
\pgfpathlineto{\pgfqpoint{3.725185in}{0.635728in}}%
\pgfpathlineto{\pgfqpoint{3.725857in}{0.654280in}}%
\pgfpathlineto{\pgfqpoint{3.726530in}{0.654280in}}%
\pgfpathlineto{\pgfqpoint{3.726530in}{0.656930in}}%
\pgfpathlineto{\pgfqpoint{3.727874in}{0.635728in}}%
\pgfpathlineto{\pgfqpoint{3.728546in}{0.635728in}}%
\pgfpathlineto{\pgfqpoint{3.728546in}{0.651629in}}%
\pgfpathlineto{\pgfqpoint{3.729890in}{0.646329in}}%
\pgfpathlineto{\pgfqpoint{3.730563in}{0.646329in}}%
\pgfpathlineto{\pgfqpoint{3.730563in}{0.633078in}}%
\pgfpathlineto{\pgfqpoint{3.731907in}{0.646329in}}%
\pgfpathlineto{\pgfqpoint{3.732579in}{0.646329in}}%
\pgfpathlineto{\pgfqpoint{3.732579in}{0.648979in}}%
\pgfpathlineto{\pgfqpoint{3.733923in}{0.648979in}}%
\pgfpathlineto{\pgfqpoint{3.734596in}{0.648979in}}%
\pgfpathlineto{\pgfqpoint{3.734596in}{0.659580in}}%
\pgfpathlineto{\pgfqpoint{3.735268in}{0.643679in}}%
\pgfpathlineto{\pgfqpoint{3.735940in}{0.648979in}}%
\pgfpathlineto{\pgfqpoint{3.736612in}{0.648979in}}%
\pgfpathlineto{\pgfqpoint{3.736612in}{0.659580in}}%
\pgfpathlineto{\pgfqpoint{3.737956in}{0.638378in}}%
\pgfpathlineto{\pgfqpoint{3.738629in}{0.638378in}}%
\pgfpathlineto{\pgfqpoint{3.739973in}{0.651629in}}%
\pgfpathlineto{\pgfqpoint{3.740645in}{0.651629in}}%
\pgfpathlineto{\pgfqpoint{3.740645in}{0.648979in}}%
\pgfpathlineto{\pgfqpoint{3.741317in}{0.662230in}}%
\pgfpathlineto{\pgfqpoint{3.741989in}{0.651629in}}%
\pgfpathlineto{\pgfqpoint{3.743334in}{0.651629in}}%
\pgfpathlineto{\pgfqpoint{3.743334in}{0.654280in}}%
\pgfpathlineto{\pgfqpoint{3.744006in}{0.635728in}}%
\pgfpathlineto{\pgfqpoint{3.744678in}{0.646329in}}%
\pgfpathlineto{\pgfqpoint{3.745350in}{0.646329in}}%
\pgfpathlineto{\pgfqpoint{3.746022in}{0.638378in}}%
\pgfpathlineto{\pgfqpoint{3.746695in}{0.664880in}}%
\pgfpathlineto{\pgfqpoint{3.747367in}{0.664880in}}%
\pgfpathlineto{\pgfqpoint{3.748039in}{0.641029in}}%
\pgfpathlineto{\pgfqpoint{3.748711in}{0.651629in}}%
\pgfpathlineto{\pgfqpoint{3.750055in}{0.651629in}}%
\pgfpathlineto{\pgfqpoint{3.750055in}{0.656930in}}%
\pgfpathlineto{\pgfqpoint{3.750728in}{0.643679in}}%
\pgfpathlineto{\pgfqpoint{3.751400in}{0.643679in}}%
\pgfpathlineto{\pgfqpoint{3.752072in}{0.643679in}}%
\pgfpathlineto{\pgfqpoint{3.752744in}{0.651629in}}%
\pgfpathlineto{\pgfqpoint{3.753416in}{0.643679in}}%
\pgfpathlineto{\pgfqpoint{3.754088in}{0.643679in}}%
\pgfpathlineto{\pgfqpoint{3.755433in}{0.662230in}}%
\pgfpathlineto{\pgfqpoint{3.756105in}{0.662230in}}%
\pgfpathlineto{\pgfqpoint{3.756105in}{0.646329in}}%
\pgfpathlineto{\pgfqpoint{3.757449in}{0.662230in}}%
\pgfpathlineto{\pgfqpoint{3.758121in}{0.662230in}}%
\pgfpathlineto{\pgfqpoint{3.758794in}{0.646329in}}%
\pgfpathlineto{\pgfqpoint{3.759466in}{0.656930in}}%
\pgfpathlineto{\pgfqpoint{3.760138in}{0.656930in}}%
\pgfpathlineto{\pgfqpoint{3.760138in}{0.641029in}}%
\pgfpathlineto{\pgfqpoint{3.761482in}{0.662230in}}%
\pgfpathlineto{\pgfqpoint{3.762154in}{0.662230in}}%
\pgfpathlineto{\pgfqpoint{3.763499in}{0.630428in}}%
\pgfpathlineto{\pgfqpoint{3.764171in}{0.630428in}}%
\pgfpathlineto{\pgfqpoint{3.764171in}{0.662230in}}%
\pgfpathlineto{\pgfqpoint{3.765515in}{0.651629in}}%
\pgfpathlineto{\pgfqpoint{3.766187in}{0.651629in}}%
\pgfpathlineto{\pgfqpoint{3.766187in}{0.641029in}}%
\pgfpathlineto{\pgfqpoint{3.766860in}{0.654280in}}%
\pgfpathlineto{\pgfqpoint{3.767532in}{0.654280in}}%
\pgfpathlineto{\pgfqpoint{3.768204in}{0.654280in}}%
\pgfpathlineto{\pgfqpoint{3.768876in}{0.662230in}}%
\pgfpathlineto{\pgfqpoint{3.769548in}{0.638378in}}%
\pgfpathlineto{\pgfqpoint{3.770220in}{0.638378in}}%
\pgfpathlineto{\pgfqpoint{3.771565in}{0.664880in}}%
\pgfpathlineto{\pgfqpoint{3.772237in}{0.664880in}}%
\pgfpathlineto{\pgfqpoint{3.773581in}{0.654280in}}%
\pgfpathlineto{\pgfqpoint{3.774253in}{0.654280in}}%
\pgfpathlineto{\pgfqpoint{3.774253in}{0.656930in}}%
\pgfpathlineto{\pgfqpoint{3.775598in}{0.638378in}}%
\pgfpathlineto{\pgfqpoint{3.776270in}{0.638378in}}%
\pgfpathlineto{\pgfqpoint{3.776942in}{0.662230in}}%
\pgfpathlineto{\pgfqpoint{3.777614in}{0.659580in}}%
\pgfpathlineto{\pgfqpoint{3.778286in}{0.659580in}}%
\pgfpathlineto{\pgfqpoint{3.778286in}{0.670181in}}%
\pgfpathlineto{\pgfqpoint{3.779631in}{0.667530in}}%
\pgfpathlineto{\pgfqpoint{3.780303in}{0.667530in}}%
\pgfpathlineto{\pgfqpoint{3.780975in}{0.643679in}}%
\pgfpathlineto{\pgfqpoint{3.781647in}{0.656930in}}%
\pgfpathlineto{\pgfqpoint{3.782319in}{0.656930in}}%
\pgfpathlineto{\pgfqpoint{3.782319in}{0.664880in}}%
\pgfpathlineto{\pgfqpoint{3.782992in}{0.646329in}}%
\pgfpathlineto{\pgfqpoint{3.783664in}{0.654280in}}%
\pgfpathlineto{\pgfqpoint{3.784336in}{0.654280in}}%
\pgfpathlineto{\pgfqpoint{3.784336in}{0.659580in}}%
\pgfpathlineto{\pgfqpoint{3.785680in}{0.646329in}}%
\pgfpathlineto{\pgfqpoint{3.786352in}{0.646329in}}%
\pgfpathlineto{\pgfqpoint{3.786352in}{0.641029in}}%
\pgfpathlineto{\pgfqpoint{3.787697in}{0.662230in}}%
\pgfpathlineto{\pgfqpoint{3.788369in}{0.662230in}}%
\pgfpathlineto{\pgfqpoint{3.788369in}{0.672831in}}%
\pgfpathlineto{\pgfqpoint{3.789713in}{0.648979in}}%
\pgfpathlineto{\pgfqpoint{3.790385in}{0.648979in}}%
\pgfpathlineto{\pgfqpoint{3.790385in}{0.656930in}}%
\pgfpathlineto{\pgfqpoint{3.791058in}{0.646329in}}%
\pgfpathlineto{\pgfqpoint{3.791730in}{0.651629in}}%
\pgfpathlineto{\pgfqpoint{3.792402in}{0.651629in}}%
\pgfpathlineto{\pgfqpoint{3.792402in}{0.659580in}}%
\pgfpathlineto{\pgfqpoint{3.793746in}{0.659580in}}%
\pgfpathlineto{\pgfqpoint{3.794418in}{0.659580in}}%
\pgfpathlineto{\pgfqpoint{3.794418in}{0.656930in}}%
\pgfpathlineto{\pgfqpoint{3.795763in}{0.667530in}}%
\pgfpathlineto{\pgfqpoint{3.797107in}{0.667530in}}%
\pgfpathlineto{\pgfqpoint{3.797107in}{0.648979in}}%
\pgfpathlineto{\pgfqpoint{3.797779in}{0.678131in}}%
\pgfpathlineto{\pgfqpoint{3.798451in}{0.670181in}}%
\pgfpathlineto{\pgfqpoint{3.799124in}{0.670181in}}%
\pgfpathlineto{\pgfqpoint{3.799124in}{0.667530in}}%
\pgfpathlineto{\pgfqpoint{3.800468in}{0.667530in}}%
\pgfpathlineto{\pgfqpoint{3.801140in}{0.667530in}}%
\pgfpathlineto{\pgfqpoint{3.801812in}{0.648979in}}%
\pgfpathlineto{\pgfqpoint{3.802484in}{0.662230in}}%
\pgfpathlineto{\pgfqpoint{3.803157in}{0.662230in}}%
\pgfpathlineto{\pgfqpoint{3.803157in}{0.680781in}}%
\pgfpathlineto{\pgfqpoint{3.804501in}{0.675481in}}%
\pgfpathlineto{\pgfqpoint{3.805173in}{0.675481in}}%
\pgfpathlineto{\pgfqpoint{3.806517in}{0.654280in}}%
\pgfpathlineto{\pgfqpoint{3.807190in}{0.654280in}}%
\pgfpathlineto{\pgfqpoint{3.807862in}{0.675481in}}%
\pgfpathlineto{\pgfqpoint{3.808534in}{0.648979in}}%
\pgfpathlineto{\pgfqpoint{3.809206in}{0.648979in}}%
\pgfpathlineto{\pgfqpoint{3.809206in}{0.670181in}}%
\pgfpathlineto{\pgfqpoint{3.810550in}{0.670181in}}%
\pgfpathlineto{\pgfqpoint{3.811223in}{0.670181in}}%
\pgfpathlineto{\pgfqpoint{3.811895in}{0.654280in}}%
\pgfpathlineto{\pgfqpoint{3.812567in}{0.662230in}}%
\pgfpathlineto{\pgfqpoint{3.813239in}{0.662230in}}%
\pgfpathlineto{\pgfqpoint{3.813911in}{0.659580in}}%
\pgfpathlineto{\pgfqpoint{3.814584in}{0.672831in}}%
\pgfpathlineto{\pgfqpoint{3.815256in}{0.672831in}}%
\pgfpathlineto{\pgfqpoint{3.815256in}{0.675481in}}%
\pgfpathlineto{\pgfqpoint{3.816600in}{0.667530in}}%
\pgfpathlineto{\pgfqpoint{3.817272in}{0.667530in}}%
\pgfpathlineto{\pgfqpoint{3.818617in}{0.686082in}}%
\pgfpathlineto{\pgfqpoint{3.819289in}{0.686082in}}%
\pgfpathlineto{\pgfqpoint{3.820633in}{0.648979in}}%
\pgfpathlineto{\pgfqpoint{3.821305in}{0.648979in}}%
\pgfpathlineto{\pgfqpoint{3.822650in}{0.678131in}}%
\pgfpathlineto{\pgfqpoint{3.823994in}{0.678131in}}%
\pgfpathlineto{\pgfqpoint{3.825338in}{0.656930in}}%
\pgfpathlineto{\pgfqpoint{3.826010in}{0.656930in}}%
\pgfpathlineto{\pgfqpoint{3.826010in}{0.667530in}}%
\pgfpathlineto{\pgfqpoint{3.827355in}{0.656930in}}%
\pgfpathlineto{\pgfqpoint{3.828027in}{0.656930in}}%
\pgfpathlineto{\pgfqpoint{3.828027in}{0.654280in}}%
\pgfpathlineto{\pgfqpoint{3.828699in}{0.659580in}}%
\pgfpathlineto{\pgfqpoint{3.829371in}{0.654280in}}%
\pgfpathlineto{\pgfqpoint{3.830043in}{0.654280in}}%
\pgfpathlineto{\pgfqpoint{3.830043in}{0.667530in}}%
\pgfpathlineto{\pgfqpoint{3.831388in}{0.662230in}}%
\pgfpathlineto{\pgfqpoint{3.832732in}{0.662230in}}%
\pgfpathlineto{\pgfqpoint{3.832732in}{0.678131in}}%
\pgfpathlineto{\pgfqpoint{3.834076in}{0.654280in}}%
\pgfpathlineto{\pgfqpoint{3.834749in}{0.654280in}}%
\pgfpathlineto{\pgfqpoint{3.834749in}{0.667530in}}%
\pgfpathlineto{\pgfqpoint{3.836093in}{0.662230in}}%
\pgfpathlineto{\pgfqpoint{3.836765in}{0.662230in}}%
\pgfpathlineto{\pgfqpoint{3.838109in}{0.688732in}}%
\pgfpathlineto{\pgfqpoint{3.838782in}{0.688732in}}%
\pgfpathlineto{\pgfqpoint{3.838782in}{0.664880in}}%
\pgfpathlineto{\pgfqpoint{3.840126in}{0.670181in}}%
\pgfpathlineto{\pgfqpoint{3.840798in}{0.670181in}}%
\pgfpathlineto{\pgfqpoint{3.840798in}{0.667530in}}%
\pgfpathlineto{\pgfqpoint{3.841470in}{0.678131in}}%
\pgfpathlineto{\pgfqpoint{3.842142in}{0.672831in}}%
\pgfpathlineto{\pgfqpoint{3.842815in}{0.672831in}}%
\pgfpathlineto{\pgfqpoint{3.842815in}{0.694032in}}%
\pgfpathlineto{\pgfqpoint{3.844159in}{0.662230in}}%
\pgfpathlineto{\pgfqpoint{3.844831in}{0.662230in}}%
\pgfpathlineto{\pgfqpoint{3.844831in}{0.683431in}}%
\pgfpathlineto{\pgfqpoint{3.845503in}{0.659580in}}%
\pgfpathlineto{\pgfqpoint{3.846175in}{0.664880in}}%
\pgfpathlineto{\pgfqpoint{3.846848in}{0.664880in}}%
\pgfpathlineto{\pgfqpoint{3.846848in}{0.643679in}}%
\pgfpathlineto{\pgfqpoint{3.848192in}{0.683431in}}%
\pgfpathlineto{\pgfqpoint{3.848864in}{0.683431in}}%
\pgfpathlineto{\pgfqpoint{3.848864in}{0.664880in}}%
\pgfpathlineto{\pgfqpoint{3.850208in}{0.664880in}}%
\pgfpathlineto{\pgfqpoint{3.850881in}{0.664880in}}%
\pgfpathlineto{\pgfqpoint{3.850881in}{0.672831in}}%
\pgfpathlineto{\pgfqpoint{3.852225in}{0.667530in}}%
\pgfpathlineto{\pgfqpoint{3.852897in}{0.667530in}}%
\pgfpathlineto{\pgfqpoint{3.853569in}{0.659580in}}%
\pgfpathlineto{\pgfqpoint{3.854241in}{0.672831in}}%
\pgfpathlineto{\pgfqpoint{3.854914in}{0.672831in}}%
\pgfpathlineto{\pgfqpoint{3.854914in}{0.678131in}}%
\pgfpathlineto{\pgfqpoint{3.855586in}{0.670181in}}%
\pgfpathlineto{\pgfqpoint{3.856258in}{0.672831in}}%
\pgfpathlineto{\pgfqpoint{3.856930in}{0.672831in}}%
\pgfpathlineto{\pgfqpoint{3.856930in}{0.670181in}}%
\pgfpathlineto{\pgfqpoint{3.857602in}{0.691382in}}%
\pgfpathlineto{\pgfqpoint{3.858274in}{0.672831in}}%
\pgfpathlineto{\pgfqpoint{3.858947in}{0.672831in}}%
\pgfpathlineto{\pgfqpoint{3.859619in}{0.686082in}}%
\pgfpathlineto{\pgfqpoint{3.860291in}{0.675481in}}%
\pgfpathlineto{\pgfqpoint{3.860963in}{0.675481in}}%
\pgfpathlineto{\pgfqpoint{3.860963in}{0.664880in}}%
\pgfpathlineto{\pgfqpoint{3.861635in}{0.683431in}}%
\pgfpathlineto{\pgfqpoint{3.862307in}{0.664880in}}%
\pgfpathlineto{\pgfqpoint{3.862980in}{0.664880in}}%
\pgfpathlineto{\pgfqpoint{3.862980in}{0.699333in}}%
\pgfpathlineto{\pgfqpoint{3.863652in}{0.662230in}}%
\pgfpathlineto{\pgfqpoint{3.864324in}{0.662230in}}%
\pgfpathlineto{\pgfqpoint{3.864996in}{0.662230in}}%
\pgfpathlineto{\pgfqpoint{3.865668in}{0.694032in}}%
\pgfpathlineto{\pgfqpoint{3.866340in}{0.667530in}}%
\pgfpathlineto{\pgfqpoint{3.867013in}{0.667530in}}%
\pgfpathlineto{\pgfqpoint{3.867685in}{0.656930in}}%
\pgfpathlineto{\pgfqpoint{3.868357in}{0.675481in}}%
\pgfpathlineto{\pgfqpoint{3.869029in}{0.675481in}}%
\pgfpathlineto{\pgfqpoint{3.869701in}{0.654280in}}%
\pgfpathlineto{\pgfqpoint{3.870373in}{0.680781in}}%
\pgfpathlineto{\pgfqpoint{3.871046in}{0.680781in}}%
\pgfpathlineto{\pgfqpoint{3.871046in}{0.667530in}}%
\pgfpathlineto{\pgfqpoint{3.872390in}{0.667530in}}%
\pgfpathlineto{\pgfqpoint{3.873062in}{0.667530in}}%
\pgfpathlineto{\pgfqpoint{3.873734in}{0.680781in}}%
\pgfpathlineto{\pgfqpoint{3.874406in}{0.670181in}}%
\pgfpathlineto{\pgfqpoint{3.875079in}{0.670181in}}%
\pgfpathlineto{\pgfqpoint{3.876423in}{0.704633in}}%
\pgfpathlineto{\pgfqpoint{3.877095in}{0.704633in}}%
\pgfpathlineto{\pgfqpoint{3.877095in}{0.680781in}}%
\pgfpathlineto{\pgfqpoint{3.878439in}{0.680781in}}%
\pgfpathlineto{\pgfqpoint{3.879112in}{0.680781in}}%
\pgfpathlineto{\pgfqpoint{3.880456in}{0.667530in}}%
\pgfpathlineto{\pgfqpoint{3.881128in}{0.667530in}}%
\pgfpathlineto{\pgfqpoint{3.882472in}{0.691382in}}%
\pgfpathlineto{\pgfqpoint{3.883145in}{0.691382in}}%
\pgfpathlineto{\pgfqpoint{3.883145in}{0.701983in}}%
\pgfpathlineto{\pgfqpoint{3.883817in}{0.680781in}}%
\pgfpathlineto{\pgfqpoint{3.884489in}{0.683431in}}%
\pgfpathlineto{\pgfqpoint{3.885161in}{0.683431in}}%
\pgfpathlineto{\pgfqpoint{3.885833in}{0.707283in}}%
\pgfpathlineto{\pgfqpoint{3.886505in}{0.670181in}}%
\pgfpathlineto{\pgfqpoint{3.887178in}{0.670181in}}%
\pgfpathlineto{\pgfqpoint{3.887178in}{0.691382in}}%
\pgfpathlineto{\pgfqpoint{3.888522in}{0.675481in}}%
\pgfpathlineto{\pgfqpoint{3.889194in}{0.675481in}}%
\pgfpathlineto{\pgfqpoint{3.889866in}{0.699333in}}%
\pgfpathlineto{\pgfqpoint{3.890538in}{0.680781in}}%
\pgfpathlineto{\pgfqpoint{3.891211in}{0.680781in}}%
\pgfpathlineto{\pgfqpoint{3.892555in}{0.670181in}}%
\pgfpathlineto{\pgfqpoint{3.893227in}{0.670181in}}%
\pgfpathlineto{\pgfqpoint{3.893899in}{0.667530in}}%
\pgfpathlineto{\pgfqpoint{3.894571in}{0.678131in}}%
\pgfpathlineto{\pgfqpoint{3.895244in}{0.678131in}}%
\pgfpathlineto{\pgfqpoint{3.896588in}{0.712583in}}%
\pgfpathlineto{\pgfqpoint{3.897932in}{0.712583in}}%
\pgfpathlineto{\pgfqpoint{3.898604in}{0.670181in}}%
\pgfpathlineto{\pgfqpoint{3.899277in}{0.694032in}}%
\pgfpathlineto{\pgfqpoint{3.899949in}{0.694032in}}%
\pgfpathlineto{\pgfqpoint{3.900621in}{0.680781in}}%
\pgfpathlineto{\pgfqpoint{3.901293in}{0.680781in}}%
\pgfpathlineto{\pgfqpoint{3.901965in}{0.680781in}}%
\pgfpathlineto{\pgfqpoint{3.901965in}{0.694032in}}%
\pgfpathlineto{\pgfqpoint{3.903310in}{0.664880in}}%
\pgfpathlineto{\pgfqpoint{3.903982in}{0.664880in}}%
\pgfpathlineto{\pgfqpoint{3.904654in}{0.691382in}}%
\pgfpathlineto{\pgfqpoint{3.905326in}{0.688732in}}%
\pgfpathlineto{\pgfqpoint{3.905998in}{0.688732in}}%
\pgfpathlineto{\pgfqpoint{3.905998in}{0.694032in}}%
\pgfpathlineto{\pgfqpoint{3.907343in}{0.691382in}}%
\pgfpathlineto{\pgfqpoint{3.908015in}{0.691382in}}%
\pgfpathlineto{\pgfqpoint{3.908015in}{0.683431in}}%
\pgfpathlineto{\pgfqpoint{3.909359in}{0.704633in}}%
\pgfpathlineto{\pgfqpoint{3.910031in}{0.704633in}}%
\pgfpathlineto{\pgfqpoint{3.910703in}{0.712583in}}%
\pgfpathlineto{\pgfqpoint{3.911376in}{0.683431in}}%
\pgfpathlineto{\pgfqpoint{3.912048in}{0.683431in}}%
\pgfpathlineto{\pgfqpoint{3.912048in}{0.688732in}}%
\pgfpathlineto{\pgfqpoint{3.912720in}{0.670181in}}%
\pgfpathlineto{\pgfqpoint{3.913392in}{0.680781in}}%
\pgfpathlineto{\pgfqpoint{3.914064in}{0.680781in}}%
\pgfpathlineto{\pgfqpoint{3.914736in}{0.723184in}}%
\pgfpathlineto{\pgfqpoint{3.915409in}{0.709933in}}%
\pgfpathlineto{\pgfqpoint{3.916081in}{0.709933in}}%
\pgfpathlineto{\pgfqpoint{3.917425in}{0.680781in}}%
\pgfpathlineto{\pgfqpoint{3.918097in}{0.680781in}}%
\pgfpathlineto{\pgfqpoint{3.918769in}{0.678131in}}%
\pgfpathlineto{\pgfqpoint{3.919442in}{0.712583in}}%
\pgfpathlineto{\pgfqpoint{3.920114in}{0.712583in}}%
\pgfpathlineto{\pgfqpoint{3.920786in}{0.683431in}}%
\pgfpathlineto{\pgfqpoint{3.921458in}{0.707283in}}%
\pgfpathlineto{\pgfqpoint{3.922130in}{0.707283in}}%
\pgfpathlineto{\pgfqpoint{3.923475in}{0.686082in}}%
\pgfpathlineto{\pgfqpoint{3.924147in}{0.686082in}}%
\pgfpathlineto{\pgfqpoint{3.924147in}{0.680781in}}%
\pgfpathlineto{\pgfqpoint{3.924819in}{0.696682in}}%
\pgfpathlineto{\pgfqpoint{3.925491in}{0.691382in}}%
\pgfpathlineto{\pgfqpoint{3.926163in}{0.691382in}}%
\pgfpathlineto{\pgfqpoint{3.926163in}{0.707283in}}%
\pgfpathlineto{\pgfqpoint{3.926836in}{0.686082in}}%
\pgfpathlineto{\pgfqpoint{3.927508in}{0.691382in}}%
\pgfpathlineto{\pgfqpoint{3.928852in}{0.691382in}}%
\pgfpathlineto{\pgfqpoint{3.928852in}{0.672831in}}%
\pgfpathlineto{\pgfqpoint{3.930196in}{0.704633in}}%
\pgfpathlineto{\pgfqpoint{3.930869in}{0.704633in}}%
\pgfpathlineto{\pgfqpoint{3.931541in}{0.688732in}}%
\pgfpathlineto{\pgfqpoint{3.932213in}{0.733785in}}%
\pgfpathlineto{\pgfqpoint{3.932885in}{0.733785in}}%
\pgfpathlineto{\pgfqpoint{3.933557in}{0.672831in}}%
\pgfpathlineto{\pgfqpoint{3.934229in}{0.699333in}}%
\pgfpathlineto{\pgfqpoint{3.935574in}{0.699333in}}%
\pgfpathlineto{\pgfqpoint{3.935574in}{0.728484in}}%
\pgfpathlineto{\pgfqpoint{3.936918in}{0.680781in}}%
\pgfpathlineto{\pgfqpoint{3.937590in}{0.680781in}}%
\pgfpathlineto{\pgfqpoint{3.937590in}{0.728484in}}%
\pgfpathlineto{\pgfqpoint{3.938935in}{0.694032in}}%
\pgfpathlineto{\pgfqpoint{3.939607in}{0.694032in}}%
\pgfpathlineto{\pgfqpoint{3.939607in}{0.723184in}}%
\pgfpathlineto{\pgfqpoint{3.940951in}{0.715234in}}%
\pgfpathlineto{\pgfqpoint{3.941623in}{0.715234in}}%
\pgfpathlineto{\pgfqpoint{3.942295in}{0.701983in}}%
\pgfpathlineto{\pgfqpoint{3.942968in}{0.725834in}}%
\pgfpathlineto{\pgfqpoint{3.943640in}{0.725834in}}%
\pgfpathlineto{\pgfqpoint{3.943640in}{0.683431in}}%
\pgfpathlineto{\pgfqpoint{3.944984in}{0.704633in}}%
\pgfpathlineto{\pgfqpoint{3.945656in}{0.704633in}}%
\pgfpathlineto{\pgfqpoint{3.946328in}{0.717884in}}%
\pgfpathlineto{\pgfqpoint{3.947001in}{0.688732in}}%
\pgfpathlineto{\pgfqpoint{3.947673in}{0.688732in}}%
\pgfpathlineto{\pgfqpoint{3.949017in}{0.712583in}}%
\pgfpathlineto{\pgfqpoint{3.949689in}{0.712583in}}%
\pgfpathlineto{\pgfqpoint{3.949689in}{0.715234in}}%
\pgfpathlineto{\pgfqpoint{3.951034in}{0.683431in}}%
\pgfpathlineto{\pgfqpoint{3.951706in}{0.683431in}}%
\pgfpathlineto{\pgfqpoint{3.953050in}{0.709933in}}%
\pgfpathlineto{\pgfqpoint{3.953722in}{0.709933in}}%
\pgfpathlineto{\pgfqpoint{3.954394in}{0.683431in}}%
\pgfpathlineto{\pgfqpoint{3.955067in}{0.704633in}}%
\pgfpathlineto{\pgfqpoint{3.955739in}{0.704633in}}%
\pgfpathlineto{\pgfqpoint{3.956411in}{0.728484in}}%
\pgfpathlineto{\pgfqpoint{3.957083in}{0.717884in}}%
\pgfpathlineto{\pgfqpoint{3.957755in}{0.717884in}}%
\pgfpathlineto{\pgfqpoint{3.959100in}{0.688732in}}%
\pgfpathlineto{\pgfqpoint{3.960444in}{0.688732in}}%
\pgfpathlineto{\pgfqpoint{3.961116in}{0.723184in}}%
\pgfpathlineto{\pgfqpoint{3.961788in}{0.707283in}}%
\pgfpathlineto{\pgfqpoint{3.962460in}{0.707283in}}%
\pgfpathlineto{\pgfqpoint{3.962460in}{0.696682in}}%
\pgfpathlineto{\pgfqpoint{3.963805in}{0.717884in}}%
\pgfpathlineto{\pgfqpoint{3.964477in}{0.717884in}}%
\pgfpathlineto{\pgfqpoint{3.965821in}{0.683431in}}%
\pgfpathlineto{\pgfqpoint{3.966493in}{0.683431in}}%
\pgfpathlineto{\pgfqpoint{3.967838in}{0.717884in}}%
\pgfpathlineto{\pgfqpoint{3.968510in}{0.717884in}}%
\pgfpathlineto{\pgfqpoint{3.968510in}{0.686082in}}%
\pgfpathlineto{\pgfqpoint{3.969854in}{0.739085in}}%
\pgfpathlineto{\pgfqpoint{3.970526in}{0.739085in}}%
\pgfpathlineto{\pgfqpoint{3.971199in}{0.709933in}}%
\pgfpathlineto{\pgfqpoint{3.971871in}{0.712583in}}%
\pgfpathlineto{\pgfqpoint{3.972543in}{0.712583in}}%
\pgfpathlineto{\pgfqpoint{3.972543in}{0.731135in}}%
\pgfpathlineto{\pgfqpoint{3.973887in}{0.728484in}}%
\pgfpathlineto{\pgfqpoint{3.974559in}{0.728484in}}%
\pgfpathlineto{\pgfqpoint{3.975232in}{0.691382in}}%
\pgfpathlineto{\pgfqpoint{3.975904in}{0.709933in}}%
\pgfpathlineto{\pgfqpoint{3.976576in}{0.709933in}}%
\pgfpathlineto{\pgfqpoint{3.976576in}{0.696682in}}%
\pgfpathlineto{\pgfqpoint{3.977920in}{0.696682in}}%
\pgfpathlineto{\pgfqpoint{3.978592in}{0.696682in}}%
\pgfpathlineto{\pgfqpoint{3.979937in}{0.704633in}}%
\pgfpathlineto{\pgfqpoint{3.980609in}{0.704633in}}%
\pgfpathlineto{\pgfqpoint{3.980609in}{0.699333in}}%
\pgfpathlineto{\pgfqpoint{3.981281in}{0.715234in}}%
\pgfpathlineto{\pgfqpoint{3.981953in}{0.709933in}}%
\pgfpathlineto{\pgfqpoint{3.982625in}{0.709933in}}%
\pgfpathlineto{\pgfqpoint{3.983298in}{0.728484in}}%
\pgfpathlineto{\pgfqpoint{3.983970in}{0.717884in}}%
\pgfpathlineto{\pgfqpoint{3.984642in}{0.717884in}}%
\pgfpathlineto{\pgfqpoint{3.984642in}{0.731135in}}%
\pgfpathlineto{\pgfqpoint{3.985986in}{0.717884in}}%
\pgfpathlineto{\pgfqpoint{3.986658in}{0.717884in}}%
\pgfpathlineto{\pgfqpoint{3.987331in}{0.725834in}}%
\pgfpathlineto{\pgfqpoint{3.988003in}{0.715234in}}%
\pgfpathlineto{\pgfqpoint{3.988675in}{0.715234in}}%
\pgfpathlineto{\pgfqpoint{3.989347in}{0.739085in}}%
\pgfpathlineto{\pgfqpoint{3.990019in}{0.699333in}}%
\pgfpathlineto{\pgfqpoint{3.990691in}{0.699333in}}%
\pgfpathlineto{\pgfqpoint{3.992036in}{0.733785in}}%
\pgfpathlineto{\pgfqpoint{3.992708in}{0.733785in}}%
\pgfpathlineto{\pgfqpoint{3.992708in}{0.704633in}}%
\pgfpathlineto{\pgfqpoint{3.994052in}{0.728484in}}%
\pgfpathlineto{\pgfqpoint{3.994724in}{0.728484in}}%
\pgfpathlineto{\pgfqpoint{3.994724in}{0.715234in}}%
\pgfpathlineto{\pgfqpoint{3.996069in}{0.725834in}}%
\pgfpathlineto{\pgfqpoint{3.996741in}{0.725834in}}%
\pgfpathlineto{\pgfqpoint{3.996741in}{0.731135in}}%
\pgfpathlineto{\pgfqpoint{3.997413in}{0.717884in}}%
\pgfpathlineto{\pgfqpoint{3.998085in}{0.723184in}}%
\pgfpathlineto{\pgfqpoint{3.998757in}{0.723184in}}%
\pgfpathlineto{\pgfqpoint{4.000102in}{0.704633in}}%
\pgfpathlineto{\pgfqpoint{4.000774in}{0.704633in}}%
\pgfpathlineto{\pgfqpoint{4.001446in}{0.725834in}}%
\pgfpathlineto{\pgfqpoint{4.002118in}{0.704633in}}%
\pgfpathlineto{\pgfqpoint{4.002790in}{0.704633in}}%
\pgfpathlineto{\pgfqpoint{4.003463in}{0.691382in}}%
\pgfpathlineto{\pgfqpoint{4.004135in}{0.720534in}}%
\pgfpathlineto{\pgfqpoint{4.004807in}{0.720534in}}%
\pgfpathlineto{\pgfqpoint{4.005479in}{0.739085in}}%
\pgfpathlineto{\pgfqpoint{4.006151in}{0.688732in}}%
\pgfpathlineto{\pgfqpoint{4.006823in}{0.688732in}}%
\pgfpathlineto{\pgfqpoint{4.006823in}{0.754986in}}%
\pgfpathlineto{\pgfqpoint{4.008168in}{0.699333in}}%
\pgfpathlineto{\pgfqpoint{4.008840in}{0.699333in}}%
\pgfpathlineto{\pgfqpoint{4.009512in}{0.749686in}}%
\pgfpathlineto{\pgfqpoint{4.010184in}{0.699333in}}%
\pgfpathlineto{\pgfqpoint{4.010856in}{0.699333in}}%
\pgfpathlineto{\pgfqpoint{4.012201in}{0.747036in}}%
\pgfpathlineto{\pgfqpoint{4.012873in}{0.747036in}}%
\pgfpathlineto{\pgfqpoint{4.014217in}{0.712583in}}%
\pgfpathlineto{\pgfqpoint{4.014889in}{0.712583in}}%
\pgfpathlineto{\pgfqpoint{4.014889in}{0.741735in}}%
\pgfpathlineto{\pgfqpoint{4.016234in}{0.733785in}}%
\pgfpathlineto{\pgfqpoint{4.016906in}{0.733785in}}%
\pgfpathlineto{\pgfqpoint{4.017578in}{0.717884in}}%
\pgfpathlineto{\pgfqpoint{4.018250in}{0.717884in}}%
\pgfpathlineto{\pgfqpoint{4.018922in}{0.717884in}}%
\pgfpathlineto{\pgfqpoint{4.019595in}{0.749686in}}%
\pgfpathlineto{\pgfqpoint{4.020267in}{0.728484in}}%
\pgfpathlineto{\pgfqpoint{4.020939in}{0.728484in}}%
\pgfpathlineto{\pgfqpoint{4.020939in}{0.707283in}}%
\pgfpathlineto{\pgfqpoint{4.021611in}{0.757636in}}%
\pgfpathlineto{\pgfqpoint{4.022283in}{0.717884in}}%
\pgfpathlineto{\pgfqpoint{4.022955in}{0.717884in}}%
\pgfpathlineto{\pgfqpoint{4.023628in}{0.747036in}}%
\pgfpathlineto{\pgfqpoint{4.024300in}{0.741735in}}%
\pgfpathlineto{\pgfqpoint{4.024972in}{0.741735in}}%
\pgfpathlineto{\pgfqpoint{4.024972in}{0.749686in}}%
\pgfpathlineto{\pgfqpoint{4.025644in}{0.699333in}}%
\pgfpathlineto{\pgfqpoint{4.026316in}{0.715234in}}%
\pgfpathlineto{\pgfqpoint{4.026988in}{0.715234in}}%
\pgfpathlineto{\pgfqpoint{4.026988in}{0.747036in}}%
\pgfpathlineto{\pgfqpoint{4.028333in}{0.707283in}}%
\pgfpathlineto{\pgfqpoint{4.029005in}{0.707283in}}%
\pgfpathlineto{\pgfqpoint{4.029005in}{0.739085in}}%
\pgfpathlineto{\pgfqpoint{4.030349in}{0.728484in}}%
\pgfpathlineto{\pgfqpoint{4.031021in}{0.728484in}}%
\pgfpathlineto{\pgfqpoint{4.031021in}{0.715234in}}%
\pgfpathlineto{\pgfqpoint{4.032366in}{0.765587in}}%
\pgfpathlineto{\pgfqpoint{4.033038in}{0.765587in}}%
\pgfpathlineto{\pgfqpoint{4.033710in}{0.707283in}}%
\pgfpathlineto{\pgfqpoint{4.034382in}{0.728484in}}%
\pgfpathlineto{\pgfqpoint{4.035054in}{0.728484in}}%
\pgfpathlineto{\pgfqpoint{4.035054in}{0.741735in}}%
\pgfpathlineto{\pgfqpoint{4.035727in}{0.723184in}}%
\pgfpathlineto{\pgfqpoint{4.036399in}{0.725834in}}%
\pgfpathlineto{\pgfqpoint{4.037071in}{0.725834in}}%
\pgfpathlineto{\pgfqpoint{4.037743in}{0.709933in}}%
\pgfpathlineto{\pgfqpoint{4.038415in}{0.760287in}}%
\pgfpathlineto{\pgfqpoint{4.039088in}{0.760287in}}%
\pgfpathlineto{\pgfqpoint{4.039088in}{0.725834in}}%
\pgfpathlineto{\pgfqpoint{4.040432in}{0.765587in}}%
\pgfpathlineto{\pgfqpoint{4.041104in}{0.765587in}}%
\pgfpathlineto{\pgfqpoint{4.042448in}{0.723184in}}%
\pgfpathlineto{\pgfqpoint{4.043121in}{0.723184in}}%
\pgfpathlineto{\pgfqpoint{4.043121in}{0.707283in}}%
\pgfpathlineto{\pgfqpoint{4.043793in}{0.744386in}}%
\pgfpathlineto{\pgfqpoint{4.044465in}{0.744386in}}%
\pgfpathlineto{\pgfqpoint{4.045137in}{0.744386in}}%
\pgfpathlineto{\pgfqpoint{4.045137in}{0.723184in}}%
\pgfpathlineto{\pgfqpoint{4.046481in}{0.725834in}}%
\pgfpathlineto{\pgfqpoint{4.047826in}{0.725834in}}%
\pgfpathlineto{\pgfqpoint{4.049170in}{0.744386in}}%
\pgfpathlineto{\pgfqpoint{4.049842in}{0.744386in}}%
\pgfpathlineto{\pgfqpoint{4.049842in}{0.728484in}}%
\pgfpathlineto{\pgfqpoint{4.051187in}{0.744386in}}%
\pgfpathlineto{\pgfqpoint{4.051859in}{0.744386in}}%
\pgfpathlineto{\pgfqpoint{4.051859in}{0.723184in}}%
\pgfpathlineto{\pgfqpoint{4.053203in}{0.778838in}}%
\pgfpathlineto{\pgfqpoint{4.053875in}{0.778838in}}%
\pgfpathlineto{\pgfqpoint{4.053875in}{0.715234in}}%
\pgfpathlineto{\pgfqpoint{4.055220in}{0.749686in}}%
\pgfpathlineto{\pgfqpoint{4.055892in}{0.749686in}}%
\pgfpathlineto{\pgfqpoint{4.057236in}{0.723184in}}%
\pgfpathlineto{\pgfqpoint{4.057908in}{0.723184in}}%
\pgfpathlineto{\pgfqpoint{4.059253in}{0.752336in}}%
\pgfpathlineto{\pgfqpoint{4.059925in}{0.752336in}}%
\pgfpathlineto{\pgfqpoint{4.059925in}{0.760287in}}%
\pgfpathlineto{\pgfqpoint{4.061269in}{0.728484in}}%
\pgfpathlineto{\pgfqpoint{4.061941in}{0.728484in}}%
\pgfpathlineto{\pgfqpoint{4.062613in}{0.736435in}}%
\pgfpathlineto{\pgfqpoint{4.063286in}{0.720534in}}%
\pgfpathlineto{\pgfqpoint{4.063958in}{0.720534in}}%
\pgfpathlineto{\pgfqpoint{4.063958in}{0.768237in}}%
\pgfpathlineto{\pgfqpoint{4.064630in}{0.715234in}}%
\pgfpathlineto{\pgfqpoint{4.065302in}{0.757636in}}%
\pgfpathlineto{\pgfqpoint{4.065974in}{0.757636in}}%
\pgfpathlineto{\pgfqpoint{4.067319in}{0.731135in}}%
\pgfpathlineto{\pgfqpoint{4.067991in}{0.731135in}}%
\pgfpathlineto{\pgfqpoint{4.067991in}{0.744386in}}%
\pgfpathlineto{\pgfqpoint{4.068663in}{0.709933in}}%
\pgfpathlineto{\pgfqpoint{4.069335in}{0.739085in}}%
\pgfpathlineto{\pgfqpoint{4.070007in}{0.739085in}}%
\pgfpathlineto{\pgfqpoint{4.071352in}{0.694032in}}%
\pgfpathlineto{\pgfqpoint{4.072024in}{0.694032in}}%
\pgfpathlineto{\pgfqpoint{4.072696in}{0.739085in}}%
\pgfpathlineto{\pgfqpoint{4.073368in}{0.736435in}}%
\pgfpathlineto{\pgfqpoint{4.074040in}{0.736435in}}%
\pgfpathlineto{\pgfqpoint{4.074040in}{0.694032in}}%
\pgfpathlineto{\pgfqpoint{4.075385in}{0.736435in}}%
\pgfpathlineto{\pgfqpoint{4.076057in}{0.736435in}}%
\pgfpathlineto{\pgfqpoint{4.076729in}{0.717884in}}%
\pgfpathlineto{\pgfqpoint{4.077401in}{0.728484in}}%
\pgfpathlineto{\pgfqpoint{4.078073in}{0.728484in}}%
\pgfpathlineto{\pgfqpoint{4.078073in}{0.720534in}}%
\pgfpathlineto{\pgfqpoint{4.078745in}{0.757636in}}%
\pgfpathlineto{\pgfqpoint{4.079418in}{0.728484in}}%
\pgfpathlineto{\pgfqpoint{4.080090in}{0.728484in}}%
\pgfpathlineto{\pgfqpoint{4.080090in}{0.720534in}}%
\pgfpathlineto{\pgfqpoint{4.080762in}{0.741735in}}%
\pgfpathlineto{\pgfqpoint{4.081434in}{0.725834in}}%
\pgfpathlineto{\pgfqpoint{4.082106in}{0.725834in}}%
\pgfpathlineto{\pgfqpoint{4.082106in}{0.731135in}}%
\pgfpathlineto{\pgfqpoint{4.082778in}{0.704633in}}%
\pgfpathlineto{\pgfqpoint{4.083451in}{0.731135in}}%
\pgfpathlineto{\pgfqpoint{4.084123in}{0.731135in}}%
\pgfpathlineto{\pgfqpoint{4.085467in}{0.765587in}}%
\pgfpathlineto{\pgfqpoint{4.086139in}{0.765587in}}%
\pgfpathlineto{\pgfqpoint{4.086139in}{0.704633in}}%
\pgfpathlineto{\pgfqpoint{4.087484in}{0.752336in}}%
\pgfpathlineto{\pgfqpoint{4.088156in}{0.752336in}}%
\pgfpathlineto{\pgfqpoint{4.088828in}{0.709933in}}%
\pgfpathlineto{\pgfqpoint{4.089500in}{0.731135in}}%
\pgfpathlineto{\pgfqpoint{4.090172in}{0.731135in}}%
\pgfpathlineto{\pgfqpoint{4.091517in}{0.770887in}}%
\pgfpathlineto{\pgfqpoint{4.092189in}{0.770887in}}%
\pgfpathlineto{\pgfqpoint{4.093533in}{0.709933in}}%
\pgfpathlineto{\pgfqpoint{4.094205in}{0.709933in}}%
\pgfpathlineto{\pgfqpoint{4.094205in}{0.739085in}}%
\pgfpathlineto{\pgfqpoint{4.095550in}{0.712583in}}%
\pgfpathlineto{\pgfqpoint{4.096222in}{0.712583in}}%
\pgfpathlineto{\pgfqpoint{4.096894in}{0.709933in}}%
\pgfpathlineto{\pgfqpoint{4.097566in}{0.739085in}}%
\pgfpathlineto{\pgfqpoint{4.098238in}{0.739085in}}%
\pgfpathlineto{\pgfqpoint{4.098238in}{0.765587in}}%
\pgfpathlineto{\pgfqpoint{4.098910in}{0.723184in}}%
\pgfpathlineto{\pgfqpoint{4.099583in}{0.744386in}}%
\pgfpathlineto{\pgfqpoint{4.100255in}{0.744386in}}%
\pgfpathlineto{\pgfqpoint{4.101599in}{0.728484in}}%
\pgfpathlineto{\pgfqpoint{4.102271in}{0.728484in}}%
\pgfpathlineto{\pgfqpoint{4.102271in}{0.744386in}}%
\pgfpathlineto{\pgfqpoint{4.103616in}{0.736435in}}%
\pgfpathlineto{\pgfqpoint{4.104288in}{0.736435in}}%
\pgfpathlineto{\pgfqpoint{4.104288in}{0.749686in}}%
\pgfpathlineto{\pgfqpoint{4.105632in}{0.733785in}}%
\pgfpathlineto{\pgfqpoint{4.106304in}{0.733785in}}%
\pgfpathlineto{\pgfqpoint{4.106304in}{0.736435in}}%
\pgfpathlineto{\pgfqpoint{4.107649in}{0.731135in}}%
\pgfpathlineto{\pgfqpoint{4.108321in}{0.731135in}}%
\pgfpathlineto{\pgfqpoint{4.108993in}{0.717884in}}%
\pgfpathlineto{\pgfqpoint{4.109665in}{0.733785in}}%
\pgfpathlineto{\pgfqpoint{4.110337in}{0.733785in}}%
\pgfpathlineto{\pgfqpoint{4.111682in}{0.762937in}}%
\pgfpathlineto{\pgfqpoint{4.112354in}{0.762937in}}%
\pgfpathlineto{\pgfqpoint{4.112354in}{0.747036in}}%
\pgfpathlineto{\pgfqpoint{4.113698in}{0.749686in}}%
\pgfpathlineto{\pgfqpoint{4.114370in}{0.749686in}}%
\pgfpathlineto{\pgfqpoint{4.115715in}{0.723184in}}%
\pgfpathlineto{\pgfqpoint{4.116387in}{0.723184in}}%
\pgfpathlineto{\pgfqpoint{4.116387in}{0.736435in}}%
\pgfpathlineto{\pgfqpoint{4.117731in}{0.736435in}}%
\pgfpathlineto{\pgfqpoint{4.118403in}{0.736435in}}%
\pgfpathlineto{\pgfqpoint{4.118403in}{0.715234in}}%
\pgfpathlineto{\pgfqpoint{4.119075in}{0.752336in}}%
\pgfpathlineto{\pgfqpoint{4.119748in}{0.733785in}}%
\pgfpathlineto{\pgfqpoint{4.120420in}{0.733785in}}%
\pgfpathlineto{\pgfqpoint{4.121092in}{0.720534in}}%
\pgfpathlineto{\pgfqpoint{4.121764in}{0.752336in}}%
\pgfpathlineto{\pgfqpoint{4.122436in}{0.752336in}}%
\pgfpathlineto{\pgfqpoint{4.123108in}{0.728484in}}%
\pgfpathlineto{\pgfqpoint{4.123108in}{0.757636in}}%
\pgfpathlineto{\pgfqpoint{4.123781in}{0.728484in}}%
\pgfpathlineto{\pgfqpoint{4.124453in}{0.728484in}}%
\pgfpathlineto{\pgfqpoint{4.124453in}{0.760287in}}%
\pgfpathlineto{\pgfqpoint{4.125797in}{0.725834in}}%
\pgfpathlineto{\pgfqpoint{4.126469in}{0.725834in}}%
\pgfpathlineto{\pgfqpoint{4.126469in}{0.768237in}}%
\pgfpathlineto{\pgfqpoint{4.127814in}{0.754986in}}%
\pgfpathlineto{\pgfqpoint{4.128486in}{0.754986in}}%
\pgfpathlineto{\pgfqpoint{4.129158in}{0.717884in}}%
\pgfpathlineto{\pgfqpoint{4.129830in}{0.717884in}}%
\pgfpathlineto{\pgfqpoint{4.130502in}{0.717884in}}%
\pgfpathlineto{\pgfqpoint{4.130502in}{0.691382in}}%
\pgfpathlineto{\pgfqpoint{4.131847in}{0.733785in}}%
\pgfpathlineto{\pgfqpoint{4.132519in}{0.733785in}}%
\pgfpathlineto{\pgfqpoint{4.132519in}{0.747036in}}%
\pgfpathlineto{\pgfqpoint{4.133191in}{0.709933in}}%
\pgfpathlineto{\pgfqpoint{4.133863in}{0.717884in}}%
\pgfpathlineto{\pgfqpoint{4.134535in}{0.717884in}}%
\pgfpathlineto{\pgfqpoint{4.135207in}{0.770887in}}%
\pgfpathlineto{\pgfqpoint{4.135880in}{0.728484in}}%
\pgfpathlineto{\pgfqpoint{4.136552in}{0.728484in}}%
\pgfpathlineto{\pgfqpoint{4.136552in}{0.712583in}}%
\pgfpathlineto{\pgfqpoint{4.137896in}{0.715234in}}%
\pgfpathlineto{\pgfqpoint{4.138568in}{0.715234in}}%
\pgfpathlineto{\pgfqpoint{4.139240in}{0.776188in}}%
\pgfpathlineto{\pgfqpoint{4.139913in}{0.728484in}}%
\pgfpathlineto{\pgfqpoint{4.140585in}{0.728484in}}%
\pgfpathlineto{\pgfqpoint{4.141929in}{0.747036in}}%
\pgfpathlineto{\pgfqpoint{4.142601in}{0.747036in}}%
\pgfpathlineto{\pgfqpoint{4.142601in}{0.709933in}}%
\pgfpathlineto{\pgfqpoint{4.143273in}{0.752336in}}%
\pgfpathlineto{\pgfqpoint{4.143946in}{0.731135in}}%
\pgfpathlineto{\pgfqpoint{4.144618in}{0.731135in}}%
\pgfpathlineto{\pgfqpoint{4.144618in}{0.760287in}}%
\pgfpathlineto{\pgfqpoint{4.145962in}{0.720534in}}%
\pgfpathlineto{\pgfqpoint{4.146634in}{0.720534in}}%
\pgfpathlineto{\pgfqpoint{4.147306in}{0.696682in}}%
\pgfpathlineto{\pgfqpoint{4.147979in}{0.757636in}}%
\pgfpathlineto{\pgfqpoint{4.148651in}{0.757636in}}%
\pgfpathlineto{\pgfqpoint{4.149995in}{0.731135in}}%
\pgfpathlineto{\pgfqpoint{4.150667in}{0.731135in}}%
\pgfpathlineto{\pgfqpoint{4.150667in}{0.744386in}}%
\pgfpathlineto{\pgfqpoint{4.152012in}{0.717884in}}%
\pgfpathlineto{\pgfqpoint{4.152684in}{0.717884in}}%
\pgfpathlineto{\pgfqpoint{4.154028in}{0.762937in}}%
\pgfpathlineto{\pgfqpoint{4.154700in}{0.762937in}}%
\pgfpathlineto{\pgfqpoint{4.155373in}{0.731135in}}%
\pgfpathlineto{\pgfqpoint{4.156045in}{0.770887in}}%
\pgfpathlineto{\pgfqpoint{4.156717in}{0.770887in}}%
\pgfpathlineto{\pgfqpoint{4.156717in}{0.728484in}}%
\pgfpathlineto{\pgfqpoint{4.158061in}{0.747036in}}%
\pgfpathlineto{\pgfqpoint{4.158733in}{0.747036in}}%
\pgfpathlineto{\pgfqpoint{4.158733in}{0.728484in}}%
\pgfpathlineto{\pgfqpoint{4.159406in}{0.776188in}}%
\pgfpathlineto{\pgfqpoint{4.160078in}{0.757636in}}%
\pgfpathlineto{\pgfqpoint{4.160750in}{0.757636in}}%
\pgfpathlineto{\pgfqpoint{4.162094in}{0.720534in}}%
\pgfpathlineto{\pgfqpoint{4.162766in}{0.720534in}}%
\pgfpathlineto{\pgfqpoint{4.163439in}{0.781488in}}%
\pgfpathlineto{\pgfqpoint{4.164111in}{0.760287in}}%
\pgfpathlineto{\pgfqpoint{4.164783in}{0.760287in}}%
\pgfpathlineto{\pgfqpoint{4.164783in}{0.762937in}}%
\pgfpathlineto{\pgfqpoint{4.165455in}{0.720534in}}%
\pgfpathlineto{\pgfqpoint{4.166127in}{0.736435in}}%
\pgfpathlineto{\pgfqpoint{4.166799in}{0.736435in}}%
\pgfpathlineto{\pgfqpoint{4.166799in}{0.723184in}}%
\pgfpathlineto{\pgfqpoint{4.167472in}{0.778838in}}%
\pgfpathlineto{\pgfqpoint{4.168144in}{0.765587in}}%
\pgfpathlineto{\pgfqpoint{4.168816in}{0.765587in}}%
\pgfpathlineto{\pgfqpoint{4.169488in}{0.768237in}}%
\pgfpathlineto{\pgfqpoint{4.170160in}{0.725834in}}%
\pgfpathlineto{\pgfqpoint{4.170832in}{0.725834in}}%
\pgfpathlineto{\pgfqpoint{4.172177in}{0.749686in}}%
\pgfpathlineto{\pgfqpoint{4.172849in}{0.749686in}}%
\pgfpathlineto{\pgfqpoint{4.174193in}{0.707283in}}%
\pgfpathlineto{\pgfqpoint{4.174865in}{0.707283in}}%
\pgfpathlineto{\pgfqpoint{4.176210in}{0.781488in}}%
\pgfpathlineto{\pgfqpoint{4.176882in}{0.781488in}}%
\pgfpathlineto{\pgfqpoint{4.176882in}{0.715234in}}%
\pgfpathlineto{\pgfqpoint{4.178226in}{0.768237in}}%
\pgfpathlineto{\pgfqpoint{4.178898in}{0.768237in}}%
\pgfpathlineto{\pgfqpoint{4.178898in}{0.725834in}}%
\pgfpathlineto{\pgfqpoint{4.179571in}{0.776188in}}%
\pgfpathlineto{\pgfqpoint{4.180243in}{0.728484in}}%
\pgfpathlineto{\pgfqpoint{4.180915in}{0.728484in}}%
\pgfpathlineto{\pgfqpoint{4.180915in}{0.757636in}}%
\pgfpathlineto{\pgfqpoint{4.182259in}{0.754986in}}%
\pgfpathlineto{\pgfqpoint{4.182931in}{0.754986in}}%
\pgfpathlineto{\pgfqpoint{4.183604in}{0.720534in}}%
\pgfpathlineto{\pgfqpoint{4.184276in}{0.741735in}}%
\pgfpathlineto{\pgfqpoint{4.184948in}{0.741735in}}%
\pgfpathlineto{\pgfqpoint{4.185620in}{0.757636in}}%
\pgfpathlineto{\pgfqpoint{4.186292in}{0.757636in}}%
\pgfpathlineto{\pgfqpoint{4.186964in}{0.757636in}}%
\pgfpathlineto{\pgfqpoint{4.186964in}{0.773537in}}%
\pgfpathlineto{\pgfqpoint{4.187637in}{0.754986in}}%
\pgfpathlineto{\pgfqpoint{4.188309in}{0.757636in}}%
\pgfpathlineto{\pgfqpoint{4.188981in}{0.757636in}}%
\pgfpathlineto{\pgfqpoint{4.188981in}{0.736435in}}%
\pgfpathlineto{\pgfqpoint{4.189653in}{0.765587in}}%
\pgfpathlineto{\pgfqpoint{4.190325in}{0.744386in}}%
\pgfpathlineto{\pgfqpoint{4.190997in}{0.744386in}}%
\pgfpathlineto{\pgfqpoint{4.190997in}{0.752336in}}%
\pgfpathlineto{\pgfqpoint{4.192342in}{0.749686in}}%
\pgfpathlineto{\pgfqpoint{4.193014in}{0.749686in}}%
\pgfpathlineto{\pgfqpoint{4.193014in}{0.720534in}}%
\pgfpathlineto{\pgfqpoint{4.193686in}{0.757636in}}%
\pgfpathlineto{\pgfqpoint{4.194358in}{0.752336in}}%
\pgfpathlineto{\pgfqpoint{4.195030in}{0.752336in}}%
\pgfpathlineto{\pgfqpoint{4.195030in}{0.757636in}}%
\pgfpathlineto{\pgfqpoint{4.196375in}{0.723184in}}%
\pgfpathlineto{\pgfqpoint{4.197047in}{0.723184in}}%
\pgfpathlineto{\pgfqpoint{4.197047in}{0.749686in}}%
\pgfpathlineto{\pgfqpoint{4.198391in}{0.731135in}}%
\pgfpathlineto{\pgfqpoint{4.199063in}{0.731135in}}%
\pgfpathlineto{\pgfqpoint{4.199736in}{0.752336in}}%
\pgfpathlineto{\pgfqpoint{4.200408in}{0.749686in}}%
\pgfpathlineto{\pgfqpoint{4.201080in}{0.749686in}}%
\pgfpathlineto{\pgfqpoint{4.201080in}{0.768237in}}%
\pgfpathlineto{\pgfqpoint{4.202424in}{0.725834in}}%
\pgfpathlineto{\pgfqpoint{4.203096in}{0.725834in}}%
\pgfpathlineto{\pgfqpoint{4.203096in}{0.744386in}}%
\pgfpathlineto{\pgfqpoint{4.204441in}{0.731135in}}%
\pgfpathlineto{\pgfqpoint{4.205113in}{0.731135in}}%
\pgfpathlineto{\pgfqpoint{4.205113in}{0.765587in}}%
\pgfpathlineto{\pgfqpoint{4.206457in}{0.717884in}}%
\pgfpathlineto{\pgfqpoint{4.207129in}{0.717884in}}%
\pgfpathlineto{\pgfqpoint{4.208474in}{0.770887in}}%
\pgfpathlineto{\pgfqpoint{4.209146in}{0.770887in}}%
\pgfpathlineto{\pgfqpoint{4.209146in}{0.741735in}}%
\pgfpathlineto{\pgfqpoint{4.210490in}{0.773537in}}%
\pgfpathlineto{\pgfqpoint{4.211162in}{0.773537in}}%
\pgfpathlineto{\pgfqpoint{4.211162in}{0.752336in}}%
\pgfpathlineto{\pgfqpoint{4.211835in}{0.778838in}}%
\pgfpathlineto{\pgfqpoint{4.212507in}{0.754986in}}%
\pgfpathlineto{\pgfqpoint{4.213179in}{0.754986in}}%
\pgfpathlineto{\pgfqpoint{4.214523in}{0.728484in}}%
\pgfpathlineto{\pgfqpoint{4.215195in}{0.728484in}}%
\pgfpathlineto{\pgfqpoint{4.215868in}{0.770887in}}%
\pgfpathlineto{\pgfqpoint{4.216540in}{0.723184in}}%
\pgfpathlineto{\pgfqpoint{4.217212in}{0.723184in}}%
\pgfpathlineto{\pgfqpoint{4.217884in}{0.768237in}}%
\pgfpathlineto{\pgfqpoint{4.218556in}{0.717884in}}%
\pgfpathlineto{\pgfqpoint{4.219228in}{0.717884in}}%
\pgfpathlineto{\pgfqpoint{4.220573in}{0.747036in}}%
\pgfpathlineto{\pgfqpoint{4.221245in}{0.747036in}}%
\pgfpathlineto{\pgfqpoint{4.222589in}{0.720534in}}%
\pgfpathlineto{\pgfqpoint{4.223261in}{0.720534in}}%
\pgfpathlineto{\pgfqpoint{4.223934in}{0.768237in}}%
\pgfpathlineto{\pgfqpoint{4.224606in}{0.725834in}}%
\pgfpathlineto{\pgfqpoint{4.225278in}{0.725834in}}%
\pgfpathlineto{\pgfqpoint{4.226622in}{0.754986in}}%
\pgfpathlineto{\pgfqpoint{4.227294in}{0.754986in}}%
\pgfpathlineto{\pgfqpoint{4.227294in}{0.760287in}}%
\pgfpathlineto{\pgfqpoint{4.228639in}{0.736435in}}%
\pgfpathlineto{\pgfqpoint{4.229311in}{0.736435in}}%
\pgfpathlineto{\pgfqpoint{4.229311in}{0.762937in}}%
\pgfpathlineto{\pgfqpoint{4.230655in}{0.725834in}}%
\pgfpathlineto{\pgfqpoint{4.231327in}{0.725834in}}%
\pgfpathlineto{\pgfqpoint{4.232672in}{0.757636in}}%
\pgfpathlineto{\pgfqpoint{4.233344in}{0.757636in}}%
\pgfpathlineto{\pgfqpoint{4.233344in}{0.770887in}}%
\pgfpathlineto{\pgfqpoint{4.234688in}{0.747036in}}%
\pgfpathlineto{\pgfqpoint{4.235360in}{0.747036in}}%
\pgfpathlineto{\pgfqpoint{4.236705in}{0.778838in}}%
\pgfpathlineto{\pgfqpoint{4.237377in}{0.778838in}}%
\pgfpathlineto{\pgfqpoint{4.237377in}{0.747036in}}%
\pgfpathlineto{\pgfqpoint{4.238049in}{0.784138in}}%
\pgfpathlineto{\pgfqpoint{4.238721in}{0.776188in}}%
\pgfpathlineto{\pgfqpoint{4.239393in}{0.776188in}}%
\pgfpathlineto{\pgfqpoint{4.240066in}{0.741735in}}%
\pgfpathlineto{\pgfqpoint{4.240738in}{0.744386in}}%
\pgfpathlineto{\pgfqpoint{4.241410in}{0.744386in}}%
\pgfpathlineto{\pgfqpoint{4.242754in}{0.778838in}}%
\pgfpathlineto{\pgfqpoint{4.243426in}{0.778838in}}%
\pgfpathlineto{\pgfqpoint{4.244099in}{0.739085in}}%
\pgfpathlineto{\pgfqpoint{4.244771in}{0.797389in}}%
\pgfpathlineto{\pgfqpoint{4.245443in}{0.797389in}}%
\pgfpathlineto{\pgfqpoint{4.246115in}{0.733785in}}%
\pgfpathlineto{\pgfqpoint{4.246787in}{0.765587in}}%
\pgfpathlineto{\pgfqpoint{4.247459in}{0.765587in}}%
\pgfpathlineto{\pgfqpoint{4.247459in}{0.731135in}}%
\pgfpathlineto{\pgfqpoint{4.248804in}{0.760287in}}%
\pgfpathlineto{\pgfqpoint{4.249476in}{0.760287in}}%
\pgfpathlineto{\pgfqpoint{4.249476in}{0.741735in}}%
\pgfpathlineto{\pgfqpoint{4.250148in}{0.765587in}}%
\pgfpathlineto{\pgfqpoint{4.250820in}{0.744386in}}%
\pgfpathlineto{\pgfqpoint{4.251492in}{0.744386in}}%
\pgfpathlineto{\pgfqpoint{4.252165in}{0.768237in}}%
\pgfpathlineto{\pgfqpoint{4.252837in}{0.715234in}}%
\pgfpathlineto{\pgfqpoint{4.253509in}{0.715234in}}%
\pgfpathlineto{\pgfqpoint{4.253509in}{0.781488in}}%
\pgfpathlineto{\pgfqpoint{4.254853in}{0.778838in}}%
\pgfpathlineto{\pgfqpoint{4.255525in}{0.778838in}}%
\pgfpathlineto{\pgfqpoint{4.256198in}{0.760287in}}%
\pgfpathlineto{\pgfqpoint{4.256870in}{0.773537in}}%
\pgfpathlineto{\pgfqpoint{4.257542in}{0.773537in}}%
\pgfpathlineto{\pgfqpoint{4.258214in}{0.786788in}}%
\pgfpathlineto{\pgfqpoint{4.258886in}{0.733785in}}%
\pgfpathlineto{\pgfqpoint{4.259558in}{0.733785in}}%
\pgfpathlineto{\pgfqpoint{4.260903in}{0.789439in}}%
\pgfpathlineto{\pgfqpoint{4.261575in}{0.789439in}}%
\pgfpathlineto{\pgfqpoint{4.261575in}{0.747036in}}%
\pgfpathlineto{\pgfqpoint{4.262919in}{0.752336in}}%
\pgfpathlineto{\pgfqpoint{4.263591in}{0.752336in}}%
\pgfpathlineto{\pgfqpoint{4.263591in}{0.747036in}}%
\pgfpathlineto{\pgfqpoint{4.264936in}{0.752336in}}%
\pgfpathlineto{\pgfqpoint{4.265608in}{0.752336in}}%
\pgfpathlineto{\pgfqpoint{4.265608in}{0.784138in}}%
\pgfpathlineto{\pgfqpoint{4.266952in}{0.768237in}}%
\pgfpathlineto{\pgfqpoint{4.267625in}{0.768237in}}%
\pgfpathlineto{\pgfqpoint{4.267625in}{0.757636in}}%
\pgfpathlineto{\pgfqpoint{4.268969in}{0.760287in}}%
\pgfpathlineto{\pgfqpoint{4.269641in}{0.760287in}}%
\pgfpathlineto{\pgfqpoint{4.269641in}{0.789439in}}%
\pgfpathlineto{\pgfqpoint{4.270313in}{0.744386in}}%
\pgfpathlineto{\pgfqpoint{4.270985in}{0.784138in}}%
\pgfpathlineto{\pgfqpoint{4.271658in}{0.784138in}}%
\pgfpathlineto{\pgfqpoint{4.272330in}{0.741735in}}%
\pgfpathlineto{\pgfqpoint{4.273002in}{0.749686in}}%
\pgfpathlineto{\pgfqpoint{4.273674in}{0.749686in}}%
\pgfpathlineto{\pgfqpoint{4.273674in}{0.728484in}}%
\pgfpathlineto{\pgfqpoint{4.274346in}{0.776188in}}%
\pgfpathlineto{\pgfqpoint{4.275018in}{0.736435in}}%
\pgfpathlineto{\pgfqpoint{4.275691in}{0.736435in}}%
\pgfpathlineto{\pgfqpoint{4.275691in}{0.765587in}}%
\pgfpathlineto{\pgfqpoint{4.276363in}{0.715234in}}%
\pgfpathlineto{\pgfqpoint{4.277035in}{0.736435in}}%
\pgfpathlineto{\pgfqpoint{4.277707in}{0.736435in}}%
\pgfpathlineto{\pgfqpoint{4.277707in}{0.784138in}}%
\pgfpathlineto{\pgfqpoint{4.279051in}{0.776188in}}%
\pgfpathlineto{\pgfqpoint{4.279724in}{0.776188in}}%
\pgfpathlineto{\pgfqpoint{4.279724in}{0.757636in}}%
\pgfpathlineto{\pgfqpoint{4.281068in}{0.778838in}}%
\pgfpathlineto{\pgfqpoint{4.281740in}{0.778838in}}%
\pgfpathlineto{\pgfqpoint{4.282412in}{0.728484in}}%
\pgfpathlineto{\pgfqpoint{4.283084in}{0.739085in}}%
\pgfpathlineto{\pgfqpoint{4.283757in}{0.739085in}}%
\pgfpathlineto{\pgfqpoint{4.284429in}{0.770887in}}%
\pgfpathlineto{\pgfqpoint{4.285101in}{0.768237in}}%
\pgfpathlineto{\pgfqpoint{4.285773in}{0.768237in}}%
\pgfpathlineto{\pgfqpoint{4.286445in}{0.723184in}}%
\pgfpathlineto{\pgfqpoint{4.287117in}{0.810640in}}%
\pgfpathlineto{\pgfqpoint{4.287790in}{0.810640in}}%
\pgfpathlineto{\pgfqpoint{4.287790in}{0.754986in}}%
\pgfpathlineto{\pgfqpoint{4.289134in}{0.768237in}}%
\pgfpathlineto{\pgfqpoint{4.289806in}{0.768237in}}%
\pgfpathlineto{\pgfqpoint{4.290478in}{0.749686in}}%
\pgfpathlineto{\pgfqpoint{4.291150in}{0.784138in}}%
\pgfpathlineto{\pgfqpoint{4.291823in}{0.784138in}}%
\pgfpathlineto{\pgfqpoint{4.291823in}{0.794739in}}%
\pgfpathlineto{\pgfqpoint{4.293167in}{0.741735in}}%
\pgfpathlineto{\pgfqpoint{4.293839in}{0.741735in}}%
\pgfpathlineto{\pgfqpoint{4.294511in}{0.765587in}}%
\pgfpathlineto{\pgfqpoint{4.295183in}{0.733785in}}%
\pgfpathlineto{\pgfqpoint{4.295856in}{0.733785in}}%
\pgfpathlineto{\pgfqpoint{4.297200in}{0.760287in}}%
\pgfpathlineto{\pgfqpoint{4.297872in}{0.760287in}}%
\pgfpathlineto{\pgfqpoint{4.297872in}{0.749686in}}%
\pgfpathlineto{\pgfqpoint{4.298544in}{0.810640in}}%
\pgfpathlineto{\pgfqpoint{4.299216in}{0.773537in}}%
\pgfpathlineto{\pgfqpoint{4.299889in}{0.773537in}}%
\pgfpathlineto{\pgfqpoint{4.299889in}{0.733785in}}%
\pgfpathlineto{\pgfqpoint{4.300561in}{0.786788in}}%
\pgfpathlineto{\pgfqpoint{4.301233in}{0.749686in}}%
\pgfpathlineto{\pgfqpoint{4.301905in}{0.749686in}}%
\pgfpathlineto{\pgfqpoint{4.301905in}{0.807990in}}%
\pgfpathlineto{\pgfqpoint{4.303249in}{0.749686in}}%
\pgfpathlineto{\pgfqpoint{4.303922in}{0.749686in}}%
\pgfpathlineto{\pgfqpoint{4.303922in}{0.794739in}}%
\pgfpathlineto{\pgfqpoint{4.305266in}{0.784138in}}%
\pgfpathlineto{\pgfqpoint{4.305938in}{0.784138in}}%
\pgfpathlineto{\pgfqpoint{4.305938in}{0.725834in}}%
\pgfpathlineto{\pgfqpoint{4.307282in}{0.762937in}}%
\pgfpathlineto{\pgfqpoint{4.307955in}{0.762937in}}%
\pgfpathlineto{\pgfqpoint{4.308627in}{0.792089in}}%
\pgfpathlineto{\pgfqpoint{4.309299in}{0.749686in}}%
\pgfpathlineto{\pgfqpoint{4.309971in}{0.749686in}}%
\pgfpathlineto{\pgfqpoint{4.309971in}{0.786788in}}%
\pgfpathlineto{\pgfqpoint{4.311315in}{0.778838in}}%
\pgfpathlineto{\pgfqpoint{4.311988in}{0.778838in}}%
\pgfpathlineto{\pgfqpoint{4.311988in}{0.784138in}}%
\pgfpathlineto{\pgfqpoint{4.313332in}{0.747036in}}%
\pgfpathlineto{\pgfqpoint{4.314004in}{0.747036in}}%
\pgfpathlineto{\pgfqpoint{4.314004in}{0.768237in}}%
\pgfpathlineto{\pgfqpoint{4.315348in}{0.768237in}}%
\pgfpathlineto{\pgfqpoint{4.316021in}{0.768237in}}%
\pgfpathlineto{\pgfqpoint{4.316021in}{0.747036in}}%
\pgfpathlineto{\pgfqpoint{4.316693in}{0.778838in}}%
\pgfpathlineto{\pgfqpoint{4.317365in}{0.760287in}}%
\pgfpathlineto{\pgfqpoint{4.318037in}{0.760287in}}%
\pgfpathlineto{\pgfqpoint{4.318709in}{0.778838in}}%
\pgfpathlineto{\pgfqpoint{4.319381in}{0.778838in}}%
\pgfpathlineto{\pgfqpoint{4.320054in}{0.778838in}}%
\pgfpathlineto{\pgfqpoint{4.320726in}{0.754986in}}%
\pgfpathlineto{\pgfqpoint{4.321398in}{0.792089in}}%
\pgfpathlineto{\pgfqpoint{4.322070in}{0.792089in}}%
\pgfpathlineto{\pgfqpoint{4.323414in}{0.744386in}}%
\pgfpathlineto{\pgfqpoint{4.324087in}{0.744386in}}%
\pgfpathlineto{\pgfqpoint{4.324759in}{0.784138in}}%
\pgfpathlineto{\pgfqpoint{4.325431in}{0.749686in}}%
\pgfpathlineto{\pgfqpoint{4.326103in}{0.749686in}}%
\pgfpathlineto{\pgfqpoint{4.327447in}{0.810640in}}%
\pgfpathlineto{\pgfqpoint{4.328120in}{0.810640in}}%
\pgfpathlineto{\pgfqpoint{4.328120in}{0.747036in}}%
\pgfpathlineto{\pgfqpoint{4.329464in}{0.784138in}}%
\pgfpathlineto{\pgfqpoint{4.330136in}{0.784138in}}%
\pgfpathlineto{\pgfqpoint{4.330808in}{0.789439in}}%
\pgfpathlineto{\pgfqpoint{4.331480in}{0.744386in}}%
\pgfpathlineto{\pgfqpoint{4.332825in}{0.744386in}}%
\pgfpathlineto{\pgfqpoint{4.332825in}{0.733785in}}%
\pgfpathlineto{\pgfqpoint{4.334169in}{0.792089in}}%
\pgfpathlineto{\pgfqpoint{4.334841in}{0.792089in}}%
\pgfpathlineto{\pgfqpoint{4.334841in}{0.749686in}}%
\pgfpathlineto{\pgfqpoint{4.336186in}{0.805340in}}%
\pgfpathlineto{\pgfqpoint{4.336858in}{0.805340in}}%
\pgfpathlineto{\pgfqpoint{4.338202in}{0.736435in}}%
\pgfpathlineto{\pgfqpoint{4.338874in}{0.736435in}}%
\pgfpathlineto{\pgfqpoint{4.340219in}{0.754986in}}%
\pgfpathlineto{\pgfqpoint{4.340891in}{0.754986in}}%
\pgfpathlineto{\pgfqpoint{4.342235in}{0.784138in}}%
\pgfpathlineto{\pgfqpoint{4.342907in}{0.784138in}}%
\pgfpathlineto{\pgfqpoint{4.343579in}{0.757636in}}%
\pgfpathlineto{\pgfqpoint{4.344252in}{0.823891in}}%
\pgfpathlineto{\pgfqpoint{4.344924in}{0.823891in}}%
\pgfpathlineto{\pgfqpoint{4.345596in}{0.752336in}}%
\pgfpathlineto{\pgfqpoint{4.346268in}{0.752336in}}%
\pgfpathlineto{\pgfqpoint{4.346940in}{0.752336in}}%
\pgfpathlineto{\pgfqpoint{4.347612in}{0.736435in}}%
\pgfpathlineto{\pgfqpoint{4.348285in}{0.770887in}}%
\pgfpathlineto{\pgfqpoint{4.349629in}{0.770887in}}%
\pgfpathlineto{\pgfqpoint{4.350301in}{0.757636in}}%
\pgfpathlineto{\pgfqpoint{4.350973in}{0.805340in}}%
\pgfpathlineto{\pgfqpoint{4.351645in}{0.805340in}}%
\pgfpathlineto{\pgfqpoint{4.351645in}{0.752336in}}%
\pgfpathlineto{\pgfqpoint{4.352990in}{0.800039in}}%
\pgfpathlineto{\pgfqpoint{4.353662in}{0.800039in}}%
\pgfpathlineto{\pgfqpoint{4.353662in}{0.762937in}}%
\pgfpathlineto{\pgfqpoint{4.354334in}{0.815940in}}%
\pgfpathlineto{\pgfqpoint{4.355006in}{0.797389in}}%
\pgfpathlineto{\pgfqpoint{4.355678in}{0.797389in}}%
\pgfpathlineto{\pgfqpoint{4.355678in}{0.752336in}}%
\pgfpathlineto{\pgfqpoint{4.356351in}{0.823891in}}%
\pgfpathlineto{\pgfqpoint{4.357023in}{0.762937in}}%
\pgfpathlineto{\pgfqpoint{4.357695in}{0.762937in}}%
\pgfpathlineto{\pgfqpoint{4.358367in}{0.781488in}}%
\pgfpathlineto{\pgfqpoint{4.359039in}{0.768237in}}%
\pgfpathlineto{\pgfqpoint{4.360384in}{0.768237in}}%
\pgfpathlineto{\pgfqpoint{4.360384in}{0.733785in}}%
\pgfpathlineto{\pgfqpoint{4.361056in}{0.781488in}}%
\pgfpathlineto{\pgfqpoint{4.361728in}{0.765587in}}%
\pgfpathlineto{\pgfqpoint{4.362400in}{0.765587in}}%
\pgfpathlineto{\pgfqpoint{4.362400in}{0.784138in}}%
\pgfpathlineto{\pgfqpoint{4.363744in}{0.781488in}}%
\pgfpathlineto{\pgfqpoint{4.364417in}{0.781488in}}%
\pgfpathlineto{\pgfqpoint{4.365089in}{0.784138in}}%
\pgfpathlineto{\pgfqpoint{4.365761in}{0.757636in}}%
\pgfpathlineto{\pgfqpoint{4.366433in}{0.757636in}}%
\pgfpathlineto{\pgfqpoint{4.367105in}{0.754986in}}%
\pgfpathlineto{\pgfqpoint{4.367777in}{0.773537in}}%
\pgfpathlineto{\pgfqpoint{4.368450in}{0.773537in}}%
\pgfpathlineto{\pgfqpoint{4.368450in}{0.765587in}}%
\pgfpathlineto{\pgfqpoint{4.369122in}{0.789439in}}%
\pgfpathlineto{\pgfqpoint{4.369794in}{0.768237in}}%
\pgfpathlineto{\pgfqpoint{4.370466in}{0.768237in}}%
\pgfpathlineto{\pgfqpoint{4.370466in}{0.754986in}}%
\pgfpathlineto{\pgfqpoint{4.371810in}{0.765587in}}%
\pgfpathlineto{\pgfqpoint{4.372483in}{0.765587in}}%
\pgfpathlineto{\pgfqpoint{4.372483in}{0.760287in}}%
\pgfpathlineto{\pgfqpoint{4.373827in}{0.794739in}}%
\pgfpathlineto{\pgfqpoint{4.374499in}{0.794739in}}%
\pgfpathlineto{\pgfqpoint{4.374499in}{0.813290in}}%
\pgfpathlineto{\pgfqpoint{4.375843in}{0.770887in}}%
\pgfpathlineto{\pgfqpoint{4.376516in}{0.770887in}}%
\pgfpathlineto{\pgfqpoint{4.377860in}{0.739085in}}%
\pgfpathlineto{\pgfqpoint{4.378532in}{0.739085in}}%
\pgfpathlineto{\pgfqpoint{4.379877in}{0.813290in}}%
\pgfpathlineto{\pgfqpoint{4.380549in}{0.813290in}}%
\pgfpathlineto{\pgfqpoint{4.380549in}{0.776188in}}%
\pgfpathlineto{\pgfqpoint{4.381893in}{0.810640in}}%
\pgfpathlineto{\pgfqpoint{4.382565in}{0.810640in}}%
\pgfpathlineto{\pgfqpoint{4.383910in}{0.754986in}}%
\pgfpathlineto{\pgfqpoint{4.384582in}{0.754986in}}%
\pgfpathlineto{\pgfqpoint{4.385926in}{0.797389in}}%
\pgfpathlineto{\pgfqpoint{4.386598in}{0.797389in}}%
\pgfpathlineto{\pgfqpoint{4.387943in}{0.781488in}}%
\pgfpathlineto{\pgfqpoint{4.388615in}{0.781488in}}%
\pgfpathlineto{\pgfqpoint{4.388615in}{0.768237in}}%
\pgfpathlineto{\pgfqpoint{4.389287in}{0.786788in}}%
\pgfpathlineto{\pgfqpoint{4.389959in}{0.776188in}}%
\pgfpathlineto{\pgfqpoint{4.390631in}{0.776188in}}%
\pgfpathlineto{\pgfqpoint{4.391303in}{0.781488in}}%
\pgfpathlineto{\pgfqpoint{4.391976in}{0.765587in}}%
\pgfpathlineto{\pgfqpoint{4.392648in}{0.765587in}}%
\pgfpathlineto{\pgfqpoint{4.392648in}{0.810640in}}%
\pgfpathlineto{\pgfqpoint{4.393992in}{0.784138in}}%
\pgfpathlineto{\pgfqpoint{4.394664in}{0.784138in}}%
\pgfpathlineto{\pgfqpoint{4.394664in}{0.744386in}}%
\pgfpathlineto{\pgfqpoint{4.396009in}{0.784138in}}%
\pgfpathlineto{\pgfqpoint{4.397353in}{0.784138in}}%
\pgfpathlineto{\pgfqpoint{4.398697in}{0.797389in}}%
\pgfpathlineto{\pgfqpoint{4.399369in}{0.797389in}}%
\pgfpathlineto{\pgfqpoint{4.399369in}{0.765587in}}%
\pgfpathlineto{\pgfqpoint{4.400714in}{0.768237in}}%
\pgfpathlineto{\pgfqpoint{4.401386in}{0.768237in}}%
\pgfpathlineto{\pgfqpoint{4.401386in}{0.778838in}}%
\pgfpathlineto{\pgfqpoint{4.402058in}{0.749686in}}%
\pgfpathlineto{\pgfqpoint{4.402730in}{0.757636in}}%
\pgfpathlineto{\pgfqpoint{4.403402in}{0.757636in}}%
\pgfpathlineto{\pgfqpoint{4.404747in}{0.800039in}}%
\pgfpathlineto{\pgfqpoint{4.405419in}{0.800039in}}%
\pgfpathlineto{\pgfqpoint{4.406091in}{0.760287in}}%
\pgfpathlineto{\pgfqpoint{4.406763in}{0.765587in}}%
\pgfpathlineto{\pgfqpoint{4.407435in}{0.765587in}}%
\pgfpathlineto{\pgfqpoint{4.407435in}{0.752336in}}%
\pgfpathlineto{\pgfqpoint{4.408108in}{0.794739in}}%
\pgfpathlineto{\pgfqpoint{4.408780in}{0.792089in}}%
\pgfpathlineto{\pgfqpoint{4.409452in}{0.792089in}}%
\pgfpathlineto{\pgfqpoint{4.410796in}{0.762937in}}%
\pgfpathlineto{\pgfqpoint{4.411468in}{0.762937in}}%
\pgfpathlineto{\pgfqpoint{4.412141in}{0.789439in}}%
\pgfpathlineto{\pgfqpoint{4.412813in}{0.760287in}}%
\pgfpathlineto{\pgfqpoint{4.413485in}{0.760287in}}%
\pgfpathlineto{\pgfqpoint{4.414829in}{0.794739in}}%
\pgfpathlineto{\pgfqpoint{4.415501in}{0.794739in}}%
\pgfpathlineto{\pgfqpoint{4.415501in}{0.754986in}}%
\pgfpathlineto{\pgfqpoint{4.416174in}{0.797389in}}%
\pgfpathlineto{\pgfqpoint{4.416846in}{0.768237in}}%
\pgfpathlineto{\pgfqpoint{4.418190in}{0.768237in}}%
\pgfpathlineto{\pgfqpoint{4.418862in}{0.805340in}}%
\pgfpathlineto{\pgfqpoint{4.419534in}{0.784138in}}%
\pgfpathlineto{\pgfqpoint{4.420207in}{0.784138in}}%
\pgfpathlineto{\pgfqpoint{4.420207in}{0.762937in}}%
\pgfpathlineto{\pgfqpoint{4.421551in}{0.789439in}}%
\pgfpathlineto{\pgfqpoint{4.422223in}{0.789439in}}%
\pgfpathlineto{\pgfqpoint{4.422895in}{0.765587in}}%
\pgfpathlineto{\pgfqpoint{4.423567in}{0.794739in}}%
\pgfpathlineto{\pgfqpoint{4.424240in}{0.794739in}}%
\pgfpathlineto{\pgfqpoint{4.424912in}{0.749686in}}%
\pgfpathlineto{\pgfqpoint{4.425584in}{0.805340in}}%
\pgfpathlineto{\pgfqpoint{4.426256in}{0.805340in}}%
\pgfpathlineto{\pgfqpoint{4.426256in}{0.739085in}}%
\pgfpathlineto{\pgfqpoint{4.427600in}{0.776188in}}%
\pgfpathlineto{\pgfqpoint{4.428273in}{0.776188in}}%
\pgfpathlineto{\pgfqpoint{4.428945in}{0.829191in}}%
\pgfpathlineto{\pgfqpoint{4.429617in}{0.770887in}}%
\pgfpathlineto{\pgfqpoint{4.430289in}{0.770887in}}%
\pgfpathlineto{\pgfqpoint{4.430961in}{0.789439in}}%
\pgfpathlineto{\pgfqpoint{4.431633in}{0.770887in}}%
\pgfpathlineto{\pgfqpoint{4.432306in}{0.770887in}}%
\pgfpathlineto{\pgfqpoint{4.433650in}{0.807990in}}%
\pgfpathlineto{\pgfqpoint{4.434322in}{0.807990in}}%
\pgfpathlineto{\pgfqpoint{4.434322in}{0.818590in}}%
\pgfpathlineto{\pgfqpoint{4.434994in}{0.765587in}}%
\pgfpathlineto{\pgfqpoint{4.435666in}{0.770887in}}%
\pgfpathlineto{\pgfqpoint{4.436339in}{0.770887in}}%
\pgfpathlineto{\pgfqpoint{4.436339in}{0.789439in}}%
\pgfpathlineto{\pgfqpoint{4.437683in}{0.731135in}}%
\pgfpathlineto{\pgfqpoint{4.438355in}{0.731135in}}%
\pgfpathlineto{\pgfqpoint{4.438355in}{0.807990in}}%
\pgfpathlineto{\pgfqpoint{4.439699in}{0.749686in}}%
\pgfpathlineto{\pgfqpoint{4.440372in}{0.749686in}}%
\pgfpathlineto{\pgfqpoint{4.441044in}{0.786788in}}%
\pgfpathlineto{\pgfqpoint{4.441716in}{0.786788in}}%
\pgfpathlineto{\pgfqpoint{4.442388in}{0.786788in}}%
\pgfpathlineto{\pgfqpoint{4.442388in}{0.805340in}}%
\pgfpathlineto{\pgfqpoint{4.443732in}{0.757636in}}%
\pgfpathlineto{\pgfqpoint{4.444405in}{0.757636in}}%
\pgfpathlineto{\pgfqpoint{4.445077in}{0.834492in}}%
\pgfpathlineto{\pgfqpoint{4.445749in}{0.786788in}}%
\pgfpathlineto{\pgfqpoint{4.446421in}{0.786788in}}%
\pgfpathlineto{\pgfqpoint{4.446421in}{0.792089in}}%
\pgfpathlineto{\pgfqpoint{4.447765in}{0.760287in}}%
\pgfpathlineto{\pgfqpoint{4.448438in}{0.760287in}}%
\pgfpathlineto{\pgfqpoint{4.449110in}{0.805340in}}%
\pgfpathlineto{\pgfqpoint{4.449782in}{0.770887in}}%
\pgfpathlineto{\pgfqpoint{4.450454in}{0.770887in}}%
\pgfpathlineto{\pgfqpoint{4.451126in}{0.813290in}}%
\pgfpathlineto{\pgfqpoint{4.451798in}{0.813290in}}%
\pgfpathlineto{\pgfqpoint{4.452471in}{0.813290in}}%
\pgfpathlineto{\pgfqpoint{4.452471in}{0.757636in}}%
\pgfpathlineto{\pgfqpoint{4.453815in}{0.810640in}}%
\pgfpathlineto{\pgfqpoint{4.454487in}{0.810640in}}%
\pgfpathlineto{\pgfqpoint{4.454487in}{0.826541in}}%
\pgfpathlineto{\pgfqpoint{4.455159in}{0.784138in}}%
\pgfpathlineto{\pgfqpoint{4.455831in}{0.789439in}}%
\pgfpathlineto{\pgfqpoint{4.456504in}{0.789439in}}%
\pgfpathlineto{\pgfqpoint{4.456504in}{0.818590in}}%
\pgfpathlineto{\pgfqpoint{4.457848in}{0.789439in}}%
\pgfpathlineto{\pgfqpoint{4.458520in}{0.789439in}}%
\pgfpathlineto{\pgfqpoint{4.459864in}{0.818590in}}%
\pgfpathlineto{\pgfqpoint{4.460537in}{0.818590in}}%
\pgfpathlineto{\pgfqpoint{4.460537in}{0.778838in}}%
\pgfpathlineto{\pgfqpoint{4.461881in}{0.800039in}}%
\pgfpathlineto{\pgfqpoint{4.462553in}{0.800039in}}%
\pgfpathlineto{\pgfqpoint{4.463225in}{0.815940in}}%
\pgfpathlineto{\pgfqpoint{4.463897in}{0.778838in}}%
\pgfpathlineto{\pgfqpoint{4.464570in}{0.778838in}}%
\pgfpathlineto{\pgfqpoint{4.464570in}{0.800039in}}%
\pgfpathlineto{\pgfqpoint{4.465242in}{0.765587in}}%
\pgfpathlineto{\pgfqpoint{4.465914in}{0.794739in}}%
\pgfpathlineto{\pgfqpoint{4.466586in}{0.794739in}}%
\pgfpathlineto{\pgfqpoint{4.467258in}{0.823891in}}%
\pgfpathlineto{\pgfqpoint{4.467930in}{0.818590in}}%
\pgfpathlineto{\pgfqpoint{4.468603in}{0.818590in}}%
\pgfpathlineto{\pgfqpoint{4.468603in}{0.762937in}}%
\pgfpathlineto{\pgfqpoint{4.469947in}{0.784138in}}%
\pgfpathlineto{\pgfqpoint{4.470619in}{0.784138in}}%
\pgfpathlineto{\pgfqpoint{4.471963in}{0.826541in}}%
\pgfpathlineto{\pgfqpoint{4.472636in}{0.826541in}}%
\pgfpathlineto{\pgfqpoint{4.472636in}{0.802689in}}%
\pgfpathlineto{\pgfqpoint{4.473308in}{0.845092in}}%
\pgfpathlineto{\pgfqpoint{4.473980in}{0.805340in}}%
\pgfpathlineto{\pgfqpoint{4.474652in}{0.805340in}}%
\pgfpathlineto{\pgfqpoint{4.475324in}{0.757636in}}%
\pgfpathlineto{\pgfqpoint{4.475996in}{0.815940in}}%
\pgfpathlineto{\pgfqpoint{4.476669in}{0.815940in}}%
\pgfpathlineto{\pgfqpoint{4.477341in}{0.839792in}}%
\pgfpathlineto{\pgfqpoint{4.478013in}{0.839792in}}%
\pgfpathlineto{\pgfqpoint{4.478685in}{0.839792in}}%
\pgfpathlineto{\pgfqpoint{4.480029in}{0.768237in}}%
\pgfpathlineto{\pgfqpoint{4.480702in}{0.768237in}}%
\pgfpathlineto{\pgfqpoint{4.481374in}{0.847742in}}%
\pgfpathlineto{\pgfqpoint{4.482046in}{0.829191in}}%
\pgfpathlineto{\pgfqpoint{4.482718in}{0.829191in}}%
\pgfpathlineto{\pgfqpoint{4.482718in}{0.797389in}}%
\pgfpathlineto{\pgfqpoint{4.484062in}{0.813290in}}%
\pgfpathlineto{\pgfqpoint{4.485407in}{0.813290in}}%
\pgfpathlineto{\pgfqpoint{4.486079in}{0.757636in}}%
\pgfpathlineto{\pgfqpoint{4.486751in}{0.802689in}}%
\pgfpathlineto{\pgfqpoint{4.487423in}{0.802689in}}%
\pgfpathlineto{\pgfqpoint{4.487423in}{0.847742in}}%
\pgfpathlineto{\pgfqpoint{4.488095in}{0.792089in}}%
\pgfpathlineto{\pgfqpoint{4.488768in}{0.802689in}}%
\pgfpathlineto{\pgfqpoint{4.489440in}{0.802689in}}%
\pgfpathlineto{\pgfqpoint{4.489440in}{0.813290in}}%
\pgfpathlineto{\pgfqpoint{4.490784in}{0.768237in}}%
\pgfpathlineto{\pgfqpoint{4.491456in}{0.768237in}}%
\pgfpathlineto{\pgfqpoint{4.492129in}{0.829191in}}%
\pgfpathlineto{\pgfqpoint{4.492801in}{0.805340in}}%
\pgfpathlineto{\pgfqpoint{4.493473in}{0.805340in}}%
\pgfpathlineto{\pgfqpoint{4.493473in}{0.850393in}}%
\pgfpathlineto{\pgfqpoint{4.494817in}{0.842442in}}%
\pgfpathlineto{\pgfqpoint{4.495489in}{0.842442in}}%
\pgfpathlineto{\pgfqpoint{4.495489in}{0.823891in}}%
\pgfpathlineto{\pgfqpoint{4.496834in}{0.829191in}}%
\pgfpathlineto{\pgfqpoint{4.498178in}{0.829191in}}%
\pgfpathlineto{\pgfqpoint{4.498178in}{0.834492in}}%
\pgfpathlineto{\pgfqpoint{4.498850in}{0.815940in}}%
\pgfpathlineto{\pgfqpoint{4.499522in}{0.826541in}}%
\pgfpathlineto{\pgfqpoint{4.500867in}{0.826541in}}%
\pgfpathlineto{\pgfqpoint{4.502211in}{0.855693in}}%
\pgfpathlineto{\pgfqpoint{4.502883in}{0.855693in}}%
\pgfpathlineto{\pgfqpoint{4.502883in}{0.898096in}}%
\pgfpathlineto{\pgfqpoint{4.503555in}{0.797389in}}%
\pgfpathlineto{\pgfqpoint{4.504228in}{0.821241in}}%
\pgfpathlineto{\pgfqpoint{4.504900in}{0.821241in}}%
\pgfpathlineto{\pgfqpoint{4.506244in}{0.842442in}}%
\pgfpathlineto{\pgfqpoint{4.506916in}{0.842442in}}%
\pgfpathlineto{\pgfqpoint{4.506916in}{0.871594in}}%
\pgfpathlineto{\pgfqpoint{4.508261in}{0.807990in}}%
\pgfpathlineto{\pgfqpoint{4.508933in}{0.807990in}}%
\pgfpathlineto{\pgfqpoint{4.509605in}{0.850393in}}%
\pgfpathlineto{\pgfqpoint{4.510277in}{0.839792in}}%
\pgfpathlineto{\pgfqpoint{4.510949in}{0.839792in}}%
\pgfpathlineto{\pgfqpoint{4.511621in}{0.834492in}}%
\pgfpathlineto{\pgfqpoint{4.512294in}{0.847742in}}%
\pgfpathlineto{\pgfqpoint{4.512966in}{0.847742in}}%
\pgfpathlineto{\pgfqpoint{4.514310in}{0.821241in}}%
\pgfpathlineto{\pgfqpoint{4.514982in}{0.821241in}}%
\pgfpathlineto{\pgfqpoint{4.514982in}{0.831841in}}%
\pgfpathlineto{\pgfqpoint{4.516327in}{0.813290in}}%
\pgfpathlineto{\pgfqpoint{4.516999in}{0.813290in}}%
\pgfpathlineto{\pgfqpoint{4.517671in}{0.892795in}}%
\pgfpathlineto{\pgfqpoint{4.518343in}{0.847742in}}%
\pgfpathlineto{\pgfqpoint{4.519015in}{0.847742in}}%
\pgfpathlineto{\pgfqpoint{4.519687in}{0.882195in}}%
\pgfpathlineto{\pgfqpoint{4.520360in}{0.882195in}}%
\pgfpathlineto{\pgfqpoint{4.521032in}{0.882195in}}%
\pgfpathlineto{\pgfqpoint{4.521704in}{0.834492in}}%
\pgfpathlineto{\pgfqpoint{4.522376in}{0.845092in}}%
\pgfpathlineto{\pgfqpoint{4.523048in}{0.845092in}}%
\pgfpathlineto{\pgfqpoint{4.523720in}{0.837142in}}%
\pgfpathlineto{\pgfqpoint{4.524393in}{0.863644in}}%
\pgfpathlineto{\pgfqpoint{4.525065in}{0.863644in}}%
\pgfpathlineto{\pgfqpoint{4.525737in}{0.813290in}}%
\pgfpathlineto{\pgfqpoint{4.526409in}{0.863644in}}%
\pgfpathlineto{\pgfqpoint{4.527081in}{0.863644in}}%
\pgfpathlineto{\pgfqpoint{4.527081in}{0.890145in}}%
\pgfpathlineto{\pgfqpoint{4.528426in}{0.802689in}}%
\pgfpathlineto{\pgfqpoint{4.529098in}{0.802689in}}%
\pgfpathlineto{\pgfqpoint{4.529770in}{0.884845in}}%
\pgfpathlineto{\pgfqpoint{4.530442in}{0.842442in}}%
\pgfpathlineto{\pgfqpoint{4.531114in}{0.842442in}}%
\pgfpathlineto{\pgfqpoint{4.531114in}{0.858343in}}%
\pgfpathlineto{\pgfqpoint{4.531786in}{0.823891in}}%
\pgfpathlineto{\pgfqpoint{4.532459in}{0.847742in}}%
\pgfpathlineto{\pgfqpoint{4.533131in}{0.847742in}}%
\pgfpathlineto{\pgfqpoint{4.533131in}{0.813290in}}%
\pgfpathlineto{\pgfqpoint{4.533803in}{0.858343in}}%
\pgfpathlineto{\pgfqpoint{4.534475in}{0.858343in}}%
\pgfpathlineto{\pgfqpoint{4.535147in}{0.858343in}}%
\pgfpathlineto{\pgfqpoint{4.535147in}{0.874244in}}%
\pgfpathlineto{\pgfqpoint{4.536492in}{0.871594in}}%
\pgfpathlineto{\pgfqpoint{4.537164in}{0.871594in}}%
\pgfpathlineto{\pgfqpoint{4.537164in}{0.911347in}}%
\pgfpathlineto{\pgfqpoint{4.537836in}{0.829191in}}%
\pgfpathlineto{\pgfqpoint{4.538508in}{0.868944in}}%
\pgfpathlineto{\pgfqpoint{4.539180in}{0.868944in}}%
\pgfpathlineto{\pgfqpoint{4.539180in}{0.834492in}}%
\pgfpathlineto{\pgfqpoint{4.540525in}{0.903396in}}%
\pgfpathlineto{\pgfqpoint{4.541197in}{0.903396in}}%
\pgfpathlineto{\pgfqpoint{4.542541in}{0.831841in}}%
\pgfpathlineto{\pgfqpoint{4.543213in}{0.831841in}}%
\pgfpathlineto{\pgfqpoint{4.543213in}{0.879545in}}%
\pgfpathlineto{\pgfqpoint{4.544558in}{0.853043in}}%
\pgfpathlineto{\pgfqpoint{4.545230in}{0.853043in}}%
\pgfpathlineto{\pgfqpoint{4.546574in}{0.903396in}}%
\pgfpathlineto{\pgfqpoint{4.547246in}{0.903396in}}%
\pgfpathlineto{\pgfqpoint{4.548591in}{0.850393in}}%
\pgfpathlineto{\pgfqpoint{4.549263in}{0.850393in}}%
\pgfpathlineto{\pgfqpoint{4.549263in}{0.929898in}}%
\pgfpathlineto{\pgfqpoint{4.550607in}{0.903396in}}%
\pgfpathlineto{\pgfqpoint{4.551279in}{0.903396in}}%
\pgfpathlineto{\pgfqpoint{4.551279in}{0.884845in}}%
\pgfpathlineto{\pgfqpoint{4.552624in}{0.924598in}}%
\pgfpathlineto{\pgfqpoint{4.553296in}{0.924598in}}%
\pgfpathlineto{\pgfqpoint{4.553968in}{0.876894in}}%
\pgfpathlineto{\pgfqpoint{4.554640in}{0.903396in}}%
\pgfpathlineto{\pgfqpoint{4.555312in}{0.903396in}}%
\pgfpathlineto{\pgfqpoint{4.555312in}{0.890145in}}%
\pgfpathlineto{\pgfqpoint{4.555984in}{0.929898in}}%
\pgfpathlineto{\pgfqpoint{4.556657in}{0.903396in}}%
\pgfpathlineto{\pgfqpoint{4.557329in}{0.903396in}}%
\pgfpathlineto{\pgfqpoint{4.558001in}{0.929898in}}%
\pgfpathlineto{\pgfqpoint{4.558673in}{0.879545in}}%
\pgfpathlineto{\pgfqpoint{4.559345in}{0.879545in}}%
\pgfpathlineto{\pgfqpoint{4.560017in}{0.874244in}}%
\pgfpathlineto{\pgfqpoint{4.560690in}{0.916647in}}%
\pgfpathlineto{\pgfqpoint{4.561362in}{0.916647in}}%
\pgfpathlineto{\pgfqpoint{4.561362in}{0.876894in}}%
\pgfpathlineto{\pgfqpoint{4.562034in}{0.945799in}}%
\pgfpathlineto{\pgfqpoint{4.562706in}{0.895446in}}%
\pgfpathlineto{\pgfqpoint{4.563378in}{0.895446in}}%
\pgfpathlineto{\pgfqpoint{4.564050in}{0.940499in}}%
\pgfpathlineto{\pgfqpoint{4.564723in}{0.908697in}}%
\pgfpathlineto{\pgfqpoint{4.565395in}{0.908697in}}%
\pgfpathlineto{\pgfqpoint{4.566739in}{0.890145in}}%
\pgfpathlineto{\pgfqpoint{4.567411in}{0.890145in}}%
\pgfpathlineto{\pgfqpoint{4.567411in}{0.882195in}}%
\pgfpathlineto{\pgfqpoint{4.568083in}{0.913997in}}%
\pgfpathlineto{\pgfqpoint{4.568756in}{0.884845in}}%
\pgfpathlineto{\pgfqpoint{4.569428in}{0.884845in}}%
\pgfpathlineto{\pgfqpoint{4.570100in}{0.940499in}}%
\pgfpathlineto{\pgfqpoint{4.570772in}{0.879545in}}%
\pgfpathlineto{\pgfqpoint{4.571444in}{0.879545in}}%
\pgfpathlineto{\pgfqpoint{4.572116in}{0.951099in}}%
\pgfpathlineto{\pgfqpoint{4.572789in}{0.935198in}}%
\pgfpathlineto{\pgfqpoint{4.573461in}{0.935198in}}%
\pgfpathlineto{\pgfqpoint{4.573461in}{0.868944in}}%
\pgfpathlineto{\pgfqpoint{4.574805in}{1.001453in}}%
\pgfpathlineto{\pgfqpoint{4.575477in}{1.001453in}}%
\pgfpathlineto{\pgfqpoint{4.575477in}{0.908697in}}%
\pgfpathlineto{\pgfqpoint{4.576822in}{0.951099in}}%
\pgfpathlineto{\pgfqpoint{4.577494in}{0.951099in}}%
\pgfpathlineto{\pgfqpoint{4.577494in}{0.985552in}}%
\pgfpathlineto{\pgfqpoint{4.578166in}{0.911347in}}%
\pgfpathlineto{\pgfqpoint{4.578838in}{0.921947in}}%
\pgfpathlineto{\pgfqpoint{4.579510in}{0.921947in}}%
\pgfpathlineto{\pgfqpoint{4.580855in}{0.956400in}}%
\pgfpathlineto{\pgfqpoint{4.581527in}{0.956400in}}%
\pgfpathlineto{\pgfqpoint{4.581527in}{0.985552in}}%
\pgfpathlineto{\pgfqpoint{4.582199in}{0.943149in}}%
\pgfpathlineto{\pgfqpoint{4.582871in}{0.953750in}}%
\pgfpathlineto{\pgfqpoint{4.583543in}{0.953750in}}%
\pgfpathlineto{\pgfqpoint{4.584215in}{0.993502in}}%
\pgfpathlineto{\pgfqpoint{4.584888in}{0.892795in}}%
\pgfpathlineto{\pgfqpoint{4.585560in}{0.892795in}}%
\pgfpathlineto{\pgfqpoint{4.586232in}{0.982901in}}%
\pgfpathlineto{\pgfqpoint{4.586904in}{0.935198in}}%
\pgfpathlineto{\pgfqpoint{4.587576in}{0.935198in}}%
\pgfpathlineto{\pgfqpoint{4.588921in}{0.988202in}}%
\pgfpathlineto{\pgfqpoint{4.589593in}{0.988202in}}%
\pgfpathlineto{\pgfqpoint{4.589593in}{0.919297in}}%
\pgfpathlineto{\pgfqpoint{4.590937in}{0.977601in}}%
\pgfpathlineto{\pgfqpoint{4.591609in}{0.977601in}}%
\pgfpathlineto{\pgfqpoint{4.591609in}{0.956400in}}%
\pgfpathlineto{\pgfqpoint{4.592954in}{0.996152in}}%
\pgfpathlineto{\pgfqpoint{4.593626in}{0.996152in}}%
\pgfpathlineto{\pgfqpoint{4.593626in}{1.012053in}}%
\pgfpathlineto{\pgfqpoint{4.594970in}{0.974951in}}%
\pgfpathlineto{\pgfqpoint{4.595642in}{0.974951in}}%
\pgfpathlineto{\pgfqpoint{4.596314in}{0.953750in}}%
\pgfpathlineto{\pgfqpoint{4.596987in}{1.059757in}}%
\pgfpathlineto{\pgfqpoint{4.597659in}{1.059757in}}%
\pgfpathlineto{\pgfqpoint{4.598331in}{0.959050in}}%
\pgfpathlineto{\pgfqpoint{4.599003in}{0.969651in}}%
\pgfpathlineto{\pgfqpoint{4.599675in}{0.969651in}}%
\pgfpathlineto{\pgfqpoint{4.599675in}{1.001453in}}%
\pgfpathlineto{\pgfqpoint{4.601020in}{0.990852in}}%
\pgfpathlineto{\pgfqpoint{4.601692in}{0.990852in}}%
\pgfpathlineto{\pgfqpoint{4.602364in}{0.985552in}}%
\pgfpathlineto{\pgfqpoint{4.603036in}{1.075658in}}%
\pgfpathlineto{\pgfqpoint{4.603708in}{1.075658in}}%
\pgfpathlineto{\pgfqpoint{4.604381in}{0.993502in}}%
\pgfpathlineto{\pgfqpoint{4.605053in}{1.070357in}}%
\pgfpathlineto{\pgfqpoint{4.605725in}{1.070357in}}%
\pgfpathlineto{\pgfqpoint{4.607069in}{0.974951in}}%
\pgfpathlineto{\pgfqpoint{4.607741in}{0.974951in}}%
\pgfpathlineto{\pgfqpoint{4.608414in}{1.062407in}}%
\pgfpathlineto{\pgfqpoint{4.609086in}{1.014704in}}%
\pgfpathlineto{\pgfqpoint{4.609758in}{1.014704in}}%
\pgfpathlineto{\pgfqpoint{4.609758in}{0.993502in}}%
\pgfpathlineto{\pgfqpoint{4.610430in}{1.027954in}}%
\pgfpathlineto{\pgfqpoint{4.611102in}{0.993502in}}%
\pgfpathlineto{\pgfqpoint{4.611774in}{0.993502in}}%
\pgfpathlineto{\pgfqpoint{4.612447in}{1.009403in}}%
\pgfpathlineto{\pgfqpoint{4.613119in}{0.998803in}}%
\pgfpathlineto{\pgfqpoint{4.613791in}{0.998803in}}%
\pgfpathlineto{\pgfqpoint{4.614463in}{1.022654in}}%
\pgfpathlineto{\pgfqpoint{4.615135in}{0.988202in}}%
\pgfpathlineto{\pgfqpoint{4.615807in}{0.988202in}}%
\pgfpathlineto{\pgfqpoint{4.617152in}{1.083608in}}%
\pgfpathlineto{\pgfqpoint{4.617824in}{1.083608in}}%
\pgfpathlineto{\pgfqpoint{4.617824in}{1.128661in}}%
\pgfpathlineto{\pgfqpoint{4.618496in}{1.046506in}}%
\pgfpathlineto{\pgfqpoint{4.619168in}{1.067707in}}%
\pgfpathlineto{\pgfqpoint{4.619840in}{1.067707in}}%
\pgfpathlineto{\pgfqpoint{4.619840in}{1.027954in}}%
\pgfpathlineto{\pgfqpoint{4.620513in}{1.149863in}}%
\pgfpathlineto{\pgfqpoint{4.621185in}{1.083608in}}%
\pgfpathlineto{\pgfqpoint{4.621857in}{1.083608in}}%
\pgfpathlineto{\pgfqpoint{4.623201in}{1.049156in}}%
\pgfpathlineto{\pgfqpoint{4.623873in}{1.049156in}}%
\pgfpathlineto{\pgfqpoint{4.624546in}{1.144562in}}%
\pgfpathlineto{\pgfqpoint{4.625218in}{1.057106in}}%
\pgfpathlineto{\pgfqpoint{4.625890in}{1.057106in}}%
\pgfpathlineto{\pgfqpoint{4.626562in}{1.025304in}}%
\pgfpathlineto{\pgfqpoint{4.627234in}{1.059757in}}%
\pgfpathlineto{\pgfqpoint{4.627906in}{1.059757in}}%
\pgfpathlineto{\pgfqpoint{4.628579in}{1.035905in}}%
\pgfpathlineto{\pgfqpoint{4.629251in}{1.147212in}}%
\pgfpathlineto{\pgfqpoint{4.629923in}{1.147212in}}%
\pgfpathlineto{\pgfqpoint{4.629923in}{1.041205in}}%
\pgfpathlineto{\pgfqpoint{4.631267in}{1.075658in}}%
\pgfpathlineto{\pgfqpoint{4.631939in}{1.075658in}}%
\pgfpathlineto{\pgfqpoint{4.631939in}{1.094209in}}%
\pgfpathlineto{\pgfqpoint{4.633284in}{1.088909in}}%
\pgfpathlineto{\pgfqpoint{4.633956in}{1.088909in}}%
\pgfpathlineto{\pgfqpoint{4.633956in}{1.110110in}}%
\pgfpathlineto{\pgfqpoint{4.634628in}{1.059757in}}%
\pgfpathlineto{\pgfqpoint{4.635300in}{1.107460in}}%
\pgfpathlineto{\pgfqpoint{4.635972in}{1.107460in}}%
\pgfpathlineto{\pgfqpoint{4.636645in}{1.099509in}}%
\pgfpathlineto{\pgfqpoint{4.637317in}{1.128661in}}%
\pgfpathlineto{\pgfqpoint{4.637989in}{1.128661in}}%
\pgfpathlineto{\pgfqpoint{4.637989in}{1.067707in}}%
\pgfpathlineto{\pgfqpoint{4.639333in}{1.192265in}}%
\pgfpathlineto{\pgfqpoint{4.640005in}{1.192265in}}%
\pgfpathlineto{\pgfqpoint{4.640678in}{1.099509in}}%
\pgfpathlineto{\pgfqpoint{4.641350in}{1.118060in}}%
\pgfpathlineto{\pgfqpoint{4.642022in}{1.118060in}}%
\pgfpathlineto{\pgfqpoint{4.642022in}{1.226718in}}%
\pgfpathlineto{\pgfqpoint{4.643366in}{1.104810in}}%
\pgfpathlineto{\pgfqpoint{4.644038in}{1.104810in}}%
\pgfpathlineto{\pgfqpoint{4.644038in}{1.083608in}}%
\pgfpathlineto{\pgfqpoint{4.645383in}{1.157813in}}%
\pgfpathlineto{\pgfqpoint{4.646055in}{1.157813in}}%
\pgfpathlineto{\pgfqpoint{4.646055in}{1.126011in}}%
\pgfpathlineto{\pgfqpoint{4.646727in}{1.247919in}}%
\pgfpathlineto{\pgfqpoint{4.647399in}{1.221417in}}%
\pgfpathlineto{\pgfqpoint{4.648071in}{1.221417in}}%
\pgfpathlineto{\pgfqpoint{4.648071in}{1.133962in}}%
\pgfpathlineto{\pgfqpoint{4.649416in}{1.197566in}}%
\pgfpathlineto{\pgfqpoint{4.650088in}{1.197566in}}%
\pgfpathlineto{\pgfqpoint{4.650760in}{1.126011in}}%
\pgfpathlineto{\pgfqpoint{4.651432in}{1.186965in}}%
\pgfpathlineto{\pgfqpoint{4.652104in}{1.186965in}}%
\pgfpathlineto{\pgfqpoint{4.652777in}{1.229368in}}%
\pgfpathlineto{\pgfqpoint{4.653449in}{1.136612in}}%
\pgfpathlineto{\pgfqpoint{4.654121in}{1.136612in}}%
\pgfpathlineto{\pgfqpoint{4.654793in}{1.176364in}}%
\pgfpathlineto{\pgfqpoint{4.655465in}{1.144562in}}%
\pgfpathlineto{\pgfqpoint{4.656137in}{1.144562in}}%
\pgfpathlineto{\pgfqpoint{4.656137in}{1.126011in}}%
\pgfpathlineto{\pgfqpoint{4.657482in}{1.163113in}}%
\pgfpathlineto{\pgfqpoint{4.658154in}{1.163113in}}%
\pgfpathlineto{\pgfqpoint{4.659498in}{1.245269in}}%
\pgfpathlineto{\pgfqpoint{4.660170in}{1.245269in}}%
\pgfpathlineto{\pgfqpoint{4.660170in}{1.200216in}}%
\pgfpathlineto{\pgfqpoint{4.661515in}{1.250569in}}%
\pgfpathlineto{\pgfqpoint{4.662187in}{1.250569in}}%
\pgfpathlineto{\pgfqpoint{4.662187in}{1.263820in}}%
\pgfpathlineto{\pgfqpoint{4.663531in}{1.242619in}}%
\pgfpathlineto{\pgfqpoint{4.664203in}{1.242619in}}%
\pgfpathlineto{\pgfqpoint{4.664876in}{1.274421in}}%
\pgfpathlineto{\pgfqpoint{4.665548in}{1.216117in}}%
\pgfpathlineto{\pgfqpoint{4.666220in}{1.216117in}}%
\pgfpathlineto{\pgfqpoint{4.666220in}{1.250569in}}%
\pgfpathlineto{\pgfqpoint{4.666892in}{1.176364in}}%
\pgfpathlineto{\pgfqpoint{4.667564in}{1.213467in}}%
\pgfpathlineto{\pgfqpoint{4.668236in}{1.213467in}}%
\pgfpathlineto{\pgfqpoint{4.668909in}{1.277071in}}%
\pgfpathlineto{\pgfqpoint{4.669581in}{1.250569in}}%
\pgfpathlineto{\pgfqpoint{4.670253in}{1.250569in}}%
\pgfpathlineto{\pgfqpoint{4.670925in}{1.242619in}}%
\pgfpathlineto{\pgfqpoint{4.671597in}{1.306223in}}%
\pgfpathlineto{\pgfqpoint{4.672269in}{1.306223in}}%
\pgfpathlineto{\pgfqpoint{4.672942in}{1.279721in}}%
\pgfpathlineto{\pgfqpoint{4.673614in}{1.311523in}}%
\pgfpathlineto{\pgfqpoint{4.674286in}{1.311523in}}%
\pgfpathlineto{\pgfqpoint{4.674958in}{1.229368in}}%
\pgfpathlineto{\pgfqpoint{4.675630in}{1.290322in}}%
\pgfpathlineto{\pgfqpoint{4.676302in}{1.290322in}}%
\pgfpathlineto{\pgfqpoint{4.676302in}{1.194916in}}%
\pgfpathlineto{\pgfqpoint{4.676975in}{1.303573in}}%
\pgfpathlineto{\pgfqpoint{4.677647in}{1.261170in}}%
\pgfpathlineto{\pgfqpoint{4.678319in}{1.261170in}}%
\pgfpathlineto{\pgfqpoint{4.678319in}{1.234668in}}%
\pgfpathlineto{\pgfqpoint{4.678991in}{1.303573in}}%
\pgfpathlineto{\pgfqpoint{4.679663in}{1.303573in}}%
\pgfpathlineto{\pgfqpoint{4.680335in}{1.303573in}}%
\pgfpathlineto{\pgfqpoint{4.681008in}{1.208166in}}%
\pgfpathlineto{\pgfqpoint{4.681680in}{1.330075in}}%
\pgfpathlineto{\pgfqpoint{4.682352in}{1.330075in}}%
\pgfpathlineto{\pgfqpoint{4.682352in}{1.255870in}}%
\pgfpathlineto{\pgfqpoint{4.683696in}{1.308873in}}%
\pgfpathlineto{\pgfqpoint{4.684368in}{1.308873in}}%
\pgfpathlineto{\pgfqpoint{4.685041in}{1.330075in}}%
\pgfpathlineto{\pgfqpoint{4.685713in}{1.245269in}}%
\pgfpathlineto{\pgfqpoint{4.686385in}{1.245269in}}%
\pgfpathlineto{\pgfqpoint{4.687057in}{1.335375in}}%
\pgfpathlineto{\pgfqpoint{4.687729in}{1.271771in}}%
\pgfpathlineto{\pgfqpoint{4.688401in}{1.271771in}}%
\pgfpathlineto{\pgfqpoint{4.688401in}{1.396329in}}%
\pgfpathlineto{\pgfqpoint{4.689746in}{1.345976in}}%
\pgfpathlineto{\pgfqpoint{4.690418in}{1.345976in}}%
\pgfpathlineto{\pgfqpoint{4.690418in}{1.287672in}}%
\pgfpathlineto{\pgfqpoint{4.691762in}{1.345976in}}%
\pgfpathlineto{\pgfqpoint{4.692434in}{1.345976in}}%
\pgfpathlineto{\pgfqpoint{4.693107in}{1.364527in}}%
\pgfpathlineto{\pgfqpoint{4.693779in}{1.285022in}}%
\pgfpathlineto{\pgfqpoint{4.694451in}{1.285022in}}%
\pgfpathlineto{\pgfqpoint{4.695123in}{1.422831in}}%
\pgfpathlineto{\pgfqpoint{4.695795in}{1.375128in}}%
\pgfpathlineto{\pgfqpoint{4.696467in}{1.375128in}}%
\pgfpathlineto{\pgfqpoint{4.696467in}{1.242619in}}%
\pgfpathlineto{\pgfqpoint{4.697812in}{1.367177in}}%
\pgfpathlineto{\pgfqpoint{4.698484in}{1.367177in}}%
\pgfpathlineto{\pgfqpoint{4.698484in}{1.290322in}}%
\pgfpathlineto{\pgfqpoint{4.699828in}{1.422831in}}%
\pgfpathlineto{\pgfqpoint{4.700500in}{1.422831in}}%
\pgfpathlineto{\pgfqpoint{4.701845in}{1.314174in}}%
\pgfpathlineto{\pgfqpoint{4.702517in}{1.314174in}}%
\pgfpathlineto{\pgfqpoint{4.703861in}{1.361877in}}%
\pgfpathlineto{\pgfqpoint{4.704533in}{1.361877in}}%
\pgfpathlineto{\pgfqpoint{4.704533in}{1.430781in}}%
\pgfpathlineto{\pgfqpoint{4.705878in}{1.295622in}}%
\pgfpathlineto{\pgfqpoint{4.706550in}{1.295622in}}%
\pgfpathlineto{\pgfqpoint{4.707894in}{1.465234in}}%
\pgfpathlineto{\pgfqpoint{4.708566in}{1.465234in}}%
\pgfpathlineto{\pgfqpoint{4.708566in}{1.319474in}}%
\pgfpathlineto{\pgfqpoint{4.709911in}{1.398979in}}%
\pgfpathlineto{\pgfqpoint{4.710583in}{1.398979in}}%
\pgfpathlineto{\pgfqpoint{4.710583in}{1.359227in}}%
\pgfpathlineto{\pgfqpoint{4.711927in}{1.449333in}}%
\pgfpathlineto{\pgfqpoint{4.712599in}{1.449333in}}%
\pgfpathlineto{\pgfqpoint{4.713944in}{1.388378in}}%
\pgfpathlineto{\pgfqpoint{4.714616in}{1.388378in}}%
\pgfpathlineto{\pgfqpoint{4.714616in}{1.428131in}}%
\pgfpathlineto{\pgfqpoint{4.715960in}{1.372477in}}%
\pgfpathlineto{\pgfqpoint{4.716633in}{1.372477in}}%
\pgfpathlineto{\pgfqpoint{4.717305in}{1.459933in}}%
\pgfpathlineto{\pgfqpoint{4.717977in}{1.404280in}}%
\pgfpathlineto{\pgfqpoint{4.718649in}{1.404280in}}%
\pgfpathlineto{\pgfqpoint{4.719321in}{1.356576in}}%
\pgfpathlineto{\pgfqpoint{4.719993in}{1.422831in}}%
\pgfpathlineto{\pgfqpoint{4.720666in}{1.422831in}}%
\pgfpathlineto{\pgfqpoint{4.720666in}{1.414880in}}%
\pgfpathlineto{\pgfqpoint{4.722010in}{1.420181in}}%
\pgfpathlineto{\pgfqpoint{4.722682in}{1.420181in}}%
\pgfpathlineto{\pgfqpoint{4.723354in}{1.481135in}}%
\pgfpathlineto{\pgfqpoint{4.724026in}{1.338025in}}%
\pgfpathlineto{\pgfqpoint{4.724699in}{1.338025in}}%
\pgfpathlineto{\pgfqpoint{4.724699in}{1.475834in}}%
\pgfpathlineto{\pgfqpoint{4.726043in}{1.446682in}}%
\pgfpathlineto{\pgfqpoint{4.726715in}{1.446682in}}%
\pgfpathlineto{\pgfqpoint{4.726715in}{1.475834in}}%
\pgfpathlineto{\pgfqpoint{4.727387in}{1.425481in}}%
\pgfpathlineto{\pgfqpoint{4.728059in}{1.446682in}}%
\pgfpathlineto{\pgfqpoint{4.728732in}{1.446682in}}%
\pgfpathlineto{\pgfqpoint{4.728732in}{1.557990in}}%
\pgfpathlineto{\pgfqpoint{4.730076in}{1.422831in}}%
\pgfpathlineto{\pgfqpoint{4.730748in}{1.422831in}}%
\pgfpathlineto{\pgfqpoint{4.732092in}{1.597742in}}%
\pgfpathlineto{\pgfqpoint{4.732765in}{1.597742in}}%
\pgfpathlineto{\pgfqpoint{4.732765in}{1.430781in}}%
\pgfpathlineto{\pgfqpoint{4.734109in}{1.542089in}}%
\pgfpathlineto{\pgfqpoint{4.734781in}{1.542089in}}%
\pgfpathlineto{\pgfqpoint{4.736125in}{1.409580in}}%
\pgfpathlineto{\pgfqpoint{4.736798in}{1.409580in}}%
\pgfpathlineto{\pgfqpoint{4.736798in}{1.573891in}}%
\pgfpathlineto{\pgfqpoint{4.737470in}{1.391029in}}%
\pgfpathlineto{\pgfqpoint{4.738142in}{1.547389in}}%
\pgfpathlineto{\pgfqpoint{4.738814in}{1.547389in}}%
\pgfpathlineto{\pgfqpoint{4.738814in}{1.581841in}}%
\pgfpathlineto{\pgfqpoint{4.740158in}{1.454633in}}%
\pgfpathlineto{\pgfqpoint{4.740831in}{1.454633in}}%
\pgfpathlineto{\pgfqpoint{4.741503in}{1.497036in}}%
\pgfpathlineto{\pgfqpoint{4.742175in}{1.489085in}}%
\pgfpathlineto{\pgfqpoint{4.742847in}{1.489085in}}%
\pgfpathlineto{\pgfqpoint{4.742847in}{1.433431in}}%
\pgfpathlineto{\pgfqpoint{4.744191in}{1.626894in}}%
\pgfpathlineto{\pgfqpoint{4.744864in}{1.626894in}}%
\pgfpathlineto{\pgfqpoint{4.744864in}{1.483785in}}%
\pgfpathlineto{\pgfqpoint{4.746208in}{1.552689in}}%
\pgfpathlineto{\pgfqpoint{4.746880in}{1.552689in}}%
\pgfpathlineto{\pgfqpoint{4.747552in}{1.449333in}}%
\pgfpathlineto{\pgfqpoint{4.747552in}{1.571241in}}%
\pgfpathlineto{\pgfqpoint{4.748224in}{1.483785in}}%
\pgfpathlineto{\pgfqpoint{4.748897in}{1.483785in}}%
\pgfpathlineto{\pgfqpoint{4.750241in}{1.603043in}}%
\pgfpathlineto{\pgfqpoint{4.750913in}{1.603043in}}%
\pgfpathlineto{\pgfqpoint{4.750913in}{1.459933in}}%
\pgfpathlineto{\pgfqpoint{4.752257in}{1.491735in}}%
\pgfpathlineto{\pgfqpoint{4.752930in}{1.491735in}}%
\pgfpathlineto{\pgfqpoint{4.752930in}{1.483785in}}%
\pgfpathlineto{\pgfqpoint{4.753602in}{1.595092in}}%
\pgfpathlineto{\pgfqpoint{4.754274in}{1.489085in}}%
\pgfpathlineto{\pgfqpoint{4.754946in}{1.489085in}}%
\pgfpathlineto{\pgfqpoint{4.754946in}{1.576541in}}%
\pgfpathlineto{\pgfqpoint{4.756290in}{1.523537in}}%
\pgfpathlineto{\pgfqpoint{4.756963in}{1.523537in}}%
\pgfpathlineto{\pgfqpoint{4.756963in}{1.478484in}}%
\pgfpathlineto{\pgfqpoint{4.758307in}{1.571241in}}%
\pgfpathlineto{\pgfqpoint{4.758979in}{1.571241in}}%
\pgfpathlineto{\pgfqpoint{4.759651in}{1.605693in}}%
\pgfpathlineto{\pgfqpoint{4.760323in}{1.467884in}}%
\pgfpathlineto{\pgfqpoint{4.760996in}{1.467884in}}%
\pgfpathlineto{\pgfqpoint{4.762340in}{1.690499in}}%
\pgfpathlineto{\pgfqpoint{4.763012in}{1.690499in}}%
\pgfpathlineto{\pgfqpoint{4.763684in}{1.542089in}}%
\pgfpathlineto{\pgfqpoint{4.764356in}{1.547389in}}%
\pgfpathlineto{\pgfqpoint{4.765029in}{1.547389in}}%
\pgfpathlineto{\pgfqpoint{4.765701in}{1.653396in}}%
\pgfpathlineto{\pgfqpoint{4.766373in}{1.587142in}}%
\pgfpathlineto{\pgfqpoint{4.767045in}{1.587142in}}%
\pgfpathlineto{\pgfqpoint{4.767717in}{1.497036in}}%
\pgfpathlineto{\pgfqpoint{4.768389in}{1.637495in}}%
\pgfpathlineto{\pgfqpoint{4.769062in}{1.637495in}}%
\pgfpathlineto{\pgfqpoint{4.770406in}{1.589792in}}%
\pgfpathlineto{\pgfqpoint{4.771078in}{1.589792in}}%
\pgfpathlineto{\pgfqpoint{4.771078in}{1.475834in}}%
\pgfpathlineto{\pgfqpoint{4.771750in}{1.634845in}}%
\pgfpathlineto{\pgfqpoint{4.772422in}{1.571241in}}%
\pgfpathlineto{\pgfqpoint{4.773767in}{1.571241in}}%
\pgfpathlineto{\pgfqpoint{4.773767in}{1.581841in}}%
\pgfpathlineto{\pgfqpoint{4.774439in}{1.512937in}}%
\pgfpathlineto{\pgfqpoint{4.775111in}{1.557990in}}%
\pgfpathlineto{\pgfqpoint{4.775783in}{1.557990in}}%
\pgfpathlineto{\pgfqpoint{4.775783in}{1.536788in}}%
\pgfpathlineto{\pgfqpoint{4.777128in}{1.730251in}}%
\pgfpathlineto{\pgfqpoint{4.777800in}{1.730251in}}%
\pgfpathlineto{\pgfqpoint{4.778472in}{1.536788in}}%
\pgfpathlineto{\pgfqpoint{4.779144in}{1.579191in}}%
\pgfpathlineto{\pgfqpoint{4.779816in}{1.579191in}}%
\pgfpathlineto{\pgfqpoint{4.779816in}{1.645446in}}%
\pgfpathlineto{\pgfqpoint{4.780488in}{1.568590in}}%
\pgfpathlineto{\pgfqpoint{4.781161in}{1.618944in}}%
\pgfpathlineto{\pgfqpoint{4.781833in}{1.618944in}}%
\pgfpathlineto{\pgfqpoint{4.781833in}{1.565940in}}%
\pgfpathlineto{\pgfqpoint{4.782505in}{1.682548in}}%
\pgfpathlineto{\pgfqpoint{4.783177in}{1.610993in}}%
\pgfpathlineto{\pgfqpoint{4.783849in}{1.610993in}}%
\pgfpathlineto{\pgfqpoint{4.783849in}{1.584492in}}%
\pgfpathlineto{\pgfqpoint{4.784521in}{1.685198in}}%
\pgfpathlineto{\pgfqpoint{4.785194in}{1.595092in}}%
\pgfpathlineto{\pgfqpoint{4.785866in}{1.595092in}}%
\pgfpathlineto{\pgfqpoint{4.785866in}{1.518237in}}%
\pgfpathlineto{\pgfqpoint{4.786538in}{1.645446in}}%
\pgfpathlineto{\pgfqpoint{4.787210in}{1.632195in}}%
\pgfpathlineto{\pgfqpoint{4.787882in}{1.632195in}}%
\pgfpathlineto{\pgfqpoint{4.788554in}{1.640145in}}%
\pgfpathlineto{\pgfqpoint{4.789227in}{1.597742in}}%
\pgfpathlineto{\pgfqpoint{4.789899in}{1.597742in}}%
\pgfpathlineto{\pgfqpoint{4.790571in}{1.677248in}}%
\pgfpathlineto{\pgfqpoint{4.791243in}{1.528838in}}%
\pgfpathlineto{\pgfqpoint{4.791915in}{1.528838in}}%
\pgfpathlineto{\pgfqpoint{4.793260in}{1.621594in}}%
\pgfpathlineto{\pgfqpoint{4.793932in}{1.621594in}}%
\pgfpathlineto{\pgfqpoint{4.793932in}{1.579191in}}%
\pgfpathlineto{\pgfqpoint{4.795276in}{1.640145in}}%
\pgfpathlineto{\pgfqpoint{4.796620in}{1.640145in}}%
\pgfpathlineto{\pgfqpoint{4.797965in}{1.557990in}}%
\pgfpathlineto{\pgfqpoint{4.798637in}{1.557990in}}%
\pgfpathlineto{\pgfqpoint{4.799309in}{1.502336in}}%
\pgfpathlineto{\pgfqpoint{4.799981in}{1.666647in}}%
\pgfpathlineto{\pgfqpoint{4.800653in}{1.666647in}}%
\pgfpathlineto{\pgfqpoint{4.801998in}{1.473184in}}%
\pgfpathlineto{\pgfqpoint{4.802670in}{1.473184in}}%
\pgfpathlineto{\pgfqpoint{4.802670in}{1.587142in}}%
\pgfpathlineto{\pgfqpoint{4.804014in}{1.573891in}}%
\pgfpathlineto{\pgfqpoint{4.804686in}{1.573891in}}%
\pgfpathlineto{\pgfqpoint{4.804686in}{1.560640in}}%
\pgfpathlineto{\pgfqpoint{4.805359in}{1.613644in}}%
\pgfpathlineto{\pgfqpoint{4.806031in}{1.592442in}}%
\pgfpathlineto{\pgfqpoint{4.806703in}{1.592442in}}%
\pgfpathlineto{\pgfqpoint{4.806703in}{1.475834in}}%
\pgfpathlineto{\pgfqpoint{4.807375in}{1.603043in}}%
\pgfpathlineto{\pgfqpoint{4.808047in}{1.507636in}}%
\pgfpathlineto{\pgfqpoint{4.808719in}{1.507636in}}%
\pgfpathlineto{\pgfqpoint{4.808719in}{1.656046in}}%
\pgfpathlineto{\pgfqpoint{4.810064in}{1.608343in}}%
\pgfpathlineto{\pgfqpoint{4.810736in}{1.608343in}}%
\pgfpathlineto{\pgfqpoint{4.810736in}{1.645446in}}%
\pgfpathlineto{\pgfqpoint{4.812080in}{1.518237in}}%
\pgfpathlineto{\pgfqpoint{4.812752in}{1.518237in}}%
\pgfpathlineto{\pgfqpoint{4.812752in}{1.661347in}}%
\pgfpathlineto{\pgfqpoint{4.814097in}{1.563290in}}%
\pgfpathlineto{\pgfqpoint{4.814769in}{1.563290in}}%
\pgfpathlineto{\pgfqpoint{4.814769in}{1.663997in}}%
\pgfpathlineto{\pgfqpoint{4.816113in}{1.587142in}}%
\pgfpathlineto{\pgfqpoint{4.816785in}{1.587142in}}%
\pgfpathlineto{\pgfqpoint{4.816785in}{1.573891in}}%
\pgfpathlineto{\pgfqpoint{4.817458in}{1.621594in}}%
\pgfpathlineto{\pgfqpoint{4.818130in}{1.608343in}}%
\pgfpathlineto{\pgfqpoint{4.818802in}{1.608343in}}%
\pgfpathlineto{\pgfqpoint{4.819474in}{1.661347in}}%
\pgfpathlineto{\pgfqpoint{4.820146in}{1.581841in}}%
\pgfpathlineto{\pgfqpoint{4.820818in}{1.581841in}}%
\pgfpathlineto{\pgfqpoint{4.821491in}{1.555340in}}%
\pgfpathlineto{\pgfqpoint{4.822163in}{1.621594in}}%
\pgfpathlineto{\pgfqpoint{4.822835in}{1.621594in}}%
\pgfpathlineto{\pgfqpoint{4.823507in}{1.478484in}}%
\pgfpathlineto{\pgfqpoint{4.824179in}{1.626894in}}%
\pgfpathlineto{\pgfqpoint{4.824851in}{1.626894in}}%
\pgfpathlineto{\pgfqpoint{4.825524in}{1.563290in}}%
\pgfpathlineto{\pgfqpoint{4.826196in}{1.650746in}}%
\pgfpathlineto{\pgfqpoint{4.826868in}{1.650746in}}%
\pgfpathlineto{\pgfqpoint{4.826868in}{1.679898in}}%
\pgfpathlineto{\pgfqpoint{4.828212in}{1.497036in}}%
\pgfpathlineto{\pgfqpoint{4.828884in}{1.497036in}}%
\pgfpathlineto{\pgfqpoint{4.829557in}{1.677248in}}%
\pgfpathlineto{\pgfqpoint{4.830229in}{1.539439in}}%
\pgfpathlineto{\pgfqpoint{4.830901in}{1.539439in}}%
\pgfpathlineto{\pgfqpoint{4.830901in}{1.557990in}}%
\pgfpathlineto{\pgfqpoint{4.831573in}{1.504986in}}%
\pgfpathlineto{\pgfqpoint{4.832245in}{1.531488in}}%
\pgfpathlineto{\pgfqpoint{4.832918in}{1.531488in}}%
\pgfpathlineto{\pgfqpoint{4.832918in}{1.648096in}}%
\pgfpathlineto{\pgfqpoint{4.834262in}{1.576541in}}%
\pgfpathlineto{\pgfqpoint{4.834934in}{1.576541in}}%
\pgfpathlineto{\pgfqpoint{4.835606in}{1.507636in}}%
\pgfpathlineto{\pgfqpoint{4.836278in}{1.621594in}}%
\pgfpathlineto{\pgfqpoint{4.836951in}{1.621594in}}%
\pgfpathlineto{\pgfqpoint{4.837623in}{1.526188in}}%
\pgfpathlineto{\pgfqpoint{4.838295in}{1.565940in}}%
\pgfpathlineto{\pgfqpoint{4.838967in}{1.565940in}}%
\pgfpathlineto{\pgfqpoint{4.838967in}{1.465234in}}%
\pgfpathlineto{\pgfqpoint{4.839639in}{1.592442in}}%
\pgfpathlineto{\pgfqpoint{4.840311in}{1.579191in}}%
\pgfpathlineto{\pgfqpoint{4.840984in}{1.579191in}}%
\pgfpathlineto{\pgfqpoint{4.840984in}{1.512937in}}%
\pgfpathlineto{\pgfqpoint{4.842328in}{1.637495in}}%
\pgfpathlineto{\pgfqpoint{4.843000in}{1.637495in}}%
\pgfpathlineto{\pgfqpoint{4.844344in}{1.502336in}}%
\pgfpathlineto{\pgfqpoint{4.845017in}{1.502336in}}%
\pgfpathlineto{\pgfqpoint{4.845017in}{1.592442in}}%
\pgfpathlineto{\pgfqpoint{4.846361in}{1.536788in}}%
\pgfpathlineto{\pgfqpoint{4.847033in}{1.536788in}}%
\pgfpathlineto{\pgfqpoint{4.847033in}{1.587142in}}%
\pgfpathlineto{\pgfqpoint{4.848377in}{1.531488in}}%
\pgfpathlineto{\pgfqpoint{4.849050in}{1.531488in}}%
\pgfpathlineto{\pgfqpoint{4.849722in}{1.473184in}}%
\pgfpathlineto{\pgfqpoint{4.850394in}{1.608343in}}%
\pgfpathlineto{\pgfqpoint{4.851066in}{1.608343in}}%
\pgfpathlineto{\pgfqpoint{4.851738in}{1.557990in}}%
\pgfpathlineto{\pgfqpoint{4.852410in}{1.587142in}}%
\pgfpathlineto{\pgfqpoint{4.853083in}{1.587142in}}%
\pgfpathlineto{\pgfqpoint{4.853083in}{1.486435in}}%
\pgfpathlineto{\pgfqpoint{4.853755in}{1.632195in}}%
\pgfpathlineto{\pgfqpoint{4.854427in}{1.486435in}}%
\pgfpathlineto{\pgfqpoint{4.855099in}{1.486435in}}%
\pgfpathlineto{\pgfqpoint{4.855771in}{1.475834in}}%
\pgfpathlineto{\pgfqpoint{4.856443in}{1.528838in}}%
\pgfpathlineto{\pgfqpoint{4.857116in}{1.528838in}}%
\pgfpathlineto{\pgfqpoint{4.857116in}{1.605693in}}%
\pgfpathlineto{\pgfqpoint{4.858460in}{1.489085in}}%
\pgfpathlineto{\pgfqpoint{4.859132in}{1.489085in}}%
\pgfpathlineto{\pgfqpoint{4.860476in}{1.550039in}}%
\pgfpathlineto{\pgfqpoint{4.861149in}{1.550039in}}%
\pgfpathlineto{\pgfqpoint{4.861149in}{1.576541in}}%
\pgfpathlineto{\pgfqpoint{4.861821in}{1.512937in}}%
\pgfpathlineto{\pgfqpoint{4.862493in}{1.576541in}}%
\pgfpathlineto{\pgfqpoint{4.863165in}{1.576541in}}%
\pgfpathlineto{\pgfqpoint{4.864509in}{1.459933in}}%
\pgfpathlineto{\pgfqpoint{4.865182in}{1.459933in}}%
\pgfpathlineto{\pgfqpoint{4.866526in}{1.544739in}}%
\pgfpathlineto{\pgfqpoint{4.867198in}{1.544739in}}%
\pgfpathlineto{\pgfqpoint{4.867198in}{1.589792in}}%
\pgfpathlineto{\pgfqpoint{4.868542in}{1.483785in}}%
\pgfpathlineto{\pgfqpoint{4.869215in}{1.483785in}}%
\pgfpathlineto{\pgfqpoint{4.870559in}{1.542089in}}%
\pgfpathlineto{\pgfqpoint{4.871231in}{1.542089in}}%
\pgfpathlineto{\pgfqpoint{4.871231in}{1.404280in}}%
\pgfpathlineto{\pgfqpoint{4.872575in}{1.536788in}}%
\pgfpathlineto{\pgfqpoint{4.873248in}{1.536788in}}%
\pgfpathlineto{\pgfqpoint{4.873248in}{1.391029in}}%
\pgfpathlineto{\pgfqpoint{4.873920in}{1.565940in}}%
\pgfpathlineto{\pgfqpoint{4.874592in}{1.557990in}}%
\pgfpathlineto{\pgfqpoint{4.875264in}{1.557990in}}%
\pgfpathlineto{\pgfqpoint{4.875264in}{1.438732in}}%
\pgfpathlineto{\pgfqpoint{4.875936in}{1.579191in}}%
\pgfpathlineto{\pgfqpoint{4.876608in}{1.473184in}}%
\pgfpathlineto{\pgfqpoint{4.877281in}{1.473184in}}%
\pgfpathlineto{\pgfqpoint{4.877281in}{1.457283in}}%
\pgfpathlineto{\pgfqpoint{4.877953in}{1.518237in}}%
\pgfpathlineto{\pgfqpoint{4.878625in}{1.504986in}}%
\pgfpathlineto{\pgfqpoint{4.879297in}{1.504986in}}%
\pgfpathlineto{\pgfqpoint{4.879297in}{1.520887in}}%
\pgfpathlineto{\pgfqpoint{4.880641in}{1.430781in}}%
\pgfpathlineto{\pgfqpoint{4.881314in}{1.430781in}}%
\pgfpathlineto{\pgfqpoint{4.881986in}{1.414880in}}%
\pgfpathlineto{\pgfqpoint{4.882658in}{1.555340in}}%
\pgfpathlineto{\pgfqpoint{4.883330in}{1.555340in}}%
\pgfpathlineto{\pgfqpoint{4.884002in}{1.451983in}}%
\pgfpathlineto{\pgfqpoint{4.884674in}{1.512937in}}%
\pgfpathlineto{\pgfqpoint{4.885347in}{1.512937in}}%
\pgfpathlineto{\pgfqpoint{4.885347in}{1.531488in}}%
\pgfpathlineto{\pgfqpoint{4.886691in}{1.438732in}}%
\pgfpathlineto{\pgfqpoint{4.887363in}{1.438732in}}%
\pgfpathlineto{\pgfqpoint{4.888035in}{1.520887in}}%
\pgfpathlineto{\pgfqpoint{4.888707in}{1.462583in}}%
\pgfpathlineto{\pgfqpoint{4.889380in}{1.462583in}}%
\pgfpathlineto{\pgfqpoint{4.890052in}{1.361877in}}%
\pgfpathlineto{\pgfqpoint{4.890724in}{1.542089in}}%
\pgfpathlineto{\pgfqpoint{4.891396in}{1.542089in}}%
\pgfpathlineto{\pgfqpoint{4.892740in}{1.404280in}}%
\pgfpathlineto{\pgfqpoint{4.893413in}{1.404280in}}%
\pgfpathlineto{\pgfqpoint{4.894085in}{1.380428in}}%
\pgfpathlineto{\pgfqpoint{4.894757in}{1.465234in}}%
\pgfpathlineto{\pgfqpoint{4.895429in}{1.465234in}}%
\pgfpathlineto{\pgfqpoint{4.895429in}{1.494386in}}%
\pgfpathlineto{\pgfqpoint{4.896101in}{1.459933in}}%
\pgfpathlineto{\pgfqpoint{4.896773in}{1.470534in}}%
\pgfpathlineto{\pgfqpoint{4.897446in}{1.470534in}}%
\pgfpathlineto{\pgfqpoint{4.897446in}{1.345976in}}%
\pgfpathlineto{\pgfqpoint{4.898790in}{1.497036in}}%
\pgfpathlineto{\pgfqpoint{4.899462in}{1.497036in}}%
\pgfpathlineto{\pgfqpoint{4.899462in}{1.499686in}}%
\pgfpathlineto{\pgfqpoint{4.900134in}{1.393679in}}%
\pgfpathlineto{\pgfqpoint{4.900806in}{1.401629in}}%
\pgfpathlineto{\pgfqpoint{4.901479in}{1.401629in}}%
\pgfpathlineto{\pgfqpoint{4.902151in}{1.327424in}}%
\pgfpathlineto{\pgfqpoint{4.902823in}{1.473184in}}%
\pgfpathlineto{\pgfqpoint{4.903495in}{1.473184in}}%
\pgfpathlineto{\pgfqpoint{4.903495in}{1.372477in}}%
\pgfpathlineto{\pgfqpoint{4.904167in}{1.483785in}}%
\pgfpathlineto{\pgfqpoint{4.904839in}{1.414880in}}%
\pgfpathlineto{\pgfqpoint{4.905512in}{1.414880in}}%
\pgfpathlineto{\pgfqpoint{4.906184in}{1.428131in}}%
\pgfpathlineto{\pgfqpoint{4.906856in}{1.348626in}}%
\pgfpathlineto{\pgfqpoint{4.907528in}{1.348626in}}%
\pgfpathlineto{\pgfqpoint{4.908872in}{1.428131in}}%
\pgfpathlineto{\pgfqpoint{4.909545in}{1.428131in}}%
\pgfpathlineto{\pgfqpoint{4.909545in}{1.345976in}}%
\pgfpathlineto{\pgfqpoint{4.910889in}{1.361877in}}%
\pgfpathlineto{\pgfqpoint{4.911561in}{1.361877in}}%
\pgfpathlineto{\pgfqpoint{4.912905in}{1.478484in}}%
\pgfpathlineto{\pgfqpoint{4.913578in}{1.478484in}}%
\pgfpathlineto{\pgfqpoint{4.914922in}{1.345976in}}%
\pgfpathlineto{\pgfqpoint{4.915594in}{1.345976in}}%
\pgfpathlineto{\pgfqpoint{4.916938in}{1.422831in}}%
\pgfpathlineto{\pgfqpoint{4.917611in}{1.422831in}}%
\pgfpathlineto{\pgfqpoint{4.918283in}{1.345976in}}%
\pgfpathlineto{\pgfqpoint{4.918955in}{1.396329in}}%
\pgfpathlineto{\pgfqpoint{4.919627in}{1.396329in}}%
\pgfpathlineto{\pgfqpoint{4.919627in}{1.316824in}}%
\pgfpathlineto{\pgfqpoint{4.920971in}{1.369827in}}%
\pgfpathlineto{\pgfqpoint{4.921644in}{1.369827in}}%
\pgfpathlineto{\pgfqpoint{4.922316in}{1.324774in}}%
\pgfpathlineto{\pgfqpoint{4.922988in}{1.369827in}}%
\pgfpathlineto{\pgfqpoint{4.923660in}{1.369827in}}%
\pgfpathlineto{\pgfqpoint{4.924332in}{1.396329in}}%
\pgfpathlineto{\pgfqpoint{4.925004in}{1.356576in}}%
\pgfpathlineto{\pgfqpoint{4.925677in}{1.356576in}}%
\pgfpathlineto{\pgfqpoint{4.925677in}{1.322124in}}%
\pgfpathlineto{\pgfqpoint{4.926349in}{1.404280in}}%
\pgfpathlineto{\pgfqpoint{4.927021in}{1.324774in}}%
\pgfpathlineto{\pgfqpoint{4.927693in}{1.324774in}}%
\pgfpathlineto{\pgfqpoint{4.927693in}{1.414880in}}%
\pgfpathlineto{\pgfqpoint{4.929037in}{1.279721in}}%
\pgfpathlineto{\pgfqpoint{4.929710in}{1.279721in}}%
\pgfpathlineto{\pgfqpoint{4.929710in}{1.396329in}}%
\pgfpathlineto{\pgfqpoint{4.931054in}{1.295622in}}%
\pgfpathlineto{\pgfqpoint{4.931726in}{1.295622in}}%
\pgfpathlineto{\pgfqpoint{4.931726in}{1.285022in}}%
\pgfpathlineto{\pgfqpoint{4.932398in}{1.300923in}}%
\pgfpathlineto{\pgfqpoint{4.933070in}{1.287672in}}%
\pgfpathlineto{\pgfqpoint{4.933743in}{1.287672in}}%
\pgfpathlineto{\pgfqpoint{4.933743in}{1.353926in}}%
\pgfpathlineto{\pgfqpoint{4.935087in}{1.322124in}}%
\pgfpathlineto{\pgfqpoint{4.935759in}{1.322124in}}%
\pgfpathlineto{\pgfqpoint{4.935759in}{1.364527in}}%
\pgfpathlineto{\pgfqpoint{4.937103in}{1.298272in}}%
\pgfpathlineto{\pgfqpoint{4.937776in}{1.298272in}}%
\pgfpathlineto{\pgfqpoint{4.939120in}{1.375128in}}%
\pgfpathlineto{\pgfqpoint{4.939792in}{1.375128in}}%
\pgfpathlineto{\pgfqpoint{4.941136in}{1.263820in}}%
\pgfpathlineto{\pgfqpoint{4.941809in}{1.263820in}}%
\pgfpathlineto{\pgfqpoint{4.943153in}{1.361877in}}%
\pgfpathlineto{\pgfqpoint{4.943825in}{1.361877in}}%
\pgfpathlineto{\pgfqpoint{4.945170in}{1.287672in}}%
\pgfpathlineto{\pgfqpoint{4.945842in}{1.287672in}}%
\pgfpathlineto{\pgfqpoint{4.946514in}{1.205516in}}%
\pgfpathlineto{\pgfqpoint{4.947186in}{1.245269in}}%
\pgfpathlineto{\pgfqpoint{4.947858in}{1.245269in}}%
\pgfpathlineto{\pgfqpoint{4.947858in}{1.287672in}}%
\pgfpathlineto{\pgfqpoint{4.949203in}{1.194916in}}%
\pgfpathlineto{\pgfqpoint{4.949875in}{1.194916in}}%
\pgfpathlineto{\pgfqpoint{4.950547in}{1.314174in}}%
\pgfpathlineto{\pgfqpoint{4.951219in}{1.208166in}}%
\pgfpathlineto{\pgfqpoint{4.951891in}{1.208166in}}%
\pgfpathlineto{\pgfqpoint{4.951891in}{1.292972in}}%
\pgfpathlineto{\pgfqpoint{4.953236in}{1.234668in}}%
\pgfpathlineto{\pgfqpoint{4.953908in}{1.234668in}}%
\pgfpathlineto{\pgfqpoint{4.953908in}{1.232018in}}%
\pgfpathlineto{\pgfqpoint{4.955252in}{1.271771in}}%
\pgfpathlineto{\pgfqpoint{4.955924in}{1.271771in}}%
\pgfpathlineto{\pgfqpoint{4.956596in}{1.316824in}}%
\pgfpathlineto{\pgfqpoint{4.957269in}{1.298272in}}%
\pgfpathlineto{\pgfqpoint{4.957941in}{1.298272in}}%
\pgfpathlineto{\pgfqpoint{4.958613in}{1.274421in}}%
\pgfpathlineto{\pgfqpoint{4.959285in}{1.319474in}}%
\pgfpathlineto{\pgfqpoint{4.959957in}{1.319474in}}%
\pgfpathlineto{\pgfqpoint{4.960629in}{1.218767in}}%
\pgfpathlineto{\pgfqpoint{4.961302in}{1.282371in}}%
\pgfpathlineto{\pgfqpoint{4.961974in}{1.282371in}}%
\pgfpathlineto{\pgfqpoint{4.962646in}{1.216117in}}%
\pgfpathlineto{\pgfqpoint{4.963318in}{1.218767in}}%
\pgfpathlineto{\pgfqpoint{4.963990in}{1.218767in}}%
\pgfpathlineto{\pgfqpoint{4.963990in}{1.308873in}}%
\pgfpathlineto{\pgfqpoint{4.965335in}{1.269121in}}%
\pgfpathlineto{\pgfqpoint{4.966007in}{1.269121in}}%
\pgfpathlineto{\pgfqpoint{4.967351in}{1.176364in}}%
\pgfpathlineto{\pgfqpoint{4.968023in}{1.176364in}}%
\pgfpathlineto{\pgfqpoint{4.968023in}{1.171064in}}%
\pgfpathlineto{\pgfqpoint{4.969368in}{1.247919in}}%
\pgfpathlineto{\pgfqpoint{4.970040in}{1.247919in}}%
\pgfpathlineto{\pgfqpoint{4.970040in}{1.184315in}}%
\pgfpathlineto{\pgfqpoint{4.971384in}{1.277071in}}%
\pgfpathlineto{\pgfqpoint{4.972056in}{1.277071in}}%
\pgfpathlineto{\pgfqpoint{4.972056in}{1.197566in}}%
\pgfpathlineto{\pgfqpoint{4.973401in}{1.239969in}}%
\pgfpathlineto{\pgfqpoint{4.974073in}{1.239969in}}%
\pgfpathlineto{\pgfqpoint{4.974073in}{1.126011in}}%
\pgfpathlineto{\pgfqpoint{4.975417in}{1.234668in}}%
\pgfpathlineto{\pgfqpoint{4.976089in}{1.234668in}}%
\pgfpathlineto{\pgfqpoint{4.977434in}{1.139262in}}%
\pgfpathlineto{\pgfqpoint{4.978106in}{1.139262in}}%
\pgfpathlineto{\pgfqpoint{4.978106in}{1.266470in}}%
\pgfpathlineto{\pgfqpoint{4.979450in}{1.232018in}}%
\pgfpathlineto{\pgfqpoint{4.980122in}{1.232018in}}%
\pgfpathlineto{\pgfqpoint{4.980122in}{1.155163in}}%
\pgfpathlineto{\pgfqpoint{4.981467in}{1.186965in}}%
\pgfpathlineto{\pgfqpoint{4.982139in}{1.186965in}}%
\pgfpathlineto{\pgfqpoint{4.983483in}{1.160463in}}%
\pgfpathlineto{\pgfqpoint{4.984155in}{1.160463in}}%
\pgfpathlineto{\pgfqpoint{4.984155in}{1.232018in}}%
\pgfpathlineto{\pgfqpoint{4.985500in}{1.181665in}}%
\pgfpathlineto{\pgfqpoint{4.986172in}{1.181665in}}%
\pgfpathlineto{\pgfqpoint{4.986172in}{1.096859in}}%
\pgfpathlineto{\pgfqpoint{4.987516in}{1.245269in}}%
\pgfpathlineto{\pgfqpoint{4.988188in}{1.245269in}}%
\pgfpathlineto{\pgfqpoint{4.988188in}{1.160463in}}%
\pgfpathlineto{\pgfqpoint{4.989533in}{1.173714in}}%
\pgfpathlineto{\pgfqpoint{4.990205in}{1.173714in}}%
\pgfpathlineto{\pgfqpoint{4.990877in}{1.149863in}}%
\pgfpathlineto{\pgfqpoint{4.991549in}{1.157813in}}%
\pgfpathlineto{\pgfqpoint{4.992221in}{1.157813in}}%
\pgfpathlineto{\pgfqpoint{4.992221in}{1.099509in}}%
\pgfpathlineto{\pgfqpoint{4.993566in}{1.107460in}}%
\pgfpathlineto{\pgfqpoint{4.994238in}{1.107460in}}%
\pgfpathlineto{\pgfqpoint{4.995582in}{1.184315in}}%
\pgfpathlineto{\pgfqpoint{4.996254in}{1.184315in}}%
\pgfpathlineto{\pgfqpoint{4.997599in}{1.126011in}}%
\pgfpathlineto{\pgfqpoint{4.998271in}{1.126011in}}%
\pgfpathlineto{\pgfqpoint{4.998271in}{1.139262in}}%
\pgfpathlineto{\pgfqpoint{4.999615in}{1.126011in}}%
\pgfpathlineto{\pgfqpoint{5.000287in}{1.126011in}}%
\pgfpathlineto{\pgfqpoint{5.001632in}{1.163113in}}%
\pgfpathlineto{\pgfqpoint{5.002304in}{1.163113in}}%
\pgfpathlineto{\pgfqpoint{5.003648in}{1.107460in}}%
\pgfpathlineto{\pgfqpoint{5.004320in}{1.107460in}}%
\pgfpathlineto{\pgfqpoint{5.004992in}{1.157813in}}%
\pgfpathlineto{\pgfqpoint{5.005665in}{1.104810in}}%
\pgfpathlineto{\pgfqpoint{5.006337in}{1.104810in}}%
\pgfpathlineto{\pgfqpoint{5.007681in}{1.216117in}}%
\pgfpathlineto{\pgfqpoint{5.008353in}{1.216117in}}%
\pgfpathlineto{\pgfqpoint{5.008353in}{1.152513in}}%
\pgfpathlineto{\pgfqpoint{5.009698in}{1.179015in}}%
\pgfpathlineto{\pgfqpoint{5.010370in}{1.179015in}}%
\pgfpathlineto{\pgfqpoint{5.011042in}{1.086258in}}%
\pgfpathlineto{\pgfqpoint{5.011714in}{1.136612in}}%
\pgfpathlineto{\pgfqpoint{5.012386in}{1.136612in}}%
\pgfpathlineto{\pgfqpoint{5.012386in}{1.147212in}}%
\pgfpathlineto{\pgfqpoint{5.013731in}{1.086258in}}%
\pgfpathlineto{\pgfqpoint{5.014403in}{1.086258in}}%
\pgfpathlineto{\pgfqpoint{5.014403in}{1.075658in}}%
\pgfpathlineto{\pgfqpoint{5.015747in}{1.133962in}}%
\pgfpathlineto{\pgfqpoint{5.016419in}{1.133962in}}%
\pgfpathlineto{\pgfqpoint{5.016419in}{1.141912in}}%
\pgfpathlineto{\pgfqpoint{5.017091in}{1.049156in}}%
\pgfpathlineto{\pgfqpoint{5.017764in}{1.070357in}}%
\pgfpathlineto{\pgfqpoint{5.018436in}{1.070357in}}%
\pgfpathlineto{\pgfqpoint{5.018436in}{1.083608in}}%
\pgfpathlineto{\pgfqpoint{5.019780in}{1.083608in}}%
\pgfpathlineto{\pgfqpoint{5.020452in}{1.083608in}}%
\pgfpathlineto{\pgfqpoint{5.020452in}{1.096859in}}%
\pgfpathlineto{\pgfqpoint{5.021797in}{1.078308in}}%
\pgfpathlineto{\pgfqpoint{5.022469in}{1.078308in}}%
\pgfpathlineto{\pgfqpoint{5.022469in}{1.065057in}}%
\pgfpathlineto{\pgfqpoint{5.023813in}{1.102159in}}%
\pgfpathlineto{\pgfqpoint{5.024485in}{1.102159in}}%
\pgfpathlineto{\pgfqpoint{5.025830in}{1.078308in}}%
\pgfpathlineto{\pgfqpoint{5.026502in}{1.078308in}}%
\pgfpathlineto{\pgfqpoint{5.026502in}{1.033255in}}%
\pgfpathlineto{\pgfqpoint{5.027174in}{1.094209in}}%
\pgfpathlineto{\pgfqpoint{5.027846in}{1.043856in}}%
\pgfpathlineto{\pgfqpoint{5.028518in}{1.043856in}}%
\pgfpathlineto{\pgfqpoint{5.028518in}{1.070357in}}%
\pgfpathlineto{\pgfqpoint{5.029863in}{1.033255in}}%
\pgfpathlineto{\pgfqpoint{5.030535in}{1.033255in}}%
\pgfpathlineto{\pgfqpoint{5.031207in}{1.078308in}}%
\pgfpathlineto{\pgfqpoint{5.031879in}{0.985552in}}%
\pgfpathlineto{\pgfqpoint{5.032551in}{0.985552in}}%
\pgfpathlineto{\pgfqpoint{5.033223in}{1.102159in}}%
\pgfpathlineto{\pgfqpoint{5.033896in}{1.067707in}}%
\pgfpathlineto{\pgfqpoint{5.034568in}{1.067707in}}%
\pgfpathlineto{\pgfqpoint{5.035240in}{0.977601in}}%
\pgfpathlineto{\pgfqpoint{5.035912in}{0.982901in}}%
\pgfpathlineto{\pgfqpoint{5.036584in}{0.982901in}}%
\pgfpathlineto{\pgfqpoint{5.036584in}{1.118060in}}%
\pgfpathlineto{\pgfqpoint{5.037929in}{1.080958in}}%
\pgfpathlineto{\pgfqpoint{5.038601in}{1.080958in}}%
\pgfpathlineto{\pgfqpoint{5.038601in}{0.967000in}}%
\pgfpathlineto{\pgfqpoint{5.039945in}{1.012053in}}%
\pgfpathlineto{\pgfqpoint{5.040617in}{1.012053in}}%
\pgfpathlineto{\pgfqpoint{5.040617in}{1.001453in}}%
\pgfpathlineto{\pgfqpoint{5.041962in}{1.073007in}}%
\pgfpathlineto{\pgfqpoint{5.042634in}{1.073007in}}%
\pgfpathlineto{\pgfqpoint{5.043306in}{1.059757in}}%
\pgfpathlineto{\pgfqpoint{5.043978in}{1.067707in}}%
\pgfpathlineto{\pgfqpoint{5.044650in}{1.067707in}}%
\pgfpathlineto{\pgfqpoint{5.045322in}{1.014704in}}%
\pgfpathlineto{\pgfqpoint{5.045995in}{1.027954in}}%
\pgfpathlineto{\pgfqpoint{5.046667in}{1.027954in}}%
\pgfpathlineto{\pgfqpoint{5.046667in}{0.990852in}}%
\pgfpathlineto{\pgfqpoint{5.047339in}{1.065057in}}%
\pgfpathlineto{\pgfqpoint{5.048011in}{1.041205in}}%
\pgfpathlineto{\pgfqpoint{5.048683in}{1.041205in}}%
\pgfpathlineto{\pgfqpoint{5.048683in}{1.022654in}}%
\pgfpathlineto{\pgfqpoint{5.050028in}{1.070357in}}%
\pgfpathlineto{\pgfqpoint{5.050700in}{1.070357in}}%
\pgfpathlineto{\pgfqpoint{5.050700in}{1.014704in}}%
\pgfpathlineto{\pgfqpoint{5.052044in}{1.022654in}}%
\pgfpathlineto{\pgfqpoint{5.052716in}{1.022654in}}%
\pgfpathlineto{\pgfqpoint{5.053388in}{0.998803in}}%
\pgfpathlineto{\pgfqpoint{5.054061in}{0.998803in}}%
\pgfpathlineto{\pgfqpoint{5.054733in}{0.998803in}}%
\pgfpathlineto{\pgfqpoint{5.055405in}{0.972301in}}%
\pgfpathlineto{\pgfqpoint{5.056077in}{1.020004in}}%
\pgfpathlineto{\pgfqpoint{5.056749in}{1.020004in}}%
\pgfpathlineto{\pgfqpoint{5.056749in}{1.054456in}}%
\pgfpathlineto{\pgfqpoint{5.058094in}{1.009403in}}%
\pgfpathlineto{\pgfqpoint{5.058766in}{1.009403in}}%
\pgfpathlineto{\pgfqpoint{5.058766in}{1.017354in}}%
\pgfpathlineto{\pgfqpoint{5.060110in}{0.998803in}}%
\pgfpathlineto{\pgfqpoint{5.060782in}{0.998803in}}%
\pgfpathlineto{\pgfqpoint{5.060782in}{1.014704in}}%
\pgfpathlineto{\pgfqpoint{5.062127in}{0.959050in}}%
\pgfpathlineto{\pgfqpoint{5.062799in}{0.959050in}}%
\pgfpathlineto{\pgfqpoint{5.064143in}{1.035905in}}%
\pgfpathlineto{\pgfqpoint{5.064815in}{1.035905in}}%
\pgfpathlineto{\pgfqpoint{5.065488in}{0.969651in}}%
\pgfpathlineto{\pgfqpoint{5.066160in}{1.038555in}}%
\pgfpathlineto{\pgfqpoint{5.066832in}{1.038555in}}%
\pgfpathlineto{\pgfqpoint{5.066832in}{0.977601in}}%
\pgfpathlineto{\pgfqpoint{5.068176in}{1.043856in}}%
\pgfpathlineto{\pgfqpoint{5.068848in}{1.043856in}}%
\pgfpathlineto{\pgfqpoint{5.070193in}{0.993502in}}%
\pgfpathlineto{\pgfqpoint{5.070865in}{0.993502in}}%
\pgfpathlineto{\pgfqpoint{5.071537in}{0.945799in}}%
\pgfpathlineto{\pgfqpoint{5.072209in}{0.956400in}}%
\pgfpathlineto{\pgfqpoint{5.072881in}{0.956400in}}%
\pgfpathlineto{\pgfqpoint{5.073554in}{1.014704in}}%
\pgfpathlineto{\pgfqpoint{5.074226in}{0.993502in}}%
\pgfpathlineto{\pgfqpoint{5.074898in}{0.993502in}}%
\pgfpathlineto{\pgfqpoint{5.074898in}{0.948449in}}%
\pgfpathlineto{\pgfqpoint{5.075570in}{0.998803in}}%
\pgfpathlineto{\pgfqpoint{5.076242in}{0.953750in}}%
\pgfpathlineto{\pgfqpoint{5.076914in}{0.953750in}}%
\pgfpathlineto{\pgfqpoint{5.077587in}{1.025304in}}%
\pgfpathlineto{\pgfqpoint{5.078259in}{0.985552in}}%
\pgfpathlineto{\pgfqpoint{5.078931in}{0.985552in}}%
\pgfpathlineto{\pgfqpoint{5.079603in}{0.982901in}}%
\pgfpathlineto{\pgfqpoint{5.080275in}{1.001453in}}%
\pgfpathlineto{\pgfqpoint{5.080947in}{1.001453in}}%
\pgfpathlineto{\pgfqpoint{5.081620in}{1.025304in}}%
\pgfpathlineto{\pgfqpoint{5.082292in}{0.927248in}}%
\pgfpathlineto{\pgfqpoint{5.082964in}{0.927248in}}%
\pgfpathlineto{\pgfqpoint{5.082964in}{0.990852in}}%
\pgfpathlineto{\pgfqpoint{5.084308in}{0.940499in}}%
\pgfpathlineto{\pgfqpoint{5.084980in}{0.940499in}}%
\pgfpathlineto{\pgfqpoint{5.085653in}{0.972301in}}%
\pgfpathlineto{\pgfqpoint{5.086325in}{0.951099in}}%
\pgfpathlineto{\pgfqpoint{5.086997in}{0.951099in}}%
\pgfpathlineto{\pgfqpoint{5.086997in}{0.906046in}}%
\pgfpathlineto{\pgfqpoint{5.088341in}{0.935198in}}%
\pgfpathlineto{\pgfqpoint{5.089013in}{0.935198in}}%
\pgfpathlineto{\pgfqpoint{5.089686in}{0.998803in}}%
\pgfpathlineto{\pgfqpoint{5.090358in}{0.935198in}}%
\pgfpathlineto{\pgfqpoint{5.091030in}{0.935198in}}%
\pgfpathlineto{\pgfqpoint{5.091030in}{1.009403in}}%
\pgfpathlineto{\pgfqpoint{5.092374in}{0.900746in}}%
\pgfpathlineto{\pgfqpoint{5.093046in}{0.900746in}}%
\pgfpathlineto{\pgfqpoint{5.093719in}{0.985552in}}%
\pgfpathlineto{\pgfqpoint{5.094391in}{0.876894in}}%
\pgfpathlineto{\pgfqpoint{5.095063in}{0.876894in}}%
\pgfpathlineto{\pgfqpoint{5.096407in}{0.951099in}}%
\pgfpathlineto{\pgfqpoint{5.097079in}{0.951099in}}%
\pgfpathlineto{\pgfqpoint{5.097079in}{0.977601in}}%
\pgfpathlineto{\pgfqpoint{5.098424in}{0.924598in}}%
\pgfpathlineto{\pgfqpoint{5.099096in}{0.924598in}}%
\pgfpathlineto{\pgfqpoint{5.099096in}{0.906046in}}%
\pgfpathlineto{\pgfqpoint{5.099768in}{0.977601in}}%
\pgfpathlineto{\pgfqpoint{5.100440in}{0.967000in}}%
\pgfpathlineto{\pgfqpoint{5.101112in}{0.967000in}}%
\pgfpathlineto{\pgfqpoint{5.101785in}{0.972301in}}%
\pgfpathlineto{\pgfqpoint{5.102457in}{0.908697in}}%
\pgfpathlineto{\pgfqpoint{5.103129in}{0.908697in}}%
\pgfpathlineto{\pgfqpoint{5.103801in}{0.948449in}}%
\pgfpathlineto{\pgfqpoint{5.104473in}{0.945799in}}%
\pgfpathlineto{\pgfqpoint{5.105818in}{0.945799in}}%
\pgfpathlineto{\pgfqpoint{5.105818in}{0.935198in}}%
\pgfpathlineto{\pgfqpoint{5.107162in}{0.980251in}}%
\pgfpathlineto{\pgfqpoint{5.107834in}{0.980251in}}%
\pgfpathlineto{\pgfqpoint{5.109178in}{0.895446in}}%
\pgfpathlineto{\pgfqpoint{5.109851in}{0.895446in}}%
\pgfpathlineto{\pgfqpoint{5.110523in}{0.951099in}}%
\pgfpathlineto{\pgfqpoint{5.111195in}{0.937848in}}%
\pgfpathlineto{\pgfqpoint{5.111867in}{0.937848in}}%
\pgfpathlineto{\pgfqpoint{5.112539in}{0.967000in}}%
\pgfpathlineto{\pgfqpoint{5.113211in}{0.890145in}}%
\pgfpathlineto{\pgfqpoint{5.113884in}{0.890145in}}%
\pgfpathlineto{\pgfqpoint{5.115228in}{0.953750in}}%
\pgfpathlineto{\pgfqpoint{5.115900in}{0.953750in}}%
\pgfpathlineto{\pgfqpoint{5.115900in}{0.906046in}}%
\pgfpathlineto{\pgfqpoint{5.117244in}{0.937848in}}%
\pgfpathlineto{\pgfqpoint{5.117917in}{0.937848in}}%
\pgfpathlineto{\pgfqpoint{5.119261in}{0.916647in}}%
\pgfpathlineto{\pgfqpoint{5.120605in}{0.916647in}}%
\pgfpathlineto{\pgfqpoint{5.121277in}{0.884845in}}%
\pgfpathlineto{\pgfqpoint{5.121950in}{0.932548in}}%
\pgfpathlineto{\pgfqpoint{5.122622in}{0.932548in}}%
\pgfpathlineto{\pgfqpoint{5.122622in}{0.879545in}}%
\pgfpathlineto{\pgfqpoint{5.123966in}{0.929898in}}%
\pgfpathlineto{\pgfqpoint{5.124638in}{0.929898in}}%
\pgfpathlineto{\pgfqpoint{5.125310in}{0.882195in}}%
\pgfpathlineto{\pgfqpoint{5.125983in}{0.895446in}}%
\pgfpathlineto{\pgfqpoint{5.126655in}{0.895446in}}%
\pgfpathlineto{\pgfqpoint{5.126655in}{0.932548in}}%
\pgfpathlineto{\pgfqpoint{5.127327in}{0.890145in}}%
\pgfpathlineto{\pgfqpoint{5.127999in}{0.932548in}}%
\pgfpathlineto{\pgfqpoint{5.128671in}{0.932548in}}%
\pgfpathlineto{\pgfqpoint{5.130016in}{0.866294in}}%
\pgfpathlineto{\pgfqpoint{5.130688in}{0.866294in}}%
\pgfpathlineto{\pgfqpoint{5.130688in}{0.853043in}}%
\pgfpathlineto{\pgfqpoint{5.131360in}{0.911347in}}%
\pgfpathlineto{\pgfqpoint{5.132032in}{0.903396in}}%
\pgfpathlineto{\pgfqpoint{5.132704in}{0.903396in}}%
\pgfpathlineto{\pgfqpoint{5.132704in}{0.916647in}}%
\pgfpathlineto{\pgfqpoint{5.133376in}{0.860993in}}%
\pgfpathlineto{\pgfqpoint{5.134049in}{0.913997in}}%
\pgfpathlineto{\pgfqpoint{5.134721in}{0.913997in}}%
\pgfpathlineto{\pgfqpoint{5.136065in}{0.876894in}}%
\pgfpathlineto{\pgfqpoint{5.136737in}{0.876894in}}%
\pgfpathlineto{\pgfqpoint{5.136737in}{0.937848in}}%
\pgfpathlineto{\pgfqpoint{5.138082in}{0.884845in}}%
\pgfpathlineto{\pgfqpoint{5.138754in}{0.884845in}}%
\pgfpathlineto{\pgfqpoint{5.138754in}{0.921947in}}%
\pgfpathlineto{\pgfqpoint{5.140098in}{0.903396in}}%
\pgfpathlineto{\pgfqpoint{5.141442in}{0.903396in}}%
\pgfpathlineto{\pgfqpoint{5.142115in}{0.927248in}}%
\pgfpathlineto{\pgfqpoint{5.142787in}{0.882195in}}%
\pgfpathlineto{\pgfqpoint{5.143459in}{0.882195in}}%
\pgfpathlineto{\pgfqpoint{5.143459in}{0.839792in}}%
\pgfpathlineto{\pgfqpoint{5.144803in}{0.884845in}}%
\pgfpathlineto{\pgfqpoint{5.145475in}{0.884845in}}%
\pgfpathlineto{\pgfqpoint{5.146820in}{0.858343in}}%
\pgfpathlineto{\pgfqpoint{5.147492in}{0.858343in}}%
\pgfpathlineto{\pgfqpoint{5.148164in}{0.919297in}}%
\pgfpathlineto{\pgfqpoint{5.148836in}{0.879545in}}%
\pgfpathlineto{\pgfqpoint{5.149508in}{0.879545in}}%
\pgfpathlineto{\pgfqpoint{5.149508in}{0.906046in}}%
\pgfpathlineto{\pgfqpoint{5.150853in}{0.845092in}}%
\pgfpathlineto{\pgfqpoint{5.151525in}{0.845092in}}%
\pgfpathlineto{\pgfqpoint{5.152197in}{0.895446in}}%
\pgfpathlineto{\pgfqpoint{5.152869in}{0.868944in}}%
\pgfpathlineto{\pgfqpoint{5.153541in}{0.868944in}}%
\pgfpathlineto{\pgfqpoint{5.154214in}{0.892795in}}%
\pgfpathlineto{\pgfqpoint{5.154886in}{0.892795in}}%
\pgfpathlineto{\pgfqpoint{5.155558in}{0.892795in}}%
\pgfpathlineto{\pgfqpoint{5.156230in}{0.887495in}}%
\pgfpathlineto{\pgfqpoint{5.156902in}{0.908697in}}%
\pgfpathlineto{\pgfqpoint{5.157574in}{0.908697in}}%
\pgfpathlineto{\pgfqpoint{5.158919in}{0.858343in}}%
\pgfpathlineto{\pgfqpoint{5.159591in}{0.858343in}}%
\pgfpathlineto{\pgfqpoint{5.159591in}{0.887495in}}%
\pgfpathlineto{\pgfqpoint{5.160935in}{0.847742in}}%
\pgfpathlineto{\pgfqpoint{5.161607in}{0.847742in}}%
\pgfpathlineto{\pgfqpoint{5.162280in}{0.916647in}}%
\pgfpathlineto{\pgfqpoint{5.162952in}{0.853043in}}%
\pgfpathlineto{\pgfqpoint{5.163624in}{0.853043in}}%
\pgfpathlineto{\pgfqpoint{5.163624in}{0.892795in}}%
\pgfpathlineto{\pgfqpoint{5.164968in}{0.874244in}}%
\pgfpathlineto{\pgfqpoint{5.165640in}{0.874244in}}%
\pgfpathlineto{\pgfqpoint{5.166313in}{0.884845in}}%
\pgfpathlineto{\pgfqpoint{5.166985in}{0.855693in}}%
\pgfpathlineto{\pgfqpoint{5.167657in}{0.855693in}}%
\pgfpathlineto{\pgfqpoint{5.168329in}{0.858343in}}%
\pgfpathlineto{\pgfqpoint{5.169001in}{0.831841in}}%
\pgfpathlineto{\pgfqpoint{5.169674in}{0.831841in}}%
\pgfpathlineto{\pgfqpoint{5.170346in}{0.921947in}}%
\pgfpathlineto{\pgfqpoint{5.171018in}{0.853043in}}%
\pgfpathlineto{\pgfqpoint{5.171690in}{0.853043in}}%
\pgfpathlineto{\pgfqpoint{5.172362in}{0.839792in}}%
\pgfpathlineto{\pgfqpoint{5.173034in}{0.863644in}}%
\pgfpathlineto{\pgfqpoint{5.173707in}{0.863644in}}%
\pgfpathlineto{\pgfqpoint{5.173707in}{0.890145in}}%
\pgfpathlineto{\pgfqpoint{5.175051in}{0.839792in}}%
\pgfpathlineto{\pgfqpoint{5.175723in}{0.839792in}}%
\pgfpathlineto{\pgfqpoint{5.176395in}{0.884845in}}%
\pgfpathlineto{\pgfqpoint{5.177067in}{0.871594in}}%
\pgfpathlineto{\pgfqpoint{5.177740in}{0.871594in}}%
\pgfpathlineto{\pgfqpoint{5.177740in}{0.892795in}}%
\pgfpathlineto{\pgfqpoint{5.178412in}{0.850393in}}%
\pgfpathlineto{\pgfqpoint{5.179084in}{0.892795in}}%
\pgfpathlineto{\pgfqpoint{5.179756in}{0.892795in}}%
\pgfpathlineto{\pgfqpoint{5.181100in}{0.831841in}}%
\pgfpathlineto{\pgfqpoint{5.181773in}{0.831841in}}%
\pgfpathlineto{\pgfqpoint{5.181773in}{0.882195in}}%
\pgfpathlineto{\pgfqpoint{5.182445in}{0.810640in}}%
\pgfpathlineto{\pgfqpoint{5.183117in}{0.850393in}}%
\pgfpathlineto{\pgfqpoint{5.183789in}{0.850393in}}%
\pgfpathlineto{\pgfqpoint{5.183789in}{0.845092in}}%
\pgfpathlineto{\pgfqpoint{5.185133in}{0.903396in}}%
\pgfpathlineto{\pgfqpoint{5.185806in}{0.903396in}}%
\pgfpathlineto{\pgfqpoint{5.187150in}{0.826541in}}%
\pgfpathlineto{\pgfqpoint{5.187822in}{0.826541in}}%
\pgfpathlineto{\pgfqpoint{5.188494in}{0.900746in}}%
\pgfpathlineto{\pgfqpoint{5.189166in}{0.826541in}}%
\pgfpathlineto{\pgfqpoint{5.191183in}{0.826541in}}%
\pgfpathlineto{\pgfqpoint{5.191183in}{0.900746in}}%
\pgfpathlineto{\pgfqpoint{5.192527in}{0.860993in}}%
\pgfpathlineto{\pgfqpoint{5.193199in}{0.860993in}}%
\pgfpathlineto{\pgfqpoint{5.194544in}{0.797389in}}%
\pgfpathlineto{\pgfqpoint{5.195216in}{0.797389in}}%
\pgfpathlineto{\pgfqpoint{5.195216in}{0.879545in}}%
\pgfpathlineto{\pgfqpoint{5.196560in}{0.847742in}}%
\pgfpathlineto{\pgfqpoint{5.197232in}{0.847742in}}%
\pgfpathlineto{\pgfqpoint{5.197905in}{0.860993in}}%
\pgfpathlineto{\pgfqpoint{5.198577in}{0.818590in}}%
\pgfpathlineto{\pgfqpoint{5.199249in}{0.818590in}}%
\pgfpathlineto{\pgfqpoint{5.199249in}{0.813290in}}%
\pgfpathlineto{\pgfqpoint{5.200593in}{0.847742in}}%
\pgfpathlineto{\pgfqpoint{5.201265in}{0.847742in}}%
\pgfpathlineto{\pgfqpoint{5.201938in}{0.781488in}}%
\pgfpathlineto{\pgfqpoint{5.202610in}{0.847742in}}%
\pgfpathlineto{\pgfqpoint{5.203282in}{0.847742in}}%
\pgfpathlineto{\pgfqpoint{5.203282in}{0.800039in}}%
\pgfpathlineto{\pgfqpoint{5.204626in}{0.805340in}}%
\pgfpathlineto{\pgfqpoint{5.205298in}{0.805340in}}%
\pgfpathlineto{\pgfqpoint{5.205298in}{0.789439in}}%
\pgfpathlineto{\pgfqpoint{5.205971in}{0.853043in}}%
\pgfpathlineto{\pgfqpoint{5.206643in}{0.847742in}}%
\pgfpathlineto{\pgfqpoint{5.207315in}{0.847742in}}%
\pgfpathlineto{\pgfqpoint{5.207315in}{0.866294in}}%
\pgfpathlineto{\pgfqpoint{5.207987in}{0.792089in}}%
\pgfpathlineto{\pgfqpoint{5.208659in}{0.847742in}}%
\pgfpathlineto{\pgfqpoint{5.209331in}{0.847742in}}%
\pgfpathlineto{\pgfqpoint{5.209331in}{0.805340in}}%
\pgfpathlineto{\pgfqpoint{5.210676in}{0.823891in}}%
\pgfpathlineto{\pgfqpoint{5.211348in}{0.823891in}}%
\pgfpathlineto{\pgfqpoint{5.212020in}{0.847742in}}%
\pgfpathlineto{\pgfqpoint{5.212692in}{0.834492in}}%
\pgfpathlineto{\pgfqpoint{5.213364in}{0.834492in}}%
\pgfpathlineto{\pgfqpoint{5.214037in}{0.794739in}}%
\pgfpathlineto{\pgfqpoint{5.214709in}{0.815940in}}%
\pgfpathlineto{\pgfqpoint{5.215381in}{0.815940in}}%
\pgfpathlineto{\pgfqpoint{5.216053in}{0.810640in}}%
\pgfpathlineto{\pgfqpoint{5.216725in}{0.842442in}}%
\pgfpathlineto{\pgfqpoint{5.217397in}{0.842442in}}%
\pgfpathlineto{\pgfqpoint{5.218070in}{0.792089in}}%
\pgfpathlineto{\pgfqpoint{5.218070in}{0.863644in}}%
\pgfpathlineto{\pgfqpoint{5.218742in}{0.823891in}}%
\pgfpathlineto{\pgfqpoint{5.219414in}{0.823891in}}%
\pgfpathlineto{\pgfqpoint{5.220086in}{0.789439in}}%
\pgfpathlineto{\pgfqpoint{5.220758in}{0.831841in}}%
\pgfpathlineto{\pgfqpoint{5.221430in}{0.831841in}}%
\pgfpathlineto{\pgfqpoint{5.222103in}{0.802689in}}%
\pgfpathlineto{\pgfqpoint{5.222775in}{0.853043in}}%
\pgfpathlineto{\pgfqpoint{5.223447in}{0.853043in}}%
\pgfpathlineto{\pgfqpoint{5.224119in}{0.807990in}}%
\pgfpathlineto{\pgfqpoint{5.224791in}{0.850393in}}%
\pgfpathlineto{\pgfqpoint{5.225463in}{0.850393in}}%
\pgfpathlineto{\pgfqpoint{5.225463in}{0.807990in}}%
\pgfpathlineto{\pgfqpoint{5.226808in}{0.818590in}}%
\pgfpathlineto{\pgfqpoint{5.227480in}{0.818590in}}%
\pgfpathlineto{\pgfqpoint{5.227480in}{0.831841in}}%
\pgfpathlineto{\pgfqpoint{5.228152in}{0.810640in}}%
\pgfpathlineto{\pgfqpoint{5.228824in}{0.829191in}}%
\pgfpathlineto{\pgfqpoint{5.229496in}{0.829191in}}%
\pgfpathlineto{\pgfqpoint{5.230841in}{0.776188in}}%
\pgfpathlineto{\pgfqpoint{5.231513in}{0.776188in}}%
\pgfpathlineto{\pgfqpoint{5.231513in}{0.842442in}}%
\pgfpathlineto{\pgfqpoint{5.232857in}{0.839792in}}%
\pgfpathlineto{\pgfqpoint{5.233529in}{0.839792in}}%
\pgfpathlineto{\pgfqpoint{5.234202in}{0.794739in}}%
\pgfpathlineto{\pgfqpoint{5.234874in}{0.823891in}}%
\pgfpathlineto{\pgfqpoint{5.235546in}{0.823891in}}%
\pgfpathlineto{\pgfqpoint{5.235546in}{0.858343in}}%
\pgfpathlineto{\pgfqpoint{5.236218in}{0.757636in}}%
\pgfpathlineto{\pgfqpoint{5.236890in}{0.794739in}}%
\pgfpathlineto{\pgfqpoint{5.237562in}{0.794739in}}%
\pgfpathlineto{\pgfqpoint{5.237562in}{0.802689in}}%
\pgfpathlineto{\pgfqpoint{5.238235in}{0.789439in}}%
\pgfpathlineto{\pgfqpoint{5.238907in}{0.800039in}}%
\pgfpathlineto{\pgfqpoint{5.239579in}{0.800039in}}%
\pgfpathlineto{\pgfqpoint{5.239579in}{0.757636in}}%
\pgfpathlineto{\pgfqpoint{5.240923in}{0.807990in}}%
\pgfpathlineto{\pgfqpoint{5.241595in}{0.807990in}}%
\pgfpathlineto{\pgfqpoint{5.241595in}{0.778838in}}%
\pgfpathlineto{\pgfqpoint{5.242268in}{0.831841in}}%
\pgfpathlineto{\pgfqpoint{5.242940in}{0.781488in}}%
\pgfpathlineto{\pgfqpoint{5.243612in}{0.781488in}}%
\pgfpathlineto{\pgfqpoint{5.244284in}{0.778838in}}%
\pgfpathlineto{\pgfqpoint{5.244956in}{0.792089in}}%
\pgfpathlineto{\pgfqpoint{5.245628in}{0.792089in}}%
\pgfpathlineto{\pgfqpoint{5.245628in}{0.821241in}}%
\pgfpathlineto{\pgfqpoint{5.246301in}{0.765587in}}%
\pgfpathlineto{\pgfqpoint{5.246973in}{0.784138in}}%
\pgfpathlineto{\pgfqpoint{5.247645in}{0.784138in}}%
\pgfpathlineto{\pgfqpoint{5.247645in}{0.797389in}}%
\pgfpathlineto{\pgfqpoint{5.248989in}{0.765587in}}%
\pgfpathlineto{\pgfqpoint{5.249661in}{0.765587in}}%
\pgfpathlineto{\pgfqpoint{5.250334in}{0.847742in}}%
\pgfpathlineto{\pgfqpoint{5.251006in}{0.792089in}}%
\pgfpathlineto{\pgfqpoint{5.251678in}{0.792089in}}%
\pgfpathlineto{\pgfqpoint{5.251678in}{0.805340in}}%
\pgfpathlineto{\pgfqpoint{5.252350in}{0.776188in}}%
\pgfpathlineto{\pgfqpoint{5.253022in}{0.781488in}}%
\pgfpathlineto{\pgfqpoint{5.253694in}{0.781488in}}%
\pgfpathlineto{\pgfqpoint{5.254367in}{0.815940in}}%
\pgfpathlineto{\pgfqpoint{5.255039in}{0.815940in}}%
\pgfpathlineto{\pgfqpoint{5.255711in}{0.815940in}}%
\pgfpathlineto{\pgfqpoint{5.255711in}{0.778838in}}%
\pgfpathlineto{\pgfqpoint{5.256383in}{0.829191in}}%
\pgfpathlineto{\pgfqpoint{5.257055in}{0.792089in}}%
\pgfpathlineto{\pgfqpoint{5.257727in}{0.792089in}}%
\pgfpathlineto{\pgfqpoint{5.257727in}{0.823891in}}%
\pgfpathlineto{\pgfqpoint{5.259072in}{0.810640in}}%
\pgfpathlineto{\pgfqpoint{5.259744in}{0.810640in}}%
\pgfpathlineto{\pgfqpoint{5.260416in}{0.786788in}}%
\pgfpathlineto{\pgfqpoint{5.260416in}{0.845092in}}%
\pgfpathlineto{\pgfqpoint{5.261088in}{0.805340in}}%
\pgfpathlineto{\pgfqpoint{5.261760in}{0.805340in}}%
\pgfpathlineto{\pgfqpoint{5.262433in}{0.810640in}}%
\pgfpathlineto{\pgfqpoint{5.263105in}{0.770887in}}%
\pgfpathlineto{\pgfqpoint{5.263777in}{0.770887in}}%
\pgfpathlineto{\pgfqpoint{5.263777in}{0.839792in}}%
\pgfpathlineto{\pgfqpoint{5.265121in}{0.765587in}}%
\pgfpathlineto{\pgfqpoint{5.265793in}{0.765587in}}%
\pgfpathlineto{\pgfqpoint{5.266466in}{0.807990in}}%
\pgfpathlineto{\pgfqpoint{5.267138in}{0.778838in}}%
\pgfpathlineto{\pgfqpoint{5.267810in}{0.778838in}}%
\pgfpathlineto{\pgfqpoint{5.267810in}{0.813290in}}%
\pgfpathlineto{\pgfqpoint{5.269154in}{0.797389in}}%
\pgfpathlineto{\pgfqpoint{5.269826in}{0.797389in}}%
\pgfpathlineto{\pgfqpoint{5.269826in}{0.786788in}}%
\pgfpathlineto{\pgfqpoint{5.271171in}{0.797389in}}%
\pgfpathlineto{\pgfqpoint{5.271843in}{0.797389in}}%
\pgfpathlineto{\pgfqpoint{5.271843in}{0.826541in}}%
\pgfpathlineto{\pgfqpoint{5.272515in}{0.784138in}}%
\pgfpathlineto{\pgfqpoint{5.273187in}{0.797389in}}%
\pgfpathlineto{\pgfqpoint{5.273859in}{0.797389in}}%
\pgfpathlineto{\pgfqpoint{5.274532in}{0.842442in}}%
\pgfpathlineto{\pgfqpoint{5.275204in}{0.760287in}}%
\pgfpathlineto{\pgfqpoint{5.275876in}{0.760287in}}%
\pgfpathlineto{\pgfqpoint{5.276548in}{0.789439in}}%
\pgfpathlineto{\pgfqpoint{5.277220in}{0.760287in}}%
\pgfpathlineto{\pgfqpoint{5.277892in}{0.760287in}}%
\pgfpathlineto{\pgfqpoint{5.277892in}{0.805340in}}%
\pgfpathlineto{\pgfqpoint{5.279237in}{0.776188in}}%
\pgfpathlineto{\pgfqpoint{5.279909in}{0.776188in}}%
\pgfpathlineto{\pgfqpoint{5.280581in}{0.765587in}}%
\pgfpathlineto{\pgfqpoint{5.281253in}{0.800039in}}%
\pgfpathlineto{\pgfqpoint{5.281926in}{0.800039in}}%
\pgfpathlineto{\pgfqpoint{5.282598in}{0.770887in}}%
\pgfpathlineto{\pgfqpoint{5.283270in}{0.829191in}}%
\pgfpathlineto{\pgfqpoint{5.283942in}{0.829191in}}%
\pgfpathlineto{\pgfqpoint{5.285286in}{0.739085in}}%
\pgfpathlineto{\pgfqpoint{5.285959in}{0.739085in}}%
\pgfpathlineto{\pgfqpoint{5.285959in}{0.786788in}}%
\pgfpathlineto{\pgfqpoint{5.287303in}{0.786788in}}%
\pgfpathlineto{\pgfqpoint{5.287975in}{0.786788in}}%
\pgfpathlineto{\pgfqpoint{5.287975in}{0.794739in}}%
\pgfpathlineto{\pgfqpoint{5.289319in}{0.768237in}}%
\pgfpathlineto{\pgfqpoint{5.290664in}{0.768237in}}%
\pgfpathlineto{\pgfqpoint{5.290664in}{0.744386in}}%
\pgfpathlineto{\pgfqpoint{5.291336in}{0.792089in}}%
\pgfpathlineto{\pgfqpoint{5.292008in}{0.778838in}}%
\pgfpathlineto{\pgfqpoint{5.292680in}{0.778838in}}%
\pgfpathlineto{\pgfqpoint{5.292680in}{0.749686in}}%
\pgfpathlineto{\pgfqpoint{5.294025in}{0.794739in}}%
\pgfpathlineto{\pgfqpoint{5.295369in}{0.794739in}}%
\pgfpathlineto{\pgfqpoint{5.295369in}{0.792089in}}%
\pgfpathlineto{\pgfqpoint{5.296713in}{0.821241in}}%
\pgfpathlineto{\pgfqpoint{5.297385in}{0.821241in}}%
\pgfpathlineto{\pgfqpoint{5.297385in}{0.765587in}}%
\pgfpathlineto{\pgfqpoint{5.298730in}{0.773537in}}%
\pgfpathlineto{\pgfqpoint{5.299402in}{0.773537in}}%
\pgfpathlineto{\pgfqpoint{5.300074in}{0.760287in}}%
\pgfpathlineto{\pgfqpoint{5.300746in}{0.842442in}}%
\pgfpathlineto{\pgfqpoint{5.301418in}{0.842442in}}%
\pgfpathlineto{\pgfqpoint{5.302091in}{0.749686in}}%
\pgfpathlineto{\pgfqpoint{5.302763in}{0.784138in}}%
\pgfpathlineto{\pgfqpoint{5.304107in}{0.784138in}}%
\pgfpathlineto{\pgfqpoint{5.304779in}{0.810640in}}%
\pgfpathlineto{\pgfqpoint{5.305451in}{0.773537in}}%
\pgfpathlineto{\pgfqpoint{5.306124in}{0.773537in}}%
\pgfpathlineto{\pgfqpoint{5.306124in}{0.762937in}}%
\pgfpathlineto{\pgfqpoint{5.307468in}{0.802689in}}%
\pgfpathlineto{\pgfqpoint{5.308140in}{0.802689in}}%
\pgfpathlineto{\pgfqpoint{5.308140in}{0.752336in}}%
\pgfpathlineto{\pgfqpoint{5.309484in}{0.770887in}}%
\pgfpathlineto{\pgfqpoint{5.310157in}{0.770887in}}%
\pgfpathlineto{\pgfqpoint{5.310829in}{0.747036in}}%
\pgfpathlineto{\pgfqpoint{5.311501in}{0.800039in}}%
\pgfpathlineto{\pgfqpoint{5.312173in}{0.800039in}}%
\pgfpathlineto{\pgfqpoint{5.312173in}{0.765587in}}%
\pgfpathlineto{\pgfqpoint{5.313517in}{0.792089in}}%
\pgfpathlineto{\pgfqpoint{5.314190in}{0.792089in}}%
\pgfpathlineto{\pgfqpoint{5.314190in}{0.762937in}}%
\pgfpathlineto{\pgfqpoint{5.315534in}{0.784138in}}%
\pgfpathlineto{\pgfqpoint{5.316206in}{0.784138in}}%
\pgfpathlineto{\pgfqpoint{5.316878in}{0.754986in}}%
\pgfpathlineto{\pgfqpoint{5.317550in}{0.797389in}}%
\pgfpathlineto{\pgfqpoint{5.318223in}{0.797389in}}%
\pgfpathlineto{\pgfqpoint{5.318223in}{0.754986in}}%
\pgfpathlineto{\pgfqpoint{5.319567in}{0.754986in}}%
\pgfpathlineto{\pgfqpoint{5.320239in}{0.754986in}}%
\pgfpathlineto{\pgfqpoint{5.320239in}{0.789439in}}%
\pgfpathlineto{\pgfqpoint{5.320911in}{0.752336in}}%
\pgfpathlineto{\pgfqpoint{5.321583in}{0.760287in}}%
\pgfpathlineto{\pgfqpoint{5.322256in}{0.760287in}}%
\pgfpathlineto{\pgfqpoint{5.322256in}{0.749686in}}%
\pgfpathlineto{\pgfqpoint{5.322928in}{0.773537in}}%
\pgfpathlineto{\pgfqpoint{5.323600in}{0.770887in}}%
\pgfpathlineto{\pgfqpoint{5.325616in}{0.770887in}}%
\pgfpathlineto{\pgfqpoint{5.326961in}{0.752336in}}%
\pgfpathlineto{\pgfqpoint{5.327633in}{0.752336in}}%
\pgfpathlineto{\pgfqpoint{5.328305in}{0.744386in}}%
\pgfpathlineto{\pgfqpoint{5.328977in}{0.778838in}}%
\pgfpathlineto{\pgfqpoint{5.330322in}{0.778838in}}%
\pgfpathlineto{\pgfqpoint{5.330322in}{0.749686in}}%
\pgfpathlineto{\pgfqpoint{5.331666in}{0.757636in}}%
\pgfpathlineto{\pgfqpoint{5.332338in}{0.757636in}}%
\pgfpathlineto{\pgfqpoint{5.333010in}{0.728484in}}%
\pgfpathlineto{\pgfqpoint{5.333682in}{0.786788in}}%
\pgfpathlineto{\pgfqpoint{5.334355in}{0.786788in}}%
\pgfpathlineto{\pgfqpoint{5.334355in}{0.749686in}}%
\pgfpathlineto{\pgfqpoint{5.335699in}{0.768237in}}%
\pgfpathlineto{\pgfqpoint{5.336371in}{0.768237in}}%
\pgfpathlineto{\pgfqpoint{5.337043in}{0.802689in}}%
\pgfpathlineto{\pgfqpoint{5.337715in}{0.765587in}}%
\pgfpathlineto{\pgfqpoint{5.339060in}{0.765587in}}%
\pgfpathlineto{\pgfqpoint{5.340404in}{0.789439in}}%
\pgfpathlineto{\pgfqpoint{5.341076in}{0.789439in}}%
\pgfpathlineto{\pgfqpoint{5.341748in}{0.739085in}}%
\pgfpathlineto{\pgfqpoint{5.342421in}{0.762937in}}%
\pgfpathlineto{\pgfqpoint{5.343093in}{0.762937in}}%
\pgfpathlineto{\pgfqpoint{5.343093in}{0.749686in}}%
\pgfpathlineto{\pgfqpoint{5.344437in}{0.770887in}}%
\pgfpathlineto{\pgfqpoint{5.345781in}{0.770887in}}%
\pgfpathlineto{\pgfqpoint{5.345781in}{0.762937in}}%
\pgfpathlineto{\pgfqpoint{5.347126in}{0.800039in}}%
\pgfpathlineto{\pgfqpoint{5.347798in}{0.800039in}}%
\pgfpathlineto{\pgfqpoint{5.348470in}{0.741735in}}%
\pgfpathlineto{\pgfqpoint{5.349142in}{0.762937in}}%
\pgfpathlineto{\pgfqpoint{5.349814in}{0.762937in}}%
\pgfpathlineto{\pgfqpoint{5.349814in}{0.752336in}}%
\pgfpathlineto{\pgfqpoint{5.351159in}{0.776188in}}%
\pgfpathlineto{\pgfqpoint{5.351831in}{0.776188in}}%
\pgfpathlineto{\pgfqpoint{5.353175in}{0.765587in}}%
\pgfpathlineto{\pgfqpoint{5.353847in}{0.765587in}}%
\pgfpathlineto{\pgfqpoint{5.354520in}{0.736435in}}%
\pgfpathlineto{\pgfqpoint{5.355192in}{0.747036in}}%
\pgfpathlineto{\pgfqpoint{5.355864in}{0.747036in}}%
\pgfpathlineto{\pgfqpoint{5.355864in}{0.786788in}}%
\pgfpathlineto{\pgfqpoint{5.357208in}{0.733785in}}%
\pgfpathlineto{\pgfqpoint{5.357880in}{0.733785in}}%
\pgfpathlineto{\pgfqpoint{5.357880in}{0.797389in}}%
\pgfpathlineto{\pgfqpoint{5.359225in}{0.770887in}}%
\pgfpathlineto{\pgfqpoint{5.360569in}{0.770887in}}%
\pgfpathlineto{\pgfqpoint{5.361913in}{0.736435in}}%
\pgfpathlineto{\pgfqpoint{5.362586in}{0.736435in}}%
\pgfpathlineto{\pgfqpoint{5.363930in}{0.760287in}}%
\pgfpathlineto{\pgfqpoint{5.364602in}{0.760287in}}%
\pgfpathlineto{\pgfqpoint{5.364602in}{0.736435in}}%
\pgfpathlineto{\pgfqpoint{5.365274in}{0.770887in}}%
\pgfpathlineto{\pgfqpoint{5.365946in}{0.736435in}}%
\pgfpathlineto{\pgfqpoint{5.366619in}{0.736435in}}%
\pgfpathlineto{\pgfqpoint{5.366619in}{0.765587in}}%
\pgfpathlineto{\pgfqpoint{5.367963in}{0.725834in}}%
\pgfpathlineto{\pgfqpoint{5.368635in}{0.725834in}}%
\pgfpathlineto{\pgfqpoint{5.368635in}{0.747036in}}%
\pgfpathlineto{\pgfqpoint{5.369979in}{0.747036in}}%
\pgfpathlineto{\pgfqpoint{5.370652in}{0.747036in}}%
\pgfpathlineto{\pgfqpoint{5.370652in}{0.733785in}}%
\pgfpathlineto{\pgfqpoint{5.371996in}{0.773537in}}%
\pgfpathlineto{\pgfqpoint{5.372668in}{0.773537in}}%
\pgfpathlineto{\pgfqpoint{5.373340in}{0.725834in}}%
\pgfpathlineto{\pgfqpoint{5.374012in}{0.731135in}}%
\pgfpathlineto{\pgfqpoint{5.374685in}{0.731135in}}%
\pgfpathlineto{\pgfqpoint{5.376029in}{0.768237in}}%
\pgfpathlineto{\pgfqpoint{5.376701in}{0.768237in}}%
\pgfpathlineto{\pgfqpoint{5.377373in}{0.741735in}}%
\pgfpathlineto{\pgfqpoint{5.378045in}{0.752336in}}%
\pgfpathlineto{\pgfqpoint{5.378718in}{0.752336in}}%
\pgfpathlineto{\pgfqpoint{5.378718in}{0.736435in}}%
\pgfpathlineto{\pgfqpoint{5.380062in}{0.749686in}}%
\pgfpathlineto{\pgfqpoint{5.380734in}{0.749686in}}%
\pgfpathlineto{\pgfqpoint{5.381406in}{0.720534in}}%
\pgfpathlineto{\pgfqpoint{5.382078in}{0.754986in}}%
\pgfpathlineto{\pgfqpoint{5.382751in}{0.754986in}}%
\pgfpathlineto{\pgfqpoint{5.382751in}{0.770887in}}%
\pgfpathlineto{\pgfqpoint{5.383423in}{0.725834in}}%
\pgfpathlineto{\pgfqpoint{5.384095in}{0.739085in}}%
\pgfpathlineto{\pgfqpoint{5.384767in}{0.739085in}}%
\pgfpathlineto{\pgfqpoint{5.384767in}{0.765587in}}%
\pgfpathlineto{\pgfqpoint{5.386111in}{0.744386in}}%
\pgfpathlineto{\pgfqpoint{5.386784in}{0.744386in}}%
\pgfpathlineto{\pgfqpoint{5.386784in}{0.765587in}}%
\pgfpathlineto{\pgfqpoint{5.387456in}{0.723184in}}%
\pgfpathlineto{\pgfqpoint{5.388128in}{0.752336in}}%
\pgfpathlineto{\pgfqpoint{5.388800in}{0.752336in}}%
\pgfpathlineto{\pgfqpoint{5.388800in}{0.770887in}}%
\pgfpathlineto{\pgfqpoint{5.390144in}{0.752336in}}%
\pgfpathlineto{\pgfqpoint{5.390817in}{0.752336in}}%
\pgfpathlineto{\pgfqpoint{5.390817in}{0.741735in}}%
\pgfpathlineto{\pgfqpoint{5.391489in}{0.773537in}}%
\pgfpathlineto{\pgfqpoint{5.392161in}{0.744386in}}%
\pgfpathlineto{\pgfqpoint{5.392833in}{0.744386in}}%
\pgfpathlineto{\pgfqpoint{5.392833in}{0.725834in}}%
\pgfpathlineto{\pgfqpoint{5.394178in}{0.760287in}}%
\pgfpathlineto{\pgfqpoint{5.394850in}{0.760287in}}%
\pgfpathlineto{\pgfqpoint{5.394850in}{0.717884in}}%
\pgfpathlineto{\pgfqpoint{5.396194in}{0.749686in}}%
\pgfpathlineto{\pgfqpoint{5.396866in}{0.749686in}}%
\pgfpathlineto{\pgfqpoint{5.396866in}{0.754986in}}%
\pgfpathlineto{\pgfqpoint{5.398211in}{0.725834in}}%
\pgfpathlineto{\pgfqpoint{5.398883in}{0.725834in}}%
\pgfpathlineto{\pgfqpoint{5.398883in}{0.747036in}}%
\pgfpathlineto{\pgfqpoint{5.399555in}{0.715234in}}%
\pgfpathlineto{\pgfqpoint{5.400227in}{0.717884in}}%
\pgfpathlineto{\pgfqpoint{5.400899in}{0.717884in}}%
\pgfpathlineto{\pgfqpoint{5.400899in}{0.760287in}}%
\pgfpathlineto{\pgfqpoint{5.402244in}{0.747036in}}%
\pgfpathlineto{\pgfqpoint{5.402916in}{0.747036in}}%
\pgfpathlineto{\pgfqpoint{5.402916in}{0.776188in}}%
\pgfpathlineto{\pgfqpoint{5.404260in}{0.765587in}}%
\pgfpathlineto{\pgfqpoint{5.404932in}{0.765587in}}%
\pgfpathlineto{\pgfqpoint{5.405604in}{0.739085in}}%
\pgfpathlineto{\pgfqpoint{5.406277in}{0.754986in}}%
\pgfpathlineto{\pgfqpoint{5.406949in}{0.754986in}}%
\pgfpathlineto{\pgfqpoint{5.406949in}{0.760287in}}%
\pgfpathlineto{\pgfqpoint{5.408293in}{0.736435in}}%
\pgfpathlineto{\pgfqpoint{5.408965in}{0.736435in}}%
\pgfpathlineto{\pgfqpoint{5.409637in}{0.752336in}}%
\pgfpathlineto{\pgfqpoint{5.410310in}{0.723184in}}%
\pgfpathlineto{\pgfqpoint{5.410982in}{0.723184in}}%
\pgfpathlineto{\pgfqpoint{5.410982in}{0.765587in}}%
\pgfpathlineto{\pgfqpoint{5.412326in}{0.731135in}}%
\pgfpathlineto{\pgfqpoint{5.412998in}{0.731135in}}%
\pgfpathlineto{\pgfqpoint{5.412998in}{0.760287in}}%
\pgfpathlineto{\pgfqpoint{5.414343in}{0.725834in}}%
\pgfpathlineto{\pgfqpoint{5.415015in}{0.725834in}}%
\pgfpathlineto{\pgfqpoint{5.415015in}{0.765587in}}%
\pgfpathlineto{\pgfqpoint{5.416359in}{0.728484in}}%
\pgfpathlineto{\pgfqpoint{5.417031in}{0.728484in}}%
\pgfpathlineto{\pgfqpoint{5.417031in}{0.723184in}}%
\pgfpathlineto{\pgfqpoint{5.417703in}{0.770887in}}%
\pgfpathlineto{\pgfqpoint{5.418376in}{0.723184in}}%
\pgfpathlineto{\pgfqpoint{5.419048in}{0.723184in}}%
\pgfpathlineto{\pgfqpoint{5.419048in}{0.715234in}}%
\pgfpathlineto{\pgfqpoint{5.419720in}{0.733785in}}%
\pgfpathlineto{\pgfqpoint{5.420392in}{0.731135in}}%
\pgfpathlineto{\pgfqpoint{5.421064in}{0.731135in}}%
\pgfpathlineto{\pgfqpoint{5.421736in}{0.725834in}}%
\pgfpathlineto{\pgfqpoint{5.422409in}{0.754986in}}%
\pgfpathlineto{\pgfqpoint{5.423081in}{0.754986in}}%
\pgfpathlineto{\pgfqpoint{5.424425in}{0.778838in}}%
\pgfpathlineto{\pgfqpoint{5.425097in}{0.778838in}}%
\pgfpathlineto{\pgfqpoint{5.426442in}{0.717884in}}%
\pgfpathlineto{\pgfqpoint{5.427114in}{0.717884in}}%
\pgfpathlineto{\pgfqpoint{5.427114in}{0.715234in}}%
\pgfpathlineto{\pgfqpoint{5.428458in}{0.754986in}}%
\pgfpathlineto{\pgfqpoint{5.429130in}{0.754986in}}%
\pgfpathlineto{\pgfqpoint{5.429802in}{0.694032in}}%
\pgfpathlineto{\pgfqpoint{5.430475in}{0.720534in}}%
\pgfpathlineto{\pgfqpoint{5.431147in}{0.720534in}}%
\pgfpathlineto{\pgfqpoint{5.431819in}{0.776188in}}%
\pgfpathlineto{\pgfqpoint{5.432491in}{0.712583in}}%
\pgfpathlineto{\pgfqpoint{5.433163in}{0.712583in}}%
\pgfpathlineto{\pgfqpoint{5.433835in}{0.736435in}}%
\pgfpathlineto{\pgfqpoint{5.434508in}{0.699333in}}%
\pgfpathlineto{\pgfqpoint{5.435180in}{0.699333in}}%
\pgfpathlineto{\pgfqpoint{5.435180in}{0.770887in}}%
\pgfpathlineto{\pgfqpoint{5.436524in}{0.725834in}}%
\pgfpathlineto{\pgfqpoint{5.437196in}{0.725834in}}%
\pgfpathlineto{\pgfqpoint{5.437196in}{0.752336in}}%
\pgfpathlineto{\pgfqpoint{5.437868in}{0.712583in}}%
\pgfpathlineto{\pgfqpoint{5.438541in}{0.733785in}}%
\pgfpathlineto{\pgfqpoint{5.439213in}{0.733785in}}%
\pgfpathlineto{\pgfqpoint{5.439213in}{0.757636in}}%
\pgfpathlineto{\pgfqpoint{5.440557in}{0.704633in}}%
\pgfpathlineto{\pgfqpoint{5.441229in}{0.704633in}}%
\pgfpathlineto{\pgfqpoint{5.441901in}{0.752336in}}%
\pgfpathlineto{\pgfqpoint{5.442574in}{0.747036in}}%
\pgfpathlineto{\pgfqpoint{5.443246in}{0.747036in}}%
\pgfpathlineto{\pgfqpoint{5.443246in}{0.752336in}}%
\pgfpathlineto{\pgfqpoint{5.444590in}{0.728484in}}%
\pgfpathlineto{\pgfqpoint{5.445934in}{0.728484in}}%
\pgfpathlineto{\pgfqpoint{5.445934in}{0.754986in}}%
\pgfpathlineto{\pgfqpoint{5.447279in}{0.712583in}}%
\pgfpathlineto{\pgfqpoint{5.447951in}{0.712583in}}%
\pgfpathlineto{\pgfqpoint{5.448623in}{0.741735in}}%
\pgfpathlineto{\pgfqpoint{5.449295in}{0.720534in}}%
\pgfpathlineto{\pgfqpoint{5.449967in}{0.720534in}}%
\pgfpathlineto{\pgfqpoint{5.450640in}{0.757636in}}%
\pgfpathlineto{\pgfqpoint{5.451312in}{0.754986in}}%
\pgfpathlineto{\pgfqpoint{5.451984in}{0.754986in}}%
\pgfpathlineto{\pgfqpoint{5.453328in}{0.723184in}}%
\pgfpathlineto{\pgfqpoint{5.454000in}{0.723184in}}%
\pgfpathlineto{\pgfqpoint{5.454000in}{0.749686in}}%
\pgfpathlineto{\pgfqpoint{5.455345in}{0.715234in}}%
\pgfpathlineto{\pgfqpoint{5.456017in}{0.715234in}}%
\pgfpathlineto{\pgfqpoint{5.456017in}{0.757636in}}%
\pgfpathlineto{\pgfqpoint{5.456689in}{0.699333in}}%
\pgfpathlineto{\pgfqpoint{5.457361in}{0.752336in}}%
\pgfpathlineto{\pgfqpoint{5.458033in}{0.752336in}}%
\pgfpathlineto{\pgfqpoint{5.458033in}{0.773537in}}%
\pgfpathlineto{\pgfqpoint{5.458706in}{0.728484in}}%
\pgfpathlineto{\pgfqpoint{5.459378in}{0.733785in}}%
\pgfpathlineto{\pgfqpoint{5.460722in}{0.733785in}}%
\pgfpathlineto{\pgfqpoint{5.460722in}{0.765587in}}%
\pgfpathlineto{\pgfqpoint{5.462066in}{0.725834in}}%
\pgfpathlineto{\pgfqpoint{5.462739in}{0.725834in}}%
\pgfpathlineto{\pgfqpoint{5.463411in}{0.717884in}}%
\pgfpathlineto{\pgfqpoint{5.464083in}{0.762937in}}%
\pgfpathlineto{\pgfqpoint{5.464755in}{0.762937in}}%
\pgfpathlineto{\pgfqpoint{5.465427in}{0.739085in}}%
\pgfpathlineto{\pgfqpoint{5.466099in}{0.744386in}}%
\pgfpathlineto{\pgfqpoint{5.466772in}{0.744386in}}%
\pgfpathlineto{\pgfqpoint{5.466772in}{0.749686in}}%
\pgfpathlineto{\pgfqpoint{5.468116in}{0.704633in}}%
\pgfpathlineto{\pgfqpoint{5.468788in}{0.704633in}}%
\pgfpathlineto{\pgfqpoint{5.469460in}{0.736435in}}%
\pgfpathlineto{\pgfqpoint{5.470132in}{0.709933in}}%
\pgfpathlineto{\pgfqpoint{5.470805in}{0.709933in}}%
\pgfpathlineto{\pgfqpoint{5.470805in}{0.773537in}}%
\pgfpathlineto{\pgfqpoint{5.472149in}{0.717884in}}%
\pgfpathlineto{\pgfqpoint{5.472821in}{0.717884in}}%
\pgfpathlineto{\pgfqpoint{5.472821in}{0.747036in}}%
\pgfpathlineto{\pgfqpoint{5.474165in}{0.731135in}}%
\pgfpathlineto{\pgfqpoint{5.474838in}{0.731135in}}%
\pgfpathlineto{\pgfqpoint{5.474838in}{0.728484in}}%
\pgfpathlineto{\pgfqpoint{5.475510in}{0.757636in}}%
\pgfpathlineto{\pgfqpoint{5.476182in}{0.757636in}}%
\pgfpathlineto{\pgfqpoint{5.476854in}{0.757636in}}%
\pgfpathlineto{\pgfqpoint{5.476854in}{0.709933in}}%
\pgfpathlineto{\pgfqpoint{5.478198in}{0.762937in}}%
\pgfpathlineto{\pgfqpoint{5.478871in}{0.762937in}}%
\pgfpathlineto{\pgfqpoint{5.479543in}{0.715234in}}%
\pgfpathlineto{\pgfqpoint{5.479543in}{0.765587in}}%
\pgfpathlineto{\pgfqpoint{5.480215in}{0.715234in}}%
\pgfpathlineto{\pgfqpoint{5.480887in}{0.715234in}}%
\pgfpathlineto{\pgfqpoint{5.480887in}{0.770887in}}%
\pgfpathlineto{\pgfqpoint{5.482231in}{0.728484in}}%
\pgfpathlineto{\pgfqpoint{5.482904in}{0.728484in}}%
\pgfpathlineto{\pgfqpoint{5.483576in}{0.749686in}}%
\pgfpathlineto{\pgfqpoint{5.484248in}{0.712583in}}%
\pgfpathlineto{\pgfqpoint{5.484920in}{0.712583in}}%
\pgfpathlineto{\pgfqpoint{5.484920in}{0.747036in}}%
\pgfpathlineto{\pgfqpoint{5.486264in}{0.715234in}}%
\pgfpathlineto{\pgfqpoint{5.486937in}{0.715234in}}%
\pgfpathlineto{\pgfqpoint{5.488281in}{0.776188in}}%
\pgfpathlineto{\pgfqpoint{5.488953in}{0.776188in}}%
\pgfpathlineto{\pgfqpoint{5.489625in}{0.709933in}}%
\pgfpathlineto{\pgfqpoint{5.490297in}{0.723184in}}%
\pgfpathlineto{\pgfqpoint{5.490970in}{0.723184in}}%
\pgfpathlineto{\pgfqpoint{5.491642in}{0.773537in}}%
\pgfpathlineto{\pgfqpoint{5.492314in}{0.717884in}}%
\pgfpathlineto{\pgfqpoint{5.492986in}{0.717884in}}%
\pgfpathlineto{\pgfqpoint{5.494330in}{0.757636in}}%
\pgfpathlineto{\pgfqpoint{5.495003in}{0.757636in}}%
\pgfpathlineto{\pgfqpoint{5.496347in}{0.707283in}}%
\pgfpathlineto{\pgfqpoint{5.497019in}{0.707283in}}%
\pgfpathlineto{\pgfqpoint{5.497691in}{0.768237in}}%
\pgfpathlineto{\pgfqpoint{5.498363in}{0.723184in}}%
\pgfpathlineto{\pgfqpoint{5.499036in}{0.723184in}}%
\pgfpathlineto{\pgfqpoint{5.500380in}{0.744386in}}%
\pgfpathlineto{\pgfqpoint{5.501052in}{0.744386in}}%
\pgfpathlineto{\pgfqpoint{5.501724in}{0.723184in}}%
\pgfpathlineto{\pgfqpoint{5.502396in}{0.728484in}}%
\pgfpathlineto{\pgfqpoint{5.503069in}{0.728484in}}%
\pgfpathlineto{\pgfqpoint{5.503069in}{0.707283in}}%
\pgfpathlineto{\pgfqpoint{5.504413in}{0.712583in}}%
\pgfpathlineto{\pgfqpoint{5.505085in}{0.712583in}}%
\pgfpathlineto{\pgfqpoint{5.506429in}{0.760287in}}%
\pgfpathlineto{\pgfqpoint{5.507102in}{0.760287in}}%
\pgfpathlineto{\pgfqpoint{5.507774in}{0.704633in}}%
\pgfpathlineto{\pgfqpoint{5.508446in}{0.739085in}}%
\pgfpathlineto{\pgfqpoint{5.509118in}{0.739085in}}%
\pgfpathlineto{\pgfqpoint{5.509118in}{0.744386in}}%
\pgfpathlineto{\pgfqpoint{5.509790in}{0.704633in}}%
\pgfpathlineto{\pgfqpoint{5.510463in}{0.731135in}}%
\pgfpathlineto{\pgfqpoint{5.511135in}{0.731135in}}%
\pgfpathlineto{\pgfqpoint{5.511135in}{0.720534in}}%
\pgfpathlineto{\pgfqpoint{5.511807in}{0.736435in}}%
\pgfpathlineto{\pgfqpoint{5.512479in}{0.728484in}}%
\pgfpathlineto{\pgfqpoint{5.513151in}{0.728484in}}%
\pgfpathlineto{\pgfqpoint{5.513823in}{0.723184in}}%
\pgfpathlineto{\pgfqpoint{5.514496in}{0.757636in}}%
\pgfpathlineto{\pgfqpoint{5.515168in}{0.757636in}}%
\pgfpathlineto{\pgfqpoint{5.515840in}{0.728484in}}%
\pgfpathlineto{\pgfqpoint{5.516512in}{0.731135in}}%
\pgfpathlineto{\pgfqpoint{5.517184in}{0.731135in}}%
\pgfpathlineto{\pgfqpoint{5.517856in}{0.752336in}}%
\pgfpathlineto{\pgfqpoint{5.518529in}{0.704633in}}%
\pgfpathlineto{\pgfqpoint{5.519201in}{0.704633in}}%
\pgfpathlineto{\pgfqpoint{5.519873in}{0.723184in}}%
\pgfpathlineto{\pgfqpoint{5.520545in}{0.715234in}}%
\pgfpathlineto{\pgfqpoint{5.521217in}{0.715234in}}%
\pgfpathlineto{\pgfqpoint{5.521889in}{0.699333in}}%
\pgfpathlineto{\pgfqpoint{5.522562in}{0.752336in}}%
\pgfpathlineto{\pgfqpoint{5.523234in}{0.752336in}}%
\pgfpathlineto{\pgfqpoint{5.523234in}{0.699333in}}%
\pgfpathlineto{\pgfqpoint{5.524578in}{0.754986in}}%
\pgfpathlineto{\pgfqpoint{5.525250in}{0.754986in}}%
\pgfpathlineto{\pgfqpoint{5.525250in}{0.723184in}}%
\pgfpathlineto{\pgfqpoint{5.526595in}{0.739085in}}%
\pgfpathlineto{\pgfqpoint{5.527267in}{0.739085in}}%
\pgfpathlineto{\pgfqpoint{5.527267in}{0.720534in}}%
\pgfpathlineto{\pgfqpoint{5.528611in}{0.736435in}}%
\pgfpathlineto{\pgfqpoint{5.529283in}{0.736435in}}%
\pgfpathlineto{\pgfqpoint{5.529955in}{0.762937in}}%
\pgfpathlineto{\pgfqpoint{5.530628in}{0.704633in}}%
\pgfpathlineto{\pgfqpoint{5.531300in}{0.704633in}}%
\pgfpathlineto{\pgfqpoint{5.531972in}{0.733785in}}%
\pgfpathlineto{\pgfqpoint{5.532644in}{0.728484in}}%
\pgfpathlineto{\pgfqpoint{5.533316in}{0.728484in}}%
\pgfpathlineto{\pgfqpoint{5.533316in}{0.725834in}}%
\pgfpathlineto{\pgfqpoint{5.533988in}{0.731135in}}%
\pgfpathlineto{\pgfqpoint{5.534661in}{0.725834in}}%
\pgfpathlineto{\pgfqpoint{5.535333in}{0.725834in}}%
\pgfpathlineto{\pgfqpoint{5.536677in}{0.701983in}}%
\pgfpathlineto{\pgfqpoint{5.537349in}{0.701983in}}%
\pgfpathlineto{\pgfqpoint{5.538021in}{0.733785in}}%
\pgfpathlineto{\pgfqpoint{5.538694in}{0.733785in}}%
\pgfpathlineto{\pgfqpoint{5.539366in}{0.733785in}}%
\pgfpathlineto{\pgfqpoint{5.539366in}{0.747036in}}%
\pgfpathlineto{\pgfqpoint{5.540038in}{0.712583in}}%
\pgfpathlineto{\pgfqpoint{5.540710in}{0.731135in}}%
\pgfpathlineto{\pgfqpoint{5.541382in}{0.731135in}}%
\pgfpathlineto{\pgfqpoint{5.541382in}{0.709933in}}%
\pgfpathlineto{\pgfqpoint{5.542727in}{0.733785in}}%
\pgfpathlineto{\pgfqpoint{5.543399in}{0.733785in}}%
\pgfpathlineto{\pgfqpoint{5.544743in}{0.720534in}}%
\pgfpathlineto{\pgfqpoint{5.545415in}{0.720534in}}%
\pgfpathlineto{\pgfqpoint{5.546087in}{0.725834in}}%
\pgfpathlineto{\pgfqpoint{5.546760in}{0.712583in}}%
\pgfpathlineto{\pgfqpoint{5.547432in}{0.712583in}}%
\pgfpathlineto{\pgfqpoint{5.547432in}{0.701983in}}%
\pgfpathlineto{\pgfqpoint{5.548776in}{0.744386in}}%
\pgfpathlineto{\pgfqpoint{5.549448in}{0.744386in}}%
\pgfpathlineto{\pgfqpoint{5.550120in}{0.733785in}}%
\pgfpathlineto{\pgfqpoint{5.550793in}{0.741735in}}%
\pgfpathlineto{\pgfqpoint{5.551465in}{0.741735in}}%
\pgfpathlineto{\pgfqpoint{5.552137in}{0.712583in}}%
\pgfpathlineto{\pgfqpoint{5.552809in}{0.715234in}}%
\pgfpathlineto{\pgfqpoint{5.553481in}{0.715234in}}%
\pgfpathlineto{\pgfqpoint{5.553481in}{0.741735in}}%
\pgfpathlineto{\pgfqpoint{5.554826in}{0.723184in}}%
\pgfpathlineto{\pgfqpoint{5.555498in}{0.723184in}}%
\pgfpathlineto{\pgfqpoint{5.555498in}{0.739085in}}%
\pgfpathlineto{\pgfqpoint{5.556842in}{0.731135in}}%
\pgfpathlineto{\pgfqpoint{5.557514in}{0.731135in}}%
\pgfpathlineto{\pgfqpoint{5.558186in}{0.715234in}}%
\pgfpathlineto{\pgfqpoint{5.558859in}{0.728484in}}%
\pgfpathlineto{\pgfqpoint{5.559531in}{0.728484in}}%
\pgfpathlineto{\pgfqpoint{5.559531in}{0.744386in}}%
\pgfpathlineto{\pgfqpoint{5.560875in}{0.720534in}}%
\pgfpathlineto{\pgfqpoint{5.561547in}{0.720534in}}%
\pgfpathlineto{\pgfqpoint{5.561547in}{0.731135in}}%
\pgfpathlineto{\pgfqpoint{5.562219in}{0.691382in}}%
\pgfpathlineto{\pgfqpoint{5.562892in}{0.720534in}}%
\pgfpathlineto{\pgfqpoint{5.563564in}{0.720534in}}%
\pgfpathlineto{\pgfqpoint{5.563564in}{0.749686in}}%
\pgfpathlineto{\pgfqpoint{5.564908in}{0.725834in}}%
\pgfpathlineto{\pgfqpoint{5.565580in}{0.725834in}}%
\pgfpathlineto{\pgfqpoint{5.565580in}{0.709933in}}%
\pgfpathlineto{\pgfqpoint{5.566925in}{0.736435in}}%
\pgfpathlineto{\pgfqpoint{5.567597in}{0.736435in}}%
\pgfpathlineto{\pgfqpoint{5.567597in}{0.683431in}}%
\pgfpathlineto{\pgfqpoint{5.568941in}{0.754986in}}%
\pgfpathlineto{\pgfqpoint{5.569613in}{0.754986in}}%
\pgfpathlineto{\pgfqpoint{5.569613in}{0.712583in}}%
\pgfpathlineto{\pgfqpoint{5.570958in}{0.728484in}}%
\pgfpathlineto{\pgfqpoint{5.571630in}{0.728484in}}%
\pgfpathlineto{\pgfqpoint{5.571630in}{0.717884in}}%
\pgfpathlineto{\pgfqpoint{5.572302in}{0.741735in}}%
\pgfpathlineto{\pgfqpoint{5.572974in}{0.725834in}}%
\pgfpathlineto{\pgfqpoint{5.573646in}{0.725834in}}%
\pgfpathlineto{\pgfqpoint{5.574318in}{0.712583in}}%
\pgfpathlineto{\pgfqpoint{5.574991in}{0.736435in}}%
\pgfpathlineto{\pgfqpoint{5.575663in}{0.736435in}}%
\pgfpathlineto{\pgfqpoint{5.575663in}{0.720534in}}%
\pgfpathlineto{\pgfqpoint{5.576335in}{0.739085in}}%
\pgfpathlineto{\pgfqpoint{5.577007in}{0.723184in}}%
\pgfpathlineto{\pgfqpoint{5.577679in}{0.723184in}}%
\pgfpathlineto{\pgfqpoint{5.577679in}{0.741735in}}%
\pgfpathlineto{\pgfqpoint{5.579024in}{0.707283in}}%
\pgfpathlineto{\pgfqpoint{5.579696in}{0.707283in}}%
\pgfpathlineto{\pgfqpoint{5.581040in}{0.717884in}}%
\pgfpathlineto{\pgfqpoint{5.581712in}{0.717884in}}%
\pgfpathlineto{\pgfqpoint{5.581712in}{0.707283in}}%
\pgfpathlineto{\pgfqpoint{5.582384in}{0.739085in}}%
\pgfpathlineto{\pgfqpoint{5.583057in}{0.725834in}}%
\pgfpathlineto{\pgfqpoint{5.583729in}{0.725834in}}%
\pgfpathlineto{\pgfqpoint{5.583729in}{0.741735in}}%
\pgfpathlineto{\pgfqpoint{5.585073in}{0.715234in}}%
\pgfpathlineto{\pgfqpoint{5.585745in}{0.715234in}}%
\pgfpathlineto{\pgfqpoint{5.585745in}{0.733785in}}%
\pgfpathlineto{\pgfqpoint{5.586417in}{0.709933in}}%
\pgfpathlineto{\pgfqpoint{5.587090in}{0.725834in}}%
\pgfpathlineto{\pgfqpoint{5.587762in}{0.725834in}}%
\pgfpathlineto{\pgfqpoint{5.589106in}{0.701983in}}%
\pgfpathlineto{\pgfqpoint{5.589778in}{0.701983in}}%
\pgfpathlineto{\pgfqpoint{5.591123in}{0.739085in}}%
\pgfpathlineto{\pgfqpoint{5.591795in}{0.739085in}}%
\pgfpathlineto{\pgfqpoint{5.591795in}{0.717884in}}%
\pgfpathlineto{\pgfqpoint{5.593139in}{0.720534in}}%
\pgfpathlineto{\pgfqpoint{5.593811in}{0.720534in}}%
\pgfpathlineto{\pgfqpoint{5.593811in}{0.725834in}}%
\pgfpathlineto{\pgfqpoint{5.594483in}{0.699333in}}%
\pgfpathlineto{\pgfqpoint{5.595156in}{0.720534in}}%
\pgfpathlineto{\pgfqpoint{5.595828in}{0.720534in}}%
\pgfpathlineto{\pgfqpoint{5.596500in}{0.696682in}}%
\pgfpathlineto{\pgfqpoint{5.597172in}{0.731135in}}%
\pgfpathlineto{\pgfqpoint{5.597844in}{0.731135in}}%
\pgfpathlineto{\pgfqpoint{5.597844in}{0.704633in}}%
\pgfpathlineto{\pgfqpoint{5.599189in}{0.712583in}}%
\pgfpathlineto{\pgfqpoint{5.599861in}{0.712583in}}%
\pgfpathlineto{\pgfqpoint{5.601205in}{0.728484in}}%
\pgfpathlineto{\pgfqpoint{5.601877in}{0.728484in}}%
\pgfpathlineto{\pgfqpoint{5.601877in}{0.699333in}}%
\pgfpathlineto{\pgfqpoint{5.603222in}{0.760287in}}%
\pgfpathlineto{\pgfqpoint{5.603894in}{0.760287in}}%
\pgfpathlineto{\pgfqpoint{5.605238in}{0.717884in}}%
\pgfpathlineto{\pgfqpoint{5.605910in}{0.717884in}}%
\pgfpathlineto{\pgfqpoint{5.607255in}{0.741735in}}%
\pgfpathlineto{\pgfqpoint{5.607927in}{0.741735in}}%
\pgfpathlineto{\pgfqpoint{5.607927in}{0.699333in}}%
\pgfpathlineto{\pgfqpoint{5.609271in}{0.707283in}}%
\pgfpathlineto{\pgfqpoint{5.609943in}{0.707283in}}%
\pgfpathlineto{\pgfqpoint{5.609943in}{0.749686in}}%
\pgfpathlineto{\pgfqpoint{5.610615in}{0.699333in}}%
\pgfpathlineto{\pgfqpoint{5.611288in}{0.699333in}}%
\pgfpathlineto{\pgfqpoint{5.611960in}{0.699333in}}%
\pgfpathlineto{\pgfqpoint{5.611960in}{0.733785in}}%
\pgfpathlineto{\pgfqpoint{5.613304in}{0.691382in}}%
\pgfpathlineto{\pgfqpoint{5.613976in}{0.691382in}}%
\pgfpathlineto{\pgfqpoint{5.614648in}{0.731135in}}%
\pgfpathlineto{\pgfqpoint{5.615321in}{0.717884in}}%
\pgfpathlineto{\pgfqpoint{5.615993in}{0.717884in}}%
\pgfpathlineto{\pgfqpoint{5.616665in}{0.699333in}}%
\pgfpathlineto{\pgfqpoint{5.617337in}{0.739085in}}%
\pgfpathlineto{\pgfqpoint{5.618009in}{0.739085in}}%
\pgfpathlineto{\pgfqpoint{5.618009in}{0.701983in}}%
\pgfpathlineto{\pgfqpoint{5.619354in}{0.715234in}}%
\pgfpathlineto{\pgfqpoint{5.620026in}{0.715234in}}%
\pgfpathlineto{\pgfqpoint{5.620698in}{0.723184in}}%
\pgfpathlineto{\pgfqpoint{5.621370in}{0.717884in}}%
\pgfpathlineto{\pgfqpoint{5.622042in}{0.717884in}}%
\pgfpathlineto{\pgfqpoint{5.622042in}{0.739085in}}%
\pgfpathlineto{\pgfqpoint{5.623387in}{0.731135in}}%
\pgfpathlineto{\pgfqpoint{5.624059in}{0.731135in}}%
\pgfpathlineto{\pgfqpoint{5.624059in}{0.747036in}}%
\pgfpathlineto{\pgfqpoint{5.625403in}{0.707283in}}%
\pgfpathlineto{\pgfqpoint{5.626075in}{0.707283in}}%
\pgfpathlineto{\pgfqpoint{5.626075in}{0.728484in}}%
\pgfpathlineto{\pgfqpoint{5.627420in}{0.709933in}}%
\pgfpathlineto{\pgfqpoint{5.628092in}{0.709933in}}%
\pgfpathlineto{\pgfqpoint{5.628092in}{0.736435in}}%
\pgfpathlineto{\pgfqpoint{5.628764in}{0.696682in}}%
\pgfpathlineto{\pgfqpoint{5.629436in}{0.731135in}}%
\pgfpathlineto{\pgfqpoint{5.630781in}{0.731135in}}%
\pgfpathlineto{\pgfqpoint{5.630781in}{0.739085in}}%
\pgfpathlineto{\pgfqpoint{5.632125in}{0.709933in}}%
\pgfpathlineto{\pgfqpoint{5.632797in}{0.709933in}}%
\pgfpathlineto{\pgfqpoint{5.632797in}{0.699333in}}%
\pgfpathlineto{\pgfqpoint{5.633469in}{0.736435in}}%
\pgfpathlineto{\pgfqpoint{5.634141in}{0.723184in}}%
\pgfpathlineto{\pgfqpoint{5.634814in}{0.723184in}}%
\pgfpathlineto{\pgfqpoint{5.634814in}{0.728484in}}%
\pgfpathlineto{\pgfqpoint{5.635486in}{0.712583in}}%
\pgfpathlineto{\pgfqpoint{5.636158in}{0.720534in}}%
\pgfpathlineto{\pgfqpoint{5.636830in}{0.720534in}}%
\pgfpathlineto{\pgfqpoint{5.637502in}{0.709933in}}%
\pgfpathlineto{\pgfqpoint{5.638174in}{0.725834in}}%
\pgfpathlineto{\pgfqpoint{5.638847in}{0.725834in}}%
\pgfpathlineto{\pgfqpoint{5.638847in}{0.712583in}}%
\pgfpathlineto{\pgfqpoint{5.639519in}{0.728484in}}%
\pgfpathlineto{\pgfqpoint{5.640191in}{0.720534in}}%
\pgfpathlineto{\pgfqpoint{5.640863in}{0.720534in}}%
\pgfpathlineto{\pgfqpoint{5.640863in}{0.739085in}}%
\pgfpathlineto{\pgfqpoint{5.642207in}{0.731135in}}%
\pgfpathlineto{\pgfqpoint{5.642880in}{0.731135in}}%
\pgfpathlineto{\pgfqpoint{5.643552in}{0.744386in}}%
\pgfpathlineto{\pgfqpoint{5.644224in}{0.699333in}}%
\pgfpathlineto{\pgfqpoint{5.644896in}{0.699333in}}%
\pgfpathlineto{\pgfqpoint{5.644896in}{0.725834in}}%
\pgfpathlineto{\pgfqpoint{5.646240in}{0.720534in}}%
\pgfpathlineto{\pgfqpoint{5.646913in}{0.720534in}}%
\pgfpathlineto{\pgfqpoint{5.646913in}{0.707283in}}%
\pgfpathlineto{\pgfqpoint{5.648257in}{0.733785in}}%
\pgfpathlineto{\pgfqpoint{5.648929in}{0.733785in}}%
\pgfpathlineto{\pgfqpoint{5.650273in}{0.696682in}}%
\pgfpathlineto{\pgfqpoint{5.650946in}{0.696682in}}%
\pgfpathlineto{\pgfqpoint{5.651618in}{0.744386in}}%
\pgfpathlineto{\pgfqpoint{5.652290in}{0.707283in}}%
\pgfpathlineto{\pgfqpoint{5.652962in}{0.707283in}}%
\pgfpathlineto{\pgfqpoint{5.652962in}{0.696682in}}%
\pgfpathlineto{\pgfqpoint{5.653634in}{0.715234in}}%
\pgfpathlineto{\pgfqpoint{5.654306in}{0.712583in}}%
\pgfpathlineto{\pgfqpoint{5.654979in}{0.712583in}}%
\pgfpathlineto{\pgfqpoint{5.655651in}{0.728484in}}%
\pgfpathlineto{\pgfqpoint{5.656323in}{0.701983in}}%
\pgfpathlineto{\pgfqpoint{5.656995in}{0.701983in}}%
\pgfpathlineto{\pgfqpoint{5.657667in}{0.723184in}}%
\pgfpathlineto{\pgfqpoint{5.658339in}{0.720534in}}%
\pgfpathlineto{\pgfqpoint{5.659012in}{0.720534in}}%
\pgfpathlineto{\pgfqpoint{5.659684in}{0.686082in}}%
\pgfpathlineto{\pgfqpoint{5.660356in}{0.731135in}}%
\pgfpathlineto{\pgfqpoint{5.661028in}{0.731135in}}%
\pgfpathlineto{\pgfqpoint{5.661700in}{0.707283in}}%
\pgfpathlineto{\pgfqpoint{5.662372in}{0.728484in}}%
\pgfpathlineto{\pgfqpoint{5.663045in}{0.728484in}}%
\pgfpathlineto{\pgfqpoint{5.664389in}{0.696682in}}%
\pgfpathlineto{\pgfqpoint{5.665061in}{0.696682in}}%
\pgfpathlineto{\pgfqpoint{5.665061in}{0.694032in}}%
\pgfpathlineto{\pgfqpoint{5.666405in}{0.725834in}}%
\pgfpathlineto{\pgfqpoint{5.667078in}{0.725834in}}%
\pgfpathlineto{\pgfqpoint{5.667078in}{0.699333in}}%
\pgfpathlineto{\pgfqpoint{5.668422in}{0.715234in}}%
\pgfpathlineto{\pgfqpoint{5.669094in}{0.715234in}}%
\pgfpathlineto{\pgfqpoint{5.669766in}{0.728484in}}%
\pgfpathlineto{\pgfqpoint{5.670438in}{0.701983in}}%
\pgfpathlineto{\pgfqpoint{5.671111in}{0.701983in}}%
\pgfpathlineto{\pgfqpoint{5.671111in}{0.688732in}}%
\pgfpathlineto{\pgfqpoint{5.671783in}{0.704633in}}%
\pgfpathlineto{\pgfqpoint{5.672455in}{0.699333in}}%
\pgfpathlineto{\pgfqpoint{5.673127in}{0.699333in}}%
\pgfpathlineto{\pgfqpoint{5.674471in}{0.733785in}}%
\pgfpathlineto{\pgfqpoint{5.675816in}{0.733785in}}%
\pgfpathlineto{\pgfqpoint{5.675816in}{0.691382in}}%
\pgfpathlineto{\pgfqpoint{5.677160in}{0.715234in}}%
\pgfpathlineto{\pgfqpoint{5.677832in}{0.715234in}}%
\pgfpathlineto{\pgfqpoint{5.678504in}{0.725834in}}%
\pgfpathlineto{\pgfqpoint{5.679177in}{0.701983in}}%
\pgfpathlineto{\pgfqpoint{5.679849in}{0.701983in}}%
\pgfpathlineto{\pgfqpoint{5.679849in}{0.728484in}}%
\pgfpathlineto{\pgfqpoint{5.681193in}{0.712583in}}%
\pgfpathlineto{\pgfqpoint{5.681865in}{0.712583in}}%
\pgfpathlineto{\pgfqpoint{5.683210in}{0.688732in}}%
\pgfpathlineto{\pgfqpoint{5.683882in}{0.688732in}}%
\pgfpathlineto{\pgfqpoint{5.683882in}{0.723184in}}%
\pgfpathlineto{\pgfqpoint{5.685226in}{0.720534in}}%
\pgfpathlineto{\pgfqpoint{5.685898in}{0.720534in}}%
\pgfpathlineto{\pgfqpoint{5.685898in}{0.736435in}}%
\pgfpathlineto{\pgfqpoint{5.687243in}{0.688732in}}%
\pgfpathlineto{\pgfqpoint{5.687915in}{0.688732in}}%
\pgfpathlineto{\pgfqpoint{5.689259in}{0.739085in}}%
\pgfpathlineto{\pgfqpoint{5.689931in}{0.739085in}}%
\pgfpathlineto{\pgfqpoint{5.690603in}{0.747036in}}%
\pgfpathlineto{\pgfqpoint{5.691276in}{0.704633in}}%
\pgfpathlineto{\pgfqpoint{5.691948in}{0.704633in}}%
\pgfpathlineto{\pgfqpoint{5.693292in}{0.725834in}}%
\pgfpathlineto{\pgfqpoint{5.693964in}{0.725834in}}%
\pgfpathlineto{\pgfqpoint{5.694636in}{0.694032in}}%
\pgfpathlineto{\pgfqpoint{5.695309in}{0.736435in}}%
\pgfpathlineto{\pgfqpoint{5.695981in}{0.736435in}}%
\pgfpathlineto{\pgfqpoint{5.695981in}{0.699333in}}%
\pgfpathlineto{\pgfqpoint{5.697325in}{0.723184in}}%
\pgfpathlineto{\pgfqpoint{5.697997in}{0.723184in}}%
\pgfpathlineto{\pgfqpoint{5.698669in}{0.696682in}}%
\pgfpathlineto{\pgfqpoint{5.699342in}{0.744386in}}%
\pgfpathlineto{\pgfqpoint{5.700014in}{0.744386in}}%
\pgfpathlineto{\pgfqpoint{5.701358in}{0.707283in}}%
\pgfpathlineto{\pgfqpoint{5.702030in}{0.707283in}}%
\pgfpathlineto{\pgfqpoint{5.702702in}{0.696682in}}%
\pgfpathlineto{\pgfqpoint{5.703375in}{0.744386in}}%
\pgfpathlineto{\pgfqpoint{5.704047in}{0.744386in}}%
\pgfpathlineto{\pgfqpoint{5.705391in}{0.707283in}}%
\pgfpathlineto{\pgfqpoint{5.706063in}{0.707283in}}%
\pgfpathlineto{\pgfqpoint{5.706063in}{0.709933in}}%
\pgfpathlineto{\pgfqpoint{5.707408in}{0.691382in}}%
\pgfpathlineto{\pgfqpoint{5.708080in}{0.691382in}}%
\pgfpathlineto{\pgfqpoint{5.709424in}{0.717884in}}%
\pgfpathlineto{\pgfqpoint{5.710096in}{0.717884in}}%
\pgfpathlineto{\pgfqpoint{5.710768in}{0.699333in}}%
\pgfpathlineto{\pgfqpoint{5.711441in}{0.717884in}}%
\pgfpathlineto{\pgfqpoint{5.712113in}{0.717884in}}%
\pgfpathlineto{\pgfqpoint{5.712113in}{0.747036in}}%
\pgfpathlineto{\pgfqpoint{5.713457in}{0.680781in}}%
\pgfpathlineto{\pgfqpoint{5.714129in}{0.680781in}}%
\pgfpathlineto{\pgfqpoint{5.715474in}{0.720534in}}%
\pgfpathlineto{\pgfqpoint{5.716146in}{0.720534in}}%
\pgfpathlineto{\pgfqpoint{5.716818in}{0.768237in}}%
\pgfpathlineto{\pgfqpoint{5.717490in}{0.715234in}}%
\pgfpathlineto{\pgfqpoint{5.718162in}{0.715234in}}%
\pgfpathlineto{\pgfqpoint{5.718162in}{0.717884in}}%
\pgfpathlineto{\pgfqpoint{5.718834in}{0.701983in}}%
\pgfpathlineto{\pgfqpoint{5.719507in}{0.709933in}}%
\pgfpathlineto{\pgfqpoint{5.720179in}{0.709933in}}%
\pgfpathlineto{\pgfqpoint{5.720179in}{0.701983in}}%
\pgfpathlineto{\pgfqpoint{5.720851in}{0.739085in}}%
\pgfpathlineto{\pgfqpoint{5.721523in}{0.712583in}}%
\pgfpathlineto{\pgfqpoint{5.722195in}{0.712583in}}%
\pgfpathlineto{\pgfqpoint{5.722867in}{0.701983in}}%
\pgfpathlineto{\pgfqpoint{5.723540in}{0.739085in}}%
\pgfpathlineto{\pgfqpoint{5.724212in}{0.739085in}}%
\pgfpathlineto{\pgfqpoint{5.724884in}{0.694032in}}%
\pgfpathlineto{\pgfqpoint{5.725556in}{0.720534in}}%
\pgfpathlineto{\pgfqpoint{5.726228in}{0.720534in}}%
\pgfpathlineto{\pgfqpoint{5.726228in}{0.728484in}}%
\pgfpathlineto{\pgfqpoint{5.726900in}{0.717884in}}%
\pgfpathlineto{\pgfqpoint{5.727573in}{0.720534in}}%
\pgfpathlineto{\pgfqpoint{5.728245in}{0.720534in}}%
\pgfpathlineto{\pgfqpoint{5.728245in}{0.691382in}}%
\pgfpathlineto{\pgfqpoint{5.729589in}{0.736435in}}%
\pgfpathlineto{\pgfqpoint{5.730261in}{0.736435in}}%
\pgfpathlineto{\pgfqpoint{5.730261in}{0.699333in}}%
\pgfpathlineto{\pgfqpoint{5.730933in}{0.749686in}}%
\pgfpathlineto{\pgfqpoint{5.731606in}{0.723184in}}%
\pgfpathlineto{\pgfqpoint{5.732278in}{0.723184in}}%
\pgfpathlineto{\pgfqpoint{5.733622in}{0.733785in}}%
\pgfpathlineto{\pgfqpoint{5.734294in}{0.733785in}}%
\pgfpathlineto{\pgfqpoint{5.734967in}{0.744386in}}%
\pgfpathlineto{\pgfqpoint{5.735639in}{0.739085in}}%
\pgfpathlineto{\pgfqpoint{5.736311in}{0.739085in}}%
\pgfpathlineto{\pgfqpoint{5.736311in}{0.694032in}}%
\pgfpathlineto{\pgfqpoint{5.737655in}{0.694032in}}%
\pgfpathlineto{\pgfqpoint{5.738327in}{0.694032in}}%
\pgfpathlineto{\pgfqpoint{5.739000in}{0.712583in}}%
\pgfpathlineto{\pgfqpoint{5.739672in}{0.699333in}}%
\pgfpathlineto{\pgfqpoint{5.740344in}{0.699333in}}%
\pgfpathlineto{\pgfqpoint{5.741016in}{0.723184in}}%
\pgfpathlineto{\pgfqpoint{5.741688in}{0.675481in}}%
\pgfpathlineto{\pgfqpoint{5.742360in}{0.675481in}}%
\pgfpathlineto{\pgfqpoint{5.742360in}{0.728484in}}%
\pgfpathlineto{\pgfqpoint{5.743705in}{0.709933in}}%
\pgfpathlineto{\pgfqpoint{5.744377in}{0.709933in}}%
\pgfpathlineto{\pgfqpoint{5.745049in}{0.744386in}}%
\pgfpathlineto{\pgfqpoint{5.745721in}{0.725834in}}%
\pgfpathlineto{\pgfqpoint{5.746393in}{0.725834in}}%
\pgfpathlineto{\pgfqpoint{5.746393in}{0.715234in}}%
\pgfpathlineto{\pgfqpoint{5.747738in}{0.731135in}}%
\pgfpathlineto{\pgfqpoint{5.748410in}{0.731135in}}%
\pgfpathlineto{\pgfqpoint{5.748410in}{0.701983in}}%
\pgfpathlineto{\pgfqpoint{5.749082in}{0.747036in}}%
\pgfpathlineto{\pgfqpoint{5.749754in}{0.709933in}}%
\pgfpathlineto{\pgfqpoint{5.750426in}{0.709933in}}%
\pgfpathlineto{\pgfqpoint{5.750426in}{0.723184in}}%
\pgfpathlineto{\pgfqpoint{5.751771in}{0.709933in}}%
\pgfpathlineto{\pgfqpoint{5.752443in}{0.709933in}}%
\pgfpathlineto{\pgfqpoint{5.753115in}{0.741735in}}%
\pgfpathlineto{\pgfqpoint{5.753787in}{0.696682in}}%
\pgfpathlineto{\pgfqpoint{5.754459in}{0.696682in}}%
\pgfpathlineto{\pgfqpoint{5.755804in}{0.749686in}}%
\pgfpathlineto{\pgfqpoint{5.756476in}{0.749686in}}%
\pgfpathlineto{\pgfqpoint{5.757820in}{0.701983in}}%
\pgfpathlineto{\pgfqpoint{5.758492in}{0.701983in}}%
\pgfpathlineto{\pgfqpoint{5.759165in}{0.736435in}}%
\pgfpathlineto{\pgfqpoint{5.759837in}{0.733785in}}%
\pgfpathlineto{\pgfqpoint{5.760509in}{0.733785in}}%
\pgfpathlineto{\pgfqpoint{5.760509in}{0.701983in}}%
\pgfpathlineto{\pgfqpoint{5.761853in}{0.733785in}}%
\pgfpathlineto{\pgfqpoint{5.762525in}{0.733785in}}%
\pgfpathlineto{\pgfqpoint{5.762525in}{0.741735in}}%
\pgfpathlineto{\pgfqpoint{5.763198in}{0.720534in}}%
\pgfpathlineto{\pgfqpoint{5.763870in}{0.720534in}}%
\pgfpathlineto{\pgfqpoint{5.764542in}{0.720534in}}%
\pgfpathlineto{\pgfqpoint{5.764542in}{0.686082in}}%
\pgfpathlineto{\pgfqpoint{5.765886in}{0.749686in}}%
\pgfpathlineto{\pgfqpoint{5.766558in}{0.749686in}}%
\pgfpathlineto{\pgfqpoint{5.767231in}{0.715234in}}%
\pgfpathlineto{\pgfqpoint{5.767903in}{0.744386in}}%
\pgfpathlineto{\pgfqpoint{5.768575in}{0.744386in}}%
\pgfpathlineto{\pgfqpoint{5.769247in}{0.680781in}}%
\pgfpathlineto{\pgfqpoint{5.769919in}{0.696682in}}%
\pgfpathlineto{\pgfqpoint{5.770591in}{0.696682in}}%
\pgfpathlineto{\pgfqpoint{5.770591in}{0.694032in}}%
\pgfpathlineto{\pgfqpoint{5.771264in}{0.736435in}}%
\pgfpathlineto{\pgfqpoint{5.771936in}{0.709933in}}%
\pgfpathlineto{\pgfqpoint{5.772608in}{0.709933in}}%
\pgfpathlineto{\pgfqpoint{5.772608in}{0.696682in}}%
\pgfpathlineto{\pgfqpoint{5.773280in}{0.715234in}}%
\pgfpathlineto{\pgfqpoint{5.773952in}{0.696682in}}%
\pgfpathlineto{\pgfqpoint{5.774624in}{0.696682in}}%
\pgfpathlineto{\pgfqpoint{5.775969in}{0.747036in}}%
\pgfpathlineto{\pgfqpoint{5.776641in}{0.747036in}}%
\pgfpathlineto{\pgfqpoint{5.777313in}{0.701983in}}%
\pgfpathlineto{\pgfqpoint{5.777985in}{0.741735in}}%
\pgfpathlineto{\pgfqpoint{5.778657in}{0.741735in}}%
\pgfpathlineto{\pgfqpoint{5.779330in}{0.694032in}}%
\pgfpathlineto{\pgfqpoint{5.780002in}{0.717884in}}%
\pgfpathlineto{\pgfqpoint{5.780674in}{0.717884in}}%
\pgfpathlineto{\pgfqpoint{5.780674in}{0.696682in}}%
\pgfpathlineto{\pgfqpoint{5.782018in}{0.749686in}}%
\pgfpathlineto{\pgfqpoint{5.782690in}{0.749686in}}%
\pgfpathlineto{\pgfqpoint{5.784035in}{0.696682in}}%
\pgfpathlineto{\pgfqpoint{5.784707in}{0.696682in}}%
\pgfpathlineto{\pgfqpoint{5.784707in}{0.749686in}}%
\pgfpathlineto{\pgfqpoint{5.786051in}{0.739085in}}%
\pgfpathlineto{\pgfqpoint{5.786723in}{0.739085in}}%
\pgfpathlineto{\pgfqpoint{5.786723in}{0.707283in}}%
\pgfpathlineto{\pgfqpoint{5.788068in}{0.707283in}}%
\pgfpathlineto{\pgfqpoint{5.788740in}{0.707283in}}%
\pgfpathlineto{\pgfqpoint{5.789412in}{0.701983in}}%
\pgfpathlineto{\pgfqpoint{5.790084in}{0.728484in}}%
\pgfpathlineto{\pgfqpoint{5.790756in}{0.728484in}}%
\pgfpathlineto{\pgfqpoint{5.791429in}{0.739085in}}%
\pgfpathlineto{\pgfqpoint{5.792101in}{0.709933in}}%
\pgfpathlineto{\pgfqpoint{5.792773in}{0.709933in}}%
\pgfpathlineto{\pgfqpoint{5.792773in}{0.694032in}}%
\pgfpathlineto{\pgfqpoint{5.793445in}{0.715234in}}%
\pgfpathlineto{\pgfqpoint{5.794117in}{0.712583in}}%
\pgfpathlineto{\pgfqpoint{5.794789in}{0.712583in}}%
\pgfpathlineto{\pgfqpoint{5.794789in}{0.717884in}}%
\pgfpathlineto{\pgfqpoint{5.795462in}{0.707283in}}%
\pgfpathlineto{\pgfqpoint{5.796134in}{0.707283in}}%
\pgfpathlineto{\pgfqpoint{5.796806in}{0.707283in}}%
\pgfpathlineto{\pgfqpoint{5.797478in}{0.686082in}}%
\pgfpathlineto{\pgfqpoint{5.798150in}{0.723184in}}%
\pgfpathlineto{\pgfqpoint{5.799495in}{0.723184in}}%
\pgfpathlineto{\pgfqpoint{5.799495in}{0.757636in}}%
\pgfpathlineto{\pgfqpoint{5.800839in}{0.731135in}}%
\pgfpathlineto{\pgfqpoint{5.801511in}{0.731135in}}%
\pgfpathlineto{\pgfqpoint{5.801511in}{0.731135in}}%
\pgfusepath{stroke}%
\end{pgfscope}%
\begin{pgfscope}%
\pgfpathrectangle{\pgfqpoint{3.662674in}{0.552778in}}{\pgfqpoint{2.138715in}{1.650000in}}%
\pgfusepath{clip}%
\pgfsetrectcap%
\pgfsetroundjoin%
\pgfsetlinewidth{1.505625pt}%
\definecolor{currentstroke}{rgb}{0.172549,0.627451,0.172549}%
\pgfsetstrokecolor{currentstroke}%
\pgfsetstrokeopacity{0.800000}%
\pgfsetdash{}{0pt}%
\pgfpathmoveto{\pgfqpoint{3.663346in}{0.760287in}}%
\pgfpathlineto{\pgfqpoint{3.664018in}{0.760287in}}%
\pgfpathlineto{\pgfqpoint{3.664018in}{0.789439in}}%
\pgfpathlineto{\pgfqpoint{3.665362in}{0.789439in}}%
\pgfpathlineto{\pgfqpoint{3.666034in}{0.789439in}}%
\pgfpathlineto{\pgfqpoint{3.666034in}{0.776188in}}%
\pgfpathlineto{\pgfqpoint{3.666707in}{0.845092in}}%
\pgfpathlineto{\pgfqpoint{3.667379in}{0.818590in}}%
\pgfpathlineto{\pgfqpoint{3.668051in}{0.818590in}}%
\pgfpathlineto{\pgfqpoint{3.669395in}{0.778838in}}%
\pgfpathlineto{\pgfqpoint{3.670067in}{0.778838in}}%
\pgfpathlineto{\pgfqpoint{3.670740in}{0.797389in}}%
\pgfpathlineto{\pgfqpoint{3.671412in}{0.757636in}}%
\pgfpathlineto{\pgfqpoint{3.672084in}{0.757636in}}%
\pgfpathlineto{\pgfqpoint{3.672084in}{0.749686in}}%
\pgfpathlineto{\pgfqpoint{3.672756in}{0.770887in}}%
\pgfpathlineto{\pgfqpoint{3.673428in}{0.768237in}}%
\pgfpathlineto{\pgfqpoint{3.674100in}{0.768237in}}%
\pgfpathlineto{\pgfqpoint{3.674773in}{0.752336in}}%
\pgfpathlineto{\pgfqpoint{3.675445in}{0.786788in}}%
\pgfpathlineto{\pgfqpoint{3.676117in}{0.786788in}}%
\pgfpathlineto{\pgfqpoint{3.676117in}{0.731135in}}%
\pgfpathlineto{\pgfqpoint{3.677461in}{0.736435in}}%
\pgfpathlineto{\pgfqpoint{3.678133in}{0.736435in}}%
\pgfpathlineto{\pgfqpoint{3.678133in}{0.768237in}}%
\pgfpathlineto{\pgfqpoint{3.679478in}{0.723184in}}%
\pgfpathlineto{\pgfqpoint{3.680150in}{0.723184in}}%
\pgfpathlineto{\pgfqpoint{3.680150in}{0.744386in}}%
\pgfpathlineto{\pgfqpoint{3.681494in}{0.691382in}}%
\pgfpathlineto{\pgfqpoint{3.682166in}{0.691382in}}%
\pgfpathlineto{\pgfqpoint{3.682839in}{0.731135in}}%
\pgfpathlineto{\pgfqpoint{3.683511in}{0.728484in}}%
\pgfpathlineto{\pgfqpoint{3.684183in}{0.728484in}}%
\pgfpathlineto{\pgfqpoint{3.684183in}{0.686082in}}%
\pgfpathlineto{\pgfqpoint{3.685527in}{0.709933in}}%
\pgfpathlineto{\pgfqpoint{3.686199in}{0.709933in}}%
\pgfpathlineto{\pgfqpoint{3.686872in}{0.667530in}}%
\pgfpathlineto{\pgfqpoint{3.687544in}{0.707283in}}%
\pgfpathlineto{\pgfqpoint{3.688216in}{0.707283in}}%
\pgfpathlineto{\pgfqpoint{3.688888in}{0.683431in}}%
\pgfpathlineto{\pgfqpoint{3.689560in}{0.699333in}}%
\pgfpathlineto{\pgfqpoint{3.690232in}{0.699333in}}%
\pgfpathlineto{\pgfqpoint{3.690232in}{0.670181in}}%
\pgfpathlineto{\pgfqpoint{3.691577in}{0.680781in}}%
\pgfpathlineto{\pgfqpoint{3.692249in}{0.680781in}}%
\pgfpathlineto{\pgfqpoint{3.692249in}{0.662230in}}%
\pgfpathlineto{\pgfqpoint{3.693593in}{0.664880in}}%
\pgfpathlineto{\pgfqpoint{3.694265in}{0.664880in}}%
\pgfpathlineto{\pgfqpoint{3.695610in}{0.675481in}}%
\pgfpathlineto{\pgfqpoint{3.696282in}{0.675481in}}%
\pgfpathlineto{\pgfqpoint{3.696282in}{0.683431in}}%
\pgfpathlineto{\pgfqpoint{3.696954in}{0.659580in}}%
\pgfpathlineto{\pgfqpoint{3.697626in}{0.683431in}}%
\pgfpathlineto{\pgfqpoint{3.698298in}{0.683431in}}%
\pgfpathlineto{\pgfqpoint{3.698971in}{0.656930in}}%
\pgfpathlineto{\pgfqpoint{3.699643in}{0.667530in}}%
\pgfpathlineto{\pgfqpoint{3.700987in}{0.667530in}}%
\pgfpathlineto{\pgfqpoint{3.701659in}{0.651629in}}%
\pgfpathlineto{\pgfqpoint{3.702332in}{0.656930in}}%
\pgfpathlineto{\pgfqpoint{3.703004in}{0.656930in}}%
\pgfpathlineto{\pgfqpoint{3.703676in}{0.667530in}}%
\pgfpathlineto{\pgfqpoint{3.704348in}{0.648979in}}%
\pgfpathlineto{\pgfqpoint{3.705020in}{0.648979in}}%
\pgfpathlineto{\pgfqpoint{3.705692in}{0.643679in}}%
\pgfpathlineto{\pgfqpoint{3.706365in}{0.672831in}}%
\pgfpathlineto{\pgfqpoint{3.707037in}{0.672831in}}%
\pgfpathlineto{\pgfqpoint{3.707037in}{0.688732in}}%
\pgfpathlineto{\pgfqpoint{3.708381in}{0.651629in}}%
\pgfpathlineto{\pgfqpoint{3.709053in}{0.651629in}}%
\pgfpathlineto{\pgfqpoint{3.709053in}{0.670181in}}%
\pgfpathlineto{\pgfqpoint{3.710398in}{0.664880in}}%
\pgfpathlineto{\pgfqpoint{3.712414in}{0.664880in}}%
\pgfpathlineto{\pgfqpoint{3.713086in}{0.641029in}}%
\pgfpathlineto{\pgfqpoint{3.713758in}{0.667530in}}%
\pgfpathlineto{\pgfqpoint{3.714431in}{0.667530in}}%
\pgfpathlineto{\pgfqpoint{3.714431in}{0.670181in}}%
\pgfpathlineto{\pgfqpoint{3.715103in}{0.643679in}}%
\pgfpathlineto{\pgfqpoint{3.715775in}{0.659580in}}%
\pgfpathlineto{\pgfqpoint{3.716447in}{0.659580in}}%
\pgfpathlineto{\pgfqpoint{3.717119in}{0.651629in}}%
\pgfpathlineto{\pgfqpoint{3.717791in}{0.656930in}}%
\pgfpathlineto{\pgfqpoint{3.718464in}{0.656930in}}%
\pgfpathlineto{\pgfqpoint{3.718464in}{0.670181in}}%
\pgfpathlineto{\pgfqpoint{3.719808in}{0.659580in}}%
\pgfpathlineto{\pgfqpoint{3.721152in}{0.659580in}}%
\pgfpathlineto{\pgfqpoint{3.721152in}{0.664880in}}%
\pgfpathlineto{\pgfqpoint{3.722497in}{0.662230in}}%
\pgfpathlineto{\pgfqpoint{3.723169in}{0.662230in}}%
\pgfpathlineto{\pgfqpoint{3.723169in}{0.672831in}}%
\pgfpathlineto{\pgfqpoint{3.723841in}{0.654280in}}%
\pgfpathlineto{\pgfqpoint{3.724513in}{0.656930in}}%
\pgfpathlineto{\pgfqpoint{3.725857in}{0.656930in}}%
\pgfpathlineto{\pgfqpoint{3.726530in}{0.648979in}}%
\pgfpathlineto{\pgfqpoint{3.727202in}{0.662230in}}%
\pgfpathlineto{\pgfqpoint{3.727874in}{0.662230in}}%
\pgfpathlineto{\pgfqpoint{3.729218in}{0.680781in}}%
\pgfpathlineto{\pgfqpoint{3.729890in}{0.680781in}}%
\pgfpathlineto{\pgfqpoint{3.730563in}{0.641029in}}%
\pgfpathlineto{\pgfqpoint{3.731235in}{0.654280in}}%
\pgfpathlineto{\pgfqpoint{3.731907in}{0.654280in}}%
\pgfpathlineto{\pgfqpoint{3.732579in}{0.670181in}}%
\pgfpathlineto{\pgfqpoint{3.733251in}{0.662230in}}%
\pgfpathlineto{\pgfqpoint{3.733923in}{0.662230in}}%
\pgfpathlineto{\pgfqpoint{3.734596in}{0.675481in}}%
\pgfpathlineto{\pgfqpoint{3.735268in}{0.651629in}}%
\pgfpathlineto{\pgfqpoint{3.735940in}{0.651629in}}%
\pgfpathlineto{\pgfqpoint{3.736612in}{0.675481in}}%
\pgfpathlineto{\pgfqpoint{3.737284in}{0.667530in}}%
\pgfpathlineto{\pgfqpoint{3.737956in}{0.667530in}}%
\pgfpathlineto{\pgfqpoint{3.737956in}{0.662230in}}%
\pgfpathlineto{\pgfqpoint{3.738629in}{0.675481in}}%
\pgfpathlineto{\pgfqpoint{3.739301in}{0.672831in}}%
\pgfpathlineto{\pgfqpoint{3.739973in}{0.672831in}}%
\pgfpathlineto{\pgfqpoint{3.739973in}{0.667530in}}%
\pgfpathlineto{\pgfqpoint{3.741317in}{0.675481in}}%
\pgfpathlineto{\pgfqpoint{3.741989in}{0.675481in}}%
\pgfpathlineto{\pgfqpoint{3.743334in}{0.656930in}}%
\pgfpathlineto{\pgfqpoint{3.744006in}{0.656930in}}%
\pgfpathlineto{\pgfqpoint{3.744006in}{0.670181in}}%
\pgfpathlineto{\pgfqpoint{3.745350in}{0.667530in}}%
\pgfpathlineto{\pgfqpoint{3.746022in}{0.667530in}}%
\pgfpathlineto{\pgfqpoint{3.746022in}{0.680781in}}%
\pgfpathlineto{\pgfqpoint{3.746695in}{0.656930in}}%
\pgfpathlineto{\pgfqpoint{3.747367in}{0.672831in}}%
\pgfpathlineto{\pgfqpoint{3.748039in}{0.672831in}}%
\pgfpathlineto{\pgfqpoint{3.748711in}{0.688732in}}%
\pgfpathlineto{\pgfqpoint{3.749383in}{0.672831in}}%
\pgfpathlineto{\pgfqpoint{3.750055in}{0.672831in}}%
\pgfpathlineto{\pgfqpoint{3.750055in}{0.686082in}}%
\pgfpathlineto{\pgfqpoint{3.751400in}{0.670181in}}%
\pgfpathlineto{\pgfqpoint{3.752072in}{0.670181in}}%
\pgfpathlineto{\pgfqpoint{3.752744in}{0.659580in}}%
\pgfpathlineto{\pgfqpoint{3.753416in}{0.686082in}}%
\pgfpathlineto{\pgfqpoint{3.754088in}{0.686082in}}%
\pgfpathlineto{\pgfqpoint{3.754761in}{0.667530in}}%
\pgfpathlineto{\pgfqpoint{3.755433in}{0.683431in}}%
\pgfpathlineto{\pgfqpoint{3.756777in}{0.683431in}}%
\pgfpathlineto{\pgfqpoint{3.756777in}{0.686082in}}%
\pgfpathlineto{\pgfqpoint{3.758121in}{0.675481in}}%
\pgfpathlineto{\pgfqpoint{3.758794in}{0.675481in}}%
\pgfpathlineto{\pgfqpoint{3.758794in}{0.672831in}}%
\pgfpathlineto{\pgfqpoint{3.760138in}{0.691382in}}%
\pgfpathlineto{\pgfqpoint{3.761482in}{0.691382in}}%
\pgfpathlineto{\pgfqpoint{3.761482in}{0.686082in}}%
\pgfpathlineto{\pgfqpoint{3.762827in}{0.715234in}}%
\pgfpathlineto{\pgfqpoint{3.763499in}{0.715234in}}%
\pgfpathlineto{\pgfqpoint{3.764171in}{0.659580in}}%
\pgfpathlineto{\pgfqpoint{3.764843in}{0.672831in}}%
\pgfpathlineto{\pgfqpoint{3.765515in}{0.672831in}}%
\pgfpathlineto{\pgfqpoint{3.766187in}{0.691382in}}%
\pgfpathlineto{\pgfqpoint{3.766860in}{0.688732in}}%
\pgfpathlineto{\pgfqpoint{3.767532in}{0.688732in}}%
\pgfpathlineto{\pgfqpoint{3.767532in}{0.672831in}}%
\pgfpathlineto{\pgfqpoint{3.768876in}{0.672831in}}%
\pgfpathlineto{\pgfqpoint{3.769548in}{0.672831in}}%
\pgfpathlineto{\pgfqpoint{3.770220in}{0.704633in}}%
\pgfpathlineto{\pgfqpoint{3.770893in}{0.659580in}}%
\pgfpathlineto{\pgfqpoint{3.771565in}{0.659580in}}%
\pgfpathlineto{\pgfqpoint{3.771565in}{0.696682in}}%
\pgfpathlineto{\pgfqpoint{3.772909in}{0.686082in}}%
\pgfpathlineto{\pgfqpoint{3.773581in}{0.686082in}}%
\pgfpathlineto{\pgfqpoint{3.774253in}{0.709933in}}%
\pgfpathlineto{\pgfqpoint{3.774253in}{0.670181in}}%
\pgfpathlineto{\pgfqpoint{3.774926in}{0.686082in}}%
\pgfpathlineto{\pgfqpoint{3.775598in}{0.686082in}}%
\pgfpathlineto{\pgfqpoint{3.776270in}{0.712583in}}%
\pgfpathlineto{\pgfqpoint{3.776942in}{0.678131in}}%
\pgfpathlineto{\pgfqpoint{3.777614in}{0.678131in}}%
\pgfpathlineto{\pgfqpoint{3.778959in}{0.704633in}}%
\pgfpathlineto{\pgfqpoint{3.779631in}{0.704633in}}%
\pgfpathlineto{\pgfqpoint{3.779631in}{0.712583in}}%
\pgfpathlineto{\pgfqpoint{3.780975in}{0.694032in}}%
\pgfpathlineto{\pgfqpoint{3.781647in}{0.694032in}}%
\pgfpathlineto{\pgfqpoint{3.782319in}{0.686082in}}%
\pgfpathlineto{\pgfqpoint{3.782992in}{0.696682in}}%
\pgfpathlineto{\pgfqpoint{3.783664in}{0.696682in}}%
\pgfpathlineto{\pgfqpoint{3.783664in}{0.715234in}}%
\pgfpathlineto{\pgfqpoint{3.784336in}{0.672831in}}%
\pgfpathlineto{\pgfqpoint{3.785008in}{0.688732in}}%
\pgfpathlineto{\pgfqpoint{3.785680in}{0.688732in}}%
\pgfpathlineto{\pgfqpoint{3.786352in}{0.709933in}}%
\pgfpathlineto{\pgfqpoint{3.787025in}{0.672831in}}%
\pgfpathlineto{\pgfqpoint{3.787697in}{0.672831in}}%
\pgfpathlineto{\pgfqpoint{3.787697in}{0.712583in}}%
\pgfpathlineto{\pgfqpoint{3.789041in}{0.694032in}}%
\pgfpathlineto{\pgfqpoint{3.789713in}{0.694032in}}%
\pgfpathlineto{\pgfqpoint{3.789713in}{0.720534in}}%
\pgfpathlineto{\pgfqpoint{3.790385in}{0.683431in}}%
\pgfpathlineto{\pgfqpoint{3.791058in}{0.707283in}}%
\pgfpathlineto{\pgfqpoint{3.791730in}{0.707283in}}%
\pgfpathlineto{\pgfqpoint{3.792402in}{0.709933in}}%
\pgfpathlineto{\pgfqpoint{3.793074in}{0.683431in}}%
\pgfpathlineto{\pgfqpoint{3.793746in}{0.683431in}}%
\pgfpathlineto{\pgfqpoint{3.795091in}{0.694032in}}%
\pgfpathlineto{\pgfqpoint{3.795763in}{0.694032in}}%
\pgfpathlineto{\pgfqpoint{3.797107in}{0.715234in}}%
\pgfpathlineto{\pgfqpoint{3.797779in}{0.715234in}}%
\pgfpathlineto{\pgfqpoint{3.798451in}{0.694032in}}%
\pgfpathlineto{\pgfqpoint{3.799124in}{0.723184in}}%
\pgfpathlineto{\pgfqpoint{3.799796in}{0.723184in}}%
\pgfpathlineto{\pgfqpoint{3.801140in}{0.675481in}}%
\pgfpathlineto{\pgfqpoint{3.801812in}{0.675481in}}%
\pgfpathlineto{\pgfqpoint{3.801812in}{0.720534in}}%
\pgfpathlineto{\pgfqpoint{3.803157in}{0.699333in}}%
\pgfpathlineto{\pgfqpoint{3.803829in}{0.699333in}}%
\pgfpathlineto{\pgfqpoint{3.804501in}{0.723184in}}%
\pgfpathlineto{\pgfqpoint{3.805173in}{0.704633in}}%
\pgfpathlineto{\pgfqpoint{3.805845in}{0.704633in}}%
\pgfpathlineto{\pgfqpoint{3.807190in}{0.733785in}}%
\pgfpathlineto{\pgfqpoint{3.807862in}{0.733785in}}%
\pgfpathlineto{\pgfqpoint{3.807862in}{0.704633in}}%
\pgfpathlineto{\pgfqpoint{3.809206in}{0.752336in}}%
\pgfpathlineto{\pgfqpoint{3.809878in}{0.752336in}}%
\pgfpathlineto{\pgfqpoint{3.810550in}{0.712583in}}%
\pgfpathlineto{\pgfqpoint{3.811223in}{0.725834in}}%
\pgfpathlineto{\pgfqpoint{3.812567in}{0.725834in}}%
\pgfpathlineto{\pgfqpoint{3.812567in}{0.747036in}}%
\pgfpathlineto{\pgfqpoint{3.813239in}{0.715234in}}%
\pgfpathlineto{\pgfqpoint{3.813911in}{0.720534in}}%
\pgfpathlineto{\pgfqpoint{3.814584in}{0.720534in}}%
\pgfpathlineto{\pgfqpoint{3.815256in}{0.717884in}}%
\pgfpathlineto{\pgfqpoint{3.815928in}{0.762937in}}%
\pgfpathlineto{\pgfqpoint{3.816600in}{0.762937in}}%
\pgfpathlineto{\pgfqpoint{3.817272in}{0.725834in}}%
\pgfpathlineto{\pgfqpoint{3.817944in}{0.731135in}}%
\pgfpathlineto{\pgfqpoint{3.818617in}{0.731135in}}%
\pgfpathlineto{\pgfqpoint{3.818617in}{0.733785in}}%
\pgfpathlineto{\pgfqpoint{3.819961in}{0.731135in}}%
\pgfpathlineto{\pgfqpoint{3.821305in}{0.731135in}}%
\pgfpathlineto{\pgfqpoint{3.821305in}{0.768237in}}%
\pgfpathlineto{\pgfqpoint{3.822650in}{0.709933in}}%
\pgfpathlineto{\pgfqpoint{3.823322in}{0.709933in}}%
\pgfpathlineto{\pgfqpoint{3.824666in}{0.762937in}}%
\pgfpathlineto{\pgfqpoint{3.825338in}{0.762937in}}%
\pgfpathlineto{\pgfqpoint{3.825338in}{0.712583in}}%
\pgfpathlineto{\pgfqpoint{3.826683in}{0.762937in}}%
\pgfpathlineto{\pgfqpoint{3.827355in}{0.762937in}}%
\pgfpathlineto{\pgfqpoint{3.827355in}{0.773537in}}%
\pgfpathlineto{\pgfqpoint{3.828699in}{0.744386in}}%
\pgfpathlineto{\pgfqpoint{3.830043in}{0.744386in}}%
\pgfpathlineto{\pgfqpoint{3.831388in}{0.770887in}}%
\pgfpathlineto{\pgfqpoint{3.832060in}{0.770887in}}%
\pgfpathlineto{\pgfqpoint{3.832060in}{0.731135in}}%
\pgfpathlineto{\pgfqpoint{3.833404in}{0.749686in}}%
\pgfpathlineto{\pgfqpoint{3.834076in}{0.749686in}}%
\pgfpathlineto{\pgfqpoint{3.834076in}{0.762937in}}%
\pgfpathlineto{\pgfqpoint{3.835421in}{0.744386in}}%
\pgfpathlineto{\pgfqpoint{3.836093in}{0.744386in}}%
\pgfpathlineto{\pgfqpoint{3.836093in}{0.728484in}}%
\pgfpathlineto{\pgfqpoint{3.836765in}{0.800039in}}%
\pgfpathlineto{\pgfqpoint{3.837437in}{0.749686in}}%
\pgfpathlineto{\pgfqpoint{3.838109in}{0.749686in}}%
\pgfpathlineto{\pgfqpoint{3.838109in}{0.744386in}}%
\pgfpathlineto{\pgfqpoint{3.838782in}{0.773537in}}%
\pgfpathlineto{\pgfqpoint{3.839454in}{0.762937in}}%
\pgfpathlineto{\pgfqpoint{3.840126in}{0.762937in}}%
\pgfpathlineto{\pgfqpoint{3.840126in}{0.781488in}}%
\pgfpathlineto{\pgfqpoint{3.841470in}{0.768237in}}%
\pgfpathlineto{\pgfqpoint{3.842142in}{0.768237in}}%
\pgfpathlineto{\pgfqpoint{3.842142in}{0.784138in}}%
\pgfpathlineto{\pgfqpoint{3.843487in}{0.781488in}}%
\pgfpathlineto{\pgfqpoint{3.844159in}{0.781488in}}%
\pgfpathlineto{\pgfqpoint{3.844159in}{0.739085in}}%
\pgfpathlineto{\pgfqpoint{3.845503in}{0.741735in}}%
\pgfpathlineto{\pgfqpoint{3.846175in}{0.741735in}}%
\pgfpathlineto{\pgfqpoint{3.846175in}{0.807990in}}%
\pgfpathlineto{\pgfqpoint{3.846848in}{0.733785in}}%
\pgfpathlineto{\pgfqpoint{3.847520in}{0.768237in}}%
\pgfpathlineto{\pgfqpoint{3.848192in}{0.768237in}}%
\pgfpathlineto{\pgfqpoint{3.848192in}{0.749686in}}%
\pgfpathlineto{\pgfqpoint{3.848864in}{0.792089in}}%
\pgfpathlineto{\pgfqpoint{3.849536in}{0.784138in}}%
\pgfpathlineto{\pgfqpoint{3.850208in}{0.784138in}}%
\pgfpathlineto{\pgfqpoint{3.850881in}{0.760287in}}%
\pgfpathlineto{\pgfqpoint{3.850881in}{0.810640in}}%
\pgfpathlineto{\pgfqpoint{3.851553in}{0.781488in}}%
\pgfpathlineto{\pgfqpoint{3.852225in}{0.781488in}}%
\pgfpathlineto{\pgfqpoint{3.852225in}{0.794739in}}%
\pgfpathlineto{\pgfqpoint{3.852897in}{0.749686in}}%
\pgfpathlineto{\pgfqpoint{3.853569in}{0.784138in}}%
\pgfpathlineto{\pgfqpoint{3.854241in}{0.784138in}}%
\pgfpathlineto{\pgfqpoint{3.854241in}{0.802689in}}%
\pgfpathlineto{\pgfqpoint{3.855586in}{0.797389in}}%
\pgfpathlineto{\pgfqpoint{3.856930in}{0.797389in}}%
\pgfpathlineto{\pgfqpoint{3.857602in}{0.821241in}}%
\pgfpathlineto{\pgfqpoint{3.858274in}{0.792089in}}%
\pgfpathlineto{\pgfqpoint{3.858947in}{0.792089in}}%
\pgfpathlineto{\pgfqpoint{3.858947in}{0.770887in}}%
\pgfpathlineto{\pgfqpoint{3.860291in}{0.778838in}}%
\pgfpathlineto{\pgfqpoint{3.860963in}{0.778838in}}%
\pgfpathlineto{\pgfqpoint{3.860963in}{0.752336in}}%
\pgfpathlineto{\pgfqpoint{3.861635in}{0.815940in}}%
\pgfpathlineto{\pgfqpoint{3.862307in}{0.802689in}}%
\pgfpathlineto{\pgfqpoint{3.862980in}{0.802689in}}%
\pgfpathlineto{\pgfqpoint{3.862980in}{0.813290in}}%
\pgfpathlineto{\pgfqpoint{3.863652in}{0.784138in}}%
\pgfpathlineto{\pgfqpoint{3.864324in}{0.789439in}}%
\pgfpathlineto{\pgfqpoint{3.864996in}{0.789439in}}%
\pgfpathlineto{\pgfqpoint{3.865668in}{0.773537in}}%
\pgfpathlineto{\pgfqpoint{3.866340in}{0.823891in}}%
\pgfpathlineto{\pgfqpoint{3.867013in}{0.823891in}}%
\pgfpathlineto{\pgfqpoint{3.867685in}{0.792089in}}%
\pgfpathlineto{\pgfqpoint{3.868357in}{0.805340in}}%
\pgfpathlineto{\pgfqpoint{3.869029in}{0.805340in}}%
\pgfpathlineto{\pgfqpoint{3.869701in}{0.794739in}}%
\pgfpathlineto{\pgfqpoint{3.870373in}{0.821241in}}%
\pgfpathlineto{\pgfqpoint{3.871046in}{0.821241in}}%
\pgfpathlineto{\pgfqpoint{3.872390in}{0.784138in}}%
\pgfpathlineto{\pgfqpoint{3.873062in}{0.784138in}}%
\pgfpathlineto{\pgfqpoint{3.874406in}{0.837142in}}%
\pgfpathlineto{\pgfqpoint{3.875079in}{0.837142in}}%
\pgfpathlineto{\pgfqpoint{3.875751in}{0.813290in}}%
\pgfpathlineto{\pgfqpoint{3.875751in}{0.855693in}}%
\pgfpathlineto{\pgfqpoint{3.876423in}{0.847742in}}%
\pgfpathlineto{\pgfqpoint{3.877095in}{0.847742in}}%
\pgfpathlineto{\pgfqpoint{3.877095in}{0.818590in}}%
\pgfpathlineto{\pgfqpoint{3.878439in}{0.871594in}}%
\pgfpathlineto{\pgfqpoint{3.879112in}{0.871594in}}%
\pgfpathlineto{\pgfqpoint{3.879112in}{0.802689in}}%
\pgfpathlineto{\pgfqpoint{3.880456in}{0.837142in}}%
\pgfpathlineto{\pgfqpoint{3.881128in}{0.837142in}}%
\pgfpathlineto{\pgfqpoint{3.882472in}{0.906046in}}%
\pgfpathlineto{\pgfqpoint{3.883145in}{0.906046in}}%
\pgfpathlineto{\pgfqpoint{3.884489in}{0.831841in}}%
\pgfpathlineto{\pgfqpoint{3.885161in}{0.831841in}}%
\pgfpathlineto{\pgfqpoint{3.885161in}{0.786788in}}%
\pgfpathlineto{\pgfqpoint{3.886505in}{0.842442in}}%
\pgfpathlineto{\pgfqpoint{3.887178in}{0.842442in}}%
\pgfpathlineto{\pgfqpoint{3.887850in}{0.821241in}}%
\pgfpathlineto{\pgfqpoint{3.888522in}{0.892795in}}%
\pgfpathlineto{\pgfqpoint{3.889194in}{0.892795in}}%
\pgfpathlineto{\pgfqpoint{3.889194in}{0.823891in}}%
\pgfpathlineto{\pgfqpoint{3.890538in}{0.855693in}}%
\pgfpathlineto{\pgfqpoint{3.891211in}{0.855693in}}%
\pgfpathlineto{\pgfqpoint{3.891883in}{0.845092in}}%
\pgfpathlineto{\pgfqpoint{3.892555in}{0.845092in}}%
\pgfpathlineto{\pgfqpoint{3.893227in}{0.845092in}}%
\pgfpathlineto{\pgfqpoint{3.893227in}{0.839792in}}%
\pgfpathlineto{\pgfqpoint{3.894571in}{0.921947in}}%
\pgfpathlineto{\pgfqpoint{3.895244in}{0.921947in}}%
\pgfpathlineto{\pgfqpoint{3.896588in}{0.821241in}}%
\pgfpathlineto{\pgfqpoint{3.897260in}{0.821241in}}%
\pgfpathlineto{\pgfqpoint{3.897260in}{0.853043in}}%
\pgfpathlineto{\pgfqpoint{3.897932in}{0.813290in}}%
\pgfpathlineto{\pgfqpoint{3.898604in}{0.831841in}}%
\pgfpathlineto{\pgfqpoint{3.899277in}{0.831841in}}%
\pgfpathlineto{\pgfqpoint{3.899949in}{0.871594in}}%
\pgfpathlineto{\pgfqpoint{3.900621in}{0.845092in}}%
\pgfpathlineto{\pgfqpoint{3.901293in}{0.845092in}}%
\pgfpathlineto{\pgfqpoint{3.901293in}{0.892795in}}%
\pgfpathlineto{\pgfqpoint{3.902637in}{0.813290in}}%
\pgfpathlineto{\pgfqpoint{3.903310in}{0.813290in}}%
\pgfpathlineto{\pgfqpoint{3.903982in}{0.887495in}}%
\pgfpathlineto{\pgfqpoint{3.904654in}{0.863644in}}%
\pgfpathlineto{\pgfqpoint{3.905326in}{0.863644in}}%
\pgfpathlineto{\pgfqpoint{3.905326in}{0.908697in}}%
\pgfpathlineto{\pgfqpoint{3.906670in}{0.908697in}}%
\pgfpathlineto{\pgfqpoint{3.907343in}{0.908697in}}%
\pgfpathlineto{\pgfqpoint{3.908687in}{0.868944in}}%
\pgfpathlineto{\pgfqpoint{3.909359in}{0.868944in}}%
\pgfpathlineto{\pgfqpoint{3.909359in}{0.887495in}}%
\pgfpathlineto{\pgfqpoint{3.910703in}{0.839792in}}%
\pgfpathlineto{\pgfqpoint{3.911376in}{0.839792in}}%
\pgfpathlineto{\pgfqpoint{3.911376in}{0.895446in}}%
\pgfpathlineto{\pgfqpoint{3.912720in}{0.837142in}}%
\pgfpathlineto{\pgfqpoint{3.913392in}{0.837142in}}%
\pgfpathlineto{\pgfqpoint{3.913392in}{0.887495in}}%
\pgfpathlineto{\pgfqpoint{3.914736in}{0.879545in}}%
\pgfpathlineto{\pgfqpoint{3.915409in}{0.879545in}}%
\pgfpathlineto{\pgfqpoint{3.915409in}{0.866294in}}%
\pgfpathlineto{\pgfqpoint{3.916753in}{0.948449in}}%
\pgfpathlineto{\pgfqpoint{3.917425in}{0.948449in}}%
\pgfpathlineto{\pgfqpoint{3.917425in}{0.866294in}}%
\pgfpathlineto{\pgfqpoint{3.918769in}{0.900746in}}%
\pgfpathlineto{\pgfqpoint{3.919442in}{0.900746in}}%
\pgfpathlineto{\pgfqpoint{3.919442in}{0.913997in}}%
\pgfpathlineto{\pgfqpoint{3.920114in}{0.890145in}}%
\pgfpathlineto{\pgfqpoint{3.920786in}{0.913997in}}%
\pgfpathlineto{\pgfqpoint{3.921458in}{0.913997in}}%
\pgfpathlineto{\pgfqpoint{3.921458in}{0.961700in}}%
\pgfpathlineto{\pgfqpoint{3.922130in}{0.868944in}}%
\pgfpathlineto{\pgfqpoint{3.922802in}{0.937848in}}%
\pgfpathlineto{\pgfqpoint{3.923475in}{0.937848in}}%
\pgfpathlineto{\pgfqpoint{3.923475in}{0.871594in}}%
\pgfpathlineto{\pgfqpoint{3.924819in}{0.879545in}}%
\pgfpathlineto{\pgfqpoint{3.925491in}{0.879545in}}%
\pgfpathlineto{\pgfqpoint{3.926163in}{0.935198in}}%
\pgfpathlineto{\pgfqpoint{3.926836in}{0.916647in}}%
\pgfpathlineto{\pgfqpoint{3.927508in}{0.916647in}}%
\pgfpathlineto{\pgfqpoint{3.927508in}{0.884845in}}%
\pgfpathlineto{\pgfqpoint{3.928852in}{0.924598in}}%
\pgfpathlineto{\pgfqpoint{3.930196in}{0.924598in}}%
\pgfpathlineto{\pgfqpoint{3.930196in}{0.929898in}}%
\pgfpathlineto{\pgfqpoint{3.930869in}{0.908697in}}%
\pgfpathlineto{\pgfqpoint{3.931541in}{0.927248in}}%
\pgfpathlineto{\pgfqpoint{3.932213in}{0.927248in}}%
\pgfpathlineto{\pgfqpoint{3.932885in}{0.876894in}}%
\pgfpathlineto{\pgfqpoint{3.933557in}{0.908697in}}%
\pgfpathlineto{\pgfqpoint{3.934229in}{0.908697in}}%
\pgfpathlineto{\pgfqpoint{3.935574in}{0.974951in}}%
\pgfpathlineto{\pgfqpoint{3.936246in}{0.974951in}}%
\pgfpathlineto{\pgfqpoint{3.937590in}{0.924598in}}%
\pgfpathlineto{\pgfqpoint{3.938262in}{0.924598in}}%
\pgfpathlineto{\pgfqpoint{3.939607in}{0.972301in}}%
\pgfpathlineto{\pgfqpoint{3.940279in}{0.972301in}}%
\pgfpathlineto{\pgfqpoint{3.940951in}{0.996152in}}%
\pgfpathlineto{\pgfqpoint{3.941623in}{0.906046in}}%
\pgfpathlineto{\pgfqpoint{3.942295in}{0.906046in}}%
\pgfpathlineto{\pgfqpoint{3.942968in}{0.959050in}}%
\pgfpathlineto{\pgfqpoint{3.943640in}{0.935198in}}%
\pgfpathlineto{\pgfqpoint{3.944312in}{0.935198in}}%
\pgfpathlineto{\pgfqpoint{3.944984in}{0.951099in}}%
\pgfpathlineto{\pgfqpoint{3.945656in}{0.935198in}}%
\pgfpathlineto{\pgfqpoint{3.946328in}{0.935198in}}%
\pgfpathlineto{\pgfqpoint{3.946328in}{0.908697in}}%
\pgfpathlineto{\pgfqpoint{3.947673in}{0.951099in}}%
\pgfpathlineto{\pgfqpoint{3.948345in}{0.951099in}}%
\pgfpathlineto{\pgfqpoint{3.949689in}{0.908697in}}%
\pgfpathlineto{\pgfqpoint{3.950361in}{0.908697in}}%
\pgfpathlineto{\pgfqpoint{3.950361in}{0.953750in}}%
\pgfpathlineto{\pgfqpoint{3.951034in}{0.906046in}}%
\pgfpathlineto{\pgfqpoint{3.951706in}{0.943149in}}%
\pgfpathlineto{\pgfqpoint{3.952378in}{0.943149in}}%
\pgfpathlineto{\pgfqpoint{3.953050in}{1.022654in}}%
\pgfpathlineto{\pgfqpoint{3.953722in}{0.921947in}}%
\pgfpathlineto{\pgfqpoint{3.954394in}{0.921947in}}%
\pgfpathlineto{\pgfqpoint{3.954394in}{0.903396in}}%
\pgfpathlineto{\pgfqpoint{3.955739in}{0.945799in}}%
\pgfpathlineto{\pgfqpoint{3.956411in}{0.945799in}}%
\pgfpathlineto{\pgfqpoint{3.957083in}{1.009403in}}%
\pgfpathlineto{\pgfqpoint{3.957755in}{0.972301in}}%
\pgfpathlineto{\pgfqpoint{3.958427in}{0.972301in}}%
\pgfpathlineto{\pgfqpoint{3.958427in}{0.937848in}}%
\pgfpathlineto{\pgfqpoint{3.959772in}{0.969651in}}%
\pgfpathlineto{\pgfqpoint{3.960444in}{0.969651in}}%
\pgfpathlineto{\pgfqpoint{3.960444in}{0.940499in}}%
\pgfpathlineto{\pgfqpoint{3.961788in}{0.948449in}}%
\pgfpathlineto{\pgfqpoint{3.962460in}{0.948449in}}%
\pgfpathlineto{\pgfqpoint{3.963133in}{0.935198in}}%
\pgfpathlineto{\pgfqpoint{3.963805in}{0.948449in}}%
\pgfpathlineto{\pgfqpoint{3.964477in}{0.948449in}}%
\pgfpathlineto{\pgfqpoint{3.965149in}{1.004103in}}%
\pgfpathlineto{\pgfqpoint{3.965821in}{0.913997in}}%
\pgfpathlineto{\pgfqpoint{3.966493in}{0.913997in}}%
\pgfpathlineto{\pgfqpoint{3.967838in}{0.985552in}}%
\pgfpathlineto{\pgfqpoint{3.968510in}{0.985552in}}%
\pgfpathlineto{\pgfqpoint{3.968510in}{0.948449in}}%
\pgfpathlineto{\pgfqpoint{3.969182in}{1.035905in}}%
\pgfpathlineto{\pgfqpoint{3.969854in}{0.977601in}}%
\pgfpathlineto{\pgfqpoint{3.970526in}{0.977601in}}%
\pgfpathlineto{\pgfqpoint{3.970526in}{0.948449in}}%
\pgfpathlineto{\pgfqpoint{3.971871in}{0.959050in}}%
\pgfpathlineto{\pgfqpoint{3.972543in}{0.959050in}}%
\pgfpathlineto{\pgfqpoint{3.972543in}{1.049156in}}%
\pgfpathlineto{\pgfqpoint{3.973887in}{0.959050in}}%
\pgfpathlineto{\pgfqpoint{3.974559in}{0.959050in}}%
\pgfpathlineto{\pgfqpoint{3.974559in}{0.913997in}}%
\pgfpathlineto{\pgfqpoint{3.975232in}{1.009403in}}%
\pgfpathlineto{\pgfqpoint{3.975904in}{0.996152in}}%
\pgfpathlineto{\pgfqpoint{3.976576in}{0.996152in}}%
\pgfpathlineto{\pgfqpoint{3.976576in}{0.998803in}}%
\pgfpathlineto{\pgfqpoint{3.977920in}{0.964350in}}%
\pgfpathlineto{\pgfqpoint{3.978592in}{0.964350in}}%
\pgfpathlineto{\pgfqpoint{3.979937in}{1.006753in}}%
\pgfpathlineto{\pgfqpoint{3.980609in}{1.006753in}}%
\pgfpathlineto{\pgfqpoint{3.980609in}{0.990852in}}%
\pgfpathlineto{\pgfqpoint{3.981281in}{1.017354in}}%
\pgfpathlineto{\pgfqpoint{3.981953in}{1.009403in}}%
\pgfpathlineto{\pgfqpoint{3.982625in}{1.009403in}}%
\pgfpathlineto{\pgfqpoint{3.983970in}{1.035905in}}%
\pgfpathlineto{\pgfqpoint{3.984642in}{1.035905in}}%
\pgfpathlineto{\pgfqpoint{3.985314in}{1.075658in}}%
\pgfpathlineto{\pgfqpoint{3.985986in}{1.009403in}}%
\pgfpathlineto{\pgfqpoint{3.986658in}{1.009403in}}%
\pgfpathlineto{\pgfqpoint{3.987331in}{0.953750in}}%
\pgfpathlineto{\pgfqpoint{3.988003in}{1.086258in}}%
\pgfpathlineto{\pgfqpoint{3.988675in}{1.086258in}}%
\pgfpathlineto{\pgfqpoint{3.988675in}{0.998803in}}%
\pgfpathlineto{\pgfqpoint{3.990019in}{1.057106in}}%
\pgfpathlineto{\pgfqpoint{3.990691in}{1.057106in}}%
\pgfpathlineto{\pgfqpoint{3.990691in}{1.004103in}}%
\pgfpathlineto{\pgfqpoint{3.992036in}{1.075658in}}%
\pgfpathlineto{\pgfqpoint{3.992708in}{1.075658in}}%
\pgfpathlineto{\pgfqpoint{3.992708in}{0.993502in}}%
\pgfpathlineto{\pgfqpoint{3.994052in}{0.998803in}}%
\pgfpathlineto{\pgfqpoint{3.994724in}{0.998803in}}%
\pgfpathlineto{\pgfqpoint{3.994724in}{1.065057in}}%
\pgfpathlineto{\pgfqpoint{3.996069in}{1.051806in}}%
\pgfpathlineto{\pgfqpoint{3.996741in}{1.051806in}}%
\pgfpathlineto{\pgfqpoint{3.996741in}{0.996152in}}%
\pgfpathlineto{\pgfqpoint{3.998085in}{1.118060in}}%
\pgfpathlineto{\pgfqpoint{3.998757in}{1.118060in}}%
\pgfpathlineto{\pgfqpoint{3.998757in}{1.014704in}}%
\pgfpathlineto{\pgfqpoint{4.000102in}{1.054456in}}%
\pgfpathlineto{\pgfqpoint{4.000774in}{1.054456in}}%
\pgfpathlineto{\pgfqpoint{4.000774in}{1.004103in}}%
\pgfpathlineto{\pgfqpoint{4.002118in}{1.033255in}}%
\pgfpathlineto{\pgfqpoint{4.002790in}{1.033255in}}%
\pgfpathlineto{\pgfqpoint{4.003463in}{1.059757in}}%
\pgfpathlineto{\pgfqpoint{4.004135in}{1.057106in}}%
\pgfpathlineto{\pgfqpoint{4.004807in}{1.057106in}}%
\pgfpathlineto{\pgfqpoint{4.005479in}{1.091559in}}%
\pgfpathlineto{\pgfqpoint{4.006151in}{1.033255in}}%
\pgfpathlineto{\pgfqpoint{4.006823in}{1.033255in}}%
\pgfpathlineto{\pgfqpoint{4.007496in}{1.080958in}}%
\pgfpathlineto{\pgfqpoint{4.008168in}{1.020004in}}%
\pgfpathlineto{\pgfqpoint{4.008840in}{1.020004in}}%
\pgfpathlineto{\pgfqpoint{4.008840in}{1.014704in}}%
\pgfpathlineto{\pgfqpoint{4.010184in}{1.070357in}}%
\pgfpathlineto{\pgfqpoint{4.010856in}{1.070357in}}%
\pgfpathlineto{\pgfqpoint{4.010856in}{1.112760in}}%
\pgfpathlineto{\pgfqpoint{4.011529in}{1.030605in}}%
\pgfpathlineto{\pgfqpoint{4.012201in}{1.075658in}}%
\pgfpathlineto{\pgfqpoint{4.012873in}{1.075658in}}%
\pgfpathlineto{\pgfqpoint{4.013545in}{1.099509in}}%
\pgfpathlineto{\pgfqpoint{4.014217in}{0.985552in}}%
\pgfpathlineto{\pgfqpoint{4.014889in}{0.985552in}}%
\pgfpathlineto{\pgfqpoint{4.014889in}{0.972301in}}%
\pgfpathlineto{\pgfqpoint{4.015562in}{1.139262in}}%
\pgfpathlineto{\pgfqpoint{4.016234in}{1.083608in}}%
\pgfpathlineto{\pgfqpoint{4.016906in}{1.083608in}}%
\pgfpathlineto{\pgfqpoint{4.017578in}{1.046506in}}%
\pgfpathlineto{\pgfqpoint{4.018250in}{1.112760in}}%
\pgfpathlineto{\pgfqpoint{4.018922in}{1.112760in}}%
\pgfpathlineto{\pgfqpoint{4.018922in}{1.001453in}}%
\pgfpathlineto{\pgfqpoint{4.020267in}{1.128661in}}%
\pgfpathlineto{\pgfqpoint{4.020939in}{1.128661in}}%
\pgfpathlineto{\pgfqpoint{4.020939in}{1.038555in}}%
\pgfpathlineto{\pgfqpoint{4.021611in}{1.141912in}}%
\pgfpathlineto{\pgfqpoint{4.022283in}{1.131311in}}%
\pgfpathlineto{\pgfqpoint{4.022955in}{1.131311in}}%
\pgfpathlineto{\pgfqpoint{4.022955in}{1.025304in}}%
\pgfpathlineto{\pgfqpoint{4.024300in}{1.075658in}}%
\pgfpathlineto{\pgfqpoint{4.024972in}{1.075658in}}%
\pgfpathlineto{\pgfqpoint{4.024972in}{1.149863in}}%
\pgfpathlineto{\pgfqpoint{4.025644in}{1.046506in}}%
\pgfpathlineto{\pgfqpoint{4.026316in}{1.096859in}}%
\pgfpathlineto{\pgfqpoint{4.026988in}{1.096859in}}%
\pgfpathlineto{\pgfqpoint{4.026988in}{1.123361in}}%
\pgfpathlineto{\pgfqpoint{4.027661in}{1.035905in}}%
\pgfpathlineto{\pgfqpoint{4.028333in}{1.083608in}}%
\pgfpathlineto{\pgfqpoint{4.029005in}{1.083608in}}%
\pgfpathlineto{\pgfqpoint{4.029677in}{1.171064in}}%
\pgfpathlineto{\pgfqpoint{4.030349in}{1.078308in}}%
\pgfpathlineto{\pgfqpoint{4.031021in}{1.078308in}}%
\pgfpathlineto{\pgfqpoint{4.031021in}{1.115410in}}%
\pgfpathlineto{\pgfqpoint{4.032366in}{1.075658in}}%
\pgfpathlineto{\pgfqpoint{4.033038in}{1.075658in}}%
\pgfpathlineto{\pgfqpoint{4.033038in}{1.157813in}}%
\pgfpathlineto{\pgfqpoint{4.034382in}{1.088909in}}%
\pgfpathlineto{\pgfqpoint{4.035054in}{1.088909in}}%
\pgfpathlineto{\pgfqpoint{4.035054in}{1.043856in}}%
\pgfpathlineto{\pgfqpoint{4.035727in}{1.141912in}}%
\pgfpathlineto{\pgfqpoint{4.036399in}{1.088909in}}%
\pgfpathlineto{\pgfqpoint{4.037071in}{1.088909in}}%
\pgfpathlineto{\pgfqpoint{4.037071in}{1.171064in}}%
\pgfpathlineto{\pgfqpoint{4.038415in}{1.078308in}}%
\pgfpathlineto{\pgfqpoint{4.039088in}{1.078308in}}%
\pgfpathlineto{\pgfqpoint{4.039088in}{1.094209in}}%
\pgfpathlineto{\pgfqpoint{4.040432in}{1.091559in}}%
\pgfpathlineto{\pgfqpoint{4.041104in}{1.091559in}}%
\pgfpathlineto{\pgfqpoint{4.041776in}{1.115410in}}%
\pgfpathlineto{\pgfqpoint{4.042448in}{1.080958in}}%
\pgfpathlineto{\pgfqpoint{4.043121in}{1.080958in}}%
\pgfpathlineto{\pgfqpoint{4.043121in}{1.075658in}}%
\pgfpathlineto{\pgfqpoint{4.043793in}{1.126011in}}%
\pgfpathlineto{\pgfqpoint{4.044465in}{1.075658in}}%
\pgfpathlineto{\pgfqpoint{4.045137in}{1.075658in}}%
\pgfpathlineto{\pgfqpoint{4.045137in}{1.088909in}}%
\pgfpathlineto{\pgfqpoint{4.046481in}{1.049156in}}%
\pgfpathlineto{\pgfqpoint{4.047154in}{1.049156in}}%
\pgfpathlineto{\pgfqpoint{4.047154in}{1.131311in}}%
\pgfpathlineto{\pgfqpoint{4.048498in}{1.118060in}}%
\pgfpathlineto{\pgfqpoint{4.049170in}{1.118060in}}%
\pgfpathlineto{\pgfqpoint{4.049842in}{1.096859in}}%
\pgfpathlineto{\pgfqpoint{4.050514in}{1.149863in}}%
\pgfpathlineto{\pgfqpoint{4.051187in}{1.149863in}}%
\pgfpathlineto{\pgfqpoint{4.051187in}{1.051806in}}%
\pgfpathlineto{\pgfqpoint{4.052531in}{1.088909in}}%
\pgfpathlineto{\pgfqpoint{4.053203in}{1.088909in}}%
\pgfpathlineto{\pgfqpoint{4.053203in}{1.051806in}}%
\pgfpathlineto{\pgfqpoint{4.053875in}{1.141912in}}%
\pgfpathlineto{\pgfqpoint{4.054547in}{1.102159in}}%
\pgfpathlineto{\pgfqpoint{4.055220in}{1.102159in}}%
\pgfpathlineto{\pgfqpoint{4.055892in}{1.051806in}}%
\pgfpathlineto{\pgfqpoint{4.056564in}{1.189615in}}%
\pgfpathlineto{\pgfqpoint{4.057236in}{1.189615in}}%
\pgfpathlineto{\pgfqpoint{4.058580in}{1.118060in}}%
\pgfpathlineto{\pgfqpoint{4.059253in}{1.118060in}}%
\pgfpathlineto{\pgfqpoint{4.059253in}{1.054456in}}%
\pgfpathlineto{\pgfqpoint{4.060597in}{1.091559in}}%
\pgfpathlineto{\pgfqpoint{4.061269in}{1.091559in}}%
\pgfpathlineto{\pgfqpoint{4.061941in}{1.083608in}}%
\pgfpathlineto{\pgfqpoint{4.062613in}{1.189615in}}%
\pgfpathlineto{\pgfqpoint{4.063286in}{1.189615in}}%
\pgfpathlineto{\pgfqpoint{4.064630in}{1.099509in}}%
\pgfpathlineto{\pgfqpoint{4.065302in}{1.099509in}}%
\pgfpathlineto{\pgfqpoint{4.065302in}{1.131311in}}%
\pgfpathlineto{\pgfqpoint{4.065974in}{1.067707in}}%
\pgfpathlineto{\pgfqpoint{4.066646in}{1.067707in}}%
\pgfpathlineto{\pgfqpoint{4.067319in}{1.067707in}}%
\pgfpathlineto{\pgfqpoint{4.067991in}{1.141912in}}%
\pgfpathlineto{\pgfqpoint{4.068663in}{1.118060in}}%
\pgfpathlineto{\pgfqpoint{4.069335in}{1.118060in}}%
\pgfpathlineto{\pgfqpoint{4.070007in}{1.136612in}}%
\pgfpathlineto{\pgfqpoint{4.070679in}{1.086258in}}%
\pgfpathlineto{\pgfqpoint{4.071352in}{1.086258in}}%
\pgfpathlineto{\pgfqpoint{4.072024in}{1.080958in}}%
\pgfpathlineto{\pgfqpoint{4.072696in}{1.136612in}}%
\pgfpathlineto{\pgfqpoint{4.074040in}{1.136612in}}%
\pgfpathlineto{\pgfqpoint{4.074712in}{1.083608in}}%
\pgfpathlineto{\pgfqpoint{4.075385in}{1.112760in}}%
\pgfpathlineto{\pgfqpoint{4.076057in}{1.112760in}}%
\pgfpathlineto{\pgfqpoint{4.076729in}{1.091559in}}%
\pgfpathlineto{\pgfqpoint{4.077401in}{1.144562in}}%
\pgfpathlineto{\pgfqpoint{4.078073in}{1.144562in}}%
\pgfpathlineto{\pgfqpoint{4.079418in}{1.070357in}}%
\pgfpathlineto{\pgfqpoint{4.080090in}{1.070357in}}%
\pgfpathlineto{\pgfqpoint{4.080762in}{1.171064in}}%
\pgfpathlineto{\pgfqpoint{4.081434in}{1.136612in}}%
\pgfpathlineto{\pgfqpoint{4.082106in}{1.136612in}}%
\pgfpathlineto{\pgfqpoint{4.082106in}{1.110110in}}%
\pgfpathlineto{\pgfqpoint{4.083451in}{1.133962in}}%
\pgfpathlineto{\pgfqpoint{4.084123in}{1.133962in}}%
\pgfpathlineto{\pgfqpoint{4.084795in}{1.173714in}}%
\pgfpathlineto{\pgfqpoint{4.085467in}{1.126011in}}%
\pgfpathlineto{\pgfqpoint{4.086139in}{1.126011in}}%
\pgfpathlineto{\pgfqpoint{4.086811in}{1.020004in}}%
\pgfpathlineto{\pgfqpoint{4.087484in}{1.155163in}}%
\pgfpathlineto{\pgfqpoint{4.088156in}{1.155163in}}%
\pgfpathlineto{\pgfqpoint{4.088828in}{1.147212in}}%
\pgfpathlineto{\pgfqpoint{4.089500in}{1.171064in}}%
\pgfpathlineto{\pgfqpoint{4.090172in}{1.171064in}}%
\pgfpathlineto{\pgfqpoint{4.091517in}{1.107460in}}%
\pgfpathlineto{\pgfqpoint{4.092189in}{1.107460in}}%
\pgfpathlineto{\pgfqpoint{4.092861in}{1.171064in}}%
\pgfpathlineto{\pgfqpoint{4.093533in}{1.110110in}}%
\pgfpathlineto{\pgfqpoint{4.094205in}{1.110110in}}%
\pgfpathlineto{\pgfqpoint{4.094205in}{1.065057in}}%
\pgfpathlineto{\pgfqpoint{4.095550in}{1.155163in}}%
\pgfpathlineto{\pgfqpoint{4.096222in}{1.155163in}}%
\pgfpathlineto{\pgfqpoint{4.097566in}{1.091559in}}%
\pgfpathlineto{\pgfqpoint{4.098910in}{1.091559in}}%
\pgfpathlineto{\pgfqpoint{4.098910in}{1.149863in}}%
\pgfpathlineto{\pgfqpoint{4.100255in}{1.141912in}}%
\pgfpathlineto{\pgfqpoint{4.100927in}{1.141912in}}%
\pgfpathlineto{\pgfqpoint{4.101599in}{1.128661in}}%
\pgfpathlineto{\pgfqpoint{4.102271in}{1.181665in}}%
\pgfpathlineto{\pgfqpoint{4.102943in}{1.181665in}}%
\pgfpathlineto{\pgfqpoint{4.104288in}{1.083608in}}%
\pgfpathlineto{\pgfqpoint{4.104960in}{1.083608in}}%
\pgfpathlineto{\pgfqpoint{4.104960in}{1.181665in}}%
\pgfpathlineto{\pgfqpoint{4.106304in}{1.155163in}}%
\pgfpathlineto{\pgfqpoint{4.106976in}{1.155163in}}%
\pgfpathlineto{\pgfqpoint{4.106976in}{1.133962in}}%
\pgfpathlineto{\pgfqpoint{4.108321in}{1.176364in}}%
\pgfpathlineto{\pgfqpoint{4.108993in}{1.176364in}}%
\pgfpathlineto{\pgfqpoint{4.108993in}{1.181665in}}%
\pgfpathlineto{\pgfqpoint{4.109665in}{1.144562in}}%
\pgfpathlineto{\pgfqpoint{4.110337in}{1.149863in}}%
\pgfpathlineto{\pgfqpoint{4.111009in}{1.149863in}}%
\pgfpathlineto{\pgfqpoint{4.111682in}{1.200216in}}%
\pgfpathlineto{\pgfqpoint{4.112354in}{1.115410in}}%
\pgfpathlineto{\pgfqpoint{4.113026in}{1.115410in}}%
\pgfpathlineto{\pgfqpoint{4.113026in}{1.110110in}}%
\pgfpathlineto{\pgfqpoint{4.113698in}{1.173714in}}%
\pgfpathlineto{\pgfqpoint{4.114370in}{1.110110in}}%
\pgfpathlineto{\pgfqpoint{4.115042in}{1.110110in}}%
\pgfpathlineto{\pgfqpoint{4.116387in}{1.208166in}}%
\pgfpathlineto{\pgfqpoint{4.117059in}{1.208166in}}%
\pgfpathlineto{\pgfqpoint{4.118403in}{1.065057in}}%
\pgfpathlineto{\pgfqpoint{4.119075in}{1.065057in}}%
\pgfpathlineto{\pgfqpoint{4.120420in}{1.192265in}}%
\pgfpathlineto{\pgfqpoint{4.121092in}{1.192265in}}%
\pgfpathlineto{\pgfqpoint{4.122436in}{1.075658in}}%
\pgfpathlineto{\pgfqpoint{4.123108in}{1.075658in}}%
\pgfpathlineto{\pgfqpoint{4.123781in}{1.112760in}}%
\pgfpathlineto{\pgfqpoint{4.124453in}{1.080958in}}%
\pgfpathlineto{\pgfqpoint{4.125125in}{1.080958in}}%
\pgfpathlineto{\pgfqpoint{4.125797in}{1.149863in}}%
\pgfpathlineto{\pgfqpoint{4.126469in}{1.115410in}}%
\pgfpathlineto{\pgfqpoint{4.127141in}{1.115410in}}%
\pgfpathlineto{\pgfqpoint{4.127141in}{1.205516in}}%
\pgfpathlineto{\pgfqpoint{4.127814in}{1.051806in}}%
\pgfpathlineto{\pgfqpoint{4.128486in}{1.115410in}}%
\pgfpathlineto{\pgfqpoint{4.129158in}{1.115410in}}%
\pgfpathlineto{\pgfqpoint{4.129830in}{1.141912in}}%
\pgfpathlineto{\pgfqpoint{4.130502in}{1.088909in}}%
\pgfpathlineto{\pgfqpoint{4.131174in}{1.088909in}}%
\pgfpathlineto{\pgfqpoint{4.131174in}{1.171064in}}%
\pgfpathlineto{\pgfqpoint{4.132519in}{1.073007in}}%
\pgfpathlineto{\pgfqpoint{4.133191in}{1.073007in}}%
\pgfpathlineto{\pgfqpoint{4.133863in}{1.065057in}}%
\pgfpathlineto{\pgfqpoint{4.134535in}{1.168414in}}%
\pgfpathlineto{\pgfqpoint{4.135207in}{1.168414in}}%
\pgfpathlineto{\pgfqpoint{4.135207in}{1.083608in}}%
\pgfpathlineto{\pgfqpoint{4.136552in}{1.194916in}}%
\pgfpathlineto{\pgfqpoint{4.137224in}{1.194916in}}%
\pgfpathlineto{\pgfqpoint{4.138568in}{1.096859in}}%
\pgfpathlineto{\pgfqpoint{4.139240in}{1.096859in}}%
\pgfpathlineto{\pgfqpoint{4.140585in}{1.171064in}}%
\pgfpathlineto{\pgfqpoint{4.141257in}{1.171064in}}%
\pgfpathlineto{\pgfqpoint{4.141929in}{1.073007in}}%
\pgfpathlineto{\pgfqpoint{4.142601in}{1.163113in}}%
\pgfpathlineto{\pgfqpoint{4.143273in}{1.163113in}}%
\pgfpathlineto{\pgfqpoint{4.143273in}{1.189615in}}%
\pgfpathlineto{\pgfqpoint{4.144618in}{1.059757in}}%
\pgfpathlineto{\pgfqpoint{4.145290in}{1.059757in}}%
\pgfpathlineto{\pgfqpoint{4.146634in}{1.224068in}}%
\pgfpathlineto{\pgfqpoint{4.147306in}{1.224068in}}%
\pgfpathlineto{\pgfqpoint{4.147306in}{1.083608in}}%
\pgfpathlineto{\pgfqpoint{4.148651in}{1.152513in}}%
\pgfpathlineto{\pgfqpoint{4.149323in}{1.152513in}}%
\pgfpathlineto{\pgfqpoint{4.149323in}{1.157813in}}%
\pgfpathlineto{\pgfqpoint{4.149995in}{1.112760in}}%
\pgfpathlineto{\pgfqpoint{4.150667in}{1.139262in}}%
\pgfpathlineto{\pgfqpoint{4.151339in}{1.139262in}}%
\pgfpathlineto{\pgfqpoint{4.152684in}{1.041205in}}%
\pgfpathlineto{\pgfqpoint{4.153356in}{1.041205in}}%
\pgfpathlineto{\pgfqpoint{4.153356in}{1.099509in}}%
\pgfpathlineto{\pgfqpoint{4.154700in}{1.067707in}}%
\pgfpathlineto{\pgfqpoint{4.155373in}{1.067707in}}%
\pgfpathlineto{\pgfqpoint{4.155373in}{1.176364in}}%
\pgfpathlineto{\pgfqpoint{4.156717in}{1.123361in}}%
\pgfpathlineto{\pgfqpoint{4.157389in}{1.123361in}}%
\pgfpathlineto{\pgfqpoint{4.157389in}{1.152513in}}%
\pgfpathlineto{\pgfqpoint{4.158733in}{1.073007in}}%
\pgfpathlineto{\pgfqpoint{4.159406in}{1.073007in}}%
\pgfpathlineto{\pgfqpoint{4.159406in}{1.118060in}}%
\pgfpathlineto{\pgfqpoint{4.160750in}{1.062407in}}%
\pgfpathlineto{\pgfqpoint{4.161422in}{1.062407in}}%
\pgfpathlineto{\pgfqpoint{4.162094in}{1.096859in}}%
\pgfpathlineto{\pgfqpoint{4.162766in}{1.096859in}}%
\pgfpathlineto{\pgfqpoint{4.163439in}{1.096859in}}%
\pgfpathlineto{\pgfqpoint{4.164783in}{1.173714in}}%
\pgfpathlineto{\pgfqpoint{4.165455in}{1.173714in}}%
\pgfpathlineto{\pgfqpoint{4.165455in}{1.083608in}}%
\pgfpathlineto{\pgfqpoint{4.166799in}{1.131311in}}%
\pgfpathlineto{\pgfqpoint{4.167472in}{1.131311in}}%
\pgfpathlineto{\pgfqpoint{4.168144in}{1.080958in}}%
\pgfpathlineto{\pgfqpoint{4.168816in}{1.112760in}}%
\pgfpathlineto{\pgfqpoint{4.169488in}{1.112760in}}%
\pgfpathlineto{\pgfqpoint{4.169488in}{1.107460in}}%
\pgfpathlineto{\pgfqpoint{4.170832in}{1.144562in}}%
\pgfpathlineto{\pgfqpoint{4.171505in}{1.144562in}}%
\pgfpathlineto{\pgfqpoint{4.172849in}{1.088909in}}%
\pgfpathlineto{\pgfqpoint{4.173521in}{1.088909in}}%
\pgfpathlineto{\pgfqpoint{4.173521in}{1.059757in}}%
\pgfpathlineto{\pgfqpoint{4.174865in}{1.118060in}}%
\pgfpathlineto{\pgfqpoint{4.175538in}{1.118060in}}%
\pgfpathlineto{\pgfqpoint{4.175538in}{1.099509in}}%
\pgfpathlineto{\pgfqpoint{4.176210in}{1.155163in}}%
\pgfpathlineto{\pgfqpoint{4.176882in}{1.133962in}}%
\pgfpathlineto{\pgfqpoint{4.177554in}{1.133962in}}%
\pgfpathlineto{\pgfqpoint{4.177554in}{1.035905in}}%
\pgfpathlineto{\pgfqpoint{4.178898in}{1.078308in}}%
\pgfpathlineto{\pgfqpoint{4.179571in}{1.078308in}}%
\pgfpathlineto{\pgfqpoint{4.180243in}{1.147212in}}%
\pgfpathlineto{\pgfqpoint{4.180915in}{1.091559in}}%
\pgfpathlineto{\pgfqpoint{4.181587in}{1.091559in}}%
\pgfpathlineto{\pgfqpoint{4.181587in}{1.059757in}}%
\pgfpathlineto{\pgfqpoint{4.182931in}{1.070357in}}%
\pgfpathlineto{\pgfqpoint{4.183604in}{1.070357in}}%
\pgfpathlineto{\pgfqpoint{4.184276in}{1.171064in}}%
\pgfpathlineto{\pgfqpoint{4.184948in}{1.149863in}}%
\pgfpathlineto{\pgfqpoint{4.185620in}{1.149863in}}%
\pgfpathlineto{\pgfqpoint{4.185620in}{1.115410in}}%
\pgfpathlineto{\pgfqpoint{4.186964in}{1.120711in}}%
\pgfpathlineto{\pgfqpoint{4.187637in}{1.120711in}}%
\pgfpathlineto{\pgfqpoint{4.188309in}{1.030605in}}%
\pgfpathlineto{\pgfqpoint{4.188981in}{1.136612in}}%
\pgfpathlineto{\pgfqpoint{4.189653in}{1.136612in}}%
\pgfpathlineto{\pgfqpoint{4.189653in}{1.141912in}}%
\pgfpathlineto{\pgfqpoint{4.190997in}{1.070357in}}%
\pgfpathlineto{\pgfqpoint{4.191670in}{1.070357in}}%
\pgfpathlineto{\pgfqpoint{4.191670in}{1.136612in}}%
\pgfpathlineto{\pgfqpoint{4.193014in}{1.112760in}}%
\pgfpathlineto{\pgfqpoint{4.193686in}{1.112760in}}%
\pgfpathlineto{\pgfqpoint{4.195030in}{1.065057in}}%
\pgfpathlineto{\pgfqpoint{4.195703in}{1.065057in}}%
\pgfpathlineto{\pgfqpoint{4.195703in}{1.115410in}}%
\pgfpathlineto{\pgfqpoint{4.197047in}{1.046506in}}%
\pgfpathlineto{\pgfqpoint{4.197719in}{1.046506in}}%
\pgfpathlineto{\pgfqpoint{4.197719in}{1.088909in}}%
\pgfpathlineto{\pgfqpoint{4.199063in}{1.067707in}}%
\pgfpathlineto{\pgfqpoint{4.199736in}{1.067707in}}%
\pgfpathlineto{\pgfqpoint{4.199736in}{1.141912in}}%
\pgfpathlineto{\pgfqpoint{4.201080in}{1.133962in}}%
\pgfpathlineto{\pgfqpoint{4.201752in}{1.133962in}}%
\pgfpathlineto{\pgfqpoint{4.202424in}{1.020004in}}%
\pgfpathlineto{\pgfqpoint{4.203096in}{1.041205in}}%
\pgfpathlineto{\pgfqpoint{4.203769in}{1.041205in}}%
\pgfpathlineto{\pgfqpoint{4.204441in}{1.033255in}}%
\pgfpathlineto{\pgfqpoint{4.205113in}{1.141912in}}%
\pgfpathlineto{\pgfqpoint{4.205785in}{1.141912in}}%
\pgfpathlineto{\pgfqpoint{4.205785in}{1.059757in}}%
\pgfpathlineto{\pgfqpoint{4.207129in}{1.078308in}}%
\pgfpathlineto{\pgfqpoint{4.207802in}{1.078308in}}%
\pgfpathlineto{\pgfqpoint{4.207802in}{1.091559in}}%
\pgfpathlineto{\pgfqpoint{4.209146in}{1.017354in}}%
\pgfpathlineto{\pgfqpoint{4.209818in}{1.017354in}}%
\pgfpathlineto{\pgfqpoint{4.209818in}{1.112760in}}%
\pgfpathlineto{\pgfqpoint{4.211162in}{1.062407in}}%
\pgfpathlineto{\pgfqpoint{4.211835in}{1.062407in}}%
\pgfpathlineto{\pgfqpoint{4.211835in}{1.059757in}}%
\pgfpathlineto{\pgfqpoint{4.212507in}{1.094209in}}%
\pgfpathlineto{\pgfqpoint{4.213179in}{1.059757in}}%
\pgfpathlineto{\pgfqpoint{4.213851in}{1.059757in}}%
\pgfpathlineto{\pgfqpoint{4.213851in}{1.115410in}}%
\pgfpathlineto{\pgfqpoint{4.215195in}{1.102159in}}%
\pgfpathlineto{\pgfqpoint{4.215868in}{1.102159in}}%
\pgfpathlineto{\pgfqpoint{4.215868in}{1.168414in}}%
\pgfpathlineto{\pgfqpoint{4.216540in}{1.065057in}}%
\pgfpathlineto{\pgfqpoint{4.217212in}{1.091559in}}%
\pgfpathlineto{\pgfqpoint{4.217884in}{1.091559in}}%
\pgfpathlineto{\pgfqpoint{4.218556in}{1.038555in}}%
\pgfpathlineto{\pgfqpoint{4.219228in}{1.112760in}}%
\pgfpathlineto{\pgfqpoint{4.219901in}{1.112760in}}%
\pgfpathlineto{\pgfqpoint{4.220573in}{1.038555in}}%
\pgfpathlineto{\pgfqpoint{4.221245in}{1.049156in}}%
\pgfpathlineto{\pgfqpoint{4.221917in}{1.049156in}}%
\pgfpathlineto{\pgfqpoint{4.221917in}{1.006753in}}%
\pgfpathlineto{\pgfqpoint{4.222589in}{1.099509in}}%
\pgfpathlineto{\pgfqpoint{4.223261in}{1.086258in}}%
\pgfpathlineto{\pgfqpoint{4.223934in}{1.086258in}}%
\pgfpathlineto{\pgfqpoint{4.225278in}{1.035905in}}%
\pgfpathlineto{\pgfqpoint{4.225950in}{1.035905in}}%
\pgfpathlineto{\pgfqpoint{4.226622in}{1.014704in}}%
\pgfpathlineto{\pgfqpoint{4.227294in}{1.080958in}}%
\pgfpathlineto{\pgfqpoint{4.227967in}{1.080958in}}%
\pgfpathlineto{\pgfqpoint{4.228639in}{1.118060in}}%
\pgfpathlineto{\pgfqpoint{4.229311in}{1.107460in}}%
\pgfpathlineto{\pgfqpoint{4.229983in}{1.107460in}}%
\pgfpathlineto{\pgfqpoint{4.231327in}{1.035905in}}%
\pgfpathlineto{\pgfqpoint{4.232000in}{1.035905in}}%
\pgfpathlineto{\pgfqpoint{4.232000in}{1.012053in}}%
\pgfpathlineto{\pgfqpoint{4.233344in}{1.041205in}}%
\pgfpathlineto{\pgfqpoint{4.234016in}{1.041205in}}%
\pgfpathlineto{\pgfqpoint{4.235360in}{1.094209in}}%
\pgfpathlineto{\pgfqpoint{4.236033in}{1.094209in}}%
\pgfpathlineto{\pgfqpoint{4.236033in}{1.136612in}}%
\pgfpathlineto{\pgfqpoint{4.236705in}{1.046506in}}%
\pgfpathlineto{\pgfqpoint{4.237377in}{1.049156in}}%
\pgfpathlineto{\pgfqpoint{4.238049in}{1.049156in}}%
\pgfpathlineto{\pgfqpoint{4.238721in}{1.038555in}}%
\pgfpathlineto{\pgfqpoint{4.239393in}{1.139262in}}%
\pgfpathlineto{\pgfqpoint{4.240066in}{1.139262in}}%
\pgfpathlineto{\pgfqpoint{4.240066in}{1.046506in}}%
\pgfpathlineto{\pgfqpoint{4.241410in}{1.070357in}}%
\pgfpathlineto{\pgfqpoint{4.242082in}{1.070357in}}%
\pgfpathlineto{\pgfqpoint{4.242082in}{1.080958in}}%
\pgfpathlineto{\pgfqpoint{4.242754in}{1.049156in}}%
\pgfpathlineto{\pgfqpoint{4.243426in}{1.080958in}}%
\pgfpathlineto{\pgfqpoint{4.244099in}{1.080958in}}%
\pgfpathlineto{\pgfqpoint{4.244099in}{1.043856in}}%
\pgfpathlineto{\pgfqpoint{4.244771in}{1.112760in}}%
\pgfpathlineto{\pgfqpoint{4.245443in}{1.043856in}}%
\pgfpathlineto{\pgfqpoint{4.246115in}{1.043856in}}%
\pgfpathlineto{\pgfqpoint{4.246787in}{1.025304in}}%
\pgfpathlineto{\pgfqpoint{4.247459in}{1.067707in}}%
\pgfpathlineto{\pgfqpoint{4.248132in}{1.067707in}}%
\pgfpathlineto{\pgfqpoint{4.248132in}{1.051806in}}%
\pgfpathlineto{\pgfqpoint{4.249476in}{1.070357in}}%
\pgfpathlineto{\pgfqpoint{4.250148in}{1.070357in}}%
\pgfpathlineto{\pgfqpoint{4.251492in}{1.014704in}}%
\pgfpathlineto{\pgfqpoint{4.252165in}{1.014704in}}%
\pgfpathlineto{\pgfqpoint{4.252837in}{1.046506in}}%
\pgfpathlineto{\pgfqpoint{4.253509in}{1.030605in}}%
\pgfpathlineto{\pgfqpoint{4.254181in}{1.030605in}}%
\pgfpathlineto{\pgfqpoint{4.254181in}{0.996152in}}%
\pgfpathlineto{\pgfqpoint{4.254853in}{1.038555in}}%
\pgfpathlineto{\pgfqpoint{4.255525in}{1.022654in}}%
\pgfpathlineto{\pgfqpoint{4.256198in}{1.022654in}}%
\pgfpathlineto{\pgfqpoint{4.256870in}{1.099509in}}%
\pgfpathlineto{\pgfqpoint{4.257542in}{1.070357in}}%
\pgfpathlineto{\pgfqpoint{4.258214in}{1.070357in}}%
\pgfpathlineto{\pgfqpoint{4.258214in}{1.088909in}}%
\pgfpathlineto{\pgfqpoint{4.259558in}{1.073007in}}%
\pgfpathlineto{\pgfqpoint{4.260231in}{1.073007in}}%
\pgfpathlineto{\pgfqpoint{4.261575in}{1.022654in}}%
\pgfpathlineto{\pgfqpoint{4.262247in}{1.022654in}}%
\pgfpathlineto{\pgfqpoint{4.262919in}{1.083608in}}%
\pgfpathlineto{\pgfqpoint{4.263591in}{1.006753in}}%
\pgfpathlineto{\pgfqpoint{4.264264in}{1.006753in}}%
\pgfpathlineto{\pgfqpoint{4.265608in}{1.043856in}}%
\pgfpathlineto{\pgfqpoint{4.266280in}{1.043856in}}%
\pgfpathlineto{\pgfqpoint{4.266280in}{1.014704in}}%
\pgfpathlineto{\pgfqpoint{4.266952in}{1.067707in}}%
\pgfpathlineto{\pgfqpoint{4.267625in}{1.051806in}}%
\pgfpathlineto{\pgfqpoint{4.268297in}{1.051806in}}%
\pgfpathlineto{\pgfqpoint{4.268969in}{1.075658in}}%
\pgfpathlineto{\pgfqpoint{4.269641in}{1.020004in}}%
\pgfpathlineto{\pgfqpoint{4.270313in}{1.020004in}}%
\pgfpathlineto{\pgfqpoint{4.270313in}{1.073007in}}%
\pgfpathlineto{\pgfqpoint{4.270985in}{0.998803in}}%
\pgfpathlineto{\pgfqpoint{4.271658in}{1.051806in}}%
\pgfpathlineto{\pgfqpoint{4.272330in}{1.051806in}}%
\pgfpathlineto{\pgfqpoint{4.273002in}{1.027954in}}%
\pgfpathlineto{\pgfqpoint{4.273002in}{1.065057in}}%
\pgfpathlineto{\pgfqpoint{4.273674in}{1.051806in}}%
\pgfpathlineto{\pgfqpoint{4.274346in}{1.051806in}}%
\pgfpathlineto{\pgfqpoint{4.275691in}{1.022654in}}%
\pgfpathlineto{\pgfqpoint{4.276363in}{1.022654in}}%
\pgfpathlineto{\pgfqpoint{4.276363in}{1.051806in}}%
\pgfpathlineto{\pgfqpoint{4.277035in}{1.020004in}}%
\pgfpathlineto{\pgfqpoint{4.277707in}{1.035905in}}%
\pgfpathlineto{\pgfqpoint{4.278379in}{1.035905in}}%
\pgfpathlineto{\pgfqpoint{4.279051in}{1.075658in}}%
\pgfpathlineto{\pgfqpoint{4.279724in}{1.065057in}}%
\pgfpathlineto{\pgfqpoint{4.281068in}{1.065057in}}%
\pgfpathlineto{\pgfqpoint{4.281068in}{1.070357in}}%
\pgfpathlineto{\pgfqpoint{4.281740in}{0.988202in}}%
\pgfpathlineto{\pgfqpoint{4.282412in}{1.035905in}}%
\pgfpathlineto{\pgfqpoint{4.283084in}{1.035905in}}%
\pgfpathlineto{\pgfqpoint{4.283757in}{1.006753in}}%
\pgfpathlineto{\pgfqpoint{4.284429in}{1.062407in}}%
\pgfpathlineto{\pgfqpoint{4.285101in}{1.062407in}}%
\pgfpathlineto{\pgfqpoint{4.285773in}{0.982901in}}%
\pgfpathlineto{\pgfqpoint{4.286445in}{1.009403in}}%
\pgfpathlineto{\pgfqpoint{4.287117in}{1.009403in}}%
\pgfpathlineto{\pgfqpoint{4.287790in}{1.057106in}}%
\pgfpathlineto{\pgfqpoint{4.288462in}{0.990852in}}%
\pgfpathlineto{\pgfqpoint{4.289806in}{0.990852in}}%
\pgfpathlineto{\pgfqpoint{4.289806in}{1.057106in}}%
\pgfpathlineto{\pgfqpoint{4.291150in}{1.033255in}}%
\pgfpathlineto{\pgfqpoint{4.292495in}{1.033255in}}%
\pgfpathlineto{\pgfqpoint{4.292495in}{0.996152in}}%
\pgfpathlineto{\pgfqpoint{4.293839in}{1.067707in}}%
\pgfpathlineto{\pgfqpoint{4.294511in}{1.067707in}}%
\pgfpathlineto{\pgfqpoint{4.295856in}{0.998803in}}%
\pgfpathlineto{\pgfqpoint{4.296528in}{0.998803in}}%
\pgfpathlineto{\pgfqpoint{4.296528in}{1.062407in}}%
\pgfpathlineto{\pgfqpoint{4.297872in}{1.033255in}}%
\pgfpathlineto{\pgfqpoint{4.298544in}{1.033255in}}%
\pgfpathlineto{\pgfqpoint{4.298544in}{0.993502in}}%
\pgfpathlineto{\pgfqpoint{4.299889in}{1.086258in}}%
\pgfpathlineto{\pgfqpoint{4.300561in}{1.086258in}}%
\pgfpathlineto{\pgfqpoint{4.300561in}{0.982901in}}%
\pgfpathlineto{\pgfqpoint{4.301905in}{0.990852in}}%
\pgfpathlineto{\pgfqpoint{4.302577in}{0.990852in}}%
\pgfpathlineto{\pgfqpoint{4.302577in}{1.022654in}}%
\pgfpathlineto{\pgfqpoint{4.303922in}{1.004103in}}%
\pgfpathlineto{\pgfqpoint{4.304594in}{1.004103in}}%
\pgfpathlineto{\pgfqpoint{4.305266in}{0.980251in}}%
\pgfpathlineto{\pgfqpoint{4.305938in}{1.067707in}}%
\pgfpathlineto{\pgfqpoint{4.306610in}{1.067707in}}%
\pgfpathlineto{\pgfqpoint{4.307955in}{0.969651in}}%
\pgfpathlineto{\pgfqpoint{4.308627in}{0.969651in}}%
\pgfpathlineto{\pgfqpoint{4.309971in}{1.049156in}}%
\pgfpathlineto{\pgfqpoint{4.310643in}{1.049156in}}%
\pgfpathlineto{\pgfqpoint{4.310643in}{0.980251in}}%
\pgfpathlineto{\pgfqpoint{4.311315in}{1.054456in}}%
\pgfpathlineto{\pgfqpoint{4.311988in}{0.982901in}}%
\pgfpathlineto{\pgfqpoint{4.312660in}{0.982901in}}%
\pgfpathlineto{\pgfqpoint{4.312660in}{0.998803in}}%
\pgfpathlineto{\pgfqpoint{4.313332in}{0.972301in}}%
\pgfpathlineto{\pgfqpoint{4.314004in}{0.982901in}}%
\pgfpathlineto{\pgfqpoint{4.314676in}{0.982901in}}%
\pgfpathlineto{\pgfqpoint{4.314676in}{1.051806in}}%
\pgfpathlineto{\pgfqpoint{4.316021in}{0.990852in}}%
\pgfpathlineto{\pgfqpoint{4.316693in}{0.990852in}}%
\pgfpathlineto{\pgfqpoint{4.318037in}{1.046506in}}%
\pgfpathlineto{\pgfqpoint{4.318709in}{1.046506in}}%
\pgfpathlineto{\pgfqpoint{4.320054in}{0.993502in}}%
\pgfpathlineto{\pgfqpoint{4.320726in}{0.993502in}}%
\pgfpathlineto{\pgfqpoint{4.320726in}{0.964350in}}%
\pgfpathlineto{\pgfqpoint{4.321398in}{1.025304in}}%
\pgfpathlineto{\pgfqpoint{4.322070in}{1.001453in}}%
\pgfpathlineto{\pgfqpoint{4.322742in}{1.001453in}}%
\pgfpathlineto{\pgfqpoint{4.322742in}{0.980251in}}%
\pgfpathlineto{\pgfqpoint{4.324087in}{1.025304in}}%
\pgfpathlineto{\pgfqpoint{4.324759in}{1.025304in}}%
\pgfpathlineto{\pgfqpoint{4.326103in}{0.985552in}}%
\pgfpathlineto{\pgfqpoint{4.326775in}{0.985552in}}%
\pgfpathlineto{\pgfqpoint{4.327447in}{1.041205in}}%
\pgfpathlineto{\pgfqpoint{4.328120in}{0.996152in}}%
\pgfpathlineto{\pgfqpoint{4.329464in}{0.996152in}}%
\pgfpathlineto{\pgfqpoint{4.329464in}{0.964350in}}%
\pgfpathlineto{\pgfqpoint{4.330808in}{1.035905in}}%
\pgfpathlineto{\pgfqpoint{4.331480in}{1.035905in}}%
\pgfpathlineto{\pgfqpoint{4.332825in}{0.980251in}}%
\pgfpathlineto{\pgfqpoint{4.333497in}{0.980251in}}%
\pgfpathlineto{\pgfqpoint{4.333497in}{0.969651in}}%
\pgfpathlineto{\pgfqpoint{4.334169in}{0.998803in}}%
\pgfpathlineto{\pgfqpoint{4.334841in}{0.990852in}}%
\pgfpathlineto{\pgfqpoint{4.335513in}{0.990852in}}%
\pgfpathlineto{\pgfqpoint{4.335513in}{0.956400in}}%
\pgfpathlineto{\pgfqpoint{4.336858in}{0.961700in}}%
\pgfpathlineto{\pgfqpoint{4.337530in}{0.961700in}}%
\pgfpathlineto{\pgfqpoint{4.338202in}{0.998803in}}%
\pgfpathlineto{\pgfqpoint{4.338874in}{0.985552in}}%
\pgfpathlineto{\pgfqpoint{4.339546in}{0.985552in}}%
\pgfpathlineto{\pgfqpoint{4.340891in}{0.940499in}}%
\pgfpathlineto{\pgfqpoint{4.341563in}{0.940499in}}%
\pgfpathlineto{\pgfqpoint{4.342235in}{0.988202in}}%
\pgfpathlineto{\pgfqpoint{4.342907in}{0.927248in}}%
\pgfpathlineto{\pgfqpoint{4.343579in}{0.927248in}}%
\pgfpathlineto{\pgfqpoint{4.343579in}{0.980251in}}%
\pgfpathlineto{\pgfqpoint{4.344924in}{0.980251in}}%
\pgfpathlineto{\pgfqpoint{4.345596in}{0.980251in}}%
\pgfpathlineto{\pgfqpoint{4.345596in}{1.043856in}}%
\pgfpathlineto{\pgfqpoint{4.346940in}{0.977601in}}%
\pgfpathlineto{\pgfqpoint{4.347612in}{0.977601in}}%
\pgfpathlineto{\pgfqpoint{4.348285in}{0.943149in}}%
\pgfpathlineto{\pgfqpoint{4.348957in}{1.030605in}}%
\pgfpathlineto{\pgfqpoint{4.349629in}{1.030605in}}%
\pgfpathlineto{\pgfqpoint{4.349629in}{0.961700in}}%
\pgfpathlineto{\pgfqpoint{4.350973in}{0.982901in}}%
\pgfpathlineto{\pgfqpoint{4.351645in}{0.982901in}}%
\pgfpathlineto{\pgfqpoint{4.351645in}{1.004103in}}%
\pgfpathlineto{\pgfqpoint{4.352318in}{0.951099in}}%
\pgfpathlineto{\pgfqpoint{4.352990in}{0.959050in}}%
\pgfpathlineto{\pgfqpoint{4.353662in}{0.959050in}}%
\pgfpathlineto{\pgfqpoint{4.355006in}{1.004103in}}%
\pgfpathlineto{\pgfqpoint{4.355678in}{1.004103in}}%
\pgfpathlineto{\pgfqpoint{4.355678in}{0.929898in}}%
\pgfpathlineto{\pgfqpoint{4.357023in}{0.996152in}}%
\pgfpathlineto{\pgfqpoint{4.357695in}{0.996152in}}%
\pgfpathlineto{\pgfqpoint{4.357695in}{0.977601in}}%
\pgfpathlineto{\pgfqpoint{4.359039in}{0.980251in}}%
\pgfpathlineto{\pgfqpoint{4.360384in}{0.980251in}}%
\pgfpathlineto{\pgfqpoint{4.361728in}{1.012053in}}%
\pgfpathlineto{\pgfqpoint{4.362400in}{1.012053in}}%
\pgfpathlineto{\pgfqpoint{4.363744in}{0.972301in}}%
\pgfpathlineto{\pgfqpoint{4.364417in}{0.972301in}}%
\pgfpathlineto{\pgfqpoint{4.365089in}{0.924598in}}%
\pgfpathlineto{\pgfqpoint{4.365761in}{1.017354in}}%
\pgfpathlineto{\pgfqpoint{4.366433in}{1.017354in}}%
\pgfpathlineto{\pgfqpoint{4.367105in}{0.921947in}}%
\pgfpathlineto{\pgfqpoint{4.367777in}{0.956400in}}%
\pgfpathlineto{\pgfqpoint{4.368450in}{0.956400in}}%
\pgfpathlineto{\pgfqpoint{4.369122in}{0.903396in}}%
\pgfpathlineto{\pgfqpoint{4.369794in}{0.998803in}}%
\pgfpathlineto{\pgfqpoint{4.370466in}{0.998803in}}%
\pgfpathlineto{\pgfqpoint{4.371138in}{1.014704in}}%
\pgfpathlineto{\pgfqpoint{4.371810in}{0.972301in}}%
\pgfpathlineto{\pgfqpoint{4.372483in}{0.972301in}}%
\pgfpathlineto{\pgfqpoint{4.373155in}{0.948449in}}%
\pgfpathlineto{\pgfqpoint{4.373827in}{0.974951in}}%
\pgfpathlineto{\pgfqpoint{4.374499in}{0.974951in}}%
\pgfpathlineto{\pgfqpoint{4.374499in}{0.961700in}}%
\pgfpathlineto{\pgfqpoint{4.375843in}{0.990852in}}%
\pgfpathlineto{\pgfqpoint{4.376516in}{0.990852in}}%
\pgfpathlineto{\pgfqpoint{4.377188in}{1.004103in}}%
\pgfpathlineto{\pgfqpoint{4.377860in}{0.959050in}}%
\pgfpathlineto{\pgfqpoint{4.378532in}{0.959050in}}%
\pgfpathlineto{\pgfqpoint{4.378532in}{0.996152in}}%
\pgfpathlineto{\pgfqpoint{4.379877in}{0.929898in}}%
\pgfpathlineto{\pgfqpoint{4.380549in}{0.929898in}}%
\pgfpathlineto{\pgfqpoint{4.381893in}{1.001453in}}%
\pgfpathlineto{\pgfqpoint{4.382565in}{1.001453in}}%
\pgfpathlineto{\pgfqpoint{4.383237in}{1.004103in}}%
\pgfpathlineto{\pgfqpoint{4.383910in}{0.916647in}}%
\pgfpathlineto{\pgfqpoint{4.384582in}{0.916647in}}%
\pgfpathlineto{\pgfqpoint{4.384582in}{0.908697in}}%
\pgfpathlineto{\pgfqpoint{4.385254in}{0.943149in}}%
\pgfpathlineto{\pgfqpoint{4.385926in}{0.929898in}}%
\pgfpathlineto{\pgfqpoint{4.386598in}{0.929898in}}%
\pgfpathlineto{\pgfqpoint{4.386598in}{1.014704in}}%
\pgfpathlineto{\pgfqpoint{4.387270in}{0.916647in}}%
\pgfpathlineto{\pgfqpoint{4.387943in}{0.953750in}}%
\pgfpathlineto{\pgfqpoint{4.388615in}{0.953750in}}%
\pgfpathlineto{\pgfqpoint{4.389959in}{0.929898in}}%
\pgfpathlineto{\pgfqpoint{4.390631in}{0.929898in}}%
\pgfpathlineto{\pgfqpoint{4.391303in}{0.974951in}}%
\pgfpathlineto{\pgfqpoint{4.391976in}{0.884845in}}%
\pgfpathlineto{\pgfqpoint{4.392648in}{0.884845in}}%
\pgfpathlineto{\pgfqpoint{4.393320in}{0.948449in}}%
\pgfpathlineto{\pgfqpoint{4.393992in}{0.919297in}}%
\pgfpathlineto{\pgfqpoint{4.394664in}{0.919297in}}%
\pgfpathlineto{\pgfqpoint{4.394664in}{0.998803in}}%
\pgfpathlineto{\pgfqpoint{4.395336in}{0.916647in}}%
\pgfpathlineto{\pgfqpoint{4.396009in}{0.964350in}}%
\pgfpathlineto{\pgfqpoint{4.396681in}{0.964350in}}%
\pgfpathlineto{\pgfqpoint{4.396681in}{0.927248in}}%
\pgfpathlineto{\pgfqpoint{4.398025in}{0.974951in}}%
\pgfpathlineto{\pgfqpoint{4.398697in}{0.974951in}}%
\pgfpathlineto{\pgfqpoint{4.400042in}{0.906046in}}%
\pgfpathlineto{\pgfqpoint{4.400714in}{0.906046in}}%
\pgfpathlineto{\pgfqpoint{4.401386in}{0.948449in}}%
\pgfpathlineto{\pgfqpoint{4.402058in}{0.943149in}}%
\pgfpathlineto{\pgfqpoint{4.402730in}{0.943149in}}%
\pgfpathlineto{\pgfqpoint{4.402730in}{0.964350in}}%
\pgfpathlineto{\pgfqpoint{4.403402in}{0.935198in}}%
\pgfpathlineto{\pgfqpoint{4.404075in}{0.951099in}}%
\pgfpathlineto{\pgfqpoint{4.404747in}{0.951099in}}%
\pgfpathlineto{\pgfqpoint{4.404747in}{0.916647in}}%
\pgfpathlineto{\pgfqpoint{4.406091in}{0.951099in}}%
\pgfpathlineto{\pgfqpoint{4.406763in}{0.951099in}}%
\pgfpathlineto{\pgfqpoint{4.407435in}{0.903396in}}%
\pgfpathlineto{\pgfqpoint{4.408108in}{0.959050in}}%
\pgfpathlineto{\pgfqpoint{4.408780in}{0.959050in}}%
\pgfpathlineto{\pgfqpoint{4.408780in}{0.940499in}}%
\pgfpathlineto{\pgfqpoint{4.410124in}{0.948449in}}%
\pgfpathlineto{\pgfqpoint{4.410796in}{0.948449in}}%
\pgfpathlineto{\pgfqpoint{4.411468in}{0.935198in}}%
\pgfpathlineto{\pgfqpoint{4.412141in}{0.988202in}}%
\pgfpathlineto{\pgfqpoint{4.412813in}{0.988202in}}%
\pgfpathlineto{\pgfqpoint{4.414157in}{0.916647in}}%
\pgfpathlineto{\pgfqpoint{4.414829in}{0.916647in}}%
\pgfpathlineto{\pgfqpoint{4.414829in}{0.908697in}}%
\pgfpathlineto{\pgfqpoint{4.416174in}{0.998803in}}%
\pgfpathlineto{\pgfqpoint{4.416846in}{0.998803in}}%
\pgfpathlineto{\pgfqpoint{4.416846in}{0.892795in}}%
\pgfpathlineto{\pgfqpoint{4.418190in}{0.908697in}}%
\pgfpathlineto{\pgfqpoint{4.418862in}{0.908697in}}%
\pgfpathlineto{\pgfqpoint{4.418862in}{0.945799in}}%
\pgfpathlineto{\pgfqpoint{4.419534in}{0.892795in}}%
\pgfpathlineto{\pgfqpoint{4.420207in}{0.927248in}}%
\pgfpathlineto{\pgfqpoint{4.420879in}{0.927248in}}%
\pgfpathlineto{\pgfqpoint{4.420879in}{0.913997in}}%
\pgfpathlineto{\pgfqpoint{4.422223in}{0.940499in}}%
\pgfpathlineto{\pgfqpoint{4.422895in}{0.940499in}}%
\pgfpathlineto{\pgfqpoint{4.422895in}{0.932548in}}%
\pgfpathlineto{\pgfqpoint{4.423567in}{0.953750in}}%
\pgfpathlineto{\pgfqpoint{4.424240in}{0.948449in}}%
\pgfpathlineto{\pgfqpoint{4.424912in}{0.948449in}}%
\pgfpathlineto{\pgfqpoint{4.424912in}{0.911347in}}%
\pgfpathlineto{\pgfqpoint{4.425584in}{0.985552in}}%
\pgfpathlineto{\pgfqpoint{4.426256in}{0.953750in}}%
\pgfpathlineto{\pgfqpoint{4.427600in}{0.953750in}}%
\pgfpathlineto{\pgfqpoint{4.427600in}{0.961700in}}%
\pgfpathlineto{\pgfqpoint{4.428945in}{0.874244in}}%
\pgfpathlineto{\pgfqpoint{4.429617in}{0.874244in}}%
\pgfpathlineto{\pgfqpoint{4.430961in}{0.972301in}}%
\pgfpathlineto{\pgfqpoint{4.431633in}{0.972301in}}%
\pgfpathlineto{\pgfqpoint{4.432978in}{0.919297in}}%
\pgfpathlineto{\pgfqpoint{4.433650in}{0.919297in}}%
\pgfpathlineto{\pgfqpoint{4.434322in}{0.900746in}}%
\pgfpathlineto{\pgfqpoint{4.434994in}{0.956400in}}%
\pgfpathlineto{\pgfqpoint{4.435666in}{0.956400in}}%
\pgfpathlineto{\pgfqpoint{4.435666in}{0.959050in}}%
\pgfpathlineto{\pgfqpoint{4.436339in}{0.900746in}}%
\pgfpathlineto{\pgfqpoint{4.437011in}{0.945799in}}%
\pgfpathlineto{\pgfqpoint{4.437683in}{0.945799in}}%
\pgfpathlineto{\pgfqpoint{4.437683in}{0.903396in}}%
\pgfpathlineto{\pgfqpoint{4.438355in}{0.988202in}}%
\pgfpathlineto{\pgfqpoint{4.439027in}{0.906046in}}%
\pgfpathlineto{\pgfqpoint{4.439699in}{0.906046in}}%
\pgfpathlineto{\pgfqpoint{4.441044in}{0.953750in}}%
\pgfpathlineto{\pgfqpoint{4.441716in}{0.953750in}}%
\pgfpathlineto{\pgfqpoint{4.442388in}{0.882195in}}%
\pgfpathlineto{\pgfqpoint{4.443060in}{0.892795in}}%
\pgfpathlineto{\pgfqpoint{4.443732in}{0.892795in}}%
\pgfpathlineto{\pgfqpoint{4.445077in}{0.943149in}}%
\pgfpathlineto{\pgfqpoint{4.445749in}{0.943149in}}%
\pgfpathlineto{\pgfqpoint{4.447093in}{0.903396in}}%
\pgfpathlineto{\pgfqpoint{4.447765in}{0.903396in}}%
\pgfpathlineto{\pgfqpoint{4.448438in}{0.943149in}}%
\pgfpathlineto{\pgfqpoint{4.449110in}{0.903396in}}%
\pgfpathlineto{\pgfqpoint{4.449782in}{0.903396in}}%
\pgfpathlineto{\pgfqpoint{4.449782in}{0.913997in}}%
\pgfpathlineto{\pgfqpoint{4.450454in}{0.900746in}}%
\pgfpathlineto{\pgfqpoint{4.451126in}{0.908697in}}%
\pgfpathlineto{\pgfqpoint{4.451798in}{0.908697in}}%
\pgfpathlineto{\pgfqpoint{4.451798in}{0.903396in}}%
\pgfpathlineto{\pgfqpoint{4.453143in}{0.943149in}}%
\pgfpathlineto{\pgfqpoint{4.453815in}{0.943149in}}%
\pgfpathlineto{\pgfqpoint{4.453815in}{0.903396in}}%
\pgfpathlineto{\pgfqpoint{4.455159in}{0.921947in}}%
\pgfpathlineto{\pgfqpoint{4.455831in}{0.921947in}}%
\pgfpathlineto{\pgfqpoint{4.455831in}{0.924598in}}%
\pgfpathlineto{\pgfqpoint{4.457176in}{0.884845in}}%
\pgfpathlineto{\pgfqpoint{4.457848in}{0.884845in}}%
\pgfpathlineto{\pgfqpoint{4.457848in}{0.903396in}}%
\pgfpathlineto{\pgfqpoint{4.459192in}{0.898096in}}%
\pgfpathlineto{\pgfqpoint{4.459864in}{0.898096in}}%
\pgfpathlineto{\pgfqpoint{4.459864in}{0.956400in}}%
\pgfpathlineto{\pgfqpoint{4.461209in}{0.847742in}}%
\pgfpathlineto{\pgfqpoint{4.461881in}{0.847742in}}%
\pgfpathlineto{\pgfqpoint{4.461881in}{0.935198in}}%
\pgfpathlineto{\pgfqpoint{4.463225in}{0.924598in}}%
\pgfpathlineto{\pgfqpoint{4.463897in}{0.924598in}}%
\pgfpathlineto{\pgfqpoint{4.463897in}{0.863644in}}%
\pgfpathlineto{\pgfqpoint{4.464570in}{0.927248in}}%
\pgfpathlineto{\pgfqpoint{4.465242in}{0.921947in}}%
\pgfpathlineto{\pgfqpoint{4.465914in}{0.921947in}}%
\pgfpathlineto{\pgfqpoint{4.466586in}{0.860993in}}%
\pgfpathlineto{\pgfqpoint{4.467258in}{0.967000in}}%
\pgfpathlineto{\pgfqpoint{4.467930in}{0.967000in}}%
\pgfpathlineto{\pgfqpoint{4.469275in}{0.839792in}}%
\pgfpathlineto{\pgfqpoint{4.469947in}{0.839792in}}%
\pgfpathlineto{\pgfqpoint{4.470619in}{0.903396in}}%
\pgfpathlineto{\pgfqpoint{4.471291in}{0.903396in}}%
\pgfpathlineto{\pgfqpoint{4.471963in}{0.903396in}}%
\pgfpathlineto{\pgfqpoint{4.471963in}{0.919297in}}%
\pgfpathlineto{\pgfqpoint{4.473308in}{0.858343in}}%
\pgfpathlineto{\pgfqpoint{4.473980in}{0.858343in}}%
\pgfpathlineto{\pgfqpoint{4.473980in}{0.876894in}}%
\pgfpathlineto{\pgfqpoint{4.475324in}{0.874244in}}%
\pgfpathlineto{\pgfqpoint{4.475996in}{0.874244in}}%
\pgfpathlineto{\pgfqpoint{4.475996in}{0.948449in}}%
\pgfpathlineto{\pgfqpoint{4.477341in}{0.853043in}}%
\pgfpathlineto{\pgfqpoint{4.478013in}{0.853043in}}%
\pgfpathlineto{\pgfqpoint{4.478013in}{0.916647in}}%
\pgfpathlineto{\pgfqpoint{4.479357in}{0.884845in}}%
\pgfpathlineto{\pgfqpoint{4.480029in}{0.884845in}}%
\pgfpathlineto{\pgfqpoint{4.480029in}{0.876894in}}%
\pgfpathlineto{\pgfqpoint{4.481374in}{0.921947in}}%
\pgfpathlineto{\pgfqpoint{4.482046in}{0.921947in}}%
\pgfpathlineto{\pgfqpoint{4.482046in}{0.882195in}}%
\pgfpathlineto{\pgfqpoint{4.482718in}{0.924598in}}%
\pgfpathlineto{\pgfqpoint{4.483390in}{0.924598in}}%
\pgfpathlineto{\pgfqpoint{4.484062in}{0.924598in}}%
\pgfpathlineto{\pgfqpoint{4.484735in}{0.895446in}}%
\pgfpathlineto{\pgfqpoint{4.485407in}{0.898096in}}%
\pgfpathlineto{\pgfqpoint{4.486079in}{0.898096in}}%
\pgfpathlineto{\pgfqpoint{4.486079in}{0.900746in}}%
\pgfpathlineto{\pgfqpoint{4.487423in}{0.876894in}}%
\pgfpathlineto{\pgfqpoint{4.488095in}{0.876894in}}%
\pgfpathlineto{\pgfqpoint{4.488768in}{0.924598in}}%
\pgfpathlineto{\pgfqpoint{4.489440in}{0.858343in}}%
\pgfpathlineto{\pgfqpoint{4.490112in}{0.858343in}}%
\pgfpathlineto{\pgfqpoint{4.490784in}{0.903396in}}%
\pgfpathlineto{\pgfqpoint{4.491456in}{0.868944in}}%
\pgfpathlineto{\pgfqpoint{4.492129in}{0.868944in}}%
\pgfpathlineto{\pgfqpoint{4.492801in}{0.921947in}}%
\pgfpathlineto{\pgfqpoint{4.493473in}{0.887495in}}%
\pgfpathlineto{\pgfqpoint{4.494145in}{0.887495in}}%
\pgfpathlineto{\pgfqpoint{4.495489in}{0.945799in}}%
\pgfpathlineto{\pgfqpoint{4.496162in}{0.945799in}}%
\pgfpathlineto{\pgfqpoint{4.496162in}{0.924598in}}%
\pgfpathlineto{\pgfqpoint{4.496834in}{0.951099in}}%
\pgfpathlineto{\pgfqpoint{4.497506in}{0.937848in}}%
\pgfpathlineto{\pgfqpoint{4.498178in}{0.937848in}}%
\pgfpathlineto{\pgfqpoint{4.498850in}{0.847742in}}%
\pgfpathlineto{\pgfqpoint{4.498850in}{0.943149in}}%
\pgfpathlineto{\pgfqpoint{4.499522in}{0.866294in}}%
\pgfpathlineto{\pgfqpoint{4.500195in}{0.866294in}}%
\pgfpathlineto{\pgfqpoint{4.500195in}{0.853043in}}%
\pgfpathlineto{\pgfqpoint{4.500867in}{0.895446in}}%
\pgfpathlineto{\pgfqpoint{4.501539in}{0.895446in}}%
\pgfpathlineto{\pgfqpoint{4.502211in}{0.895446in}}%
\pgfpathlineto{\pgfqpoint{4.502883in}{0.884845in}}%
\pgfpathlineto{\pgfqpoint{4.503555in}{0.919297in}}%
\pgfpathlineto{\pgfqpoint{4.504228in}{0.919297in}}%
\pgfpathlineto{\pgfqpoint{4.504900in}{0.868944in}}%
\pgfpathlineto{\pgfqpoint{4.505572in}{0.900746in}}%
\pgfpathlineto{\pgfqpoint{4.506244in}{0.900746in}}%
\pgfpathlineto{\pgfqpoint{4.506244in}{0.937848in}}%
\pgfpathlineto{\pgfqpoint{4.507588in}{0.839792in}}%
\pgfpathlineto{\pgfqpoint{4.508261in}{0.839792in}}%
\pgfpathlineto{\pgfqpoint{4.508933in}{0.903396in}}%
\pgfpathlineto{\pgfqpoint{4.509605in}{0.887495in}}%
\pgfpathlineto{\pgfqpoint{4.510277in}{0.887495in}}%
\pgfpathlineto{\pgfqpoint{4.511621in}{0.908697in}}%
\pgfpathlineto{\pgfqpoint{4.512294in}{0.908697in}}%
\pgfpathlineto{\pgfqpoint{4.512294in}{0.948449in}}%
\pgfpathlineto{\pgfqpoint{4.512966in}{0.900746in}}%
\pgfpathlineto{\pgfqpoint{4.513638in}{0.900746in}}%
\pgfpathlineto{\pgfqpoint{4.514310in}{0.900746in}}%
\pgfpathlineto{\pgfqpoint{4.514310in}{0.935198in}}%
\pgfpathlineto{\pgfqpoint{4.515654in}{0.887495in}}%
\pgfpathlineto{\pgfqpoint{4.516327in}{0.887495in}}%
\pgfpathlineto{\pgfqpoint{4.516327in}{0.945799in}}%
\pgfpathlineto{\pgfqpoint{4.517671in}{0.895446in}}%
\pgfpathlineto{\pgfqpoint{4.518343in}{0.895446in}}%
\pgfpathlineto{\pgfqpoint{4.519015in}{0.863644in}}%
\pgfpathlineto{\pgfqpoint{4.519687in}{0.890145in}}%
\pgfpathlineto{\pgfqpoint{4.520360in}{0.890145in}}%
\pgfpathlineto{\pgfqpoint{4.521032in}{0.866294in}}%
\pgfpathlineto{\pgfqpoint{4.521704in}{0.892795in}}%
\pgfpathlineto{\pgfqpoint{4.522376in}{0.892795in}}%
\pgfpathlineto{\pgfqpoint{4.522376in}{0.919297in}}%
\pgfpathlineto{\pgfqpoint{4.523720in}{0.879545in}}%
\pgfpathlineto{\pgfqpoint{4.524393in}{0.879545in}}%
\pgfpathlineto{\pgfqpoint{4.525737in}{0.924598in}}%
\pgfpathlineto{\pgfqpoint{4.526409in}{0.924598in}}%
\pgfpathlineto{\pgfqpoint{4.527753in}{0.866294in}}%
\pgfpathlineto{\pgfqpoint{4.528426in}{0.866294in}}%
\pgfpathlineto{\pgfqpoint{4.529770in}{0.937848in}}%
\pgfpathlineto{\pgfqpoint{4.530442in}{0.937848in}}%
\pgfpathlineto{\pgfqpoint{4.530442in}{0.876894in}}%
\pgfpathlineto{\pgfqpoint{4.531786in}{0.884845in}}%
\pgfpathlineto{\pgfqpoint{4.532459in}{0.884845in}}%
\pgfpathlineto{\pgfqpoint{4.532459in}{0.916647in}}%
\pgfpathlineto{\pgfqpoint{4.533803in}{0.876894in}}%
\pgfpathlineto{\pgfqpoint{4.534475in}{0.876894in}}%
\pgfpathlineto{\pgfqpoint{4.534475in}{0.890145in}}%
\pgfpathlineto{\pgfqpoint{4.535819in}{0.839792in}}%
\pgfpathlineto{\pgfqpoint{4.536492in}{0.839792in}}%
\pgfpathlineto{\pgfqpoint{4.537836in}{0.935198in}}%
\pgfpathlineto{\pgfqpoint{4.538508in}{0.935198in}}%
\pgfpathlineto{\pgfqpoint{4.538508in}{0.943149in}}%
\pgfpathlineto{\pgfqpoint{4.539852in}{0.874244in}}%
\pgfpathlineto{\pgfqpoint{4.540525in}{0.874244in}}%
\pgfpathlineto{\pgfqpoint{4.541197in}{0.853043in}}%
\pgfpathlineto{\pgfqpoint{4.541869in}{0.908697in}}%
\pgfpathlineto{\pgfqpoint{4.542541in}{0.908697in}}%
\pgfpathlineto{\pgfqpoint{4.542541in}{0.845092in}}%
\pgfpathlineto{\pgfqpoint{4.543885in}{0.919297in}}%
\pgfpathlineto{\pgfqpoint{4.544558in}{0.919297in}}%
\pgfpathlineto{\pgfqpoint{4.545902in}{0.831841in}}%
\pgfpathlineto{\pgfqpoint{4.546574in}{0.831841in}}%
\pgfpathlineto{\pgfqpoint{4.547246in}{0.887495in}}%
\pgfpathlineto{\pgfqpoint{4.547918in}{0.847742in}}%
\pgfpathlineto{\pgfqpoint{4.548591in}{0.847742in}}%
\pgfpathlineto{\pgfqpoint{4.549935in}{0.927248in}}%
\pgfpathlineto{\pgfqpoint{4.550607in}{0.927248in}}%
\pgfpathlineto{\pgfqpoint{4.550607in}{0.929898in}}%
\pgfpathlineto{\pgfqpoint{4.551279in}{0.884845in}}%
\pgfpathlineto{\pgfqpoint{4.551951in}{0.913997in}}%
\pgfpathlineto{\pgfqpoint{4.552624in}{0.913997in}}%
\pgfpathlineto{\pgfqpoint{4.552624in}{0.921947in}}%
\pgfpathlineto{\pgfqpoint{4.553296in}{0.895446in}}%
\pgfpathlineto{\pgfqpoint{4.553968in}{0.900746in}}%
\pgfpathlineto{\pgfqpoint{4.554640in}{0.900746in}}%
\pgfpathlineto{\pgfqpoint{4.554640in}{0.940499in}}%
\pgfpathlineto{\pgfqpoint{4.555984in}{0.911347in}}%
\pgfpathlineto{\pgfqpoint{4.556657in}{0.911347in}}%
\pgfpathlineto{\pgfqpoint{4.557329in}{0.863644in}}%
\pgfpathlineto{\pgfqpoint{4.558001in}{0.908697in}}%
\pgfpathlineto{\pgfqpoint{4.558673in}{0.908697in}}%
\pgfpathlineto{\pgfqpoint{4.559345in}{0.980251in}}%
\pgfpathlineto{\pgfqpoint{4.560017in}{0.961700in}}%
\pgfpathlineto{\pgfqpoint{4.560690in}{0.961700in}}%
\pgfpathlineto{\pgfqpoint{4.561362in}{0.890145in}}%
\pgfpathlineto{\pgfqpoint{4.562034in}{0.900746in}}%
\pgfpathlineto{\pgfqpoint{4.562706in}{0.900746in}}%
\pgfpathlineto{\pgfqpoint{4.562706in}{0.845092in}}%
\pgfpathlineto{\pgfqpoint{4.563378in}{0.916647in}}%
\pgfpathlineto{\pgfqpoint{4.564050in}{0.882195in}}%
\pgfpathlineto{\pgfqpoint{4.564723in}{0.882195in}}%
\pgfpathlineto{\pgfqpoint{4.565395in}{0.900746in}}%
\pgfpathlineto{\pgfqpoint{4.566067in}{0.850393in}}%
\pgfpathlineto{\pgfqpoint{4.566739in}{0.850393in}}%
\pgfpathlineto{\pgfqpoint{4.567411in}{0.911347in}}%
\pgfpathlineto{\pgfqpoint{4.568083in}{0.874244in}}%
\pgfpathlineto{\pgfqpoint{4.568756in}{0.874244in}}%
\pgfpathlineto{\pgfqpoint{4.568756in}{0.858343in}}%
\pgfpathlineto{\pgfqpoint{4.569428in}{0.887495in}}%
\pgfpathlineto{\pgfqpoint{4.570100in}{0.879545in}}%
\pgfpathlineto{\pgfqpoint{4.570772in}{0.879545in}}%
\pgfpathlineto{\pgfqpoint{4.572116in}{0.974951in}}%
\pgfpathlineto{\pgfqpoint{4.572789in}{0.974951in}}%
\pgfpathlineto{\pgfqpoint{4.573461in}{0.906046in}}%
\pgfpathlineto{\pgfqpoint{4.574133in}{0.921947in}}%
\pgfpathlineto{\pgfqpoint{4.574805in}{0.921947in}}%
\pgfpathlineto{\pgfqpoint{4.574805in}{0.882195in}}%
\pgfpathlineto{\pgfqpoint{4.575477in}{0.943149in}}%
\pgfpathlineto{\pgfqpoint{4.576149in}{0.906046in}}%
\pgfpathlineto{\pgfqpoint{4.576822in}{0.906046in}}%
\pgfpathlineto{\pgfqpoint{4.578166in}{0.948449in}}%
\pgfpathlineto{\pgfqpoint{4.578838in}{0.948449in}}%
\pgfpathlineto{\pgfqpoint{4.580182in}{0.868944in}}%
\pgfpathlineto{\pgfqpoint{4.580855in}{0.868944in}}%
\pgfpathlineto{\pgfqpoint{4.582199in}{0.967000in}}%
\pgfpathlineto{\pgfqpoint{4.582871in}{0.967000in}}%
\pgfpathlineto{\pgfqpoint{4.583543in}{0.919297in}}%
\pgfpathlineto{\pgfqpoint{4.584215in}{0.953750in}}%
\pgfpathlineto{\pgfqpoint{4.584888in}{0.953750in}}%
\pgfpathlineto{\pgfqpoint{4.584888in}{0.924598in}}%
\pgfpathlineto{\pgfqpoint{4.585560in}{0.964350in}}%
\pgfpathlineto{\pgfqpoint{4.586232in}{0.940499in}}%
\pgfpathlineto{\pgfqpoint{4.586904in}{0.940499in}}%
\pgfpathlineto{\pgfqpoint{4.587576in}{0.887495in}}%
\pgfpathlineto{\pgfqpoint{4.588248in}{0.903396in}}%
\pgfpathlineto{\pgfqpoint{4.588921in}{0.903396in}}%
\pgfpathlineto{\pgfqpoint{4.590265in}{0.969651in}}%
\pgfpathlineto{\pgfqpoint{4.590937in}{0.969651in}}%
\pgfpathlineto{\pgfqpoint{4.591609in}{0.906046in}}%
\pgfpathlineto{\pgfqpoint{4.592281in}{0.906046in}}%
\pgfpathlineto{\pgfqpoint{4.592954in}{0.906046in}}%
\pgfpathlineto{\pgfqpoint{4.592954in}{0.903396in}}%
\pgfpathlineto{\pgfqpoint{4.593626in}{0.967000in}}%
\pgfpathlineto{\pgfqpoint{4.594298in}{0.927248in}}%
\pgfpathlineto{\pgfqpoint{4.594970in}{0.927248in}}%
\pgfpathlineto{\pgfqpoint{4.594970in}{0.921947in}}%
\pgfpathlineto{\pgfqpoint{4.596314in}{0.935198in}}%
\pgfpathlineto{\pgfqpoint{4.596987in}{0.935198in}}%
\pgfpathlineto{\pgfqpoint{4.598331in}{0.985552in}}%
\pgfpathlineto{\pgfqpoint{4.599003in}{0.985552in}}%
\pgfpathlineto{\pgfqpoint{4.599003in}{0.929898in}}%
\pgfpathlineto{\pgfqpoint{4.600347in}{0.945799in}}%
\pgfpathlineto{\pgfqpoint{4.601020in}{0.945799in}}%
\pgfpathlineto{\pgfqpoint{4.601020in}{0.876894in}}%
\pgfpathlineto{\pgfqpoint{4.601692in}{0.972301in}}%
\pgfpathlineto{\pgfqpoint{4.602364in}{0.911347in}}%
\pgfpathlineto{\pgfqpoint{4.603036in}{0.911347in}}%
\pgfpathlineto{\pgfqpoint{4.603036in}{0.961700in}}%
\pgfpathlineto{\pgfqpoint{4.604381in}{0.921947in}}%
\pgfpathlineto{\pgfqpoint{4.605053in}{0.921947in}}%
\pgfpathlineto{\pgfqpoint{4.605053in}{0.972301in}}%
\pgfpathlineto{\pgfqpoint{4.605725in}{0.906046in}}%
\pgfpathlineto{\pgfqpoint{4.606397in}{0.948449in}}%
\pgfpathlineto{\pgfqpoint{4.607069in}{0.948449in}}%
\pgfpathlineto{\pgfqpoint{4.607069in}{1.012053in}}%
\pgfpathlineto{\pgfqpoint{4.608414in}{0.943149in}}%
\pgfpathlineto{\pgfqpoint{4.609086in}{0.943149in}}%
\pgfpathlineto{\pgfqpoint{4.610430in}{0.980251in}}%
\pgfpathlineto{\pgfqpoint{4.611774in}{0.980251in}}%
\pgfpathlineto{\pgfqpoint{4.612447in}{0.927248in}}%
\pgfpathlineto{\pgfqpoint{4.613119in}{1.012053in}}%
\pgfpathlineto{\pgfqpoint{4.613791in}{1.012053in}}%
\pgfpathlineto{\pgfqpoint{4.614463in}{0.964350in}}%
\pgfpathlineto{\pgfqpoint{4.615135in}{0.980251in}}%
\pgfpathlineto{\pgfqpoint{4.615807in}{0.980251in}}%
\pgfpathlineto{\pgfqpoint{4.615807in}{1.020004in}}%
\pgfpathlineto{\pgfqpoint{4.617152in}{0.940499in}}%
\pgfpathlineto{\pgfqpoint{4.617824in}{0.940499in}}%
\pgfpathlineto{\pgfqpoint{4.618496in}{1.067707in}}%
\pgfpathlineto{\pgfqpoint{4.619168in}{0.964350in}}%
\pgfpathlineto{\pgfqpoint{4.619840in}{0.964350in}}%
\pgfpathlineto{\pgfqpoint{4.619840in}{0.951099in}}%
\pgfpathlineto{\pgfqpoint{4.621185in}{0.990852in}}%
\pgfpathlineto{\pgfqpoint{4.621857in}{0.990852in}}%
\pgfpathlineto{\pgfqpoint{4.621857in}{0.927248in}}%
\pgfpathlineto{\pgfqpoint{4.623201in}{1.014704in}}%
\pgfpathlineto{\pgfqpoint{4.623873in}{1.014704in}}%
\pgfpathlineto{\pgfqpoint{4.623873in}{1.065057in}}%
\pgfpathlineto{\pgfqpoint{4.625218in}{0.935198in}}%
\pgfpathlineto{\pgfqpoint{4.625890in}{0.935198in}}%
\pgfpathlineto{\pgfqpoint{4.626562in}{0.977601in}}%
\pgfpathlineto{\pgfqpoint{4.627234in}{0.956400in}}%
\pgfpathlineto{\pgfqpoint{4.627906in}{0.956400in}}%
\pgfpathlineto{\pgfqpoint{4.627906in}{1.009403in}}%
\pgfpathlineto{\pgfqpoint{4.629251in}{1.004103in}}%
\pgfpathlineto{\pgfqpoint{4.629923in}{1.004103in}}%
\pgfpathlineto{\pgfqpoint{4.631267in}{0.919297in}}%
\pgfpathlineto{\pgfqpoint{4.631939in}{0.919297in}}%
\pgfpathlineto{\pgfqpoint{4.633284in}{1.006753in}}%
\pgfpathlineto{\pgfqpoint{4.633956in}{1.006753in}}%
\pgfpathlineto{\pgfqpoint{4.635300in}{0.977601in}}%
\pgfpathlineto{\pgfqpoint{4.635972in}{0.977601in}}%
\pgfpathlineto{\pgfqpoint{4.636645in}{0.959050in}}%
\pgfpathlineto{\pgfqpoint{4.637317in}{1.006753in}}%
\pgfpathlineto{\pgfqpoint{4.637989in}{1.006753in}}%
\pgfpathlineto{\pgfqpoint{4.637989in}{0.932548in}}%
\pgfpathlineto{\pgfqpoint{4.639333in}{1.030605in}}%
\pgfpathlineto{\pgfqpoint{4.640005in}{1.030605in}}%
\pgfpathlineto{\pgfqpoint{4.640678in}{0.967000in}}%
\pgfpathlineto{\pgfqpoint{4.641350in}{1.033255in}}%
\pgfpathlineto{\pgfqpoint{4.642022in}{1.033255in}}%
\pgfpathlineto{\pgfqpoint{4.643366in}{0.996152in}}%
\pgfpathlineto{\pgfqpoint{4.644038in}{0.996152in}}%
\pgfpathlineto{\pgfqpoint{4.644038in}{0.974951in}}%
\pgfpathlineto{\pgfqpoint{4.644711in}{1.043856in}}%
\pgfpathlineto{\pgfqpoint{4.645383in}{1.033255in}}%
\pgfpathlineto{\pgfqpoint{4.646055in}{1.033255in}}%
\pgfpathlineto{\pgfqpoint{4.646055in}{1.054456in}}%
\pgfpathlineto{\pgfqpoint{4.647399in}{0.969651in}}%
\pgfpathlineto{\pgfqpoint{4.648744in}{0.969651in}}%
\pgfpathlineto{\pgfqpoint{4.649416in}{1.046506in}}%
\pgfpathlineto{\pgfqpoint{4.650088in}{1.043856in}}%
\pgfpathlineto{\pgfqpoint{4.650760in}{1.043856in}}%
\pgfpathlineto{\pgfqpoint{4.650760in}{0.996152in}}%
\pgfpathlineto{\pgfqpoint{4.652104in}{1.038555in}}%
\pgfpathlineto{\pgfqpoint{4.652777in}{1.038555in}}%
\pgfpathlineto{\pgfqpoint{4.652777in}{1.001453in}}%
\pgfpathlineto{\pgfqpoint{4.654121in}{1.059757in}}%
\pgfpathlineto{\pgfqpoint{4.654793in}{1.059757in}}%
\pgfpathlineto{\pgfqpoint{4.655465in}{0.996152in}}%
\pgfpathlineto{\pgfqpoint{4.656137in}{1.059757in}}%
\pgfpathlineto{\pgfqpoint{4.657482in}{1.059757in}}%
\pgfpathlineto{\pgfqpoint{4.657482in}{1.080958in}}%
\pgfpathlineto{\pgfqpoint{4.658826in}{1.025304in}}%
\pgfpathlineto{\pgfqpoint{4.659498in}{1.025304in}}%
\pgfpathlineto{\pgfqpoint{4.660170in}{0.967000in}}%
\pgfpathlineto{\pgfqpoint{4.660843in}{1.017354in}}%
\pgfpathlineto{\pgfqpoint{4.661515in}{1.017354in}}%
\pgfpathlineto{\pgfqpoint{4.661515in}{0.985552in}}%
\pgfpathlineto{\pgfqpoint{4.662187in}{1.067707in}}%
\pgfpathlineto{\pgfqpoint{4.662859in}{1.020004in}}%
\pgfpathlineto{\pgfqpoint{4.663531in}{1.020004in}}%
\pgfpathlineto{\pgfqpoint{4.663531in}{1.080958in}}%
\pgfpathlineto{\pgfqpoint{4.664203in}{1.012053in}}%
\pgfpathlineto{\pgfqpoint{4.664876in}{1.062407in}}%
\pgfpathlineto{\pgfqpoint{4.665548in}{1.062407in}}%
\pgfpathlineto{\pgfqpoint{4.666220in}{1.025304in}}%
\pgfpathlineto{\pgfqpoint{4.666892in}{1.049156in}}%
\pgfpathlineto{\pgfqpoint{4.667564in}{1.049156in}}%
\pgfpathlineto{\pgfqpoint{4.667564in}{1.059757in}}%
\pgfpathlineto{\pgfqpoint{4.668909in}{1.033255in}}%
\pgfpathlineto{\pgfqpoint{4.669581in}{1.033255in}}%
\pgfpathlineto{\pgfqpoint{4.670253in}{1.030605in}}%
\pgfpathlineto{\pgfqpoint{4.670925in}{1.080958in}}%
\pgfpathlineto{\pgfqpoint{4.671597in}{1.080958in}}%
\pgfpathlineto{\pgfqpoint{4.672942in}{1.014704in}}%
\pgfpathlineto{\pgfqpoint{4.673614in}{1.014704in}}%
\pgfpathlineto{\pgfqpoint{4.674958in}{1.102159in}}%
\pgfpathlineto{\pgfqpoint{4.675630in}{1.102159in}}%
\pgfpathlineto{\pgfqpoint{4.676302in}{1.107460in}}%
\pgfpathlineto{\pgfqpoint{4.676975in}{1.078308in}}%
\pgfpathlineto{\pgfqpoint{4.677647in}{1.078308in}}%
\pgfpathlineto{\pgfqpoint{4.678991in}{1.022654in}}%
\pgfpathlineto{\pgfqpoint{4.679663in}{1.022654in}}%
\pgfpathlineto{\pgfqpoint{4.680335in}{1.083608in}}%
\pgfpathlineto{\pgfqpoint{4.681008in}{1.075658in}}%
\pgfpathlineto{\pgfqpoint{4.682352in}{1.075658in}}%
\pgfpathlineto{\pgfqpoint{4.683024in}{1.128661in}}%
\pgfpathlineto{\pgfqpoint{4.683696in}{1.083608in}}%
\pgfpathlineto{\pgfqpoint{4.684368in}{1.083608in}}%
\pgfpathlineto{\pgfqpoint{4.685041in}{1.115410in}}%
\pgfpathlineto{\pgfqpoint{4.685713in}{1.070357in}}%
\pgfpathlineto{\pgfqpoint{4.686385in}{1.070357in}}%
\pgfpathlineto{\pgfqpoint{4.686385in}{1.054456in}}%
\pgfpathlineto{\pgfqpoint{4.687729in}{1.067707in}}%
\pgfpathlineto{\pgfqpoint{4.688401in}{1.067707in}}%
\pgfpathlineto{\pgfqpoint{4.689074in}{1.123361in}}%
\pgfpathlineto{\pgfqpoint{4.689746in}{1.051806in}}%
\pgfpathlineto{\pgfqpoint{4.690418in}{1.051806in}}%
\pgfpathlineto{\pgfqpoint{4.691090in}{1.096859in}}%
\pgfpathlineto{\pgfqpoint{4.691762in}{1.091559in}}%
\pgfpathlineto{\pgfqpoint{4.692434in}{1.091559in}}%
\pgfpathlineto{\pgfqpoint{4.692434in}{1.070357in}}%
\pgfpathlineto{\pgfqpoint{4.693779in}{1.094209in}}%
\pgfpathlineto{\pgfqpoint{4.694451in}{1.094209in}}%
\pgfpathlineto{\pgfqpoint{4.694451in}{1.112760in}}%
\pgfpathlineto{\pgfqpoint{4.695123in}{1.067707in}}%
\pgfpathlineto{\pgfqpoint{4.695795in}{1.102159in}}%
\pgfpathlineto{\pgfqpoint{4.696467in}{1.102159in}}%
\pgfpathlineto{\pgfqpoint{4.696467in}{1.022654in}}%
\pgfpathlineto{\pgfqpoint{4.697812in}{1.184315in}}%
\pgfpathlineto{\pgfqpoint{4.698484in}{1.184315in}}%
\pgfpathlineto{\pgfqpoint{4.699156in}{1.059757in}}%
\pgfpathlineto{\pgfqpoint{4.699828in}{1.123361in}}%
\pgfpathlineto{\pgfqpoint{4.700500in}{1.123361in}}%
\pgfpathlineto{\pgfqpoint{4.700500in}{1.057106in}}%
\pgfpathlineto{\pgfqpoint{4.701173in}{1.136612in}}%
\pgfpathlineto{\pgfqpoint{4.701845in}{1.094209in}}%
\pgfpathlineto{\pgfqpoint{4.702517in}{1.094209in}}%
\pgfpathlineto{\pgfqpoint{4.702517in}{1.091559in}}%
\pgfpathlineto{\pgfqpoint{4.703861in}{1.139262in}}%
\pgfpathlineto{\pgfqpoint{4.704533in}{1.139262in}}%
\pgfpathlineto{\pgfqpoint{4.705206in}{1.073007in}}%
\pgfpathlineto{\pgfqpoint{4.705878in}{1.152513in}}%
\pgfpathlineto{\pgfqpoint{4.706550in}{1.152513in}}%
\pgfpathlineto{\pgfqpoint{4.707894in}{1.123361in}}%
\pgfpathlineto{\pgfqpoint{4.708566in}{1.123361in}}%
\pgfpathlineto{\pgfqpoint{4.709239in}{1.094209in}}%
\pgfpathlineto{\pgfqpoint{4.709911in}{1.139262in}}%
\pgfpathlineto{\pgfqpoint{4.710583in}{1.139262in}}%
\pgfpathlineto{\pgfqpoint{4.711255in}{1.070357in}}%
\pgfpathlineto{\pgfqpoint{4.711927in}{1.102159in}}%
\pgfpathlineto{\pgfqpoint{4.712599in}{1.102159in}}%
\pgfpathlineto{\pgfqpoint{4.713272in}{1.149863in}}%
\pgfpathlineto{\pgfqpoint{4.713944in}{1.131311in}}%
\pgfpathlineto{\pgfqpoint{4.714616in}{1.131311in}}%
\pgfpathlineto{\pgfqpoint{4.715288in}{1.189615in}}%
\pgfpathlineto{\pgfqpoint{4.715960in}{1.096859in}}%
\pgfpathlineto{\pgfqpoint{4.716633in}{1.096859in}}%
\pgfpathlineto{\pgfqpoint{4.717977in}{1.181665in}}%
\pgfpathlineto{\pgfqpoint{4.718649in}{1.181665in}}%
\pgfpathlineto{\pgfqpoint{4.718649in}{1.128661in}}%
\pgfpathlineto{\pgfqpoint{4.719993in}{1.131311in}}%
\pgfpathlineto{\pgfqpoint{4.720666in}{1.131311in}}%
\pgfpathlineto{\pgfqpoint{4.720666in}{1.118060in}}%
\pgfpathlineto{\pgfqpoint{4.722010in}{1.149863in}}%
\pgfpathlineto{\pgfqpoint{4.722682in}{1.149863in}}%
\pgfpathlineto{\pgfqpoint{4.723354in}{1.176364in}}%
\pgfpathlineto{\pgfqpoint{4.724026in}{1.139262in}}%
\pgfpathlineto{\pgfqpoint{4.724699in}{1.139262in}}%
\pgfpathlineto{\pgfqpoint{4.724699in}{1.155163in}}%
\pgfpathlineto{\pgfqpoint{4.725371in}{1.104810in}}%
\pgfpathlineto{\pgfqpoint{4.726043in}{1.136612in}}%
\pgfpathlineto{\pgfqpoint{4.726715in}{1.136612in}}%
\pgfpathlineto{\pgfqpoint{4.727387in}{1.186965in}}%
\pgfpathlineto{\pgfqpoint{4.728059in}{1.128661in}}%
\pgfpathlineto{\pgfqpoint{4.728732in}{1.128661in}}%
\pgfpathlineto{\pgfqpoint{4.728732in}{1.099509in}}%
\pgfpathlineto{\pgfqpoint{4.730076in}{1.239969in}}%
\pgfpathlineto{\pgfqpoint{4.730748in}{1.239969in}}%
\pgfpathlineto{\pgfqpoint{4.732092in}{1.123361in}}%
\pgfpathlineto{\pgfqpoint{4.732765in}{1.123361in}}%
\pgfpathlineto{\pgfqpoint{4.733437in}{1.186965in}}%
\pgfpathlineto{\pgfqpoint{4.734109in}{1.110110in}}%
\pgfpathlineto{\pgfqpoint{4.734781in}{1.110110in}}%
\pgfpathlineto{\pgfqpoint{4.735453in}{1.194916in}}%
\pgfpathlineto{\pgfqpoint{4.736125in}{1.194916in}}%
\pgfpathlineto{\pgfqpoint{4.736798in}{1.194916in}}%
\pgfpathlineto{\pgfqpoint{4.737470in}{1.102159in}}%
\pgfpathlineto{\pgfqpoint{4.738142in}{1.186965in}}%
\pgfpathlineto{\pgfqpoint{4.738814in}{1.186965in}}%
\pgfpathlineto{\pgfqpoint{4.739486in}{1.210817in}}%
\pgfpathlineto{\pgfqpoint{4.740158in}{1.149863in}}%
\pgfpathlineto{\pgfqpoint{4.740831in}{1.149863in}}%
\pgfpathlineto{\pgfqpoint{4.740831in}{1.179015in}}%
\pgfpathlineto{\pgfqpoint{4.742175in}{1.160463in}}%
\pgfpathlineto{\pgfqpoint{4.742847in}{1.160463in}}%
\pgfpathlineto{\pgfqpoint{4.742847in}{1.152513in}}%
\pgfpathlineto{\pgfqpoint{4.744191in}{1.261170in}}%
\pgfpathlineto{\pgfqpoint{4.744864in}{1.261170in}}%
\pgfpathlineto{\pgfqpoint{4.746208in}{1.197566in}}%
\pgfpathlineto{\pgfqpoint{4.747552in}{1.197566in}}%
\pgfpathlineto{\pgfqpoint{4.747552in}{1.186965in}}%
\pgfpathlineto{\pgfqpoint{4.748224in}{1.239969in}}%
\pgfpathlineto{\pgfqpoint{4.748897in}{1.186965in}}%
\pgfpathlineto{\pgfqpoint{4.749569in}{1.186965in}}%
\pgfpathlineto{\pgfqpoint{4.749569in}{1.149863in}}%
\pgfpathlineto{\pgfqpoint{4.750913in}{1.226718in}}%
\pgfpathlineto{\pgfqpoint{4.751585in}{1.226718in}}%
\pgfpathlineto{\pgfqpoint{4.751585in}{1.253219in}}%
\pgfpathlineto{\pgfqpoint{4.752257in}{1.224068in}}%
\pgfpathlineto{\pgfqpoint{4.752930in}{1.226718in}}%
\pgfpathlineto{\pgfqpoint{4.753602in}{1.226718in}}%
\pgfpathlineto{\pgfqpoint{4.753602in}{1.253219in}}%
\pgfpathlineto{\pgfqpoint{4.754274in}{1.171064in}}%
\pgfpathlineto{\pgfqpoint{4.754946in}{1.186965in}}%
\pgfpathlineto{\pgfqpoint{4.755618in}{1.186965in}}%
\pgfpathlineto{\pgfqpoint{4.756290in}{1.295622in}}%
\pgfpathlineto{\pgfqpoint{4.756963in}{1.200216in}}%
\pgfpathlineto{\pgfqpoint{4.757635in}{1.200216in}}%
\pgfpathlineto{\pgfqpoint{4.757635in}{1.285022in}}%
\pgfpathlineto{\pgfqpoint{4.758979in}{1.261170in}}%
\pgfpathlineto{\pgfqpoint{4.759651in}{1.261170in}}%
\pgfpathlineto{\pgfqpoint{4.760323in}{1.192265in}}%
\pgfpathlineto{\pgfqpoint{4.760996in}{1.279721in}}%
\pgfpathlineto{\pgfqpoint{4.761668in}{1.279721in}}%
\pgfpathlineto{\pgfqpoint{4.761668in}{1.332725in}}%
\pgfpathlineto{\pgfqpoint{4.763012in}{1.221417in}}%
\pgfpathlineto{\pgfqpoint{4.763684in}{1.221417in}}%
\pgfpathlineto{\pgfqpoint{4.763684in}{1.279721in}}%
\pgfpathlineto{\pgfqpoint{4.764356in}{1.205516in}}%
\pgfpathlineto{\pgfqpoint{4.765029in}{1.213467in}}%
\pgfpathlineto{\pgfqpoint{4.765701in}{1.213467in}}%
\pgfpathlineto{\pgfqpoint{4.765701in}{1.245269in}}%
\pgfpathlineto{\pgfqpoint{4.767045in}{1.208166in}}%
\pgfpathlineto{\pgfqpoint{4.767717in}{1.208166in}}%
\pgfpathlineto{\pgfqpoint{4.768389in}{1.165764in}}%
\pgfpathlineto{\pgfqpoint{4.769062in}{1.239969in}}%
\pgfpathlineto{\pgfqpoint{4.769734in}{1.239969in}}%
\pgfpathlineto{\pgfqpoint{4.770406in}{1.277071in}}%
\pgfpathlineto{\pgfqpoint{4.771078in}{1.224068in}}%
\pgfpathlineto{\pgfqpoint{4.771750in}{1.224068in}}%
\pgfpathlineto{\pgfqpoint{4.771750in}{1.205516in}}%
\pgfpathlineto{\pgfqpoint{4.773095in}{1.285022in}}%
\pgfpathlineto{\pgfqpoint{4.773767in}{1.285022in}}%
\pgfpathlineto{\pgfqpoint{4.773767in}{1.226718in}}%
\pgfpathlineto{\pgfqpoint{4.775111in}{1.245269in}}%
\pgfpathlineto{\pgfqpoint{4.775783in}{1.245269in}}%
\pgfpathlineto{\pgfqpoint{4.776455in}{1.184315in}}%
\pgfpathlineto{\pgfqpoint{4.777128in}{1.274421in}}%
\pgfpathlineto{\pgfqpoint{4.777800in}{1.274421in}}%
\pgfpathlineto{\pgfqpoint{4.778472in}{1.194916in}}%
\pgfpathlineto{\pgfqpoint{4.779144in}{1.263820in}}%
\pgfpathlineto{\pgfqpoint{4.779816in}{1.263820in}}%
\pgfpathlineto{\pgfqpoint{4.779816in}{1.202866in}}%
\pgfpathlineto{\pgfqpoint{4.780488in}{1.269121in}}%
\pgfpathlineto{\pgfqpoint{4.781161in}{1.234668in}}%
\pgfpathlineto{\pgfqpoint{4.781833in}{1.234668in}}%
\pgfpathlineto{\pgfqpoint{4.781833in}{1.263820in}}%
\pgfpathlineto{\pgfqpoint{4.782505in}{1.218767in}}%
\pgfpathlineto{\pgfqpoint{4.783177in}{1.253219in}}%
\pgfpathlineto{\pgfqpoint{4.783849in}{1.253219in}}%
\pgfpathlineto{\pgfqpoint{4.783849in}{1.330075in}}%
\pgfpathlineto{\pgfqpoint{4.785194in}{1.295622in}}%
\pgfpathlineto{\pgfqpoint{4.785866in}{1.295622in}}%
\pgfpathlineto{\pgfqpoint{4.785866in}{1.250569in}}%
\pgfpathlineto{\pgfqpoint{4.786538in}{1.396329in}}%
\pgfpathlineto{\pgfqpoint{4.787210in}{1.375128in}}%
\pgfpathlineto{\pgfqpoint{4.787882in}{1.375128in}}%
\pgfpathlineto{\pgfqpoint{4.788554in}{1.303573in}}%
\pgfpathlineto{\pgfqpoint{4.789227in}{1.314174in}}%
\pgfpathlineto{\pgfqpoint{4.789899in}{1.314174in}}%
\pgfpathlineto{\pgfqpoint{4.790571in}{1.253219in}}%
\pgfpathlineto{\pgfqpoint{4.791243in}{1.282371in}}%
\pgfpathlineto{\pgfqpoint{4.791915in}{1.282371in}}%
\pgfpathlineto{\pgfqpoint{4.792587in}{1.237318in}}%
\pgfpathlineto{\pgfqpoint{4.793260in}{1.356576in}}%
\pgfpathlineto{\pgfqpoint{4.793932in}{1.356576in}}%
\pgfpathlineto{\pgfqpoint{4.793932in}{1.290322in}}%
\pgfpathlineto{\pgfqpoint{4.794604in}{1.359227in}}%
\pgfpathlineto{\pgfqpoint{4.795276in}{1.327424in}}%
\pgfpathlineto{\pgfqpoint{4.795948in}{1.327424in}}%
\pgfpathlineto{\pgfqpoint{4.795948in}{1.245269in}}%
\pgfpathlineto{\pgfqpoint{4.796620in}{1.359227in}}%
\pgfpathlineto{\pgfqpoint{4.797293in}{1.269121in}}%
\pgfpathlineto{\pgfqpoint{4.797965in}{1.269121in}}%
\pgfpathlineto{\pgfqpoint{4.798637in}{1.356576in}}%
\pgfpathlineto{\pgfqpoint{4.799309in}{1.338025in}}%
\pgfpathlineto{\pgfqpoint{4.799981in}{1.338025in}}%
\pgfpathlineto{\pgfqpoint{4.799981in}{1.345976in}}%
\pgfpathlineto{\pgfqpoint{4.800653in}{1.247919in}}%
\pgfpathlineto{\pgfqpoint{4.801326in}{1.282371in}}%
\pgfpathlineto{\pgfqpoint{4.801998in}{1.282371in}}%
\pgfpathlineto{\pgfqpoint{4.801998in}{1.277071in}}%
\pgfpathlineto{\pgfqpoint{4.802670in}{1.353926in}}%
\pgfpathlineto{\pgfqpoint{4.803342in}{1.308873in}}%
\pgfpathlineto{\pgfqpoint{4.804014in}{1.308873in}}%
\pgfpathlineto{\pgfqpoint{4.804014in}{1.361877in}}%
\pgfpathlineto{\pgfqpoint{4.805359in}{1.229368in}}%
\pgfpathlineto{\pgfqpoint{4.806031in}{1.229368in}}%
\pgfpathlineto{\pgfqpoint{4.806031in}{1.340675in}}%
\pgfpathlineto{\pgfqpoint{4.807375in}{1.324774in}}%
\pgfpathlineto{\pgfqpoint{4.808047in}{1.324774in}}%
\pgfpathlineto{\pgfqpoint{4.808719in}{1.345976in}}%
\pgfpathlineto{\pgfqpoint{4.809392in}{1.266470in}}%
\pgfpathlineto{\pgfqpoint{4.810064in}{1.266470in}}%
\pgfpathlineto{\pgfqpoint{4.811408in}{1.340675in}}%
\pgfpathlineto{\pgfqpoint{4.812080in}{1.340675in}}%
\pgfpathlineto{\pgfqpoint{4.812080in}{1.330075in}}%
\pgfpathlineto{\pgfqpoint{4.813425in}{1.428131in}}%
\pgfpathlineto{\pgfqpoint{4.814097in}{1.428131in}}%
\pgfpathlineto{\pgfqpoint{4.814097in}{1.311523in}}%
\pgfpathlineto{\pgfqpoint{4.815441in}{1.338025in}}%
\pgfpathlineto{\pgfqpoint{4.816113in}{1.338025in}}%
\pgfpathlineto{\pgfqpoint{4.816113in}{1.285022in}}%
\pgfpathlineto{\pgfqpoint{4.816785in}{1.356576in}}%
\pgfpathlineto{\pgfqpoint{4.817458in}{1.314174in}}%
\pgfpathlineto{\pgfqpoint{4.818130in}{1.314174in}}%
\pgfpathlineto{\pgfqpoint{4.818802in}{1.351276in}}%
\pgfpathlineto{\pgfqpoint{4.819474in}{1.279721in}}%
\pgfpathlineto{\pgfqpoint{4.820146in}{1.279721in}}%
\pgfpathlineto{\pgfqpoint{4.821491in}{1.385728in}}%
\pgfpathlineto{\pgfqpoint{4.822163in}{1.385728in}}%
\pgfpathlineto{\pgfqpoint{4.822163in}{1.372477in}}%
\pgfpathlineto{\pgfqpoint{4.823507in}{1.444032in}}%
\pgfpathlineto{\pgfqpoint{4.824179in}{1.444032in}}%
\pgfpathlineto{\pgfqpoint{4.824851in}{1.306223in}}%
\pgfpathlineto{\pgfqpoint{4.825524in}{1.377778in}}%
\pgfpathlineto{\pgfqpoint{4.826196in}{1.377778in}}%
\pgfpathlineto{\pgfqpoint{4.826868in}{1.330075in}}%
\pgfpathlineto{\pgfqpoint{4.827540in}{1.422831in}}%
\pgfpathlineto{\pgfqpoint{4.828212in}{1.422831in}}%
\pgfpathlineto{\pgfqpoint{4.828212in}{1.364527in}}%
\pgfpathlineto{\pgfqpoint{4.829557in}{1.414880in}}%
\pgfpathlineto{\pgfqpoint{4.830229in}{1.414880in}}%
\pgfpathlineto{\pgfqpoint{4.830229in}{1.332725in}}%
\pgfpathlineto{\pgfqpoint{4.831573in}{1.332725in}}%
\pgfpathlineto{\pgfqpoint{4.832245in}{1.332725in}}%
\pgfpathlineto{\pgfqpoint{4.832918in}{1.412230in}}%
\pgfpathlineto{\pgfqpoint{4.833590in}{1.401629in}}%
\pgfpathlineto{\pgfqpoint{4.834262in}{1.401629in}}%
\pgfpathlineto{\pgfqpoint{4.834934in}{1.282371in}}%
\pgfpathlineto{\pgfqpoint{4.835606in}{1.345976in}}%
\pgfpathlineto{\pgfqpoint{4.836278in}{1.345976in}}%
\pgfpathlineto{\pgfqpoint{4.836951in}{1.298272in}}%
\pgfpathlineto{\pgfqpoint{4.836951in}{1.364527in}}%
\pgfpathlineto{\pgfqpoint{4.837623in}{1.343325in}}%
\pgfpathlineto{\pgfqpoint{4.838295in}{1.343325in}}%
\pgfpathlineto{\pgfqpoint{4.838295in}{1.504986in}}%
\pgfpathlineto{\pgfqpoint{4.839639in}{1.398979in}}%
\pgfpathlineto{\pgfqpoint{4.840311in}{1.398979in}}%
\pgfpathlineto{\pgfqpoint{4.840311in}{1.417530in}}%
\pgfpathlineto{\pgfqpoint{4.841656in}{1.327424in}}%
\pgfpathlineto{\pgfqpoint{4.842328in}{1.327424in}}%
\pgfpathlineto{\pgfqpoint{4.842328in}{1.401629in}}%
\pgfpathlineto{\pgfqpoint{4.843000in}{1.316824in}}%
\pgfpathlineto{\pgfqpoint{4.843672in}{1.330075in}}%
\pgfpathlineto{\pgfqpoint{4.844344in}{1.330075in}}%
\pgfpathlineto{\pgfqpoint{4.844344in}{1.412230in}}%
\pgfpathlineto{\pgfqpoint{4.845689in}{1.404280in}}%
\pgfpathlineto{\pgfqpoint{4.846361in}{1.404280in}}%
\pgfpathlineto{\pgfqpoint{4.847033in}{1.391029in}}%
\pgfpathlineto{\pgfqpoint{4.847705in}{1.467884in}}%
\pgfpathlineto{\pgfqpoint{4.848377in}{1.467884in}}%
\pgfpathlineto{\pgfqpoint{4.849050in}{1.428131in}}%
\pgfpathlineto{\pgfqpoint{4.849722in}{1.462583in}}%
\pgfpathlineto{\pgfqpoint{4.850394in}{1.462583in}}%
\pgfpathlineto{\pgfqpoint{4.851738in}{1.345976in}}%
\pgfpathlineto{\pgfqpoint{4.852410in}{1.345976in}}%
\pgfpathlineto{\pgfqpoint{4.853755in}{1.414880in}}%
\pgfpathlineto{\pgfqpoint{4.854427in}{1.414880in}}%
\pgfpathlineto{\pgfqpoint{4.855771in}{1.438732in}}%
\pgfpathlineto{\pgfqpoint{4.856443in}{1.438732in}}%
\pgfpathlineto{\pgfqpoint{4.856443in}{1.372477in}}%
\pgfpathlineto{\pgfqpoint{4.857788in}{1.377778in}}%
\pgfpathlineto{\pgfqpoint{4.858460in}{1.377778in}}%
\pgfpathlineto{\pgfqpoint{4.859132in}{1.467884in}}%
\pgfpathlineto{\pgfqpoint{4.859804in}{1.367177in}}%
\pgfpathlineto{\pgfqpoint{4.860476in}{1.367177in}}%
\pgfpathlineto{\pgfqpoint{4.860476in}{1.457283in}}%
\pgfpathlineto{\pgfqpoint{4.861821in}{1.356576in}}%
\pgfpathlineto{\pgfqpoint{4.862493in}{1.356576in}}%
\pgfpathlineto{\pgfqpoint{4.863165in}{1.465234in}}%
\pgfpathlineto{\pgfqpoint{4.863837in}{1.406930in}}%
\pgfpathlineto{\pgfqpoint{4.864509in}{1.406930in}}%
\pgfpathlineto{\pgfqpoint{4.864509in}{1.457283in}}%
\pgfpathlineto{\pgfqpoint{4.865182in}{1.375128in}}%
\pgfpathlineto{\pgfqpoint{4.865854in}{1.441382in}}%
\pgfpathlineto{\pgfqpoint{4.866526in}{1.441382in}}%
\pgfpathlineto{\pgfqpoint{4.866526in}{1.380428in}}%
\pgfpathlineto{\pgfqpoint{4.867870in}{1.438732in}}%
\pgfpathlineto{\pgfqpoint{4.868542in}{1.438732in}}%
\pgfpathlineto{\pgfqpoint{4.869215in}{1.465234in}}%
\pgfpathlineto{\pgfqpoint{4.869887in}{1.404280in}}%
\pgfpathlineto{\pgfqpoint{4.870559in}{1.404280in}}%
\pgfpathlineto{\pgfqpoint{4.870559in}{1.483785in}}%
\pgfpathlineto{\pgfqpoint{4.871903in}{1.425481in}}%
\pgfpathlineto{\pgfqpoint{4.872575in}{1.425481in}}%
\pgfpathlineto{\pgfqpoint{4.872575in}{1.454633in}}%
\pgfpathlineto{\pgfqpoint{4.873248in}{1.393679in}}%
\pgfpathlineto{\pgfqpoint{4.873920in}{1.451983in}}%
\pgfpathlineto{\pgfqpoint{4.874592in}{1.451983in}}%
\pgfpathlineto{\pgfqpoint{4.874592in}{1.459933in}}%
\pgfpathlineto{\pgfqpoint{4.875936in}{1.377778in}}%
\pgfpathlineto{\pgfqpoint{4.876608in}{1.377778in}}%
\pgfpathlineto{\pgfqpoint{4.876608in}{1.367177in}}%
\pgfpathlineto{\pgfqpoint{4.877281in}{1.502336in}}%
\pgfpathlineto{\pgfqpoint{4.877953in}{1.425481in}}%
\pgfpathlineto{\pgfqpoint{4.878625in}{1.425481in}}%
\pgfpathlineto{\pgfqpoint{4.878625in}{1.356576in}}%
\pgfpathlineto{\pgfqpoint{4.879297in}{1.441382in}}%
\pgfpathlineto{\pgfqpoint{4.879969in}{1.422831in}}%
\pgfpathlineto{\pgfqpoint{4.880641in}{1.422831in}}%
\pgfpathlineto{\pgfqpoint{4.880641in}{1.391029in}}%
\pgfpathlineto{\pgfqpoint{4.881986in}{1.499686in}}%
\pgfpathlineto{\pgfqpoint{4.882658in}{1.499686in}}%
\pgfpathlineto{\pgfqpoint{4.882658in}{1.343325in}}%
\pgfpathlineto{\pgfqpoint{4.884002in}{1.486435in}}%
\pgfpathlineto{\pgfqpoint{4.884674in}{1.486435in}}%
\pgfpathlineto{\pgfqpoint{4.886019in}{1.428131in}}%
\pgfpathlineto{\pgfqpoint{4.886691in}{1.428131in}}%
\pgfpathlineto{\pgfqpoint{4.886691in}{1.459933in}}%
\pgfpathlineto{\pgfqpoint{4.887363in}{1.393679in}}%
\pgfpathlineto{\pgfqpoint{4.888035in}{1.417530in}}%
\pgfpathlineto{\pgfqpoint{4.888707in}{1.417530in}}%
\pgfpathlineto{\pgfqpoint{4.888707in}{1.512937in}}%
\pgfpathlineto{\pgfqpoint{4.890052in}{1.499686in}}%
\pgfpathlineto{\pgfqpoint{4.890724in}{1.499686in}}%
\pgfpathlineto{\pgfqpoint{4.891396in}{1.398979in}}%
\pgfpathlineto{\pgfqpoint{4.892068in}{1.462583in}}%
\pgfpathlineto{\pgfqpoint{4.892740in}{1.462583in}}%
\pgfpathlineto{\pgfqpoint{4.893413in}{1.425481in}}%
\pgfpathlineto{\pgfqpoint{4.894085in}{1.536788in}}%
\pgfpathlineto{\pgfqpoint{4.894757in}{1.536788in}}%
\pgfpathlineto{\pgfqpoint{4.895429in}{1.438732in}}%
\pgfpathlineto{\pgfqpoint{4.896101in}{1.515587in}}%
\pgfpathlineto{\pgfqpoint{4.896773in}{1.515587in}}%
\pgfpathlineto{\pgfqpoint{4.897446in}{1.436082in}}%
\pgfpathlineto{\pgfqpoint{4.898118in}{1.518237in}}%
\pgfpathlineto{\pgfqpoint{4.898790in}{1.518237in}}%
\pgfpathlineto{\pgfqpoint{4.900134in}{1.414880in}}%
\pgfpathlineto{\pgfqpoint{4.900806in}{1.414880in}}%
\pgfpathlineto{\pgfqpoint{4.900806in}{1.385728in}}%
\pgfpathlineto{\pgfqpoint{4.902151in}{1.494386in}}%
\pgfpathlineto{\pgfqpoint{4.902823in}{1.494386in}}%
\pgfpathlineto{\pgfqpoint{4.903495in}{1.536788in}}%
\pgfpathlineto{\pgfqpoint{4.904167in}{1.322124in}}%
\pgfpathlineto{\pgfqpoint{4.904839in}{1.322124in}}%
\pgfpathlineto{\pgfqpoint{4.905512in}{1.534138in}}%
\pgfpathlineto{\pgfqpoint{4.906184in}{1.391029in}}%
\pgfpathlineto{\pgfqpoint{4.906856in}{1.391029in}}%
\pgfpathlineto{\pgfqpoint{4.908200in}{1.489085in}}%
\pgfpathlineto{\pgfqpoint{4.908872in}{1.489085in}}%
\pgfpathlineto{\pgfqpoint{4.908872in}{1.446682in}}%
\pgfpathlineto{\pgfqpoint{4.910217in}{1.523537in}}%
\pgfpathlineto{\pgfqpoint{4.910889in}{1.523537in}}%
\pgfpathlineto{\pgfqpoint{4.911561in}{1.417530in}}%
\pgfpathlineto{\pgfqpoint{4.912233in}{1.465234in}}%
\pgfpathlineto{\pgfqpoint{4.912905in}{1.465234in}}%
\pgfpathlineto{\pgfqpoint{4.912905in}{1.544739in}}%
\pgfpathlineto{\pgfqpoint{4.914250in}{1.510287in}}%
\pgfpathlineto{\pgfqpoint{4.914922in}{1.510287in}}%
\pgfpathlineto{\pgfqpoint{4.915594in}{1.314174in}}%
\pgfpathlineto{\pgfqpoint{4.916266in}{1.478484in}}%
\pgfpathlineto{\pgfqpoint{4.916938in}{1.478484in}}%
\pgfpathlineto{\pgfqpoint{4.916938in}{1.428131in}}%
\pgfpathlineto{\pgfqpoint{4.917611in}{1.510287in}}%
\pgfpathlineto{\pgfqpoint{4.918283in}{1.478484in}}%
\pgfpathlineto{\pgfqpoint{4.918955in}{1.478484in}}%
\pgfpathlineto{\pgfqpoint{4.918955in}{1.369827in}}%
\pgfpathlineto{\pgfqpoint{4.919627in}{1.497036in}}%
\pgfpathlineto{\pgfqpoint{4.920299in}{1.478484in}}%
\pgfpathlineto{\pgfqpoint{4.920971in}{1.478484in}}%
\pgfpathlineto{\pgfqpoint{4.922316in}{1.414880in}}%
\pgfpathlineto{\pgfqpoint{4.922988in}{1.414880in}}%
\pgfpathlineto{\pgfqpoint{4.924332in}{1.457283in}}%
\pgfpathlineto{\pgfqpoint{4.925004in}{1.457283in}}%
\pgfpathlineto{\pgfqpoint{4.925677in}{1.483785in}}%
\pgfpathlineto{\pgfqpoint{4.926349in}{1.457283in}}%
\pgfpathlineto{\pgfqpoint{4.927021in}{1.457283in}}%
\pgfpathlineto{\pgfqpoint{4.927693in}{1.515587in}}%
\pgfpathlineto{\pgfqpoint{4.928365in}{1.436082in}}%
\pgfpathlineto{\pgfqpoint{4.929037in}{1.436082in}}%
\pgfpathlineto{\pgfqpoint{4.929037in}{1.375128in}}%
\pgfpathlineto{\pgfqpoint{4.930382in}{1.481135in}}%
\pgfpathlineto{\pgfqpoint{4.931054in}{1.481135in}}%
\pgfpathlineto{\pgfqpoint{4.931726in}{1.486435in}}%
\pgfpathlineto{\pgfqpoint{4.932398in}{1.324774in}}%
\pgfpathlineto{\pgfqpoint{4.933070in}{1.324774in}}%
\pgfpathlineto{\pgfqpoint{4.934415in}{1.483785in}}%
\pgfpathlineto{\pgfqpoint{4.935087in}{1.483785in}}%
\pgfpathlineto{\pgfqpoint{4.935087in}{1.428131in}}%
\pgfpathlineto{\pgfqpoint{4.936431in}{1.510287in}}%
\pgfpathlineto{\pgfqpoint{4.937103in}{1.510287in}}%
\pgfpathlineto{\pgfqpoint{4.937776in}{1.441382in}}%
\pgfpathlineto{\pgfqpoint{4.938448in}{1.465234in}}%
\pgfpathlineto{\pgfqpoint{4.939120in}{1.465234in}}%
\pgfpathlineto{\pgfqpoint{4.939120in}{1.430781in}}%
\pgfpathlineto{\pgfqpoint{4.940464in}{1.457283in}}%
\pgfpathlineto{\pgfqpoint{4.941136in}{1.457283in}}%
\pgfpathlineto{\pgfqpoint{4.941136in}{1.444032in}}%
\pgfpathlineto{\pgfqpoint{4.942481in}{1.526188in}}%
\pgfpathlineto{\pgfqpoint{4.943153in}{1.526188in}}%
\pgfpathlineto{\pgfqpoint{4.944497in}{1.441382in}}%
\pgfpathlineto{\pgfqpoint{4.945170in}{1.441382in}}%
\pgfpathlineto{\pgfqpoint{4.945170in}{1.497036in}}%
\pgfpathlineto{\pgfqpoint{4.946514in}{1.438732in}}%
\pgfpathlineto{\pgfqpoint{4.947186in}{1.438732in}}%
\pgfpathlineto{\pgfqpoint{4.948530in}{1.581841in}}%
\pgfpathlineto{\pgfqpoint{4.949203in}{1.581841in}}%
\pgfpathlineto{\pgfqpoint{4.950547in}{1.422831in}}%
\pgfpathlineto{\pgfqpoint{4.951219in}{1.422831in}}%
\pgfpathlineto{\pgfqpoint{4.951891in}{1.526188in}}%
\pgfpathlineto{\pgfqpoint{4.952563in}{1.441382in}}%
\pgfpathlineto{\pgfqpoint{4.953236in}{1.441382in}}%
\pgfpathlineto{\pgfqpoint{4.953236in}{1.504986in}}%
\pgfpathlineto{\pgfqpoint{4.953908in}{1.396329in}}%
\pgfpathlineto{\pgfqpoint{4.954580in}{1.398979in}}%
\pgfpathlineto{\pgfqpoint{4.955252in}{1.398979in}}%
\pgfpathlineto{\pgfqpoint{4.955924in}{1.422831in}}%
\pgfpathlineto{\pgfqpoint{4.956596in}{1.412230in}}%
\pgfpathlineto{\pgfqpoint{4.957269in}{1.412230in}}%
\pgfpathlineto{\pgfqpoint{4.957941in}{1.489085in}}%
\pgfpathlineto{\pgfqpoint{4.958613in}{1.412230in}}%
\pgfpathlineto{\pgfqpoint{4.959285in}{1.412230in}}%
\pgfpathlineto{\pgfqpoint{4.959285in}{1.467884in}}%
\pgfpathlineto{\pgfqpoint{4.960629in}{1.412230in}}%
\pgfpathlineto{\pgfqpoint{4.961302in}{1.412230in}}%
\pgfpathlineto{\pgfqpoint{4.961302in}{1.361877in}}%
\pgfpathlineto{\pgfqpoint{4.961974in}{1.486435in}}%
\pgfpathlineto{\pgfqpoint{4.962646in}{1.414880in}}%
\pgfpathlineto{\pgfqpoint{4.963318in}{1.414880in}}%
\pgfpathlineto{\pgfqpoint{4.963990in}{1.438732in}}%
\pgfpathlineto{\pgfqpoint{4.964662in}{1.383078in}}%
\pgfpathlineto{\pgfqpoint{4.965335in}{1.383078in}}%
\pgfpathlineto{\pgfqpoint{4.966007in}{1.497036in}}%
\pgfpathlineto{\pgfqpoint{4.966679in}{1.465234in}}%
\pgfpathlineto{\pgfqpoint{4.967351in}{1.465234in}}%
\pgfpathlineto{\pgfqpoint{4.968023in}{1.377778in}}%
\pgfpathlineto{\pgfqpoint{4.968695in}{1.422831in}}%
\pgfpathlineto{\pgfqpoint{4.969368in}{1.422831in}}%
\pgfpathlineto{\pgfqpoint{4.969368in}{1.539439in}}%
\pgfpathlineto{\pgfqpoint{4.970040in}{1.420181in}}%
\pgfpathlineto{\pgfqpoint{4.970712in}{1.481135in}}%
\pgfpathlineto{\pgfqpoint{4.971384in}{1.481135in}}%
\pgfpathlineto{\pgfqpoint{4.972728in}{1.375128in}}%
\pgfpathlineto{\pgfqpoint{4.973401in}{1.375128in}}%
\pgfpathlineto{\pgfqpoint{4.973401in}{1.520887in}}%
\pgfpathlineto{\pgfqpoint{4.974745in}{1.393679in}}%
\pgfpathlineto{\pgfqpoint{4.975417in}{1.393679in}}%
\pgfpathlineto{\pgfqpoint{4.976089in}{1.491735in}}%
\pgfpathlineto{\pgfqpoint{4.976761in}{1.444032in}}%
\pgfpathlineto{\pgfqpoint{4.977434in}{1.444032in}}%
\pgfpathlineto{\pgfqpoint{4.978106in}{1.422831in}}%
\pgfpathlineto{\pgfqpoint{4.978778in}{1.436082in}}%
\pgfpathlineto{\pgfqpoint{4.979450in}{1.436082in}}%
\pgfpathlineto{\pgfqpoint{4.979450in}{1.295622in}}%
\pgfpathlineto{\pgfqpoint{4.980794in}{1.523537in}}%
\pgfpathlineto{\pgfqpoint{4.981467in}{1.523537in}}%
\pgfpathlineto{\pgfqpoint{4.981467in}{1.380428in}}%
\pgfpathlineto{\pgfqpoint{4.982811in}{1.383078in}}%
\pgfpathlineto{\pgfqpoint{4.983483in}{1.383078in}}%
\pgfpathlineto{\pgfqpoint{4.983483in}{1.502336in}}%
\pgfpathlineto{\pgfqpoint{4.984827in}{1.353926in}}%
\pgfpathlineto{\pgfqpoint{4.985500in}{1.353926in}}%
\pgfpathlineto{\pgfqpoint{4.985500in}{1.462583in}}%
\pgfpathlineto{\pgfqpoint{4.986844in}{1.383078in}}%
\pgfpathlineto{\pgfqpoint{4.987516in}{1.383078in}}%
\pgfpathlineto{\pgfqpoint{4.988188in}{1.343325in}}%
\pgfpathlineto{\pgfqpoint{4.988860in}{1.409580in}}%
\pgfpathlineto{\pgfqpoint{4.989533in}{1.409580in}}%
\pgfpathlineto{\pgfqpoint{4.989533in}{1.459933in}}%
\pgfpathlineto{\pgfqpoint{4.990877in}{1.430781in}}%
\pgfpathlineto{\pgfqpoint{4.991549in}{1.430781in}}%
\pgfpathlineto{\pgfqpoint{4.992221in}{1.388378in}}%
\pgfpathlineto{\pgfqpoint{4.992893in}{1.393679in}}%
\pgfpathlineto{\pgfqpoint{4.993566in}{1.393679in}}%
\pgfpathlineto{\pgfqpoint{4.994238in}{1.383078in}}%
\pgfpathlineto{\pgfqpoint{4.994910in}{1.436082in}}%
\pgfpathlineto{\pgfqpoint{4.995582in}{1.436082in}}%
\pgfpathlineto{\pgfqpoint{4.995582in}{1.369827in}}%
\pgfpathlineto{\pgfqpoint{4.996926in}{1.401629in}}%
\pgfpathlineto{\pgfqpoint{4.997599in}{1.401629in}}%
\pgfpathlineto{\pgfqpoint{4.998271in}{1.457283in}}%
\pgfpathlineto{\pgfqpoint{4.998943in}{1.308873in}}%
\pgfpathlineto{\pgfqpoint{4.999615in}{1.308873in}}%
\pgfpathlineto{\pgfqpoint{4.999615in}{1.459933in}}%
\pgfpathlineto{\pgfqpoint{5.000959in}{1.375128in}}%
\pgfpathlineto{\pgfqpoint{5.001632in}{1.375128in}}%
\pgfpathlineto{\pgfqpoint{5.002304in}{1.430781in}}%
\pgfpathlineto{\pgfqpoint{5.002976in}{1.383078in}}%
\pgfpathlineto{\pgfqpoint{5.003648in}{1.383078in}}%
\pgfpathlineto{\pgfqpoint{5.004992in}{1.343325in}}%
\pgfpathlineto{\pgfqpoint{5.005665in}{1.343325in}}%
\pgfpathlineto{\pgfqpoint{5.006337in}{1.459933in}}%
\pgfpathlineto{\pgfqpoint{5.007009in}{1.420181in}}%
\pgfpathlineto{\pgfqpoint{5.007681in}{1.420181in}}%
\pgfpathlineto{\pgfqpoint{5.007681in}{1.290322in}}%
\pgfpathlineto{\pgfqpoint{5.009025in}{1.380428in}}%
\pgfpathlineto{\pgfqpoint{5.009698in}{1.380428in}}%
\pgfpathlineto{\pgfqpoint{5.009698in}{1.335375in}}%
\pgfpathlineto{\pgfqpoint{5.011042in}{1.338025in}}%
\pgfpathlineto{\pgfqpoint{5.011714in}{1.338025in}}%
\pgfpathlineto{\pgfqpoint{5.012386in}{1.406930in}}%
\pgfpathlineto{\pgfqpoint{5.013058in}{1.359227in}}%
\pgfpathlineto{\pgfqpoint{5.013731in}{1.359227in}}%
\pgfpathlineto{\pgfqpoint{5.013731in}{1.327424in}}%
\pgfpathlineto{\pgfqpoint{5.015075in}{1.388378in}}%
\pgfpathlineto{\pgfqpoint{5.015747in}{1.388378in}}%
\pgfpathlineto{\pgfqpoint{5.015747in}{1.316824in}}%
\pgfpathlineto{\pgfqpoint{5.017091in}{1.425481in}}%
\pgfpathlineto{\pgfqpoint{5.017764in}{1.425481in}}%
\pgfpathlineto{\pgfqpoint{5.018436in}{1.345976in}}%
\pgfpathlineto{\pgfqpoint{5.019108in}{1.491735in}}%
\pgfpathlineto{\pgfqpoint{5.019780in}{1.491735in}}%
\pgfpathlineto{\pgfqpoint{5.020452in}{1.311523in}}%
\pgfpathlineto{\pgfqpoint{5.021124in}{1.343325in}}%
\pgfpathlineto{\pgfqpoint{5.021797in}{1.343325in}}%
\pgfpathlineto{\pgfqpoint{5.022469in}{1.383078in}}%
\pgfpathlineto{\pgfqpoint{5.023141in}{1.311523in}}%
\pgfpathlineto{\pgfqpoint{5.023813in}{1.311523in}}%
\pgfpathlineto{\pgfqpoint{5.024485in}{1.292972in}}%
\pgfpathlineto{\pgfqpoint{5.025157in}{1.377778in}}%
\pgfpathlineto{\pgfqpoint{5.025830in}{1.377778in}}%
\pgfpathlineto{\pgfqpoint{5.025830in}{1.287672in}}%
\pgfpathlineto{\pgfqpoint{5.026502in}{1.409580in}}%
\pgfpathlineto{\pgfqpoint{5.027174in}{1.338025in}}%
\pgfpathlineto{\pgfqpoint{5.027846in}{1.338025in}}%
\pgfpathlineto{\pgfqpoint{5.027846in}{1.420181in}}%
\pgfpathlineto{\pgfqpoint{5.029190in}{1.359227in}}%
\pgfpathlineto{\pgfqpoint{5.029863in}{1.359227in}}%
\pgfpathlineto{\pgfqpoint{5.029863in}{1.229368in}}%
\pgfpathlineto{\pgfqpoint{5.030535in}{1.433431in}}%
\pgfpathlineto{\pgfqpoint{5.031207in}{1.330075in}}%
\pgfpathlineto{\pgfqpoint{5.031879in}{1.330075in}}%
\pgfpathlineto{\pgfqpoint{5.031879in}{1.338025in}}%
\pgfpathlineto{\pgfqpoint{5.032551in}{1.303573in}}%
\pgfpathlineto{\pgfqpoint{5.033223in}{1.324774in}}%
\pgfpathlineto{\pgfqpoint{5.033896in}{1.324774in}}%
\pgfpathlineto{\pgfqpoint{5.033896in}{1.247919in}}%
\pgfpathlineto{\pgfqpoint{5.035240in}{1.300923in}}%
\pgfpathlineto{\pgfqpoint{5.035912in}{1.300923in}}%
\pgfpathlineto{\pgfqpoint{5.036584in}{1.393679in}}%
\pgfpathlineto{\pgfqpoint{5.037256in}{1.285022in}}%
\pgfpathlineto{\pgfqpoint{5.037929in}{1.285022in}}%
\pgfpathlineto{\pgfqpoint{5.037929in}{1.306223in}}%
\pgfpathlineto{\pgfqpoint{5.039273in}{1.300923in}}%
\pgfpathlineto{\pgfqpoint{5.039945in}{1.300923in}}%
\pgfpathlineto{\pgfqpoint{5.040617in}{1.277071in}}%
\pgfpathlineto{\pgfqpoint{5.041289in}{1.327424in}}%
\pgfpathlineto{\pgfqpoint{5.041962in}{1.327424in}}%
\pgfpathlineto{\pgfqpoint{5.041962in}{1.391029in}}%
\pgfpathlineto{\pgfqpoint{5.043306in}{1.210817in}}%
\pgfpathlineto{\pgfqpoint{5.043978in}{1.210817in}}%
\pgfpathlineto{\pgfqpoint{5.043978in}{1.391029in}}%
\pgfpathlineto{\pgfqpoint{5.045322in}{1.367177in}}%
\pgfpathlineto{\pgfqpoint{5.045995in}{1.367177in}}%
\pgfpathlineto{\pgfqpoint{5.047339in}{1.247919in}}%
\pgfpathlineto{\pgfqpoint{5.048011in}{1.247919in}}%
\pgfpathlineto{\pgfqpoint{5.048011in}{1.322124in}}%
\pgfpathlineto{\pgfqpoint{5.048683in}{1.194916in}}%
\pgfpathlineto{\pgfqpoint{5.049355in}{1.300923in}}%
\pgfpathlineto{\pgfqpoint{5.050028in}{1.300923in}}%
\pgfpathlineto{\pgfqpoint{5.050028in}{1.242619in}}%
\pgfpathlineto{\pgfqpoint{5.050700in}{1.303573in}}%
\pgfpathlineto{\pgfqpoint{5.051372in}{1.285022in}}%
\pgfpathlineto{\pgfqpoint{5.052044in}{1.285022in}}%
\pgfpathlineto{\pgfqpoint{5.052044in}{1.316824in}}%
\pgfpathlineto{\pgfqpoint{5.052716in}{1.208166in}}%
\pgfpathlineto{\pgfqpoint{5.053388in}{1.213467in}}%
\pgfpathlineto{\pgfqpoint{5.054061in}{1.213467in}}%
\pgfpathlineto{\pgfqpoint{5.054733in}{1.258520in}}%
\pgfpathlineto{\pgfqpoint{5.055405in}{1.221417in}}%
\pgfpathlineto{\pgfqpoint{5.056077in}{1.221417in}}%
\pgfpathlineto{\pgfqpoint{5.057422in}{1.332725in}}%
\pgfpathlineto{\pgfqpoint{5.058094in}{1.332725in}}%
\pgfpathlineto{\pgfqpoint{5.059438in}{1.224068in}}%
\pgfpathlineto{\pgfqpoint{5.060110in}{1.224068in}}%
\pgfpathlineto{\pgfqpoint{5.060110in}{1.274421in}}%
\pgfpathlineto{\pgfqpoint{5.061455in}{1.218767in}}%
\pgfpathlineto{\pgfqpoint{5.062127in}{1.218767in}}%
\pgfpathlineto{\pgfqpoint{5.062127in}{1.290322in}}%
\pgfpathlineto{\pgfqpoint{5.063471in}{1.200216in}}%
\pgfpathlineto{\pgfqpoint{5.064143in}{1.200216in}}%
\pgfpathlineto{\pgfqpoint{5.064143in}{1.263820in}}%
\pgfpathlineto{\pgfqpoint{5.065488in}{1.152513in}}%
\pgfpathlineto{\pgfqpoint{5.066160in}{1.152513in}}%
\pgfpathlineto{\pgfqpoint{5.066832in}{1.266470in}}%
\pgfpathlineto{\pgfqpoint{5.067504in}{1.149863in}}%
\pgfpathlineto{\pgfqpoint{5.068176in}{1.149863in}}%
\pgfpathlineto{\pgfqpoint{5.069521in}{1.242619in}}%
\pgfpathlineto{\pgfqpoint{5.070193in}{1.242619in}}%
\pgfpathlineto{\pgfqpoint{5.070193in}{1.269121in}}%
\pgfpathlineto{\pgfqpoint{5.071537in}{1.247919in}}%
\pgfpathlineto{\pgfqpoint{5.072209in}{1.247919in}}%
\pgfpathlineto{\pgfqpoint{5.072881in}{1.200216in}}%
\pgfpathlineto{\pgfqpoint{5.073554in}{1.213467in}}%
\pgfpathlineto{\pgfqpoint{5.074898in}{1.213467in}}%
\pgfpathlineto{\pgfqpoint{5.074898in}{1.234668in}}%
\pgfpathlineto{\pgfqpoint{5.075570in}{1.080958in}}%
\pgfpathlineto{\pgfqpoint{5.076242in}{1.213467in}}%
\pgfpathlineto{\pgfqpoint{5.076914in}{1.213467in}}%
\pgfpathlineto{\pgfqpoint{5.078259in}{1.123361in}}%
\pgfpathlineto{\pgfqpoint{5.078931in}{1.123361in}}%
\pgfpathlineto{\pgfqpoint{5.079603in}{1.224068in}}%
\pgfpathlineto{\pgfqpoint{5.080275in}{1.218767in}}%
\pgfpathlineto{\pgfqpoint{5.080947in}{1.218767in}}%
\pgfpathlineto{\pgfqpoint{5.080947in}{1.136612in}}%
\pgfpathlineto{\pgfqpoint{5.082292in}{1.171064in}}%
\pgfpathlineto{\pgfqpoint{5.082964in}{1.171064in}}%
\pgfpathlineto{\pgfqpoint{5.082964in}{1.165764in}}%
\pgfpathlineto{\pgfqpoint{5.084308in}{1.197566in}}%
\pgfpathlineto{\pgfqpoint{5.084980in}{1.197566in}}%
\pgfpathlineto{\pgfqpoint{5.085653in}{1.173714in}}%
\pgfpathlineto{\pgfqpoint{5.086325in}{1.224068in}}%
\pgfpathlineto{\pgfqpoint{5.086997in}{1.224068in}}%
\pgfpathlineto{\pgfqpoint{5.086997in}{1.149863in}}%
\pgfpathlineto{\pgfqpoint{5.088341in}{1.253219in}}%
\pgfpathlineto{\pgfqpoint{5.089013in}{1.253219in}}%
\pgfpathlineto{\pgfqpoint{5.089686in}{1.184315in}}%
\pgfpathlineto{\pgfqpoint{5.090358in}{1.247919in}}%
\pgfpathlineto{\pgfqpoint{5.091030in}{1.247919in}}%
\pgfpathlineto{\pgfqpoint{5.092374in}{1.131311in}}%
\pgfpathlineto{\pgfqpoint{5.093046in}{1.131311in}}%
\pgfpathlineto{\pgfqpoint{5.093046in}{1.234668in}}%
\pgfpathlineto{\pgfqpoint{5.093719in}{1.128661in}}%
\pgfpathlineto{\pgfqpoint{5.094391in}{1.221417in}}%
\pgfpathlineto{\pgfqpoint{5.095063in}{1.221417in}}%
\pgfpathlineto{\pgfqpoint{5.096407in}{1.102159in}}%
\pgfpathlineto{\pgfqpoint{5.097079in}{1.102159in}}%
\pgfpathlineto{\pgfqpoint{5.097752in}{1.163113in}}%
\pgfpathlineto{\pgfqpoint{5.098424in}{1.118060in}}%
\pgfpathlineto{\pgfqpoint{5.099096in}{1.118060in}}%
\pgfpathlineto{\pgfqpoint{5.099096in}{1.110110in}}%
\pgfpathlineto{\pgfqpoint{5.099768in}{1.149863in}}%
\pgfpathlineto{\pgfqpoint{5.100440in}{1.110110in}}%
\pgfpathlineto{\pgfqpoint{5.101112in}{1.110110in}}%
\pgfpathlineto{\pgfqpoint{5.101112in}{1.073007in}}%
\pgfpathlineto{\pgfqpoint{5.102457in}{1.080958in}}%
\pgfpathlineto{\pgfqpoint{5.103129in}{1.080958in}}%
\pgfpathlineto{\pgfqpoint{5.103801in}{1.120711in}}%
\pgfpathlineto{\pgfqpoint{5.104473in}{1.110110in}}%
\pgfpathlineto{\pgfqpoint{5.105145in}{1.110110in}}%
\pgfpathlineto{\pgfqpoint{5.105145in}{1.107460in}}%
\pgfpathlineto{\pgfqpoint{5.105818in}{1.139262in}}%
\pgfpathlineto{\pgfqpoint{5.106490in}{1.115410in}}%
\pgfpathlineto{\pgfqpoint{5.107162in}{1.115410in}}%
\pgfpathlineto{\pgfqpoint{5.107162in}{1.131311in}}%
\pgfpathlineto{\pgfqpoint{5.108506in}{1.078308in}}%
\pgfpathlineto{\pgfqpoint{5.109178in}{1.078308in}}%
\pgfpathlineto{\pgfqpoint{5.109851in}{1.152513in}}%
\pgfpathlineto{\pgfqpoint{5.110523in}{1.096859in}}%
\pgfpathlineto{\pgfqpoint{5.111195in}{1.096859in}}%
\pgfpathlineto{\pgfqpoint{5.111195in}{1.054456in}}%
\pgfpathlineto{\pgfqpoint{5.112539in}{1.118060in}}%
\pgfpathlineto{\pgfqpoint{5.113211in}{1.118060in}}%
\pgfpathlineto{\pgfqpoint{5.114556in}{1.057106in}}%
\pgfpathlineto{\pgfqpoint{5.115228in}{1.057106in}}%
\pgfpathlineto{\pgfqpoint{5.115228in}{1.078308in}}%
\pgfpathlineto{\pgfqpoint{5.116572in}{1.067707in}}%
\pgfpathlineto{\pgfqpoint{5.117244in}{1.067707in}}%
\pgfpathlineto{\pgfqpoint{5.117917in}{1.136612in}}%
\pgfpathlineto{\pgfqpoint{5.118589in}{1.043856in}}%
\pgfpathlineto{\pgfqpoint{5.119261in}{1.043856in}}%
\pgfpathlineto{\pgfqpoint{5.119261in}{1.152513in}}%
\pgfpathlineto{\pgfqpoint{5.120605in}{1.099509in}}%
\pgfpathlineto{\pgfqpoint{5.121277in}{1.099509in}}%
\pgfpathlineto{\pgfqpoint{5.121950in}{1.043856in}}%
\pgfpathlineto{\pgfqpoint{5.122622in}{1.083608in}}%
\pgfpathlineto{\pgfqpoint{5.123294in}{1.083608in}}%
\pgfpathlineto{\pgfqpoint{5.123966in}{1.120711in}}%
\pgfpathlineto{\pgfqpoint{5.124638in}{1.025304in}}%
\pgfpathlineto{\pgfqpoint{5.125310in}{1.025304in}}%
\pgfpathlineto{\pgfqpoint{5.125310in}{1.102159in}}%
\pgfpathlineto{\pgfqpoint{5.126655in}{1.030605in}}%
\pgfpathlineto{\pgfqpoint{5.127327in}{1.030605in}}%
\pgfpathlineto{\pgfqpoint{5.128671in}{1.173714in}}%
\pgfpathlineto{\pgfqpoint{5.129343in}{1.173714in}}%
\pgfpathlineto{\pgfqpoint{5.130016in}{1.022654in}}%
\pgfpathlineto{\pgfqpoint{5.130688in}{1.110110in}}%
\pgfpathlineto{\pgfqpoint{5.131360in}{1.110110in}}%
\pgfpathlineto{\pgfqpoint{5.131360in}{0.996152in}}%
\pgfpathlineto{\pgfqpoint{5.132704in}{1.030605in}}%
\pgfpathlineto{\pgfqpoint{5.133376in}{1.030605in}}%
\pgfpathlineto{\pgfqpoint{5.133376in}{1.083608in}}%
\pgfpathlineto{\pgfqpoint{5.134049in}{1.025304in}}%
\pgfpathlineto{\pgfqpoint{5.134721in}{1.062407in}}%
\pgfpathlineto{\pgfqpoint{5.135393in}{1.062407in}}%
\pgfpathlineto{\pgfqpoint{5.135393in}{1.091559in}}%
\pgfpathlineto{\pgfqpoint{5.136065in}{0.982901in}}%
\pgfpathlineto{\pgfqpoint{5.136737in}{1.022654in}}%
\pgfpathlineto{\pgfqpoint{5.137409in}{1.022654in}}%
\pgfpathlineto{\pgfqpoint{5.137409in}{1.083608in}}%
\pgfpathlineto{\pgfqpoint{5.138082in}{1.006753in}}%
\pgfpathlineto{\pgfqpoint{5.138754in}{1.012053in}}%
\pgfpathlineto{\pgfqpoint{5.139426in}{1.012053in}}%
\pgfpathlineto{\pgfqpoint{5.140098in}{1.086258in}}%
\pgfpathlineto{\pgfqpoint{5.140770in}{1.025304in}}%
\pgfpathlineto{\pgfqpoint{5.141442in}{1.025304in}}%
\pgfpathlineto{\pgfqpoint{5.141442in}{0.996152in}}%
\pgfpathlineto{\pgfqpoint{5.142115in}{1.038555in}}%
\pgfpathlineto{\pgfqpoint{5.142787in}{1.025304in}}%
\pgfpathlineto{\pgfqpoint{5.143459in}{1.025304in}}%
\pgfpathlineto{\pgfqpoint{5.144803in}{0.977601in}}%
\pgfpathlineto{\pgfqpoint{5.145475in}{0.977601in}}%
\pgfpathlineto{\pgfqpoint{5.146148in}{0.948449in}}%
\pgfpathlineto{\pgfqpoint{5.146820in}{1.041205in}}%
\pgfpathlineto{\pgfqpoint{5.147492in}{1.041205in}}%
\pgfpathlineto{\pgfqpoint{5.147492in}{1.049156in}}%
\pgfpathlineto{\pgfqpoint{5.148164in}{0.969651in}}%
\pgfpathlineto{\pgfqpoint{5.148836in}{0.996152in}}%
\pgfpathlineto{\pgfqpoint{5.149508in}{0.996152in}}%
\pgfpathlineto{\pgfqpoint{5.150181in}{1.083608in}}%
\pgfpathlineto{\pgfqpoint{5.150853in}{1.041205in}}%
\pgfpathlineto{\pgfqpoint{5.151525in}{1.041205in}}%
\pgfpathlineto{\pgfqpoint{5.152197in}{0.996152in}}%
\pgfpathlineto{\pgfqpoint{5.152869in}{0.998803in}}%
\pgfpathlineto{\pgfqpoint{5.153541in}{0.998803in}}%
\pgfpathlineto{\pgfqpoint{5.153541in}{1.033255in}}%
\pgfpathlineto{\pgfqpoint{5.154886in}{0.977601in}}%
\pgfpathlineto{\pgfqpoint{5.155558in}{0.977601in}}%
\pgfpathlineto{\pgfqpoint{5.155558in}{0.948449in}}%
\pgfpathlineto{\pgfqpoint{5.156902in}{1.065057in}}%
\pgfpathlineto{\pgfqpoint{5.157574in}{1.065057in}}%
\pgfpathlineto{\pgfqpoint{5.157574in}{0.932548in}}%
\pgfpathlineto{\pgfqpoint{5.158919in}{0.940499in}}%
\pgfpathlineto{\pgfqpoint{5.159591in}{0.940499in}}%
\pgfpathlineto{\pgfqpoint{5.160263in}{0.964350in}}%
\pgfpathlineto{\pgfqpoint{5.160935in}{0.921947in}}%
\pgfpathlineto{\pgfqpoint{5.161607in}{0.921947in}}%
\pgfpathlineto{\pgfqpoint{5.162280in}{0.969651in}}%
\pgfpathlineto{\pgfqpoint{5.162952in}{0.961700in}}%
\pgfpathlineto{\pgfqpoint{5.163624in}{0.961700in}}%
\pgfpathlineto{\pgfqpoint{5.163624in}{0.953750in}}%
\pgfpathlineto{\pgfqpoint{5.164296in}{0.993502in}}%
\pgfpathlineto{\pgfqpoint{5.164968in}{0.982901in}}%
\pgfpathlineto{\pgfqpoint{5.165640in}{0.982901in}}%
\pgfpathlineto{\pgfqpoint{5.166313in}{0.911347in}}%
\pgfpathlineto{\pgfqpoint{5.166985in}{0.948449in}}%
\pgfpathlineto{\pgfqpoint{5.167657in}{0.948449in}}%
\pgfpathlineto{\pgfqpoint{5.167657in}{0.940499in}}%
\pgfpathlineto{\pgfqpoint{5.168329in}{0.961700in}}%
\pgfpathlineto{\pgfqpoint{5.169001in}{0.956400in}}%
\pgfpathlineto{\pgfqpoint{5.169674in}{0.956400in}}%
\pgfpathlineto{\pgfqpoint{5.169674in}{0.900746in}}%
\pgfpathlineto{\pgfqpoint{5.171018in}{0.913997in}}%
\pgfpathlineto{\pgfqpoint{5.171690in}{0.913997in}}%
\pgfpathlineto{\pgfqpoint{5.171690in}{0.961700in}}%
\pgfpathlineto{\pgfqpoint{5.173034in}{0.937848in}}%
\pgfpathlineto{\pgfqpoint{5.173707in}{0.937848in}}%
\pgfpathlineto{\pgfqpoint{5.174379in}{0.924598in}}%
\pgfpathlineto{\pgfqpoint{5.175051in}{0.924598in}}%
\pgfpathlineto{\pgfqpoint{5.175723in}{0.924598in}}%
\pgfpathlineto{\pgfqpoint{5.176395in}{0.916647in}}%
\pgfpathlineto{\pgfqpoint{5.177067in}{0.967000in}}%
\pgfpathlineto{\pgfqpoint{5.177740in}{0.967000in}}%
\pgfpathlineto{\pgfqpoint{5.179084in}{0.887495in}}%
\pgfpathlineto{\pgfqpoint{5.179756in}{0.887495in}}%
\pgfpathlineto{\pgfqpoint{5.180428in}{0.940499in}}%
\pgfpathlineto{\pgfqpoint{5.181100in}{0.927248in}}%
\pgfpathlineto{\pgfqpoint{5.181773in}{0.927248in}}%
\pgfpathlineto{\pgfqpoint{5.182445in}{0.882195in}}%
\pgfpathlineto{\pgfqpoint{5.183117in}{0.911347in}}%
\pgfpathlineto{\pgfqpoint{5.183789in}{0.911347in}}%
\pgfpathlineto{\pgfqpoint{5.184461in}{0.929898in}}%
\pgfpathlineto{\pgfqpoint{5.185133in}{0.927248in}}%
\pgfpathlineto{\pgfqpoint{5.185806in}{0.927248in}}%
\pgfpathlineto{\pgfqpoint{5.185806in}{0.940499in}}%
\pgfpathlineto{\pgfqpoint{5.187150in}{0.874244in}}%
\pgfpathlineto{\pgfqpoint{5.187822in}{0.874244in}}%
\pgfpathlineto{\pgfqpoint{5.189166in}{0.908697in}}%
\pgfpathlineto{\pgfqpoint{5.189839in}{0.908697in}}%
\pgfpathlineto{\pgfqpoint{5.191183in}{0.858343in}}%
\pgfpathlineto{\pgfqpoint{5.191855in}{0.858343in}}%
\pgfpathlineto{\pgfqpoint{5.191855in}{0.913997in}}%
\pgfpathlineto{\pgfqpoint{5.193199in}{0.868944in}}%
\pgfpathlineto{\pgfqpoint{5.193872in}{0.868944in}}%
\pgfpathlineto{\pgfqpoint{5.194544in}{0.921947in}}%
\pgfpathlineto{\pgfqpoint{5.195216in}{0.866294in}}%
\pgfpathlineto{\pgfqpoint{5.195888in}{0.866294in}}%
\pgfpathlineto{\pgfqpoint{5.196560in}{0.890145in}}%
\pgfpathlineto{\pgfqpoint{5.197232in}{0.876894in}}%
\pgfpathlineto{\pgfqpoint{5.197905in}{0.876894in}}%
\pgfpathlineto{\pgfqpoint{5.197905in}{0.829191in}}%
\pgfpathlineto{\pgfqpoint{5.198577in}{0.895446in}}%
\pgfpathlineto{\pgfqpoint{5.199249in}{0.868944in}}%
\pgfpathlineto{\pgfqpoint{5.199921in}{0.868944in}}%
\pgfpathlineto{\pgfqpoint{5.200593in}{0.916647in}}%
\pgfpathlineto{\pgfqpoint{5.201265in}{0.884845in}}%
\pgfpathlineto{\pgfqpoint{5.201938in}{0.884845in}}%
\pgfpathlineto{\pgfqpoint{5.201938in}{0.863644in}}%
\pgfpathlineto{\pgfqpoint{5.203282in}{0.863644in}}%
\pgfpathlineto{\pgfqpoint{5.203954in}{0.863644in}}%
\pgfpathlineto{\pgfqpoint{5.204626in}{0.882195in}}%
\pgfpathlineto{\pgfqpoint{5.205298in}{0.882195in}}%
\pgfpathlineto{\pgfqpoint{5.205971in}{0.882195in}}%
\pgfpathlineto{\pgfqpoint{5.207315in}{0.847742in}}%
\pgfpathlineto{\pgfqpoint{5.208659in}{0.847742in}}%
\pgfpathlineto{\pgfqpoint{5.208659in}{0.855693in}}%
\pgfpathlineto{\pgfqpoint{5.210004in}{0.810640in}}%
\pgfpathlineto{\pgfqpoint{5.210676in}{0.810640in}}%
\pgfpathlineto{\pgfqpoint{5.210676in}{0.892795in}}%
\pgfpathlineto{\pgfqpoint{5.212020in}{0.845092in}}%
\pgfpathlineto{\pgfqpoint{5.212692in}{0.845092in}}%
\pgfpathlineto{\pgfqpoint{5.212692in}{0.940499in}}%
\pgfpathlineto{\pgfqpoint{5.214037in}{0.831841in}}%
\pgfpathlineto{\pgfqpoint{5.214709in}{0.831841in}}%
\pgfpathlineto{\pgfqpoint{5.214709in}{0.818590in}}%
\pgfpathlineto{\pgfqpoint{5.215381in}{0.863644in}}%
\pgfpathlineto{\pgfqpoint{5.216053in}{0.834492in}}%
\pgfpathlineto{\pgfqpoint{5.216725in}{0.834492in}}%
\pgfpathlineto{\pgfqpoint{5.218070in}{0.876894in}}%
\pgfpathlineto{\pgfqpoint{5.218742in}{0.876894in}}%
\pgfpathlineto{\pgfqpoint{5.219414in}{0.810640in}}%
\pgfpathlineto{\pgfqpoint{5.220086in}{0.826541in}}%
\pgfpathlineto{\pgfqpoint{5.220758in}{0.826541in}}%
\pgfpathlineto{\pgfqpoint{5.220758in}{0.847742in}}%
\pgfpathlineto{\pgfqpoint{5.222103in}{0.845092in}}%
\pgfpathlineto{\pgfqpoint{5.222775in}{0.845092in}}%
\pgfpathlineto{\pgfqpoint{5.224119in}{0.797389in}}%
\pgfpathlineto{\pgfqpoint{5.224791in}{0.797389in}}%
\pgfpathlineto{\pgfqpoint{5.225463in}{0.839792in}}%
\pgfpathlineto{\pgfqpoint{5.226136in}{0.800039in}}%
\pgfpathlineto{\pgfqpoint{5.228152in}{0.800039in}}%
\pgfpathlineto{\pgfqpoint{5.228152in}{0.855693in}}%
\pgfpathlineto{\pgfqpoint{5.229496in}{0.815940in}}%
\pgfpathlineto{\pgfqpoint{5.230169in}{0.815940in}}%
\pgfpathlineto{\pgfqpoint{5.230169in}{0.837142in}}%
\pgfpathlineto{\pgfqpoint{5.231513in}{0.786788in}}%
\pgfpathlineto{\pgfqpoint{5.232185in}{0.786788in}}%
\pgfpathlineto{\pgfqpoint{5.232185in}{0.821241in}}%
\pgfpathlineto{\pgfqpoint{5.233529in}{0.818590in}}%
\pgfpathlineto{\pgfqpoint{5.234202in}{0.818590in}}%
\pgfpathlineto{\pgfqpoint{5.234874in}{0.792089in}}%
\pgfpathlineto{\pgfqpoint{5.235546in}{0.813290in}}%
\pgfpathlineto{\pgfqpoint{5.236890in}{0.813290in}}%
\pgfpathlineto{\pgfqpoint{5.236890in}{0.802689in}}%
\pgfpathlineto{\pgfqpoint{5.237562in}{0.834492in}}%
\pgfpathlineto{\pgfqpoint{5.238235in}{0.818590in}}%
\pgfpathlineto{\pgfqpoint{5.238907in}{0.818590in}}%
\pgfpathlineto{\pgfqpoint{5.240251in}{0.757636in}}%
\pgfpathlineto{\pgfqpoint{5.240923in}{0.757636in}}%
\pgfpathlineto{\pgfqpoint{5.241595in}{0.805340in}}%
\pgfpathlineto{\pgfqpoint{5.242268in}{0.802689in}}%
\pgfpathlineto{\pgfqpoint{5.242940in}{0.802689in}}%
\pgfpathlineto{\pgfqpoint{5.242940in}{0.781488in}}%
\pgfpathlineto{\pgfqpoint{5.243612in}{0.815940in}}%
\pgfpathlineto{\pgfqpoint{5.244284in}{0.810640in}}%
\pgfpathlineto{\pgfqpoint{5.244956in}{0.810640in}}%
\pgfpathlineto{\pgfqpoint{5.245628in}{0.781488in}}%
\pgfpathlineto{\pgfqpoint{5.246301in}{0.786788in}}%
\pgfpathlineto{\pgfqpoint{5.246973in}{0.786788in}}%
\pgfpathlineto{\pgfqpoint{5.246973in}{0.805340in}}%
\pgfpathlineto{\pgfqpoint{5.247645in}{0.784138in}}%
\pgfpathlineto{\pgfqpoint{5.248317in}{0.794739in}}%
\pgfpathlineto{\pgfqpoint{5.248989in}{0.794739in}}%
\pgfpathlineto{\pgfqpoint{5.249661in}{0.818590in}}%
\pgfpathlineto{\pgfqpoint{5.250334in}{0.749686in}}%
\pgfpathlineto{\pgfqpoint{5.251006in}{0.749686in}}%
\pgfpathlineto{\pgfqpoint{5.251006in}{0.807990in}}%
\pgfpathlineto{\pgfqpoint{5.252350in}{0.765587in}}%
\pgfpathlineto{\pgfqpoint{5.253022in}{0.765587in}}%
\pgfpathlineto{\pgfqpoint{5.253022in}{0.760287in}}%
\pgfpathlineto{\pgfqpoint{5.254367in}{0.823891in}}%
\pgfpathlineto{\pgfqpoint{5.255039in}{0.823891in}}%
\pgfpathlineto{\pgfqpoint{5.255711in}{0.754986in}}%
\pgfpathlineto{\pgfqpoint{5.256383in}{0.813290in}}%
\pgfpathlineto{\pgfqpoint{5.257055in}{0.813290in}}%
\pgfpathlineto{\pgfqpoint{5.257727in}{0.752336in}}%
\pgfpathlineto{\pgfqpoint{5.258400in}{0.802689in}}%
\pgfpathlineto{\pgfqpoint{5.259072in}{0.802689in}}%
\pgfpathlineto{\pgfqpoint{5.259072in}{0.762937in}}%
\pgfpathlineto{\pgfqpoint{5.260416in}{0.773537in}}%
\pgfpathlineto{\pgfqpoint{5.261760in}{0.773537in}}%
\pgfpathlineto{\pgfqpoint{5.262433in}{0.818590in}}%
\pgfpathlineto{\pgfqpoint{5.263105in}{0.736435in}}%
\pgfpathlineto{\pgfqpoint{5.263777in}{0.736435in}}%
\pgfpathlineto{\pgfqpoint{5.264449in}{0.786788in}}%
\pgfpathlineto{\pgfqpoint{5.265121in}{0.786788in}}%
\pgfpathlineto{\pgfqpoint{5.265793in}{0.786788in}}%
\pgfpathlineto{\pgfqpoint{5.265793in}{0.760287in}}%
\pgfpathlineto{\pgfqpoint{5.267138in}{0.786788in}}%
\pgfpathlineto{\pgfqpoint{5.267810in}{0.786788in}}%
\pgfpathlineto{\pgfqpoint{5.267810in}{0.757636in}}%
\pgfpathlineto{\pgfqpoint{5.269154in}{0.781488in}}%
\pgfpathlineto{\pgfqpoint{5.269826in}{0.781488in}}%
\pgfpathlineto{\pgfqpoint{5.271171in}{0.736435in}}%
\pgfpathlineto{\pgfqpoint{5.271843in}{0.736435in}}%
\pgfpathlineto{\pgfqpoint{5.271843in}{0.760287in}}%
\pgfpathlineto{\pgfqpoint{5.273187in}{0.752336in}}%
\pgfpathlineto{\pgfqpoint{5.273859in}{0.752336in}}%
\pgfpathlineto{\pgfqpoint{5.273859in}{0.781488in}}%
\pgfpathlineto{\pgfqpoint{5.274532in}{0.744386in}}%
\pgfpathlineto{\pgfqpoint{5.275204in}{0.752336in}}%
\pgfpathlineto{\pgfqpoint{5.276548in}{0.752336in}}%
\pgfpathlineto{\pgfqpoint{5.277220in}{0.765587in}}%
\pgfpathlineto{\pgfqpoint{5.277892in}{0.749686in}}%
\pgfpathlineto{\pgfqpoint{5.278565in}{0.749686in}}%
\pgfpathlineto{\pgfqpoint{5.279237in}{0.744386in}}%
\pgfpathlineto{\pgfqpoint{5.279909in}{0.757636in}}%
\pgfpathlineto{\pgfqpoint{5.280581in}{0.757636in}}%
\pgfpathlineto{\pgfqpoint{5.280581in}{0.733785in}}%
\pgfpathlineto{\pgfqpoint{5.281253in}{0.765587in}}%
\pgfpathlineto{\pgfqpoint{5.281926in}{0.739085in}}%
\pgfpathlineto{\pgfqpoint{5.282598in}{0.739085in}}%
\pgfpathlineto{\pgfqpoint{5.283270in}{0.725834in}}%
\pgfpathlineto{\pgfqpoint{5.283942in}{0.733785in}}%
\pgfpathlineto{\pgfqpoint{5.284614in}{0.733785in}}%
\pgfpathlineto{\pgfqpoint{5.284614in}{0.765587in}}%
\pgfpathlineto{\pgfqpoint{5.285959in}{0.757636in}}%
\pgfpathlineto{\pgfqpoint{5.286631in}{0.757636in}}%
\pgfpathlineto{\pgfqpoint{5.286631in}{0.725834in}}%
\pgfpathlineto{\pgfqpoint{5.287975in}{0.765587in}}%
\pgfpathlineto{\pgfqpoint{5.288647in}{0.765587in}}%
\pgfpathlineto{\pgfqpoint{5.289992in}{0.717884in}}%
\pgfpathlineto{\pgfqpoint{5.290664in}{0.717884in}}%
\pgfpathlineto{\pgfqpoint{5.290664in}{0.731135in}}%
\pgfpathlineto{\pgfqpoint{5.292008in}{0.723184in}}%
\pgfpathlineto{\pgfqpoint{5.292680in}{0.723184in}}%
\pgfpathlineto{\pgfqpoint{5.292680in}{0.754986in}}%
\pgfpathlineto{\pgfqpoint{5.294025in}{0.731135in}}%
\pgfpathlineto{\pgfqpoint{5.294697in}{0.731135in}}%
\pgfpathlineto{\pgfqpoint{5.295369in}{0.754986in}}%
\pgfpathlineto{\pgfqpoint{5.296041in}{0.715234in}}%
\pgfpathlineto{\pgfqpoint{5.296713in}{0.715234in}}%
\pgfpathlineto{\pgfqpoint{5.296713in}{0.765587in}}%
\pgfpathlineto{\pgfqpoint{5.298058in}{0.712583in}}%
\pgfpathlineto{\pgfqpoint{5.298730in}{0.712583in}}%
\pgfpathlineto{\pgfqpoint{5.298730in}{0.709933in}}%
\pgfpathlineto{\pgfqpoint{5.300074in}{0.731135in}}%
\pgfpathlineto{\pgfqpoint{5.300746in}{0.731135in}}%
\pgfpathlineto{\pgfqpoint{5.301418in}{0.709933in}}%
\pgfpathlineto{\pgfqpoint{5.302091in}{0.720534in}}%
\pgfpathlineto{\pgfqpoint{5.302763in}{0.720534in}}%
\pgfpathlineto{\pgfqpoint{5.304107in}{0.731135in}}%
\pgfpathlineto{\pgfqpoint{5.304779in}{0.731135in}}%
\pgfpathlineto{\pgfqpoint{5.306124in}{0.736435in}}%
\pgfpathlineto{\pgfqpoint{5.306796in}{0.736435in}}%
\pgfpathlineto{\pgfqpoint{5.307468in}{0.717884in}}%
\pgfpathlineto{\pgfqpoint{5.308140in}{0.736435in}}%
\pgfpathlineto{\pgfqpoint{5.308812in}{0.736435in}}%
\pgfpathlineto{\pgfqpoint{5.309484in}{0.701983in}}%
\pgfpathlineto{\pgfqpoint{5.310157in}{0.731135in}}%
\pgfpathlineto{\pgfqpoint{5.310829in}{0.731135in}}%
\pgfpathlineto{\pgfqpoint{5.310829in}{0.736435in}}%
\pgfpathlineto{\pgfqpoint{5.311501in}{0.696682in}}%
\pgfpathlineto{\pgfqpoint{5.312173in}{0.709933in}}%
\pgfpathlineto{\pgfqpoint{5.312845in}{0.709933in}}%
\pgfpathlineto{\pgfqpoint{5.312845in}{0.704633in}}%
\pgfpathlineto{\pgfqpoint{5.314190in}{0.733785in}}%
\pgfpathlineto{\pgfqpoint{5.314862in}{0.733785in}}%
\pgfpathlineto{\pgfqpoint{5.314862in}{0.691382in}}%
\pgfpathlineto{\pgfqpoint{5.316206in}{0.725834in}}%
\pgfpathlineto{\pgfqpoint{5.317550in}{0.725834in}}%
\pgfpathlineto{\pgfqpoint{5.317550in}{0.709933in}}%
\pgfpathlineto{\pgfqpoint{5.318223in}{0.739085in}}%
\pgfpathlineto{\pgfqpoint{5.318895in}{0.709933in}}%
\pgfpathlineto{\pgfqpoint{5.319567in}{0.709933in}}%
\pgfpathlineto{\pgfqpoint{5.319567in}{0.688732in}}%
\pgfpathlineto{\pgfqpoint{5.320911in}{0.720534in}}%
\pgfpathlineto{\pgfqpoint{5.321583in}{0.720534in}}%
\pgfpathlineto{\pgfqpoint{5.322256in}{0.691382in}}%
\pgfpathlineto{\pgfqpoint{5.322928in}{0.699333in}}%
\pgfpathlineto{\pgfqpoint{5.323600in}{0.699333in}}%
\pgfpathlineto{\pgfqpoint{5.324272in}{0.686082in}}%
\pgfpathlineto{\pgfqpoint{5.324944in}{0.704633in}}%
\pgfpathlineto{\pgfqpoint{5.326289in}{0.704633in}}%
\pgfpathlineto{\pgfqpoint{5.326289in}{0.672831in}}%
\pgfpathlineto{\pgfqpoint{5.326961in}{0.717884in}}%
\pgfpathlineto{\pgfqpoint{5.327633in}{0.715234in}}%
\pgfpathlineto{\pgfqpoint{5.328305in}{0.715234in}}%
\pgfpathlineto{\pgfqpoint{5.328305in}{0.683431in}}%
\pgfpathlineto{\pgfqpoint{5.329649in}{0.694032in}}%
\pgfpathlineto{\pgfqpoint{5.330322in}{0.694032in}}%
\pgfpathlineto{\pgfqpoint{5.330322in}{0.691382in}}%
\pgfpathlineto{\pgfqpoint{5.330994in}{0.715234in}}%
\pgfpathlineto{\pgfqpoint{5.331666in}{0.707283in}}%
\pgfpathlineto{\pgfqpoint{5.332338in}{0.707283in}}%
\pgfpathlineto{\pgfqpoint{5.332338in}{0.688732in}}%
\pgfpathlineto{\pgfqpoint{5.333010in}{0.712583in}}%
\pgfpathlineto{\pgfqpoint{5.333682in}{0.694032in}}%
\pgfpathlineto{\pgfqpoint{5.334355in}{0.694032in}}%
\pgfpathlineto{\pgfqpoint{5.335027in}{0.717884in}}%
\pgfpathlineto{\pgfqpoint{5.335027in}{0.664880in}}%
\pgfpathlineto{\pgfqpoint{5.335699in}{0.675481in}}%
\pgfpathlineto{\pgfqpoint{5.336371in}{0.675481in}}%
\pgfpathlineto{\pgfqpoint{5.336371in}{0.696682in}}%
\pgfpathlineto{\pgfqpoint{5.337715in}{0.691382in}}%
\pgfpathlineto{\pgfqpoint{5.338388in}{0.691382in}}%
\pgfpathlineto{\pgfqpoint{5.338388in}{0.731135in}}%
\pgfpathlineto{\pgfqpoint{5.339060in}{0.686082in}}%
\pgfpathlineto{\pgfqpoint{5.339732in}{0.699333in}}%
\pgfpathlineto{\pgfqpoint{5.340404in}{0.699333in}}%
\pgfpathlineto{\pgfqpoint{5.341076in}{0.667530in}}%
\pgfpathlineto{\pgfqpoint{5.341748in}{0.688732in}}%
\pgfpathlineto{\pgfqpoint{5.343093in}{0.688732in}}%
\pgfpathlineto{\pgfqpoint{5.343765in}{0.709933in}}%
\pgfpathlineto{\pgfqpoint{5.344437in}{0.707283in}}%
\pgfpathlineto{\pgfqpoint{5.345109in}{0.707283in}}%
\pgfpathlineto{\pgfqpoint{5.345109in}{0.664880in}}%
\pgfpathlineto{\pgfqpoint{5.346454in}{0.678131in}}%
\pgfpathlineto{\pgfqpoint{5.347126in}{0.678131in}}%
\pgfpathlineto{\pgfqpoint{5.347126in}{0.686082in}}%
\pgfpathlineto{\pgfqpoint{5.348470in}{0.683431in}}%
\pgfpathlineto{\pgfqpoint{5.349142in}{0.683431in}}%
\pgfpathlineto{\pgfqpoint{5.349142in}{0.699333in}}%
\pgfpathlineto{\pgfqpoint{5.350487in}{0.683431in}}%
\pgfpathlineto{\pgfqpoint{5.351159in}{0.683431in}}%
\pgfpathlineto{\pgfqpoint{5.351159in}{0.680781in}}%
\pgfpathlineto{\pgfqpoint{5.352503in}{0.709933in}}%
\pgfpathlineto{\pgfqpoint{5.353175in}{0.709933in}}%
\pgfpathlineto{\pgfqpoint{5.353847in}{0.675481in}}%
\pgfpathlineto{\pgfqpoint{5.354520in}{0.696682in}}%
\pgfpathlineto{\pgfqpoint{5.355192in}{0.696682in}}%
\pgfpathlineto{\pgfqpoint{5.355192in}{0.686082in}}%
\pgfpathlineto{\pgfqpoint{5.356536in}{0.688732in}}%
\pgfpathlineto{\pgfqpoint{5.357208in}{0.688732in}}%
\pgfpathlineto{\pgfqpoint{5.357880in}{0.672831in}}%
\pgfpathlineto{\pgfqpoint{5.358553in}{0.694032in}}%
\pgfpathlineto{\pgfqpoint{5.359225in}{0.694032in}}%
\pgfpathlineto{\pgfqpoint{5.359897in}{0.670181in}}%
\pgfpathlineto{\pgfqpoint{5.360569in}{0.707283in}}%
\pgfpathlineto{\pgfqpoint{5.361241in}{0.707283in}}%
\pgfpathlineto{\pgfqpoint{5.362586in}{0.670181in}}%
\pgfpathlineto{\pgfqpoint{5.363258in}{0.670181in}}%
\pgfpathlineto{\pgfqpoint{5.363930in}{0.691382in}}%
\pgfpathlineto{\pgfqpoint{5.364602in}{0.686082in}}%
\pgfpathlineto{\pgfqpoint{5.365274in}{0.686082in}}%
\pgfpathlineto{\pgfqpoint{5.365946in}{0.688732in}}%
\pgfpathlineto{\pgfqpoint{5.366619in}{0.678131in}}%
\pgfpathlineto{\pgfqpoint{5.367291in}{0.678131in}}%
\pgfpathlineto{\pgfqpoint{5.367291in}{0.688732in}}%
\pgfpathlineto{\pgfqpoint{5.367963in}{0.667530in}}%
\pgfpathlineto{\pgfqpoint{5.368635in}{0.680781in}}%
\pgfpathlineto{\pgfqpoint{5.369307in}{0.680781in}}%
\pgfpathlineto{\pgfqpoint{5.369307in}{0.691382in}}%
\pgfpathlineto{\pgfqpoint{5.370652in}{0.662230in}}%
\pgfpathlineto{\pgfqpoint{5.371324in}{0.662230in}}%
\pgfpathlineto{\pgfqpoint{5.371324in}{0.691382in}}%
\pgfpathlineto{\pgfqpoint{5.372668in}{0.686082in}}%
\pgfpathlineto{\pgfqpoint{5.373340in}{0.686082in}}%
\pgfpathlineto{\pgfqpoint{5.374012in}{0.670181in}}%
\pgfpathlineto{\pgfqpoint{5.374685in}{0.680781in}}%
\pgfpathlineto{\pgfqpoint{5.375357in}{0.680781in}}%
\pgfpathlineto{\pgfqpoint{5.375357in}{0.691382in}}%
\pgfpathlineto{\pgfqpoint{5.376701in}{0.667530in}}%
\pgfpathlineto{\pgfqpoint{5.377373in}{0.667530in}}%
\pgfpathlineto{\pgfqpoint{5.377373in}{0.686082in}}%
\pgfpathlineto{\pgfqpoint{5.378718in}{0.680781in}}%
\pgfpathlineto{\pgfqpoint{5.379390in}{0.680781in}}%
\pgfpathlineto{\pgfqpoint{5.379390in}{0.691382in}}%
\pgfpathlineto{\pgfqpoint{5.380734in}{0.675481in}}%
\pgfpathlineto{\pgfqpoint{5.381406in}{0.675481in}}%
\pgfpathlineto{\pgfqpoint{5.382751in}{0.659580in}}%
\pgfpathlineto{\pgfqpoint{5.383423in}{0.659580in}}%
\pgfpathlineto{\pgfqpoint{5.384095in}{0.672831in}}%
\pgfpathlineto{\pgfqpoint{5.384767in}{0.667530in}}%
\pgfpathlineto{\pgfqpoint{5.385439in}{0.667530in}}%
\pgfpathlineto{\pgfqpoint{5.385439in}{0.678131in}}%
\pgfpathlineto{\pgfqpoint{5.386784in}{0.662230in}}%
\pgfpathlineto{\pgfqpoint{5.387456in}{0.662230in}}%
\pgfpathlineto{\pgfqpoint{5.388128in}{0.659580in}}%
\pgfpathlineto{\pgfqpoint{5.388800in}{0.672831in}}%
\pgfpathlineto{\pgfqpoint{5.389472in}{0.672831in}}%
\pgfpathlineto{\pgfqpoint{5.389472in}{0.664880in}}%
\pgfpathlineto{\pgfqpoint{5.390817in}{0.670181in}}%
\pgfpathlineto{\pgfqpoint{5.391489in}{0.670181in}}%
\pgfpathlineto{\pgfqpoint{5.392833in}{0.675481in}}%
\pgfpathlineto{\pgfqpoint{5.393505in}{0.675481in}}%
\pgfpathlineto{\pgfqpoint{5.393505in}{0.672831in}}%
\pgfpathlineto{\pgfqpoint{5.394178in}{0.680781in}}%
\pgfpathlineto{\pgfqpoint{5.394850in}{0.680781in}}%
\pgfpathlineto{\pgfqpoint{5.395522in}{0.680781in}}%
\pgfpathlineto{\pgfqpoint{5.395522in}{0.667530in}}%
\pgfpathlineto{\pgfqpoint{5.396194in}{0.688732in}}%
\pgfpathlineto{\pgfqpoint{5.396866in}{0.670181in}}%
\pgfpathlineto{\pgfqpoint{5.397538in}{0.670181in}}%
\pgfpathlineto{\pgfqpoint{5.397538in}{0.667530in}}%
\pgfpathlineto{\pgfqpoint{5.398883in}{0.683431in}}%
\pgfpathlineto{\pgfqpoint{5.399555in}{0.683431in}}%
\pgfpathlineto{\pgfqpoint{5.400227in}{0.688732in}}%
\pgfpathlineto{\pgfqpoint{5.400899in}{0.654280in}}%
\pgfpathlineto{\pgfqpoint{5.401571in}{0.654280in}}%
\pgfpathlineto{\pgfqpoint{5.402244in}{0.678131in}}%
\pgfpathlineto{\pgfqpoint{5.402916in}{0.667530in}}%
\pgfpathlineto{\pgfqpoint{5.403588in}{0.667530in}}%
\pgfpathlineto{\pgfqpoint{5.404260in}{0.659580in}}%
\pgfpathlineto{\pgfqpoint{5.404932in}{0.694032in}}%
\pgfpathlineto{\pgfqpoint{5.405604in}{0.694032in}}%
\pgfpathlineto{\pgfqpoint{5.405604in}{0.701983in}}%
\pgfpathlineto{\pgfqpoint{5.406949in}{0.672831in}}%
\pgfpathlineto{\pgfqpoint{5.407621in}{0.672831in}}%
\pgfpathlineto{\pgfqpoint{5.407621in}{0.675481in}}%
\pgfpathlineto{\pgfqpoint{5.408965in}{0.643679in}}%
\pgfpathlineto{\pgfqpoint{5.409637in}{0.643679in}}%
\pgfpathlineto{\pgfqpoint{5.410982in}{0.688732in}}%
\pgfpathlineto{\pgfqpoint{5.411654in}{0.688732in}}%
\pgfpathlineto{\pgfqpoint{5.412998in}{0.659580in}}%
\pgfpathlineto{\pgfqpoint{5.413670in}{0.659580in}}%
\pgfpathlineto{\pgfqpoint{5.413670in}{0.667530in}}%
\pgfpathlineto{\pgfqpoint{5.415015in}{0.667530in}}%
\pgfpathlineto{\pgfqpoint{5.416359in}{0.667530in}}%
\pgfpathlineto{\pgfqpoint{5.416359in}{0.659580in}}%
\pgfpathlineto{\pgfqpoint{5.417703in}{0.688732in}}%
\pgfpathlineto{\pgfqpoint{5.418376in}{0.688732in}}%
\pgfpathlineto{\pgfqpoint{5.419720in}{0.664880in}}%
\pgfpathlineto{\pgfqpoint{5.420392in}{0.664880in}}%
\pgfpathlineto{\pgfqpoint{5.420392in}{0.683431in}}%
\pgfpathlineto{\pgfqpoint{5.421064in}{0.656930in}}%
\pgfpathlineto{\pgfqpoint{5.421736in}{0.675481in}}%
\pgfpathlineto{\pgfqpoint{5.422409in}{0.675481in}}%
\pgfpathlineto{\pgfqpoint{5.422409in}{0.662230in}}%
\pgfpathlineto{\pgfqpoint{5.423081in}{0.686082in}}%
\pgfpathlineto{\pgfqpoint{5.423753in}{0.667530in}}%
\pgfpathlineto{\pgfqpoint{5.424425in}{0.667530in}}%
\pgfpathlineto{\pgfqpoint{5.425097in}{0.680781in}}%
\pgfpathlineto{\pgfqpoint{5.425769in}{0.656930in}}%
\pgfpathlineto{\pgfqpoint{5.426442in}{0.656930in}}%
\pgfpathlineto{\pgfqpoint{5.427114in}{0.678131in}}%
\pgfpathlineto{\pgfqpoint{5.427786in}{0.662230in}}%
\pgfpathlineto{\pgfqpoint{5.428458in}{0.662230in}}%
\pgfpathlineto{\pgfqpoint{5.428458in}{0.643679in}}%
\pgfpathlineto{\pgfqpoint{5.429130in}{0.699333in}}%
\pgfpathlineto{\pgfqpoint{5.429802in}{0.659580in}}%
\pgfpathlineto{\pgfqpoint{5.430475in}{0.659580in}}%
\pgfpathlineto{\pgfqpoint{5.430475in}{0.648979in}}%
\pgfpathlineto{\pgfqpoint{5.431147in}{0.664880in}}%
\pgfpathlineto{\pgfqpoint{5.431819in}{0.659580in}}%
\pgfpathlineto{\pgfqpoint{5.433163in}{0.659580in}}%
\pgfpathlineto{\pgfqpoint{5.433835in}{0.643679in}}%
\pgfpathlineto{\pgfqpoint{5.434508in}{0.683431in}}%
\pgfpathlineto{\pgfqpoint{5.435180in}{0.683431in}}%
\pgfpathlineto{\pgfqpoint{5.436524in}{0.664880in}}%
\pgfpathlineto{\pgfqpoint{5.437196in}{0.664880in}}%
\pgfpathlineto{\pgfqpoint{5.437196in}{0.667530in}}%
\pgfpathlineto{\pgfqpoint{5.438541in}{0.659580in}}%
\pgfpathlineto{\pgfqpoint{5.439213in}{0.659580in}}%
\pgfpathlineto{\pgfqpoint{5.439213in}{0.680781in}}%
\pgfpathlineto{\pgfqpoint{5.439885in}{0.651629in}}%
\pgfpathlineto{\pgfqpoint{5.440557in}{0.670181in}}%
\pgfpathlineto{\pgfqpoint{5.441229in}{0.670181in}}%
\pgfpathlineto{\pgfqpoint{5.442574in}{0.656930in}}%
\pgfpathlineto{\pgfqpoint{5.443246in}{0.656930in}}%
\pgfpathlineto{\pgfqpoint{5.444590in}{0.678131in}}%
\pgfpathlineto{\pgfqpoint{5.445262in}{0.678131in}}%
\pgfpathlineto{\pgfqpoint{5.446607in}{0.654280in}}%
\pgfpathlineto{\pgfqpoint{5.447951in}{0.654280in}}%
\pgfpathlineto{\pgfqpoint{5.449295in}{0.670181in}}%
\pgfpathlineto{\pgfqpoint{5.449967in}{0.670181in}}%
\pgfpathlineto{\pgfqpoint{5.451312in}{0.656930in}}%
\pgfpathlineto{\pgfqpoint{5.451984in}{0.656930in}}%
\pgfpathlineto{\pgfqpoint{5.453328in}{0.670181in}}%
\pgfpathlineto{\pgfqpoint{5.454000in}{0.670181in}}%
\pgfpathlineto{\pgfqpoint{5.454000in}{0.664880in}}%
\pgfpathlineto{\pgfqpoint{5.455345in}{0.664880in}}%
\pgfpathlineto{\pgfqpoint{5.456017in}{0.664880in}}%
\pgfpathlineto{\pgfqpoint{5.456017in}{0.659580in}}%
\pgfpathlineto{\pgfqpoint{5.457361in}{0.664880in}}%
\pgfpathlineto{\pgfqpoint{5.458033in}{0.664880in}}%
\pgfpathlineto{\pgfqpoint{5.458706in}{0.648979in}}%
\pgfpathlineto{\pgfqpoint{5.459378in}{0.675481in}}%
\pgfpathlineto{\pgfqpoint{5.460050in}{0.675481in}}%
\pgfpathlineto{\pgfqpoint{5.460050in}{0.688732in}}%
\pgfpathlineto{\pgfqpoint{5.460722in}{0.638378in}}%
\pgfpathlineto{\pgfqpoint{5.461394in}{0.662230in}}%
\pgfpathlineto{\pgfqpoint{5.462066in}{0.662230in}}%
\pgfpathlineto{\pgfqpoint{5.462066in}{0.651629in}}%
\pgfpathlineto{\pgfqpoint{5.463411in}{0.659580in}}%
\pgfpathlineto{\pgfqpoint{5.464083in}{0.659580in}}%
\pgfpathlineto{\pgfqpoint{5.464083in}{0.654280in}}%
\pgfpathlineto{\pgfqpoint{5.464755in}{0.670181in}}%
\pgfpathlineto{\pgfqpoint{5.465427in}{0.667530in}}%
\pgfpathlineto{\pgfqpoint{5.466099in}{0.667530in}}%
\pgfpathlineto{\pgfqpoint{5.467444in}{0.659580in}}%
\pgfpathlineto{\pgfqpoint{5.468788in}{0.659580in}}%
\pgfpathlineto{\pgfqpoint{5.468788in}{0.646329in}}%
\pgfpathlineto{\pgfqpoint{5.470132in}{0.678131in}}%
\pgfpathlineto{\pgfqpoint{5.470805in}{0.678131in}}%
\pgfpathlineto{\pgfqpoint{5.472149in}{0.656930in}}%
\pgfpathlineto{\pgfqpoint{5.472821in}{0.656930in}}%
\pgfpathlineto{\pgfqpoint{5.472821in}{0.670181in}}%
\pgfpathlineto{\pgfqpoint{5.474165in}{0.651629in}}%
\pgfpathlineto{\pgfqpoint{5.474838in}{0.651629in}}%
\pgfpathlineto{\pgfqpoint{5.475510in}{0.672831in}}%
\pgfpathlineto{\pgfqpoint{5.476182in}{0.672831in}}%
\pgfpathlineto{\pgfqpoint{5.476854in}{0.672831in}}%
\pgfpathlineto{\pgfqpoint{5.478198in}{0.662230in}}%
\pgfpathlineto{\pgfqpoint{5.479543in}{0.662230in}}%
\pgfpathlineto{\pgfqpoint{5.479543in}{0.654280in}}%
\pgfpathlineto{\pgfqpoint{5.480215in}{0.678131in}}%
\pgfpathlineto{\pgfqpoint{5.480887in}{0.654280in}}%
\pgfpathlineto{\pgfqpoint{5.481559in}{0.654280in}}%
\pgfpathlineto{\pgfqpoint{5.481559in}{0.651629in}}%
\pgfpathlineto{\pgfqpoint{5.482904in}{0.667530in}}%
\pgfpathlineto{\pgfqpoint{5.483576in}{0.667530in}}%
\pgfpathlineto{\pgfqpoint{5.483576in}{0.670181in}}%
\pgfpathlineto{\pgfqpoint{5.484920in}{0.654280in}}%
\pgfpathlineto{\pgfqpoint{5.485592in}{0.654280in}}%
\pgfpathlineto{\pgfqpoint{5.486264in}{0.659580in}}%
\pgfpathlineto{\pgfqpoint{5.486937in}{0.659580in}}%
\pgfpathlineto{\pgfqpoint{5.487609in}{0.659580in}}%
\pgfpathlineto{\pgfqpoint{5.488281in}{0.654280in}}%
\pgfpathlineto{\pgfqpoint{5.488953in}{0.678131in}}%
\pgfpathlineto{\pgfqpoint{5.489625in}{0.678131in}}%
\pgfpathlineto{\pgfqpoint{5.490970in}{0.656930in}}%
\pgfpathlineto{\pgfqpoint{5.491642in}{0.656930in}}%
\pgfpathlineto{\pgfqpoint{5.491642in}{0.651629in}}%
\pgfpathlineto{\pgfqpoint{5.492314in}{0.662230in}}%
\pgfpathlineto{\pgfqpoint{5.492986in}{0.654280in}}%
\pgfpathlineto{\pgfqpoint{5.493658in}{0.654280in}}%
\pgfpathlineto{\pgfqpoint{5.493658in}{0.675481in}}%
\pgfpathlineto{\pgfqpoint{5.495003in}{0.662230in}}%
\pgfpathlineto{\pgfqpoint{5.495675in}{0.662230in}}%
\pgfpathlineto{\pgfqpoint{5.495675in}{0.667530in}}%
\pgfpathlineto{\pgfqpoint{5.497019in}{0.659580in}}%
\pgfpathlineto{\pgfqpoint{5.497691in}{0.659580in}}%
\pgfpathlineto{\pgfqpoint{5.497691in}{0.656930in}}%
\pgfpathlineto{\pgfqpoint{5.499036in}{0.656930in}}%
\pgfpathlineto{\pgfqpoint{5.499708in}{0.656930in}}%
\pgfpathlineto{\pgfqpoint{5.499708in}{0.664880in}}%
\pgfpathlineto{\pgfqpoint{5.501052in}{0.664880in}}%
\pgfpathlineto{\pgfqpoint{5.501724in}{0.664880in}}%
\pgfpathlineto{\pgfqpoint{5.501724in}{0.648979in}}%
\pgfpathlineto{\pgfqpoint{5.503069in}{0.659580in}}%
\pgfpathlineto{\pgfqpoint{5.503741in}{0.659580in}}%
\pgfpathlineto{\pgfqpoint{5.503741in}{0.651629in}}%
\pgfpathlineto{\pgfqpoint{5.505085in}{0.659580in}}%
\pgfpathlineto{\pgfqpoint{5.505757in}{0.659580in}}%
\pgfpathlineto{\pgfqpoint{5.506429in}{0.643679in}}%
\pgfpathlineto{\pgfqpoint{5.507102in}{0.646329in}}%
\pgfpathlineto{\pgfqpoint{5.507774in}{0.646329in}}%
\pgfpathlineto{\pgfqpoint{5.507774in}{0.675481in}}%
\pgfpathlineto{\pgfqpoint{5.509118in}{0.662230in}}%
\pgfpathlineto{\pgfqpoint{5.509790in}{0.662230in}}%
\pgfpathlineto{\pgfqpoint{5.510463in}{0.659580in}}%
\pgfpathlineto{\pgfqpoint{5.511135in}{0.670181in}}%
\pgfpathlineto{\pgfqpoint{5.511807in}{0.670181in}}%
\pgfpathlineto{\pgfqpoint{5.512479in}{0.654280in}}%
\pgfpathlineto{\pgfqpoint{5.513151in}{0.656930in}}%
\pgfpathlineto{\pgfqpoint{5.513823in}{0.656930in}}%
\pgfpathlineto{\pgfqpoint{5.513823in}{0.667530in}}%
\pgfpathlineto{\pgfqpoint{5.514496in}{0.646329in}}%
\pgfpathlineto{\pgfqpoint{5.515168in}{0.651629in}}%
\pgfpathlineto{\pgfqpoint{5.515840in}{0.651629in}}%
\pgfpathlineto{\pgfqpoint{5.515840in}{0.670181in}}%
\pgfpathlineto{\pgfqpoint{5.517184in}{0.664880in}}%
\pgfpathlineto{\pgfqpoint{5.517856in}{0.664880in}}%
\pgfpathlineto{\pgfqpoint{5.517856in}{0.643679in}}%
\pgfpathlineto{\pgfqpoint{5.519201in}{0.675481in}}%
\pgfpathlineto{\pgfqpoint{5.519873in}{0.675481in}}%
\pgfpathlineto{\pgfqpoint{5.519873in}{0.656930in}}%
\pgfpathlineto{\pgfqpoint{5.521217in}{0.656930in}}%
\pgfpathlineto{\pgfqpoint{5.521889in}{0.656930in}}%
\pgfpathlineto{\pgfqpoint{5.521889in}{0.672831in}}%
\pgfpathlineto{\pgfqpoint{5.523234in}{0.656930in}}%
\pgfpathlineto{\pgfqpoint{5.524578in}{0.656930in}}%
\pgfpathlineto{\pgfqpoint{5.524578in}{0.651629in}}%
\pgfpathlineto{\pgfqpoint{5.525922in}{0.656930in}}%
\pgfpathlineto{\pgfqpoint{5.526595in}{0.656930in}}%
\pgfpathlineto{\pgfqpoint{5.527267in}{0.667530in}}%
\pgfpathlineto{\pgfqpoint{5.527939in}{0.641029in}}%
\pgfpathlineto{\pgfqpoint{5.528611in}{0.641029in}}%
\pgfpathlineto{\pgfqpoint{5.528611in}{0.670181in}}%
\pgfpathlineto{\pgfqpoint{5.529955in}{0.656930in}}%
\pgfpathlineto{\pgfqpoint{5.530628in}{0.656930in}}%
\pgfpathlineto{\pgfqpoint{5.531300in}{0.672831in}}%
\pgfpathlineto{\pgfqpoint{5.531972in}{0.648979in}}%
\pgfpathlineto{\pgfqpoint{5.532644in}{0.648979in}}%
\pgfpathlineto{\pgfqpoint{5.532644in}{0.667530in}}%
\pgfpathlineto{\pgfqpoint{5.533316in}{0.643679in}}%
\pgfpathlineto{\pgfqpoint{5.533988in}{0.659580in}}%
\pgfpathlineto{\pgfqpoint{5.534661in}{0.659580in}}%
\pgfpathlineto{\pgfqpoint{5.534661in}{0.662230in}}%
\pgfpathlineto{\pgfqpoint{5.536005in}{0.646329in}}%
\pgfpathlineto{\pgfqpoint{5.536677in}{0.646329in}}%
\pgfpathlineto{\pgfqpoint{5.537349in}{0.670181in}}%
\pgfpathlineto{\pgfqpoint{5.538021in}{0.659580in}}%
\pgfpathlineto{\pgfqpoint{5.539366in}{0.659580in}}%
\pgfpathlineto{\pgfqpoint{5.540038in}{0.662230in}}%
\pgfpathlineto{\pgfqpoint{5.540710in}{0.648979in}}%
\pgfpathlineto{\pgfqpoint{5.541382in}{0.648979in}}%
\pgfpathlineto{\pgfqpoint{5.542054in}{0.638378in}}%
\pgfpathlineto{\pgfqpoint{5.542727in}{0.664880in}}%
\pgfpathlineto{\pgfqpoint{5.543399in}{0.664880in}}%
\pgfpathlineto{\pgfqpoint{5.543399in}{0.651629in}}%
\pgfpathlineto{\pgfqpoint{5.544743in}{0.656930in}}%
\pgfpathlineto{\pgfqpoint{5.545415in}{0.656930in}}%
\pgfpathlineto{\pgfqpoint{5.546087in}{0.670181in}}%
\pgfpathlineto{\pgfqpoint{5.546760in}{0.667530in}}%
\pgfpathlineto{\pgfqpoint{5.547432in}{0.667530in}}%
\pgfpathlineto{\pgfqpoint{5.547432in}{0.654280in}}%
\pgfpathlineto{\pgfqpoint{5.548104in}{0.672831in}}%
\pgfpathlineto{\pgfqpoint{5.548776in}{0.664880in}}%
\pgfpathlineto{\pgfqpoint{5.550120in}{0.664880in}}%
\pgfpathlineto{\pgfqpoint{5.550120in}{0.659580in}}%
\pgfpathlineto{\pgfqpoint{5.551465in}{0.659580in}}%
\pgfpathlineto{\pgfqpoint{5.552137in}{0.659580in}}%
\pgfpathlineto{\pgfqpoint{5.552137in}{0.675481in}}%
\pgfpathlineto{\pgfqpoint{5.552809in}{0.651629in}}%
\pgfpathlineto{\pgfqpoint{5.553481in}{0.662230in}}%
\pgfpathlineto{\pgfqpoint{5.554153in}{0.662230in}}%
\pgfpathlineto{\pgfqpoint{5.554153in}{0.648979in}}%
\pgfpathlineto{\pgfqpoint{5.555498in}{0.670181in}}%
\pgfpathlineto{\pgfqpoint{5.556170in}{0.670181in}}%
\pgfpathlineto{\pgfqpoint{5.556842in}{0.672831in}}%
\pgfpathlineto{\pgfqpoint{5.557514in}{0.659580in}}%
\pgfpathlineto{\pgfqpoint{5.558186in}{0.659580in}}%
\pgfpathlineto{\pgfqpoint{5.558859in}{0.648979in}}%
\pgfpathlineto{\pgfqpoint{5.559531in}{0.664880in}}%
\pgfpathlineto{\pgfqpoint{5.560203in}{0.664880in}}%
\pgfpathlineto{\pgfqpoint{5.560875in}{0.641029in}}%
\pgfpathlineto{\pgfqpoint{5.561547in}{0.648979in}}%
\pgfpathlineto{\pgfqpoint{5.562219in}{0.648979in}}%
\pgfpathlineto{\pgfqpoint{5.563564in}{0.662230in}}%
\pgfpathlineto{\pgfqpoint{5.564908in}{0.662230in}}%
\pgfpathlineto{\pgfqpoint{5.564908in}{0.659580in}}%
\pgfpathlineto{\pgfqpoint{5.566252in}{0.659580in}}%
\pgfpathlineto{\pgfqpoint{5.566925in}{0.659580in}}%
\pgfpathlineto{\pgfqpoint{5.566925in}{0.646329in}}%
\pgfpathlineto{\pgfqpoint{5.568269in}{0.656930in}}%
\pgfpathlineto{\pgfqpoint{5.568941in}{0.656930in}}%
\pgfpathlineto{\pgfqpoint{5.568941in}{0.648979in}}%
\pgfpathlineto{\pgfqpoint{5.570285in}{0.648979in}}%
\pgfpathlineto{\pgfqpoint{5.571630in}{0.648979in}}%
\pgfpathlineto{\pgfqpoint{5.572302in}{0.643679in}}%
\pgfpathlineto{\pgfqpoint{5.572974in}{0.664880in}}%
\pgfpathlineto{\pgfqpoint{5.573646in}{0.664880in}}%
\pgfpathlineto{\pgfqpoint{5.574991in}{0.648979in}}%
\pgfpathlineto{\pgfqpoint{5.575663in}{0.648979in}}%
\pgfpathlineto{\pgfqpoint{5.575663in}{0.664880in}}%
\pgfpathlineto{\pgfqpoint{5.576335in}{0.643679in}}%
\pgfpathlineto{\pgfqpoint{5.577007in}{0.659580in}}%
\pgfpathlineto{\pgfqpoint{5.578351in}{0.659580in}}%
\pgfpathlineto{\pgfqpoint{5.579024in}{0.654280in}}%
\pgfpathlineto{\pgfqpoint{5.579696in}{0.664880in}}%
\pgfpathlineto{\pgfqpoint{5.580368in}{0.664880in}}%
\pgfpathlineto{\pgfqpoint{5.581712in}{0.643679in}}%
\pgfpathlineto{\pgfqpoint{5.582384in}{0.643679in}}%
\pgfpathlineto{\pgfqpoint{5.583729in}{0.667530in}}%
\pgfpathlineto{\pgfqpoint{5.585073in}{0.667530in}}%
\pgfpathlineto{\pgfqpoint{5.585073in}{0.646329in}}%
\pgfpathlineto{\pgfqpoint{5.586417in}{0.646329in}}%
\pgfpathlineto{\pgfqpoint{5.587090in}{0.646329in}}%
\pgfpathlineto{\pgfqpoint{5.587762in}{0.659580in}}%
\pgfpathlineto{\pgfqpoint{5.588434in}{0.648979in}}%
\pgfpathlineto{\pgfqpoint{5.589106in}{0.648979in}}%
\pgfpathlineto{\pgfqpoint{5.589106in}{0.646329in}}%
\pgfpathlineto{\pgfqpoint{5.590450in}{0.672831in}}%
\pgfpathlineto{\pgfqpoint{5.591123in}{0.672831in}}%
\pgfpathlineto{\pgfqpoint{5.591123in}{0.659580in}}%
\pgfpathlineto{\pgfqpoint{5.592467in}{0.659580in}}%
\pgfpathlineto{\pgfqpoint{5.593139in}{0.659580in}}%
\pgfpathlineto{\pgfqpoint{5.593139in}{0.667530in}}%
\pgfpathlineto{\pgfqpoint{5.594483in}{0.648979in}}%
\pgfpathlineto{\pgfqpoint{5.595156in}{0.648979in}}%
\pgfpathlineto{\pgfqpoint{5.595156in}{0.654280in}}%
\pgfpathlineto{\pgfqpoint{5.596500in}{0.651629in}}%
\pgfpathlineto{\pgfqpoint{5.597172in}{0.651629in}}%
\pgfpathlineto{\pgfqpoint{5.597172in}{0.659580in}}%
\pgfpathlineto{\pgfqpoint{5.598516in}{0.641029in}}%
\pgfpathlineto{\pgfqpoint{5.599189in}{0.641029in}}%
\pgfpathlineto{\pgfqpoint{5.600533in}{0.656930in}}%
\pgfpathlineto{\pgfqpoint{5.601205in}{0.656930in}}%
\pgfpathlineto{\pgfqpoint{5.601205in}{0.667530in}}%
\pgfpathlineto{\pgfqpoint{5.602549in}{0.648979in}}%
\pgfpathlineto{\pgfqpoint{5.603222in}{0.648979in}}%
\pgfpathlineto{\pgfqpoint{5.603222in}{0.664880in}}%
\pgfpathlineto{\pgfqpoint{5.604566in}{0.651629in}}%
\pgfpathlineto{\pgfqpoint{5.605238in}{0.651629in}}%
\pgfpathlineto{\pgfqpoint{5.605910in}{0.659580in}}%
\pgfpathlineto{\pgfqpoint{5.606582in}{0.641029in}}%
\pgfpathlineto{\pgfqpoint{5.607255in}{0.641029in}}%
\pgfpathlineto{\pgfqpoint{5.607255in}{0.659580in}}%
\pgfpathlineto{\pgfqpoint{5.608599in}{0.651629in}}%
\pgfpathlineto{\pgfqpoint{5.609271in}{0.651629in}}%
\pgfpathlineto{\pgfqpoint{5.609271in}{0.646329in}}%
\pgfpathlineto{\pgfqpoint{5.610615in}{0.672831in}}%
\pgfpathlineto{\pgfqpoint{5.611288in}{0.672831in}}%
\pgfpathlineto{\pgfqpoint{5.612632in}{0.654280in}}%
\pgfpathlineto{\pgfqpoint{5.613976in}{0.654280in}}%
\pgfpathlineto{\pgfqpoint{5.614648in}{0.667530in}}%
\pgfpathlineto{\pgfqpoint{5.615321in}{0.662230in}}%
\pgfpathlineto{\pgfqpoint{5.615993in}{0.662230in}}%
\pgfpathlineto{\pgfqpoint{5.617337in}{0.646329in}}%
\pgfpathlineto{\pgfqpoint{5.618009in}{0.646329in}}%
\pgfpathlineto{\pgfqpoint{5.618681in}{0.662230in}}%
\pgfpathlineto{\pgfqpoint{5.619354in}{0.651629in}}%
\pgfpathlineto{\pgfqpoint{5.620026in}{0.651629in}}%
\pgfpathlineto{\pgfqpoint{5.620026in}{0.648979in}}%
\pgfpathlineto{\pgfqpoint{5.621370in}{0.672831in}}%
\pgfpathlineto{\pgfqpoint{5.622042in}{0.672831in}}%
\pgfpathlineto{\pgfqpoint{5.622042in}{0.638378in}}%
\pgfpathlineto{\pgfqpoint{5.623387in}{0.654280in}}%
\pgfpathlineto{\pgfqpoint{5.624059in}{0.654280in}}%
\pgfpathlineto{\pgfqpoint{5.624059in}{0.651629in}}%
\pgfpathlineto{\pgfqpoint{5.625403in}{0.656930in}}%
\pgfpathlineto{\pgfqpoint{5.626075in}{0.656930in}}%
\pgfpathlineto{\pgfqpoint{5.626075in}{0.646329in}}%
\pgfpathlineto{\pgfqpoint{5.627420in}{0.648979in}}%
\pgfpathlineto{\pgfqpoint{5.628092in}{0.648979in}}%
\pgfpathlineto{\pgfqpoint{5.628764in}{0.667530in}}%
\pgfpathlineto{\pgfqpoint{5.629436in}{0.664880in}}%
\pgfpathlineto{\pgfqpoint{5.630108in}{0.664880in}}%
\pgfpathlineto{\pgfqpoint{5.630781in}{0.667530in}}%
\pgfpathlineto{\pgfqpoint{5.631453in}{0.656930in}}%
\pgfpathlineto{\pgfqpoint{5.632125in}{0.656930in}}%
\pgfpathlineto{\pgfqpoint{5.632125in}{0.648979in}}%
\pgfpathlineto{\pgfqpoint{5.633469in}{0.654280in}}%
\pgfpathlineto{\pgfqpoint{5.634141in}{0.654280in}}%
\pgfpathlineto{\pgfqpoint{5.634141in}{0.659580in}}%
\pgfpathlineto{\pgfqpoint{5.634814in}{0.648979in}}%
\pgfpathlineto{\pgfqpoint{5.635486in}{0.654280in}}%
\pgfpathlineto{\pgfqpoint{5.636158in}{0.654280in}}%
\pgfpathlineto{\pgfqpoint{5.636158in}{0.667530in}}%
\pgfpathlineto{\pgfqpoint{5.636830in}{0.648979in}}%
\pgfpathlineto{\pgfqpoint{5.637502in}{0.654280in}}%
\pgfpathlineto{\pgfqpoint{5.638174in}{0.654280in}}%
\pgfpathlineto{\pgfqpoint{5.638174in}{0.643679in}}%
\pgfpathlineto{\pgfqpoint{5.638847in}{0.670181in}}%
\pgfpathlineto{\pgfqpoint{5.639519in}{0.648979in}}%
\pgfpathlineto{\pgfqpoint{5.640191in}{0.648979in}}%
\pgfpathlineto{\pgfqpoint{5.640191in}{0.643679in}}%
\pgfpathlineto{\pgfqpoint{5.640863in}{0.651629in}}%
\pgfpathlineto{\pgfqpoint{5.641535in}{0.648979in}}%
\pgfpathlineto{\pgfqpoint{5.643552in}{0.648979in}}%
\pgfpathlineto{\pgfqpoint{5.643552in}{0.659580in}}%
\pgfpathlineto{\pgfqpoint{5.644896in}{0.646329in}}%
\pgfpathlineto{\pgfqpoint{5.645568in}{0.646329in}}%
\pgfpathlineto{\pgfqpoint{5.646913in}{0.659580in}}%
\pgfpathlineto{\pgfqpoint{5.647585in}{0.659580in}}%
\pgfpathlineto{\pgfqpoint{5.647585in}{0.651629in}}%
\pgfpathlineto{\pgfqpoint{5.648929in}{0.659580in}}%
\pgfpathlineto{\pgfqpoint{5.649601in}{0.659580in}}%
\pgfpathlineto{\pgfqpoint{5.649601in}{0.656930in}}%
\pgfpathlineto{\pgfqpoint{5.650946in}{0.656930in}}%
\pgfpathlineto{\pgfqpoint{5.651618in}{0.656930in}}%
\pgfpathlineto{\pgfqpoint{5.651618in}{0.638378in}}%
\pgfpathlineto{\pgfqpoint{5.652962in}{0.672831in}}%
\pgfpathlineto{\pgfqpoint{5.653634in}{0.672831in}}%
\pgfpathlineto{\pgfqpoint{5.653634in}{0.651629in}}%
\pgfpathlineto{\pgfqpoint{5.654979in}{0.651629in}}%
\pgfpathlineto{\pgfqpoint{5.655651in}{0.651629in}}%
\pgfpathlineto{\pgfqpoint{5.655651in}{0.667530in}}%
\pgfpathlineto{\pgfqpoint{5.656995in}{0.654280in}}%
\pgfpathlineto{\pgfqpoint{5.657667in}{0.654280in}}%
\pgfpathlineto{\pgfqpoint{5.658339in}{0.648979in}}%
\pgfpathlineto{\pgfqpoint{5.659012in}{0.659580in}}%
\pgfpathlineto{\pgfqpoint{5.659684in}{0.659580in}}%
\pgfpathlineto{\pgfqpoint{5.660356in}{0.651629in}}%
\pgfpathlineto{\pgfqpoint{5.661028in}{0.654280in}}%
\pgfpathlineto{\pgfqpoint{5.661700in}{0.654280in}}%
\pgfpathlineto{\pgfqpoint{5.662372in}{0.678131in}}%
\pgfpathlineto{\pgfqpoint{5.663045in}{0.648979in}}%
\pgfpathlineto{\pgfqpoint{5.663717in}{0.648979in}}%
\pgfpathlineto{\pgfqpoint{5.663717in}{0.675481in}}%
\pgfpathlineto{\pgfqpoint{5.665061in}{0.662230in}}%
\pgfpathlineto{\pgfqpoint{5.665733in}{0.662230in}}%
\pgfpathlineto{\pgfqpoint{5.665733in}{0.646329in}}%
\pgfpathlineto{\pgfqpoint{5.666405in}{0.672831in}}%
\pgfpathlineto{\pgfqpoint{5.667078in}{0.664880in}}%
\pgfpathlineto{\pgfqpoint{5.667750in}{0.664880in}}%
\pgfpathlineto{\pgfqpoint{5.667750in}{0.643679in}}%
\pgfpathlineto{\pgfqpoint{5.668422in}{0.678131in}}%
\pgfpathlineto{\pgfqpoint{5.669094in}{0.664880in}}%
\pgfpathlineto{\pgfqpoint{5.669766in}{0.664880in}}%
\pgfpathlineto{\pgfqpoint{5.669766in}{0.656930in}}%
\pgfpathlineto{\pgfqpoint{5.671111in}{0.670181in}}%
\pgfpathlineto{\pgfqpoint{5.671783in}{0.670181in}}%
\pgfpathlineto{\pgfqpoint{5.672455in}{0.675481in}}%
\pgfpathlineto{\pgfqpoint{5.673127in}{0.651629in}}%
\pgfpathlineto{\pgfqpoint{5.673799in}{0.651629in}}%
\pgfpathlineto{\pgfqpoint{5.674471in}{0.662230in}}%
\pgfpathlineto{\pgfqpoint{5.675144in}{0.656930in}}%
\pgfpathlineto{\pgfqpoint{5.675816in}{0.656930in}}%
\pgfpathlineto{\pgfqpoint{5.675816in}{0.646329in}}%
\pgfpathlineto{\pgfqpoint{5.677160in}{0.656930in}}%
\pgfpathlineto{\pgfqpoint{5.677832in}{0.656930in}}%
\pgfpathlineto{\pgfqpoint{5.678504in}{0.675481in}}%
\pgfpathlineto{\pgfqpoint{5.679177in}{0.670181in}}%
\pgfpathlineto{\pgfqpoint{5.679849in}{0.670181in}}%
\pgfpathlineto{\pgfqpoint{5.679849in}{0.643679in}}%
\pgfpathlineto{\pgfqpoint{5.681193in}{0.648979in}}%
\pgfpathlineto{\pgfqpoint{5.681865in}{0.648979in}}%
\pgfpathlineto{\pgfqpoint{5.681865in}{0.664880in}}%
\pgfpathlineto{\pgfqpoint{5.682537in}{0.646329in}}%
\pgfpathlineto{\pgfqpoint{5.683210in}{0.656930in}}%
\pgfpathlineto{\pgfqpoint{5.683882in}{0.656930in}}%
\pgfpathlineto{\pgfqpoint{5.683882in}{0.667530in}}%
\pgfpathlineto{\pgfqpoint{5.685226in}{0.667530in}}%
\pgfpathlineto{\pgfqpoint{5.685898in}{0.667530in}}%
\pgfpathlineto{\pgfqpoint{5.686570in}{0.648979in}}%
\pgfpathlineto{\pgfqpoint{5.687243in}{0.648979in}}%
\pgfpathlineto{\pgfqpoint{5.687915in}{0.648979in}}%
\pgfpathlineto{\pgfqpoint{5.687915in}{0.667530in}}%
\pgfpathlineto{\pgfqpoint{5.689259in}{0.651629in}}%
\pgfpathlineto{\pgfqpoint{5.689931in}{0.651629in}}%
\pgfpathlineto{\pgfqpoint{5.689931in}{0.675481in}}%
\pgfpathlineto{\pgfqpoint{5.690603in}{0.648979in}}%
\pgfpathlineto{\pgfqpoint{5.691276in}{0.664880in}}%
\pgfpathlineto{\pgfqpoint{5.691948in}{0.664880in}}%
\pgfpathlineto{\pgfqpoint{5.692620in}{0.641029in}}%
\pgfpathlineto{\pgfqpoint{5.693292in}{0.641029in}}%
\pgfpathlineto{\pgfqpoint{5.693964in}{0.641029in}}%
\pgfpathlineto{\pgfqpoint{5.694636in}{0.664880in}}%
\pgfpathlineto{\pgfqpoint{5.695309in}{0.659580in}}%
\pgfpathlineto{\pgfqpoint{5.695981in}{0.659580in}}%
\pgfpathlineto{\pgfqpoint{5.696653in}{0.648979in}}%
\pgfpathlineto{\pgfqpoint{5.697325in}{0.654280in}}%
\pgfpathlineto{\pgfqpoint{5.697997in}{0.654280in}}%
\pgfpathlineto{\pgfqpoint{5.697997in}{0.672831in}}%
\pgfpathlineto{\pgfqpoint{5.699342in}{0.643679in}}%
\pgfpathlineto{\pgfqpoint{5.700014in}{0.643679in}}%
\pgfpathlineto{\pgfqpoint{5.701358in}{0.662230in}}%
\pgfpathlineto{\pgfqpoint{5.702702in}{0.662230in}}%
\pgfpathlineto{\pgfqpoint{5.704047in}{0.643679in}}%
\pgfpathlineto{\pgfqpoint{5.704719in}{0.643679in}}%
\pgfpathlineto{\pgfqpoint{5.706063in}{0.656930in}}%
\pgfpathlineto{\pgfqpoint{5.706735in}{0.656930in}}%
\pgfpathlineto{\pgfqpoint{5.707408in}{0.664880in}}%
\pgfpathlineto{\pgfqpoint{5.708080in}{0.654280in}}%
\pgfpathlineto{\pgfqpoint{5.708752in}{0.654280in}}%
\pgfpathlineto{\pgfqpoint{5.708752in}{0.664880in}}%
\pgfpathlineto{\pgfqpoint{5.710096in}{0.654280in}}%
\pgfpathlineto{\pgfqpoint{5.711441in}{0.654280in}}%
\pgfpathlineto{\pgfqpoint{5.712113in}{0.659580in}}%
\pgfpathlineto{\pgfqpoint{5.712785in}{0.638378in}}%
\pgfpathlineto{\pgfqpoint{5.713457in}{0.638378in}}%
\pgfpathlineto{\pgfqpoint{5.714129in}{0.664880in}}%
\pgfpathlineto{\pgfqpoint{5.714801in}{0.662230in}}%
\pgfpathlineto{\pgfqpoint{5.716146in}{0.662230in}}%
\pgfpathlineto{\pgfqpoint{5.716818in}{0.654280in}}%
\pgfpathlineto{\pgfqpoint{5.717490in}{0.654280in}}%
\pgfpathlineto{\pgfqpoint{5.718162in}{0.654280in}}%
\pgfpathlineto{\pgfqpoint{5.718162in}{0.641029in}}%
\pgfpathlineto{\pgfqpoint{5.718834in}{0.675481in}}%
\pgfpathlineto{\pgfqpoint{5.719507in}{0.651629in}}%
\pgfpathlineto{\pgfqpoint{5.720851in}{0.651629in}}%
\pgfpathlineto{\pgfqpoint{5.721523in}{0.659580in}}%
\pgfpathlineto{\pgfqpoint{5.722195in}{0.646329in}}%
\pgfpathlineto{\pgfqpoint{5.722867in}{0.646329in}}%
\pgfpathlineto{\pgfqpoint{5.723540in}{0.656930in}}%
\pgfpathlineto{\pgfqpoint{5.724212in}{0.648979in}}%
\pgfpathlineto{\pgfqpoint{5.724884in}{0.648979in}}%
\pgfpathlineto{\pgfqpoint{5.725556in}{0.646329in}}%
\pgfpathlineto{\pgfqpoint{5.726228in}{0.656930in}}%
\pgfpathlineto{\pgfqpoint{5.727573in}{0.656930in}}%
\pgfpathlineto{\pgfqpoint{5.727573in}{0.664880in}}%
\pgfpathlineto{\pgfqpoint{5.728917in}{0.664880in}}%
\pgfpathlineto{\pgfqpoint{5.729589in}{0.664880in}}%
\pgfpathlineto{\pgfqpoint{5.730261in}{0.646329in}}%
\pgfpathlineto{\pgfqpoint{5.730933in}{0.659580in}}%
\pgfpathlineto{\pgfqpoint{5.731606in}{0.659580in}}%
\pgfpathlineto{\pgfqpoint{5.731606in}{0.662230in}}%
\pgfpathlineto{\pgfqpoint{5.732950in}{0.641029in}}%
\pgfpathlineto{\pgfqpoint{5.733622in}{0.641029in}}%
\pgfpathlineto{\pgfqpoint{5.733622in}{0.654280in}}%
\pgfpathlineto{\pgfqpoint{5.734967in}{0.646329in}}%
\pgfpathlineto{\pgfqpoint{5.735639in}{0.646329in}}%
\pgfpathlineto{\pgfqpoint{5.736311in}{0.672831in}}%
\pgfpathlineto{\pgfqpoint{5.736983in}{0.656930in}}%
\pgfpathlineto{\pgfqpoint{5.738327in}{0.656930in}}%
\pgfpathlineto{\pgfqpoint{5.739672in}{0.664880in}}%
\pgfpathlineto{\pgfqpoint{5.740344in}{0.664880in}}%
\pgfpathlineto{\pgfqpoint{5.741688in}{0.641029in}}%
\pgfpathlineto{\pgfqpoint{5.742360in}{0.641029in}}%
\pgfpathlineto{\pgfqpoint{5.742360in}{0.662230in}}%
\pgfpathlineto{\pgfqpoint{5.743033in}{0.638378in}}%
\pgfpathlineto{\pgfqpoint{5.743705in}{0.648979in}}%
\pgfpathlineto{\pgfqpoint{5.744377in}{0.648979in}}%
\pgfpathlineto{\pgfqpoint{5.745049in}{0.646329in}}%
\pgfpathlineto{\pgfqpoint{5.745721in}{0.667530in}}%
\pgfpathlineto{\pgfqpoint{5.746393in}{0.667530in}}%
\pgfpathlineto{\pgfqpoint{5.747066in}{0.641029in}}%
\pgfpathlineto{\pgfqpoint{5.747738in}{0.659580in}}%
\pgfpathlineto{\pgfqpoint{5.748410in}{0.659580in}}%
\pgfpathlineto{\pgfqpoint{5.748410in}{0.648979in}}%
\pgfpathlineto{\pgfqpoint{5.749754in}{0.664880in}}%
\pgfpathlineto{\pgfqpoint{5.750426in}{0.664880in}}%
\pgfpathlineto{\pgfqpoint{5.751771in}{0.656930in}}%
\pgfpathlineto{\pgfqpoint{5.752443in}{0.656930in}}%
\pgfpathlineto{\pgfqpoint{5.753115in}{0.648979in}}%
\pgfpathlineto{\pgfqpoint{5.753787in}{0.654280in}}%
\pgfpathlineto{\pgfqpoint{5.754459in}{0.654280in}}%
\pgfpathlineto{\pgfqpoint{5.755132in}{0.651629in}}%
\pgfpathlineto{\pgfqpoint{5.755804in}{0.667530in}}%
\pgfpathlineto{\pgfqpoint{5.756476in}{0.667530in}}%
\pgfpathlineto{\pgfqpoint{5.756476in}{0.656930in}}%
\pgfpathlineto{\pgfqpoint{5.757820in}{0.662230in}}%
\pgfpathlineto{\pgfqpoint{5.758492in}{0.662230in}}%
\pgfpathlineto{\pgfqpoint{5.759837in}{0.646329in}}%
\pgfpathlineto{\pgfqpoint{5.760509in}{0.646329in}}%
\pgfpathlineto{\pgfqpoint{5.761853in}{0.670181in}}%
\pgfpathlineto{\pgfqpoint{5.762525in}{0.670181in}}%
\pgfpathlineto{\pgfqpoint{5.762525in}{0.641029in}}%
\pgfpathlineto{\pgfqpoint{5.763870in}{0.651629in}}%
\pgfpathlineto{\pgfqpoint{5.764542in}{0.651629in}}%
\pgfpathlineto{\pgfqpoint{5.764542in}{0.656930in}}%
\pgfpathlineto{\pgfqpoint{5.765886in}{0.643679in}}%
\pgfpathlineto{\pgfqpoint{5.766558in}{0.643679in}}%
\pgfpathlineto{\pgfqpoint{5.767231in}{0.664880in}}%
\pgfpathlineto{\pgfqpoint{5.767903in}{0.662230in}}%
\pgfpathlineto{\pgfqpoint{5.768575in}{0.662230in}}%
\pgfpathlineto{\pgfqpoint{5.768575in}{0.641029in}}%
\pgfpathlineto{\pgfqpoint{5.769919in}{0.651629in}}%
\pgfpathlineto{\pgfqpoint{5.770591in}{0.651629in}}%
\pgfpathlineto{\pgfqpoint{5.771936in}{0.662230in}}%
\pgfpathlineto{\pgfqpoint{5.772608in}{0.662230in}}%
\pgfpathlineto{\pgfqpoint{5.773280in}{0.648979in}}%
\pgfpathlineto{\pgfqpoint{5.773952in}{0.662230in}}%
\pgfpathlineto{\pgfqpoint{5.774624in}{0.662230in}}%
\pgfpathlineto{\pgfqpoint{5.775969in}{0.646329in}}%
\pgfpathlineto{\pgfqpoint{5.776641in}{0.646329in}}%
\pgfpathlineto{\pgfqpoint{5.777985in}{0.659580in}}%
\pgfpathlineto{\pgfqpoint{5.778657in}{0.659580in}}%
\pgfpathlineto{\pgfqpoint{5.779330in}{0.651629in}}%
\pgfpathlineto{\pgfqpoint{5.780002in}{0.656930in}}%
\pgfpathlineto{\pgfqpoint{5.780674in}{0.656930in}}%
\pgfpathlineto{\pgfqpoint{5.780674in}{0.654280in}}%
\pgfpathlineto{\pgfqpoint{5.782018in}{0.656930in}}%
\pgfpathlineto{\pgfqpoint{5.782690in}{0.656930in}}%
\pgfpathlineto{\pgfqpoint{5.782690in}{0.646329in}}%
\pgfpathlineto{\pgfqpoint{5.784035in}{0.670181in}}%
\pgfpathlineto{\pgfqpoint{5.784707in}{0.670181in}}%
\pgfpathlineto{\pgfqpoint{5.786051in}{0.643679in}}%
\pgfpathlineto{\pgfqpoint{5.786723in}{0.643679in}}%
\pgfpathlineto{\pgfqpoint{5.788068in}{0.672831in}}%
\pgfpathlineto{\pgfqpoint{5.788740in}{0.672831in}}%
\pgfpathlineto{\pgfqpoint{5.789412in}{0.646329in}}%
\pgfpathlineto{\pgfqpoint{5.790084in}{0.656930in}}%
\pgfpathlineto{\pgfqpoint{5.790756in}{0.656930in}}%
\pgfpathlineto{\pgfqpoint{5.790756in}{0.641029in}}%
\pgfpathlineto{\pgfqpoint{5.791429in}{0.659580in}}%
\pgfpathlineto{\pgfqpoint{5.792101in}{0.648979in}}%
\pgfpathlineto{\pgfqpoint{5.792773in}{0.648979in}}%
\pgfpathlineto{\pgfqpoint{5.792773in}{0.664880in}}%
\pgfpathlineto{\pgfqpoint{5.793445in}{0.643679in}}%
\pgfpathlineto{\pgfqpoint{5.794117in}{0.648979in}}%
\pgfpathlineto{\pgfqpoint{5.794789in}{0.648979in}}%
\pgfpathlineto{\pgfqpoint{5.795462in}{0.662230in}}%
\pgfpathlineto{\pgfqpoint{5.796134in}{0.646329in}}%
\pgfpathlineto{\pgfqpoint{5.796806in}{0.646329in}}%
\pgfpathlineto{\pgfqpoint{5.798150in}{0.656930in}}%
\pgfpathlineto{\pgfqpoint{5.798822in}{0.656930in}}%
\pgfpathlineto{\pgfqpoint{5.799495in}{0.662230in}}%
\pgfpathlineto{\pgfqpoint{5.800167in}{0.648979in}}%
\pgfpathlineto{\pgfqpoint{5.800839in}{0.648979in}}%
\pgfpathlineto{\pgfqpoint{5.800839in}{0.641029in}}%
\pgfpathlineto{\pgfqpoint{5.801511in}{0.641029in}}%
\pgfusepath{stroke}%
\end{pgfscope}%
\begin{pgfscope}%
\pgfsetrectcap%
\pgfsetmiterjoin%
\pgfsetlinewidth{0.803000pt}%
\definecolor{currentstroke}{rgb}{0.000000,0.000000,0.000000}%
\pgfsetstrokecolor{currentstroke}%
\pgfsetdash{}{0pt}%
\pgfpathmoveto{\pgfqpoint{3.662674in}{0.552778in}}%
\pgfpathlineto{\pgfqpoint{3.662674in}{2.202778in}}%
\pgfusepath{stroke}%
\end{pgfscope}%
\begin{pgfscope}%
\pgfsetrectcap%
\pgfsetmiterjoin%
\pgfsetlinewidth{0.803000pt}%
\definecolor{currentstroke}{rgb}{0.000000,0.000000,0.000000}%
\pgfsetstrokecolor{currentstroke}%
\pgfsetdash{}{0pt}%
\pgfpathmoveto{\pgfqpoint{5.801389in}{0.552778in}}%
\pgfpathlineto{\pgfqpoint{5.801389in}{2.202778in}}%
\pgfusepath{stroke}%
\end{pgfscope}%
\begin{pgfscope}%
\pgfsetrectcap%
\pgfsetmiterjoin%
\pgfsetlinewidth{0.803000pt}%
\definecolor{currentstroke}{rgb}{0.000000,0.000000,0.000000}%
\pgfsetstrokecolor{currentstroke}%
\pgfsetdash{}{0pt}%
\pgfpathmoveto{\pgfqpoint{3.662674in}{0.552778in}}%
\pgfpathlineto{\pgfqpoint{5.801389in}{0.552778in}}%
\pgfusepath{stroke}%
\end{pgfscope}%
\begin{pgfscope}%
\pgfsetrectcap%
\pgfsetmiterjoin%
\pgfsetlinewidth{0.803000pt}%
\definecolor{currentstroke}{rgb}{0.000000,0.000000,0.000000}%
\pgfsetstrokecolor{currentstroke}%
\pgfsetdash{}{0pt}%
\pgfpathmoveto{\pgfqpoint{3.662674in}{2.202778in}}%
\pgfpathlineto{\pgfqpoint{5.801389in}{2.202778in}}%
\pgfusepath{stroke}%
\end{pgfscope}%
\begin{pgfscope}%
\definecolor{textcolor}{rgb}{0.000000,0.000000,0.000000}%
\pgfsetstrokecolor{textcolor}%
\pgfsetfillcolor{textcolor}%
\pgftext[x=4.732031in,y=2.286111in,,base]{\color{textcolor}\rmfamily\fontsize{12.000000}{14.400000}\selectfont Energiespektrum B}%
\end{pgfscope}%
\begin{pgfscope}%
\pgfsetbuttcap%
\pgfsetmiterjoin%
\definecolor{currentfill}{rgb}{1.000000,1.000000,1.000000}%
\pgfsetfillcolor{currentfill}%
\pgfsetfillopacity{0.800000}%
\pgfsetlinewidth{1.003750pt}%
\definecolor{currentstroke}{rgb}{0.800000,0.800000,0.800000}%
\pgfsetstrokecolor{currentstroke}%
\pgfsetstrokeopacity{0.800000}%
\pgfsetdash{}{0pt}%
\pgfpathmoveto{\pgfqpoint{3.759896in}{1.510834in}}%
\pgfpathlineto{\pgfqpoint{4.608090in}{1.510834in}}%
\pgfpathquadraticcurveto{\pgfqpoint{4.635868in}{1.510834in}}{\pgfqpoint{4.635868in}{1.538612in}}%
\pgfpathlineto{\pgfqpoint{4.635868in}{2.105556in}}%
\pgfpathquadraticcurveto{\pgfqpoint{4.635868in}{2.133333in}}{\pgfqpoint{4.608090in}{2.133333in}}%
\pgfpathlineto{\pgfqpoint{3.759896in}{2.133333in}}%
\pgfpathquadraticcurveto{\pgfqpoint{3.732118in}{2.133333in}}{\pgfqpoint{3.732118in}{2.105556in}}%
\pgfpathlineto{\pgfqpoint{3.732118in}{1.538612in}}%
\pgfpathquadraticcurveto{\pgfqpoint{3.732118in}{1.510834in}}{\pgfqpoint{3.759896in}{1.510834in}}%
\pgfpathclose%
\pgfusepath{stroke,fill}%
\end{pgfscope}%
\begin{pgfscope}%
\pgfsetrectcap%
\pgfsetroundjoin%
\pgfsetlinewidth{1.505625pt}%
\definecolor{currentstroke}{rgb}{0.121569,0.466667,0.705882}%
\pgfsetstrokecolor{currentstroke}%
\pgfsetdash{}{0pt}%
\pgfpathmoveto{\pgfqpoint{3.787674in}{2.029167in}}%
\pgfpathlineto{\pgfqpoint{4.065451in}{2.029167in}}%
\pgfusepath{stroke}%
\end{pgfscope}%
\begin{pgfscope}%
\definecolor{textcolor}{rgb}{0.000000,0.000000,0.000000}%
\pgfsetstrokecolor{textcolor}%
\pgfsetfillcolor{textcolor}%
\pgftext[x=4.176563in,y=1.980556in,left,base]{\color{textcolor}\rmfamily\fontsize{10.000000}{12.000000}\selectfont Mitte}%
\end{pgfscope}%
\begin{pgfscope}%
\pgfsetrectcap%
\pgfsetroundjoin%
\pgfsetlinewidth{1.505625pt}%
\definecolor{currentstroke}{rgb}{1.000000,0.498039,0.054902}%
\pgfsetstrokecolor{currentstroke}%
\pgfsetstrokeopacity{0.800000}%
\pgfsetdash{}{0pt}%
\pgfpathmoveto{\pgfqpoint{3.787674in}{1.835556in}}%
\pgfpathlineto{\pgfqpoint{4.065451in}{1.835556in}}%
\pgfusepath{stroke}%
\end{pgfscope}%
\begin{pgfscope}%
\definecolor{textcolor}{rgb}{0.000000,0.000000,0.000000}%
\pgfsetstrokecolor{textcolor}%
\pgfsetfillcolor{textcolor}%
\pgftext[x=4.176563in,y=1.786945in,left,base]{\color{textcolor}\rmfamily\fontsize{10.000000}{12.000000}\selectfont Rechts}%
\end{pgfscope}%
\begin{pgfscope}%
\pgfsetrectcap%
\pgfsetroundjoin%
\pgfsetlinewidth{1.505625pt}%
\definecolor{currentstroke}{rgb}{0.172549,0.627451,0.172549}%
\pgfsetstrokecolor{currentstroke}%
\pgfsetstrokeopacity{0.800000}%
\pgfsetdash{}{0pt}%
\pgfpathmoveto{\pgfqpoint{3.787674in}{1.641945in}}%
\pgfpathlineto{\pgfqpoint{4.065451in}{1.641945in}}%
\pgfusepath{stroke}%
\end{pgfscope}%
\begin{pgfscope}%
\definecolor{textcolor}{rgb}{0.000000,0.000000,0.000000}%
\pgfsetstrokecolor{textcolor}%
\pgfsetfillcolor{textcolor}%
\pgftext[x=4.176563in,y=1.593334in,left,base]{\color{textcolor}\rmfamily\fontsize{10.000000}{12.000000}\selectfont Links}%
\end{pgfscope}%
\end{pgfpicture}%
\makeatother%
\endgroup%

  \caption[Vergleich der Quellpositionen]{Zeit und Energiespektren f\"ur Verschiedene
    Quellpositionen. (Links \(=\) bei Detektor A), relevante Ausschnitte}
  \label{fig:calibration-comp}
\end{figure}

Die Z\"ahlrate ist generell f\"ur die mittlere Quellposition am
wenigsten breit verteilt, da die Anordnung so am symetrischsten- und
der l\"angste Detektorabstand am geringsten ist. Im Zeitspektrum sieht
man eine Verschiebung des Peaks entsprechend des Laufzeitunterschiedes
der Positionen. F\"ur den Vergleich der Energiespektra eignet sich
Detektor B am besten, da er der urspr\"unglichen konfiguration
entspricht. So erkennt man, das sich f\"ur die Linke Position (bei
Detektor A) entsprechend eine h\"ohere Z\"ahlrate f\"ur
niederenergetische Ereignisse bei Detektor B ergibt (Analog bei Det. A
f\"ur die rechte Position). Dies k\"onnte unter anderem von Streuueng
bedingt sein. Dabei ist die Verteilung des jeweils anderen Detektors
bei h\"oheren Energien angehoben (gleiche Gesamtereignisszahl,
Umverteilung entsprechend h\"oherer ungestreuter Energie).  Die Peaks
sind generell zu niedrigeren Energien verschoben.

Bildet man die Differenz der Peakpositionen im Zeitspektrum f\"ur die Linke und
die rechte Quellposition (mittlere wird bei Mittelung redundant) so
ergibt sich die Doppelte Lichtlaufzeit zwischen den Detektoren (\(t_0
+ \mathfrak{t} - (t_0
- \mathfrak{t}) = 2\mathfrak{t}\)).

Um die Peakpositionen zu erhalten, wurden Gaussfunktionen \"uber die
Zeitspektren gefittet. Als ma\ss{} f\"ur die Peakbreite und damit die
(statistische) Unsicherheit wurde \(1/10\) des \(\sigma\) Parameters
der Gaussfunktion genutzt (heuristische Absch\"atzung). Nicht explizit
betrachtet werden hier der Endliche Quellendurchmesser
\SI{2.5}{\centi\meter} sowie der Restabstand der Quelle zum Detektor
(kleiner \SI{1}{\centi\meter}). Der Prozess wird
in~\ref{fig:calibration-lenght_det} illustriert.

\begin{figure}[H]\centering
  %% Creator: Matplotlib, PGF backend
%%
%% To include the figure in your LaTeX document, write
%%   \input{<filename>.pgf}
%%
%% Make sure the required packages are loaded in your preamble
%%   \usepackage{pgf}
%%
%% Figures using additional raster images can only be included by \input if
%% they are in the same directory as the main LaTeX file. For loading figures
%% from other directories you can use the `import` package
%%   \usepackage{import}
%% and then include the figures with
%%   \import{<path to file>}{<filename>.pgf}
%%
%% Matplotlib used the following preamble
%%   \usepackage{fontspec}
%%
\begingroup%
\makeatletter%
\begin{pgfpicture}%
\pgfpathrectangle{\pgfpointorigin}{\pgfqpoint{5.000000in}{4.000000in}}%
\pgfusepath{use as bounding box, clip}%
\begin{pgfscope}%
\pgfsetbuttcap%
\pgfsetmiterjoin%
\definecolor{currentfill}{rgb}{1.000000,1.000000,1.000000}%
\pgfsetfillcolor{currentfill}%
\pgfsetlinewidth{0.000000pt}%
\definecolor{currentstroke}{rgb}{1.000000,1.000000,1.000000}%
\pgfsetstrokecolor{currentstroke}%
\pgfsetdash{}{0pt}%
\pgfpathmoveto{\pgfqpoint{0.000000in}{0.000000in}}%
\pgfpathlineto{\pgfqpoint{5.000000in}{0.000000in}}%
\pgfpathlineto{\pgfqpoint{5.000000in}{4.000000in}}%
\pgfpathlineto{\pgfqpoint{0.000000in}{4.000000in}}%
\pgfpathclose%
\pgfusepath{fill}%
\end{pgfscope}%
\begin{pgfscope}%
\pgfsetbuttcap%
\pgfsetmiterjoin%
\definecolor{currentfill}{rgb}{1.000000,1.000000,1.000000}%
\pgfsetfillcolor{currentfill}%
\pgfsetlinewidth{0.000000pt}%
\definecolor{currentstroke}{rgb}{0.000000,0.000000,0.000000}%
\pgfsetstrokecolor{currentstroke}%
\pgfsetstrokeopacity{0.000000}%
\pgfsetdash{}{0pt}%
\pgfpathmoveto{\pgfqpoint{0.736806in}{0.600955in}}%
\pgfpathlineto{\pgfqpoint{4.801389in}{0.600955in}}%
\pgfpathlineto{\pgfqpoint{4.801389in}{3.801389in}}%
\pgfpathlineto{\pgfqpoint{0.736806in}{3.801389in}}%
\pgfpathclose%
\pgfusepath{fill}%
\end{pgfscope}%
\begin{pgfscope}%
\pgfpathrectangle{\pgfqpoint{0.736806in}{0.600955in}}{\pgfqpoint{4.064583in}{3.200434in}}%
\pgfusepath{clip}%
\pgfsetbuttcap%
\pgfsetmiterjoin%
\definecolor{currentfill}{rgb}{0.121569,0.466667,0.705882}%
\pgfsetfillcolor{currentfill}%
\pgfsetfillopacity{0.200000}%
\pgfsetlinewidth{1.003750pt}%
\definecolor{currentstroke}{rgb}{0.121569,0.466667,0.705882}%
\pgfsetstrokecolor{currentstroke}%
\pgfsetstrokeopacity{0.200000}%
\pgfsetdash{}{0pt}%
\pgfpathmoveto{\pgfqpoint{3.101550in}{0.600955in}}%
\pgfpathlineto{\pgfqpoint{3.101550in}{3.801389in}}%
\pgfpathlineto{\pgfqpoint{3.575979in}{3.801389in}}%
\pgfpathlineto{\pgfqpoint{3.575979in}{0.600955in}}%
\pgfpathclose%
\pgfusepath{stroke,fill}%
\end{pgfscope}%
\begin{pgfscope}%
\pgfpathrectangle{\pgfqpoint{0.736806in}{0.600955in}}{\pgfqpoint{4.064583in}{3.200434in}}%
\pgfusepath{clip}%
\pgfsetbuttcap%
\pgfsetmiterjoin%
\definecolor{currentfill}{rgb}{1.000000,0.498039,0.054902}%
\pgfsetfillcolor{currentfill}%
\pgfsetfillopacity{0.200000}%
\pgfsetlinewidth{1.003750pt}%
\definecolor{currentstroke}{rgb}{1.000000,0.498039,0.054902}%
\pgfsetstrokecolor{currentstroke}%
\pgfsetstrokeopacity{0.200000}%
\pgfsetdash{}{0pt}%
\pgfpathmoveto{\pgfqpoint{1.807751in}{0.600955in}}%
\pgfpathlineto{\pgfqpoint{1.807751in}{3.801389in}}%
\pgfpathlineto{\pgfqpoint{2.343490in}{3.801389in}}%
\pgfpathlineto{\pgfqpoint{2.343490in}{0.600955in}}%
\pgfpathclose%
\pgfusepath{stroke,fill}%
\end{pgfscope}%
\begin{pgfscope}%
\pgfpathrectangle{\pgfqpoint{0.736806in}{0.600955in}}{\pgfqpoint{4.064583in}{3.200434in}}%
\pgfusepath{clip}%
\pgfsetbuttcap%
\pgfsetmiterjoin%
\definecolor{currentfill}{rgb}{0.172549,0.627451,0.172549}%
\pgfsetfillcolor{currentfill}%
\pgfsetfillopacity{0.200000}%
\pgfsetlinewidth{1.003750pt}%
\definecolor{currentstroke}{rgb}{0.172549,0.627451,0.172549}%
\pgfsetstrokecolor{currentstroke}%
\pgfsetstrokeopacity{0.200000}%
\pgfsetdash{}{0pt}%
\pgfpathmoveto{\pgfqpoint{2.347696in}{0.600955in}}%
\pgfpathlineto{\pgfqpoint{2.347696in}{3.801389in}}%
\pgfpathlineto{\pgfqpoint{2.643439in}{3.801389in}}%
\pgfpathlineto{\pgfqpoint{2.643439in}{0.600955in}}%
\pgfpathclose%
\pgfusepath{stroke,fill}%
\end{pgfscope}%

  \caption[Abstandsbestimmung]{Bestimmung der Peakpositionen der Zeitspektren zwecks der
    Berechnung des Detektorabstandes. Die farbigen Intervalmarkierung
    stellen die gesch\"atzte Abweichung dar.}
  \label{fig:calibration-lenght_det}
\end{figure}

Entsprechend ergibt sich dann der Detektorabstand zu:
\begin{align}
  \label{eq:abst}
  \mathfrak{t} &= \SI{2.2\pm .6}{\nano\second} \\
  D &= \frac{\mathfrak{t}}{2c} = \SI{330\pm 90}{\milli\meter}
\end{align}

\subsection{Theoriebeispiel}
\label{sec:theobei}
Zur Verbesserung des verst\"andnisses der Projektions- und
Rekonstruktionsprozesse, werden diese hier anhand eines einfachen
Beispiels nachvollzogen.

\begin{figure}[htp]
  \centering
  \begin{subfigure}[t]{.25\textwidth}
    \centering
    \includegraphics[width=.6\textwidth]{../auswertung/figs/theory/source.pdf}
    \caption{Ausgangsmatrix}
    \label{fig:theory-source}
  \end{subfigure}
  \begin{subfigure}[t]{.25\textwidth}
    \centering
    \includegraphics[width=.6\textwidth]{../auswertung/figs/theory/projection.pdf}
    \caption{Sinogram}
    \label{fig:theory-projection}
  \end{subfigure}
  \begin{subfigure}[t]{.25\textwidth}
    \centering
    \includegraphics[width=.6\textwidth]{../auswertung/figs/theory/convoluted.pdf}
    \caption{Gefiltertes Sinogram}
    \label{fig:theory-convoluted}
  \end{subfigure}
  \begin{subfigure}[t]{.25\textwidth}
    \centering
    \includegraphics[width=.6\textwidth]{../auswertung/figs/theory/rec_simple.pdf}
    \caption{Einfache R\"uckprojektion}
    \label{fig:theory-rec_simple}
  \end{subfigure}
  \begin{subfigure}[t]{.25\textwidth}
    \centering
    \includegraphics[width=.6\textwidth]{../auswertung/figs/theory/rec_filtered.pdf}
    \caption{Gefilterte R\"uckprojektion}
    \label{fig:theory-rec_filtered}
  \end{subfigure}
  \caption[Graustufendarstellung der
  Beispielmatrizen]{Graustufendarstellung der Matrizen aus den
    Teilschritten des Beispiels. Es wurden jeweils die Matrixelemente
    in das Interval \([0,1]\) reskaliert.}
  \label{fig:graubei}
\end{figure}

Im ersten Schritt wird die Projektion~\ref{fig:theory-projection} des
Ausgangsbildes~\ref{fig:theory-source} aus Verschiedenen Winkeln
berechenet. Dabei wurden die diagonalen entsprechend gewichtet.

\begin{equation}
  \label{eq:proj}
  \mathfrak{M}_0 =
  \begin{pmatrix}
    0 & 0 & 0 & 0 & 2\\
    0 & 9 & 0 & 0 & 0\\
    0 & 0 & 0 & 0 & 0\\
    0 & 7 & 0 & 8 & 0\\
    0 & 0 & 0 & 0 & 0\\
  \end{pmatrix}
  \implies
  \left(
    \begin{array}{ccccc|c}
      0. & 16. & 0. & 8. & 2 & 0^\circ\\
      2.73 & 4.95 & 15.47 & 0.68 & 0.28 & 45^\circ\\
      0. & 15. & 0. & 9. & 2. & 90^\circ\\
      3.12 & 5.24 & 8.19 & 5.85 & 3.51 & 135^\circ\\
    \end{array}\right) = \mathfrak{P}_0
\end{equation}

Ein gefiltertes Sinogram~\ref{fig:theory-convoluted} ergibt sich durch
Faltung der Zeilen von \(\mathfrak{M}_0\) mit
\(F = \mqty(-.1 & .25 & -.1)\).

\begin{equation}
  \label{eq:filter}
  \mathfrak{M}_1 = \mathfrak{M}_0 * F =
  \begin{pmatrix}
    -1.6 & 4. & -2.4 & 1.8 & -0.3\\
    0.188 & -0.583 & 3.304 & -1.405 & 0.002\\
    -1.5 & 3.75 & -2.4 & 2.05 & -0.4\\
    0.256 & 0.179 & 0.938 & 0.293 & 0.292\\
  \end{pmatrix}
\end{equation}

Die R\"uckprojektion des Einfachen sinogramms
ergib~\ref{fig:theory-rec_simple}.

{\footnotesize
\setlength{\arraycolsep}{2.5pt}

\begin{align}
  \label{eq:simplerepr}
  \overbrace{\begin{pmatrix}
      0. & 16. & 0. & 8. & 2.\\
      0. & 16. & 0. & 8. & 2.\\
      0. & 16. & 0. & 8. & 2.\\
      0. & 16. & 0. & 8. & 2.\\
      0. & 16. & 0. & 8. & 2.\\
    \end{pmatrix}}^{0^\circ} & + \frac{1}{2}\overbrace{\begin{pmatrix}
      15.864 & 8.408 & 0.837 & 0.287 & 0.076\\
      12.636 & 15.864 & 8.408 & 0.837 & 0.287\\
      6.569 & 12.636 & 15.864 & 8.408 & 0.837\\
      2.78 & 6.569 & 12.636 & 15.864 & 8.408\\
      0.737 & 2.78 & 6.569 & 12.636 & 15.864\\
    \end{pmatrix}}^{45^\circ} \nonumber \\ + \overbrace{\begin{pmatrix}
      2. & 2. & 2. & 2. & 2.\\
      9. & 9. & 9. & 9. & 9.\\
      0. & 0. & 0. & 0. & 0.\\
      15. & 15. & 15. & 15. & 15.\\
      0. & 0. & 0. & 0. & 0.\\
    \end{pmatrix}}^{90^\circ} &+ \frac{1}{2}\overbrace{\begin{pmatrix}
      0.842 & 3.172 & 7.12 & 9.283 & 8.966\\
      3.172 & 7.12 & 9.283 & 8.966 & 9.886\\
      7.12 & 9.283 & 8.966 & 9.886 & 8.003\\
      9.283 & 8.966 & 9.886 & 8.003 & 3.569\\
      8.966 & 9.886 & 8.003 & 3.569 & 0.948\\
    \end{pmatrix}}^{135^\circ}\nonumber \\
  & = \begin{pmatrix}
    10.353 & 23.79 & 5.978 & 14.785 & 8.521\\
    16.904 & 36.492 & 17.845 & 21.902 & 16.087\\
    6.844 & 26.959 & 12.415 & 17.147 & 6.42\\
    21.031 & 38.768 & 26.261 & 34.933 & 22.988\\
    4.852 & 22.333 & 7.286 & 16.102 & 10.406\\
  \end{pmatrix} = \mathfrak{M}_1
\end{align}}

Aus dem gefilterten Sinogram ergibt sich auf \"ahnliche
Weise~\ref{fig:theory-rec_filtered}
{\footnotesize
\setlength{\arraycolsep}{2.5pt}

\begin{align}
  \label{eq:simplerepr}
  \overbrace{\begin{pmatrix}
      -1.6 & 4. & -2.4 & 1.8 & -0.3\\
      -1.6 & 4. & -2.4 & 1.8 & -0.3\\
      -1.6 & 4. & -2.4 & 1.8 & -0.3\\
      -1.6 & 4. & -2.4 & 1.8 & -0.3\\
      -1.6 & 4. & -2.4 & 1.8 & -0.3\\
    \end{pmatrix}}^{0^\circ} & + \frac{1}{2}\overbrace{\begin{pmatrix}
      3.165 & 0.261 & -1.305 & -0.012 & 0.001\\
      1.076 & 3.165 & 0.261 & -1.305 & -0.012\\
      -0.407 & 1.076 & 3.165 & 0.261 & -1.305\\
      0.182 & -0.407 & 1.076 & 3.165 & 0.261\\
      0.051 & 0.182 & -0.407 & 1.076 & 3.165\\
    \end{pmatrix}}^{45^\circ} \nonumber \\ +
  \overbrace{\begin{pmatrix}
      -0.4 & -0.4 & -0.4 & -0.4 & -0.4\\
      2.05 & 2.05 & 2.05 & 2.05 & 2.05\\
      -2.4 & -2.4 & -2.4 & -2.4 & -2.4\\
      3.75 & 3.75 & 3.75 & 3.75 & 3.75\\
      -1.5 & -1.5 & -1.5 & -1.5 & -1.5\\
    \end{pmatrix}}^{90^\circ} &+ \frac{1}{2}\overbrace{\begin{pmatrix}
      0.069 & 0.258 & 0.351 & 0.646 & 0.972\\
      0.258 & 0.351 & 0.646 & 0.972 & 0.759\\
      0.351 & 0.646 & 0.972 & 0.759 & 0.486\\
      0.646 & 0.972 & 0.759 & 0.486 & 0.295\\
      0.972 & 0.759 & 0.486 & 0.295 & 0.079\\
    \end{pmatrix}}^{135^\circ}\nonumber \\
           &= \begin{pmatrix}
             -0.383 & 3.86 & -3.277 & 1.717 & -0.214\\
             1.117 & 7.808 & 0.104 & 3.683 & 2.123\\
             -4.028 & 2.461 & -2.732 & -0.09 & -3.11\\
             2.564 & 8.032 & 2.267 & 7.375 & 3.728\\
             -2.589 & 2.97 & -3.861 & 0.985 & -0.178\\
           \end{pmatrix} = \mathfrak{M}_2
\end{align}
}

Es ist zu erkennen, das in beiden rekonstruktionen die starken Signale
\((1,1),\,(3,1),\,(3,3)\) klar zu identifizerien, wenngleich die
Unterschiede in der signalst\"arke nicht im urspr\"unglichen
Verh\"altniss stehen. Die gefilterte R\"uckprojektion weist in den
Randfeldern und im Mittleren Feld einen h\"oheren Kontrast auf,
erzeugt aber dennoch nur ein geringf\"ugig besseres und in manchen
Bereichen (Ecken) sogar ein schlechteres Bild. Das schwache Signal
\((0,4)\) wurde in beiden F\"allen nicht rekonstruiert. F\"uhrt man
die Rechnung ohne diesen Punkt aus, ergibt sich kaum ein
Unterschied. Schwache Signale werden also nicht gut reproduziert.

\newpage
\section{Verzeichnisse}
\label{sec:literatur}

\listoffigures

\listoftables

\printbibliography
\end{document}
