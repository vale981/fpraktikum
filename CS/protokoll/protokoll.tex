\documentclass[slug=CS, room=Andreas-Schubert-Bau\,\ Labor\ 406,
supervisor=Juliane\ Volkmer, coursedate=29.\ 11.\ 2019]{../../Lab_Report_LaTeX/lab_report}

\title{Comptonstreuung}
\author{Oliver Matthes, Valentin Boettcher}
\usepackage[version=4]{mhchem}
\usepackage{todonotes}
\graphicspath{ {figs/} }
\newcommand{\cs}{\emph{Comptonstreuung }}
\usepackage{circuitikz}
\usepackage{subcaption}
\usepackage{ amssymb }
\usepackage{tabularx}
\usepackage{pgf}
\sisetup{math-celsius = {}^{\circ}\kern-\scriptspace C}
\usepackage[ngerman]{babel}

% bib
\addbibresource{protokoll.bib}

\begin{document}
\maketitle

\section{Einleitung}
\label{sec:einl}

\cs ist in einem Energiebereich von hundert Kiloelektronenvolt bis hin zu wenigen
Megaelektronenvolt der Wechselwirkungsprozess zwischen Photonen und Materie, der am
 wahrscheinlichsten auftritt und deswegen in vielen physikalischen Bereichen beachtet werden muss.
Neben \cs gibt es natürlich auch noch andere Wechselwirkungsprozesse wie der
Photoeffekt, der vor allem bei geringen Photonenenergien auftritt oder die Paarbildung bei
der man Energien von mindestens zwei Elektronenmassen braucht, um ein Elektron-Positron-Paar
zu erzeugen. Diese Prozesse werden in diesem Versuch allerdings nicht betrachtet.\\

Um Aussagen über die Wahrscheinlichkeit, dass ein Photon mit einem Elektron wechselwirkt,
treffen zu können, definiert man den Wirkungsquerschnitt:

\begin{equation}\label{eq:wirkquer}
	\sigma = \frac{N}{\Phi}
\end{equation}

\begin{tabular}{llll}
	 & \(N\)    & ... & mittlere Wechselwirkungsanzahl eines Teilchens mit einem atomaren Target \\
	 & \(\Phi\) & ... & Teilchenfluenz, dem das Target ausgesetzt ist
\end{tabular}\\

Der Wechselwirkungsquerschnitt der inkohärenten Streuung und damit des Comptoneffekts ist
proportional zur Ordnungszahl des Atoms (\(\sigma_i \propto\) Z).\\

\subsection{Inkohärente Streuung}
\label{sec:inkostreu}

Wichtig, um \cs beschreiben zu können, ist der Prozess der \emph{inkohärenten Streuung}.
Dabei überträgt das Photon bei der Wechselwirkung mit einem an einem Atomkern gebundenen
Elektron einen Teil seiner Energie auf dieses, so dass es den gebundenen Zustand verlassen kann.
Vernachlässigt man bei diesem Prozess die Bindungsenergie des Elektrons, nennt man diesen \cs,
da Arthur Holly Compton 1922 diese Annahme traf, um diesen Effekt zu beschreiben.\\

\subsubsection{Comptonstreuung}
\label{sec:cs}

Um \cs zu beschreiben, geht man, wie oben schon erwähnt, von quasi freien Elektronen aus.
Diese Annahme trifft besonders gut auf Metalle zu (im Experiment werden wir mit einem Aluminiumtarget arbeiten).
Wird ein Photon an einem Elektron gestreut, ändert sich seine Energie sowie seine 
Bewegungsrichtung um einen polaren Streuwinkel \(\theta\). Nutzt man den Energie-
und den Impulserhaltungssatz aus und setzt den Photonenimpuls \(p = E/c\) ein, so erhält man
einen Ausdruck für die Energie des Wechselwirkungsphotons nach der Interaktion:

\begin{gather}
	E(\mu) = \frac{E'}{1 + \kappa(1 - \mu)} \label{eq:photoenergie}\\
	\kappa = \frac{E'}{m_0c^2}\\
	\mu = \cos\theta
\end{gather}

\begin{tabular}{llll}
	& \(E'\)    & ... & Photonenenergie vor dem Stoß
\end{tabular}\\

Die Ruheenergie des Elektrons beträgt:

\begin{equation}\label{key}
	E_{e^-} = m_0c^2 = \SI{511}{\kilo\electronvolt}
\end{equation}\\

Die maximal mögliche Energie, die ein Photon nach der Streuung haben kann, ist also dessen
Ausgangsenergie. Bei einem Streuwinkel von \(\theta = 0^\circ\) (Vorwärtsstreuung) folgt
\(\mu \rightarrow 1 \implies E(\mu) \rightarrow E'\).
Die Minimalenergie wird bei \(\mu = -1\), also bei \(\theta = 180^\circ\), erreicht, da
hier gilt:

\begin{equation}\label{eq:emax}
	E(\mu) = \frac{E'}{1 + 2\kappa}
\end{equation}

Je größer der polare Streuwinkel \(\theta\) des Photons ist, desto mehr Energie wird beim Stoß
an das Elektron übertragen. Je größer außerdem die Ausgangsenergie des Photons, desto größer ist
der Energieverlust bei der Streuung und desto höher ist zudem die Winkelabhängigkeit des
Energieverlustes.\\

\section{Verzeichnisse}

\label{sec:literatur}

\listoffigures

\listoftables

\printbibliography
\end{document}