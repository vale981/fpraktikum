\documentclass[slug=SZ, room=Hermann-Krone-Bau\,\ Labor\ 1.25, supervisor=Tim\ Ziegler]{../../Lab_Report_LaTeX/lab_report}

\title{Solarzelle}
\author{Oliver Matthes, Valentin Boettcher}
\usepackage[version=4]{mhchem}
\usepackage{todonotes}
\graphicspath{ {figs/} }

\usepackage[ngerman]{babel}

% bib
\addbibresource{protokoll.bib}

\begin{document}
\maketitle

\section{Einleitung}
\label{sec:einl}

Die Energiegewinnung aus erneuerbaren Energien spielt eine entscheidende Rolle, wenn es darum geht,
aus der Energieproduktion mittels fossiler Energieträger auszusteigen.
Auch Solarzellen steuern dazu einen wichtigen Beitrag bei. Deswegen ist es wichtig, diese
Technologie weiterzuentwickeln.

Solarzellen wandeln durch Lichtabsorption Strahlung in elektrische Energie um (photovoltaischer Effekt).
Dafür müssen Solarzellen die eintreffende Strahlung natürlich absorbieren. 
Außerdem muss es aufgrund dieser Absorption zu einer Anregung von beweglichen Ladungsträgern
(positiven und negativen) kommen, die von einander getrennt werden müssen.

Zur Erfüllung dieser Kriterien, benötigt man einen Übergang zwischen zwei verschieden dotierten
Halbleitern (p-n-Übergang)(vgl. ref...).

\subsection{Halbleiter}
\label{sec:halbleiter}

Die beste Erklärung der elektrischen Eigenschaften von Halbleitern liefert das Bändermodell.
Dieses Modell besteht aus Energiebändern und Bandlücken.

In einem einzelnem Atom können Elektronen nur diskrete Energiewerte annehmen.
Kristalle allerdings bestehen aus sehr vielen Atomen (~10^{23}), mit einem geringen Abstand zu einander,
der dazu führt, dass die Wellenfunktionen der Elektronen überlappen und somit die Energieniveaus in sehr
viele Unterniveaus aufspalten, die praktisch kontinuierlich aussehen.
Zwischen diesen Energiebändern befinden sich Bandlücken, die einen nicht erlaubten Bereich darstellen und
einen Abstand $ \mathit{E_g} $ besitzen.

Das bei einer Temperatur von $ T=0 K $ höchste vollbesetzte Band nennt man das \emph{Valenzband}.
Die maximale Energie, die die Elektronen bei $ T=0 K $ besitzen \emph{Fermienergie}. Das nächst höhere Band ist
also nicht vollständig besetzt, weswegen sich Ladungsträger ziemlich gut auf diesem fortbewegen können, da
ihnen viele unbesetzte Zustände zur Verfügung stehen.
Aufgrund dieser Eigenschaft wird jenes Band als \emph{Leitungsband} bezeichnet.
Um ein Elektron also aus dem Valenz- in das Leitungsband anzuheben, muss es die Bandlücke überqueren,
wofür es genügend Energie benötigt. Diese erhält es durch die Absorption von Strahlung der Energie:

\begin{equation}\label{eq:bandenenergie}
	E_g = h\nu
\end{equation}

Bei einer Temperatur von $ T=0 K $ sind Halbleiter ebenso wie Isolatoren nichtleitend.
Der Unterschied zwischen den Beiden ist die Größe der Bandlücke. Diese ist bei Isolatoren relativ groß,
bei Halbleitern hingegen eher klein, sodass schon geringe Energien ausreichen, um Elektronen aus dem Valenz-
in das Leitungsband anzuheben.

Der Unterschied zwischen den beiden ist die Größe der Bandlücke. Diese ist bei Isolatoren relativ groß,
bei Halbleitern hingegen eher klein, sodass schon geringe Energien ausreichen, um Elektronen aus dem Valenz-
in das Leitungsband anzuheben.

\subsection{Dotierung von Halbleitern}
\label{sec:dotierung}

Unter Dotierung versteht man die "Verunreinigung" des eigentlichen Halbleitermaterials mit Fremdatomen, um
die Eigenschaften dieses Halbleiters zu verändern.
Man unterscheidet dabei zwischen \emph{n-dotierten Halbleitern} und \emph{p-dotierten Halbleitern}.

\begin{description}
	
	\item[n-dotierte Halbleiter] Bringt man in einen Siliziumkristall, dessen Atome je vier Valenzelektronen
	besitzen, ein paar Atome, die beispielsweise fünf Valenzelektronen (z.B. Phosphor) haben, so binden die 
	vier Siliziumelektronen vier der Elektronen der Fremdatome. Ein Außenelektron es Phosphors bleibt also
	ungebunden und dient als Ladungsträger. Die nun positiv geladenen Phosphoratome sitzen fest im Kristall,
	können sich also nicht bewegen und dienen deswegen nicht als Ladungsträger.
	Da thermisch angeregte Elektron-Loch-Paare in dotierten Halbleitern relativ selten vorkommen und die 
	beweglichen Elektronen der Hauptladungsträger sind, nennt man diese \emph{Majoritätsladungsträger}, die
	Elektron-Loch-Paare entsprechend \emph{Minoritätsladungsträger}.
	
	\item[p-dotierte Halbleiter] Bei p-dotierten Halbleitern macht man genau das Gegenteil von dem, was man
	bei den n-dotierten getan hat. Statt Fremdatome mit fünf bringt man solche mit drei Valenzelektronen
	in den Siliziumkristall ein. Das nun fehlende Elektron steuert das Silizium bei. Dadurch entsteht eine
	frei bewegliche positive Ladung, ein so genanntes Loch, das jetzt den \emph{Majoritätsladungsträger}
	darstellt.
	
\end{description}

Durch die Dotierung kommt es zu einem Ladungsträgerungleichgewicht, das die Fermie-Energie in Richtung des
Majoritätsladungsträger enthaltenden Bandes.

\subsection{p-n-Übergang von Halbleitern}
\label{sec:pnüber}

Ein p-n-Übergang findet statt, wenn man einen p-dotierten und einen n-dotierten Halbleiter in Kontakt miteinander
bringt. Im n-Gebiet befinden sich mehr Elektronen als im p-Gebiet. Dadurch kommt es zu einem Konzentrationsgefälle
und die Löcher diffundieren Richtung n-Gebiet, die Elektronen Richtung p-Gebiet. Treffen beide Ladungsträger
aufeinander rekombinieren sie. Aufgrund dessen sinkt die Zahl der Ladungsträger nahe der Grenze der beiden
Halbleiter und es entsteht eine so genannte \emph{Verarmungszone}. Die Atomen, mit denen der Halbleiter
dotiert worden ist, sind, wie in \ref{sec:dotierung} unbeweglich. Deswegen bleiben diese in der Verarmungszone
zurück und es entsteht ein negativ geladener Bereich im p-dotierten und ein positiv geladener im
n-dotierten Halbleiter. Diese beiden Bereiche zusammen werden als \emph{Raumladungszone} bezeichnet.
In dieser Zone entsteht also durch diese festen Ladungen eine Potentialdifferenz, die der Diffusion der
beweglichen Ladungen entgegen wirkt. Im Gleichgewicht zwischen Diffusion und Feldstrom ist die 
\emph{Raumladungszone} gleich der \emph{Verarmungszone}.
\section{Literatur}
\label{sec:literatur}

\printbibliography
\end{document}
