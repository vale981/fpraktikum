\documentclass[slug=SZ, room=Hermann-Krone-Bau\,\ Labor\ 1.25, supervisor=Tim\ Ziegler]{../../Lab_Report_LaTeX/lab_report}

\title{Solarzelle}
\author{Oliver Matthes, Valentin Boettcher}
\usepackage[version=4]{mhchem}
\usepackage{todonotes}
\graphicspath{ {figs/} }

\usepackage[ngerman]{babel}

% bib
\addbibresource{protokoll.bib}

\begin{document}
\maketitle

\section{Einleitung}
\label{sec:einl}

Die Energiegewinnung aus erneuerbaren Energien spielt eine entscheidende Rolle, wenn es darum geht,
aus der Energieproduktion mittels fossiler Energieträger auszusteigen.
Auch Solarzellen steuern dazu einen wichtigen Beitrag bei. Deswegen ist es wichtig, diese
Technologie weiterzuentwickeln.

Solarzellen wandeln durch Lichtabsorption Strahlung in elektrische Energie um (photovoltaischer Effekt).
Dafür müssen Solarzellen die eintreffende Strahlung natürlich absorbieren. 
Außerdem muss es aufgrund dieser Absorption zu einer Anregung von beweglichen Ladungsträgern
(positiven und negativen) kommen, die von einander getrennt werden müssen.

Zur Erfüllung dieser Kriterien, benötigt man einen Übergang zwischen zwei verschieden dotierten
Halbleitern (p-n-Übergang)(vgl. ref...).

\subsection{Halbleiter}
\label{sec:halbleiter}

Die beste Erklärung der elektrischen Eigenschaften von Halbleitern liefert das Bändermodell.
Dieses Modell besteht aus Energiebändern und Bandlücken.

In einem einzelnem Atom können Elektronen nur diskrete Energiewerte annehmen.
Kristalle allerdings bestehen aus sehr vielen Atomen (~10^{23}), mit einem geringen Abstand zu einander,
der dazu führt, dass die Wellenfunktionen der Elektronen überlappen und somit die Energieniveaus in sehr
viele Unterniveaus aufspalten, die praktisch kontinuierlich aussehen.
Zwischen diesen Energiebändern befinden sich Bandlücken, die einen nicht erlaubten Bereich darstellen und
einen Abstand $ \mathit{E_g} $ besitzen.

\todo{Bänder erläutern.}

Bei einer Temperatur von $ T=0 K $ sind Halbleiter ebenso wie Isolatoren nichtleitend.
Der Unterschied zwischen den Beiden ist die Größe der Bandlücke. Diese ist bei Isolatoren relativ groß,
bei Halbleitern hingegen eher klein, sodass schon geringe Energien ausreichen, um Elektronen aus dem Valenz-
in das Leitungsband anzuheben.

\section{Literatur}
\label{sec:literatur}

\printbibliography
\end{document}
